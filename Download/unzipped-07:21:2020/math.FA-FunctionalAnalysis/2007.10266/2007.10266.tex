\documentclass[reqno]{amsproc}
\usepackage{hyperref,color,amssymb}

\newcommand{\cA}{\mathcal{A}}
\newcommand{\cB}{\mathcal{B}}
\newcommand{\cK}{\mathcal{K}}
\newcommand{\cM}{\mathcal{M}}
\newcommand{\fM}{\mathfrak{M}}

\newcommand{\alg}{\operatorname{alg}}
\newcommand{\eps}{\varepsilon}
\newcommand{\dR}{{\bf\dot{\R}}}

\newcommand{\C}{\mathbb{C}}
\newcommand{\N}{\mathbb{N}}
\newcommand{\R}{\mathbb{R}}

\newtheorem{theorem}{Theorem}[section]
\newtheorem{lemma}[theorem]{Lemma}
\newtheorem{corollary}[theorem]{Corollary}
\newtheorem{question}[theorem]{Question}
\theoremstyle{definition}
\newtheorem{definition}[theorem]{Definition}
\theoremstyle{remark}
\newtheorem{remark}[theorem]{Remark}
\numberwithin{equation}{section}

\title[Algebras of convolution type operators]
{Algebras of convolution type operators with continuous data do not 
always contain all rank one operators}
%%%----------------------------------------------------------------------------
\author[A. Karlovich]{Alexei Karlovich}
\address{%%
Centro de Matem\'atica e Aplica\c{c}\~oes\\
Departamento de Matem\'atica\\
Faculdade de Ci\^encias e Tecnologia\\
Universidade Nova de Lisboa\\
Quinta da Torre\\
2829--516 Caparica\\
Portugal} \email{oyk@fct.unl.pt}
%%%----------------------------------------------------------------------------
\author[E. Shargorodsky]{Eugene Shargorodsky}
\address{%
Department of Mathematics\\
King's College London\\
Strand, London WC2R 2LS\\
United Kingdom
\ and \
Technische Universit\"at Dresden\\
Fakult\"at Mathematik\\
01062 Dresden\\
Germany}
\email{eugene.shargorodsky@kcl.ac.uk}


\date{\today}
\begin{document}
%%%----------------------------------------------------------------------------
\begin{abstract}
Let $X(\mathbb{R})$ be a separable Banach function space such that the 
Hardy-Littlewood maximal operator is bounded $X(\mathbb{R})$ 
and on its associate space $X'(\mathbb{R})$. The algebra $C_X(\dR)$
of continuous Fourier multipliers on $X(\mathbb{R})$ is defined as the 
closure of the set of continuous functions of bounded variation on 
$\dR=\mathbb{R}\cup\{\infty\}$ with respect to the multiplier norm. It 
was proved by C. Fernandes, Yu. Karlovich and the first author
\cite{FKK19} that if the space $X(\R)$ is reflexive, then the ideal of compact 
operators is contained in the Banach algebra $\mathcal{A}_{X(\mathbb{R})}$ 
generated by all multiplication operators $aI$ by continuous functions 
$a\in C(\dR)$ and by all Fourier convolution operators $W^0(b)$ with symbols 
$b\in C_X(\dR)$. We show that there are separable and non-reflexive 
Banach function spaces $X(\mathbb{R})$ such that the algebra 
$\mathcal{A}_{X(\mathbb{R})}$ does not contain all rank one operators. 
In particular, this happens in the case of the Lorentz spaces 
$L^{p,1}(\mathbb{R})$ with $1<p<\infty$.
\end{abstract}
\maketitle
%%%----------------------------------------------------------------------------
\section{Introduction}
We denote by $\mathcal{S}(\R)$ the Schwartz class of all infinitely 
differentiable and rapidly decaying functions (see, e.g., 
\cite[Section~2.2.1]{G14}). Let $F$ denote the Fourier transform, defined
on $\mathcal{S}(\R)$ by
\[
(Ff)(x):=\widehat{f}(x):=\int_\R f(t)e^{itx}\,dt,
\quad
x\in\R,
\]
and let $F^{-1}$ be the inverse of $F$ defined on $\mathcal{S}(\R)$ 
by
\[
(F^{-1}g)(t)=\frac{1}{2\pi}\int_\R g(x)e^{-itx}\,d x,
\quad
t\in\R.
\]
It is well known that these operators extend uniquely to the space
$L^2(\R)$. As usual, we will use symbols $F$ and $F^{-1}$ for the
direct and inverse Fourier transform on $L^2(\R)$. 
The Fourier convolution operator 
\[
W^0(a):=F^{-1}aF
\]
is bounded on the space $L^2(\R)$ for every $a\in L^\infty(\R)$.

In this paper we will study algebras of operators generated by 
operators of multiplication $aI$ and Fourier convolution operators
$W^0(b)$ on so-called Banach function spaces in the case when both $a$ and $b$
are continuous. We postpone a formal definition of a Banach function 
space $X(\R)$ and its associate space $X'(\R)$ until Section~\ref{sec:BFS}. 
The Lebesgue spaces $L^p(\R)$ with 
$1\le p\le\infty$ constitute the most important example of Banach function
spaces. The class of Banach function spaces includes classical Orlicz
spaces $L^\Phi(\R)$, Lorentz spaces $L^{p,q}(\R)$, all other 
rearrangement-invariant spaces, as well as, (non-rearrangement-in\-vari\-ant) 
weighted Lebesgue spaces $L^p(\R,w)$ and 
variable Lebesgue spaces $L^{p(\cdot)}(\R)$.

Let $X(\R)$ be a separable Banach function space. Then $L^2(\R)\cap X(\R)$
is dense in $X(\R)$ (see Lemma~\ref{le:density} below). A function 
$a\in L^\infty(\R)$ is called a Fourier multiplier on $X(\R)$ if the 
convolution operator $W^0(a):=F^{-1}aF$ maps $L^2(\R)\cap X(\R)$ into 
$X(\R)$ and extends to a bounded linear operator on $X(\R)$. The function 
$a$ is called the symbol of the Fourier convolution operator $W^0(a)$. 
The set $\cM_{X(\R)}$ of all Fourier multipliers on  $X(\R)$ is a unital 
normed algebra under pointwise operations and the norm
\[
\left\|a\right\|_{\cM_{X(\R)}}:=\left\|W^0(a)\right\|_{\cB(X(\R))},
\]
where $\cB(X(\R))$ denotes the Banach algebra of all bounded linear operators
on the space $X(\R)$. Let $\cK(X(\R))$ denote the ideal of all compact
operators in the Banach algebra $\cB(X(\R))$.

Recall that the (non-centered) Hardy-Littlewood maximal function $Mf$ of a
function $f\in L_{\rm loc}^1(\R)$ is defined by
\[
(Mf)(x):=\sup_{Q\ni x}\frac{1}{|Q|}\int_Q|f(y)|\,dy,
\]
where the supremum is taken over all intervals $Q\subset\R$ of finite length
containing $x$. The Hardy-Littlewood maximal operator $M$ defined by the rule 
$f\mapsto Mf$ is a sublinear operator.

Suppose that $a:\R\to\C$ is a function of bounded variation $V(a)$ given 
by
\[
V(a):=\sup \sum_{k=1}^n |a(x_k)-a(x_{k-1})|,
\]
where the supremum is taken over all partitions of $\R$ of the form
\[
-\infty<x_0<x_1<\dots<x_n<+\infty
\]
with $n\in\N$. The set $V(\R)$ of all functions of bounded variation
on $\R$ with the norm
\[
\|a\|_{V(\R)}:=\|a\|_{L^\infty(\R)}+V(a)
\]
is a unital non-separable Banach algebra.

Let $X(\R)$ be a separable Banach function space such that the
Hardy-Littlewood maximal operator $M$ is bounded on $X(\R)$ and on its
associate space $X'(\R)$. It follows from \cite[Theorem~4.3]{K15a}
that if a function $a:\R\to\C$ has a bounded variation $V(a)$, then 
the convolution operator $W^0(a)$ is bounded on the space $X(\R)$ and
%%%
\begin{equation}\label{eq:Stechkin}
\|W^0(a)\|_{\cB(X(\R))}
\le
c_{X}\|a\|_{V(\R)}
\end{equation}
%%%
where $c_{X}$ is a positive constant depending only on $X(\R)$.

For Lebesgue spaces $L^p(\R)$, $1<p<\infty$, inequality~\eqref{eq:Stechkin} is
usually called Stechkin's inequality.
We refer to \cite[Theorem~2.11]{D79} for the proof of \eqref{eq:Stechkin}
in the case of Lebesgue spaces $L^p(\R)$ with $c_{L^p}=\|S\|_{\cB(L^p(\R))}$,
where $S$ is the Cauchy singular integral operator.

Let $C(\dR)$ denote the $C^*$-algebra of continuous functions on the one-point 
compactification $\dR=\R\cup\{\infty\}$ of the real line. For a subset 
$\mathfrak{S}$ of a Banach space $\mathcal{E}$, we denote by
$\operatorname{clos}_\mathcal{E}(\mathfrak{S})$ the closure of $\mathfrak{S}$
with respect to the norm of $\mathcal{E}$. Consider the following algebra of 
continuous Fourier multipliers:
%%%
\begin{equation}\label{eq:algebra-CX}
C_{X}(\dR):=\operatorname{clos}_{\cM_{X(\R)}}
\big(C(\dR)\cap V(\R)\big).
\end{equation}
%%%
It follows Theorem \ref{th:continuous-embedding} below that 
$C_X(\dR)\subset C(\dR)$. 
The aim of this paper is to continue the study of the smallest Banach subalgebra
\[
\cA_{X(\R)}
:=
\operatorname{alg}\{aI,W^0(b)\ :\ a\in C(\dR),\ b\in C_X(\dR)\}
\]
of the algebra $\cB(X(\R))$ that contains all operators of multiplication $aI$ 
by functions $a\in C(\dR)$ and all Fourier convolution operators $W^0(b)$ with 
symbols $b\in C_X(\dR)$ started in the setting of reflexive Banach function
spaces in \cite{FKK19}. The main result of that paper says the following.
%%%----------------------------------------------------------------------------
\begin{theorem}[{\cite[Theorem~1.1]{FKK19}}]
\label{th:FKK}
Let $X(\R)$ be a reflexive Banach function space such that the Hardy-Littlewood 
maximal operator $M$ is bounded on $X(\R)$ and on its associate space 
$X'(\R)$. Then the ideal of compact operators $\cK(X(\R))$ is contained in the 
Banach algebra $\cA_{X(\R)}$.
\end{theorem}
%%%----------------------------------------------------------------------------
Note that results of this kind are well known in the setting of (weighted)
Lebesgue spaces (see, e.g., \cite[Lemma~6.1]{KILH13a}, \cite[Theorem~5.2.1 and 
Proposition~5.8.1]{RSS11} and also \cite[Lemma~8.23]{BK97}, 
\cite[Theorem~4.1.5]{RSS11}). They constitute the first step in the Fredholm 
study of more general algebras of convolution type operators with more 
general function algebras in place of $C(\dR)$ and $C_X(\dR)$, respectively 
(see, e.g., \cite{KILH12,KILH13a,KILH13b}), by means of local principles 
(see, e.g., \cite[Sections~1.30--1.35]{BS06}).

Let $\mathfrak{A}$ be a Banach algebra with unit $e$.  The center 
$\operatorname{Cen}\mathfrak{A}$ of $\mathfrak{A}$ is the set of all elements 
$z\in\mathfrak{A}$ with the property that $za = az$ for all $a\in\mathfrak{A}$.
One can successfully apply the Allan-Douglas local principle 
\cite[Section~1.35]{BS06} to the algebra $\mathfrak{A}$ if it possesses 
a (hopefully large) closed subalgebra $\mathfrak{C}$ lying in its center.
Having applications of the Allan-Douglas local principle in mind,
the authors of \cite{FKK19} asked whether the quotient algebra
\[
\cA_{X(\R)}^\pi:= \cA_{X(\R)}/\cK(X(\R))
\]
is commutative under the assumptions of Theorem~\ref{th:FKK}. Our first
result is the positive answer to \cite[Question~1.2]{FKK19}.
%%%----------------------------------------------------------------------------
\begin{theorem}
\label{th:algebra-Api}
Let $X(\R)$ be a reflexive Banach function space such that the Hardy-Littlewood 
maximal operator $M$ is bounded on $X(\R)$ and on its associate space 
$X'(\R)$. Then the quotient algebra $\cA_{X(\R)}^\pi$ is commutative.
\end{theorem}
%%%---------------------------------------------------------------------------
It is well known that a Banach function space $X(\R)$ is reflexive if and 
only if the space $X(\R)$ and its associate space $X'(\R)$ are separable 
(see \cite[Chap.~1, Corollaries 4.4 and 5.6]{BS88}). So, it is natural to 
ask whether the assumption of the reflexivity of the space $X(\R)$ in 
Theorem~\ref{th:FKK} can be relaxed to the assumption of the separability 
of the space $X(\R)$. Our main result says that this is impossible.
%%%----------------------------------------------------------------------------
\begin{theorem}[Main result]
\label{th:main}
There exists a separable non-reflexive Banach function space $X(\R)$ such that
\begin{enumerate}
\item[(a)]
the Hardy-Littlewood maximal operator is bounded on $X(\R)$ and on 
its associate space $X'(\R)$;
\item[(b)]
the algebra $\cA_{X(\R)}$ does not contain all rank one operators.
\end{enumerate}
\end{theorem}
%%%----------------------------------------------------------------------------
This theorem means that usual methods of the Fredholm study of algebras
of convolution type operators with discontinuous data on non-reflexive
separable Banach function spaces will require a modification to overcome
an obstacle that certain compact operators do not belong to the algebra
$\cA_{X(\R)}$ and, therefore, the quotient algebra $\cA_{X(\R)}/\cK(X(\R))$
cannot be defined. 

In fact, one can take in Theorem~\ref{th:main}
familiar separable and non-reflexive classical Lorentz 
spaces $L^{p,1}(\R)$ with $1<p<\infty$. Let us recall their 
definition. The distribution function $\mu_f$ of a
measurable function $f:\R\to\C$ is given by
\[
\mu_f(\lambda):=|\{x\in\R:|f(x)|>\lambda\}|,
\quad\lambda\ge 0.
\]
The non-increasing
rearrangement of $f$ is the function $f^*$ defined on $[0,\infty)$ by
\[
f^*(t)=\inf\{\lambda:\mu_f(\lambda)\le t\},
\quad t\ge 0
\]
(see, e.g., \cite[Chap.~3, Definitions 1.1 and 1.5]{BS88}).

For given $1 < p < \infty$ and $1 \le q \le \infty$, the Lorentz
space $L^{p, q}(\mathbb{R})$ consist of all measurable 
functions  $f : \mathbb{R} \to \mathbb{C}$ such that the norm
\[
\|f\|_{(p, q)} 
:=  
\left\{\begin{array}{cl}
\displaystyle\left(\int_0^\infty \left(t^{1/p} f^{**}(t)\right)^q\, 
\frac{dt}{t}\right)^{1/q},     
&  q < \infty,
\\[3mm]
\displaystyle\sup_{0 < t < \infty} \left(t^{1/p} f^{**}(t)\right),   
&   q = \infty,
\end{array}\right.
\]
is finite, where 
\[
f^{**}(x) := \frac1x\, \int_0^x f^*(t)\, dt
\]
(see \cite[Chap.~4, Lemma 4.5]{BS88}).
%%%----------------------------------------------------------------------------
\begin{theorem}\label{th:main-Lorentz}
Let $1<p<\infty$. The Lorentz space $L^{p,1}(\R)$ is a separable
and non-reflexive Banach function space satisfying assumption {\rm(a)}
of Theorem~{\rm\ref{th:main}} and such that the algebra $\cA_{L^{p,1}(\R)}$ 
does not contain all rank one operators.
\end{theorem}
%%%----------------------------------------------------------------------------
The paper is organized as follows. 
In Section~\ref{sec:auxiliary}, we collect definitions of a Banach function
space and its associate space $X'(\R)$, recall that the set of Fourier
multipliers $\cM_{X(\R)}$ on a separable Banach function space $X(\R)$, 
such that the Hardy-Littlewood maximal operator $M$ is bounded on $X(\R)$
and on its associate space $X'(\R)$, is continuously embedded into 
$L^\infty(\R)$. Consequently, $\cM_{X(\R)}$ is a unital Banach algebra.
Further, we prove several lemmas on approximation of continuous
functions (or Fourier multipliers) vanishing at infinity by
compactly supported continuous functions (or Fourier multipliers, respectively).

In Section~\ref{sec:commutativity}, we show that if $X(\R)$ is a
separable Banach function space  such that the Hardy-Littlewood maximal 
operator $M$ is bounded on $X(\R)$ and on its associate space $X'(\R)$
and $a\in C(\dR)$, $b\in C_X(\dR)$, then the commutator
$aW^0(b)-W^0(b)aI$ is compact on the space $X(\R)$. Combining this result 
with Theorem~\ref{th:FKK}, we arrive at Theorem~\ref{th:algebra-Api}.

Section~\ref{sec:main-proof} is devoted to the proof of 
Theorems~\ref{th:main} and~\ref{th:main-Lorentz}. 
For $R>0$, let $\chi_{\{R\}}:=\chi_{\R\setminus[-R,R]}$.
We show that
if $a$ is a compactly supported continuous function and
$b$ is a compactly supported function of bounded variation,
then the norm of the operator $aW^0(b)\chi_{\{R\}}I$  goes to zero as 
$R\to\infty$. If a Banach function space $X(\R)$ is separable and 
non-reflexive, its associate space $X'(\R)$ may contain a function
$g$ such that $\|g\chi_{\{R\}}\|_{X'(\R)}$ is bounded away from
zero for all $R>0$ (this cannot happen if $X(\R)$ is reflexive).
If, in addition, the Hardy-Littlewood operator is bounded on $X(\R)$
and on its associate space $X(\R)$, then we show that 
for every $h\in X(\R)\setminus\{0\}$ the rank one operator
\[
(T_{g,h}f)(x) := h(x) \int_\R g(y) f(y)\, dy 
\]
does not belong to the algebra $\cA_{X(\R)}$, which implies
Theorem~\ref{th:main} under the assumption that the 
function $g\in X'(\R)$  mentioned  above does indeed exist. 
Let $1<p<\infty$ and $1/p+1/p'=1$.
Finally, we prove Theorem~\ref{th:main-Lorentz} first recalling that
the classical Lorentz space $L^{p,1}(\R)$ is a separable non-reflexive 
Banach function space with the associate space $L^{p',\infty}(\R)$, that 
the Hardy-Littlewood maximal operator is bounded on both $L^{p,1}(\R)$ 
and $L^{p',\infty}(\R)$; and then showing that the function 
$g(x)=|x|^{-1/p'}$ belongs to $L^{p',\infty}(\R)$ and 
$\|\chi_{\{R\}}g\|_{(p',\infty)}$ is bounded away from zero for all 
$R>0$. This completes the proof of Theorem~\ref{th:main-Lorentz} and,
thus, of Theorem~\ref{th:main}.

In Section~\ref{sec:SO}, we define the algebra of continuous
Fourier multipliers $C_{X}^0(\dR)$ as the closure of $\C\dot{+} C_c^\infty(\R)$,
where $C_c^\infty(\R)$ is the set of smooth compactly supported functions
and $\C$ denotes the set of constant functions. It is not difficult
to see that $C_X^0(\dR)\subset C_X(\dR)$. We do not know whether
these algebras coincide, in general. We prove a possible refinement of 
Theorem~\ref{th:FKK} for the algebra $\cA^0_{X(\R)}$, where the latter
algebra is defined in the same way as the algebra $\cA_{X(\R)}$
with $C_X(\dR)$ replaced by $C_X^0(\dR)$. Further, we recall
the definition of the set of slowly oscillating functions $SO^\diamond$
and slowly oscillating Fourier multipliers $SO_{X(\R)}^\diamond$
(see \cite{FKK20,K15c}). Since $C(\dR)\subset SO^\diamond$ and
$C_X^0(\dR)\subset SO_{X(\R)}^\diamond$, we conclude that
the ideal of compact operators $\cK(X(\R))$ is contained
in the algebra $\mathcal{D}_{X(\R)}$ generated by the operators $aI$
with $a\in SO^\diamond$ and $b\in SO_{X(\R)}^\diamond$ under the assumptions
that $X(\R)$ is a reflexive Banach function space such that the 
Hardy-Littlewood maximal operator is bounded on $X(\R)$ and on its associate
space $X'(\R)$.
%%%----------------------------------------------------------------------------
\section{Auxiliary results}\label{sec:auxiliary}
\subsection{Banach function spaces}\label{sec:BFS}
The set of all Lebesgue measurable complex-valued functions on $\R$ is denoted
by $\fM(\R)$. Let $\fM^+(\R)$ be the subset of functions in $\fM(\R)$ whose
values lie  in $[0,\infty]$. For a measurable set $E\subset\R$, 
its Lebesgue measure and the characteristic function are denoted by $|E|$ and
$\chi_E$, respectively. Following \cite[Chap.~1, Definition~1.1]{BS88}, a 
mapping $\rho:\fM^+(\R)\to [0,\infty]$ is called a Banach function norm if,
for all functions $f,g, f_n \ (n\in\N)$ in $\fM^+(\R)$, for all
constants $a\ge 0$, and for all measurable subsets $E$ of $\R$,
the following properties hold:
\begin{eqnarray*}
{\rm (A1)} &\quad & \rho(f)=0  \Leftrightarrow  f=0\ \mbox{a.e.}, \
\rho(af)=a\rho(f), \
\rho(f+g) \le \rho(f)+\rho(g),\\
{\rm (A2)} &\quad &0\le g \le f \ \mbox{a.e.} \ \Rightarrow \ \rho(g)
\le \rho(f)
\quad\mbox{(the lattice property)},
\\
{\rm (A3)} &\quad &0\le f_n \uparrow f \ \mbox{a.e.} \ \Rightarrow \
       \rho(f_n) \uparrow \rho(f)\quad\mbox{(the Fatou property)},\\
{\rm (A4)} &\quad & |E|<\infty \Rightarrow \rho(\chi_E) <\infty,\\
{\rm (A5)} &\quad & |E|<\infty \Rightarrow \int_E f(x)\,dx \le C_E\rho(f),
\end{eqnarray*}
%%%%
where $C_E \in (0,\infty)$ may depend on $E$ and $\rho$ but is
independent of $f$. When functions differing only on a set of measure zero
are identified, the set $X(\R)$ of functions $f\in\fM(\R)$
for which $\rho(|f|)<\infty$ is called a Banach function space. For each
$f\in X(\R)$, the norm of $f$ is defined by
\[
\left\|f\right\|_{X(\R)} :=\rho(|f|).
\]
With this norm and under natural linear space operations, the set $X(\R)$ 
becomes a Banach space (see \cite[Chap.~1, Theorems~1.4 and~1.6]{BS88}). 
If $\rho$ is a Banach function norm, its associate norm $\rho'$ is defined on
$\fM^+(\R)$ by
\[
\rho'(g):=\sup\left\{
\int_{\R} f(x)g(x)\,dx \ : \ f\in \fM^+(\R), \ \rho(f) \le 1
\right\}, \quad g\in \fM^+(\R).
\]
By \cite[Chap.~1, Theorem~2.2]{BS88}, $\rho'$ is itself 
a Banach function norm.
The Banach function space $X'(\R)$ determined by the Banach function norm
$\rho'$ is called the associate space (K\"othe dual) of $X(\R)$.
The associate space $X'(\R)$ is a subspace of the (Banach) dual
space $[X(\R)]^*$.
%%%----------------------------------------------------------------------------
\subsection{Density of nice functions in Banach function spaces}
Let $C_c(\R)$ and $C_c^\infty(\R)$ denote the sets of continuous compactly
supported functions on $\R$ and infinitely differentiable compactly supported
functions on $\R$, respectively.
%%%----------------------------------------------------------------------------
\begin{lemma}\label{le:density}
Let $X(\R)$ be a separable Banach function space. Then the sets $C_c(\R)$,
$C_c^\infty(\R)$ and $L^2(\R)\cap X(\R)$ are dense in the space $X(\R)$.
\end{lemma}
%%%----------------------------------------------------------------------------
The density of $C_c(\R)$ and $C_c^\infty(\R)$ in $X(\R)$ is shown in 
\cite[Lemma~2.12]{KS14}. Since $C_c(\R)\subset L^2(\R)\cap X(\R)\subset X(\R)$,
we conclude that $L^2(\R)\cap X(\R)$ is dense in $X(\R)$.
%----------------------------------------------------------------------------
\subsection{Banach algebra \boldmath{$\cM_{X(\R)}$} of Fourier multipliers}
The following result plays an important role in this paper.
%%%----------------------------------------------------------------------------
\begin{theorem}
\label{th:continuous-embedding}
Let $X(\R)$ be a separable Banach function space such that the
Hardy-Littlewood maximal operator $M$ is bounded on  $X(\R)$ or
on its associate space $X'(\R)$. If $a\in\cM_{X(\R)}$, then
%%%
\begin{equation}\label{eq:continuous-embedding}
\|a\|_{L^\infty(\R)}\le\|a\|_{\cM_{X(\R)}}.
\end{equation}
%%%
The constant $1$ on the right-hand side of \eqref{eq:continuous-embedding}
is best possible.
\end{theorem}
%%%----------------------------------------------------------------------------
\begin{proof}
If the Hardy-Littlewood maximal operator $M$ is bounded on the space $X(\R)$
or on its associate space $X'(\R)$, then in view of \cite[Lemma~3.2]{H12}
we have
\[
\sup_{-\infty<a<b<\infty}
\frac{1}{b-a}\|\chi_{(a,b)}\|_{X(\R)}\|\chi_{(a,b)}\|_{X'(\R)}<\infty.
\]
If this condition is fulfilled, then inequality \eqref{eq:continuous-embedding}
follows from \cite[inequality (1.2) and Corollary~4.2]{KS19}.
\end{proof}
%%%----------------------------------------------------------------------------
Inequality \eqref{eq:continuous-embedding} was established earlier in
\cite[Theorem~1]{K15b} with some constant on the right-hand side that depends 
on the space $X(\R)$ under the assumption that the operator $M$ is bounded on 
both $X(\R)$ and $X'(\R)$ (see  also \cite[Theorem~2.4]{FKK-AFA}).

Since \eqref{eq:continuous-embedding} is available, an easy adaptation of
the proof of \cite[Proposition 2.5.13]{G14} leads to the following
(we refer to the proof of \cite[Corollary~1]{K15b} for details).
%%%----------------------------------------------------------------------------
\begin{corollary}
Let $X(\R)$ be a separable Banach function space such that the
Hardy-Littlewood maximal operator $M$ is bounded on $X(\R)$ or
on its associate space $X'(\R)$. Then the set of Fourier multipliers
$\cM_{X(\R)}$ is a Banach algebra under pointwise operations and the norm
$\|\cdot\|_{\cM_{X(\R)}}$.
\end{corollary}
%%%----------------------------------------------------------------------------
\subsection{Approximation of continuous functions vanishing at infinity}
Let $C_0(\R)$ denote the set of all continuous functions on $\R$ that vanish 
at $\pm\infty$. 
%%%----------------------------------------------------------------------------
\begin{lemma}\label{le:approximating-vanishing}
For a function $\upsilon \in C_c^\infty(\mathbb{R})$ such that 
$0 \le \upsilon \le 1$ and $\upsilon(x) = 1$ when $|x| \le 1$, let 
\[
\upsilon_n(x) := \upsilon(x/n), \quad x\in\R, \quad n \in \mathbb{N}.
\]
\begin{enumerate}
\item[(a)] If $a\in C_0(\R)$, then
%%%
\begin{equation}\label{eq:approximating-vanishing-1}
\lim_{n\to\infty}\|a-\upsilon_n a\|_{L^\infty(\R)}=0.
\end{equation}
%%%

\item[(b)] If $a\in C_0(\R)\cap V(\R)$, then
\begin{equation}\label{eq:approximating-vanishing-2}
\lim_{n\to\infty}\|a-\upsilon_n a\|_{V(\R)}=0.
\end{equation}
\end{enumerate}
\end{lemma}
%%%----------------------------------------------------------------------------
\begin{proof}
(a) If $a\in C_0(\R)$, then for every $\eps>0$ there exists $N\in\N$
such that 
\[
\sup_{x\in\R\setminus[-N,N]}|a(x)|<\frac{\eps}{2}.
\]
For all $n>N$ and $x\in[-N,N]$, we have $v_n(x)=1$. Since $0\le\upsilon_n\le 1$,
for $n>N$, we get
\[
\|a-\upsilon_n a\|_{L^\infty(\R)}=
\sup_{x\in\R\setminus[-N,N]}|a(x)-\upsilon_n(x)a(x)|
\le 2\sup_{x\in\R\setminus[-N,N]}|a(x)|<\eps,
\]
which completes the proof of equality \eqref{eq:approximating-vanishing-1}.

(b) Let $V(g;\Omega)$ denote the total variation of a function $g$ 
over a union of intervals $\Omega \subset \mathbb{R}$. Then for all $n\in\N$,
\begin{align}
V(a - \upsilon_n a) 
=& 
V(a(1 - \upsilon_n);\mathbb{R}\setminus [-n, n]) 
\nonumber\\
\le & 
V(a;\mathbb{R}\setminus [-n, n]) 
\|1 - \upsilon_n\|_{L^\infty(\mathbb{R}\setminus [-n, n])} 
\nonumber\\
&+ 
\|a\|_{L^\infty(\mathbb{R}\setminus [-n, n])} 
V(1 - \upsilon_n;\mathbb{R}\setminus [-n, n]) 
\nonumber\\
\le & 
V(a;\mathbb{R}\setminus [-n, n]) 
+ 
\|a\|_{L^\infty(\mathbb{R}\setminus [-n, n])} V(\upsilon).
\label{eq:approximating-vanishing-3}
\end{align}
%%%
Since $a\in C_0(\R)$, we have
\begin{equation}\label{eq:approximating-vanishing-4}
\lim_{n\to\infty}\|a\|_{L^\infty(\R\setminus[-n,n])}=0
\end{equation}
%%%
(see the proof of part (a)).
On the other hand,
%%%
\begin{equation}\label{eq:approximating-vanishing-5}
\lim_{n\to\infty}V(a;\mathbb{R}\setminus [-n, n])
=\lim_{n\to\infty}V(a)-V(a;\R\setminus[-n,n])=V(a)-V(a)=0.
\end{equation}
%%%
It follows from \eqref{eq:approximating-vanishing-3}--%
\eqref{eq:approximating-vanishing-5} that
%%%
\begin{equation}\label{eq:approximating-vanishing-6}
\lim_{n\to\infty}V(a-\upsilon_n a)=0.
\end{equation}
%%%
Combining equalities \eqref{eq:approximating-vanishing-1}
and \eqref{eq:approximating-vanishing-6},
we arrive at equality \eqref{eq:approximating-vanishing-2}.
\end{proof}
%%%----------------------------------------------------------------------------
\subsection{Approximation of continuous Fourier multipliers vanishing 
at infinity}
The first result in this section says that continuous Fourier
multipliers vanishing at infinity can be approximated by continuous
functions of bounded variation vanishing at infinity.
%%%----------------------------------------------------------------------------
\begin{lemma}\label{le:approximating-vanishing-multipliers}
Let $X(\R)$ be a Banach function space such that the Hardy-Little\-wood
maximal operator $M$ is bounded on $X(\R)$ and on its associate
space $X'(\R)$. If $b\in C_X(\dR)$ is such that $b(\infty)=0$, then
there exists a sequence $\{b_n\}_{n=1}^\infty$ of functions in
$C_0(\R)\cap V(\R)$ such that
\[
\lim_{n\to\infty}\|b_n-b\|_{\cM_{X(\R)}}=0.
\]
\end{lemma}
%%%----------------------------------------------------------------------------
\begin{proof}
It follows from the definition of $C_X(\dR)$ that there exists a sequence
$\{d_n\}_{n=1}^\infty$ in $C(\dR)\cap V(\R)$ such that
%%%
\begin{equation}\label{eq:approximating-vanishing-multipliers-1}
\lim_{n\to\infty}\|d_n-b\|_{\cM_{X(\R)}}=0.
\end{equation}
%%%
Take $b_n:=d_n-d_n(\infty)$. Then $b_n\in C_0(\R)\cap V(\R)$. It follows
\eqref{eq:approximating-vanishing-multipliers-1} and 
Theorem~\ref{th:continuous-embedding} that $\{d_n\}_{n=1}^\infty$ converges
uniformly to $b$ on $\R$. In particular,
%%%
\begin{equation}\label{eq:approximating-vanishing-multipliers-2}
\lim_{n\to\infty}d_n(\infty)=b(\infty)=0,
\end{equation}
%%%
Combining \eqref{eq:approximating-vanishing-multipliers-1} and
\eqref{eq:approximating-vanishing-multipliers-2}, we see that
%%%
\begin{align*}
\lim_{n\to\infty}\|b_n-b\|_{\cM_{X(\R)}}
&=
\lim_{n\to\infty}\|d_n-d_n(\infty)-b\|_{\cM_{X(\R)}}
\\
&\le
\lim_{n\to\infty}\|d_n-b\|_{\cM_{X(\R)}}
+\lim_{n\to\infty}|d_n(\infty)|=0,
\end{align*}
%%%
which completes the proof.
\end{proof}
%%%----------------------------------------------------------------------------

%%%---------------------------------------------------------------------------
\section{Commutativity of the algebra $\cA_{X(\R)}^\pi$}
\label{sec:commutativity}
\subsection{Compactness of convolution operators from a subspace of compactly 
supported functions of $L^1(\R)$ to a subspace of compactly
supported functions of $C(\R)$}
%%%---------------------------------------------------------------------------
Let $C^k(\mathbb{R})$, $k = 0, 1, 2, \dots$ be the space of functions with 
continuous bounded derivatives of all orders up to $k$,
$C(\mathbb{R}) = C^0(\mathbb{R})$. For any space of functions $Y(\mathbb{R})$ 
and any $R > 0$, let $Y_{[R]}(\mathbb{R})$ denote the subspace of 
$Y(\mathbb{R})$ consisting of functions with supports in $[-R, R]$.
As usual, the support of a function $f:\R\to\C$ will be denoted, by 
$\operatorname{supp}f$.
%%%---------------------------------------------------------------------------
\begin{lemma}\label{le:compact-convolution}
Suppose that $R_1,R_2>0$. If $k\in C^1(\R)$ is such that 
$\operatorname{supp}k\subset[-R_1,R_1]$, then the convolution operator with 
the kernel $k$ defined by
%%%
\begin{equation}\label{eq:compact-convolution}
(Kf)(x):=(k*f)(x)=\int_\R k(x-y)f(y)\,dy,
\quad x\in\R,
\end{equation}
%%%
is compact from the space $L^1_{[R_2]}(\R)$ to the space $C_{[R_1+R_2]}(\R)$.
\end{lemma}
%%%---------------------------------------------------------------------------
\begin{proof}
It follows from \cite[Propositions~4.18 and 4.20]{B11} that the operator
$K$ is bounded from the space $L^1_{[R_2]}(\R)$ to the space 
$C^1_{[R_1+R_2]}(\R)$. Further, by the Arzel\`a-Ascoli theorem
(see, e.g., \cite[Theorems~2.2.12 and 2.5.10]{PKJF13}), the space 
$C_{[R_1+R_2]}^1(\R)$ is compactly embedded into the space 
$C_{[R_1+R_2]}(\R)$, which completes the proof.
\end{proof}
%%%---------------------------------------------------------------------------
\subsection{Compactness of products of Fourier convolution
operators and multiplication operators}
The main step in the proof of Theorem~\ref{th:algebra-Api} consists
of proving the following. 
%%%---------------------------------------------------------------------------
\begin{theorem}\label{th:compactness-products}
Let $X(\R)$ be a separable Banach function space such that the Hardy-Littlewood 
maximal operator $M$ is bounded on  $X(\R)$ and on its associate 
space $X'(\R)$. If $a \in C(\dR)$ and $b \in C_X(\dR)$ are such 
that $a(\infty)= b(\infty)=0$, then 
\[
aW^0(b), W^0(b)aI \in \mathcal{K}(X(\mathbb{R})).
\]
\end{theorem}
%%%---------------------------------------------------------------------------
\begin{proof}
A part of the proof is quite standard (see, e.g., 
\cite[Theorem 5.3.1(i)]{RSS11}).
It follows from Lemma~\ref{le:approximating-vanishing-multipliers} that there 
exists a sequence $\{b_n\}_{n=1}^\infty$ of functions in $C_0(\R)\cap V(\R)$
such that $\|b_n-b\|_{\cM_{X(\R)}}\to 0$ as $n\to\infty$. Then
\[
\|aW^0(b)-aW^0(b_n)\|_{\cB(X(\R))}\to 0,
\quad
\|W^0(b)aI-W^0(b_n)aI\|_{\cB(X(\R))}\to 0
\]
as $n\to\infty$. So, we can assume without loss of generality that 
$b\in C_0(\R)\cap V(\R)$.

Let $\{\upsilon_n\}_{n=1}^\infty$ be the sequence of functions 
in $C_c^\infty(\R)$ as in Lemma~\ref{le:approximating-vanishing}. It
follows from the Stechkin type inequality \eqref{eq:Stechkin} that for all 
$n\in\N$,
%%%
\begin{align*}
&
\|aW^0(b)-\upsilon_n aW^0(\upsilon_n b)\|_{\cB(X(\R))}
\\
&\quad
\le 
\|(a-\upsilon_n a)W^0(b)\|_{\cB(X(\R))}
+
\|\upsilon_n a W^0(b-\upsilon_n b)\|_{\cB(X(\R))}
\\
&\quad
\le 
\|a-\upsilon_n a\|_{L^\infty(\R)}\|b\|_{\cM_{X(\R)}}
+ 
c_X\|a\|_{L^\infty(\R)}\|b-\upsilon_n b\|_{V(\R)}.
\end{align*}
This inequality and Lemma~\ref{le:approximating-vanishing} imply that
\[
\lim_{n\to\infty}\|aW^0(b)-\upsilon_n aW^0(\upsilon_n b)\|_{\cB(X(\R))}=0.
\]
Analogously we can show that
\[
\lim_{n\to\infty}\|W^0(b)aI-W^0(\upsilon_n b)\upsilon_n a I\|_{\cB(X(\R))}=0.
\]
Since $\upsilon_n a\in C_c(\R)$ and $\upsilon_n b\in C_c(\R)\cap V(\R)$
for all $n\in\N$, we can assume without loss of generality that 
$a\in C_c(\R)$ and $b\in C_c(\R)\cap V(\R)$.

Take $a_0, b_0 \in C_c^\infty(\mathbb{R})$ such that $a_0 a = a$ and 
$b_0 b = b$. Then 
\[
aW^0(b) = a(a_0W^0(b_0))W^0(b) , \quad W^0(b)aI  = W^0(b)(W^0(b_0)a_0I )aI .
\]
Hence it is enough to prove that 
$a_0W^0(b_0), W^0(b_0)a_0I \in \mathcal{K}(X(\mathbb{R}))$.

Since $F^{-1}b_0 \in \mathcal{S}(\mathbb{R})$, it is easy to see that 
$\upsilon_n F^{-1}b_0 \to F^{-1}b_0$ in $\mathcal{S}(\mathbb{R})$ as
$n \to \infty$. Then $b_n := F\left(\upsilon_n F^{-1}b_0\right) \to b_0$ 
in $\mathcal{S}(\mathbb{R})$ as $n\to\infty$. It is easy to see that the 
convergence in $\mathcal{S}(\R)$ implies the convergence in $V(\R)$.
Therefore $\|b_n - b_0\|_{V(\mathbb{R})} \to 0$ as $n \to \infty$. 
It follows from the Stechkin type inequality \eqref{eq:Stechkin} that
%%%
\begin{align*}
&
\lim_{n\to\infty}\|a_0W^0(b_n)-a_0W^0(b_0)\|_{\cB(X(\R))}
\le 
c_X\|a_0\|_{L^\infty(\R)}\lim_{n\to\infty}\|b_n-b_0\|_{V(\R)}
=0,
\\
&
\lim_{n\to\infty}\|W^0(b_n)a_0I-W^0(b_0)a_0I\|_{\cB(X(\R))}
\le 
c_X\|a_0\|_{L^\infty(\R)}\lim_{n\to\infty}\|b_n-b_0\|_{V(\R)}
=0.
\end{align*}
%%%
Thus, it is sufficient to prove that
$a_0W^0(b_n), W^0(b_n)a_0I \in \mathcal{K}(X(\mathbb{R}))$ for all $n\in\N$. 
Let 
$k_n:=F^{-1}b_n=\upsilon_n F^{-1}b_0\in C_c^\infty(\R)$.
It follows from the convolution theorem for the inverse Fourier
transform (see, e.g., \cite[Proposition~2.2.11, statement (12)]{G14}) that
for all $n\in\N$ and $f\in C_c^\infty(\R)$,
%%%
\begin{align}\label{eq:compactness-products-1}
W^0(b_n)f
&=
F^{-1}(b_nFf)=(F^{-1}b_n)*F^{-1}(Ff)
\nonumber\\
&=
(F^{-1}b_n)*f=k_n*f=:K_nf,
\end{align}
%%%
where $K_n$ is the convolution operator with the kernel $k_n$ defined by
\eqref{eq:compact-convolution}.
In view of Lemma~\ref{le:density}, equality \eqref{eq:compactness-products-1}
remains valid for all $f\in X(\R)$. 

Take $R_1,R_2>0$ such that $\operatorname{supp}k_n\subset[-R_1,R_2]$
and $\operatorname{supp}a_0\subset[-R_2,R_2]$. Equality 
\eqref{eq:compactness-products-1} implies that
\[
a_0W^0(b_n)=a_0K_n\chi_{[-R_1-R_2,R_1+R_2]}I.
\]
It follows from Axiom (A5) that there exists 
$C_{[-R_1-R_2,R_1+R_2]}\in(0,\infty)$ such that for all $f\in X(\R)$,
\[
\int_{-R_1-R_2}^{R_1+R_2}|f(x)|dx
\le 
C_{[-R_1-R_2,R_1+R_2]} \|f\|_{X(\R)},
\]
which means that the operator $\chi_{[-R_1-R_2,R_1+R_2]}I$ is bounded
from the space $X(\R)$ to the space $L^1_{[R_1+R_2]}(\R)$.
By Lemma~\ref{le:compact-convolution}, the operator $K_n$ is compact
from the space $L^1_{[R_1+R_2]}(\R)$ to the space $C_{[2R_1+R_2]}(\R)$.
It follows from Axiom (A2) that the operator 
$a_0I: C_{[2R_1+R_2]}(\R)\to X(\R)$ is bounded. Thus, for every $n\in\N$,
the operator $a_0W^0(b_n):X(\R)\to X(\R)$ is compact as the composition
of the bounded operator $\chi_{[-R_1-R_2,R_1+R_2]}I:X(\R)\to 
L^1_{[R_1+R_2]}(\R)$, the compact operator $T_n:L_{[R_1+R_2]}^1(\R)\to
C_{[2R_1+R_2]}(\R)$, and the bounded operator $aI:C_{[2R_1+R_2]}(\R)\to
X(\R)$.

Similarly, for every $n\in\N$, the operator $W^0(b_n) a_0I:X(\R)\to X(\R)$
is compact as the composition of the bounded operator
$a_0I:X(\R)\to L_{[R_2]}^1(\R)$, the compact operator
$K_n:L_{[R_2]}^1(\R)\to C_{[R_1+R_2]}(\R)$, and the bounded
operator $I:C_{[R_1+R_2]}(\R)\to X(\R)$.
\end{proof}
%%%---------------------------------------------------------------------------
\subsection{Compactness of commutators of Fourier convolution
operators and multiplication operators} 
The previous theorem implies the following.
%%%---------------------------------------------------------------------------
\begin{corollary}\label{co:compactness-commutators}
Let $X(\R)$ be a separable Banach function space such that the Hardy-Littlewood 
maximal operator $M$ is bounded on $X(\R)$ and on its associate 
space $X'(\R)$. If $a \in C(\dR)$ and $b \in C_X(\dR)$, then 
\[
[aI, W^0(b)] := aW^0(b) - W^0(b)aI \in \mathcal{K}(X(\mathbb{R})).
\]
\end{corollary}
%%%----------------------------------------------------------------------------
\begin{proof}
Since $a = a(\infty) + \widetilde{a}$ and $b = b(\infty) + \widetilde{b}$, 
where $\widetilde{a} \in C(\dR)$, $\widetilde{b} \in C_X(\dR)$, and 
$\widetilde{a}(\infty) = 0 = \widetilde{b}(\infty)$,
Theorem~\ref{th:compactness-products} implies that
%%%
\begin{align*}
[aI, W^0(b)] 
&= 
(a(\infty) + \widetilde{a}) (b(\infty) + W^0(\widetilde{b})) 
- 
(b(\infty) + W^0(\widetilde{b}))(a(\infty) + \widetilde{a}) I 
\\
& = 
\widetilde{a} W^0(\widetilde{b}) - W^0(\widetilde{b}) \widetilde{a}I 
\in \mathcal{K}(X(\mathbb{R})).
\qedhere
\end{align*}
\end{proof}
%%%----------------------------------------------------------------------------
\subsection{Proof of Theorem~\ref{th:algebra-Api}}
Since a Banach function space $X(\R)$ is reflexive if and only if the space 
$X(\R)$ and its associate space $X'(\R)$ are separable 
(see \cite[Chap.~1, Corollaries 4.4 and 5.6]{BS88}), 
Theorem~\ref{th:algebra-Api} follows from Theorem~\ref{th:FKK} and
Corollary~\ref{co:compactness-commutators}.
\qed

%%%----------------------------------------------------------------------------
\section{Proof of the main result}
\label{sec:main-proof}
%%%----------------------------------------------------------------------------
\subsection{Estimate for the norm of a product of multiplication operators and
a Fourier convolution operator}
For $n\in\N_0:=\N\cup\{0\}$, let
\[
\ell_n(x) := \frac{\log^n (1 + |x|)}{1 + |x|}, \quad x \in \mathbb{R}.
\]
%%%----------------------------------------------------------------------------
\begin{lemma}\label{le:ell-n}
If $Y(\R)$ is a Banach function space such that the Hardy-Littlewood
maximal operator $M$ is bounded on it, then $\ell_n \in Y(\mathbb{R})$
for all $n\in\N_0$.
\end{lemma}
%%%----------------------------------------------------------------------------
\begin{proof}
Since $\chi_{[-1, 1]} \in Y(\mathbb{R})$ by Axiom (A4), 
$M\chi_{[-1, 1]} \in Y(\mathbb{R})$. It is easy to see
that $0 \le \ell_0 \le M\chi_{[-1, 1]}$ (see \cite[Example 2.1.4]{G14}). 
Hence $\ell_0 \in Y(\mathbb{R})$ in view of Axiom (A2).

Now let $k\in\N_0$. It follows from the definition of the 
Hardy-Littlewood maximal operator that for $x\ne 0$,
\[
(M\ell_k)(x)
\ge 
\left\{\begin{array}{lll}
\displaystyle
\frac{1}{x+\eps}\int_0^{x+\eps}\frac{\log^k(1+|t|)}{1+|t|}dt
&\mbox{if}& x,\eps>0,
\\[4mm]
\displaystyle
\frac{1}{-x-\eps}\int_{x+\eps}^0\frac{\log^k(1+|t|)}{1+|t|}dt
&\mbox{if}& x,\eps<0.
\end{array}\right.
\]
Passing to the limit as $\eps\to 0^\pm$, we obtain for $x\ne 0$,
\begin{align*}
(M\ell_k)(x)
&\ge 
\left\{\begin{array}{lll}
\displaystyle
\frac{1}{x}\int_0^{x}\frac{\log^k(1+|t|)}{1+|t|}dt
&\mbox{if}& x>0,
\\[4mm]
\displaystyle
\frac{1}{-x}\int_{x}^0\frac{\log^k(1+|t|)}{1+|t|}dt
&\mbox{if}& x<0
\end{array}\right.
=
\frac{1}{|x|}\int_0^{|x|}\frac{\log^k(1+t)}{1+t}dt
\\
&=
\frac{1}{(k+1)|x|}\log^{k+1}(1+|x|)
\ge
\frac{1}{k+1}\ell_{k+1}(x).
\end{align*}
So
\[
0\le\ell_{k+1}\le(k+1)M\ell_k,\quad k\in\N_0,
\]
and one gets by induction that $\ell_n\in Y(\R)$ for all $n\in\N_0$.
\end{proof}
%%%----------------------------------------------------------------------------
For $R>0$, let $\chi_{\{R\}} := \chi_{\mathbb{R}\setminus[-R, R]}$.
%%%----------------------------------------------------------------------------
\begin{theorem}\label{th:key-estimate}
Let $X(\mathbb{R})$ be a separable Banach function space such 
that the Hardy-Littlewood maximal operator is bounded on $X(\R)$ 
and on its associate space $X'(\mathbb{R})$. 
Let $a \in C_c(\R)$ and $b \in C_c(\mathbb{R})\cap V(\mathbb{R})$. 
Then for every $n \in \mathbb{N}_0$, there exists a constant 
$c_n(a,b)\in (0,\infty)$ depending only on $a,b$ and $n$,
such that for all $R>0$,
%%%
\begin{equation}\label{eq:key-estimate-0}
\|aW^0(b)\chi_{\{R\}}I\|_{\cB(X(\R))} \le \frac{c_n(a,b)}{\log^n(R + 2)}.
\end{equation}
\end{theorem}
%%%----------------------------------------------------------------------------
\begin{proof}
Since $b\in C_c(\R)\subset L^1(\R)$, it follows from the convolution theorem
for the inverse Fourier transform (see, e.g., \cite[Theorem~11.66]{A20})
that for $f\in C_c^\infty(\R)$,
%%%
\begin{equation}\label{eq:key-estimate-1}
W^0(b)f=F^{-1}(b\cdot Ff)=(F^{-1}b)*F^{-1}(Ff)=:k*f,
\end{equation}
%%%
where $k:=F^{-1}b$. In view of Lemma~\ref{le:density}, formula
\eqref{eq:key-estimate-1} remains valid for all $f\in X(\R)$.
Since $b\in V(\R)$, using integration by parts, similarly to the proof
of \cite[Chap.~I, Theorem~4.5]{K76}, we get for $x\in\R$,
\[
k(x)
=
(F^{-1}b)(x)
=
\frac{1}{2\pi}\int_\R e^{-ix\xi}b(\xi)\,d\xi
=
\frac{1}{2\pi ix}\int_\R e^{-ix\xi}db(\xi),
\]
and hence
%%%
\begin{equation}\label{eq:key-estimate-2}
|k(x)|\le \frac{V(b)}{2\pi|x|},
\quad
x\in\R.
\end{equation}
%%%
Take $R_1>0$ such that $\operatorname{supp}a\subset[-R_1,R_1]$. If
$x\in[-R_1,R_1]$ and $|y|>R\ge\max\{2R_1,1\}$, then
%%%
\begin{equation}\label{eq:key-estimate-3}
|x-y|\ge|y|-|x|\ge|y|-R_1\ge|y|-\frac{|y|}{2}=\frac{|y|}{2}\ge\frac{|y|+1}{4}
\end{equation}
%%%
and
%%%
\begin{equation}\label{eq:key-estimate-4}
\log(R+1)\ge\frac{1}{2}\log(R+2).
\end{equation}
%%%
Combining \eqref{eq:key-estimate-2}--\eqref{eq:key-estimate-4} and
taking into account the definition of $\ell_n$, we get for every
$x\in[-R_1,R_1]$, $R\ge\max\{2R_1,1\}$, and $n\in\N_0$,
%%%
\begin{align}
|k(x-y)|\chi_{\{R\}}(y)
&\le 
\frac{V(b)}{2\pi|x-y|}\chi_{\{R\}}(y)
\le 
\frac{2V(b)}{\pi(1+|y|)}\chi_{\{R\}}(y)
\nonumber\\
&\le 
\frac{2V(b)}{\pi\log^n(R+1)}\ell_n(y)
\le 
\frac{2^{n+1}V(b)}{\pi\log^n(R+2)}\ell_n(y).
\label{eq:key-estimate-5}
\end{align}
%%%
It follows from \eqref{eq:key-estimate-5}, Lemma~\ref{le:ell-n} and
H\"older's inequality for Banach function spaces (see 
\cite[Chap.~1, Theorem~2.4]{BS88}) that for $x\in[-R_1,R_1]$, 
$R\ge\max\{2R_1,1\}$, $n\in\N_0$ and $f\in X(\R)$,
%%%
\begin{align}
|k*(\chi_{\{R\}}f)(x)|
&=
\left|\int_\R k(x-y)\chi_{\{R\}}(y)f(y)\,dy\right|
\nonumber\\
&\le 
\|k(x-\cdot)\chi_{\{R\}}\|_{X'(\R)}\|f\|_{X(\R)}
\nonumber\\
&\le 
\frac{2^{n+1}V(b)\|\ell_n\|_{X'(\R)}\|f\|_{X(\R)}}{\pi\log^n(R+2)}.
\label{eq:key-estimate-6}
\end{align}
%%%
It follows from Axiom (A4) that $\chi_{[-R_1,R_1]}\in X(\R)$.
Since $\operatorname{supp}a\subset[-R_1,R_1]$, in view of Axiom (A2), 
equality \eqref{eq:key-estimate-1} and inequality \eqref{eq:key-estimate-6},
we obtain for $R\ge\max\{2R_1,1\}$, $f\in X(\R)$ and $n\in\N_0$,
%%%
\begin{align}
\|aW^0(b)\chi_{\{R\}}f\|_{X(\R)}
&\le 
\|a\|_{L^\infty(\R)}\|\chi_{[-R_1,R_1]}\|_{X(\R)}
\operatornamewithlimits{ess\,sup}_{x\in[-R_1,R_1]}|k*(\chi_{\{R\}}f)(x)|
\nonumber\\
&\le 
\frac{2^{n+1}V(b)\|\ell_n\|_{X'(\R)}\|\chi_{[-R_1,R_1]}\|_{X(\R)}}
{\pi\log^n(R+2)}
\|f\|_{X(\R)}.
\label{eq:key-estimate-7}
\end{align}
%%%
If $R\in(0,\max\{2R_1,1\})$, then
$\log(R+2)\le\log(2+\max\{2R_1,1\})$ and
%%%
\begin{align}
\|aW^0(b)\chi_{\{R\}}I\|_{\cB(X(\R))}
&\le 
\|aW^0(b)\|_{\cB(X(\R))}
\nonumber\\
&\le 
\frac{\log^n(2+\max\{2R_1,1\})\|aW^0(b)\|_{\cB(X(\R))}}{\log^n(R+2)}.
\label{eq:key-estimate-8}
\end{align}
%%%
It follows from \eqref{eq:key-estimate-7} and \eqref{eq:key-estimate-8}
that \eqref{eq:key-estimate-0} is fulfilled with
\begin{align*}
c_n(a,b):=\max\bigg\{ &
\frac{2^{n+1}}{\pi}V(b)\|\ell_n\|_{X'(\R)}\|\chi_{[-R_1,R_1]}\|_{X(\R)},
\\
&
\log^n(2+\max\{2R_1,1\})\|aW^0(b)\|_{\cB(X(\R))}\bigg\},
\end{align*}
which completes the proof.
\end{proof}
%%%----------------------------------------------------------------------------
\subsection{Sufficient condition on the space $X(\R)$ implying that the 
algebra $\cA_{X(\R)}$ does not contain all rank one operators}
Now we prove a conditional statement, which will lead 
to the proof of Theorem~\ref{th:main}.
%%%----------------------------------------------------------------------------
\begin{theorem}\label{th:main-key-step}
Let $X(\mathbb{R})$ be a separable non-reflexive Banach function space such 
that the Hardy-Littlewood maximal operator is bounded on $X(\R)$ 
and on its associate space $X'(\mathbb{R})$. Suppose that there exist a 
function $g \in X'(\mathbb{R})$ and a constant $\delta > 0$ such that 
$\|\chi_{\{R\}}g\|_{X'(\mathbb{R})} \ge \delta$ for all $R > 0$. Then for 
any function $h \in X(\mathbb{R})\setminus\{0\}$, the rank one operator 
$T_{g,h} \in \mathcal{B}(X(\mathbb{R}))$, defined by
\[
(T_{g,h}f)(x) := h(x) \int_\R g(y) f(y)\, dy ,
\]
does not belong to the algebra $\mathcal{A}_{X(\mathbb{R})}$.
\end{theorem}
%%%---------------------------------------------------------------------------
\begin{proof}
Fix $h\in X(\R)\setminus\{0\}$. Suppose the contrary: $T_{g,h}\in\cA_{X(\R)}$.
Fix $\eps>0$. By the definition of the algebra $\cA_{X(\R)}$ there exist
numbers $N,M\in\N$ and operators 
\[
A_{ij}\in\{aI,W^0(b)\ : \ a\in C(\dR),\ b\in C_X(\dR)\}
\]
for $i\in\{1,\dots,N\}$, $j\in\{1,\dots,M\}$ such that
%%%
\begin{equation}\label{eq:main-key-step-1}
\left\|
T_{g,h}-\sum_{i=1}^N A_{i1}\dots A_{iM}
\right\|_{\cB(X(\R))}<\frac{\eps}{6}.
\end{equation}
%%%
Put
%%%
\begin{equation}\label{eq:main-key-step-2}
L:=2\max\big\{
\|A_{ij}\|_{\cB(X(\R))}
\ :\
i\in\{1,\dots,M\},\
j\in\{1,\dots,M\}\big\}
\end{equation}
%%%
Let $b_1,\dots,b_r\in C_X(\dR)$ be such that for $k\in\{1,\dots,r\}$,
\[
W^0(b_k)\in
\big\{A_{ij} \ :\
i\in\{1,\dots,M\},\
j\in\{1,\dots,M\}\big\} 
\setminus
\big\{aI\ : \ a\in C(\dR)\big\}
\]
and $a_1,\dots, a_s\in C(\dR)$ be such that for $l\in\{1,\dots,s\}$,
\[
a_lI
\in
\big\{A_{ij} \ :\
i\in\{1,\dots,M\},\
j\in\{1,\dots,M\}\big\} 
\setminus
\big\{W^0(b_k)\ : \ k\in\{1,\dots,r\}\big\}.
\]
It follows from the definition of the algebra $C_X(\dR)$ that for every 
$k\in\{1,\dots,r\}$ there exists a function $c_k\in C(\dR)\cap V(\R)$
such that
%%%
\begin{equation}\label{eq:main-key-step-3}
\|W^0(b_k)-W^0(c_k)\|_{\cB(X(\R))}
=
\|b_k-c_k\|_{\cM_{X(\R)}}
<
\min\left\{\frac{\eps}{6NML^{M-1}},\frac{L}{4}\right\}.
\end{equation}
%%%
Further, in view of Lemma~\ref{le:approximating-vanishing}(b)
and the Stechkin-type inequality \eqref{eq:Stechkin}, there exists
a function $\widetilde{b}_k\in C_c(\R)\cap V(\R)$ such that
%%%
\begin{align}
\|W^0(c_k)-c_k(\infty)I-W^0(\widetilde{b}_k)\|_{\cB(X(\R))}
&=
\|c_k-c_k(\infty)-\widetilde{b}_k\|_{\cM_{X(\R))}}
\nonumber\\
&\le 
c_X\|c_k-c_k(\infty)-\widetilde{b}_k\|_{V(\R)}
\nonumber\\
&<
\min\left\{\frac{\eps}{6NML^{M-1}},\frac{L}{4}\right\}.
\label{eq:main-key-step-4}
\end{align}
%%%
Combining \eqref{eq:main-key-step-3} and \eqref{eq:main-key-step-4},
we get
\[
\|W^0(b_k)-c_k(\infty)I-W^0(\widetilde{b}_k)\|_{\cB(X(\R))}
<
\min\left\{\frac{\eps}{3NML^{M-1}},\frac{L}{2}\right\}.
\]
Analogously, by Lemma~\ref{le:approximating-vanishing}(a), for every
$l\in\{1,\dots,s\}$, there exists $\widetilde{a}_l\in C_c(\R)$
such that
\[
\|a_lI-a_l(\infty)I-\widetilde{a}_lI\|_{\cB(X(\R))}
\le 
\|a_l-a_l(\infty)-\widetilde{a}_l\|_{L^\infty(\R)}
<
\min\left\{\frac{\eps}{3NML^{M-1}},\frac{L}{2}\right\}.
\]
We have shown that for every $i\in\{1,\dots,N\}$ and $j\in\{1,\dots,N\}$
there exists an operator
%%%
\begin{equation}\label{eq:main-key-step-5}
B_{ij}\in\big\{cI+\widetilde{a}I,cI+W^0(\widetilde{b})\ :\
c\in\C,\ \widetilde{a}\in C_c(\R),\ \widetilde{b}\in C_c(\R)\cap V(\R)\big\}
\end{equation}
%%%
such that
\[
\|A_{ij}-B_{ij}\|_{\cB(X(\R))}
<
\min\left\{\frac{\eps}{3NML^{M-1}},\frac{L}{2}\right\}.
\]
Then, taking into account \eqref{eq:main-key-step-2}, we get
%%%
\begin{align}
&
\left\|
\sum_{i=1}^N A_{i1}\dots A_{iM}
-
\sum_{i=1}^N B_{i1}\dots B_{iM}
\right\|_{\cB(X(\R))}
\nonumber\\
&\quad=
\left\|
\sum_{i=1}^N\sum_{j=1}^M 
A_{i1}\dots A_{i,j-1}
(A_{ij}-B_{ij})B_{i,j+1}\dots B_{iM}
\right\|_{\cB(X(\R))}
\nonumber\\
&\quad\le
\sum_{i=1}^N\sum_{j=1}^M
\left(\prod_{k=1}^{j-1}\|A_{ik}\|_{\cB(X(\R))}\right)
\|A_{ij}-B_{ij}\|_{\cB(X(\R))} 
\left(\prod_{l=j+1}^{M}\|B_{il}\|_{\cB(X(\R))}\right)
\nonumber\\
&\quad<
\sum_{i=1}^N\sum_{j=1}^M
\left(\frac{L}{2}\right)^{j-1}
\frac{\eps}{3NML^{M-1}}
\left(\frac{L}{2}+\frac{L}{2}\right)^{M-j}
<
\sum_{i=1}^N\sum_{j=1}^M
\frac{\eps}{3NM}
=
\frac{\eps}{3}.
\label{eq:main-key-step-6}
\end{align}
%%%
It follows from \eqref{eq:main-key-step-1} and
\eqref{eq:main-key-step-6} that
%%%
\begin{equation}\label{eq:main-key-step-7}
\|T_{g,h}-T_\eps\|_{\cB(X(\R))}
<
\frac{\eps}{6}+\frac{\eps}{3}
=
\frac{\eps}{2},
\end{equation}
%%%
where
\[
T_\eps:=\sum_{i=1}^N B_{i1}\dots B_{iM}.
\]
Taking into account \eqref{eq:main-key-step-5}, we can rearrange
terms and write the operator $T_\eps$ in the form
%%%
\begin{equation}\label{eq:main-key-step-8}
T_\eps=cI+W^0(\widetilde{b}_0)
+
\sum_{i=1}^p D_{1,i}\widetilde{a}_{1,i}I
+
\sum_{j=1}^t D_{2,j}\widetilde{a}_{2,j}W^0(\widetilde{b}_j),
\end{equation}
%%%
where $c\in\C$, $\widetilde{b}_j\in C_c(\R)\cap V(\R)$
for $j\in\{0,\dots,t\}$, 
$\widetilde{a}_{1,i},\widetilde{a}_{2,j}\in C_c(\R)$
and $D_{1,i},D_{2,j}$ are some operators in
$\cA_{X(\R)}\setminus\{0\}$
for $i\in\{1,\dots,p\}$ and $j\in\{1,\dots,t\}$.

Since the space $X(\R)$ is separable, it follows from 
\cite[Chap.~1, Definition~3.1 and Corollary~5.6]{BS88} that there
exists $R_1>0$ such that 
$\|\chi_{\{R_1\}}h\|_{X(\R)}\le\frac{1}{2}\|h\|_{X(\R)}$. Then
%%%
\begin{equation}\label{eq:main-key-step-9}
\|\chi_{R_1}h\|_{X(\R)}
\ge 
\|h\|_{X(\R)}-\|\chi_{\{R_1\}}h\|_{X(\R)}
\ge 
\frac{1}{2}\|h\|_{X(\R)},
\end{equation}
%%%
where 
\[
\chi_{R_1}:=1-\chi_{\{R_1\}}=\chi_{[-R_1,R_1]}.
\]
Since $\widetilde{a}_{1,i}\in C_c(\R)$ for $i=1,\dots,p$, there
exists $R_2>R_1$ such that for $R\ge R_2$,
%%%
\begin{equation}\label{eq:main-key-step-10}
\chi_{R_1}(cI)\chi_{\{R\}}I
+
\chi_{R_1}\sum_{i=1}^p D_{1,i}\widetilde{a}_{1,i}\chi_{\{R\}}I=0.
\end{equation}
%%%
Let $\widetilde{a}_0\in C_c(\R)$ be such that $\widetilde{a}_0=1$
for $x\in[-R_1,R_1]$. Then
\[
\chi_{R_1}W^0(\widetilde{b}_0)=\chi_{R_1}\widetilde{a}_0 W^0(\widetilde{b}_0).
\]
It follows from Theorem~\ref{th:key-estimate} that there exists 
$R_0>R_2$ such that for all $R\ge R_0$ and $j\in\{1,\dots,t\}$,
%%%
\begin{align}
&
\|\chi_{R_1}\widetilde{a}_0W^0(\widetilde{b}_0)\chi_{\{R\}}I\|_{\cB(X(\R))}
\le 
\|\widetilde{a}_0W^0(\widetilde{b}_0)\chi_{\{R\}}I\|_{\cB(X(\R))}
<
\frac{\eps}{2(t+1)},
\label{eq:main-key-step-11}
\\
&
\|\widetilde{a}_{2,j}W^0(\widetilde{b}_j)\chi_{\{R\}}I\|_{\cB(X(\R))}
<
\frac{\eps}{2(t+1)\|D_{2,j}\|_{\cB(X(\R))}}.
\label{eq:main-key-step-12}
\end{align}
%%%
Combining \eqref{eq:main-key-step-8} and
\eqref{eq:main-key-step-10}--\eqref{eq:main-key-step-12}, we see that
for all $R\ge R_0$,
%%%
\begin{equation}\label{eq:main-key-step-13}
\|\chi_{R_1}T_\eps\chi_{\{R\}}I\|_{\cB(X(\R))}
<\frac{\eps}{2(t+1)}+\sum_{j=1}^t\frac{\eps}{2(t+1)}=\frac{\eps}{2}.
\end{equation}
%%%
It follows from \eqref{eq:main-key-step-7} and \eqref{eq:main-key-step-13}
that for all $R\ge R_0$,
%%%
\begin{align}
\|\chi_{R_1} T_{g,h}\chi_{\{R\}}I\|_{\cB(X(\R))}
&
\le
\|\chi_{R_1} (T_{g,h}-T_\eps)\chi_{\{R\}}I\|_{\cB(X(\R))}
+
\|\chi_{R_1} T_\eps\chi_{\{R\}}I\|_{\cB(X(\R))}
\nonumber\\
&\le
\|T_{g,h}-T_\eps\|_{\cB(X(\R))}+\frac{\eps}{2}<\eps.
\label{eq:main-key-step-14}
\end{align}

On the other hand, in view of \cite[Chap.~1, Lemma~2.8]{BS88},
we have
%%%
\begin{align*}
&
\|\chi_{R_1} T_{g,h}\chi_{\{R\}}I\|_{\cB(X(\R))}
\\
&\quad=
\sup\left\{
\left\|\chi_{R_1}h\int_\R g(y)\chi_{\{R\}}(y)f(y)\,dy\right\|_{X(\R)}
\ :\
f\in X(\R),\ \|f\|_{X(\R)}\le 1
\right\}
\\
&\quad=
\sup\left\{
\left|\int_\R g(y)\chi_{\{R\}}(y)f(y)\,dy\right|
\|\chi_{R_1}h\|_{X(\R)}
\ :\
f\in X(\R),\ \|f\|_{X(\R)}\le 1
\right\}
\\
&\quad=\|\chi_{R_1}h\|_{X(\R)}
\sup\left\{
\left|\int_\R g(y)\chi_{\{R\}}(y)f(y)\,dy\right|
\ :\
f\in X(\R),\ \|f\|_{X(\R)}\le 1
\right\}
\\
&\quad=
\|\chi_{R_1}h\|_{X(\R)}\|g\chi_{\{R\}}\|_{X'(\R)}.
\end{align*}
%%%%
This equality, inequality \eqref{eq:main-key-step-9} and inequality
$\|\chi_{\{R\}}g\|_{X'(\R)}\ge\delta$ imply that
%%%
\begin{equation}\label{eq:main-key-step-15}
\|\chi_{R_1}T_{g,h}\chi_{\{R\}}I\|_{\cB(X(\R))}
\ge\frac{\delta}{2}\|h\|_{X(\R)}.
\end{equation}
%%%
Inequalities \eqref{eq:main-key-step-14} and \eqref{eq:main-key-step-15}
yield a contradiction for $\eps\le\frac{\delta}{2}\|h\|_{X(\R)}$.
\end{proof}
%%%----------------------------------------------------------------------------
\begin{remark}
Note that a Banach function spaces $X(\R)$ is reflexive if and only if
$X(\R)$ and its associate space $X'(\R)$ are separable (see 
\cite[Chap.~1, Corollaries~4.4 and 5.6]{BS88}). In turn, if $X'(\R)$ is 
separable, then for any $g\in X'(\R)$ one has 
$\|\chi_{\{R\}}g\|_{X'(\R)}\to 0$ as $R\to\infty$ in view of 
\cite[Chap.~1, Definition~3.1 and Corollary~5.6]{BS88}.
\end{remark}
%%%----------------------------------------------------------------------------
To complete the proof of Theorem~\ref{th:main}, we have to show that
there exists a separable non-reflexive Banach function space satisfying the
hypotheses of Theorem~\ref{th:main-key-step}. In the next subsection, we will
show that the classical Lorentz spaces $L^{p,1}(\R)$, $1<p<\infty$,
perfectly fit our needs. 
%%%---------------------------------------------------------------------------
\subsection{Proof of Theorem~\ref{th:main-Lorentz}}
The space $X(\mathbb{R}) = L^{p, 1}(\mathbb{R})$ is separable and 
\[
\left[L^{p, 1}(\mathbb{R})\right]^* 
= 
\left(L^{p, 1}\right)' (\mathbb{R})
= 
L^{p', \infty}(\mathbb{R}),
\]
where $1/p + 1/p' = 1$ (see \cite[Chap. 1, Corollaries 4.3 and 5.6, 
Chap. 4, Corollary 4.8]{BS88}). 
It is also known that 
\[
L^{p,1}(\R)\subsetneqq [L^{p',\infty}(\R)]^*=[L^{p,1}(\R)]^{**}
\]
(see \cite[p.~83]{C75}). Hence $L^{p,1}(\R)$ is non-reflexive.
The lower and upper Boyd indices of 
$L^{p, 1}(\mathbb{R})$ (resp., of $L^{p', \infty}(\mathbb{R})$) are both 
equal to $1/p$ (resp., to $1/p'$); see \cite[Chap. 4, Theorem 4.6]{BS88}. 
Hence the Hardy-Littlewood maximal operator is bounded on the space 
$X(\mathbb{R})$ and on its associate space $X'(\mathbb{R})$ in view of the 
Lorentz-Shimogaki theorem (see \cite[Chap. 3, Theorem 5.17]{BS88}). Thus, the 
space $L^{p,1}(\R)$ is a separable non-reflexive Banach function space
satisfying condition (a) of Theorem~\ref{th:main}.

Consider the function $g(x)=|x|^{-1/p'}$. Its distribution function is
\[
\mu_g(\lambda)
=
|\{x\in\R:|x|^{-1/p'}>\lambda\}|
=
|\{x\in\R:|x|<\lambda^{-p'}\}|=2\lambda^{-p'}, 
\quad
\lambda\ge 0,
\]
and its non-increasing rearrangement is
\[
g^*(t)
=
\inf\{\lambda\ge 0: 2\lambda^{-p'}\le t\}
=
\inf\{\lambda\ge 0: 2^{1/p'}t^{-1/p'}\le \lambda\}
=
2^{1/p'}t^{-1/p'},
\ t\ge 0.
\]
Then
\[
g^{**}(t)
=
\frac{1}{t}\int_0^t 2^{1/p'}y^{-1/p'}dy
=\frac{2^{1/p'}t^{-1/p'}}{1-1/p'}
=
2^{1/p'}pt^{-1/p'},
\quad t\ge 0
\]
and
\[
\|g\|_{(p',\infty)}=2^{1/p'}p<\infty.
\]
The distribution function of $\chi_{\{R\}}g$ for every $R>0$ is given by
%%%
\begin{align*}
\mu_{\chi_{\{R\}}g}(\lambda)
&=
|\{x\in\R:\chi_{\{R\}}(x)g(x)>\lambda\}|
=
\left\{\begin{array}{lll}
2\lambda^{-p'}-2R & \mbox{if}& 0\le\lambda<R^{-1/p'},
\\
0 &\mbox{if}& \lambda\ge R^{-1/p'}.
\end{array}\right.
\end{align*}
%%%
Then
%%%
\begin{align*}
(\chi_{\{R\}}g)^*(t)
&=
\inf\{\lambda\ge 0 : 2\lambda^{-p'}-2R\le t\}
=
\inf\left\{\lambda\ge 0:\lambda^{-p'}\le\frac{t}{2}+R\right\}
\\
&=
\inf\left\{\lambda\ge 0:\frac{2}{t+2R}\le\lambda^{p'}\right\}
=
2^{1/p'}(t+2R)^{-1/p'},
\quad
t\ge 0.
\end{align*}
%%%
Since $(\chi_{\{R\}}g)^*$ is non-increasing, we have
$(\chi_{\{R\}}g)^{**}\ge (\chi_{\{R\}}g)^*$ and
\[
\|\chi_{\{R\}}g\|_{(p',\infty)}
\ge 
\sup_{0<t<\infty}\left(t^{1/p'}(\chi_{\{R\}}g)^*(t)\right)
=
2^{1/p'}\sup_{0<t<\infty}
\left(\frac{t}{t+2R}\right)^{1/p'}
=
2^{1/p'}.
\]
Thus, the conditions of Theorem~\ref{th:main-key-step} are satisfied for
$X(\R)=L^{p,1}(\R)$, $g(x)=|x|^{-1/p'}$ and $\delta=2^{1/p'}$.
The desired result now follows from that theorem.
\qed
%%%----------------------------------------------------------------------------
\section{Final remarks on algebras of convolution type operators with 
continuous and slowly oscillating data}
\label{sec:SO}
\subsection{Algebra $C_X^0(\dR)$ of continuous Fourier multipliers}
%%%----------------------------------------------------------------------------
We first show that the Schwartz functions lie in the closure of the set 
of smooth compactly supported functions with respect to the multiplier norm.
%%%----------------------------------------------------------------------------
\begin{lemma}\label{le:Schwartz-in-CX0}
Let $X(\R)$ be a Banach function space such that the Hardy-Little\-wood
maximal operator $M$ is bounded on $X(\R)$ and on its associate
space $X'(\R)$. Then 
\[
\mathcal{S}(\R)\subset \operatorname{clos}_{\cM_{X(\R)}}\big(C_c^\infty(\R)\big).
\]
\end{lemma}
%%%----------------------------------------------------------------------------
\begin{proof}
If $a\in\mathcal{S}(\R)$, then in view of the mean value theorem,
$a\in C_0(\R)\cap V(\R)$. Moreover, $\upsilon_n a\in C_c^\infty(\R)$
for all $n\in\N$, where $\upsilon_n$ are the functions from 
Lemma~\ref{le:approximating-vanishing}. Combining 
Lemma~\ref{le:approximating-vanishing}(b) with the Stechkin-type inequality 
\eqref{eq:Stechkin}, we see that
\[
\lim_{n\to\infty}\|a-\upsilon_n a\|_{\cM_{X(\R)}}
\le 
c_X\lim_{n\to\infty}\|a-\upsilon_n a\|_{V(\R)}=0.
\]
Hence 
$a\in\operatorname{clos}_{\cM_{X(\R)}}\big(C_c^\infty(\R)\big)$.
\end{proof}
%%%----------------------------------------------------------------------------
Let $\C$ stand for the constant complex-valued functions on $\R$. Notice that 
$C(\dR)$ decomposes into the direct sum $C(\dR)=\C\mathbf{\dot{+}}C_0(\R)$. 
It follows from the mean value theorem
that
%%%
\begin{equation}\label{eq:embedding-of-classes-of-conitnuous-functions}
\C\mathbf{\dot{+}}C_c^\infty(\R)\subset C(\dR)\cap V(\R).
\end{equation}

Suppose $X(\R)$ is a separable Banach function space such that the 
Hardy-Little\-wood maximal operator $M$ is bounded on $X(\R)$
and on its associate space $X'(\R)$. Along with the algebra
$C_X(\dR)$ of continuous Fourier multipliers defined by
\eqref{eq:algebra-CX}, consider the following algebra of continuous 
Fourier multipliers:
%%%
\begin{equation}\label{eq:algebra-CX0}
C_{X}^0(\dR):=\operatorname{clos}_{\cM_{X(\R)}}
\big(\C\mathbf{\dot{+}}C_c^\infty(\R)\big).
\end{equation}
%%%


It follows from embeddings 
\eqref{eq:embedding-of-classes-of-conitnuous-functions} and definitions 
\eqref{eq:algebra-CX} and \eqref{eq:algebra-CX0} that
%%%
\begin{equation}\label{eq:embeddings-of-classes-of-continuous-multipliers}
C_{X}^0(\dR)\subset C_{X}(\dR).
\end{equation}

For large classes of Banach function spaces, including separable
rearrangement-invariant Banach function with nontirvial Boyd indices,
weighted Lebesgue spaces with Muckenhoupt weights, reflexive variable 
Lebesgue spaces $L^{p(\cdot)}(\R)$ such that the Hardy-Littlewood
maximal operator $M$ is bounded on $L^{p(\cdot)}(\R)$, the above
embedding becomes equality (see \cite[Theorem~3.3]{FK20} and
\cite[Theorem~1.1]{K20}). Proofs of \cite[Theorem~3.3]{FK20}
and \cite[Theorem~1.1]{K20} are based on an interpoaltion argument. 
Unfortunately, interpolation tools are not available in the general setting
of Banach function spaces. So, we arrive at the following.
%%%----------------------------------------------------------------------------
\begin{question}\label{question-1}
Is it true that $C_{X}^0(\dR)= C_{X}(\dR)$ for an arbitrary
separable Banach function space $X(\R)$ such that the Hardy-Littlewood
maximal operator is bounded on $X(\R)$ and on its associate space
$X'(\R)$?
\end{question}

%%%----------------------------------------------------------------------------
\subsection{The ideal of compact operators is contained in the 
algebra of convolution type operators with continuous data}
Since we do not know the answer on Question~\ref{question-1}, 
along with the Banach algebra $\cA_{X(\R)}$, we will also consider
the smallest Banach subalgebra
\[
\cA_{X(\R)}^0
:=
\operatorname{alg}\{aI,W^0(b)\ :\ a\in C(\dR),\ b\in C_X^0(\dR)\}
\]
of the algebra $\cB(X(\R))$ that contain all operators of multiplication $aI$ 
by functions $a\in C(\dR)$ and all Fourier convolution operators $W^0(b)$ with 
symbols $b\in C_X^0(\dR)$.

If the answer to Question~\ref{question-1} is negative, then 
the following result provides a refinement of Theorem~\ref{th:FKK}.
%%%----------------------------------------------------------------------------
\begin{theorem}\label{th:FKK-refined}
Let $X(\R)$ be a reflexive Banach function space such that the Hardy-Littlewood 
maximal operator $M$ is bounded on $X(\R)$ and on its associate 
space $X'(\R)$.  Then the ideal of compact operators $\cK(X(\R))$ is contained 
in the Banach algebra $\cA_{X(\R)}^0$.
\end{theorem}
%%%----------------------------------------------------------------------------
The proof of Theorem~\ref{th:FKK-refined} repeats word-by-word the proof
of Theorem~\ref{th:FKK} with \cite[Lemma~4.2]{FKK19} replaced by the 
following.
%%%----------------------------------------------------------------------------
\begin{lemma}\label{le:one-dimensional-operator}
Let $X(\R)$ be a separable Banach function space such that the Hardy-Littlewood 
maximal operator $M$ is bounded on $X(\R)$ and on its associate 
space $X'(\R)$. Suppose $a,b\in C_c(\R)$ and a
one-dimensional operator $T_1$ is defined on the space $X(\R)$ by
%%%
\begin{equation}\label{eq:one-dimensional-operator-1}
(T_1f)(x)=a(x)\int_\R b(y)f(y)\,dy.
\end{equation}
%%%
Then there exists a function $c\in C_{X}^0(\dR)$ 
such that $T_1=aW^0(c)bI$.
\end{lemma}
%%%----------------------------------------------------------------------------
\begin{proof}
The idea of the proof is borrowed from \cite[Lemma~6.1]{KILH13a}
(see also \cite[Proposition~5.8.1]{RSS11}). Since $a,b\in C_c(\R)$, there
exists a number $M>0$ such that the set $\{x-y:x\in\operatorname{supp}a,
y\in\operatorname{supp}b\}$ is contained in the segment $[-M,M]$ for
certain $M>0$. By the smooth version of Urysohn's lemma (see, e.g.,
\cite[Proposition~6.5]{F09}), there exists $k\in C_c^\infty(\R)$
such that $0\le k(x)\le 1$ for $x\in\R$, $k(x)=1$ for $x\in[-M,M]$ and $k(x)=0$
for $x\in\R\setminus(-2M,2M)$. Then \eqref{eq:one-dimensional-operator-1}
can be rewritten in the form
\[
(T_1f)(x)=a(x)\int_\R k(x-y)b(y)f(y)\,dy=\big(aW^0(\widehat{k})bf\big)(x),
\quad
x\in\R.
\]
It follows from \cite[Example~2.2.2 and Proposition~2.2.11]{G14} that
$C_c^\infty(\R)\subset\mathcal{S}(\R)$ and $\widehat{k}\in\mathcal{S}(\R)$.
By Lemma~\ref{le:Schwartz-in-CX0}, 
$c:=\widehat{k}\in\mathcal{S}(\R)\subset
\operatorname{clos}_{\cM_{X(\R)}}\big(C_c^\infty(\R)\big)\subset C_X^0(\dR)$.
\end{proof}
%%%----------------------------------------------------------------------------
\subsection{Slowly oscillating Fourier multipliers}
\label{sec:SO-multipliers}
For a set $E\subset\dR$ and a function
$f:\dR\to\C$ in $L^\infty(\R)$, let the oscillation of
$f$ over $E$ be defined by
\[
\operatorname{osc}(f,E)
:=
\operatornamewithlimits{ess\,sup}_{s,t\in E}|f(s)-f(t)|.
\]
Following \cite[Section~4]{BFK06} and
\cite[Section~2.1]{KILH12}, \cite[Section~2.1]{KILH13a},
we say that a function $f\in L^\infty(\R)$ is slowly
oscillating at a point $\lambda\in\dR$ if for every $r\in(0,1)$ or,
equivalently, for some $r\in(0,1)$, one has
\[
\begin{array}{lll}
\lim\limits_{x\to 0+}
\operatorname{osc}\big(f,\lambda+([-x,-rx]\cup[rx,x])\big)=0
&\mbox{if}& \lambda\in\R,
\\
\lim\limits_{x\to +\infty}
\operatorname{osc}\big(f,[-x,-rx]\cup[rx,x]\big)=0
&\mbox{if}& \lambda=\infty.
\end{array}
\]
For every $\lambda\in\dR$, let $SO_\lambda$ denote the $C^*$-subalgebra of
$L^\infty(\R)$ defined by
\[
SO_\lambda:=\left\{f\in C_b(\dR\setminus\{\lambda\})\ :\ f
\mbox{ slowly oscillates at }\lambda\right\},
\]
where
$C_b(\dR\setminus\{\lambda\}):=C(\dR\setminus\{\lambda\})\cap L^\infty(\R)$.

Let $SO^\diamond$ be the smallest $C^*$-subalgebra of $L^\infty(\R)$ that
contains all the $C^*$-algebras $SO_\lambda$ with $\lambda\in\dR$.
The functions in $SO^\diamond$ are called slowly oscillating functions.


For a point $\lambda\in\dR$, let $C^3(\R\setminus\{\lambda\})$ be the set of
all three times continuously differentiable functions
$a:\R\setminus\{\lambda\}\to\C$.
Following \cite[Section~2.4]{KILH12} and \cite[Section~2.3]{KILH13a}, consider
the commutative Banach algebras
\[
SO_\lambda^3:=\left\{
a\in SO_\lambda\cap C^3(\R\setminus\{\lambda\})\ :\
\lim_{x\to\lambda}(D_\lambda^k a)(x)=0,
\ k=1,2,3
\right\}
\]
equipped with the norm
\[
\|a\|_{SO_\lambda^3}:=
\sum_{j=0}^3\frac{1}{j!}\left\|D_\lambda^ka\right\|_{L^\infty(\R)},
\]
where $(D_\lambda a)(x)=(x-\lambda) a'(x)$ for $\lambda\in\R$ and
$(D_\lambda a)(x)=xa'(x)$ for $\lambda=\infty$.

The following result leads us to the definition of slowly oscillating
Fourier multipliers.
%%%----------------------------------------------------------------------------
\begin{theorem}[{\cite[Theorem~2.5]{K15c}}]
\label{th:boundedness-convolution-SO}
Let $X(\R)$ be a separable Banach function space such that the
Hardy-Littlewood maximal operator $M$ is bounded on $X(\R)$ and on its
associate space $X'(\R)$. If $\lambda\in\dR$ and $a\in SO_\lambda^3$, then
the convolution operator $W^0(a)$ is bounded on the space $X(\R)$ and
%%%
\[
\|W^0(a)\|_{\cB(X(\R))}
\le
d_{X}\|a\|_{SO_\lambda^3},
\]
where $d_{X}$ is a positive constant depending only on $X(\R)$.
\end{theorem}
%%%----------------------------------------------------------------------------
Let $SO_{\lambda,X(\R)}$ denote the closure of $SO_\lambda^3$ in the norm of
$\cM_{X(\R)}$. Further, let $SO_{X(\R)}^\diamond$ be the smallest Banach
subalgebra of $\cM_{X(\R)}$ that contains all the Banach algebras
$SO_{\lambda,X(\R)}$ for $\lambda\in\dR$. The functions in
$SO_{X(\R)}^\diamond$ will be called slowly oscillating Fourier multipliers.
%%%----------------------------------------------------------------------------
\subsection{The ideal of compact operators is contained in the 
algebra of convolution type operators with slowly oscillating data}
Consider the smallest Banach subalgebra
\[
\mathcal{D}_{X(\R)}
:=
\operatorname{alg}\{aI,W^0(b)\ :\ a\in SO^\diamond,\ b\in SO^\diamond_{X(\R)}\}
\]
of the algebra $\cB(X(\R))$ that contain all operators of multiplication $aI$ 
by slowly oscillating functions $a\in SO^\diamond$ and all Fourier convolution
operators $W^0(b)$ with slowly oscillating symbols $b\in SO_{X(\R)}^\diamond$.

Now we are in a position to formulate the main result of this section.
%%%----------------------------------------------------------------------------
\begin{theorem}\label{th:compacts-in-D}
Let $X(\R)$ be a reflexive Banach function space such that the Hardy-Littlewood 
maximal operator $M$ is bounded on $X(\R)$ and on its associate space 
$X'(\R)$. Then the ideal of compact operators $\cK(X(\R))$ is contained in the 
Banach algebra $\mathcal{D}_{X(\R)}$.
\end{theorem}
%%%----------------------------------------------------------------------------
\begin{proof}
It is clear that $C(\dR)\subset SO_\infty$. It is also easy to see that
\[
\C\dot{+} C_c^\infty(\R)\subset SO_\infty^3. 
\]
Hence
\[
C_X^0(\dR)\subset SO_{\infty,X(\R)}\subset SO_{X(\R)}^\diamond.
\]
Thus $\cA_{X(\R)}^0\subset\mathcal{D}_{X(\R)}$ and the desired result
follows from Theorem~\ref{th:FKK-refined}.
\end{proof}
%%%----------------------------------------------------------------------------
Under the assumptions of Theorem~\ref{th:compacts-in-D}, we can define
the quotient algebra
\[
\mathcal{D}_{X(\R)}^\pi:=\mathcal{D}_{X(\R)}/\mathcal{K}(X(\R)).
\]
We conclude this section with the following.
%%%----------------------------------------------------------------------------
\begin{question}
Let $X(\R)$ be a reflexive Banach function space such that the Hardy-Littlewood 
maximal operator $M$ is bounded on $X(\R)$ and on its associate space 
$X'(\R)$. Is it true that the quotient algebra $\mathcal{D}_{X(\R)}^\pi$
is commutative?
\end{question}
%%%----------------------------------------------------------------------------
We know that the answer is positive for some particular cases of Banach 
function spaces.
For Lebesgue spaces $L^p(\R,w)$, $1<p<\infty$, with Muckenhoupt weights $w$,
the positive answer to the above question follows from 
\cite[Theorem~4.6]{KILH13a}, whose proof relies on a version of the 
Krasnosel'skii interpolation theorem for compact operators
(see, e.g., \cite[Corollary~5.3]{K12}).
The answer is also positive for reflexive variable Lebesgue spaces
$L^{p(\cdot)}(\R)$ such that the Hardy-Littlewood maximal operator $M$
is bounded on $L^{p(\cdot)}(\R)$. It is based on a similar interpolation
argument (see \cite[Lemma 6.4]{K15a}). However, as far as we know, for 
arbitrary Banach function spaces, interpolation tools are not available.
%%%----------------------------------------------------------------------------

\bigskip
\textbf{Acknowledgments.} 
This work was partially supported by the Funda\c{c}\~ao para a Ci\^encia e a
Tecnologia (Portu\-guese Foundation for Science and Technology)
through the project
UIDB/MAT/00297/2020 (Centro de Matem\'atica e Aplica\c{c}\~oes).

We are grateful to Helena Mascarenhas, who asked the first author about
a possibility of refinement of Theorem~\ref{th:FKK} contained in 
Theorem~\ref{th:FKK-refined}.
%%%----------------------------------------------------------------------------
\bibliographystyle{amsplain}
\begin{thebibliography}{99}
\bibitem{A20}
S. Axler,
\textit{Measure, Integration \& Real Analysis}.
Springer, Berlin, 2020.

\bibitem{BFK06}
M. A. Bastos, C. A. Fernandes, and Yu. I. Karlovich,
\textit{$C^*$-algebras of integral operators with piecewise slowly
oscillating coefficients and shifts acting freely},
Integr. Equ. Oper. Theory \textbf{55} (2006), 19--67.

\bibitem{BS88} 
C. Bennett and R. Sharpley, 
\textit{Interpolation of Operators}.
Academic Press, Boston, 1988. 

\bibitem{BK97}
A. B\"ottcher and Yu. I. Karlovich,
\textit{Carleson Curves, Muckenhoupt Weights, and Toeplitz Operators}.
Birkh\"auser Verlag, Basel, Boston, Berlin, 1997.

\bibitem{BS06}
A, B\"ottcher and B. Silbermann,
\textit{Analysis of Toeplitz Operators.}
2nd ed., Springer, Berlin, 2006

\bibitem{B11}
H. Brezis, 
\textit{Functional Analysis, Sobolev Spaces and Partial Differential 
Equations}.
Springer, New York, 2011.

\bibitem{C75}
M. Cwikel,
\textit{The dual of weak $L^p$}.
Ann. Inst. Fourier \textbf{25} (1975), no. 2, 81--126.

\bibitem{D79}
R. Duduchava, 
\textit{Integral Equations with Fixed Singularities}.
Teubner Verlagsgesellschaft, Leipzig, 1979.

\bibitem{FK20}
C. A. Fernandes and A. Yu. Karlovich, 
\textit{Semi-almost periodic Fourier multipliers on rearrangement-invariant 
spaces with suitable Muckenhoupt weights}. 
Bolet{\'\i}n de la Sociedad Matem\'atica Mexicana (2020),
doi: 10.1007/s40590-020-00276-1.

\bibitem{FKK-AFA}
C. A. Fernandes, A. Yu. Karlovich, and Yu. I. Karlovich,
\textit{Noncompactness of Fourier convolution operators on Banach function
spaces},
Ann. Funct. Anal. AFA \textbf{10} (2019), 553--561.

\bibitem{FKK19}
C. A. Fernandes, A. Yu. Karlovich, and Yu. I. Karlovich, 
\textit{Algebra of convolution type operators with continuous data on 
Banach function spaces}. 
Banach Center Publications \textbf{119} (2019), 157--171.

\bibitem{FKK20}
C. A. Fernandes, A. Yu. Karlovich, and Yu. I. Karlovich,
\textit{Calkin images of Fourier convolution operators with slowly 
oscillating symbols},
Oper. Theor. Adv. Appl. \textbf{282} (2020), in press.

\bibitem{F09}
G. B. Folland,
\textit{A Guide to Advanced Real Analysis}.
The Mathematical Association of America,
Washington, DC, 2009.

\bibitem{G14}
L. Grafakos, 
\textit{Classical Fourier Analysis}. 3rd ed.,
Springer, New York, 2014.

\bibitem{H12}
K.-P. Ho,
\textit{Atomic decomposition of Hardy spaces and
characterization of BMO via Banach function spaces}.
Analysis Mathematica \textbf{38} (2012), 173--185.

\bibitem{K15a}
A. Yu. Karlovich, 
\textit{Maximally modulated singular integral operators and their applications
to pseudodifferential operators on Banach function spaces}.
Contemp. Math. \textbf{645} (2015), 165--178.

\bibitem{K15b}
A. Yu. Karlovich,
\textit{Banach algebra of the Fourier multipliers on weighted Banach 
function spaces}.
Concr. Oper. \textbf{2} (2015), 27--36.

\bibitem{K15c}
A. Yu. Karlovich,
\textit{Commutators of convolution type operators on some Banach function 
spaces}.
Ann. Funct. Anal. AFA \textbf{6} (2015), 191--205.

\bibitem{K20}
A. Yu. Karlovich, 
\textit{Algebras of continuous Fourier multipliers on variable Lebesgue spaces}.
Mediter. J. Math. \textbf{17} (2020), paper 102, 19 pages.

\bibitem{KS19}
A. Karlovich and E. Shargorodsky,
\textit{When does the norm of a Fourier multiplier dominate its $L^\infty$ norm?}
Proc. London Math. Soc. \textbf{118} (2019), 901--941.

\bibitem{KS14}
A. Yu. Karlovich and I. M. Spitkovsky, 
\textit{The Cauchy singular integral operator on weighted variable Lebesgue 
spaces}.
Oper. Theor. Adv. Appl. \textbf{236} (2014), 275--291.

\bibitem{K12}
Yu. I. Karlovich, 
\textit{Boundedness and compactness of pseudodifferential operators with
non-regular symbols on weighted Lebesgue spaces}.
Integr. Equ. Oper. Theor. \textbf{73} (2012), 217--254.

\bibitem{KILH12}
Yu. I. Karlovich and I. Loreto Hern\'andez, 
\textit{Algebras of convolution type pperators with piecewise slowly oscillating 
data. I: Local and structural study}.
Integr. Equ. Oper. Theor. \textbf{74} (2012), 377--415.

\bibitem{KILH13a}
Yu. I. Karlovich and I. Loreto Hern\'andez, 
\textit{On convolution type operators with piecewise slowly oscillating data}.
Oper. Theor. Adv. Appl. \textbf{228} (2013), 185--207.

\bibitem{KILH13b}
Yu. I. Karlovich and I. Loreto Hern\'andez,
\textit{Algebras of convolution type operators with piecewise slowly 
oscillating data. II: Local spectra and Fredholmness.}
Integr. Equ. Oper. Theor. \textbf{75} (2013), 49–-86.

\bibitem{K76} 
Y. Katznelson, 
\textit{An Introduction to Harmonic Analysis}.
Dower Publications, New York, 1976.

\bibitem{PKJF13}
L. Pick, A. Kufner, O. John, and S. Fu\v{c}ík,
\textit{Function Spaces}.
de Gruyter, 2013.

\bibitem{RSS11}
S. Roch, P. A. Santos, and B. Silbermann,
\textit{Non-Commutative Gelfand Theories. A Tool-kit for Operator Theorists
and Numerical Analysts}.
Springer, Berlin, 2011.
\end{thebibliography}

\end{document}

%------------------------------------------------------------------------------
% End of journal.tex
%------------------------------------------------------------------------------
