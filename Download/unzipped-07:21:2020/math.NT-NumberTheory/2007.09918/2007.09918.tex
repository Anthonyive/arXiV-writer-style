\documentclass[12pt]{amsart}

\usepackage{amssymb, amscd, txfonts}
\usepackage{graphicx}
\usepackage[all]{xy}

%%%%%%% Layout %%%%%%%%%%%%%%%%%%%%%
%\setlength{\textwidth}{16cm}
%\setlength{\oddsidemargin}{0cm}
%\setlength{\evensidemargin}{0cm}
%\setlength{\topmargin}{0cm}
%\setlength{\textheight}{22.5cm}
 
\numberwithin{equation}{section}

\renewcommand{\thefootnote}{\fnsymbol{footnote}}

\sloppy

%%%%%%% Theoremstyle %%%%%%%%%%%%%%%%%%
\newtheorem{theorem}{Theorem}[section]
\newtheorem{proposition}[theorem]{Proposition}
\newtheorem{lemma}[theorem]{Lemma}
\newtheorem{corollary}[theorem]{Corollary}
\newtheorem{reduction}[theorem]{Reduction}

\theoremstyle{definition}
\newtheorem{definition}[theorem]{Definition}
\newtheorem{example}[theorem]{Example}
\newtheorem*{conjecture}{Conjecture}
\newtheorem{condition}[theorem]{Condition}

\theoremstyle{remark}
\newtheorem{remark}[theorem]{Remark}
\newtheorem{claim}[theorem]{Claim}


%%%%%%% Macro %%%%%%%%%%%%%%%%%%%%%%
\renewcommand{\hom}{\operatorname{Hom}}
\renewcommand{\ker}{\operatorname{Ker}}

\newcommand{\N}{\mathbb{N}}
\newcommand{\Z}{\mathbb{Z}}
\newcommand{\Q}{\mathbb{Q}}
\newcommand{\R}{\mathbb{R}}
\newcommand{\C}{\mathbb{C}}
\newcommand{\HH}{\mathbb{H}}
\newcommand{\proj}{{\mathbb P}}
\newcommand{\QZ}{\mathbb{Q}/\mathbb{Z}}
\newcommand{\e}{{\mathbf e}}
\newcommand{\emu}{{\mathbf e}_{\mu}}
\newcommand{\elambda}{{\mathbf e}_{\lambda}}

\newcommand{\SL}{{\rm SL}_2(\mathbb{Z})}
\newcommand{\Mp}{{\rm Mp}_2(\mathbb{Z})}
\newcommand{\pullL}{\uparrow_{L}^{L'}}
\newcommand{\pushL}{\downarrow_{L}^{L'}}
\newcommand{\pushLK}{\downarrow^{L}_{K}}
\newcommand{\ThetaK}{\Theta_{K^{+}}}
\newcommand{\ML}{M^{!}(L)}
\newcommand{\MLZ}{M^{!}(L)_{\mathbb{Z}}}
\newcommand{\MLR}{M^{!}(L)_{R}}
\newcommand{\fTheta}{\langle f\downarrow^{L}_{K}, \Theta_{K^{+}} \rangle}
\newcommand{\foneTheta}{\langle f_{1}\downarrow^{L}_{K}, \Theta_{K^{+}} \rangle}
\newcommand{\ftwoTheta}{\langle f_{2}\downarrow^{L}_{K}, \Theta_{K^{+}} \rangle}

\DeclareMathOperator{\Ad}{Ad}
\DeclareMathOperator{\aut}{Aut}
\DeclareMathOperator{\coker}{Coker}
\DeclareMathOperator{\im}{Im}
\DeclareMathOperator{\out}{Out}
\DeclareMathOperator{\sgn}{sgn}
\DeclareMathOperator{\tr}{tr}
\DeclareMathOperator{\vol}{Vol}


\begin{document}

%%%%%%% Title %%%%%%%%%%%%%%%%%%%%%%%%
\title[]{Algebra of Borcherds products}
\author[]{Shouhei Ma}
\thanks{Supported by JSPS KAKENHI 17K14158 and 20H00112.} 
\address{Department~of~Mathematics, Tokyo~Institute~of~Technology, Tokyo 152-8551, Japan}
\email{ma@math.titech.ac.jp}
\subjclass[2010]{11F37, 16S99, 11F27, 11F55, 11F50}
\keywords{Weil representation, modular forms, Borcherds products, associative ring} 
%\dedicatory{}
\maketitle

\begin{abstract}
Borcherds lift for an even lattice $L$ of signature $(p, q)$ is 
a lifting from weakly holomorphic modular forms of 
weight $(p-q)/2$ for the Weil representation of $L$. 
We introduce a product operation on the space of such modular forms, 
depending on the choice of a maximal isotropic sublattice of $L$,  
which makes this space a finitely generated filtered associative algebra, without unit element in general. 
%This is done for each choice of a maximal isotropic sublattice of $L$. 
This algebra structure is functorial with respect to embedding of lattices by the quasi-pullback map. 
We study the basic properties, 
prove for example that 
the algebra is commutative if and only if $L$ is unimodular. 
When $L$ is unimodular with $p=2$, 
the multiplicative group of Borcherds products of integral weight 
forms a subring. 
\end{abstract} 


%%%%%%%%Introduction%%%%%%%%%%%

\section{Introduction}

Since ancient, mathematicians have introduced and studied product structures 
on various mathematical objects. 
In this paper we define a product structure on a space of 
certain vector-valued modular forms of fixed weight attached to an integral quadratic form, 
that is functorial and that reflects some properties of the quadratic form. 

Let $L$ be an even lattice of signature $(p, q)$ with $p\leq q$ and  
$\rho_{L}$ be the Weil representation attached to the discriminant form of $L$. 
In \cite{Bo95}, \cite{Bo98}, Borcherds constructed a lifting from 
weakly holomorphic modular forms $f$ of weight $\sigma(L)/2=(p-q)/2$ and type $\rho_{L}$ 
to automorphic forms $\Phi(f)$ with remarkable singularity on the symmetric domain attached to $L$. 
When $p=2$ and the principal part of $f$ has integral coefficients, 
$\Phi(f)$ gives rise to a meromorphic modular form $\Psi(f)$ with infinite product expansion, 
known as Borcherds product. 

The discovery of Borcherds has stimulated the study of 
weakly holomorphic modular forms of weight $\sigma(L)/2$ and type $\rho_{L}$.  
If we consider the space of such modular forms,  
say ${\ML}$, 
it is a priori just an infinite dimensional ${\C}$-linear space. 
The purpose of this paper is to define a product operation on the space ${\ML}$, 
%essentially one for each maximal boundary component of a locally symmetric space attached to $L$, 
depending on the choice of a maximal isotropic sublattice of $L$ up to the action of an arithmetic group, 
which makes ${\ML}$ an associative ${\C}$-algebra, finitely generated and filtered but without unit element in general. 
Moreover, this product is functorial with respect to embedding of lattices by the so-called quasi-pullback operation. 
This gives a link between quadratic forms and noncommutative rings. 

In order to state our result, 
we assume that the lattice $L$ has Witt index $p$ ($=$ maximal) and  
%We write $\sigma(L)=b_{+}-b_{-}$ and let 
%${\ML}=M^{!}_{\sigma(L)/2}(\rho_{L})$ be the space of 
%weakly holomorphic modular forms of weight $\sigma(L)/2$ and type $\rho_{L}$.  
choose a maximal isotropic sublattice $I$ of $L$. 
Then $K=I^{\perp}/I$ is an even negative-definite lattice of rank $-\sigma(L)$. 
Let ${\pushLK}$ be the pushforward operation from $\rho_{L}$ to $\rho_{K}$ (\S \ref{ssec: Weil representation}), 
and ${\ThetaK}(\tau)$ be the $\rho_{K^{+}}$-valued theta series of the positive-definite lattice $K^{+}=K(-1)$. 
In \S \ref{sec: product}, we define the $\Theta$-product of $f_{1}, f_{2} \in {\ML}$ with respect to $I$ by 
\begin{equation*}
f_{1} \ast_{I} f_{2} = \langle f_{1}{\pushLK}, {\ThetaK} \rangle \cdot f_{2}.  
\end{equation*}
Then $f_{1} \ast_{I} f_{2}$ is again an element of ${\ML}$. 
By considering Fourier expansion, 
this could be viewed as a sort of product of two $\rho_{L}$-valued Laurent series, 
one suitably contracted with the theta series ${\ThetaK}$. 

In what follows, an \textit{associative algebra} is not assumed to have a unit element. 
Our basic results can be summarized as follows. 

\begin{theorem}\label{thm: main}
The $\Theta$-product $\ast_{I}$ makes ${\ML}$ a finitely generated filtered associative ${\C}$-algebra. 
%Its commutator brackets are contained in the (unique) annihilator ideal. 
The algebra ${\ML}$ has a unit element if and only if $L\simeq U\oplus \cdots \oplus U$. 
The algebra ${\ML}$ is commutative if and only if $L$ is unimodular. 

If $L'$ is a sublattice of $L$ of signature $(p, q')$ with $I_{{\Q}}\subset L'_{{\Q}}$, 
the map 
\begin{equation*}
{\ML} \to M^{!}(L'), \quad f\mapsto |I/I'|^{-1} \cdot f|_{L'}, 
\end{equation*}
is a homomorphism of ${\C}$-algebras, 
where $I'=I\cap L'$ and 
$f|_{L'}\in M^{!}(L')$ is the quasi-pullback of $f\in {\ML}$ 
as defined in \eqref{eqn: define quasi-pullback}. 
\end{theorem} 


Here the filtration on ${\ML}$ is defined by the degree of principal part. 
$U$ stands for the integral hyperbolic plane, namely 
the even unimodular lattice of signature $(1, 1)$. 
The quasi-pullback map $|_{L'}\colon {\ML}\to M^{!}(L')$  
is an operation coming from quasi-pullback of Borcherds products (\cite{Bo95}, \cite{B-K-P-SB}, \cite{Ma}), 
which is a sort of renormalized restriction.  
The kernel of ${\ML}\to M^{!}(L')$ for various sublattices $L'\subset L$ 
provide natural examples of two-sided ideals of ${\ML}$ contained in the left annihilator ideal. 
The statements in Theorem \ref{thm: main} are proved in Propositions 
\ref{prop: associativity et al}, 
\ref{prop: unimodular commutative}, 
\ref{prop: RUE}, 
\ref{prop: f.g.}, and 
\ref{prop: functorial main}. 


The algebra structure on ${\ML}$ requires the choice of $I$, 
but actually it depends only on the equivalence class of $I$ under a natural subgroup 
of the orthogonal group of $L$. 
Geometrically, when $p=2$, such equivalence classes correspond to 
maximal boundary components of the Baily-Borel compactification of the associated modular variety. 

In some special cases, $\Theta$-product is a quite simple operation. 
When $L$ is unimodular, so that $f_{1}, f_{2}$ and ${\ThetaK}=\theta_{K^{+}}$ are scalar-valued, 
$f_{1}\ast_{I}f_{2}$ is just the product $f_{1} \cdot \theta_{K^+} \cdot f_{2}$ (Example \ref{ex: unimodular}). 
When $I$ comes from $pU=U\oplus \cdots \oplus U$ embedded in $L$, 
so that we have a splitting $L=pU\oplus K$, 
$f_{1}, f_{2}$ correspond to weakly holomorphic Jacobi forms $\phi_{1}(\tau, Z), \phi_{2}(\tau, Z)$ 
of weight $0$ and index $K^+$ (see \cite{Gr}). 
Then the Jacobi form corresponding to $f_{1}\ast_{I}f_{2}$ is just 
$\phi_{1}(\tau, 0)\cdot \phi_{2}(\tau, Z)$ (Example \ref{ex: Jacobi form}). 
In general, one can say that $\Theta$-product $\ast_{I}$ is a functorial extension of  
this simple product to all pairs $(L, I)$. 

We expect that 
the complexity of the lattice $L$ (within fixed $p$ or up to direct summand of $U$) 
would be reflected in the complexity of the algebra ${\ML}$ in some way. 
The first examples are stated in Theorem \ref{thm: main}: 
commutativity and existence of (two-sided) unit element. 
More widely, we show that 
${\ML}$ has a left unit element 
if it contains a modular form with very mild singularity that is not a left zero divisor 
(Proposition \ref{prop: LUE}). 
Some reflective modular forms provide typical examples of such a modular form 
(Examples \ref{ex: reflective 1} and \ref{ex: reflective 2}). 
This might remind us of 
Borcherds' philosophy \cite{Bo00b} that 
for $L$ Lorentzian, 
existence of a reflective modular form 
should be related to interesting property of the reflection group of $L$. 

Another example of our general expectation is that 
the minimal number of generators of ${\ML}$ would reflect the size of $L$, 
and generators of small degree would have some significance (\S \ref{ssec: f.g.}). 
We prove a finiteness result on 
lattices $L$ with bounded number of generators of ${\ML}$ (Proposition \ref{prop: finiteness}). 
In the simple example $L=pU\oplus \langle -2 \rangle$, 
${\ML}$ is generated by two basic reflective modular forms (Example \ref{ex: generator}). 
We hope to find further connection between the algebra ${\ML}$ and the lattice $L$. 
It would be also a natural problem to find an interesting ${\ML}$-module. 


A natural but subtle problem is 
whether various subgroups of ${\ML}$ defined by 
arithmetic condition on the coefficients of principal part 
are closed under $\Theta$-product. 
We give a general criterion in the positive direction, 
and use it to deduce the following (\S \ref{ssec: integral part}): 
\begin{itemize}
\item The real part $M^!(L)_{{\R}}\subset {\ML}$ is closed under $\ast_{I}$.  
\item When $L$ is unimodular with $p=2$, 
the multiplicative group of Borcherds products of integral weight is closed under $\ast_{I}$. 
\end{itemize}
In general, the possible obstruction can be expressed as a $2$-cocycle in group cohomology. 


%We prove that the real part $M^{!}(L)_{{\R}}\subset {\ML}$ consisting of 
%modular forms with real principal part is closed under $\Theta$-product (Proposition \ref{prop: real}). 
%A subtler problem is whether the integral part ${\MLZ}\subset {\ML}$ is so. 
%We give a sufficient condition that guarantees $f_{1}\ast_{I} f_{2}\in {\MLZ}$ for two $f_{1}, f_{2} \in {\MLZ}$. 
%Using this criterion, we show that when $L$ is unimodular, 
%the lattice of Borcherds products of integral or half integral weight, 
%a natural sublattice of ${\MLZ}$, is closed under $\ast_{I}$ (Proposition \ref{prop: Z-part unimodular}). 
%In general, the obstruction can be expressed as a $2$-cocycle of the abelian group ${\MLZ}$ 
%with value in the module $M^{!}(L)_{{\R}}/{\MLZ}$. 


This paper is organized as follows. 
\S \ref{sec: preliminary} is recollection of modular forms for the Weil representation. 
In \S \ref{sec: product} we define $\Theta$-product. 
In \S \ref{sec: first property} we study some basic properties of the algebra ${\ML}$. 
In \S \ref{sec: functorial} we prove that $\Theta$-product is functorial. 
\S \ref{sec: first property} and \S \ref{sec: functorial} may be read independently. 

\vspace{0.2cm}

\textbf{Convention.} 
\textit{Unless stated otherwise, every ring in this paper is not assumed to be commutative nor have a unit element.} 



\section{Weil representation and modular forms}\label{sec: preliminary}

In this section we recall some basic facts about 
modular forms of Weil representation type 
following \cite{Bo98}, \cite{Br}. 
 

\subsection{Weil representation}\label{ssec: Weil representation} 

Let $L$ be an even lattice, namely 
a free abelian group of finite rank equipped with a nondegenerate symmetric bilinear form 
$(\cdot , \cdot) \colon L\times L \to {\Z}$ such that 
$(l ,l)\in 2{\Z}$ for all $l\in L$. 
When $L$ has signature $(p, q)$, we write $\sigma(L)=p-q$. 
The dual lattice of $L$ is denoted by $L^{\vee}$. 
The quotient $A_L=L^{\vee}/L$ is called the \textit{discriminant group} of $L$, 
and is endowed with the canonical ${\QZ}$-valued quadratic form 
$q_L \colon A_L\to {\QZ}$, $q_L(x)=(x, x)/2+{\Z}$, 
called the \textit{discriminant form} of $L$. 
In general, a finite abelian group $A$ endowed with a nondegenerate quadratic form 
$q \colon A\to{\QZ}$ is called a \textit{finite quadratic module}. 
We will frequently abbreviate $(A, q)$ as $A$. 
Every finite quadratic module $A$ is isometric to the discriminant form of some even lattice $L$. 
We then write $\sigma(A)=[\sigma(L)]\in {\Z}/8$. 
We denote by ${\C}A$ the group ring of $A$. 
The standard basis vector of ${\C}A$ corresponding to an element $\lambda\in A$ 
will be denoted by ${\elambda}$. 

Let ${\Mp}$ be the metaplectic double cover of ${\SL}$.   
Elements of ${\Mp}$ are pairs $(M, \phi)$ where 
$M=\begin{pmatrix}a & b \\ c & d \end{pmatrix}\in {\SL}$ 
and $\phi$ is a holomorphic function on the upper half plane such that $\phi(\tau)^2=c\tau+d$. 
The group ${\Mp}$ is generated by 
%\begin{equation*}
$T = \left( \begin{pmatrix}1&1\\ 0&1\end{pmatrix}, 1 \right)$ and %\quad  
$S = \left( \begin{pmatrix}0&-1\\ 1&0\end{pmatrix}, \sqrt{\tau} \right)$,  
%\end{equation*}
and the center of ${\Mp}$ is generated by 
$Z = S^{2} = \left( \begin{pmatrix}-1& 0\\ 0 & -1 \end{pmatrix}, \sqrt{-1} \right)$. 

The \textit{Weil representation} $\rho_A$ of ${\Mp}$ attached to a finite quadratic module $A$ 
is a unitary representation on ${\C}A$ defined by 
\begin{eqnarray*}
\rho_A(T)({\elambda}) & = & e(q(\lambda)){\elambda}, \\ 
\rho_A(S)({\elambda}) & = & 
\frac{e(-\sigma(A)/8)}{\sqrt{|A|}} \sum_{\mu\in A}e(-(\lambda, \mu)){\emu}.  
\end{eqnarray*}
Here $e(z)={\rm exp}(2\pi i z)$ for $z\in{\Q}/{\Z}$. 
We have 
\begin{equation*}
\rho_A(Z)({\elambda})  = e(-\sigma(A)/4)\mathbf{e}_{-\lambda}. 
\end{equation*}
We will also write $\rho_A=\rho_L$ when $A=A_{L}$ for an even lattice $L$. 

 
Let $A(-1)$ be the $(-1)$-scaling of $A$, 
namely the same underlying abelian group with the quadratic form $q$ replaced by $-q$. 
Then $\rho_{A(-1)}$ is canonically isomorphic to the dual representation $\rho_A^{\vee}$ of $\rho_A$. 
The isomorphism is defined by sending the standard basis of ${\C}A(-1)$ 
to the dual basis $\{ \mathbf{e}_{\lambda}^{\vee} \}$ 
of the standard basis $\{ \mathbf{e}_{\lambda} \}$ of ${\C}A$ 
through the identification $A(-1)=A$ as abelian groups. 
We will tacitly identify $\rho_{A(-1)}=\rho_{A}^{\vee}$ in this way. 
 
Let $I\subset A$ be an isotropic subgroup. 
Then $A'=I^{\perp}/I$ inherits the structure of a finite quadratic module. 
Let $p:I^{\perp}\to A'$ be the projection. 
We define linear maps 
\begin{equation}\label{eqn: pull push}
\uparrow_{A'}^{A} : {\C}A' \to {\C}A, \qquad 
\downarrow_{A'}^{A} : {\C}A \to {\C}A', 
\end{equation}
called \textit{pullback} and \textit{pushforward} respectively, by 
\begin{equation*}
{\elambda}\uparrow_{A'}^{A}=\sum_{\mu\in p^{-1}(\lambda)}{\emu}, \qquad 
{\emu}\downarrow_{A'}^{A} = 
\begin{cases}
\mathbf{e}_{p(\mu)}, & \mu\in I^{\perp}, \\ 
0, & \mu\not\in I^{\perp},  
\end{cases}
\end{equation*}
for $\lambda\in A'$ and $\mu\in A$. 
Then 
$\uparrow_{A'}^{A}$ and $\downarrow_{A'}^{A}$ are homomorphisms between the Weil representations 
(see, e.g., \cite{Bo98}, \cite{Br}, \cite{Ma}). 
Note that 
$\downarrow^{A}_{A'}\circ \uparrow^{A}_{A'}$ 
is the scalar multiplication by $|I|$. 
Note also that 
$\uparrow_{A'}^{A}$ and $\downarrow_{A'}^{A}$ 
are adjoint to each other with respect to the standard Hermitian metrics on ${\C}A$ and ${\C}A'$. 
When $A=A_{L}$ for an even lattice $L$, 
the isotropic subgroup $I$ corresponds to 
the even overlattice $L\subset L'\subset L^{\vee}$ of $L$ with $L'/L=I$. 
Then $A'=A_{L'}$. 
In this situation, we will also write 
$\uparrow_{A'}^{A} = \uparrow_{L'}^{L}$ and 
$\downarrow_{A'}^{A}=\downarrow_{L'}^{L}$.  



\subsection{Modular forms}\label{ssec: modular form}

Let $A$ be a finite quadratic module and 
let $k\in \frac{1}{2}{\Z}$ with $k\equiv \sigma(A)/2$ modulo $2{\Z}$. 
(We will be mainly interested in the case $k\leq 0$.) 
A $ {\C}A$-valued holomorphic function 
$f$ on the upper half plane is called a \textit{weakly holomorphic modular form} 
of weight $k$ and type $\rho_A$ if it satisfies 
$f(M\tau)=\phi(\tau)^{2k}\rho_A(M, \phi)f(\tau)$ 
for every $(M, \phi)\in {\Mp}$ 
and is meromorphic at the cusp. 
We write 
\begin{equation*}
f(\tau) = \sum_{\lambda\in A} \sum_{n\in q(\lambda)+{\Z}} c_{\lambda}(n) q^{n}{\elambda} 
\end{equation*}
for the Fourier expansion of $f$ 
where $q^n=\exp (2\pi in\tau)$ for $n\in {\Q}$. 
By the invariance under $Z$, we have 
$c_{-\lambda}(n)=c_{\lambda}(n)$. 
The finite sum 
$\sum_{\lambda} \sum_{n<0} c_{\lambda}(n) q^{n}{\elambda}$ 
is called the \textit{principal part} of $f$. 
When $k<0$, $f$ is determined by its principal part; 
when $k=0$, $f$ is determined by its principal part and constant term. 
According to Borcherds duality theorem (\cite{Bo00a}, \cite{Bo00b}, \cite{Br}), 
which polynomial can be realized as principal part 
is determined by certain cusp forms as follows. 

\begin{theorem}[\cite{Bo00a}, \cite{Bo00b}, \cite{Br}]
Let 
$P=\sum_{\lambda, n}c_{\lambda}(n)q^n{\elambda}$ 
be a ${\C}A$-valued polynomial % in $q^{-1}$ with fractional power 
where $\lambda\in A$ and $n\in q(\lambda)+{\Z}$ with $n<0$, 
such that $c_{-\lambda}(n)=c_{\lambda}(n)$. 
Then $P$ is the principal part of a weakly holomorphic modular form 
of weight $k\equiv \sigma(A)/2$ mod $2{\Z}$ and type $\rho_{A}$ 
if and only if 
$\sum_{n<0}c_{\lambda}(n)a_{\lambda}(-n)=0$ 
for every cusp form 
$\sum_{\lambda, m} a_{\lambda}(m)q^{m}\mathbf{e}_{\lambda}^{\vee}$ 
of weight $2-k$ and type $\rho_{A}^{\vee}$. 
\end{theorem}

This will be used in \S \ref{sec: first property}. 
The version in \cite{Bo00a} also takes the constant term into account 
and replaces cusp forms by holomorphic modular forms. 

We write $M^{!}_{k}(\rho_A)$ for the space of weakly holomorphic modular forms of weight $k$ and type $A$. 
For a subring $R$ of ${\C}$ (typically ${\Z}$ or ${\Q}$ or ${\R}$), 
we write $M^{!}_{k}(\rho_A)_{R} \subset M^{!}_{k}(\rho_A)$ 
for the subgroup of those $f$ whose principal part has coefficients in $R$. 
It is clear that 
$M^{!}_{k}(\rho_A)_{{\Z}}\otimes_{{\Z}}{\Q} = M^{!}_{k}(\rho_A)_{{\Q}}$. 
Moreover, 
McGraw's rationality theorem \cite{Mc} and Borcherds duality theorem 
tell us that 
\begin{equation*}
M^{!}_{k}(\rho_A)_{{\Q}}\otimes_{{\Q}}{\C} = M^{!}_{k}(\rho_A). 
\end{equation*}
If $f\in M^{!}_{k}(\rho_A)_{{\Q}}$, its coefficients $c_{\lambda}(0)$ of the constant term ($\lambda$ isotropic) 
are also rational number. 
This follows from the version of Borcherds duality theorem in \cite{Bo00a} 
and the rationality of Fourier coefficients of Eisenstein series 
due to Bruinier-Kuss \cite{B-K} ($\lambda=0$) and 
Schwagenscheidt \cite{Sc} ($\lambda$ general). 


Theta series are typical examples of holomorphic modular forms of Weil representation type. 
Let $N$ be an even positive-definite lattice. 
By Borcherds \cite{Bo98}, 
the $\rho_N$-valued function 
\begin{equation*}
\Theta_{N}(\tau) 
= \sum_{l\in N^{\vee}} q^{(l, l)/2}\mathbf{e}_{[l]} 
= \sum_{\lambda, n}c_{\lambda}^{N}(n)q^{n}{\elambda},  
\end{equation*}
where $c_{\lambda}^{N}(n)$ is the number of vectors $l$ in $\lambda+N\subset N^{\vee}$ such that $(l, l)=2n$, 
is a holomorphic modular form of weight ${\rm rk}(N)/2$ and type $\rho_{N}$. 
All Fourier coefficients of $\Theta_{N}(\tau)$ are nonnegative integers. 
If $N'$ is an even overlattice of $N$, we have 
$\Theta_{N'}=\Theta_{N}\!\downarrow^{N}_{N'}$. 


Let $L$ be an even lattice. 
For $A=A_{L}$ and $k=\sigma(L)/2$, 
we write 
\begin{equation*}
{\ML} = M^{!}_{\sigma(L)/2}(\rho_{L}), \qquad 
{\MLR} = M^{!}_{\sigma(L)/2}(\rho_{L})_{R}. 
\end{equation*}
We especially write 
\begin{equation*}
M^{!}=M^{!}(U\oplus \cdots \oplus U)=M^{!}(\{ 0 \}), 
\end{equation*} 
which is just the space of scalar-valued weakly holomorphic modular forms of weight $0$. 
Then $M^{!}$ is the polynomial ring in the $j$-function 
$j(\tau)=q^{-1}+744+ \cdots$. 
It is a fundamental remark that for every even lattice $L$, ${\ML}$ is a $M^{!}$-module. 


When $p=2$ and for $f\in{\MLZ}$, Borcherds \cite{Bo95}, \cite{Bo98} constructed a meromorphic modular form 
$\Psi(f)$ on the Hermitian symmetric domain attached to $L$, 
which has weight $c_{0}(0)/2\in {\Q}$ and whose divisor is 
a linear combination of Heegner divisors determined by the principal part of $f$. 
%Here, if $c_0(0)\not\in {\Z}$, $\Psi(f)$ is modular with respect to some multiplier system. 
The lifting 
$f\mapsto \Psi(f)$ is multiplicative (at least up to constant). 
%namely $\Psi(f_1+f_2)=\Psi(f_1)\cdot \Psi(f_2)$ for $f_1, f_2\in {\MLZ}$. 
Thus, at least when $R\supset {\Q}$, 
${\MLR}={\MLZ}\otimes_{{\Z}}R$ can be thought of as a scalar extension of the multiplicative group of Borcherds products. 



%%%Section3: Theta-prpoduct

\section{$\Theta$-product}\label{sec: product} 

Let $L$ be an even lattice of signature $(p, q)$ with $p\leq q$ 
and assume that $L$ has Witt index $p$. 
We choose and fix a maximal ($=$ rank $p$, primitive) isotropic sublattice $I$ of $L$. 
%This corresponds to a maximal rational boundary component of 
%the symmetric domain attached to $L$. 
In this section we define $\Theta$-product $\ast_{I}$ on the space 
${\ML}=M_{\sigma(L)/2}^{!}(\rho_{L})$ 
with respect to $I$, 
which makes ${\ML}$ an associative algebra.  
\S \ref{ssec: lattice lemma} is lattice-theoretic preliminary. 
$\Theta$-product is defined in \S \ref{ssec: theta product}. 
In \S \ref{ssec: example} we look at some examples. 


\subsection{Preliminary}\label{ssec: lattice lemma}

We first prepare a lattice-theoretic lemma. 
We write $K=I^{\perp}\cap L/I$, which is an even negative-definite lattice of rank $-\sigma(L)$. 
We shall realize $K$ as an orthogonal direct summand of a canonical overlattice of $L$. 
Let $I^{\ast}=I_{{\Q}}\cap L^{\vee}$ be the primitive hull of $I$ in the dual lattice $L^{\vee}$. 
Then $L^{\ast}=\langle L, I^{\ast} \rangle$ is an even overlattice of $L$ with $L^{\ast}/L\simeq I^{\ast}/I$. 
For $rU=U\oplus \cdots \oplus U$ ($r$ times) 
we denote by $e_{1}, f_{1}, \cdots , e_{r}, f_{r}$ its standard basis, 
namely $(e_{i}, f_{j})=\delta_{ij}$ and $(e_i, e_j)=(f_i, f_j)=0$. 
We write $I_{r}=\langle e_{1}, \cdots, e_{r} \rangle$. 

\begin{lemma}\label{lem: overlattice split}
There exists an embedding $\varphi\colon pU \hookrightarrow L^{\ast}$ 
such that $\varphi(I_{p})=I^{\ast}$.  
%and $\varphi(pU)^{\perp}\cap L' \simeq K$ naturally. 
In particular, we have 
$L^{\ast} = \varphi(pU)\oplus \varphi(pU)^{\perp} \simeq pU\oplus K$. 
The induced isometry 
$A_{L^{\ast}}\to A_{K}$ 
does not depend on the choice of $\varphi$.  
\end{lemma}

\begin{proof}
By the primitivity of $I^{\ast}$ in $L^{\vee}$, 
we have $(l, L^{\ast})=(l, L)={\Z}$ for any primitive vector $l$ in $I^{\ast}$. 
We take one such vector $l_1\in I^{\ast}$ and a vector $m_1\in L^{\ast}$ with $(l_1, m_1)=1$. 
Then $\langle l_1, m_1 \rangle \simeq U$ and 
we have a splitting $L^{\ast}=\langle l_1, m_1 \rangle \oplus L_{1}$ 
where $L_{1}=\langle l_1, m_1 \rangle^{\perp}\cap L^{\ast}$. 
The intersection 
$I_{1}=I^{\ast}\cap L_{1}$ satisfies $I^{\ast}=I_{1}\oplus {\Z}l_{1}$ 
and we have 
$(l, L_{1})=(l, L^{\ast})={\Z}$ for any primitive vector $l\in I_{1}$. 
Then we can repeat the same process for $I_{1}\subset L_{1}$. 
This eventually defines an embedding $\varphi\colon pU\hookrightarrow L^{\ast}$ with 
$\varphi(I_{p})=I^{\ast}$. 
We have natural isomorphisms 
\begin{equation*}
\varphi(pU)^{\perp}\cap L^{\ast} \stackrel{\simeq}{\to} 
(I^{\ast})^{\perp}\cap L^{\ast}/I^{\ast} = I^{\perp}\cap L/I =K. 
\end{equation*}

For the last assertion, we use the following construction. 
($I^{\ast}\subset L^{\ast}$ will be $I\subset L$ below.) 

\begin{claim}\label{claim: AL AK}
Let $L$ be an even lattice and $I\subset L$ be a primitive isotropic sublattice. 
Let $\varphi_{1}, \varphi_{2} \colon rU \hookrightarrow L$ 
be two embeddings with 
$\varphi_{1}(I_{r})=\varphi_{2}(I_{r})=I$. 
Then there exist 
\begin{itemize}
\item an isometry $\gamma_{L}$ of $L$ which preserves $I$ and acts trivially on $K=I^{\perp}/I$ and $A_{L}$, and 
\item an isometry $\gamma_{rU}$ of $rU$ which preserves $I_{r}$, 
\end{itemize}
such that $\varphi_{2}=\gamma_{L} \circ \varphi_{1} \circ \gamma_{rU}$. 
\end{claim}

If we write 
$K_{i}=\varphi_{i}(rU)^{\perp}\cap L$, 
then we have $\gamma_{L}(K_{1})=K_{2}$. 
The properties of $\gamma_{L}$ 
imply that the composition 
$A_{L}\to A_{K_{1}} \to A_{K}$ 
coincides with 
$A_{L}\to A_{K_{2}} \to A_{K}$,  
hence the last assertion of Lemma \ref{lem: overlattice split} follows.  

We prove Claim \ref{claim: AL AK} by induction on $r$. 
We may assume that $\varphi_{1}|_{I_{r}}=\varphi_{2}|_{I_{r}}$ 
by composing an isometry of $rU$ preserving $I_{r}$ 
and $\langle f_1, \cdots, f_r \rangle \simeq I_{r}^{\vee}$. 
When $r=1$, we let $l=\varphi_{i}(e_{1})$ and $m_{i}=\varphi_{i}(f_{1})$. 
Then as $\gamma_{L}$ we take the Eichler transvection $E_{l,m_{2}-m_{1}}$ (see, e.g., \cite{G-H-S}) 
which fixes $l$, sends $m_{1}$ to $m_{2}$, 
and acts trivially on $K$ and on $A_{L}$. 

%For general $r$, 
%we may assume by induction that 
%$\varphi_{1}|_{(r-1)U}=\varphi_{2}|_{(r-1)U}$. 
%Let $U\subset rU$ be the remaining component and $\varphi_{i}'=\varphi_{i}|_{U}$. 
%We write $L'=\varphi_{i}((r-1)U)^{\perp}\cap L$ and $I'=I\cap L'$. 
%Then we can apply the result in $r=1$ to 
%$I'\subset L'$ and $\varphi_{1}', \varphi_{2}'$ 
%to find a suitable isometry $\gamma$ of $L'$ and $\gamma'=\pm{\rm id}_{U}$ 
%such that $\varphi_{2}'=\gamma \circ \varphi_{1}' \circ \gamma'$. 
%Adding identity of $(r-1)U$ and $\varphi_{i}((r-1)U)$, 
%this proves Claim \ref{claim: AL AK} and so completes the proof of Lemma \ref{lem: overlattice split}. 

For general $r$, 
let $rU=(r-1)U\oplus U$ be the apparent decomposition and let  
$\varphi_{i}'=\varphi_{i}|_{(r-1)U}$ and $I'=\varphi_{i}(I_{r-1})$. 
By induction, there exists an isometry $\gamma_{L}'$ of $L$ 
which preserves $I'$ and acts trivially on $(I')^{\perp}/I'$ and $A_{L}$, 
and an isometry $\gamma_{(r-1)U}$ of $(r-1)U$ preserving $I_{r-1}$, 
such that 
$\varphi_{2}'=\gamma_{L}' \circ \varphi_{1}' \circ \gamma_{(r-1)U}$. 
Note that $\gamma_{L}'$ also preserves $I$ and acts trivially on $K$. 
We put 
$L'=\varphi_{2}((r-1)U)$, 
$L''=(L')^{\perp}\cap L$ and 
$I''=I\cap L''$. 
Then we have 
$\gamma_{L}' \circ \varphi_{1}({\Z}e_r)=\varphi_{2}({\Z}e_{r})=I''$. 
Thus we can apply the result for $r=1$ to 
$\varphi_{1}''=\gamma_{L}' \circ \varphi_{1}|_{U}$, 
$\varphi_{2}''= \varphi_{2}|_{U}$, and 
$I''\subset L''$. 
This provides us with an isometry $\gamma_{L''}$ of $L''$ 
which preserves $I''$ and acts trivially on $(I'')^{\perp}/I''\simeq K$ and $A_{L''}\simeq A_{L}$, 
and an isometry $\gamma_{U}=\pm{\rm id}_{U}$ of $U$, 
such that 
$\varphi_{2}''=\gamma_{L''}\circ \varphi_{1}'' \circ \gamma_{U}$. 
Now 
$\gamma_{L}=({\rm id}_{L'}\oplus \gamma_{L''})\circ \gamma_{L}'$ and 
$\gamma_{rU}=\gamma_{(r-1)U} \oplus \gamma_{U}$ 
satisfy the desired properties. 
\end{proof} 


\begin{remark}
The lattice $K$ can also be realized as a sublattice of $I^{\perp}\cap L$ as follows. 
We choose a basis $l_1, \cdots, l_p$ of $I$ and its dual basis 
$l_1^{\vee}, \cdots, l_p^{\vee}$ from $L^{\vee}$. 
We put $\tilde{K}=\langle l_1^{\vee}, \cdots, l_p^{\vee} \rangle^{\perp} \cap I^{\perp} \cap L$. 
By construction we have a splitting 
$I^{\perp}\cap L = I \oplus \tilde{K}$, 
so the projection gives an isometry $\tilde{K}\to K$. 
%This is a generalization of the construction in \cite{Br}. 
\end{remark}




\subsection{$\Theta$-product}\label{ssec: theta product}


We now define $\Theta$-product $\ast_{I}$ on ${\ML}$. 
%Let $K, I', L'$ be as in \S \ref{ssec: rank 2 isotropic}. 
We put $K^{+}=K(-1)$, which is an even positive-definite lattice of rank $-\sigma(L)$. 
We identify $A_{L^{\ast}}=A_{K}$ as in Lemma \ref{lem: overlattice split}. 
Let 
\begin{equation*}
{\pushLK} = \downarrow^{L}_{L^{\ast}} : A_{L} \to A_{L^{\ast}} = A_{K} 
\end{equation*}
be the pushforward operation defined in \eqref{eqn: pull push}. 
%This does not depend on the choice of the isometry $L'\simeq pU\oplus K$ as above. 
If $f\in {\ML}$, then 
$f{\pushLK}$ is an element of $M^{!}(K)$. 
We take the tensor product $f{\pushLK}\otimes {\ThetaK}$ with the theta series ${\ThetaK}$.  
This is a weakly holomorphic modular form of weight $0$ and type 
$\rho_{K}\otimes \rho_{K^{+}} \simeq \rho_{K}\otimes \rho_{K}^{\vee}$. 
Taking the contraction $\rho_{K}\otimes \rho_{K}^{\vee} \to {\C}$ 
produces a scalar-valued weakly holomorphic modular form of weight $0$, 
namely an element of $M^{!}$.  
We denote this modular function by 
\begin{equation*}
\xi(f) = \langle f{\pushLK}, {\ThetaK} \rangle \; \; \in M^{!}.  
\end{equation*}
The map $\xi \colon {\ML}\to M^{!}$ is $M^{!}$-linear. 

Now if $f_1, f_2\in {\ML}$, we define 
\begin{equation*}
f_{1} \ast_{I} f_{2} = \xi(f_{1}) \cdot f_{2} = \langle f_{1}{\pushLK}, {\ThetaK} \rangle \cdot f_2. 
\end{equation*}
This is again an element of ${\ML}$. 
The map 
\begin{equation*}
\ast_{I} : {\ML} \times {\ML} \to {\ML} 
\end{equation*}
is $M^!$-bilinear. 

Explicitly, if 
$f_{i}(\tau)=\sum_{\lambda, n}c_{\lambda}^{i}(n)q^{n}{\elambda}$ 
for $i=1, 2$ and 
${\ThetaK}(\tau)=\sum_{\nu, m} c_{\nu}^{K}(m)q^{m}\mathbf{e}_{\nu}^{\vee}$, 
the Fourier coefficients of 
$f_{1}\ast_{I}f_{2}=\sum_{\lambda, n}c_{\lambda}(n)q^{n}{\elambda}$ 
are given by 
\begin{equation*}
c_{\lambda}(n) = 
\sum_{m+l+k=n}\sum_{\mu\in J^{\perp}} 
c^{1}_{\mu}(m) \cdot c^{K}_{p(\mu)}(l) \cdot c^{2}_{\lambda}(k). 
\end{equation*}
Here 
$J=I^{\ast}/I\subset A_{L}$ 
and $p:J^{\perp}\to A_{K}$ is the projection. 
Note that even coefficients of $f_1, f_2$ in $n>0$, 
sometimes not being paid much attention, 
may contribute to the principal part of $f_1\ast_{I} f_2$. 


\begin{proposition}\label{prop: associativity et al}
We have 
%The following equalities hold. 
%
%(1) $f_1\ast_{I}(f_2+f_2') = f_1\ast_{I}f_2 + f_1\ast_{I}f_2'$. 
%
%(2) $(f_1+f_1')\ast_{I} f_2 = f_1\ast_{I}f_2 + f_1'\ast_{I}f_2$. 
\begin{equation*}
(f_1\ast_{I} f_2)\ast_{I} f_3 = f_1\ast_{I}(f_2\ast_{I} f_3) 
\end{equation*}
for $f_1, f_2, f_3 \in {\ML}$. 
Therefore $\Theta$-product $\ast_{I}$ makes ${\ML}$ an associative ${\C}$-algebra. 
Moreover, the map $\xi \colon {\ML}\to M^{!}$ is a ring homomorphism.  
%
%(4) $(f_1\ast_{I} f_2)\ast_{I} f_3 = (f_2\ast_{I} f_1)\ast_{I} f_3$. 
\end{proposition}

\begin{proof}
For the first assertion, we have 
\begin{eqnarray*}
(f_1\ast_{I} f_2)\ast_{I} f_3 
& =  & 
\xi(f_1\ast_{I}f_2) \cdot f_3  
\; = \; \xi(\xi(f_1) \cdot f_2) \cdot f_3 \\  
& = & 
\xi(f_1) \cdot \xi(f_2) \cdot f_3  
\; = \; \xi(f_1) \cdot (f_2 \ast_{I} f_3) \\
& = & 
f_{1}\ast_{I} (f_2 \ast_{I} f_3). 
\end{eqnarray*}
For the second assertion, we calculate  
\begin{equation*}
\xi(f_{1} \ast_{I} f_{2}) = 
\xi(\xi(f_{1})\cdot f_{2}) = 
\xi(f_{1})\cdot \xi(f_{2}).  
\end{equation*}
Thus $\xi$ preserves the products. 
\end{proof}
  

The algebra ${\ML}$ has the following filtration. 
For a natural number $d$ we denote by 
$M^{!}(L)_{d}\subset {\ML}$ 
the subspace of modular forms $f$ 
whose principal part has degree $\leq d$. 
Then we have 
\begin{equation*}
M^{!}(L)_{d} \ast_{I} M^{!}(L)_{d'} \subset M^{!}(L)_{d+d'}. 
\end{equation*}
Hence ${\ML}$ is a filtered algebra with this filtration. 
%We observe that 
%multiplication by the $j$-function $j(\tau)=q^{-1}+O(1)$ defines an embedding 
%$M^{!}(L)_{d}/M^{!}(L)_{d-1}\hookrightarrow M^!(L)_{d+1}/M^!(L)_{d}$ 
%for every $d$. 


By construction, this algebra structure on ${\ML}$ requires the choice of 
a maximal isotropic sublattice $I$, 
so we should write $\xi=\xi_{I}$ and ${\ML}=M^{!}(L, I)$ when we want to specify this dependence. 
In fact, the freedom of choice is finite. 
If $\gamma\colon L\to L$ is an isometry of $L$, 
then $\gamma$ acts on $A_{L}$. 
Since the induced action on ${\C}A_{L}$ preserves the Weil representation $\rho_L$, 
$\gamma$ acts on ${\ML}$. 
We have 
$\xi_{\gamma I}(\gamma f)=\xi_{I}(f)$ 
and so 
\begin{equation*}
(\gamma f_{1})\ast_{\gamma I} (\gamma f_{2}) 
= \gamma (f_{1}\ast_{I} f_{2}). 
\end{equation*}
In other words, the action of $\gamma$ on ${\ML}$ gives an isomorphism 
\begin{equation*}
\gamma : M^!(L, I) \to M^!(L, \gamma I) 
\end{equation*}
of algebras. 
In particular, 
when $\gamma$ acts trivially on $A_{L}$, 
its action on ${\ML}$ is also trivial, 
so we have 
$M^{!}(L, I)=M^{!}(L, \gamma I)$ 
as algebras. 

To summarize, 
if ${\rm O}(L)$ is the orthogonal group of $L$ and 
$\Gamma_{L}<{\rm O}(L)$ is the kernel of the reduction map ${\rm O}(L)\to {\rm O}(A_{L})$, 
then $M^!(L, I)$ depends only on the $\Gamma_{L}$-equivalence class of $I$. 
Moreover, its isomorphism class depends only on the ${\rm O}(L)$-equivalence class of $I$. 
In particular, we have only finitely many algebra structures $M^!(L, I)$ on ${\ML}$ 
for a fixed lattice $L$. 

Geometrically, the $\Gamma_{L}$-equivalence class of $I$ corresponds more or less to 
a boundary component of some compactification of the locally symmetric space associated to $\Gamma_{L}$. 
(For example, when $p=2$, a boundary curve in the Baily-Borel compactification.) 
Perhaps this geometric picture might lead one to wonder whether it is possible to interpolate 
$M^!(L, I)$ and $M^!(L, I')$ for $I\not\sim I'$ 
by some continuous family of algebraic objects. 


\subsection{Examples}\label{ssec: example}

We look at $\Theta$-product in some examples. 

\begin{example}\label{ex: unimodular}
Assume that $L$ is unimodular. 
Then $8 | \sigma(L)$. 
Modular forms of type $\rho_{L}$ are just scalar-valued modular forms. 
For any maximal isotropic sublattice $I$ we can find a splitting $L\simeq pU \oplus K$ with $I\subset pU$, 
and $K$ is also unimodular. 
In particular, ${\pushLK}$ is identity and 
${\ThetaK}=\theta_{K^{+}}$ is also scalar-valued. 
In this case, $\Theta$-product is just the product 
\begin{equation*}
f_1 \ast_{I} f_{2} = f_1 \cdot \theta_{K^{+}} \cdot f_2 
\end{equation*}
for $f_1, f_2\in {\ML}$. 
This shows that ${\ML}$ is commutative and has no zero divisor.  
Furthermore, ${\ML}$ has no unit element unless when $L=pU$. 
Indeed, if $f\in {\ML}$ is a unit element, then $f\cdot \theta_{K^{+}}=1$, 
but this is impossible when $K\ne \{ 0 \}$ 
because %$\theta_{K^{+}}(\tau)$ has a zero on the upper half plane. 
then $f$ would be a holomorphic modular form of negative weight. 
\end{example}


\begin{example}\label{ex: Jacobi form}
More generally, assume that we have a splitting 
$L=pU\oplus K$ with $I\subset pU$ ($K$ not necessarily unimodular). 
This is equivalent to $I=I^{\ast}$. 
In this situation, modular forms of type $\rho_{L}=\rho_{K}$ 
correspond to Jacobi forms of index $K^{+}$ as follows (see \cite{Gr} for more detail). 
Let 
$\Theta_{K^{+}}(\tau, Z) = \sum_{\lambda\in A_{K}} \theta_{K^{+}+\lambda}(\tau, Z)\mathbf{e}_{\lambda}^{\vee}$ 
be the $\rho_{K^{+}}$-valued Jacobi theta series. 
%where we identify $A_{K^{+}}=A_{K}$ as abelian groups 
%and $\{ \mathbf{e}_{\lambda}^{\vee} \}$ is the dual basis of $\{ {\elambda} \}$. 
If $f(\tau)=\sum_{\lambda\in A_K}f_{\lambda}(\tau){\elambda}$ 
is a weakly holomorphic modular form of weight $\sigma(L)/2$ and type $\rho_{K}$, 
the function 
\begin{equation*}
\phi(\tau, Z) 
= \langle f(\tau), \: \Theta_{K^{+}}(\tau, Z) \rangle 
= \sum_{\lambda\in A_{K}} f_{\lambda}(\tau)\theta_{K^{+}+\lambda}(\tau, Z) 
\end{equation*}
given by the contraction 
$\rho_K\otimes \rho_K^{\vee} \to {\C}$ 
is a weakly holomorphic Jacobi form of weight $0$ and index $K^{+}$. 
This gives a one-to-one correspondence between two such forms. 
Note that 
the restriction $\phi(\tau, 0)$ of $\phi(\tau, Z)$ to $Z=0$  
is just the modular function $\xi(f)$ because 
$\Theta_{K^{+}}(\tau, 0) = \Theta_{K^{+}}(\tau)$. 
%a scalar-valued weakly holomorphic modular form of weight $0$. 

Now let $f_1, f_2 \in {\ML}$ 
and $\phi_1, \phi_2$ be the corresponding Jacobi forms. 
Then the Jacobi form corresponding to $f_1\ast_{I} f_2$ is 
\begin{equation*}
\phi_1(\tau, 0) \cdot \phi_2(\tau, Z). 
\end{equation*}
Indeed, we have 
\begin{eqnarray*}
\langle f_1\ast_{I} f_2(\tau), \: \Theta_{K^{+}}(\tau, Z) \rangle 
& = & 
\langle \xi(f_{1})(\tau) \cdot f_2(\tau), \: \Theta_{K^{+}}(\tau, Z) \rangle \\ 
& = &  
 \xi(f_{1})(\tau) \cdot \langle f_2(\tau), \: \Theta_{K^{+}}(\tau, Z) \rangle \\ 
& = &  
\phi_1(\tau, 0) \cdot \phi_2(\tau, Z). 
\end{eqnarray*}
Thus Jacobi form interpretation of $\Theta$-product is simple: 
substitute $Z=0$ into $\phi_1$ to obtain a scalar-valued modular function, 
and multiply it to $\phi_2$. 
$\Theta$-product for general $(L, I)$, not necessarily coming from $pU\hookrightarrow L$, 
can be thought of as a functorial extension of this simple operation using the pushforward operation $\downarrow^{L}_{K}$. 
\end{example}






%%%Section 4: First properties

\section{Basic properties}\label{sec: first property}

In this section we study some basic properties of the algebra ${\ML}$. 
Except in Proposition \ref{prop: finiteness}, 
the reference maximal isotropic sublattice $I\subset L$ is fixed throughout. 
In \S \ref{ssec: annihilator} we study the left annihilator ideal of ${\ML}$, 
which plays a basic role in the study of ${\ML}$. 
In \S \ref{ssec: unit} we study existence/nonexistence of unit element. 
In \S \ref{ssec: f.g.} we prove that ${\ML}$ is finitely generated. 
In \S \ref{ssec: integral part} we study the problem 
whether the $R$-part ${\MLR}$ of ${\ML}$ or its variant is closed under $\ast_{I}$. 
\S \ref{ssec: unit} should be read after \S \ref{ssec: annihilator}, 
but \S \ref{ssec: f.g.} and \S \ref{ssec: integral part} may be read independently. 



\subsection{Left annihilator}\label{ssec: annihilator} 

The left annihilator ideal of ${\ML}$ 
%\begin{equation*}
%\{ \: f\in {\ML} \: | \: f\ast_{I}g=0 \; \textrm{for any} \; g\in {\ML} \: \}.  
%\end{equation*}
is a two-sided ideal of ${\ML}$. 
Since ${\ML}$ is torsion-free as a $M^{!}$-module, 
this coincides with the kernel of $\xi \colon {\ML}\to M^{!}$, 
which we denote by 
\begin{equation*}
\Theta^{\perp} = 
\{ \: f\in{\ML} \: | \: \langle f{\pushLK}, {\ThetaK} \rangle = 0 \: \}. 
\end{equation*}
This is also a sub $M^!$-module. 
Note that $\Theta^{\perp}$ also coincides with the left annihilator of any \textit{fixed} $g\ne0 \in {\ML}$. 
We have 
$(\Theta^{\perp})^{2}=0$ 
and $\Theta^{\perp}$ is the maximal nilpotent ideal of ${\ML}$, 
consisting of all nilpotent elements of ${\ML}$. 


\begin{proposition}\label{prop: theta kernel basic}
The quotient ring ${\ML}/\Theta^{\perp}$ is canonically identified with 
a nonzero ideal of the polynomial ring $M^{!}={\C}[j]$.  
%The quotient map ${\ML}\to {\ML}/\Theta^{\perp}$ 
%is universal among homomorphisms from ${\ML}$ to integral domains. 
Every homomorphism from ${\ML}$ to a ring without nonzero nilpotent element 
factors through ${\ML}\to {\ML}/\Theta^{\perp}$. 
\end{proposition}

\begin{proof}
%Since $\xi\colon {\ML}\to M^!$ is $M^!$-linear, its image is an ideal of $M^!$. 
By the definition $\Theta^{\perp}={\rm Ker}(\xi)$, 
the quotient ${\ML}/\Theta^{\perp}$ is identified with 
the image $\xi({\ML})\subset M^!$ of $\xi$. 
Since $\xi$ is a $M^!$-linear map, 
$\xi({\ML})$ is an ideal of $M^!$. 
We shall show that $\xi$ is a nonzero map. 
Since the map ${\pushLK}\colon {\ML}\to M^{!}(K)$ is surjective, 
it suffices to check that the map 
$\langle \cdot , {\ThetaK} \rangle \colon M^!(K)\to M^{!}$ is nonzero. 
This can be seen, e.g., by taking a modular form $f\in M^!(K)$ 
with Fourier expansion of the form 
$f(\tau)=q^{n}\mathbf{e}_{0}+o(q^{n})$ 
for some negative integer $n$,  
which is possible as guaranteed by Lemma \ref{lem: leading term}.  
The last assertion follows by a standard argument. 
%For the last assertion, 
%it suffices to see that 
%$\varphi(\Theta^{\perp})= \{ 0 \}$ 
%for an arbitrary ring homomorphism 
%$\varphi\colon {\ML}\to R$ 
%to an integral domain $R$. 
%If $f\in\Theta^{\perp}$ but $\varphi(f)\ne 0$, 
%then 
%$\varphi(f\ast_{I}f)=\varphi(f)\cdot \varphi(f)\ne 0$, 
%while $\varphi(f\ast_{I}f)=\varphi(0)=0$. 
%This is absurd. 
\end{proof}

By general theory of associative algebra, 
${\ML}$ has the structure of a Lie algebra by the commutator bracket 
\begin{equation*}
[ f_1, f_2 ] = f_1\ast_{I} f_2 - f_2\ast_{I} f_1. 
\end{equation*}
Since $\xi$ is a ring homomorphism and $M^!$ is commutative, 
these brackets are contained in $\Theta^{\perp}$. 
In other words, 
\begin{equation*}
f_1\ast_{I} f_2\ast_{I} f_3 = f_2\ast_{I} f_1\ast_{I} f_3. 
\end{equation*}

\begin{proposition}\label{prop: unimodular commutative} 
The following three conditions are equivalent. 

(1) $L$ is unimodular. 

(2) ${\ML}$ is commutative. 

(3) $\Theta^{\perp} = \{ 0 \}$. 

\noindent
Moreover, if $\Theta^{\perp}\ne \{ 0 \}$, we have $\dim \Theta^{\perp} = \infty$. 
\end{proposition}

\begin{proof}
(1) $\Rightarrow$ (2), (3) is observed in Example \ref{ex: unimodular}. 
(3) $\Rightarrow$ (2) holds because $[f_1, f_2]\in \Theta^{\perp}$. 
We check (2) $\Rightarrow$ (3). 
If $\Theta^{\perp}\ne \{ 0 \}$, 
we take $f_1\ne 0 \in \Theta^{\perp}$ and $f_2\not\in \Theta^{\perp}$. 
Then $f_1\ast_{I}f_2=0$ but $f_{2}\ast_{I}f_{1}\ne 0$, 
so ${\ML}$ is not commutative. 

Finally, we prove (3) $\Rightarrow$ (1). 
Suppose that $L$ is not unimodular. 
We shall show that $\dim \Theta^{\perp} = \infty$. 
We consider separately according to whether $K$ is unimodular or not. 
%(The case $n=2$ is included in the former: we have $K=\{ 0 \}$ and $\theta_{K^{+}}=1$.) 
When $K$ is unimodular, 
$\Theta^{\perp}$ coincides with the kernel of the pushforward 
${\pushLK}\colon {\ML} \to M^{!}(K)$. 
%Here modular forms in $M{!}(K)$ are scalar-valued. 
We show that 
$\dim {\rm Ker}({\pushLK})= \infty$. 
%For a natural number $d$, 
%let $M^{!}(L)_{d}$ be the space of modular forms in ${\ML}$ 
%whose principal part as a polynomial in $q^{-1}$ has degree $\leq d$. 
%We define 
%$M^{!}(K)_{d}\subset M^{!}(K)$ similarly. 
The map ${\pushLK}$ preserves the degree filtration, 
namely 
$M^!(L)_{d}{\pushLK} \subset M^!(K)_d$. 
By Borcherds duality theorem, we have 
\begin{eqnarray*}
\dim M^!(L)_d & = & |A_L/\pm 1| \cdot d+O(1), \\
%\end{equation*}
%\begin{equation*}
\dim M^!(K)_d & = & 1\cdot d+O(1), 
\end{eqnarray*}
as $d$ grows. 
Therefore 
\begin{equation*}
\dim (\ker({\pushLK})\cap M^!(L)_d) \: \geq \: (|A_L/\pm1|-1)\cdot d + O(1) \to \infty 
\end{equation*}
as $d\to \infty$. 
Here $|A_L/\pm1|>1$ because $A_{L}\ne \{ 0 \}$. 

When $K$ is not unimodular, we can still argue similarly. 
The map 
${\pushLK}\colon {\ML}\to M^!(K)$ 
is surjective as 
the composition $\downarrow^{L}_{K} \circ \uparrow^{L}_{K}$ is a nonzero scalar multiplication.  
Therefore it is sufficient to show that the subspace 
$\ker \langle \cdot, {\ThetaK} \rangle$ of $M^!(K)$ 
has dimension $\infty$. 
The map 
$\langle \cdot, {\ThetaK} \rangle \colon M^!(K)\to M^!$ 
preserves the degree filtration, 
so we have similarly 
\begin{equation*}
\dim (\ker \langle \cdot, {\ThetaK} \rangle \cap M^!(K)_d) \: \geq \: (|A_K/\pm 1|-1)\cdot d + O(1) \to \infty 
\end{equation*}
as $d\to \infty$. 
This finishes the proof of (3) $\Rightarrow$ (1). 
\end{proof}


By Proposition \ref{prop: theta kernel basic}, 
${\ML}$ is decomposed into two parts: 
the ideal $\xi({\ML})$ in the polynomial ring $M^{!}={\C}[j]$, 
and the left annihilator $\Theta^{\perp}$. 
By the proof of (2) $\Rightarrow$ (3) in Proposition \ref{prop: unimodular commutative},  
the Lie brackets $[f, g]$ generate a large part of $\Theta^{\perp}$ 
containing at least $\xi({\ML})\cdot \Theta^{\perp}$. 
In \S \ref{sec: functorial} we will see that 
the kernels of the quasi-pullback maps to sublattices of $L$ 
provide natural examples of two-sided ideals contained in $\Theta^{\perp}$. 

\begin{remark}
We have only studied the left annihilator. 
The right annihilator of a fixed $f\in {\ML}$ 
coincides with the whole ${\ML}$ if $f\in \Theta^{\perp}$, 
while it is $\{ 0 \}$ if $f\not\in \Theta^{\perp}$. 
\end{remark}



\subsection{Unit element}\label{ssec: unit}

Next we study existence/nonexistence of unit element. 
Right unit element exists only in the apparent case. 

\begin{proposition}\label{prop: RUE}
${\ML}$ has a right unit element if and only if $L=pU$. 
In this case it is actually the two-sided unit element. 
\end{proposition}

\begin{proof}
It suffices to verify the ``only if'' direction. 
Let $g\in {\ML}$ be a right unit element. 
If $L$ is not unimodular, we can take $f\ne 0 \in \Theta^{\perp}$ by Proposition \ref{prop: unimodular commutative}. 
Then $f\ast_{I}g=0\ne f$, which is absurd. 
So $L$ must be unimodular. 
Then the assertion follows from the last part of Example \ref{ex: unimodular}. 
\end{proof} 


On the other hand, 
left unit element, though still relatively rare, 
exists in more cases. 
They are exactly modular forms $f\in {\ML}$ with $\xi(f)=1$. 
In particular, if $f$ is a left unit element, 
every element of $f+\Theta^{\perp}$ is so, 
and vice versa. 


\begin{proposition}\label{prop: LUE}
(1) ${\ML}$ has a left unit element if and only if the homomorphism 
$\xi\colon {\ML}\to M^{!}$ is surjective. 
This always holds when $\sigma(L)=0$. 


(2) If there exists a modular form $f\in {\ML}\backslash \Theta^{\perp}$ with $f(\tau)=o(q^{-1})$, 
then ${\ML}$ has a left unit element. 
Such a modular form $f$ exists only when $|\sigma(L)|<24$. 
\end{proposition}

\begin{proof}
The first assertion of (1) holds because $\xi$ is $M^!$-linear. 
When $\sigma(L)=0$, 
we have $M^!(K)=M^!$ and 
$\xi={\pushLK}\colon {\ML}\to M^{!}(K)$ is surjective. 

Next we prove (2). 
If $f=o(q^{-1})$, we have 
$\xi(f) = o(q^{-1})$. 
Since Fourier expansion of elements of $M^!$ have only integral powers of $q$, 
we have in fact $\xi(f) = O(1)$. 
Hence $\xi(f)$ is a holomorphic modular function, 
namely a constant, 
which is nonzero by our assumption $f\not\in \Theta^{\perp}$. 
As for the last assertion of (2), 
we consider the product $\Delta\cdot f$ with the $\Delta$-function. 
This is a cusp form , so its weight $12+\sigma(L)/2$ must be positive. 
\end{proof}


The condition $f\not\in \Theta^{\perp}$ in Proposition \ref{prop: LUE} (2) is satisfied when 
the principal part of $f{\pushLK}$ has nonnegative (at least one nonzero) coefficients. 
Indeed, $\Theta_{K^{+}}(\tau)=\mathbf{e}_{0}^{\vee}+o(1)$ has nonnegative coefficients 
and the coefficient $c_{0}(0)$ of $f{\pushLK}$ is positive (\cite{Br}, \cite{B-K}), 
so $\xi(f)$ has nonzero constant term. 

Some reflective modular forms provide 
typical examples of modular forms as in Proposition \ref{prop: LUE} (2). 

\begin{example}\label{ex: reflective 1}
Let $L= pU \oplus \langle -2 \rangle$. 
Then $K^{+}= \langle 2 \rangle$. 
Let $\phi_{0,1}$ be the weak Jacobi form of weight $0$ and index $1$ 
constructed by Eichler-Zagier in \cite{E-Z} Theorem 9.3. 
The corresponding modular form in ${\ML}$ has Fourier expansion 
$f(\tau)=q^{-1/4}\mathbf{e}_{1}+10\mathbf{e}_{0}+o(1)$  
where $\mathbf{e}_{i}$ is the basis vector of ${\C}A_{L}$ corresponding to $[i]\in {\Z}/2\simeq A_{L}$. 
This modular form satisfies the condition in Proposition \ref{prop: LUE} (2). 
We will return to this example in Example \ref{ex: generator}. 
\end{example}


\begin{example}\label{ex: reflective 2}
More generally, let $L=pU\oplus \langle -2t \rangle$. 
Then $K=K_{t}=\langle -2t \rangle$. 
Eichler-Zagier's Jacobi form $\phi_{0,1}$ was generalized by Gritsenko-Nikulin in \cite{G-N} \S 2.2 
to Jacobi forms $\phi_{0,t}$ of weight $0$ and index $t$. 
For $t=2, 3, 4$, 
the $\rho_{K_{t}}$-valued modular form $f_{t}$ corresponding to $\phi_{0,t}$ 
has Fourier expansion 
$f_{t}(\tau) = q^{-1/4t}\mathbf{e}_{1}+a_{t}\mathbf{e}_{0}+ \cdots $ 
where 
$a_{t}=4, 2, 1$ for $t=2, 3, 4$ respectively. 
Thus $f_{t}$ for $t=2, 3, 4$ satisfy the condition in Proposition \ref{prop: LUE} (2). 
\end{example}


\subsection{Finite generation}\label{ssec: f.g.}

In this subsection we prove that ${\ML}$ is finitely generated and 
give rough estimates, from above and below, 
on the minimal number of generators. 

\begin{proposition}\label{prop: f.g.}
The algebra ${\ML}$ is finitely generated over ${\C}$. 
\end{proposition}

For the proof we need the following construction. 

\begin{lemma}\label{lem: leading term}
There exists a natural number $d_{0}$ such that 
for any pair $(\lambda, n)$ with 
$\lambda\in A_{L}$ and $n\in q(\lambda)+{\Z}$, $n<-d_{0}$,  
there exists a modular form $f_{\lambda,n}\in {\ML}$ with Fourier expansion 
$f_{\lambda,n}(\tau)=q^{n}({\elambda}+\mathbf{e}_{-\lambda})+o(q^{n})$. 
\end{lemma}

\begin{proof}
For simplicity we assume $\sigma(L)<0$; 
the case $\sigma(L)=0$ can be dealt with similarly. 
For each natural number $d$ we let $V_{d}$ be 
the space of ${\C}A_{L}$-valued polynomials of the form 
\begin{equation}\label{eqn: principal part}
\sum_{\lambda\in A_{L}} 
\sum_{\substack{-d\leq m <0 \\ m\in q(\lambda)+{\Z}}} 
c_{\lambda}(m)q^{m}{\elambda}, \qquad 
c_{\lambda}(m) = c_{-\lambda}(m).  
\end{equation}
Then $\dim V_{d} = |A_{L}/\pm 1|\cdot d$. 
The filter $M^{!}(L)_{d}$ of ${\ML}$ is canonically embedded in $V_{d}$ 
by associating the principal parts. 
Let $S=S_{2-\sigma(L)/2}(\rho_{L}^{\vee})$ 
be the space of cusp forms of weight $2-\sigma(L)/2$ and type $\rho_{L}^{\vee}$. 
By Borcherds duality theorem, 
the subspace $M^{!}(L)_{d}$ of $V_{d}$ is characterized as 
\begin{equation*}
M^{!}(L)_{d} = {\ker}(V_{d}\to S^{\vee}). 
\end{equation*}
When $d\gg 0$, 
$V_{d}\to S^{\vee}$ is surjective (\cite{Bo00a}), 
and hence 
\begin{equation*}
\dim M^{!}(L)_{d} = |A_L/\pm 1|\cdot d - \dim S. 
\end{equation*}
In particular, we find that 
\begin{equation*}
\dim M^{!}(L)_{d+1} - \dim M^{!}(L)_{d} = |A_{L}/\pm 1|. 
\end{equation*}
On the other hand, 
$M^{!}(L)_{d}$ as a subspace of $M^{!}(L)_{d+1}$ 
is the kernel of the map 
$\rho_{d}\colon M^{!}(L)_{d+1} \to {\C}(A_{L}/\pm 1)$ 
that associates coefficients of the principal part in degree $\in [-d-1, -d)$. 
Therefore $\rho_{d}$ must be surjective when $d\gg 0$. 
The form $f_{\lambda,n}$ as desired can be obtained as 
$\rho_{d}^{-1}({\elambda}+\mathbf{e}_{-\lambda})$ for suitable $d$. 
\end{proof}

We now prove Proposition \ref{prop: f.g.}. 

\begin{proof}[(Proof of Proposition \ref{prop: f.g.})]
We first define a set of generators. 
First we take $f_{0}\in{\ML}$ whose Fourier expansion is of the form 
$q^{-d_{1}}\mathbf{e}_{0}+o(q^{-d_{1}})$ for some natural number $d_{1}$. 
Next, 
letting $d_{0}$ be as in Lemma \ref{lem: leading term}, 
we put  
\begin{equation*}
\Lambda_{1} = 
\{ \: f_{\lambda,m} \: | \: \lambda\in A_{L}/\pm 1, \: m\in q(\lambda)+{\Z}, \: -d_{0}-d_{1} \leq m < -d_{0} \: \}. 
\end{equation*}
Then we take a basis of $M^{!}(L)_{d_{0}}$ and denote it by $\Lambda_{2}$. 
We shall show that $f_{0}$, $\Lambda_{1}$ and $\Lambda_{2}$ 
generate ${\ML}$ as a ${\C}$-algebra. 

By definition $M^{!}(L)_{d_{0}+d_{1}}$ is generated by 
$\Lambda_{1}\cup \Lambda_{2}$ as a ${\C}$-linear space. 
The quotient 
$M^!(L)/M^!(L)_{d_0+d_1}$ 
is generated as a ${\C}$-linear space 
by any set of modular forms whose Fourier expansion is of the form 
$q^{n}({\elambda}+\mathbf{e}_{-\lambda})+o(q^{n})$ 
where $\lambda$ varies over $A_{L}/\pm 1$ and 
$n$ varies over  $q(\lambda)+{\Z}$ with $n<-d_{0}-d_{1}$. 
Therefore it suffices to show that 
we can construct such a modular form as a product of $f_{0}$ and elements of $\Lambda_{1}$. 
Since 
$f_{0}(\tau){\pushLK}=q^{-d_1}\mathbf{e}_{0}+o(q^{-d_1})$ 
and 
${\ThetaK}(\tau)=\mathbf{e}_{0}^{\vee}+o(1)$, 
we have 
$\xi(f_0)=q^{-d_1}+o(q^{-d_1})$. 
We take 
$m\equiv n$ modulo $d_1$ from 
$-d_0-d_1 \leq m < -d_0$ 
and put 
$r=(m-n)/d_1\in {\N}$. 
Then 
\begin{eqnarray*}
& & 
f_{0} \ast_{I} \cdots \ast_{I} f_{0} \ast_{I} f_{\lambda,m} \qquad (f_{0} \: \: r \: \textrm{times}) \\ 
& = & 
%\xi(f_0)^{r} \cdot f_{\lambda,m} = 
(q^{-d_1}+o(q^{-d_1}))^{r} (q^{m}({\elambda}+\mathbf{e}_{-\lambda})+o(q^{m})) \\ 
& = & 
q^{n}({\elambda}+\mathbf{e}_{-\lambda}) + o(q^{n}). 
\end{eqnarray*}
This gives a desired modular form. 
\end{proof}


\begin{remark}\label{remark: f.g. as module}
By a similar (and easier) argument, 
using the $j$-function in place of $f_{0}$, 
we see that ${\ML}$ is also finitely generated as a $M^!$-module. 
Indeed, 
multiplication by $j(\tau)=q^{-1}+O(1)$ defines an embedding 
$M^{!}(L)_{d}/M^{!}(L)_{d-1}\hookrightarrow M^!(L)_{d+1}/M^!(L)_{d}$ 
for every $d$, 
which stabilizes to an isomorphism in $d \gg 0$. 
\end{remark} 



By the proof of Proposition \ref{prop: f.g.}, 
the number of generators can be bounded above by 
\begin{equation*}\label{eqn: bound generator}
1 + d_{1}\cdot |A_{L}/\pm 1| + \dim M^!(L)_{d_{0}} 
\: \leq \: 
1+(d_0+d_1)\cdot |A_{L}/\pm 1|.   
\end{equation*}
In this upper bound is reflected the size of $L$. 
Indeed, $|A_{L}/\pm 1|$ reflects $|A_{L}|$, and 
$d_{0}, d_{1}$ reflect $|\sigma(L)|$ by the following well-known property. 

\begin{lemma}\label{lem: sgn bound}
If $M^!(L)_{d}\ne \{ 0 \}$, then $|\sigma(L)|\leq 24d$. 
\end{lemma}

\begin{proof}
If $f\ne 0 \in M^!(L)_{d}$, 
the product $\Delta^{d}\cdot f$ with the $\Delta$-function is holomorphic also at the cusp. 
Hence its weight $\sigma(L)/2+12d$ must be nonnegative. 
\end{proof}

%It might be natural to expect that a minimal set of generators of ${\ML}$ 
%would still reflect the size of $L$ and perhaps more. 


On the other hand, a lower bound leads to the following. 

\begin{proposition}\label{prop: finiteness}
Let $p\leq q$ be fixed. 
Let $N$ be a fixed natural number. 
Then up to isometry 
there are only finitely many pairs $(L, I)$ 
of an even lattice $L$ of signature $(p, q)$ and Witt index $p$ 
and a maximal isotropic sublattice $I\subset L$ such that 
the algebra $M^!(L, I)$ can be generated by at most $N$ elements. 
\end{proposition}

\begin{proof}
In \S \ref{ssec: theta product}, we observed that 
the dependence on $I$ is finite for a fixed lattice $L$. 
Hence it is sufficient to prove finiteness of lattices $L$. 
Since $f\ast_{I}g=\xi(f)\cdot g$, 
generators of ${\ML}$ as algebra 
also serve as generators as $M^!$-module. 
Since we need at least $|A_{L}/\pm 1|$ generators as $M^!$-module, 
we obtain the bound 
\begin{equation*}
N \: \geq \: |A_{L}/\pm 1| \: > \: |A_{L}|/2. 
\end{equation*}
Then our assertion follows from 
finiteness of even lattices of fixed signature and bounded discriminant. 
\end{proof}


%Proposition \ref{prop: finiteness} 
%is one coarse way of expressing the expectation that 
%the minimal number of generators of ${\ML}$ would reflect the complexity of $L$. 
It would be a natural problem whether the finiteness still holds 
even if we let $q$ vary with $p$ fixed. 
The same statement is not true for generators \textit{as $M^!$-module}. 
Indeed, when $L$ is unimodular, 
${\ML}$ can be generated by one element as $M^!$-module (cf.~Remark \ref{remark: f.g. as module}). 




We close this subsection with some simple examples. 


\begin{example}
Assume that the obstruction space 
$S_{2-\sigma(L)/2}(\rho_{L}^{\vee})$ 
is trivial. 
(Such lattices $L$ with $p=2$ are classified in \cite{B-E-F}.) 
Then every polynomial as in \eqref{eqn: principal part} 
is the principal part of some modular form in ${\ML}$. 
In this case, using the notation in the proof of Proposition \ref{prop: f.g.}, 
we have 
$d_{0}=0$, 
$d_{1}=1$, 
$\Lambda_{2}=\emptyset$, 
and the modular form $f_{0}$ can be included in $\Lambda_{1}$. 
Therefore ${\ML}$ can be generated by modular forms 
$f_{\lambda}=q^n(\mathbf{e}_{\lambda}+\mathbf{e}_{-\lambda})+O(1)$ 
with $\lambda\in A_L/\pm1$ and $n\in q(\lambda)+{\Z}$, $-1\leq n < 0$. 
The minimal number of generators is thus equal to $|A_{L}/\pm1|$. 
The generator $f_{\lambda}$ with $\lambda\ne 0$ 
is either a left unit element or a left zero divisor 
according to Proposition \ref{prop: LUE} (2). 
\end{example}


\begin{example}\label{ex: generator}
We go back to Example \ref{ex: reflective 1} where $L=pU \oplus \langle -2 \rangle$. 
The algebra ${\ML}$ is generated by the two elements 
$f_{0}=q^{-1}\mathbf{e}_{0}+O(1)$ and 
$f_{1}=q^{-1/4}\mathbf{e}_{1}+O(1)$ 
with the relation 
$f_{1}\ast_{I} f_{1}= 12 f_{1}$ and $f_{1}\ast_{I} f_{0}= 12 f_{0}$.  
Thus the two basic reflective modular forms for $L$ 
give minimal generators of the algebra ${\ML}$.  
\end{example}




\subsection{On the integral part}\label{ssec: integral part}

One of interests in ${\ML}$ would lie in the integral part ${\MLZ}$ 
because when $p=2$ Borcherds products can be constructed from modular forms in ${\MLZ}$. 
It seems to be a subtle problem 
whether ${\MLZ}$ is closed under $\Theta$-product. 
There are examples of $f_{1}, f_{2}\in {\MLZ}$ with $f_{1}\ast_{I}f_{2}\in {\MLZ}$, 
but in general there seems to be an obstruction coming from the possibility that 
Fourier coefficients of $f\in{\MLZ}$ in positive degree might be no longer integer. 
In this subsection we study some aspects of this problem. 

We first give a sufficient condition that guarantees $f_{1}\ast_{I}f_{2}\in {\MLZ}$. 
Since the proof is similar, we work with a general subring $R$ of ${\C}$. 

\begin{proposition}\label{prop: criterion}
Let $R$ be a subring of ${\C}$. 
Let $f_{1}, f_{2}\in M^!(L)_{R}$ and 
$f_{i}(\tau) = \sum_{\lambda, n}c_{\lambda}^{i}(n)q^{n}{\elambda}$ 
be their Fourier expansion. 
Let $f_{1}\in M^!(L)_{d}$. 
Assume that 

(a) the coefficients $c_{\lambda}^{1}(0)$ of the constant term of $f_{1}$ are contained in $R$; 

(b) the coefficients $c_{\lambda}^{2}(n)$ of $f_{2}$ in  $n< d$ are contained in $R$. 

\noindent
Then $f_{1}\ast_{I}f_{2}\in M^!(L)_{R}$. 
\end{proposition}

\begin{proof}
 
We shall show that $\xi(f_{1})$ has Fourier coefficients in $R$. 
Since only coefficients of $f_{2}$ in degree $<d$ 
contribute to the principal part of $f_{1}\ast_{I}f_{2}=\xi(f_{1})f_{2}$, 
the assertion then follows from the condition (b). 

In order to show that $\xi(f_{1})$ has coefficients in $R$, 
we first note that the principal part of $f_{1}{\pushLK}$ has coefficients in $R$. 
By the condition (a), the constant term of $f_{1}{\pushLK}$ also has coefficients in $R$. 
Since ${\ThetaK}$ is holomorphic and has integral coefficients, 
we find that the principal part and the constant term of $\xi(f_{1})$ have coefficients in $R$. 
We write $\xi(f_{1})$ as a polynomial $P(j)$ of the $j$-function  
$j(\tau)=q^{-1}+744+\cdots$. 
In view of the fact that 
$j(\tau)$ has integral coefficients, 
this implies that the polynomial $P$ has coefficients in $R$. 
This in turn concludes that $\xi(f_{1})=P(j)$ has Fourier coefficients in $R$. 
\end{proof}


We apply this criterion in two cases. 

\begin{proposition}\label{prop: real}
The real part $M^{!}(L)_{{\R}}$ is closed under $\ast_{I}$. 
\end{proposition}

\begin{proof}
We prove that any modular form 
$f=\sum_{\lambda,n}c_{\lambda}(n)q^n{\elambda}$ 
in $M^!(L)_{{\R}}$ has real Fourier coefficients: 
this enables us to apply Proposition \ref{prop: criterion}. 
We use the results of Bruinier in \cite{Br} \S 1.3. 
%to which we refer for the following notation. 
Let $F_{\lambda,n}(\tau)$ be the Maass-Poincare series constructed in \cite{Br} Proposition 1.10. 
By \cite{Br} Proposition 1.12, 
$f$ can be written as a linear combination of $F_{\lambda,n}(\tau)$ as 
\begin{equation*}
f(\tau) = \frac{1}{2} \sum_{\lambda\in A_{L}} \sum_{\substack{n\in q(\lambda)+{\Z} \\ n<0}} 
c_{\lambda}(n) F_{\lambda,n}(\tau). 
\end{equation*}
Here the non-holomorphic parts $\tilde{F}_{\lambda,n}$ of $F_{\lambda,n}$ cancel out (cf.~\cite{Br} Theorem 1.17). 
By \cite{Br} Remark 1.14, 
the Fourier coefficients of $F_{\lambda,n}$ are real. 
Since $c_{\lambda}(n)$ are real in $n<0$, 
this implies that $f$ has real Fourier coefficients. 
\end{proof}

In general, arithmetic property of Fourier coefficients of Maass-Poincare series 
seems to be a subtle problem. 
See \cite{B-F-O-R}. 


Next we let $L$ be unimodular. 
We show that 
a natural subgroup of ${\MLR}$ is closed under $\ast_{I}$. 
%Since the argument is similar, we again work with a general subring $R$ of ${\C}$. 
Let $M^{!}(L)_{R}'\subset {\MLR}$ be the subgroup of those $f\in {\MLR}$ 
whose constant term $c(0)$ is also contained in $R$. 

\begin{example}
(1) When $R\supset{\Q}$, we have $M^!(L)_{R}'=M^{!}(L)_{R}$ 
by Borcherds duality theorem in \cite{Bo00b} and the rationality of Fourier coefficients of 
(scalar-valued) Eisenstein series. 

(2) When $R={\Z}$, we have in fact $c(0)\in 2{\Z}$ for $f\in M^!(L)_{{\Z}}'$ by \cite{W-W}. 
Thus, when $p=2$, $M^!(L)_{{\Z}}'$ can be thought of as 
the multiplicative group of Borcherds products of integral weight.  
\end{example}


\begin{proposition}\label{prop: Z-part unimodular} 
Let $R$ be a subring of ${\C}$. 
When $L$ is unimodular, $M^{!}(L)_{R}'$ is closed under $\ast_{I}$. 
\end{proposition}

\begin{proof}
By Proposition \ref{prop: criterion},  
it suffices to show that if $f(\tau)=\sum_{n}c(n)q^{n}$ is an element of $M^{!}(L)_{R}'$, 
its coefficients $c(n)$ in $n>0$ are also contained in $R$. 
By the same argument as in the proof of Proposition \ref{prop: criterion}, 
%the principal part and the constant term of 
%$f\cdot \theta_{K^{+}}\in M^{!}$ have coefficients in $R$. 
%Using the $j$-function similarly, 
we find that 
all Fourier coefficients of $f\cdot \theta_{K^{+}}\in M^!$ are contained in $R$. 
If we write 
$\theta_{K^{+}}(\tau)=\sum_{n\geq 0}c^{K}(n)q^{n}$ and notice that $c^{K}(0)=1$, 
this means that 
\begin{equation*}
c(n)+\sum_{m<n}c(m)c^{K}(n-m)\in R 
\end{equation*}
for every $n$. 
Since we already know that 
$c(m)\in R$ for $m\leq 0$ 
and $c^{K}(l)\in {\Z}$ for every $l$, 
induction on $n$ tells us that 
$c(n)\in R$ for every $n$. 
This proves our claim. 
\end{proof} 


\begin{corollary}
When $L$ is unimodular with $p=2$, 
the multiplicative group of Borcherds products of integral weight 
has the structure of a commutative ring under $\Theta$-product. 
\end{corollary}


We close this subsection with a remark that 
how far ${\MLR}$ is not closed under $\ast_{I}$ can be 
expressed as a $2$-cocyle in group cohomology. 
We view ${\ML}/{\MLR}$ as a ${\MLR}$-module with trivial action. 
We define a map 
\begin{equation*}
\phi : {\MLR}\times {\MLR} \to {\ML}/{\MLR} 
\end{equation*}
by 
$\phi(f_{1}, f_{2})=[f_{1}\ast_{I}f_{2}]$, 
where $[ \: ]$ means the image in ${\ML}/{\MLR}$. 

\begin{proposition}
(1) ${\MLR}$ is closed under $\ast_{I}$ if and only if $\phi\equiv 0$. 

(2) $\phi$ is a $2$-cocycle of the abelian group ${\MLR}$ with value in ${\ML}/{\MLR}$. 
\end{proposition}

\begin{proof}
(1) is obvious from the definition of $\phi$. 
For (2), the cocycle condition is 
\begin{equation*}
\phi(f_{2}, f_{3}) + \phi(f_{1}, f_{2}+f_{3}) = 
\phi(f_{1}+f_{2}, f_{3}) + \phi(f_{1}, f_{2}). 
\end{equation*}
This holds true by the bilinearity of $\ast_{I}$. 
\end{proof}

When $R\subset {\R}$, we may replace ${\ML}/{\MLR}$ with 
$M^!(L)_{{\R}}/{\MLR}$ by Proposition \ref{prop: real}. 




%%%Section 5: functoriality 

\section{Functoriality}\label{sec: functorial}  

In this section we prove that  
$\Theta$-product is functorial with respect to embedding of lattices 
if we use quasi-pullback as morphism. 
The statement is Proposition \ref{prop: functorial main}, 
and the proof is given in \S \ref{ssec: finite pullback} and  \S \ref{ssec: general case}. 
In \S \ref{ssec: functorial push} we also prove functoriality 
with respect to pushforward to a special type of overlattice. 
Except for Corollaries \ref{cor: functorial add consequence} and \ref{cor: functorial add consequence II}, 
this section may be read independently of \S \ref{sec: first property}. 



\subsection{Quasi-pullback}\label{ssec: quasi-pullback} 

Let $L$ be an even lattice of signature $(p, q)$ %with $p\leq q$, 
and $L'$ be a sublattice of $L$ of signature $(p, q')$. %with $p\leq q'$. 
We do not assume that $L'$ is primitive in $L$. 
Following \cite{Ma}, we define a linear map 
$|_{L'} \colon {\ML}\to M^{!}(L')$ as follows. 
Let 
$N=(L')^{\perp}\cap L$, 
which is a negative-definite lattice. 
We write $N^{+}=N(-1)$. 
The lattice $L'\oplus N$ is of finite index in $L$. 
Let $f\in {\ML}$. 
We first take the pullback 
$f\!\uparrow_{L}^{L'\oplus N}$, 
which is an element of $M^!(L'\oplus N)$. 
Since 
$\rho_{L'\oplus N} = \rho_{L'}\otimes \rho_{N}$, 
we can take contraction of $f\!\uparrow_{L}^{L'\oplus N}$ 
with the $\rho_{N^{+}}$-valued theta series $\Theta_{N^{+}}$ of $N^{+}$. 
This produces a $\rho_{L'}$-valued weakly holomorphic modular form of weight $\sigma(L')/2$, 
which we denote by 
\begin{equation}\label{eqn: define quasi-pullback}
f|_{L'} = \langle f\!\uparrow_{L}^{L'\oplus N}, \: \Theta_{N^{+}} \rangle \quad \in M^{!}(L'). 
\end{equation}
We call $f|_{L'}$ the \textit{quasi-pullback} of $f$ to $L'$. 
The map $|_{L'} \colon {\ML}\to M^!(L')$ is $M^!$-linear. 

The geometric significance of this operation comes from Borcherds products as follows. 
Assume that $p=2$ and $f$ has integral principal part, 
and let $\Psi(f)$ be the Borcherds product associated to $f$ 
on the Hermitian symmetric domain $\mathcal{D}_{L}$ for $L$.   
The Hermitian symmetric domain $\mathcal{D}_{L'}$ for $L'$ 
is naturally embedded in $\mathcal{D}_{L}$. 
The quasi-pullback of $\Psi(f)$ from $L$ to $L'$, 
discovered by Borcherds \cite{Bo95}, \cite{B-K-P-SB},  
is defined by 
first dividing $\Psi(f)$ by suitable linear forms to get rid of zeros and poles containing $\mathcal{D}_{L'}$, 
and then restricting the resulting form to $\mathcal{D}_{L'}\subset \mathcal{D}_{L}$. 
It is proved in \cite{Ma} that this quasi-pullback of $\Psi(f)$ 
coincides with the Borcherds product for $f|_{L'}\in M^{!}(L')$ up to constant. 
Thus the operation $|_{L'}$ defined in \eqref{eqn: define quasi-pullback} 
can be thought of as a formal ${\C}$-linear extension of 
the quasi-pullback operation on Borcherds products. 

We can now state the main result of this \S \ref{sec: functorial}. 
We assume that $p\leq q' \leq q$ and both $L$ and $L'$ have Witt index $p$. 

\begin{proposition}\label{prop: functorial main}
Let $L'\subset L$ be as above. 
Let $I$ be a maximal isotropic sublattice of $L$ 
such that $I_{{\Q}}\subset L'_{{\Q}}$. 
We set $I'=I\cap L'$. 
Then we have 
\begin{equation*}\label{eqn: functorial}
(f|_{L'})\ast_{I'}(g|_{L'}) = |I/I'|\cdot (f\ast_{I}g)|_{L'} 
\end{equation*}
for $f, g \in {\ML}$. 
In particular, the map
\begin{equation*} 
|I/I'|^{-1}\cdot |_{L'} : M^!(L, I) \to M^{!}(L', I') 
\end{equation*}
is a ring homomorphism. 
\end{proposition}


This means that the assignment 
\begin{equation*}
(L, I) \mapsto M^{!}(L, I) 
\end{equation*}
is a contravariant functor 
from the category of pairs $(L, I)$ 
to the category of associative ${\C}$-algebras, 
by assigning the morphism $|I/I'|^{-1}\cdot |_{L'}$ 
to an embedding $(L', I')\hookrightarrow (L, I)$. 

The proof of Proposition \ref{prop: functorial main} is reduced to the following assertion. 

\begin{proposition}\label{prop: functorial}
Let $L'\subset L$ and $I'\subset I$ be as in Proposition \ref{prop: functorial main}. 
We put $K=I^{\perp}\cap L / I$ and $K'=(I')^{\perp}\cap L'/I'$. 
Let 
$\xi \colon {\ML}\to M^{!}$ and 
$\xi' \colon M^{!}(L')\to M^{!}$ be the maps 
$\xi= \langle \cdot {\pushLK}, {\ThetaK} \rangle$ and 
$\xi'= \langle \cdot \! \downarrow^{L'}_{K'}, \Theta_{(K')^{+}} \rangle$ 
respectively. 
Then we have 
\begin{equation*}
\xi' \circ |_{L'} = |I/I'| \cdot \xi. 
\end{equation*}
\end{proposition}

Indeed, if we admit Proposition \ref{prop: functorial}, we can calculate 
\begin{eqnarray*}
(f|_{L'})\ast_{I'}(g|_{L'}) 
& = & 
\xi'(f|_{L'})\cdot (g|_{L'}) 
= |I/I'| \cdot \xi(f) \cdot (g|_{L'}) \\ 
& = & 
|I/I'| \cdot (\xi(f)\cdot g)|_{L'} 
= |I/I'| \cdot (f\ast_{I}g)|_{L'}. 
\end{eqnarray*}
Thus Proposition \ref{prop: functorial} implies Proposition \ref{prop: functorial main}. 

Before going on, we note some consequences.  

\begin{corollary}\label{cor: functorial add consequence}
%Let $L'\subset L$ be as in Proposition \ref{prop: functorial main} and 
%Assume that $I\subset L'$. 
Let $\Theta^{\perp}(L)\subset {\ML}$ and 
$\Theta^{\perp}(L')\subset M^!(L')$ be 
the respective left annihilators. 
Then we have 
%$\Theta^{\perp}(L)|_{L'}\subset \Theta^{\perp}(L')$ and 
$|_{L'}^{-1}(\Theta^{\perp}(L'))=\Theta^{\perp}(L)$. 
In particular, we have 
${\rm Ker}(|_{L'})\subset \Theta^{\perp}(L)$. 
The map 
${\ML}/\Theta^{\perp}(L) \to M^!(L')/\Theta^{\perp}(L')$ 
induced by $|I/I'|^{-1}\cdot |_{L'}$ 
is inclusion of ideals in the polynomial ring $M^!={\C}[j]$. 
\end{corollary}

\begin{proof}
%The inclusion 
%$\Theta^{\perp}(L)|_{L'}\subset \Theta^{\perp}(L')$ 
%follows from Proposition \ref{prop: functorial main} and 
%the last part of Proposition \ref{prop: theta kernel basic}. 
%Proposition \ref{prop: functorial} implies that 
%the map 
%${\ML}/\Theta^{\perp}(L) \to M^!(L')/\Theta^{\perp}(L')$ 
%induced by $[I:I']^{-1}\cdot |_{L'}$ 
%is compatible with $\xi$ and $\xi'$, 
%so it is inclusion of ideals in $M^!$. 
%In particular, it is injective, and hence 
The equality 
$|_{L'}^{-1}(\Theta^{\perp}(L'))=\Theta^{\perp}(L)$ 
follows from Proposition \ref{prop: functorial}. 
Since $\xi$ and $\xi'$ embed 
${\ML}/\Theta^{\perp}(L)$ and $M^!(L')/\Theta^{\perp}(L')$ 
as ideals in $M^!$ respectively, 
the last assertion follows.  
\end{proof}

%Thus the kernel of the quasi-pullback map $|_{L'}$ 
%provides a natural example of two-sided ideal of ${\ML}$ contained in $\Theta^{\perp}$. 


\begin{corollary}\label{cor: functorial add consequence II}
Assume that $L$ is unimodular and let $R$ be a subring of ${\C}$. 
Then the subgroup $|I/I'|^{-1}\cdot M^!(L)_{R}'|_{L'}$ of $M^!(L')$ is closed under $\ast_{I'}$. 
In particular, $M^{!}(L)_{R}'|_{L'}\subset M^{!}(L')_{R}$ is closed under $\ast_{I'}$. 
\end{corollary}

\begin{proof}
This follows from Propositions \ref{prop: Z-part unimodular} and \ref{prop: functorial main}. 
\end{proof}

%The rest of this section is devoted to 
The proof of Proposition \ref{prop: functorial} occupies \S \ref{ssec: finite pullback} and \S \ref{ssec: general case}. 
It is divided into two parts, 
reflecting the fact that the quasi-pullback $|_{L'}$ is composition of 
two operators $\uparrow_{L}^{L'\oplus N}$ and $\langle \cdot, \Theta_{N^{+}} \rangle$. 
In \S \ref{ssec: finite pullback} we consider the case when $L'$ is of finite index in $L$. 
In \S \ref{ssec: general case} we consider the case when the splitting $L=L'\oplus N$ holds. 
The proof in the general case is a combination of these two special cases.  


\subsection{The case of finite pullback}\label{ssec: finite pullback} 

In this subsection we prove Proposition \ref{prop: functorial} 
in the case when $L'$ is of finite index in $L$. 
In this case, the quasi-pullback $|_{L'}$ is the operation $\uparrow^{L'}_{L}$, 
and Proposition \ref{prop: functorial} takes the following form. 

\begin{lemma}\label{prop: finite pullback}
When $L'\subset L$ is of finite index, we have for $f \in {\ML}$ 
\begin{equation*}
\xi'(f\!\uparrow^{L'}_{L}) = |I/I'| \cdot \xi(f).  
\end{equation*}
\end{lemma}

This is a consequence of the following calculation in 
finite quadratic modules. 

\begin{lemma}\label{lem: finite pullback FQM}
Let $A$ be a finite quadratic module and 
$I_1, I_2 \subset A$ be two isotropic subgroups. %such that $(I_{1}, I_{2})\equiv 0$. 
We set $A_1=I_1^{\perp}/I_1$, $A_2=I_2^{\perp}/I_2$ and 
\begin{equation*}
A'= %(I_{1}^{\perp}\cap I_{2}^{\perp}) /I_1+I_2$.  
(I_{1}^{\perp}\cap I_{2}^{\perp}) / ((I_{1}\cap I_{2}^{\perp})+(I_{2}\cap I_{1}^{\perp})). 
\end{equation*}
Let $I_{2}'=I_{2}\cap I_{1}^{\perp}/I_{1}\cap I_{2}$ be the image of $I_{2}\cap I_{1}^{\perp}$ in $A_{1}$, 
and $I_{1}'=I_{1}\cap I_{2}^{\perp}/I_{1}\cap I_{2}$ be the image of $I_{1}\cap I_{2}^{\perp}$ in $A_{2}$. 
Then, under the natural isomorphism 
\begin{equation}\label{eqn: description of A'}
A' \simeq (I_{2}')^{\perp}\cap A_{1}/I_{2}' \simeq (I_{1}')^{\perp}\cap A_{2}/I_{1}', 
\end{equation}
we have 
\begin{equation}\label{eqn: pull push commute FQM}
\downarrow^{A}_{A_{2}} \circ \uparrow^{A}_{A_{1}} = 
| I_{1} \cap I_{2} |  \uparrow^{A_{2}}_{A'} \circ \downarrow^{A_{1}}_{A'} 
\end{equation}
as linear maps 
${\C}A_{1} \to {\C}A_{2}$. 
\end{lemma}

We postpone the proof of Lemma \ref{lem: finite pullback FQM} for a moment, 
and first explain how Lemma \ref{prop: finite pullback} is deduced from Lemma \ref{lem: finite pullback FQM}. 

\begin{proof}[(Proof of Lemma \ref{prop: finite pullback})]
Let $K=I^{\perp}\cap L /I$ and $K'=(I')^{\perp}\cap L' /I'$. 
We have a canonical embedding $K'\hookrightarrow K$ of finite index. 
Since ${\ThetaK}=\Theta_{(K')^{+}}\!\downarrow^{(K')^{+}}_{K^{+}}$, 
we find that  
\begin{equation*}
\xi(f) 
= \langle f{\pushLK}, {\ThetaK} \rangle 
= \langle f{\pushLK}, \: \Theta_{(K')^{+}}\!\downarrow^{(K')^{+}}_{K^{+}} \rangle 
= \langle f{\pushLK}\uparrow_{K}^{K'}, \: \Theta_{(K')^{+}} \rangle. 
\end{equation*}
On the other hand, we have 
\begin{equation*}
\xi'(f \! \uparrow^{L'}_{L}) = 
\langle f {\pullL}\downarrow^{L'}_{K'}, \: \Theta_{(K')^{+}} \rangle. 
\end{equation*}
Thus it is sufficient to show that 
\begin{equation}\label{eqn: pull push commute lattice}
\downarrow^{L'}_{K'} \circ {\pullL} = |I/I'| \uparrow^{K'}_{K} \circ {\pushLK} 
\end{equation}
as linear maps 
${\C}A_{L}\to {\C}A_{K'}$. 

We apply Lemma \ref{lem: finite pullback FQM} as follows. 
Let $I^{\ast}=I_{{\Q}}\cap L^{\vee}$ and 
$(I')^{\ast} = I_{{\Q}}\cap (L')^{\vee}$. 
We set 
$A=A_{L'}$, 
$I_{1}=L/L'$ and 
$I_{2}=(I')^{\ast}/I'$. 
Then 
$A_{1}\simeq A_{L}$ and  
$A_{2}\simeq A_{K'}$. 
%The condition $(I_{1}, I_{2})\equiv 0$ is satisfied because $I^{\ast}\subset L^{\vee}$. 
We have 
$I_{2}\cap I_{1}^{\perp}=I^{\ast}/I'$ and 
\begin{equation*}
I_{1} \cap I_{2} 
= ( L\cap \langle L', (I')^{\ast} \rangle )/L' 
= \langle L', I \rangle /L' 
= I/I'. 
\end{equation*}
This implies that 
$I_{2}'=I^{\ast}/I \subset A_{L}$ 
and $A'= A_{K}$. 
Thus we have 
\begin{equation*}
\uparrow^{A}_{A_{1}} = \uparrow^{L'}_{L}, \quad 
\downarrow^{A}_{A_{2}} = \downarrow^{L'}_{K'}, \quad 
\downarrow^{A_{1}}_{A'} = \downarrow^{L}_{K}, \quad 
\uparrow^{A_{2}}_{A'} = \uparrow^{K'}_{K},  
\end{equation*}
hence \eqref{eqn: pull push commute FQM} implies \eqref{eqn: pull push commute lattice}.  
\end{proof}


We now prove Lemma \ref{lem: finite pullback FQM}. 

\begin{proof}[(Proof of Lemma \ref{lem: finite pullback FQM})] 
We first justify the isomorphism \eqref{eqn: description of A'}, 
which also implies that $A'$ is nondegenerate. 
We write 
$\hat{I}_{1}'=I_{1}\cap I_{2}^{\perp}$ and  
$\hat{I}_{2}'=I_{2}\cap I_{1}^{\perp}$. 
We shall establish the following commutative diagram: 
\begin{equation*}\label{eqn: CD}
\xymatrix{
    & (\hat{I}_2')^{\perp}\cap I_{1}^{\perp} \ar[rd]^{p_1} &  \\ 
(\hat{I}_{1}')^{\perp}\cap I_{2}^{\perp} \ar[rd]_{p_2} & 
I_{1}^{\perp}\cap I_{2}^{\perp} \ar[r]^{p_1'} \ar@{^{(}-_>}[u] \ar[d]^{p_2'} \ar@{_{(}-_>}[l]
& (I_{2}')^{\perp}\cap A_{1} \ar[d]_{q_{2}}  \\ 
   & (I_{1}')^{\perp}\cap A_{2} \ar[r]_{q_1} & A' 
}
\end{equation*}
Here $p_i$ is the quotient map by $I_{i}$ and 
$p_{i}'$ is the restriction of $p_{i}$. 
Since we have 
$\hat{I}_{1}' = I_{1} \cap (I_{1}^{\perp}\cap I_{2}^{\perp})$ and 
\begin{equation*}
I_{1} / \hat{I}_{1}'  \simeq 
((\hat{I}_2')^{\perp}\cap I_{1}^{\perp}) / (I_{1}^{\perp}\cap I_{2}^{\perp}) 
\simeq (\hat{I}_{2}')^{\perp}/I_{2}^{\perp}, 
\end{equation*}
we see that $p_{1}'$ is surjective and is the quotient map by $\hat{I}_{1}'$. 
This induces the map  
$q_{2}\colon (I_2')^{\perp}\cap A_{1} \to A'$ 
as the quotient map by $I_2'$. 
Similarly, we find that 
$p_2'$ is the quotient map by $\hat{I}_{2}'$ and 
$q_{1}$ is induced as the quotient map by $I_{1}'$. 

We now prove \eqref{eqn: pull push commute FQM}. 
Let $\lambda\in A_{1}$. 
It suffices to show that 
\begin{equation}\label{eqn: pullback FQM}
{\elambda}\uparrow_{A_{1}}^{A}\downarrow^{A}_{A_{2}} = 
|I_{1}\cap I_{2}| \cdot {\elambda}\downarrow^{A_{1}}_{A'}\uparrow^{A_{2}}_{A'}. 
\end{equation}
When $\lambda\not\in (I_{2}')^{\perp}$, we have 
$\mathbf{e}_{\lambda}\! \downarrow^{A_{1}}_{A'}=0$. 
On the other hand, we have 
$(\tilde{\lambda}, \hat{I}_{2}')\not\equiv 0$ 
for every $\tilde{\lambda}\in I_{1}^{\perp}$ 
in the inverse image of $\lambda$. 
In particular, we have $(\tilde{\lambda}, I_2)\not\equiv 0$ and hence 
$\mathbf{e}_{\tilde{\lambda}}\downarrow^{A}_{A_{2}}=0$. 
This implies that  
${\elambda}\uparrow_{A_{1}}^{A}\downarrow^{A}_{A_{2}}=0$. 

%We first consider the case . 
%Then $\tilde{\lambda}\not\in I_{2}^{\perp}$. 
%We have 
%We shall show that ${\elambda}\!\uparrow^{A}_{A_{1}}\downarrow^{A}_{A_{2}}=0$. 
%By definition we have  
%${\elambda}\uparrow^{A}_{A_{1}}\downarrow^{A}_{A_{2}} = 
%\sum_{\lambda'\in \tilde{\lambda}+I_{1}}\mathbf{e}_{\lambda'}\downarrow^{A}_{A_{2}}$. 
%Since 
%$(\tilde{\lambda}, I_{2}) \not\equiv 0$ 
%and $(I_{1}, I_{2})\equiv 0$, 
%we have 
%$(\lambda', I_{2})\not\equiv 0$ 
%for every $\lambda' \in \tilde{\lambda}+I_{1}$. 
%Therefore 
%$\mathbf{e}_{\lambda'}\downarrow^{A}_{A_{2}}=0$. 

%Next we let 
%$\lambda\in (I_{2}')^{\perp}$. 
%Then $\tilde{\lambda}\in I_{1}^{\perp}\cap I_{2}^{\perp}$. 
%We consider the following commutative diagram: 
%\begin{equation*}
%\begin{CD}
%I_{1}^{\perp}\cap I_{2}^{\perp} @>{p_{1}}>> (I_{2}')^{\perp} \\ 
%@V{p_{2}}VV        @VV{p_{2}'}V \\ 
%(I_{1}')^{\perp} @>>{p_{1}'}>  A'
%\end{CD}
%\end{equation*}
%Here 
%$p_{1}$, $p_{2}$, $p_{1}'$, $p_{2}'$ 
%are the quotient maps by 
%$I_{1}$, $I_{2}$, $I_{1}'$, $I_{2}'$ respectively. 
%We write 
%$p=p_{2}'\circ p_{1}=p_{1}'\circ p_{2}$. 
%the quotient map by $I_{1}+I_{2}$. 
%Then we have 
%\begin{equation*}
%\mathbf{e}_{\lambda}\downarrow^{A_{1}}_{A'} \uparrow_{A'}^{A_{2}} = 
%\mathbf{e}_{p(\tilde{\lambda})}\uparrow_{A'}^{A_{2}} = 
%\sum_{\mu'\in I_{1}'} \mathbf{e}_{p_{2}(\tilde{\lambda})+\mu'}. 
%\end{equation*}
%On the other hand, we have 
%\begin{equation*}
%\mathbf{e}_{\lambda}\uparrow_{A_{1}}^{A}\downarrow^{A}_{A_{2}} 
%= \sum_{\mu\in I_{1}} \mathbf{e}_{\tilde{\lambda}+\mu}\downarrow^{A}_{A_{2}} 
%= \sum_{\mu\in I_{1}} \mathbf{e}_{p_{2}(\tilde{\lambda})+p_{2}(\mu)}.  
%\end{equation*}
%Since the fibers of the map 
%$p_{2}\colon I_{1}\to I_{1}'$ 
%consist of $|I_{1}\cap I_{2}|$ elements, 
%we obtain the desired equality \eqref{eqn: pullback FQM}. 

Next let $\lambda\in (I_{2}')^{\perp}$. 
By the above commutative diagram, 
we can choose $\tilde{\lambda}\in I_1^{\perp}\cap I_2^{\perp}$ such that 
$p_{1}'(\tilde{\lambda})=\lambda$. 
Then 
\begin{equation}\label{eqn: push then pull}
\mathbf{e}_{\lambda}\downarrow^{A_{1}}_{A'} \uparrow_{A'}^{A_{2}} \: = \: 
\mathbf{e}_{q_{2}(\lambda)}\uparrow^{A_{2}}_{A'} \: = \: 
\sum_{\mu'\in I_{1}'} \mathbf{e}_{p_{2}'(\tilde{\lambda})+\mu'}. 
\end{equation}
On the other hand, we have 
\begin{equation}\label{eqn: pull then push}
\mathbf{e}_{\lambda}\uparrow_{A_{1}}^{A}\downarrow^{A}_{A_{2}} 
\: = \:  
\sum_{\mu\in I_{1}} \mathbf{e}_{\tilde{\lambda}+\mu}\downarrow^{A}_{A_{2}}  
\: = \: 
\sum_{\mu\in \hat{I}_{1}'} \mathbf{e}_{p_{2}(\tilde{\lambda}+\mu)} 
\: = \: 
\sum_{\mu\in \hat{I}_{1}'} \mathbf{e}_{p_{2}'(\tilde{\lambda})+p_{2}'(\mu)}.  
\end{equation}
Here we used the equality 
$(\tilde{\lambda}+I_{1}) \cap I_{2}^{\perp} = \tilde{\lambda}+\hat{I}_{1}'$. 
Since the map 
$p_{2}'\colon \hat{I}_{1}'\to I_{1}'$ 
is the quotient map by $I_{1}\cap I_{2}$, 
its fibers consist of $|I_{1}\cap I_{2}|$ elements. 
Comparing \eqref{eqn: push then pull} and \eqref{eqn: pull then push},  
we obtain the desired equality \eqref{eqn: pullback FQM}. 
\end{proof}



\subsection{The split case}\label{ssec: general case} 

Next we prove Proposition \ref{prop: functorial} 
in the case when the splitting $L=L'\oplus N$ holds. 
In this case, 
$\uparrow^{L'\oplus N}_{L}$ is identity, 
$I'$ coincides with $I$, 
so Proposition \ref{prop: functorial} takes the following form. 

\begin{lemma}\label{prop: split case}
When the splitting $L=L'\oplus N$ holds, we have for $f \in {\ML}$ 
\begin{equation*}
\xi'(\langle f, \Theta_{N^{+}} \rangle) = \xi(f).  
\end{equation*}
\end{lemma}

\begin{proof}
%By definition of $\ast_{I}$ we have 
%\begin{equation*}
%\langle f_{1}\ast_{I} f_{2}, \Theta_{N^{+}} \rangle = 
%\langle f_{1}{\pushLK}, \Theta_{K^{+}} \rangle \ast_{I} \langle f_{2}, \Theta_{N^{+}} \rangle. 
%\end{equation*}
%On the other hand, if we write 
%$K'=I^{\perp}\cap L'/I$, 
Since $K=K'\oplus N$, we have 
${\ThetaK}=\Theta_{(K')^{+}} \otimes \Theta_{N^+}$ 
under the natural isomorphism 
$\rho_{K^+}\simeq \rho_{(K')^{+}}\otimes \rho_{N^+}$. 
Therefore 
\begin{eqnarray*}
\xi'(\langle f, \Theta_{N^{+}} \rangle) 
& = & 
\langle \langle f, \Theta_{N^{+}} \rangle \! \downarrow^{L'}_{K'}, \: \Theta_{(K')^{+}} \rangle 
= 
\langle \langle f{\pushLK}, \Theta_{N^{+}} \rangle, \Theta_{(K')^{+}} \rangle \\ 
& = & 
\langle f{\pushLK}, \Theta_{N^{+}}\otimes \Theta_{(K')^{+}} \rangle 
= \xi(f). 
\end{eqnarray*}
This proves the desired equality.  
\end{proof}



We can now prove Proposition \ref{prop: functorial} in the general case. 

\begin{proof}[(Proof of Proposition \ref{prop: functorial})]
Let $L'\oplus N\subset L$ and $I'\subset I$ be as in Proposition \ref{prop: functorial}. 
We write 
$L''=L'\oplus N$,  
$K''=(I')^{\perp}\cap L''/I'$ and 
$\xi''= \langle \cdot\!\downarrow^{L''}_{K''}, \Theta_{(K'')^{+}} \rangle$. 
By using Lemma \ref{prop: split case} for $L'\subset L''$ and 
Lemma \ref{prop: finite pullback} for $L''\subset L$, 
we see that  
\begin{equation*}
\xi'(f|_{L'}) 
= \xi'(\langle f\!\uparrow_{L}^{L''}, \Theta_{N^{+}} \rangle ) 
= \xi''(f\!\uparrow_{L}^{L''}) 
= |I/I'|\cdot \xi(f). 
\end{equation*}
This proves Proposition \ref{prop: functorial} in the general case. 
\end{proof}


  


\subsection{Special finite pushforward}\label{ssec: functorial push}

We close this section by noticing that 
$\Theta$-product is also covariantly functorial with respect to 
pushforward to a special type of overlattices. 
Let $I\subset L$ be as before. 

\begin{proposition}\label{prop: functorial push}
Let $L'$ be a sublattice of $L$ of finite index. 
Assume that $L= \langle L', I \rangle$. 
We set $I'=I\cap L'$. 
Then we have 
\begin{equation*}
(f\! \downarrow^{L'}_{L})\ast_{I}(g\! \downarrow^{L'}_{L}) = 
(f\ast_{I'}g)\! \downarrow^{L'}_{L} 
\end{equation*}
for $f, g\in M^!(L')$. 
\end{proposition}

\begin{proof}
We use the notation in the proof of Lemma \ref{prop: finite pullback}. 
Since $I_1=L/L'$ coincides with $I_1\cap I_2=I/I'$, 
we have $I_1 \subset I_2$. 
Hence $I_{1}'=\{ 0 \}$ 
and so the canonical embedding $K'\hookrightarrow K$ is isomorphic. 
Moreover, since $I_{2}=(I')^{\ast}/I'$ coincides with $I_{2}\cap I_{1}^{\perp}$, 
we have $(I')^{\ast}=I^{\ast}$ and hence 
$\langle L, I^{\ast} \rangle = \langle L', (I')^{\ast} \rangle$. 
These equalities imply that 
$\downarrow^{L'}_{K'} \: = \: \downarrow^{L}_{K} \circ \downarrow^{L'}_{L}$. 
Therefore we have 
\begin{equation*}
\xi'(f) = 
\langle f\! \downarrow^{L'}_{K'}, \: \Theta_{(K')^{+}} \rangle = 
\langle f\! \downarrow^{L'}_{L} \downarrow^{L}_{K}, \: \Theta_{K^{+}} \rangle = 
\xi(f\! \downarrow^{L'}_{L}). 
\end{equation*}
As in the case of quasi-pullback, 
this implies 
\begin{eqnarray*}
(f\! \downarrow^{L'}_{L})\ast_{I}(g\! \downarrow^{L'}_{L}) 
& = & 
\xi(f\! \downarrow^{L'}_{L})\cdot (g\! \downarrow^{L'}_{L}) 
\: = \: \xi'(f)\cdot (g\! \downarrow^{L'}_{L}) \\ 
& = & 
(\xi'(f)\cdot g)\! \downarrow^{L'}_{L} 
\: = \: (f\ast_{I'}g)\! \downarrow^{L'}_{L}. 
\end{eqnarray*}
This proves Proposition \ref{prop: functorial push}. 
\end{proof}



%%%%%%% Reference %%%%%%%%%%%%%%%%%%%%%%%%%%%%%

\begin{thebibliography}{99}


\bibitem{Bo95}Borcherds, R. 
\textit{Automorphic forms on $O_{s+2,2}({\R})$ and infinite products.} 
Invent. Math. \textbf{120} (1995), no. 1, 161--213. 

\bibitem{Bo98}Borcherds, R. 
\textit{Automorphic forms with singularities on Grassmannians.} 
Invent. Math. \textbf{132} (1998), no. 3, 491--562. 

\bibitem{Bo00a}Borcherds, R. 
\textit{The Gross-Kohnen-Zagier theorem in higher dimensions.} 
Duke Math. J. \textbf{97} (1999), no. 2, 219--233. 

\bibitem{Bo00b}Borcherds, R. 
\textit{Reflection groups of Lorentzian lattices.} 
Duke Math. J. \textbf{104} (2000), no. 2, 319--366. 

\bibitem{B-K-P-SB}
Borcherds, R.; Katzarkov, L.; Pantev, T.; Shepherd-Barron, N. I. 
\textit{Families of K3 surfaces.} 
J. Algebraic Geom. \textbf{7} (1998), no. 1, 183--193. 

\bibitem{B-F-O-R}Bringmann, K.; Folsom, A.; Ono, K.; Rolen, L. 
\textit{Harmonic Maass forms and mock modular forms: theory and applications.}  
American Mathematical Society, 2017. 

\bibitem{Br}Bruinier, J.~H.
\textit{Borcherds products on $O(2, l)$ and Chern classes of Heegner divisors.} 
Lecture Notes in Math. \textbf{1780}, Springer-Verlag, 2002. 

\bibitem{B-E-F}Bruinier, J.~H.; Ehlen, S.; Freitag, E. 
\textit{Lattices with many Borcherds products.}  
Math. Comp. \textbf{85} (2016), no. 300, 1953--1981. 

\bibitem{B-K}Bruinier, J.~H.; Kuss, M. 
\textit{Eisenstein series attached to lattices and modular forms on orthogonal groups.} 
Manuscripta Math. \textbf{106} (2001), no. 4, 443--459. 

\bibitem{E-Z}Eichler, M.; Zagier, D. 
\textit{The theory of Jacobi forms.} 
Birkh\"auser, 1985. 

\bibitem{Gr}Gritsenko, V.~A. 
\textit{Modular forms and moduli spaces of abelian and K3 surfaces.} 
St. Petersburg Math. J. \textbf{6} (1995), no. 6, 1179--1208.  

\bibitem{G-H-S}Gritsenko, V.; Hulek, K.; Sankaran, G.~K.
\textit{Abelianisation of orthogonal groups and the fundamental group of modular varieties.} 
J. Algebra \textbf{322} (2009), no. 2, 463--478. 

\bibitem{G-N}Gritsenko, V.~A.; Nikulin, V.~V. 
\textit{Automorphic forms and Lorentzian Kac-Moody algebras. II.}  
Internat. J. Math. \textbf{9} (1998), no. 2, 201--275. 

\bibitem{Ma}Ma, S. 
\textit{Quasi-pullback of Borcherds products.}  
Bull. Lond. Math. Soc. \textbf{51} (2019), no. 6, 1061--1078. 

\bibitem{Mc}McGraw, W.~J. 
\textit{The rationality of vector valued modular forms associated with the Weil representation.}  
Math. Ann. \textbf{326} (2003), no. 1, 105--122. 

\bibitem{Sc}Schwagenscheidt, M. 
\textit{Eisenstein series for the Weil representation.} 
J. Number Theory \textbf{193} (2018), 74--90. 

\bibitem{W-W}Wang, H.; Williams, B. 
\textit{Borcherds products of half-integral weight.}
arXiv: 2007.00055. 

\end{thebibliography}

\end{document}