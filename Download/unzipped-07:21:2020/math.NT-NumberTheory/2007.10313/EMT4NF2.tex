\documentclass[10pt,reqno]{amsart}
\usepackage{amsmath}
\usepackage{amsthm}
\usepackage{amssymb}
\usepackage{graphicx}
\usepackage{amsfonts}
\usepackage{latexsym, wasysym}
\usepackage{hyperref,subcaption}



\graphicspath{{IMAGES/}}

%%%%%%%%%%%%%%%%%%%
\usepackage{enumitem}
\setlist[itemize]{leftmargin=*}
\setlist[enumerate]{leftmargin=*}
\renewcommand{\labelenumi}{(\alph{enumi}) }
%%%%%%%%%%%%%%%%%%%


\usepackage[font=footnotesize,labelfont=bf]{caption}
\usepackage[font=scriptsize,labelfont=bf]{subcaption}

\usepackage[table]{xcolor}
\usepackage{cite}
\usepackage{multicol}

\theoremstyle{plain}
%\numberwithin{theorem}{section}
\newtheorem{theorem}{Theorem}
\newtheorem{corollary}[theorem]{Corollary}
\newtheorem{lemma}[theorem]{Lemma}
\newtheorem{proposition}[theorem]{Proposition}

\theoremstyle{definition}
\newtheorem{question}[theorem]{Question}
\newtheorem{example}[theorem]{Example}
\newtheorem{remark}[theorem]{Remark}


%\newtheorem{definition}[equation]{Definition}
%
%%\newtheorem{theorem}{Theorem}[section]
%\newtheorem{corollary}[thm]{Corollary}
%\newtheorem{lemma}[thm]{Lemma}
%\newtheorem{prop}[thm]{Proposition}
%\newtheorem{ax}{Axiom}
%\newtheorem{prob}{Problem}
%\newtheorem{quest}{Question}
%
%\theoremstyle{definition}
%\newtheorem{defn}{Definition}[section]
%
%\theoremstyle{remark}
%\newtheorem{rem}{Remark}[section]
%\newtheorem*{notation}{Notation}
%
%\newtheorem*{remark}{Remark}

\newcommand{\comment}[1]{\marginpar{\color{blue}$\bullet$ \scriptsize #1}}
\newcommand{\highlight}[1]{{\color{blue} #1}}
\newcommand{\red}[1]{{\color{red} #1}}
\newcommand{\purple}[1]{{\color{blue} #1}}


%% New Commands
\newcommand{\Q}{\mathbb{Q}}
\newcommand{\R}{\mathbb{R}}
\newcommand{\N}{\mathbb{N}}
\newcommand{\F}{\mathbb{F}}
\newcommand{\K}{\mathbb{K}}
\newcommand{\Z}{\mathbb{Z}}
\newcommand{\C}{\mathbb{C}}
\newcommand{\PP}{\mathfrak{p}}
\renewcommand{\O}{\mathcal{O}}
\newcommand{\T}{\mathsf{T}}
\newcommand{\E}{\mathcal{E}}
\newcommand{\I}{\mathcal{I}}
\newcommand{\li}{\operatorname{li}}
\newcommand{\Li}{\operatorname{Li}}
\newcommand{\Ei}{\operatorname{Ei}}

\renewcommand{\Re}{\operatorname{Re}}
\renewcommand{\Im}{\operatorname{Im}}

\renewcommand{\pmod}[1]{\,\,(\operatorname{mod}#1)}

\let\oldenumerate=\enumerate
	\def\enumerate{
	\oldenumerate
	\setlength{\itemsep}{5pt}
	}
\let\olditemize=\itemize
	\def\itemize{
	\olditemize
	\setlength{\itemsep}{5pt}
	}

\linespread{1.05}

\allowdisplaybreaks

%%
%% MAIN DOCUMENT
%%

\begin{document}

\title[Explicit Mertens' Theorems for Number Fields II]{Explicit Mertens' Theorems for Number Fields and Dedekind Zeta Residue Bounds II: Unconditional Estimates}

	\author[S.R.~Garcia]{Stephan Ramon Garcia}
	\address{Department of Mathematics, Pomona College, 610 N. College Ave., Claremont, CA 91711} 
	\email{stephan.garcia@pomona.edu}
	\urladdr{\url{http://pages.pomona.edu/~sg064747}}
	
\author[E.S.~Lee]{Ethan Simpson Lee}
\address{School of Science, UNSW Canberra at the Australian Defence Force Academy, Northcott Drive, Campbell, ACT 2612} 
\email{ethan.s.lee@student.adfa.edu.au}
\urladdr{\url{https://www.unsw.adfa.edu.au/our-people/mr-ethan-lee}}
	
\thanks{SRG supported by NSF Grant DMS-1800123.}

\subjclass[2010]{11N32, 11N05, 11N13}

\keywords{Number field, discriminant, Mertens' theorem, prime ideal, prime counting function, Chebotar\"ev density theorem, prime ideal, Dedekind zeta function, residue}

\begin{abstract}
We obtain unconditional, effective number-field analogues of the three Mertens' theorems, all with explicit constants and valid for $x\geq 2$. To this end, we also state related unconditional bounds, with explicit constants, for the residue of the corresponding Dedekind zeta-function at $s=1$.
\end{abstract}

\maketitle


\section{Introduction}

The following three results are collectively known as Mertens' theorems \cite{Mertens}:
\begin{align*}
\sum_{p\leq x} \frac{\log p}{p} &= \log x + O(1), \\
\sum_{p \leq x} \frac{1}{p} &= \log \log x + M + O\bigg(\frac{1}{\log x} \bigg), \\
\prod_{p\leq x} \bigg(1 -\frac{1}{p} \bigg) &= \frac{e^{-\gamma}}{\log x} \big(1+o(1) \big).
\end{align*}
Here $p$ denotes a prime number, $M = 0.2614\ldots$ is the Meissel--Mertens constant, and $\gamma = 0.5772\ldots$ is the Euler--Mascheroni constant. 
Assuming the generalized Riemann hypothesis (GRH), 
the authors obtained in \cite{EMT4NF1} effective number-field analogues of these results, in which the implied constants are explicit.
We approach the same family of problems here unconditionally; that is, without assuming any unproved conjectures.
Naturally, our error terms are inferior to those obtained under GRH in \cite{EMT4NF1}, but they are nevertheless completely explicit and widely applicable.

In what follows, let $\K$ denote a number field with ring of algebraic integers $\O_{\K}$,
let $n_{\K} = [\K : \Q]$ denote the degree of $\K$, let $\Delta_{\K}$ is the absolute value of the discriminant of $\K$,
and let $N(\mathfrak{a})$ denote the norm of an ideal $\mathfrak{a}\subset\O_{\K}$.
Our previous GRH-based approach used a conditional estimate due to Greni\'e--Molteni \cite[Cor.~1]{Grenie2019} 
on the prime-ideal counting function $\pi_{\K}(x) = \sum_{N(\PP)\leq x} 1$;
here  $\PP$ runs over the prime ideals in $\O_{\K}$.  They proved that
$\pi_{\K}(x) = \Li(x) + R_{\K}(x)$ for $x \geq 2$,
in which $R_{\K}(x) = O( x^{1/2}\log x)$ is completely explicit in its dependence on 
$n_{\K}$ and $\Delta_{\K}$.

Unconditional estimates on $\pi_{\K}(x)$ of strength comparable to Greni\'e--Molteni's are currently unavailable.
Lagarias--Odlyzko \cite[Thm.~1.1, Thm.~1.3]{LagariasOdlyzko} established 
an effective Chebotar\"{e}v density theorem,
a simultaneous generalization of the prime number theorem \cite{de1896fonction, hadamard1896distribution} and Landau's prime ideal theorem \cite{Landau}, both with and without GRH.
They proved unconditionally that for sufficiently large $x$,
\begin{equation}\label{eqn:PIT}
    \pi_{\K}(x) = \Li(x) + \Li(x^\beta) + O\big(x\exp\big(- c {n_{\K}}^{-1/2}(\log{x})^{1/2}\big)\big),
\end{equation}
in which $c$ is effectively computable and $\beta\in (0,1)$ is the possible exceptional zero 
of the corresponding Dedekind zeta-function, defined for $\Re{s} > 1$ by
\begin{equation}\label{eq:DedekindZeta}
\zeta_{\K}(s) = \sum_{\mathfrak{a} \subseteq \O_{\K}} \frac{1}{N(\mathfrak{a})^s}
= \prod_{\PP} \bigg(1 - \frac{1}{N(\PP)^s}\bigg)^{-1}.
\end{equation}
See \cite{Winckler} for an explicit version of \eqref{eqn:PIT}, valid for $x \geq \exp(8 n_{\K}( \log(150867\Delta_{\K}^{44/5}))^2$,
and \cite{Murty,ThornerZaman} for conditional improvements.
If the exceptional zero $\beta$ does not exist, 
then one may remove the term $\Li(x^\beta)$ from \eqref{eqn:PIT}
and other estimates of the same type. 
Otherwise, one can use the bound \cite[p.~148]{Stark}:
\begin{equation*}
\beta \leq 1 - \frac{6}{\pi \Delta_{\K}^{1/n_{\K}}}.
\end{equation*}
We also mention a result of Heilbronn \cite{Heilbronn}, 
which was later generalized by Stark \cite{StarkBS}. 
If $\mathbb{L}$ is a normal extension of $\K$ and $\K_2$ is the union of all quadratic extensions of $\K$ contained in $\mathbb{L}$,
then every real simple zero of $\zeta_{\mathbb{L}}(s)$ is a simple zero of $\zeta_{\K_2}(s)$. Thus, if $\mathbb{L}$ has no quadratic subfield, then no exceptional zero exists.


The Dedekind zeta-function \eqref{eq:DedekindZeta}
is analytic on $\Re s > 1$ and extends meromorphically to $\C$ except for a simple pole at $s=1$. 
The generalized Riemann hypothesis claims that all the zeros of 
$\zeta_{\K}(s)$ with $\Re s \in (0,1)$ satisfy $\Re s = 1/2$.
The analytic class number formula asserts
that the residue of $\zeta_{\K}(s)$ at $s=1$ is
\begin{equation}\label{eqn:residue_class_ana_form}
\kappa_{\K} = \frac{2^{r_1}(2\pi)^{r_2}h_{\K}R_{\K}}{w_{\K}\sqrt{\Delta_{\K}}},
\end{equation}
in which $r_1$ the number of real places of $\K$, $r_2$ the number of complex places of $\K$,
$w_{\K}$ the number of roots of unity in $\K$,
$h_{\K}$ the class number of $\K$, and $R_{\K}$ the regulator of $\K$ \cite{Lang}.


We do not wish our error bounds to depend upon the potential exceptional zero $\beta$
of $\zeta_{\K}$.  Consequently, we must eschew estimates of the form \eqref{eqn:PIT} in favor
of an alternative.  We use a recent result of  Debaene (see Lemma \ref{Lemma:Debaene}), who obtained an effective estimate for the
ideal-counting function for $\K$ that does not require information about the zeros of $\zeta_{\K}$ \cite[Cor.~2]{Debaene}. Debaene's proof is a delicate lattice-based argument that employs Schmidt's partition trick, the Selberg sieve in number fields, and Artin reciprocity.  This avoids the exceptional zero issue entirely.

Our overall roadmap follows Diamond--Halberstam \cite[p.~128-9]{DiamondHalberstam}, 
although significant adaptations and technical lemmas are required to make things completely explicit
(we encountered some difficulties in attempting to adapt Hardy's classical method \cite{HardyNote1927,HardyNote2935,BardestaniFreiberg}, although such an approach might still be possible).

Our main result is the following effective version of Mertens' theorems for number fields.
It is unconditional and the constants involved are completely explicit and depend only upon the two
easily-obtained parameters $n_{\K}$ and $\Delta_{\K}$;
see Remarks \ref{Remark:Kappa} and \ref{Remark:Regulator}.
Moreover, our estimates are valid for all $x \geq 2$.

\begin{theorem}\label{Theorem:Main}
Let $\K$ be a number field with $n_{\K} \geq 2$.  Then for $x \geq 2$,
\begin{align}
    \sum_{N(\PP)\leq x}\frac{\log{N(\PP)}}{N(\PP)} &\,=\, \log{x} + A_{\K}(x) ,\label{eq:A1}\tag{A1}\\[5pt]
    \sum_{N(\PP)\leq x}\frac{1}{N(\PP)} &\,=\, \log\log{x} + M_{\K} + B_{\K}(x),\label{eq:B1} \tag{B1}\\[5pt]
    \prod_{N(\PP)\leq x}\left(1 - \frac{1}{N(\PP)}\right) &\,=\, \frac{e^{-\gamma}}{\kappa_{\K}\log{x}} \big(1 + C_{\K}(x) \big),\label{eq:C1}\tag{C1}
\end{align}
in which
\begin{align}
M_{\K} \,&=\, \gamma + \log{\kappa_{\K}} + \sum_{\PP}\left[\frac{1}{N(\PP)} + \log\left(1 - \frac{1}{N(\PP)}\right)\right]\label{eq:M1}\tag{M1},\\
\Lambda_{\K} &= {n_{\K}}^{10 {n_{\K}}^2}(R_{\K}h_{\K})^{\frac{1}{n_{\K}}}(1 + \log{R_{\K} h_{\K}})^{\frac{(n_{\K}-1)^2}{n_{\K}}}, \label{eq:Lambda} \tag{$\Lambda$}\\
|A_{\K}(x)| 
&\leq n_{\K}  + 18.2 \dfrac{\Lambda_{\K}n_{\K}^2}{\kappa_{\K}} + 1,
%\leq \frac{17.1 \Lambda_{\K}n_{\K}^2 + 2\Lambda_{\K} n_{\K} + \kappa_{\K}(2n_{\K}+1)}{\kappa_{\K}}, 
\label{eq:A2}\tag{A2}\\[5pt]
|B_{\K}(x)| &\leq 
\bigg(n_{\K}  + 18.2 \dfrac{\Lambda_{\K}n_{\K}^2}{\kappa_{\K}} + 1 \bigg) \frac{2}{\log x},
%\frac{34.2 \Lambda_{\K}n_{\K}^2 + 4\Lambda_{\K} n_{\K} + \kappa_{\K}(4n_{\K}+2)}{\kappa_{\K}\log{x}} ,
\quad\text{and}\label{eq:B2}\tag{B2}\\[5pt]
|C_{\K}(x)| &\,\leq\, |E_{\K}(x)|e^{| E_{\K}(x)|} \qquad \text{with}\qquad |E_{\K}(x)| \,\leq\, \frac{n_{\K}}{x-1} + |B_{\K}(x)|.\label{eq:C2}\tag{C2}
\end{align}
In particular, $E_{\K}(x) = o(1)$, hence $C_{\K}(x) = o(1)$ as $x \to \infty$.  Furthermore,
\begin{equation}\label{eq:M2}\tag{M2}
    \gamma + \log \kappa_{\K} -n_{\K} \,\leq\, M_{\K} \,\leq\,  \gamma + \log \kappa_{\K}.
\end{equation}
All quantities above can be effectively bounded, with explicit constants, in terms of $\Delta_{\K}$ and $n_{\K}$ alone;
see the remarks below.
\end{theorem}

\begin{remark}
For $n_{\K} = 1$, that is $\K = \Q$, our approach provides similar estimates.  However,
these are weaker than the error bounds given by Rosser--Schoenfeld \cite[Thms.~5-7]{Rosser}.
Since much more is known about the behavior of the Riemann zeta function than
a generic Dedekind zeta function, this discrepancy is not surprising.
\end{remark}

\begin{remark}\label{Remark:Kappa}
Since the residue $\kappa_{\K}$ of the Dedekind zeta function $\zeta_{\K}(s)$ at $s=1$ appears
in the bound \eqref{eq:M2}, it is of interest to exhibit explicit bounds for it.  
An elegant upper bound is due to Louboutin \cite{Louboutin00}:
\begin{equation}\label{eq:LouboutinKappa}\qquad
    \kappa_{\K} \leq \left(\frac{e\log{\Delta_{\K}}}{2(n_{\K} - 1)}\right)^{n_{\K} - 1}
    \quad \text{for $n_{\K} \geq 2$}.
\end{equation}
At the level of generality we seek, the upper bound \eqref{eq:LouboutinKappa} appears to be the best
available in the literature.  Given the location (or nonexistence) of an exceptional zero of $\zeta_{\K}$,
much more can be said \cite{Louboutin01,Louboutin11, Louboutin03};
we do not discuss the topic further here.  

We also obtain several lower bounds on $\kappa_{\K}$;
all details can be found in Section \ref{sec:ResidueBounds}.  First, there is the weak, but simple, bound
\begin{equation*}
\kappa_{\K} \geq \frac{0.0884}{\sqrt{\Delta_{\K}}}.
\end{equation*}
For $n_{\K} \geq 3$, an analysis of Stark's paper \cite{StarkBS} yields
the asymptotically better bound 
\begin{equation*}
    \kappa_{\K} > \frac{0.001448029}{n_{\K} g(n_{\K}){\Delta_{\K}}^{1/n_{\K}}},
\end{equation*}
in which $g(n_{\K})=1$ if $\K$ has a normal tower over $\Q$ and 
$g(n_{\K}) = n_{\K}!$ otherwise.
Section \ref{sec:ResidueBounds} contains several improvements in various special cases.
\end{remark}

\begin{remark}\label{Remark:Regulator}
The product of the class number $h_{\K}$ and regulator $R_{\K}$ that appears in the definition of $\Lambda_{\K}$
equals $\kappa_{\K}\sqrt{\Delta_{\K}}$ up to constant factors that depend upon $n_{\K}$ in
an explicit manner \cite[(7), p.~42]{MR2145727}:
\begin{equation*}
\frac{ \sqrt{ \Delta_{\K}}}{3^{n_{\K}}} \kappa_{\K}
\,\leq\, h_{\K} R_{\K} 
\,\leq\, \frac{4 \sqrt{ \Delta_{\K}}}{\pi} \kappa_{\K}.
\end{equation*}
The upper bound above may be combined with \eqref{eq:LouboutinKappa} to replace
the $R_{\K} h_{\K}$ in \eqref{eq:Lambda} by an explicit quantity involving only $n_{\K}$ and $\Delta_{\K}$.  We leave the details to the reader.
\end{remark}


%%%%%%%%%%%%%%%%%
\subsection*{Outline of the paper}
Section \ref{Section:Proof} contains the proof of Theorem \ref{Theorem:Main},
which occupies the bulk of the paper.
In Section \ref{sec:ResidueBounds}, we state several unconditional explicit bounds for $\kappa_{\K}$.
We conclude in Section \ref{Section:Future} with remarks and future questions.

\subsection*{Acknowledgements}
The authors thank Matteo Bordignon, Tristan Freiberg, Eduardo Friedman, and Tim Trudgian for their feedback and suggestions.

%%%%%%%%%%%%%%%%%%%%%%%%%%%%%%%%%%%%%%

%%%%%%%%%%%%%%%%%%%%%%%%%%%%%%%%%%%%%%%%%%%%%%%%%%%%%%%%%%%%%%%%%%%%%%%%%%%%%%%%
\section{Proof of Theorem \ref{Theorem:Main}}\label{Section:Proof}

We split the proof of Theorem \ref{Theorem:Main} across several subsections.  The proofs of \eqref{eq:A1} and \eqref{eq:A2} are the most involved since they require a series of technical lemmas and computations to ensure that all of the constants produced remain explicit.
Once \eqref{eq:A1} and \eqref{eq:A2} are established, the rest of the proof falls into place.

In what follows, $f(x) = O^{\star}(g(x))$ means $|f(x)| \leq |g(x)|$ for all $x$ in a pre-defined range (often
$x \geq 2$).
This is similar to Landau's big-$O$ notation, except the implied constant is always one.
To begin, we require some preliminary remarks.

\subsection{Preliminaries}

Fix a number field $\K$ with $n_{\K}\geq 2$ and ring of integers $\O_{\K}$. 
Let $I_{\K}(n)$ denote the number of ideals with norm $n$, and $P_{\K}(n)$ denote the number of prime ideals in $\O_{\K}$ with norm $n$.
Borevich--Shafarevich \cite[p. 220]{Borevich} tell us that if $p^k$ is a rational prime power, then $I_{\K}(p^k)\leq (k+1)^{n_{\K}}$.
The total multiplicativity of the norm means that a non-prime ideal may have norm $p^k$, so one might suspect that a tighter bound
can be found for $P_{\K}(p^k)$. This expectation is well-founded.

If $\PP\subset\mathcal{O}_{\K}$ is a prime ideal, then it divides exactly one rational prime $p$ and $N(\mathfrak{p}) = p^k$ for some $1\leq k\leq n_{\K}$ \cite[Thm.~5.14c]{StewartTall}.
Moreover, $p\O_{\K}$ has a unique factorization
\begin{equation*}
p\O_{\K} = \PP_1^{e_1} \cdots \PP_r^{e_r}
\end{equation*}
into prime ideals $\PP_i$, where $e_i \in \N$ is the ramification index of $\PP_i$.
The $\PP_i$ are the only prime ideals in $\K$ with norm equal to a power of $p$.
In fact, $N(\PP_i) = p^{f_i}$, in which the inertia degrees $f_i$ satisfy $f_i\leq n_{\K}$ and
\begin{equation*}
e_1 f_1 + \cdots + e_r f_r = n_{\K}.
\end{equation*}
In particular, for each rational prime $p$ the corresponding inertia degrees satisfy
\begin{equation}\label{eq:InertiaSum}
    {\sum_{f_i}} f_i \leq n_{\K}\quad\text{hence}\quad P_{\K}(p^k)\leq \bigg\lfloor \frac{n_{\K}}{k} \bigg\rfloor \leq \frac{n_{\K}}{k}.
\end{equation}

We require the following technique to obtain estimates for sums over prime ideals.
Suppose $g$ is a non-negative arithmetic function and recall that a prime ideal $\PP$ with $N(\PP) \leq x$ lies over exactly one rational prime $p \leq x$.  Then
\begin{equation*}
    G(x)
    = \sum_{N(\PP)\leq x} g(N(\PP))
    \leq \sum_{p\leq x}{\sum_{f_i}} g(p^{f_i}),
\end{equation*}
in which $\sum_{f_i}$ denotes the sum over the inertia degrees $f_i$ of the prime ideals lying over $p$. 
If one can apply \eqref{eq:InertiaSum}, the previous sum can be simplified.  For example,
\begin{equation*}
    \theta_{\K}(x)
    = \sum_{N(\PP)\leq x} \log{N(\PP)} 
    \leq \sum_{p\leq x}\sum_{f_i} \log{p^{f_i}}
    = \sum_{p\leq x}\bigg(\sum_{f_i} f_i\bigg) \log{p}
    \leq n_{\K}\theta_{\Q}(x),
\end{equation*}
in which $\theta_{\Q}$ denotes the Chebyshev theta function and $\theta_{\K}$ its number-field analogue.

Finally, to avoid the problems which might arise from an exceptional zero of $\zeta_{\K}$, we introduce the ideal-counting function 
\begin{equation*}
\I_{\K}(x) = \sum_{n\leq x} I_{\K}(n).
\end{equation*}
This is the number-field analogue of the integer-counting function $\lfloor x \rfloor$. Our proof relies 
on the following unconditional result of Debaene \cite[Cor.~2]{Debaene}.

\begin{lemma}[Debaene]\label{Lemma:Debaene}
For $x \geq 1$, 
\begin{equation*}
\I_{\K}(x) = \kappa_{\K} x + O^{\star}(\Lambda_{\K}x^{1-\frac{1}{n_{\K}}}),
\end{equation*}
in which
\begin{equation*}
\Lambda_{\K} = {n_{\K}}^{10 {n_{\K}}^2}(R_{\K}h_{\K})^{\frac{1}{n_{\K}}}(1 + \log{R_{\K} h_{\K}})^{\frac{(n_{\K}-1)^2}{n_{\K}}}.
\end{equation*}
\end{lemma}

\begin{remark}
If $\K= \Q$, then $\Lambda_{\Q} = 1$ and we have
$\I_{\Q}(x) = x + O^{\star}(1)$. However, $\I_{\Q}(x) = \lfloor x \rfloor \leq x$ is more precise.
This is one reason Rosser--Schoenfeld obtain better error estimates in Mertens' theorems for $\K = \Q$ \cite{Rosser};  the fact that the Riemann zeta function has no exceptional zero provides them 
more options as well.
\end{remark}

%%%%%%%%%%%%%%%%%%%%%%%%%%%
\subsection{Preparatory lemmas to prove \eqref{eq:A1} and \eqref{eq:A2}}

Before we establish \eqref{eq:A1} and \eqref{eq:A2} in Section \ref{Section:Proof:A}, we need 
several technical lemmas to estimate
\begin{equation*}
    \sum_{N(\PP)\leq x} \I_{\K} \bigg(\frac{x}{N(\PP)}\bigg) \log{N(\PP)}\qquad\text{and}\qquad\sum_{N(\PP)\leq x} \log{N(\PP)} \sum_{j\geq 2} \I_{\K} \left(\frac{x}{N(\PP^j)}\right).
\end{equation*}
We need the following result of Rosser--Schoenfeld \cite[Thm.~9]{Rosser}: 
\begin{equation}\label{eq:Rosser}
    \theta(x) = \sum_{p\leq x}\log p  < 1.01624x < 1.1x,
    \quad \text{for $x>0$}.
\end{equation}
A recent improvement on \eqref{eq:Rosser} yields smaller constants
throughout; see Remark \ref{Remark:Theta}.

\begin{lemma}\label{Lemma:Painful}
    For $\alpha \geq 0$ and $x \geq 2$, 
    \begin{equation*}
        \sum_{p\leq x} \frac{\log p}{p^{\alpha}}
        <
        \begin{cases}
            \dfrac{1.1}{1-\alpha}x^{1-\alpha} & \text{if $0 \leq \alpha \leq 1$},\\[10pt]
            1.6 \log x & \text{if $\alpha=1$},\\[5pt]
            \dfrac{1.1\,\alpha}{(\alpha - 1)2^{\alpha-1}}  & \text{if $\alpha > 1$}.
        \end{cases}
    \end{equation*}
\end{lemma}

\begin{proof}
    For $x \geq 2$, partial summation and \eqref{eq:Rosser} yield 
    \begin{align*}
        \sum_{p\leq x} \frac{\log p}{p^{\alpha}}
        &= \frac{\theta(x)}{x^{\alpha}} + \alpha\int_2^x \frac{\theta(t)}{t^{\alpha+1}}\,dt 
        < 1.1\left( \frac{1}{x^{\alpha-1}} + \alpha \int_2^x \frac{dt}{t^{\alpha}} \right) \\
        &=
        \begin{cases}
            1.1 \left( x^{1-\alpha} + \dfrac{\alpha}{1-\alpha} \left( x^{1-\alpha} - 2^{1-\alpha} \right) \right)  & \text{if $0<\alpha < 1$},\\[10pt]
            1.1 (\log x - \log 2 + 1) & \text{if $\alpha=1$},\\[5pt]
            1.1 \left(\dfrac{1}{x^{\alpha-1}} + \dfrac{\alpha}{\alpha - 1} \left( \dfrac{1}{2^{\alpha-1}} - \dfrac{1}{x^{\alpha-1}} \right) \right)  & \text{if $\alpha > 1$},\\
        \end{cases}
    \end{align*}
    which implies the desired result.
\end{proof}

The preceding lemma and some computation yields the next lemma.

\begin{lemma}\label{Lemma:Multipart}
    For $j \in \N$, $n_{\K}\geq 2$ and $x \geq 2$,
    \begin{equation*}
        x^{1 - \frac{1}{n_{\K}}} \!\!\! \sum_{N(\PP)\leq x} \frac{\log N(\PP)}{N(\PP)^{j (1- \frac{1}{n_{\K}})}}
        <
        \begin{cases}
       1.1 n_{\K}^2 x & \text{if $j =1$},\\[5pt] 
%       1.1 n_{\K}^2 x & \text{if $n_{\K}=1$, or $n_{\K}\geq 2$ and $j =1$},\\[5pt] 
        1.6n_{\K}x^{ \frac{1}{2} }  \log x & \text{if $n_{\K}=j =2$},\\[5pt]
        \dfrac{8.8n_{\K} x^{1 - \frac{1}{n_{\K}}} }{2^{j /2}} & \text{otherwise}.
    \end{cases}
\end{equation*}
\end{lemma}

\begin{proof}
First observe that \eqref{eq:InertiaSum} implies
\begin{align}
\sum_{N(\PP)\leq x} \frac{\log N(\PP)}{N(\PP)^{j (1- \frac{1}{n_{\K}})}}
&\leq \sum_{p\leq x} \sum_{f_i} \frac{\log (p^{f_i}) }{ p^{f_i j (1- \frac{1}{n_{\K}} )} }  \nonumber \\
&\leq \sum_{p\leq x} \sum_{f_i} f_i \frac{\log p }{ p^{j (1- \frac{1}{n_{\K}}) } }  \nonumber \\
&\leq n_{\K} \sum_{p\leq x}\frac{\log p }{ p^{j (1- \frac{1}{n_{\K}}) } } \label{eq:ContinueBound}
 \end{align}
in which $\sum_{f_i}$ denotes the sum over the inertia degrees $f_i$ of the prime ideals lying over the
rational prime $p$. Next substitute 
\begin{equation*}
\alpha = j  \bigg(1- \frac{1}{n_{\K}} \bigg)
\end{equation*}
into Lemma \ref{Lemma:Painful}
and multiply through by $x^{1 - \frac{1}{n_{\K}}}$.

First, if $j =1$, then $0 < \alpha < 1$ since $n_{\K}\geq 2$ and \eqref{eq:ContinueBound} can be bounded from above by the first case of Lemma \ref{Lemma:Painful}:
\begin{equation*}
n_{\K} x^{1 - \frac{1}{n_{\K}}} \sum_{p\leq x} \frac{\log p}{p^{j (1- \frac{1}{n_{\K}})}}
< n_{\K}  x^{1 - \frac{1}{n_{\K}}} \frac{1.1}{1 - (1- \frac{1}{n_{\K}})} x^{1 - (1- \frac{1}{n_{\K}})}
= 1.1 n_{\K}^2x.
\end{equation*}
Next, if $n_{\K} = j  = 2$, then $\alpha = 1$ and the second case of Lemma \ref{Lemma:Painful} immediately yields the desired upper bound.
Finally, suppose that $n_{\K} \geq 3$ and $j  \geq 2$, or $n_{\K} =2$ and $j  \geq 3$. We first maximize
\begin{equation*}
f(\alpha) = \frac{\alpha}{\alpha -1}
\qquad \text{when $n_{\K} \geq 3$ and $j  \geq 2$, or when $n_{\K} =2$ and $j  \geq 3$}.
\end{equation*}
Under these conditions, 
\begin{equation*}
\alpha= j \left(1 - \frac{1}{n_{\K}} \right) \geq \min\left\{ 2\left(1 - \frac{1}{3}\right), 3\left(1- \frac{1}{2}\right) \right\} = \min\left\{\frac{4}{3}, \frac{3}{2} \right\} = \frac{4}{3},
\end{equation*}
and hence $f(\alpha) \leq 4$ since $f$ is decreasing on $(1,\infty)$.
In particular, the third case of Lemma \ref{Lemma:Painful} permits us to bound \eqref{eq:ContinueBound}
from above as follows:
\begin{align*}
n_{\K}x^{1 - \frac{1}{n_{\K}}} \sum_{p\leq x} \frac{\log p}{p^{j (1- \frac{1}{n_{\K}})}} < \dfrac{1.1n_{\K} \alpha x^{1 - \frac{1}{n_{\K}}}}{ \big(\alpha - 1 \big)2^{j (1- \frac{1}{n_{\K}})-1}} 
&\leq \dfrac{4.4n_{\K}x^{1 - \frac{1}{n_{\K}}}}{ 2^{j (1- \frac{1}{n_{\K}})-1}} 
\leq \frac{8.8n_{\K} x^{1 - \frac{1}{n_{\K}}} }{2^{j /2}}. \qedhere
\end{align*}
\end{proof}

Our next two lemmas are estimates obtained with the aid of Lemma \ref{Lemma:Multipart}.
The first one is rather straightforward, but the second is much more involved.

\begin{lemma}\label{Lemma:FirstPower}
For $x\geq2$,
\begin{equation*}
\sum_{N(\PP)\leq x} \I_{\K} \bigg(\frac{x}{N(\PP)}\bigg) \log{N(\PP)} 
= \kappa_{\K} x \sum_{N(\PP)\leq x} \frac{\log{N(\PP)}}{N(\PP)} 
+ 1.1 \Lambda_{\K}n_{\K}^2 O^{\star} (x) .
\end{equation*}
\end{lemma}

\begin{proof}
Lemma \ref{Lemma:Debaene} and Lemma \ref{Lemma:Multipart} with $j  = 1$ imply
\begin{align*}
&\sum_{N(\PP)\leq x} \I_{\K} \bigg(\frac{x}{N(\PP)}\bigg) \log{N(\PP)}\\
&\qquad= \sum_{N(\PP)\leq x} \Bigg( \kappa_{\K}  \bigg(\frac{x}{N(\PP)}\bigg) 
+ O^{\star}\bigg(\Lambda_{\K} \bigg(\frac{x}{N(\PP)}\bigg)^{1-\frac{1}{n_{\K}}} \Bigg) \log N(\PP)  \\
&\qquad= \kappa_{\K} x \!\!  \sum_{N(\PP)\leq x} \frac{\log{N(\PP)}}{N(\PP)} 
+ \Lambda_{\K} O^{\star}\Bigg(x^{1 - \frac{1}{n_{\K}}}\sum_{N(\PP)\leq x}  
\frac{ \log N(\PP)}{N(\PP)^{1-\frac{1}{n_{\K}}} } \Bigg)  \\
&\qquad= \kappa_{\K} x \!\!  \sum_{N(\PP)\leq x} \frac{\log{N(\PP)}}{N(\PP)} 
+ 1.1 \Lambda_{\K}n_{\K}^2 O^{\star} (x)  .\qedhere
\end{align*}
\end{proof}


\begin{lemma}\label{Lemma:HigherPowers}
For $x\geq 2$,
\begin{equation*}
\sum_{N(\PP)\leq x} \log{N(\PP)} \sum_{j\geq 2} \I_{\K} \left(\frac{x}{N(\PP^j)}\right) 
\,\leq\, \Xi_{\K}(x),
\end{equation*}
in which
\begin{equation}\label{eq:Xi}
\Xi_{\K}(x)=
\begin{cases}
\kappa_{\K} n_{\K} x + \Lambda_{\K} O^{\star}( 6.4\,x^{ \frac{1}{2} }\log x +42.5\,x^{\frac{1}{2}}  )  & \text{if $n_{\K} = 2$}, \\[5pt]
\kappa_{\K} n_{\K} x + 15.1 \,\Lambda_{\K} O^{\star}( n_{\K}^2 x^{1 - \frac{1}{n_{\K}}}  )  & \text{if $n_{\K} \geq 3$}. 
\end{cases}
\end{equation}
\end{lemma}

\begin{proof}
Lemma \ref{Lemma:Debaene} and the total multiplicativity of the norm imply that
\begin{align}
&\sum_{N(\PP)\leq x} \log{N(\PP)} \sum_{j\geq 2} \I_{\K} \left(\frac{x}{N(\PP^j)}\right) \label{eq:Quantity}\\
&\qquad= \sum_{N(\PP)\leq x} \log{N(\PP)} \sum_{j\geq 2} \Bigg( \frac{\kappa_{\K} x}{N(\PP^j)} 
+ O^{\star}\bigg( \Lambda_{\K}\bigg(\frac{x}{N(\PP^j)}\bigg)^{1 - \frac{1}{n_{\K}}}  \bigg) \Bigg) \nonumber \\
&\qquad= \kappa_{\K} x  \underbrace{ \sum_{N(\PP)\leq x} \sum_{j\geq 2} \frac{\log{N(\PP)}}{N(\PP)^j}  }_{\text{Term 1}}
+ \Lambda_{\K}  O^{\star} \Bigg(
\underbrace{ x^{1 - \frac{1}{n_{\K}}} \sum_{N(\PP)\leq x}\sum_{j\geq 2} \frac{\log{N(\PP)} }{N(\PP)^{j(1 - \frac{1}{n_{\K}})}}  }_{\text{Term 2}}
 \Bigg) .\label{eq:Terms}
\end{align}

\medskip\noindent\textsc{Term 1.}
First use \eqref{eq:InertiaSum} and obtain
\begin{align}
    \sum_{N(\PP)\leq x} \sum_{j\geq 2} \frac{\log{N(\PP)}}{N(\PP)^j}
    &\leq  \sum_{p\leq x} \sum_{f_i} \sum_{j\geq 2} \frac{\log (p^{f_i})}{(p^{f_i})^j} \nonumber\\
    &\leq  \sum_{p\leq x} \sum_{j\geq 2} \bigg(\sum_{f_i}f_i\bigg) \frac{\log{p}}{p^{j}} \nonumber\\
    &\leq  n_{\K}  \sum_{p\leq x} \sum_{j\geq 2} \frac{\log{p}}{p^{j}}, \label{eq:Pause1}
\end{align}
in which $\sum_{f_i}$ denotes the sum over the inertia degrees $f_i$ of the prime ideals lying over the
rational prime $p$. We have\footnote{\texttt{Mathematica} gives the value
$2 \sum_{p\leq x} \frac{\log{p}}{p^2} \approx 0.986182$.}%
\begin{equation}\label{eq:Barely}
\sum_{p\leq x} \sum_{j\geq 2} \frac{\log{p}}{p^j}
= \sum_{p\leq x} \frac{\log{p}}{p^2}\sum_{j\geq 0}\frac{1}{p^j} 
\leq \sum_{p\leq x} \frac{\log{p}}{p^2}\sum_{j\geq 0}\frac{1}{2^j} 
 =2 \sum_{p\leq x} \frac{\log{p}}{p^2} 
< 1,
\end{equation}
because $(\log x)/x^2$ decreases for $x\geq e^{1/2}\approx 1.65$ and  
\begin{equation*}
2 \sum_{p\leq x} \frac{\log{p}}{p^2} \leq 2\bigg(\sum_{p\leq 1009} \frac{\log p}{p^2} + \int_{1009}^{\infty} \frac{\log t}{t^2}\,dt\bigg) \leq 0.999907 < 1.
\end{equation*}
Thus, \eqref{eq:Pause1} and \eqref{eq:Barely} imply
\begin{equation*}
     \sum_{N(\PP)\leq x} \sum_{j\geq 2} \frac{\log{N(\PP)}}{N(\PP)^j}
    < n_{\K}.
\end{equation*}

\medskip\noindent\textsc{Term 2.}
Apply Lemma \ref{Lemma:Multipart} and \eqref{eq:InertiaSum} and obtain
\begin{align*}
&x^{1 - \frac{1}{n_{\K}}}\sum_{N(\PP)\leq x}  \sum_{j\geq 2} \frac{\log{N(\PP)} }{N(\PP)^{j(1 - \frac{1}{n_{\K}})}}  \\
&\qquad \leq x^{1 - \frac{1}{n_{\K}}}\sum_{p\leq x} \sum_{f_i} \sum_{j\geq 2}
\frac{\log (p^{f_i}) }{(p^{f_i})^{j(1 - \frac{1}{n_{\K}})}} \\
%&\qquad \leq  x^{1 - \frac{1}{n_{\K}}}\sum_{p \leq x}\sum_{j\geq 2} \bigg( \sum_{f_i}f_i \bigg)\frac{\log p}{p^{j(1 - \frac{1}{n_{\K}})}}  \\
&\qquad \leq n_{\K}\sum_{j\geq 2} \bigg( x^{1 - \frac{1}{n_{\K}}}\sum_{p \leq x} \frac{\log p}{p^{j(1 - \frac{1}{n_{\K}})}}\bigg) \\
&\qquad \leq 
\begin{cases}
 \displaystyle n_{\K}\Bigg( 1.6n_{\K}x^{ \frac{1}{2} }\log x +\sum_{j\geq 3}\dfrac{8.8n_{\K} x^{1 - \frac{1}{n_{\K}}} }{2^{j/2}}  \Bigg)
 & \text{if $n_{\K} = 2$}, \\
 n_{\K} \displaystyle\sum_{j\geq 2}\dfrac{8.8n_{\K} x^{1 - \frac{1}{n_{\K}}} }{2^{j/2}}  
 & \text{if $n_{\K} \geq 3$},
\end{cases}
 \\
&\qquad \leq 
\begin{cases}
 \displaystyle 4x^{\frac{1}{2}} \bigg( 1.6\log x +8.8  \sum_{j\geq 3}2^{-j/2}  \bigg)
 & \text{if $n_{\K} = 2$}, \\[5pt]
 \displaystyle 8.8 n_{\K}^2 x^{1 - \frac{1}{n_{\K}}}\sum_{j\geq 2}2^{-j/2}
 & \text{if $n_{\K} \geq 3$},
\end{cases}
 \\
&\qquad \leq 
\begin{cases}
 \displaystyle 6.4\,x^{ \frac{1}{2} }\log x +42.5\,x^{\frac{1}{2}}   
 & \text{if $n_{\K} = 2$}, \\[3pt]
 \displaystyle 15.1 \,n_{\K}^2 x^{1 - \frac{1}{n_{\K}}}
 & \text{if $n_{\K} \geq 3$}.
\end{cases}
\end{align*}
Return to \eqref{eq:Terms} and bound the quantity \eqref{eq:Quantity} using the estimates above.
\end{proof}

\subsection{Proof of \eqref{eq:A1} and \eqref{eq:A2}}\label{Section:Proof:A}

Consider
\begin{equation*}
T_{\K}(x) = \log\bigg(\prod_{N(\mathfrak{a})\leq x}N(\mathfrak{a})\bigg) = \sum_{n\leq x} I_{\K}(n)\log{n},
\end{equation*}
in which $\mathfrak{a}\subset\O_{\K}$ runs over the integral ideals of $\K$.
In the next two lemmas, we approximate $T_{\K}(x)$ in two different ways.  Comparing the resulting expressions will complete the proof of \eqref{eq:A1}. 
The following lemma is an explicit version of Weber's theorem, which states $T_{\K}(x) = \kappa_{\K} x \log x + O(x)$ \cite[p.~128]{DiamondHalberstam}.

\begin{lemma}\label{Lemma:FirstBound}
For $x \geq 2$,
\begin{equation*}
T_{\K}(x) =  \kappa_{\K} x\log x + (2\Lambda_{\K} n_{\K} +\kappa_{\K})O^{\star}(x).
\end{equation*}
\end{lemma}

\begin{proof}
Partial summation and Lemma \ref{Lemma:Debaene} imply 
\begin{align}
T_{\K}(x)
&= \sum_{n\leq x}I_{\K}(n)\log{n} = \sum_{2\leq n\leq x}I_{\K}(n)\log{n}\nonumber\\
&= \I_{\K}(x)\log{x} - \int_{2}^{x}\frac{\I_{\K}(t)}{t}dt\nonumber\\
&= \big(\kappa_{\K} x + O^{\star}(\Lambda_{\K}x^{1-\frac{1}{n_{\K}}})\big) \log{x} 
+O^{\star}\bigg( \int_{2}^{x}\frac{\kappa_{\K} t}{t}\,dt + \int_{2}^{x}\frac{\Lambda_{\K}t^{1-\frac{1}{n_{\K}}}}{t}\,dt \bigg).\label{eqn:comebacktome}
\end{align}
Since $\log{x} < n_{\K} x^{1/n_{\K}}$, it follows that
\begin{align*}
    \big(\kappa_{\K} x + O^{\star}(\Lambda_{\K}x^{1-\frac{1}{n_{\K}}})\big) \log{x}
    = \kappa_{\K} x \log{x} + \Lambda_{\K} n_{\K}O^{\star}(x).
\end{align*}
Next observe that
\begin{equation*}
    \int_2^x t^{-\frac{1}{n_{\K}}}dt
     =\left.\frac{n_{\K}}{n_{\K} - 1}t^{\frac{n_{\K}-1}{n_{\K}}}\right|_{2}^{x} 
     < \frac{n_{\K}}{n_{\K} - 1}x^{\frac{n_{\K}-1}{n_{\K}}} 
     < \frac{n_{\K}}{n_{\K} - 1}x \leq 2x,
\end{equation*}
since $\log x < x$ there and $n_{\K} \geq 2$.
The previous inequality yields
\begin{equation*}
    O^{\star}\bigg(  \int_{2}^{x}\frac{\kappa_{\K} t}{t}\,dt +\int_{2}^{x}\frac{\Lambda_{\K}t^{1-\frac{1}{n_{\K}}}}{t}\,dt \bigg)
    = \left(\kappa_{\K} + 2 \Lambda_{\K} \right)O^{\star}(x).
\end{equation*}
Apply these observations to \eqref{eqn:comebacktome} with $n_{\K} \geq 2$
to complete the proof.
\end{proof}

Now, we derive a second explicit approximation for $T_{\K}(x)$.

\begin{lemma}\label{Lemma:SecondBound}
For $x \geq 2$,
\begin{equation*}
T_{\K}(x) = \kappa_{\K} x \sum_{N(\PP)\leq x} \frac{\log{N(\PP)}}{N(\PP)} + 1.1 \Lambda_{\K}n_{\K}^2 O^{\star} (x) + O^{\star}(\Xi_{\K}(x)),
\end{equation*}
in which $\Xi_{\K}(x)$ is given by \eqref{eq:Xi}.
\end{lemma}

\begin{proof}
We require the ideal analogue of the Legendre--Chebyshev identity\footnote{Diamond--Halberstam \cite[p.~128]{DiamondHalberstam}
inform us that Landau calls this the ``Poincar\'e identity.''} \cite{Legendre},
\begin{equation*}
\prod_{N(\mathfrak{a})\leq x}N(\mathfrak{a}) = \prod_{N(\PP)\leq x}
\prod_{j\geq 1}N(\PP)^{\I_{\K} ( x / N(\PP^j) )}.
\end{equation*}
Lemma \ref{Lemma:Debaene} and Lemma \ref{Lemma:Multipart} with $j  = 1$ imply
\begin{align*}
&T_{\K}(x)
= \log\bigg(\prod_{N(\mathfrak{a})\leq x}N(\mathfrak{a})\bigg) 
= \log\bigg(\prod_{N(\PP)\leq x} \prod_{j\geq 1}N(\PP)^{\I_{\K} ( x / N(\PP^j) )} \bigg) \\
&\,\,=\sum_{N(\PP)\leq x} \log{N(\PP)} \sum_{j\geq 1} \I_{\K} \bigg(\frac{x}{N(\PP^j)}\bigg)\\
&\,\,=\sum_{N(\PP)\leq x} \I_{\K} \bigg(\frac{x}{N(\PP)}\bigg) \log{N(\PP)}  +\sum_{N(\PP)\leq x} \log{N(\PP)} \sum_{j\geq 2} \I_{\K} \left(\frac{x}{N(\PP^j)}\right)  \\
&\,\,= \underbrace{\kappa_{\K} x \sum_{N(\PP)\leq x} \frac{\log{N(\PP)}}{N(\PP)} + 1.1 \Lambda_{\K}n_{\K}^2 O^{\star} (x) }_{\text{Lemma \ref{Lemma:FirstPower}}}+
\underbrace{O^{\star}(\Xi_{\K}(x))}_{\text{Lemma \ref{Lemma:HigherPowers}}} . \qedhere
\end{align*}
\end{proof}

Equating the two expressions for $T_{\K}(x)$ from Lemmas \ref{Lemma:FirstBound} and \ref{Lemma:SecondBound} yields
\begin{align*}
\kappa_{\K} x\log x &+ (2\Lambda_{\K} n_{\K} +\kappa_{\K})O^{\star}(x)\\
&= \kappa_{\K} x \sum_{N(\PP)\leq x} \frac{\log{N(\PP)}}{N(\PP)} + 1.1 \Lambda_{\K}n_{\K}^2 
%{\color{red} + \kappa_{\K} n_{\K} }
 O^{\star} (x) + O^{\star}(\Xi_{\K}(x)).
\end{align*}
Divide by $\kappa_{\K} x$ and simplify to get
\begin{equation*}
\sum_{N(\PP)\leq x} \frac{\log{N(\PP)}}{N(\PP)}
= \log x + \frac{2\Lambda_{\K} n_{\K} +\kappa_{\K}}{\kappa_{\K}}O^{\star}(1)
+ \frac{1.1 \Lambda_{\K}n_{\K}^2 
%+ \kappa_{\K} n_{\K}
}{\kappa_{\K}} O^{\star} (1) + O^{\star}\bigg(\frac{\Xi_{\K}(x)}{\kappa_{\K}x}\bigg).
\end{equation*}
Now observe that
\begin{align*}
\frac{\Xi_{\K}(x)}{\kappa_{\K}x} 
&=
\begin{cases}
n_{\K}  + \dfrac{\Lambda_{\K}}{\kappa_{\K}} O^{\star} \big(6.4 x^{-\frac{1}{2}}\log x + 42.5 x^{ -\frac{1}{2} }   \big)  & \text{if $n_{\K} = 2$}, \\[5pt]
n_{\K}  + 15.1\, \dfrac{\Lambda_{\K}}{\kappa_{\K}} O^{\star}( n_{\K}^2 x^{ - \frac{1}{n_{\K}}}  )  & \text{if $n_{\K} \geq 3$}. 
\end{cases}
\\
&\leq n_{\K}+
\begin{cases}
\dfrac{\Lambda_{\K}}{\kappa_{\K}} O^{\star} \big(6.4 x^{-\frac{1}{2}}\log x + 42.5 x^{ -\frac{1}{2} }   \big)  & \text{if $n_{\K} = 2$}, \\[5pt]
15.1\, \dfrac{\Lambda_{\K} n_{\K}^2 }{\kappa_{\K}} O^{\star}( x^{ - \frac{1}{n_{\K}}}  )  & \text{if $n_{\K} \geq 3$}. 
\end{cases}
\\
&\leq n_{\K}+
\begin{cases}
33.2\dfrac{\Lambda_{\K}}{\kappa_{\K}}  & \text{if $n_{\K} = 2$}, \\[5pt]
15.1\, \dfrac{\Lambda_{\K}n_{\K}^2}{\kappa_{\K}}   & \text{if $n_{\K} \geq 3$}. 
\end{cases}
\\
&\leq n_{\K} + 15.1 \,  \dfrac{\Lambda_{\K}n_{\K}^2}{\kappa_{\K}} .
\end{align*}
Putting this all together, we get
\begin{align*}
\left|\sum_{N(\PP)\leq x} \frac{\log{N(\PP)}}{N(\PP)} - \log x\right|
&\leq 
 \frac{2\Lambda_{\K} n_{\K} +\kappa_{\K}}{\kappa_{\K}}
+ \frac{1.1 \Lambda_{\K}n_{\K}^2 }{\kappa_{\K}} 
+
n_{\K} + 15.1 \dfrac{\Lambda_{\K}n_{\K}^2}{\kappa_{\K}}\\
&\leq  n_{\K}  + 18.2 \dfrac{\Lambda_{\K}n_{\K}^2}{\kappa_{\K}} + 1.
\end{align*}
This yields the desired bound \eqref{eq:A2}.\qed

%%%%%%%%%%%%%%%%%%%%%%%%%%%%%%%%%%%%%%%%%%%
%%%%%%%%%%%%%%%%%%%%%%%%%%%%%%%%%%%%%%%%%%%
\subsection{Proofs of \eqref{eq:B1}, \eqref{eq:B2}, and \eqref{eq:M1}}\label{Section:Proof:B}
For $x \geq 2$, partial summation yields
\begin{align*}
    \sum_{N(\PP)\leq x}&\frac{1}{N(\PP)}
    = \sum_{N(\PP)\leq x}\frac{\log{N(\PP)}}{N(\PP)}\frac{1}{\log{N(\PP)}}\\
    &= \frac{1}{\log{x}} \sum_{N(\PP)\leq x}\frac{\log{N(\PP)}}{N(\PP)} + \int_2^x \bigg(\sum_{N(\PP)\leq t}\frac{\log{N(\PP)}}{N(\PP)}\bigg)\frac{dt}{t(\log t)^2}\\
    &= \frac{1}{\log{x}} \big(\log{x} + A_{\K}(x)\big) + \int_2^x \big(\log{t} + A_{\K}(t)\big)\frac{dt}{t(\log t)^2}\\
    &= 1 + \frac{A_{\K}(x)}{\log{x}} + \int_2^x \frac{dt}{t\log t} + \int_2^x \frac{A_{\K}(t)}{t(\log t)^2}dt\\
    &= \log\log{x} - \log\log{2}  + 1 + \frac{A_{\K}(x)}{\log{x}} + \int_2^x \frac{A_{\K}(t)}{t(\log t)^2}dt\\
    &= \log\log{x} + \underbrace{1- \log\log{2}  + \int_2^\infty \frac{A_{\K}(t)}{t(\log t)^2}dt}_{M_{\K}} 
    +  \underbrace{ \frac{A_{\K}(x)}{\log{x}} - \int_x^\infty \frac{A_{\K}(t)}{t(\log t)^2}dt }_{B_{\K}(x)},
\end{align*}
in which \eqref{eq:A2} ensures that the integral that defines $M_{\K}$ converges and
\begin{align*}
    |B_{\K}(x)|
    &\leq \frac{|A_{\K}(x)|}{\log{x}} + \int_x^\infty \frac{|A_{\K}(t)|}{t(\log t)^2}dt\\
    &\leq  \bigg(  n_{\K}  + 18.2 \dfrac{\Lambda_{\K}n_{\K}^2}{\kappa_{\K}} + 1  \bigg)
     \bigg( \frac{1}{\log x} + \int_x^\infty \frac{dt}{t(\log t)^2}\bigg)\\
    &= 2 \bigg(  n_{\K}  + 18.2 \dfrac{\Lambda_{\K}n_{\K}^2}{\kappa_{\K}} + 1  \bigg) \frac{1}{\log x}
\end{align*}

This proves \eqref{eq:B1} and \eqref{eq:B2}.
The proof of \eqref{eq:M1} is identical to that in \cite{EMT4NF1}, which followed
Ingham's approach \cite{Ingham}. Hence, we present an abbreviated version.

For $\Re s > 1$, partial summation and \eqref{eq:B1} provide
\begin{align}
\sum_{\PP}&\frac{1}{N(\PP)^s}
= \lim_{x\to\infty} \bigg( \sum_{N(\PP)\leq x}\frac{1}{N(\PP)^s} \bigg)
= \lim_{x\to\infty} \bigg( \sum_{N(\PP)\leq x}\frac{1}{N(\PP)^{s-1} N(\PP)} \bigg) \nonumber\\
&= \lim_{x\to\infty} \bigg(\frac{1}{x^{s - 1}}\sum_{N(\PP)\leq x}\frac{1}{N(\PP)}\bigg) + (s-1) \int_{2}^{\infty}\bigg(\sum_{N(\PP)\leq t}\frac{1}{N(\PP)}\bigg) \frac{dt}{t^{s}}\nonumber\\
&=(s-1) \int_{2}^{\infty}\bigg(\sum_{N(\PP)\leq t}\frac{1}{N(\PP)}\bigg) \frac{dt}{t^{s}}\nonumber\\
&=   (s-1)\int_{2}^\infty\frac{M_{\K}}{t^{s}}\,dt
+ (s-1)\int_{2}^\infty\frac{B_{\K}(t)}{t^{s}} \,dt 
+ (s-1)\int_{2}^\infty\frac{\log\log t}{t^{s}}\,dt.\label{eqn:refmel8er}
\end{align}
Let $s \to 1^+$, then the first integral tends to $M_{\K}$, the second tends to zero and the substitution $t^{s-1} = e^y$ converts the third integral into
\begin{equation*}
(s-1)\int_{2}^{\infty}\frac{\log\log{t}}{t^{s}}\,dt
= \int_{\log(2^{s-1})}^{\infty} e^{-y} \log y \, dy - 2^{1-s} \log(s-1),
\end{equation*}
which tends to $- \gamma- \log(s-1)$.  As $s \to 1^+$, the Euler product formula yields
\begin{align}
M_{\K} &-\gamma \nonumber\\
&=\log(s-1) + \sum_{\PP}\left[ \frac{1}{N(\PP)^s} + \log\left(1 - \frac{1}{N(\PP)^s}\right) \right] - \sum_{\PP}  \log\left(1 - \frac{1}{N(\PP)^s}\right) \nonumber \\
&= \log\big((s-1)\zeta_{\K}(s)\big) + \sum_{\PP}\left[ \frac{1}{N(\PP)^s} + \log\left(1 - \frac{1}{N(\PP)^s}\right) \right] \nonumber \\
&= \log \kappa_{\K} + \sum_{\PP}\left[\frac{1}{N(\PP)} + \log\left(1 - \frac{1}{N(\PP)}\right)\right], \label{eq:MKGo1}
\end{align}
which converges by comparison with $\sum_{\PP} N(\PP)^{-2}$.   \qed

%%%%%%%%%%%%%%%%%%%%%%%%%%%%%%%%%%%%%%%%%%%
\subsection{Proofs of \eqref{eq:C1} and \eqref{eq:C2}}
The proof is identical to that in \cite{EMT4NF1} so we merely sketch it here for the sake of completeness.
From \eqref{eq:MKGo1} obtain
\begin{equation}\label{eq:refmelastX}
    -\gamma - \log{\kappa_{\K}} + M_{\K} 
    = \sum_{N(\PP) \leq x}\left[\frac{1}{N(\PP)} + \log\left(1 - \frac{1}{N(\PP)}\right)\right] + F_{\K}(x).
\end{equation}
Since $0 \leq -y -\log(1-y) \leq  y^2(1-y)^{-1}$, it follows that
\begin{align}
| F_{\K}(x)|
&=- \sum_{N(\PP) > x}\left[\frac{1}{N(\PP)} + \log\left(1 - \frac{1}{N(\PP)}\right)\right] 
\leq \sum_{N(\PP) > x}\frac{1}{N(\PP) (N(\PP) - 1)} \nonumber \\
&\leq \sum_{p > x}\sum_{f_i}\frac{1}{p^{f_i} (p^{f_i} - 1)} 
< n_{\K}\sum_{m > x}\frac{1}{m (m - 1)}  
\leq \frac{n_{\K}}{ x-1 }. \label{eq:FKx}
\end{align}
Then \eqref{eq:B1} and \eqref{eq:refmelastX} imply
\begin{equation*}
-\gamma - \log{\kappa_{\K}} + M_{\K} =  \sum_{N(\PP) \leq x} \log\left(1 - \frac{1}{N(\PP)}\right) + \log\log{x} + M_{\K} +
\underbrace{B_{\K}(x) + F_{\K}(x)}_{E_{\K}(x)}.
\end{equation*}
Exponentiate and simplify to get
\begin{equation*}
    \prod_{N(\PP) \leq x} \left(1 - \frac{1}{N(\PP)}\right) 
    = \frac{e^{- \gamma}}{\kappa_{\K} \log{x}} e^{-E_{\K}(x)}
    = \frac{e^{- \gamma}}{\kappa_{\K} \log{x}}\big(1 + C_{\K}(x) \big),
\end{equation*}
where $|C_{\K}(x)| \leq  |E_{\K}(x)| e^{ |E_{\K}(x)|}$ since
$|e^t - 1|  \leq |t| e^{|t|}$ for $t \in \R$. \qed

%%%%%%%%%%%%%%%%%%%%%%%%%%%%%%%%%%%%%%%%%%%
\subsection{Proof of \eqref{eq:M2}}
From \eqref{eq:refmelastX}, 
$M_{\K} = \gamma + \log\kappa_{\K} + F_{\K}(2-\delta)$ for $\delta \in (0,1)$.  Then
\eqref{eq:FKx} yields
$$- n_{\K} \leq \liminf_{\delta\to0^+}F_{\K}(2-\delta) \leq \limsup_{\delta\to0^+}F_{\K}(2-\delta)\leq 0,$$
which proves \eqref{eq:M2}.\qed



%%%%%%%%%%%%%%%%%%%%%%%%%%%%%%%%%%%%%%%%%%%%%%%%%%%%%%%%%%%%%%%%%%%%%%%%%%%%%%%%
\section{Explicit lower bounds for the Dedekind-zeta residue}\label{sec:ResidueBounds}
For a number field $\K$, recall that $\kappa_{\K}$ denotes the residue of the Dedekind zeta function
$\zeta_{\K}(s)$ at $s=1$.  If $\K = \Q$, then $\kappa_{\K} = 1$.  Consequently, we assume 
that $n_{\K} \geq 2$. Although $\kappa_{\K}$ can be computed directly from the analytic class 
number formula \eqref{eqn:residue_class_ana_form}, it is worth investigating bounds 
that are given only in terms of the absolute value of the discriminant $\Delta_{\K}$
and the degree $n_{\K}$ of $\K$.

Since $n_{\K} = r_1 + 2r_2\geq 2$, it follows that $0.46r_1 + 0.1r_2 \geq 0.1$ and hence Zimmert's estimate \cite{Zimmert} (see also \cite[Thm.~7, p.~273]{Lang}) provides
\begin{equation*}
    \frac{R_{\K}}{w_{\K}} \geq 0.02 e^{0.46r_1 + 0.1 r_2} \geq 0.02 e^{0.1} > 0.0221.
\end{equation*}
Moreover, $2^{r_1}(2\pi)^{r_2} \geq 2^{2}(2\pi)^{0} = 4$ and hence we obtain the lower bound
\begin{equation}\label{eqn:kapp_bound_Zimmert}
    \kappa_{\K} \geq \frac{2^{r_1}(2\pi)^{r_2}R_{\K}}{w_{\K}\sqrt{\Delta_{\K}}} > \frac{4\times 0.0221}{\sqrt{\Delta_{\K}}} = \frac{0.0884}{\sqrt{\Delta_{\K}}}.
\end{equation}
We thank Eduardo Friedman for suggesting this argument to us.

Another approach is based on Stark's estimate
\begin{equation}\label{eqn:kappd_bound_Stark}
    \kappa_{\K} > \frac{c}{n_{\K} g(n_{\K}){\Delta_{\K}}^{1/n_{\K}}},
\end{equation}
in which
\begin{equation*}
    g(n_{\K}) = \begin{cases}
    1 &\text{if }\K\text{ has a normal tower over }\Q,\\
    n_{\K} ! &\text{otherwise,}
    \end{cases}
\end{equation*}
and $c$ is an effectively computable constant \cite{StarkBS}. The denominator in \eqref{eqn:kappd_bound_Stark} can be replaced by $g(n_{\K})\log{\Delta_{\K}}$ if $\K$ has no quadratic subfield. We establish here that $c = 0.001448029$ is unconditionally admissible in \eqref{eqn:kappd_bound_Stark}, with improvements available in some cases. If $n_{\K} > 2$, then \eqref{eqn:kappd_bound_Stark} is generally preferred over \eqref{eqn:kapp_bound_Zimmert}.

In what follows, we adhere to Stark's notation so that the reader may, if they wish,
confirm our calculations \cite{StarkBS}.
Stark proves the existence of effectively computable constants $c_4$ and $c_8$ such that
\begin{equation}
    \kappa_{\K} > {c_4}^{-1}\min\left\{\frac{1}{\alpha(n_{\K})\log{\Delta_{\K}}}, \frac{1}{c_8{\Delta_{\K}}^{1/n_{\K}}}\right\},
\end{equation}
in which
\begin{equation*}
    \alpha(n_{\K})=
    \begin{cases}
    4&\text{if }\K\text{ is normal over }\Q,\\
    16&\text{if }\K\text{ has a normal tower over }\Q,\\
    4n_{\K}!&\text{otherwise.}
    \end{cases}
\end{equation*}
Moreover, $\kappa_{\K} > 1/(c_4\alpha(n_{\K})\log{\Delta_{\K}})$ if $\K$ does not have a quadratic subfield. Therefore, in the case $\K$ does not have a quadratic subfield, 
\begin{align*}
    \kappa_{\K}
    &> \frac{1}{c_4c_8 {\Delta_{\K}}^{1/n_{\K}}} \min\left\{\frac{c_8{\Delta_{\K}}^{1/n_{\K}}}{\alpha(n_{\K})\log{\Delta_{\K}}}, 1\right\}\\
    &\geq \frac{1}{c_4c_8 {\Delta_{\K}}^{1/n_{\K}}} \min\left\{\frac{e c_8}{n_{\K}\alpha(n_{\K})}, 1\right\},
\end{align*}
since $e^x \geq x e$ for $x>0$ implies that 
$${\Delta_{\K}}^{1/n_{\K}} = \exp((1/n_{\K})\log{\Delta_{\K}}) \geq (e/n_{\K})\log{\Delta_{\K}}.$$ 
We find admissible constants $c_4$ and $c_8$ by tracking the 
choices made in \cite{StarkBS}. First,
\begin{equation*}
    c_4 = 2c_3 = 2e^{\frac{21}{8} + \frac{c_1}{2} - \frac{c_2}{8}\frac{\Gamma'}{\Gamma}\left(\frac{1}{2}\right)},
\end{equation*}
in which $c_1 = 0$ and $c_2 = 2/\log{3}$ are admissible options \cite[Lemma 4]{StarkBS}. Hence, $c_4 \approx 43.162115 < 43.2$ is admissible. 
Stark claims somewhat mysteriously that ``it is likely from a remark in Bateman and Grosswald \cite[p.~188]{batemanGrosswald} that $c_8 = \pi/6$ will suffice"; see Remark \ref{Remark:Pintz}.
If one proceeds with $c_8 = \pi/6$, then 
\begin{equation*}
    \kappa_{\K}
    > \frac{1}{c_4c_8 {\Delta_{\K}}^{1/n_{\K}}} \min\left\{\frac{ec_8}{n_{\K}\alpha(n_{\K})}, 1\right\}
    = \frac{e/c_4}{n_{\K}\alpha(n_{\K}){\Delta_{\K}}^{1/n_{\K}}}
    > \frac{0.06297842}{n_{\K}\alpha(n_{\K}){\Delta_{\K}}^{1/n_{\K}}},
\end{equation*}
since $(ec_8)/(n_{\K}\alpha(n_{\K})) \leq (ec_8)/8 < 0.178 < 1$. It follows that
\begin{equation*}
    \kappa_{\K} >
    \begin{cases}
    \frac{0.015744605}{n_{\K}g(n_{\K}){\Delta_{\K}}^{1/n_{\K}}}&\text{if $\K$ is normal over $\Q$},\\[5pt]
    \frac{0.003936151}{n_{\K}g(n_{\K}){\Delta_{\K}}^{1/n_{\K}}}&\text{if $\K$ has a normal tower over $\Q$},\\[5pt]
    \frac{0.015744605}{n_{\K}g(n_{\K}){\Delta_{\K}}^{1/n_{\K}}}&\text{otherwise}.
    \end{cases}
\end{equation*}
Moreover, if $\K$ does not have a quadratic subfield, then
\begin{equation*}
    \kappa_{\K}
    > \frac{1/c_4}{\alpha(n_{\K})\log{\Delta_{\K}}}
    > \begin{cases}
    \frac{0.005792116}{g(n_{\K})\log{\Delta_{\K}}}&\text{if $\K$ is normal over $\Q$},\\[5pt]
    \frac{0.001448029}{g(n_{\K})\log{\Delta_{\K}}}&\text{if $\K$ has a normal tower over $\Q$},\\[5pt]
    \frac{0.005792116}{g(n_{\K})\log{\Delta_{\K}}}&\text{otherwise}.
    \end{cases}
\end{equation*}
In summary, $c = 0.001448029$ is admissible in \eqref{eqn:kappd_bound_Stark} unconditionally; 
$c = 0.005792116$ is admissible if $\K$ is normal over $\Q$ or $\K$ does not have a normal tower over $\Q$. Moreover, if $\K$ does have a quadratic subfield, then $c = 0.003936151$ is admissible in \eqref{eqn:kappd_bound_Stark} unconditionally and $c = 0.015744605$ is admissible if $\K$ is normal over $\Q$ or $\K$ does not have a normal tower over $\Q$.


\begin{remark}\label{Remark:Pintz}
Stark suggests that $c_8 = \pi/6$ is admissible in \cite[Lem.~11]{StarkBS}.
Pintz \cite[Thm.~3]{Pintz76} proved that $c_8 = \pi/12+  o(1)$ works.  
This suggests that Stark's $c_8 = \pi / 6$ is indeed acceptable. 
Schinzel, a referee of Pintz's paper, improved the value of $c_8$ in a footnote to Pintz' paper \cite[p.~277]{Pintz76}.
For each $\varepsilon > 0$ and $\Delta_{\K}$
sufficiently large, he proved that $c_8 = (16/\pi - \varepsilon)^{-1}$ is admissible \cite[p.~277]{Pintz76}.
\end{remark}


%%%%%%%%%%%%%%%%%%%%%%%%%%%%%%%%%%%%%%%%%%%%%%%%%%%%%%%%%%%%%%%%%%%%%%%%%%%%%%%%
\section{Remarks and Open Problems}\label{Section:Future}

\begin{remark}\label{Remark:Theta}
The estimate $\theta(x) < 1.01624x$ from \eqref{eq:Rosser} has been substantially improved over the years.
The current record appears to be due to Broadbent, Kadiri, Lumley, Ng, and Wilk \cite{Broadbent}, who proved that 
\begin{equation*}
\theta(x) < (1+1.93378 \times 10^{-8})x.
\end{equation*}
This bound results in a slight improvement to the constants in
Theorem \ref{Theorem:Main}.
\end{remark}


\begin{question}
Tenenbaum \cite{Tenenbaum} recently proved a generalization of Mertens' second theorem.
Following similar arguments \textit{mutatis mutandis} it may be possible to write
\begin{equation*}
    \mathcal{S}_{\K}(k,x) = \sum_{N(\PP_1\PP_2\cdots\PP_k)}\frac{1}{N(\PP_1\PP_2\cdots\PP_k)} = P_{\K}(k,\log\log{x}) + O\left(\frac{(\log\log{x})^{k-1}}{\log{x}}\right),
\end{equation*}
for $x\geq 3$, in which $P_{\K}(k,X) = \sum_{0\leq j\leq k} \lambda_{j,k} X^j$ and $\lambda_{j,k}$ are defined as in \cite[Thm.~1]{Tenenbaum}.
Can one make the implied constant explicit in terms of $n_{\K}$ and $\Delta_{\K}$?
\end{question}


\begin{question}
In the case $\K = \Q$, Mertens' third theorem asserts
\begin{equation*}
    \prod_{p\leq x} \left(1 - \frac{1}{p}\right)^{-1} \,\sim\,\,\,e^\gamma \log{x} .
\end{equation*}
Rosser--Schoenfeld \cite{Rosser} observed that the product is less than $e^\gamma \log{x}$ for $x\leq 108$ and
they wondered if the two expressions took turns exceeding the other.
Diamond--Pintz proved that this is the case \cite{DiamondPintz}. In fact, the difference 
is at least as large as $O(\log\log\log{x}/\sqrt{x})$ infinitely often.
Does a similar phenomenon occur for $\K \neq \Q$?
\end{question}



\bibliographystyle{amsplain}
\bibliography{EMT4NF2}

\end{document}