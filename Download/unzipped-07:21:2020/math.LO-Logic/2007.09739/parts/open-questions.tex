
The computability-theoretic study of Milliken's tree theorem and its applications being completely new, this work leaves many questions open. We collect some here that seem most promising for follow-up research directions.


\section{Milliken's tree theorem in the arithmetical hierarchy}


When analyzing a mathematical problem from a computability-theoretic viewpoint, the first step usually consists in determining whether the computable instances of the problem admit arithmetical solutions, and if so, trying to identify the exact level in the arithmetical hierarchy where they stand. For example, Jockusch~\cite{Jockusch1972Ramseys} proved that every computable instance of Ramsey's theorem for $n$-tuples admits $\Pi^0_n$ solutions, and for each $n \geq 2$, constructed a computable instance of Ramsey's theorem for $n$-tuples and 2 colors with no $\Sigma^0_n$ solutions. Thus, the status of Ramsey's theorem with respect to the arithmetical hierarchy is fully determined. 

The case of Milliken's tree theorem is less clear. By \Cref{thm:milliken-arithmetic}, computable instances of Milliken's tree theorem admit arithmetical solutions. More precisely, every computable instance of Milliken's tree theorem for subtrees of height $n$ admits a $\Delta^0_{2n-1}$ solution. On the other hand, since Milliken's tree theorem generalizes Ramsey's theorem, for every $n \geq 2$, there exists a computable instance of Milliken's tree theorem for trees of height $n$ with no $\Sigma^0_n$ solutions. This leaves a gap between the lower and upper bound.


\begin{question}\label{quest:mttn-deltanp1}
Does every computable instance of Milliken's tree theorem for height $n$
admit a $\Delta^0_{n+1}$ solution?
\end{question}

The proof by Jockusch~\cite{Jockusch1972Ramseys} of the existence of a $\Pi^0_2$ solution for every computable instance of Ramsey's theorem for $n$-tuples is by an inductive argument based on the notion of prehomogeneous set. In particular, he proves that every PA degree relative to $\emptyset'$ is sufficient to compute a prehomogeneous set. Hirschfeldt and Jockusch~\cite[Theorem 2.1]{Hirschfeldt2016notions} actually proved a reversal, by constructing a computable instance of Ramsey's theorem for triples such that every prehomogeneous set is of PA degree relative to $\emptyset'$. This bound on prehomogeneous sets is sufficient to make increasing the level in the arithmetical hierarchy only by one when increasing the size of the colored tuples by one, by taking prehomogeneous sets of low degree over $\emptyset'$.

Similarly, the current upper bound of Milliken's tree theorem is proved using the corresponding notion of prehomogeneous tree, but \Cref{thm:milliken-prehomogeneous} yields only a $\Delta^0_3$ solution, which makes increase the level in the arithmetical hierarchy by 2 instead of 1 when coloring larger tuples. The following questions are still open:

\begin{question}
Given a computable instance of the Milliken's tree theorem for height $n$, does any PA degree relative to $\emptyset'$ compute a prehomogeneous infinite strong subtree? Is there always a prehomogeneous infinite strong subtree of low degree relative to $\emptyset'$?
\end{question}

A positive answer to either question would be sufficient to answer positively \Cref{quest:mttn-deltanp1}.


\section{Larger degrees and cone avoidance}

Cone avoidance is a central notion in the computability-theoretic analysis of theorems. It is the main tool for separating a theorem from $\ACA_0$ over $\omega$-structures. It is in particular a desirable property to have, and given a statement which does not admit cone avoidance, one can ask whether there exists a natural weakening of it which admits it. The analysis of Ramsey's theorem gives a good example:  Jockusch~\cite{Jockusch1972Ramseys} constructed for every $n \geq 3$ a computable instance of Ramsey's theorem for $n$-tuples whose solutions compute the halting set. In particular, this shows that Ramsey's theorem for 3-tuples does not admit cone avoidance. On the other hand, Wang~\cite[Theorem 3.2]{Wang2014Some} proved that when weakening the notion of homogeneity in Ramsey's theorem by allowing a larger number of colors, then the resulting statement admits cone avoidance. In particular, he proved that $(\forall k)\RT{3}{k, 2}$ admits cone avoidance, where $(\forall k)\RT{n}{k, \ell}$ is the statement whose instances are colorings $f: [\omega]^n \to k$ for some $k$, and whose solutions are infinite sets $H$ such that $|f[H]| \leq \ell$. In general, Wang proved that for every $n$, and every $\ell$ sufficiently large with respect to $n$, the statement  $(\forall k)\RT{n}{k, \ell}$ admits cone avoidance. Cholak and Patey~\cite{Cholak2019Thin} computed the exact bound where this threshold phenomenon happens, which happens to be $(\forall k)\RT{n}{k, C_{n-1}}$, where $C_0, C_1, \dots$ is the Catalan sequence, starting with $1, 1, 2, 5, 14, 42, \dots$

Milliken's tree theorem behaves like Ramsey's theorem with many respects. Milliken's tree theorem for pairs admits cone avoidance, while there exists a computable instance of Milliken's tree theorem for trees of height 3 whose solutions compute the halting set. By a similar investigation, we proved in \Cref{subsect:thin-milliken} that $(\forall k)\PMT{3}{k,2}$ admits cone avoidance (\Cref{thm:pmtt3k2-cone-avoidance}), where $(\forall k)\PMT{n}{k,\ell}$ is the weakening of $(\forall k)\PMT{n}{}$ where $\ell$ colors are allowed in the solutions.

The proof of \Cref{thm:pmtt3k2-cone-avoidance} goes through the existence of a level-homogeneous strong subtree. Recall that a tree $T$ is level-homogeneous with respect to a coloring $f: \Subtree{n}{T}$ if strong subtrees with the same level function get assigned the same color. This notion reduces the problem of finding an infinite strong subtree monochromatic for $f$ to the problem of finding an infinite homogeneous set. Indeed, if $T$ is level-homogeneous with respect to $f$, the color depends only on the levels, hence becomes a coloring of $[\omega]^n$. The known counter-examples to cone avoidance of Milliken's tree theorem for trees of height at least 3 as all inherited from Ramsey's theorem by defining a coloring which depends only on the levels. We proved that the statement which to a finite coloring of $\Subtree{3}{T}$, associates an infinite level-homogeneous strong subtree, admits cone avoidance. This result goes towards the intuition that the strength of Milliken's tree theorem is mainly inherited from Ramsey's theorem. It is therefore natural to wonder whether the statement of the existence of a level-homogeneous infinite strong subtree admits cone avoidance, when considering colorings of finite subtrees of larger height. Since the proof from height $n$ to height $n+1$ is usually inductive, by first proving cone avoidance for height $n$, then strong cone avoidance for height $n$, and then only cone avoidance for height $n+1$, we wonder whether the statement of the existence of a level-homogeneous infinite strong subtree admits strong cone avoidance.

\begin{question}
Given two sets $C$ and $Z$ such that $C \not \leq_T Z$, and a finite sequence of $Z$-computable, $Z$-computably bounded, infinite trees with no leaves $T_0, \dots, T_{d-1}$, does every coloring $f: \Subtree{n}{T_0, \dots, T_{d-1}} \to k$ admit a level-homogeneous tuple $(S_0, \dots, S_{d-1}) \in \Subtree{\omega}{T_0, \dots, T_{d-1}}$ such that $C \not \leq_T Z \oplus S_0 \oplus \dots \oplus S_{d-1}$?
\end{question}

A positive answer to this question would enable to make it benefit from the computability-theoretic analysis for Ramsey's theorem, and in particular would imply that $(\forall k)\PMT{n}{k,C_{n-1}}$ admits cone avoidance.

\section{Comparing the statements for pairs in reverse mathematics}

Ramsey's theorem for pairs admits a special status with respect to full Ramsey's theorem in reverse mathematics, as it admits cone avoidance, while Ramsey's theorem for larger tuples is equivalent to $\ACA_0$ over~$\RCA_0$. This threshold phenomenon was also satisfied by the Chubb-Hirst-McNicholl tree theorem (see Dzhafarov and Patey~\cite{Dzhafarov2017Coloring}) whose statement for pairs admits cone avoidance, while is equivalent to $\ACA_0$ for larger tuples, or the Erd\"os-Rado theorem (see Chong, Liu, Liu and Yang~\cite{Chong2019Strengthc}).
We therefore naturally had a particular focus on the restriction of Milliken's tree theorem for trees of height 2, and on the applications of Milliken's tree theorem restricted to pairs.

As we can see in the proof of Devlin's theorem and the Rado graph theorem using Milliken's tree theorem, both Devlin's theorem for $n$-tuples and the Rado graph theorem for graphs of size $n$ involve applications of Milliken's tree theorem for strong subtrees of height $2n-1$. This is essentially due to \Cref{lem:coded-joyce-order-to-strong-subtree}. Informally, when representing rational numbers as strings, any coloring of a pairs of rationals induces a coloring of strong subtrees of height 3, by considering the tree whose first level is the length of their meet, the second level is the length of the shortest of the two strings representing the rationals, and the third level is the longest length. 

This yields two main questions, namely, (1) whether there exists another proof of these statements of size $n$ involving only applications of Milliken's tree theorem for trees of height $n$, and (2) whether these statements should be more considered as statement about pairs or about triples. The latter question is more informal, and depends on the aspects considered. 

One aspect separating Ramsey's theorem for pairs from larger tuples is the existence of cone avoiding solutions. With this respect, Devlin's theorem admits a computable instance whose solutions all compute the halting set, while the Rado graph theorem for graphs of size 2, the Erd\"os-Rado theorem and Milliken's tree theorem for pairs are all cone avoiding. This proves in particular that Milliken's tree theorem for pairs does not imply Devlin's theorem for pairs in $\RCA_0$, and answers the first question negatively for Devlin's theorem. Another aspect which could better capture the difference between statements about pairs and about larger tuples, is the position in the arithmetical hierarchy. As explained, Jockusch~\cite{Jockusch1972Ramseys} proved the existence of a computable instance of Ramsey's theorem for triples with no $\Sigma^0_3$ solution, while every computable instance of Ramsey's theorem for pairs admits a $\Pi^0_2$ solution. Here again, using this criterium, Devlin's theorem for pairs does not seem to be a statement about pairs. Indeed, by \Cref{cor:dt2-no-sigma3} in a computable instance of Devlin's theorem for pairs with no $\Sigma^0_3$ solution. The question for the Rado graph theorem for graphs of height 2 and for the Erdos-Rado theorem remains open:

\begin{question}
Is there a computable instance of the Rado graph theorem for graphs of size 2 with no $\Sigma^0_3$ solution? Same question for the Erd\"os-Rado theorem for pairs.
\end{question}

If the answer is yes, then this would answer negatively the corresponding part of the following question.

\begin{question}
  Does $\MT{2}{}$ imply $(\forall k)\RG^2_{k, 4}$ over $\RCA_0$? Same question for $\mathrm{ER}^2$.
\end{question}

Milliken's tree theorem for trees of height 2 is a natural generalization of Ramsey's theorem for pairs, and so is the Erd\"os-Rado theorem. By \Cref{thm:dt2-implies-rt2}, this is also the case of Devlin's theorem for pairs. It is however unknown whether the Rado graph theorem for graphs of height 2 also implies Ramsey's theorem for pairs. On the positive side, the Rado graph theorem for pairs implies a stable version of Ramsey's theorem for pairs (see \Cref{thm:rg284-implies-srt22}). Thus, by the decomposition of Ramsey's theorem for pairs in its stable version and the cohesiveness principle (see Cholak, Jockusch and Slaman~\cite{Cholak2001strength}, Section 7), the question can be rephrased as whether the Rado graph theorem for pairs implies the $\COH$ over~$\RCA_0$.

\begin{question}
Does $(\forall k)\RG^2_{k, 4}$ imply $\RT 22$ over $\RCA_0$?
Equivalently, does $(\forall k)\RG^2_{k, 4}$ imply $\COH$ over $\RCA_0$?
\end{question}


Devlin's theorem for pairs and the Erd\"os-Rado theorem are both statements about colorings of pairs of rationals. The former is symmetric, in that the nature of the solution does not depend on value of the color, and is thus arguably more natural than the Erd\"os-Rado theorem.
Since the statement $\DT{2}{<\infty,2}$ is somehow combinatorially optimal with respect to coloring of pairs of  dense linear orders with no endpoints, one could expect that it implies the Erd\"os-Rado theorem. This is actually the case by \Cref{th:dt242-implies-er2}: $\DT{2}{4,2}$ implies $\mathrm{ER}^2$ over $\RCA_0$. On the other hand, $\mathrm{ER}^2$ admits cone avoidance, while $\DT{2}{4,2}$ does not. This yields the following question: is there a natural statement which implies $\mathrm{ER}^2$ and does admit cone avoidance? The notion of naturality is kept informal. When increasing the number of colors allowed in the solutions of Devlin's theorem for pairs, the statement $\DT{2}{<\infty,4}$ is the first one admitting cone avoidance (\Cref{cor:jdt4-cone-avoidance}). This yields the following question:

\begin{question}
Does $\DT{2}{<\infty,4}$ imply $\mathrm{ER}^2$ over $\RCA_0$?
\end{question}

We reproved in \Cref{subsect:er-theorem} cone avoidance of $\mathrm{ER}^2$ (which was first proved by Chong, Liu, Liu and Yang~\cite{Chong2019Strengthc}) using $\DT{2}{<\infty,4}$. This is however not an answer to the question since it involved the existence of generic sets for a particular notion of forcing, which may not belong to the model of $\RCA_0$.
