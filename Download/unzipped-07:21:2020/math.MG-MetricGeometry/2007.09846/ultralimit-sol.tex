%%%%%%%%%%%%%%%%%%%%%%%%%%%%%%

\parbf{\ref{ex:ultrapower}.}
Part \ref{SHORT.ex:ultrapower:a} follows directly from the definitions.
Further we consider $\spc{X}$ as a subset of $\spc{X}^\omega$.

\parit{\ref{SHORT.ex:ultrapower:compact}.}
Suppose $\spc{X}$ compact.
Given a sequence $(x_n)$ in $\spc{X}$, denote its $\omega$-limit in $\spc{X}^\omega$ by $x^\omega$ and its $\omega$-limit in $\spc{X}$ by $x_\omega$.

Observe that $x^\omega=\iota(x_\omega)$.
Therefore $\iota$ is onto.

If $\spc{X}$ is not compact, we can choose a sequence $(x_n)$ such that $\dist{x_m}{x_n}{}>\eps$ for fixed $\eps>0$ and $m\ne n$.
Observe that
\[\lim_{n\to\omega}\dist{x_n}{y}{\spc{X}}\ge \tfrac\eps2\]
for any $y\in\spc{X}$.
It follows that $x_\omega$ lies on the distance at least $\tfrac\eps2$ from $\spc{X}$.

\parit{\ref{SHORT.ex:ultrapower:proper}.}
A sequence of points $(x_n)$ in $\spc{X}$ will be called $\omega$-bounded if there is a real constant $C$ such that
\[\dist{p}{x_n}{\spc{X}}\le C\] 
for $\omega$-almost all $n$.

The same argument as in \ref{SHORT.ex:ultrapower:compact} shows that any $\omega$-bounded sequence has its $\omega$-limit in $\spc{X}$.
Further if $(x_n)$ is not  $\omega$-bounded, then 
\[\lim_{n\to\omega}\dist{p}{x_n}{\spc{X}}=\infty;\]
that is $x_\omega$ does not lie in the metric component of $p$ in $\spc{X}^\omega$.

\parbf{\ref{ex:lim(tree)}.}
Observe that if a path $\gamma$ in a metric tree from $p$ to $q$ pass thru a point $x$ on distance $\ell$ from $[pq]$, then 
\[\length\gamma\ge \dist{p}{q}{}+2\cdot \ell.\eqlbl{eq:+ell}\]

Suppose that $\spc{T}_n$ is a sequence of metric trees that $\omega$-converges to $\spc{T}_\omega$.
By \ref{obs:ultralimit-is-geodesic}, the space $\spc{T}_\omega$.

The uniqueness will follow from \ref{eq:+ell}.
Indeed, if for a geodesic $[p_\omega q_\omega]$ there is another geodesic $\gamma_\omega$ connecting its ends,
then it have to pass thru a point $x_\omega\notin [p_\omega q_\omega]$.
Choose a sequences $p_n,q_n,x_n\in\spc{T}_n$ such that $p_n\to p_\omega$, $q_n\to q_\omega$, $x_n\to x_\omega$ and $n\to\omega$.
Then 
\begin{align*}
\dist{p_\omega}{q_\omega}{}&=\length\gamma\ge \lim_{n\to\omega}(\dist{p_n}{x_n}{}+\dist{q_n}{x_n}{})\ge
\\
&\ge \lim_{n\to\omega}(\dist{p_n}{q_n}{}+2\cdot\ell_n)=
\\
&=\dist{p_\omega}{q_\omega}{}+2\cdot\ell_\omega.
\end{align*}
Since $x_\omega\notin [p_\omega q_\omega]$, we have that $\ell_\omega>0$ --- a contradiction.

To prove the last property consider sequence of centers of tripods $m_n$ for points $x_n,y_n,z_n\in \spc{T}_n$ and observe that its ultralimit $m_\omega$ is a the center of tripod with ends at $x_\omega,y_\omega,z_\omega\in \spc{T}_\omega$.
