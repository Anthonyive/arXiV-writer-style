     % ----------------------------------------------------------------
% AMS-LaTeX Paper ************************************************
% **** -----------------------------------------------------------
\documentclass[a4paper,11pt]{amsart}
\usepackage[foot]{amsaddr}
\newenvironment{nouppercase}{%
  \let\uppercase\relax%
  \renewcommand{\uppercasenonmath}[1]{}}{}

%\documentclass[a4paper,leqno,11pt,reqno]{amsart}
%\usepackage[showframe]{geometry}
%\usepackage{pb-diagram}
%\tolerance=1000
%\hbadness=10000
\raggedbottom
\hfuzz3pt
\usepackage{epsf,graphicx,epsfig}
\usepackage{amscd}
\usepackage{chngcntr}
\usepackage{enumitem}
\usepackage{amsmath,latexsym,amssymb,amsthm}
\usepackage[nospace,noadjust]{cite}
\usepackage{textcomp}
\usepackage{setspace,cite}
\usepackage{lscape,fancyhdr,fancybox}
\usepackage[all,cmtip]{xy}
%\usepackage[hmarginratio=1:1, vmarginratio =5:6,
%textheight=22cm,bindingoffset=1.6cm, textwidth=14.6cm]{geometry}
\setlength{\unitlength}{0.4in}
\renewcommand{\baselinestretch}{1.2}
\counterwithout{equation}{section}


\usepackage{graphicx}

\usepackage{color}
\usepackage{url}
\usepackage{enumitem}
\usepackage[mathscr]{euscript}
%\usepackage{showkeys}
%\input xy
%\xyoption{all}


\setlength{\topmargin}{0mm}
%\setlength{\topmargin}{-20mm}
\setlength{\textheight}{9in}
\setlength{\oddsidemargin}{0in}
\setlength{\evensidemargin}{0in}
\setlength{\textwidth}{6.2in}
%\setlength{\textwidth}{6.0in}

  \theoremstyle{plain}
%\theoremstyle{hhaplain}
%----------------------------------------------------------------
\vfuzz2pt % Don't report over-full v-boxes if over-edge is small
\hfuzz2pt % Don't report over-full h-boxes if over-edge is small
% THEOREMS -------------------------------------------------------
\newtheorem{theorem}{Theorem}[section]
\newtheorem{corollary}[theorem]{Corollary}
\newtheorem{lemma}[theorem]{Lemma}
\newtheorem{proposition}[theorem]{Proposition}
\newtheorem{example}[theorem]{Example}
\theoremstyle{definition}
\newtheorem{definition}[theorem]{Definition}
\theoremstyle{remark}
\newtheorem{remark}[theorem]{Remark}
\numberwithin{equation}{section}
% MATH -----------------------------------------------------------
\newcommand{\norm}[1]{\left\Vert#1\right\Vert}
\newcommand{\abs}[1]{\left\vert#1\right\vert}
\newcommand{\set}[1]{\left\{#1\right\}}
\newcommand{\Real}{\mathbb R}
\newcommand{\K}{\mathbb K}
\newcommand{\To}{\longrightarrow}
\newcommand{\BX}{\mathbf{B}(X)}
\newcommand{\A}{\mathcal{A}}
\newcommand{\corps}{\mathbb{K}}
\newcommand{\real}{\mathbb{R}}
\newcommand{\G}{\mathcal{G}}
\newcommand{\N}{\mathcal{N}}
\newcommand{\HH}{\mathcal{H}}
\newcommand{\sll}{\mathfrak{sl}_2(\mathbb{K})}
\newcommand{\Jacksll}{{}_q\mathfrak{sl}_2(\mathbb{K})}
\newcommand{\dl}{\displaystyle}
\newcommand{\ba}{\overline{\alpha}}
\newcommand{\AKMSbracket}[1]{\left[#1\right]}
\newcommand{\AKMSpara}[1]{\left(#1\right)}
\newcommand{\D}{\mathcal{L}(A)}
% ----------------------------------------------------------------
\makeatletter
\def\@settitle{\begin{center}%
  \baselineskip14\p@\relax
    %\bfseries
    \normalfont\LARGE%<- NEW
  \@title
  \end{center}%
}

\makeatother
\makeatletter
\renewcommand{\email}[2][]{%
  \ifx\emails\@empty\relax\else{\g@addto@macro\emails{,\space}}\fi%
  \@ifnotempty{#1}{\g@addto@macro\emails{\textrm{(#1)}\space}}%
  \g@addto@macro\emails{#2}%
}
\makeatother

\begin{document}

\title{${\mathcal{O}}$-O{perators} {on Hom-Lie algebras}}
\author[Satyendra Kumar Mishra and Anita Naolekar]{\textbf\normalfont\large S\MakeLowercase{atyendra} K\MakeLowercase{umar} M\MakeLowercase{ishra and }A\MakeLowercase{nita} N\MakeLowercase{aolekar}}
\email{satyamsr10@gmail.com}
\email{anita@isibang.ac.in}
\address{Statistics and Mathematics Unit, Indian Statistical Institute, Bangalore center, India.}
%\author{ 
\footnotetext{AMS Mathematics Subject Classification (2010): $17$A$30,$ $17$B$99,$ $16$T$25$.}
\footnotetext{{\it{Keyword}}: $\mathcal{O}$-operators; Hom-Lie algebras; Hom-pre-Lie algebras; deformation cohomology; differential graded Lie algebra.}
\begin{abstract}
In this article, we study $\mathcal{O}$-operators on hom-Lie algebras. We define a cohomology for $\mathcal{O}$-operators on hom-Lie algebras with respect to a representation. Any $\mathcal{O}$-operator induces a hom-pre-Lie algebra structure. We show that the cohomology of an $\mathcal{O}$-operator can be expressed in terms of certain hom-Lie algebra cohomology of the sub-adjacent hom-Lie algebra associated to the induced hom-pre-Lie algebra. If the structure maps in a hom-Lie algebra and its representation are invertible, then we can extend the above cohomology of $\mathcal{O}$-operators to a deformation cohomology by adding the space of zero cochains. Subsequently, we study linear and formal deformations of $\mathcal{O}$-operators on hom-Lie algebras in terms of the deformation cohomology. As an application, we deduce deformations of $s$-Rota-Baxter operators (of weight 0) and skew-symmetric $r$-matrices on hom-Lie algebras as particular cases of $\mathcal{O}$-operators on hom-Lie algebras.   
\end{abstract}
\maketitle

\section{\normalfont\large\textbf{Introduction}}
%, in the context of some particular deformation called $q$-deformations of Witt and Virasoro algebra of vector fields. In \cite{HALG}, hom-associative algebras are introduced as hom-Lie admissible algebras. Later on, many essential results on hom-Lie algebras and hom-associative algebras followed in \cite{DefHLIE,  HLIE01, QuasiHL1, LGMT18, HALG,  NtHOM, Sheng, UniHLie1, HALG2}. These type of algebras are becoming very popular and several new hom-algebraic structures such as hom-Hopf algebras \cite{HAD09, MSCO10, Yau3}, hom-Jordan algebras \cite{HJRD10} and hom-Poisson algebras \cite{Oliver, hom-Lie, NtHOM} are widely studied. A categorical interpretation of hom-algebra structures is given by S. Caenepeel and I. Goyvaerts in \cite{CGMON}.


%The deformation of an algebraic (or analytical) object is a fundamental tool to study the object. The deformation theory of algebraic structures was first introduced by M. Gerstenhaber for associative algebras in \cite{Ger63}-\cite{Ger74}. Gerstenhaber's theory was extended to Lie algebras by Nijenhuis and Richardson in \cite{NR66, NR67a, NR67b}. Later on, D. Balavoine \cite{Bal97} generalized algebraic deformation theory for algebras over quadratic operad and deduced formal deformation theory for Leibniz algebras. From the above-mentioned works, the deformations of algebraic structures is controlled by a certain deformation cohomology. For associative algebras, the deformation cohomology is given by the Hochschild cohomology and in the Lie algebra case, it is the Chevally-Eilenberg cohomology. Likewise, for hom-Lie algebras and hom-associative algebras, there are cohomologies with coefficients in adjoint representations (generalizing Chevally-Eilenberg and Hochschild cohomology, respectively in \cite{DefHLIE}), playing the role of the required deformation cohomologies. 


In this article, we deal with a specific type of algebraic structures, called `hom-algebraic structures'. The defining identities of such algebraic structure are twisted by endomorphisms. The first appearance of such structures was the notion of hom-Lie algebras \cite{HLIE01}, in the context of deformation of Witt and Virasoro algebras using a more general type of derivations known as `$\sigma$-derivations'. The approach of deformations using $\sigma$-derivations yields new deformations of Lie algebras and their central extensions. In particular, $q$-deformations of Witt and Virasoro algebras (For instance, see  \cite{q1,q3,q4,q5,q2,QuasiHL1,q7,q8, q9, q10}) for some of such constructions) can be described using $\sigma$-derivations. 

Our main objective is to study $\mathcal{O}$-operators (also known as relative Rota-Baxter operators) on hom-Lie algebras. In 1960, G. Baxter \cite{Baxter} first introduced the notion of Rota–Baxter operators for associative algebras. It is well-known that the Rota-Baxter operators have several applications in probability \cite{Baxter}, combinatorics \cite{Cartier, Guo, Rota}, and quantum field theory \cite{Connes}. In the 1980s, the notion of Rota-Baxter operator of weight $0$ was introduced in terms of the classical Yang-Baxter equation for Lie algebras (see \cite{Survey-Guo} for more details). Later on, B. A. Kupershmidt \cite{Kupershmidt} defined the notion of $\mathcal{O}$-operators as generalized Rota-Baxter operators to understand classical Yang-Baxter equations and related integrable systems.  

More recently, R. Tang and the coauthors \cite{Sheng3} developed formal deformation theory for $\mathcal{O}$-operators on Lie algebras. They constructed a differential graded Lie algebra (dgla) and characterize $\mathcal{O}$-operators as Maurer-Cartan elements of this dgla, which in turn allows them to introduce deformation cohomology for $\mathcal{O}$-operators. Here, we develop a cohomology for $\mathcal{O}$-operators on hom-Lie algebras. For this purpose, we follow the approach developed in \cite{Sheng3}. In particular, we use the graded Lie algebra structure on the deformation complex of hom-Lie algebras \cite{DefHLIE} and derived bracket construction by Voronov \cite{Voronov} to obtain an explicit graded Lie algebra. We characterize $\mathcal{O}$-operators on hom-Lie algebras as Maurer-Cartan elements of this graded Lie algebra, which allows us to construct a cochain complex defining a cohomology for an $\mathcal{O}$-operator on hom-Lie algebras. We interpret this cohomology as a hom-Lie algebra cohomology of a particular hom-Lie algebra with coefficients in a suitable representation. In the sequel, we show that the above-mentioned cohomology of $\mathcal{O}$-operators on hom-Lie algebras can be modified to a deformation cohomology for $\mathcal{O}$-operators if one considers the structure maps to be invertible.

The section-wise summary of the paper is as follows.
In Section $2$, we recall some basic definitions and results on hom-Lie algebras and their representations. We define a modified complex for regular hom-Lie algebras with coefficients in regular representations. We characterize hom-Lie algebras equipped with a representation in terms of Maurer-Cartan elements of a graded Lie algebra (defined in \cite{DefHLIE}).
 
In Section $3$, We define the notion of $\mathcal{O}$-operators on a hom-Lie algebra with respect to a representation. There is a graded Lie algebra structure on the deformation complex of hom-Lie algebra, which we use to obtain a graded Lie algebra whose Maurer-Cartan elements are $\mathcal{O}$-operators on hom-Lie algebras with respect to a representation. With this characterization, we get a differential graded Lie algebra associated to an $\mathcal{O}$-operator. Consequently, we define a cohomology for $\mathcal{O}$-operators on hom-Lie algebras with respect to a representation. We show that an $\mathcal{O}$-operators $T:V\rightarrow \mathfrak{g}$ on a hom-Lie algebra $(\mathfrak{g},[~,~],\alpha)$ with respect to a representation $(V,\beta,\rho)$, induces a hom-pre-Lie algebra structure on $V$. Furthermore, the cohomology of the $\mathcal{O}$-operator $T$ can be given in terms of the hom-Lie algebra cohomology of the sub adjacent hom-Lie algebra $(V,[~,~]^c,\beta)$ of the induced pre-Lie algebra with coefficients in a representation $(\mathfrak{g},\alpha,\rho_T)$. In the case when the structure maps $\alpha$ and $\beta$ are invertible, we define an extended cochain complex (including $0$-cochains) $\big(\widetilde{C}^*_{\beta,\alpha}(V,\mathfrak{g}),\delta_{\beta,\alpha}\big)$ for the $\mathcal{O}$-operator $T:V\rightarrow \mathfrak{g}$ on the hom-Lie algebra $(\mathfrak{g},[~,~],\alpha)$ with respect to the representation $(V,\beta,\rho)$.   

In Section $4$, we discuss deformations of $\mathcal{O}$-operators on hom-Lie algebras. For this section, we consider the structure maps to be invertible for both the hom-Lie algebras and their representations. We will see that the modified complex $\big(\widetilde{C}^*_{\beta,\alpha}(V,\mathfrak{g}),\delta_{\beta,\alpha}\big)$ is not possible unless we assume invertibility of the structure maps. This modified complex gives us a deformation cohomology of $\mathcal{O}$-operators. First, we consider linear deformations of $\mathcal{O}$-operators. We discuss trivial linear deformations in terms of Nijenhuis elements. Then we consider formal deformations of $\mathcal{O}$-operators. We show that equivalent formal deformations of $\mathcal{O}$-operators have cohomologous infinitesimals. We also consider the problem of extending a finite order deformation to the next higher order deformation. 

In the last section, we consider $s$-Rota-Baxter operators of weight $0$ and skew-symmetric $r$-matrices on hom-Lie algebras as particular cases of $\mathcal{O}$-operators. Consequently, we discuss deformations of $s$-Rota-Baxter operators of weight $0$ and skew-symmetric $r$-matrices on hom-Lie algebras.


\section{\normalfont\large\textbf{{Preliminaries}}}
Throughout the paper, we consider the base field to be $k$ with characteristic zero. All linear maps and tensor products are taken over the field $k$ unless otherwise stated.

\subsection{Hom-Lie algebras}
 Let us recall some definitions and results related to hom-Lie algebras and their representations. 

\begin{definition}
 A (multiplicative) hom-Lie algebra is a triplet $(\mathfrak{g},[~,~],\alpha)$, where $\mathfrak{g}$ is a vector space equipped with a skew-symmetric bilinear map $[~,~]:\mathfrak{g}\otimes \mathfrak{g}\rightarrow \mathfrak{g}$, and a linear map $\alpha:\mathfrak{g}\rightarrow \mathfrak{g}$ satisfying $\alpha[x,y]=[\alpha(x),\alpha(y)]$ such that
\begin{equation}
[\alpha(x),[y,z]]+[\alpha(y),[z,x]]+[\alpha(z),[x,y]]=0, ~~~~\mbox{for all}~~x,y,z\in \mathfrak{g}.
\end{equation}
Furthermore, if $\alpha:\mathfrak{g}\rightarrow \mathfrak{g}$ is a vector space automorphism of $\mathfrak{g}$, then the hom-Lie algebra $(\mathfrak{g},[~,~],\alpha)$ is called a regular hom-Lie algebra.
\end{definition}

\begin{example}
Given a Lie algebra $\mathfrak{g}$ with a Lie algebra homomorphism $\alpha:\mathfrak{g}\rightarrow \mathfrak{g}$, we can define a hom-Lie algebra as the triplet $(\mathfrak{g},\alpha\circ [~,~],\alpha)$, where $[~,~]$ is the underlying Lie bracket. 
\end{example}

\begin{definition}\label{Rep-hom-Lie}(\cite{Sheng}) A representation of a hom-Lie algebra $(\mathfrak{g},[~,~],\alpha)$ on a $k$-vector space $V$ with respect to $\beta\in\mathsf{End}(V)$ is a linear map $\rho: \mathfrak{g}\rightarrow \mathsf{End}(V)$ such that 
\begin{equation}
\rho(\alpha(x))(\beta(v))= \beta(\rho(x)(v)),
\end{equation}
\begin{equation}
\rho([x,y])(\beta(v))= \rho(\alpha(x))\rho(y)(v)-\rho(\alpha(y))\rho(x)(v),
\end{equation}
for all $x,y \in \mathfrak{g}$ and $v\in V$. 
%A representation $(V,\beta,\rho)$ is said to be 'regular' if the map $\beta:V\rightarrow V$ is a vector space automorphism of $V$.
\end{definition}
Let us denote a representation $\rho$ on $V$ with respect to $\beta\in \mathsf{End}(V)$ by a triple $(V,\beta,\rho)$. Throughout the paper, for simplification, we use the notation
$$\{x,v\}:=\rho(x)(v), \quad\mbox{for all } x\in \mathfrak{g}, ~v\in V.$$

\begin{example}[\cite{Sheng}]\label{adjoint representation}
 For any integer $s\geq 0$, we can define the $\alpha^s$-adjoint representation of a hom-Lie algebra $(\mathfrak{g},[~,~],\alpha)$ on $\mathfrak{g}$ as follows
$$\mathsf{ad}_x^s(y):=[\alpha^s(x),y]\quad\mbox{for all}~x,y\in \mathfrak{g}.$$
Let us denote the $\alpha^s$-adjoint representation of the hom-Lie algebra $(\mathfrak{g},[~,~],\alpha)$ by the triple $(\mathfrak{g},\alpha,\mathsf{ad}^s)$. Also, we denote $\mathsf{ad}_x^0$ simply by $\mathsf{ad}_x$ for any $x\in\mathfrak{g}$.
\end{example}

\begin{example}[\cite{hom-Liebi}]\label{coadjoint rep}
Let $(\mathfrak{g},[~,~],\alpha)$ be a regular hom-Lie algebra and $(V,\beta,\rho)$ be a hom-Lie algebra representation with $\beta$ an invertible linear map. Let us define a map $\rho^\star:\mathfrak{g}\rightarrow\mathsf{End}(V^*)$ by
\begin{align*}
\langle\rho^{\star}(x)(\xi),v\rangle :=&\langle\rho^*(\alpha(x))((\beta^{-2})^*(\xi)),v\rangle \\
=&\langle\xi,\rho(\alpha^{-1}(x))(\beta^{-2}(v))\rangle,
\end{align*}
for all $x\in\mathfrak{g}$ and $v\in V$. Then the triplet $(V^*,(\beta^{-1})^*,\rho^\star)$ is a representation of the hom-Lie algebra $(\mathfrak{g},[~,~],\alpha)$ on the dual vector space $V^*$ with respect to the map $(\beta^{-1})^*$. This is also known as the `dual representation' to $(V,\beta,\rho)$. 

In particular, let us also recall that the `coadjoint representation' of a regular hom-Lie algebra $(\mathfrak{g},[~,~],\alpha)$ on $\mathfrak{g}^*$ with respect to the map $(\alpha^{-1})^*$ is given by the triplet $(\mathfrak{g}^*,(\alpha^{-1})^*,\mathsf{ad}^\star)$, where
$$\langle \mathsf{ad}^\star(x)(\xi),y\rangle=\langle\xi, [\alpha^{-1}(x),\alpha^{-2}(x)]\rangle, \quad\mbox{for all }x,y\in\mathfrak{g},~\xi\in \mathfrak{g}^*.$$
\end{example}




\begin{lemma}\label{direct sum}
Let $(\mathfrak{g},[~,~],\alpha)$ be a hom-Lie algebra and $V$ be a vector space equipped with a linear map $\beta\in \mathsf{End}(V)$ and an action $\rho:\mathfrak{g}\otimes V\rightarrow V$. Then, the triplet $(V,\beta,\rho)$ is a hom-Lie algebra representation on $V$ if and only if $(\mathfrak{g}\oplus V,[~,~]_{\rho},\alpha+\beta)$ is a hom-Lie algebra, where the action $\mathsf{ad}^\star$ is given by 
\begin{align*}
[x+v,y+w]_{\rho}&=[x,y]+\rho(x)(w)-\rho(y)(v),\\
(\alpha+\beta)(x,v)&=(\alpha(x),\beta(v)),\quad\mbox{for all }x,y\in \mathfrak{g}~\mbox{and } v,w\in V.
\end{align*}
\end{lemma}
\begin{proof}
The proof of the lemma is straightforward (see \cite{Sheng} for details).
\end{proof}
The hom-Lie algebra $(\mathfrak{g}\oplus V,[~,~]_{\rho},\alpha+\beta)$ is called the semi-direct product hom-Lie algebra for a hom-Lie algebra $(\mathfrak{g},[~,~],\alpha)$ with a representation $(V,\beta,\rho)$.
\subsection{A cochain complex for hom-Lie algebras}
Let $(\mathfrak{g},[~,~],\alpha)$ be a hom-Lie algebra with a representation $(V, \beta, \rho)$ on a vector space $V$. We define a cochain complex $(C^*_{\alpha,\beta}(\mathfrak{g},V),\delta_{\alpha,\beta})$ for the hom-Lie algebra $(\mathfrak{g},[~,~],\alpha)$ with coefficients in the representation $(V, \beta, \rho)$. Here, 
$$C^*_{\alpha,\beta}(\mathfrak{g},V):=\bigoplus_{n\geq 1}C^n_{\alpha,\beta}(\mathfrak{g},V)$$
and $C^n_{\alpha,\beta}(\mathfrak{g},V)$ is a subspace of $Hom(\wedge^n \mathfrak{g},V)$ consisting of all those linear maps $f:\wedge^n \mathfrak{g}\rightarrow V,$ which satisfy the following condition
$$f(\alpha(x_1),\cdots,\alpha(x_n))=\beta(f(x_1,x_2,\cdots,x_n)).$$
The coboundary map $\delta_{\alpha,\beta}:C^n_{\alpha,\beta}(\mathfrak{g},V)\rightarrow C^{n+1}_{\alpha,\beta}(\mathfrak{g},V)$ is given by 
\begin{equation}\label{coboundary:hom-Lie algebra}
\begin{split}
\delta_{\alpha,\beta} f(x_1,\cdots,x_{n+1}):= &\sum_{i=1}^{n+1}(-1)^{i+1}\rho(\alpha^{n-1}(x_i))(f(x_1,\cdots,\hat{x_i},\cdots,x_{n+1}))\\
&+\sum_{i<j} (-1)^{i+j}f([x_i,x_j],\alpha(x_1),\cdots,\hat{\alpha(x_i)},\cdots,\hat{\alpha(x_j)},\cdots,\alpha(x_{n+1}))
\end{split}
\end{equation}
for all $f\in C^n_{\alpha,\beta}(\mathfrak{g},V)$ and $x_1,x_2,\cdots x_{n+1}\in \mathfrak{g}$. Let us denote by $H^*_{\alpha,\beta}(\mathfrak{g},V)$, the cohomology space  associated to the cochain complex $(C^*_{\alpha,\beta}(\mathfrak{g},V),\delta_{\alpha,\beta})$. 

\begin{remark}
This cochain complex is different from the one defined in \cite{Sheng}. If $V=\mathfrak{g},$ $\beta=\alpha,$ and the action $\rho: \mathfrak{g}\otimes V\rightarrow V$ is given by the underlying hom-Lie bracket, then we denote the above cochain complex by $(C^*_{\alpha}(\mathfrak{g},\mathfrak{g}),\delta_{\alpha})$. This complex is the same as the deformation complex of the hom-Lie algebra $(\mathfrak{g},[~,~],\alpha)$ (defined in \cite{DefHLIE}). The cohomology of the complex $(C^*_{\alpha}(\mathfrak{g},\mathfrak{g}),\delta_{\alpha})$ serves as deformation cohomology for hom-Lie algebra.
\end{remark}

\begin{remark}\label{regular hom-Lie algebra cohomology}
Let $(\mathfrak{g},[~,~],\alpha)$ be a regular hom-Lie algebra and $(V, \beta, \rho)$ be a hom-Lie algebra representation, where $\beta:V\rightarrow V$ is invertible. Then, one can define 
$$C^0_{\alpha,\beta}(\mathfrak{g},V):=\{v\in V| \beta(v)=v\},$$
and the coboundary $\delta_{\alpha,\beta}:C^0_{\alpha,\beta}(\mathfrak{g},V)\rightarrow C^1_{\alpha,\beta}(\mathfrak{g},V)$ on $0$-cochains is given by
$$\delta_{\alpha,\beta}(v)(x):=\rho(\alpha^{-1}(x))(v),~\quad\mbox{for all }v\in C^0_{\alpha,\beta}(\mathfrak{g},V) ~\mbox{and}~ x\in \mathfrak{g}.$$
Since the structure maps $\alpha$ and $\beta$ are invertible, it follows that we have a modified cochain complex $\big(\widetilde{C}^*_{\alpha,\beta}(\mathfrak{g},V),\delta_{\alpha,\beta}\big)$, where 
$$\widetilde{C}^*_{\alpha,\beta}(\mathfrak{g},V):=\bigoplus_{n\geq 0} C^n_{\alpha,\beta}(\mathfrak{g},V).$$
We denote the associated cohomology by $\widetilde{H}^*_{\alpha,\beta}(\mathfrak{g},V)$.
\end{remark}

\subsection{Hom-Lie algebras and their representations in terms of Maurer-Cartan elements}
Let $\mathfrak{g}$ be a vector space equipped with a linear map $\alpha:\mathfrak{g}\rightarrow\mathfrak{g}$. Let us consider the graded vector space $C^*_{\alpha}(\mathfrak{g};\mathfrak{g})$. Then, we recall from  \cite{DefHLIE} that for any $\varphi\in C^p_{\alpha}(\mathfrak{g},\mathfrak{g})$ and $\psi\in C^q_{\alpha}(\mathfrak{g},\mathfrak{g})$, a circle product $\varphi\circ_\alpha \psi$ is defined by the following expression
\begin{equation}\label{Defofcirc}
\begin{split}
&(\varphi \circ_{\alpha} \psi)(x_1, x_2, \ldots, x_{p+q+1})\\ &= \textstyle{\sum\limits_{\tau \in Sh (q+1, p)} (-1)^{|\tau|}
\varphi \big(\psi (x_{\tau(1)}, \ldots, x_{\tau (q+1)}),\alpha^q(x_{\tau (q+2)}) , \ldots, \alpha^q(x_{\tau (p+q+1))}) \big)},
\end{split}
\end{equation}
where $Sh (q+1, p)$ denotes the set of $(q+1,p)$ shuffles in $S_{q+p+1}$ (the symmetric group on the set $\{1,2,\cdots,q+p+1\}$). For any permutation $\tau\in S_{q+p+1}$, the notation $|\tau|$ denotes the signature of the permutation $\tau$. 

We can define a bracket of degree $-1$ on the graded vector space $C^*_{\alpha}(\mathfrak{g},\mathfrak{g})$ in terms of the circle product
\begin{equation}\label{GLB}
[\varphi, \psi]_N^{\alpha} := (-1)^{pq} \varphi \circ_{\alpha} \psi - \psi \circ_{\alpha} \varphi.
\end{equation}

Thus, by a degree shift we obtain a graded Lie algebra structure on $C^{*-1}_{\alpha}(\mathfrak{g},\mathfrak{g})$. Hom-Lie algebra structures on $(\mathfrak{g},\alpha)$  corresponds bijectively to Maurer-Cartan elements of the graded Lie algebra $C^{*-1}_{\alpha}(\mathfrak{g},\mathfrak{g})$, i.e., the elements $\mu\in C^{2}_{\alpha}(\mathfrak{g},\mathfrak{g})$ satisfying $[\mu,\mu]_N^{\alpha}=0$. If a hom-Lie algebra structure $(\mathfrak{g},[~,~],\alpha)$ corresponds to such an element $\mu\in C^{2}_{\alpha}(\mathfrak{g},\mathfrak{g})$, one obtains a differential graded Lie algebra structure on $C^{*-1}_{\alpha}(\mathfrak{g},\mathfrak{g})$ with the differential $d_{\mu}=[\mu,-]_N^{\alpha}$. This differential $d_{\mu}$ coincides with the coboundary operator $\delta_{\alpha}$ given by equation \eqref{coboundary:hom-Lie algebra}.

Let $(\mathfrak{g},\alpha)$ and $(V,\beta)$ be vector spaces equipped with linear operators. We now consider the graded Lie algebra 
$$\textstyle{\big(\mathcal{G}^*:=C^{*-1}_{\alpha}(\mathfrak{g}\oplus V,\mathfrak{g}\oplus V),~[~,~]_N^{\alpha+\beta}\big)}$$ 
associated to the pair $(\mathfrak{g}\oplus V,\alpha+\beta)$. Let $\mu: \wedge^2(\mathfrak{g})\rightarrow\mathfrak{g}$ and $\rho:\mathfrak{g}\rightarrow \mathsf{End}(V)$ be linear maps, then define a linear map $\mu+\rho:\wedge^2(\mathfrak{g}\oplus V)\rightarrow \mathfrak{g}\oplus V$ by
$$\mu+\rho(x+v,y+w)=\mu(x,y)+\rho(x)(w)-\rho(y)(v),\quad \mbox{for any }x,y\in \mathfrak{g}~~\mbox{and }v,w\in V.$$
With the above notations, we have the following proposition.

\begin{proposition}\label{Maurer-Cartan}
The map $\mu$ defines a hom-Lie algebra structure on the pair $(\mathfrak{g},\alpha)$ and the map $\rho$ defines a hom-Lie algebra representation on the pair $(V,\beta)$ if and only if $\mu+\rho$ is a Maurer-Cartan element of the graded Lie algebra $\textstyle{\big(\mathcal{G}^*,~[~,~]_N^{\alpha+\beta}\big)}$.
\end{proposition}
\begin{proof}
First, let us observe that the map $\mu+\rho\in \mathcal{G}^1$ if and only if 
$$\mu+\rho\big(\alpha+\beta(x+v),\alpha+\beta(y+w)\big)=(\alpha+\beta)\big(\mu+\rho(x+v,y+w)\big),$$
which is equivalent to the following expressions
$$\mu(\alpha(x),\alpha(y))=\alpha\big(\mu(x,y)\big)\quad \mbox{and}\quad \rho(\alpha(x))(\beta(v))=\beta(\rho(x)(v)),\quad \mbox{for all }x,y\in \mathfrak{g},~v\in V.$$
Moreover, the map $\mu+\rho$ is a Maurer-Cartan element if and only if 
\begin{equation}\label{square-zero}
\textstyle{[\mu+\rho,\mu+\rho]_N^{\alpha+\beta}(x+u,y+v,z+w)=-2~\big((\mu+\rho)\circ_{\alpha+\beta}(\mu+\rho)\big)(x+u,y+v,z+w)=0},
\end{equation}
or equivalently
\begin{align}\label{HJacobi n representation}
\nonumber
[[x,y],\alpha(z)]+[[y,z],\alpha(x)]+[[z,x],\alpha(y)]&=0,\quad \mbox{for all }x,y,z \in \mathfrak{g}\\
\rho([x,y])(\beta(w))- \rho(\alpha(x))\rho(y)(w)+\rho(\alpha(y))\rho(x)(w)&=0,\quad \mbox{for all }x,y\in \mathfrak{g}~~\mbox{and }w\in V.
\end{align} 
Note that we obtain the equation \eqref{HJacobi n representation} from equation \eqref{square-zero} by taking $u=v=0$. Hence, the result of the proposition follows.
\end{proof}


\section{\normalfont\large\textbf{{$\mathcal{O}$-operators on hom-Lie algebras}}}

In this section, we define the notion of $s$-Rota-Baxter operators and $\mathcal{O}$-operators on hom-Lie algebras. We use the graded Lie algebra structure on the deformation complex of a hom-Lie algebra and derived bracket construction to define a graded Lie algebra whose Maurer-Cartan elements are precisely the $\mathcal{O}$-operators on hom-Lie algebras. Subsequently, we obtain a differential graded Lie algebra associated to an $\mathcal{O}$-operator. 


\begin{definition}\label{def:Rota-Baxter operators}
Let $(\mathfrak{g},[~,~],\alpha)$ be a hom-Lie algebra and $s$ be a non-negative integer. Then, a linear operator $\mathcal{R}: \mathfrak{g} \rightarrow \mathfrak{g}$ is called an $s$-Rota-Baxter operator on $(\mathfrak{g},[~,~],\alpha)$ if $\mathcal{R}\circ \alpha= \alpha\circ \mathcal{R}$ and the following identity is satisfied
\begin{equation*} 
\quad\quad\quad\quad\quad[\mathcal{R}(x), \mathcal{R}(y)]= \mathcal{R}([\alpha^s \mathcal{R}(x), y]+ [x, \alpha^s \mathcal{R}(y)]+\lambda [x,y]),\quad\mbox{for all }x,y\in \mathfrak{g}.
\end{equation*}
\end{definition}


For $\alpha=\mathsf{Id}$, the Definition \ref{def:Rota-Baxter operators} coincides with the notion of Rota-Baxter operators on a Lie algebra. 

\begin{definition}\label{O-operator}
Let $(\mathfrak{g},[~,~],\alpha)$ be a hom-Lie algebra and $(V,\beta,\rho)$ be a hom-Lie algebra representation. A linear map $T:V\rightarrow \mathfrak{g}$ is called an $\mathcal{O}$-operator on $(\mathfrak{g},[~,~],\alpha)$ with respect to the representation $(V,\beta,\rho)$ if the following conditions hold
\begin{align*}
T\circ\beta&=\alpha\circ T,\\
[Tu,Tv]&=T\big(\{Tu,v\}-\{Tv,u\}\big),\quad\mbox{ for all }  u,v\in V.
\end{align*}
\end{definition}

\begin{remark}\label{5.2}
Let us recall from Example \ref{adjoint representation} that for any integer $s\geq 0$, the $\alpha^s$-adjoint representation $(\mathfrak{g}, \alpha, \mathsf{ad}^s)$ of a hom-Lie algebra $(\mathfrak{g},[~,~],\alpha)$. Then, any $s$-Rota-Baxter operator of weight $0$ on the hom-Lie algebra $(\mathfrak{g},[~,~],\alpha)$ is an $\mathcal{O}$-operator on $(\mathfrak{g},[~,~],\alpha)$ with respect to the representation $(\mathfrak{g}, \alpha, \mathsf{ad}^s)$.
The notion of $\mathcal{O}$-operators is a generalization of Rota-Baxter operators and therefore also known as relative Rota-Baxter operator.
\end{remark}


\begin{example}
If $\alpha=\mathsf{Id}_{\mathfrak{g}}$ and $\beta=\mathsf{Id}_V$, then the Definition \ref{O-operator} coincides with the notion of $\mathcal{O}$-operators on a Lie algebra. 
\end{example}

\begin{example}
Let $T:V\rightarrow \mathfrak{g}$ be an $\mathcal{O}$-operator on a Lie algebra $(\mathfrak{g},[~,~])$ with respect to a Lie algebra representation $\rho$ on $V$. A pair $(\phi_{\mathfrak{g}},\phi_V)$ is an endomorphism of the $\mathcal{O}$-operator $T$ if 
\begin{align*}
T\circ \phi_{V}&=\phi_{\mathfrak{g}}\circ T\quad \mbox{and}\\
\rho(\phi_{\mathfrak{g}}(x))(\phi_{V}(v))&=\phi_V(\rho(x)(v)), \quad \mbox{for all } x\in \mathfrak{g}, ~v\in V. 
\end{align*}

 Let us consider the hom-Lie algebra $(\mathfrak{g},[~,~]_{\phi_\mathfrak{g}},\phi_{\mathfrak{g}})$ obtained by composition, where the hom-Lie bracket is given by 
  $$[~,~]_{\phi_\mathfrak{g}}:=\phi_{\mathfrak{g}}\circ [~,~].$$ 
If we consider the composition $\rho_{\phi_V}:=\phi_V\circ \rho$, then the triplet $(V,\phi_V,\rho_{\phi_V})$ is a hom-Lie algebra representation of $(\mathfrak{g},[~,~]_{\phi_\mathfrak{g}},\phi_{\mathfrak{g}})$. Moreover, 
$$[T(v),T(w)]_{\phi_\mathfrak{g}}=\phi_{\mathfrak{g}}[T(v),T(w)]=\phi_{\mathfrak{g}}\big(T(\rho(T(v))(w)-\rho(T(w))(v))\big),$$
and 
$$T\big(\rho_{\phi_V}(T(v))(w)-\rho_{\phi_V}(T(w))(v)\big)=T\big(\phi_V(\rho(T(v))(w)-\rho(T(w))(v))\big), $$
for all $v,w \in V$. Clearly, it follows that the map $T:V\rightarrow \mathfrak{g}$ is an $\mathcal{O}$-operator on hom-Lie algebra $(\mathfrak{g},[~,~]_{\phi_\mathfrak{g}},\phi_{\mathfrak{g}})$ with respect to the hom-Lie algebra representation $(V,\phi_V,\rho_{\phi_V})$.
\end{example}

The following proposition gives a characterization of an $\mathcal{O}$-operator $T$ in terms of a hom-Lie subalgebra structure on the graph of $T$.

\begin{proposition}
A map $T:V\rightarrow \mathfrak{g}$ is an $\mathcal{O}$-operator on $(\mathfrak{g},[~,~],\alpha)$ with respect to the representation $(V,\beta,\rho)$ if and only if the graph of the map $T$ $$Gr(T)=\{(T(v),v)|~v\in V\}$$
is a  hom-Lie subalgebra of the semi-direct product hom-Lie algebra $(\mathfrak{g}\oplus V,[~,~]_{\rho},\alpha+\beta)$, defined in Lemma \ref{direct sum}.
\end{proposition}

It is known that $\mathcal{O}$-operators on Lie algebras can be characterized in terms of the Nijenhuis operators. In \cite{Sheng}, Nijenhuis operators on hom-Lie algebras are defined as follows.

\begin{definition}
A linear map $N:\mathfrak{g}\rightarrow \mathfrak{g}$ is called a Nijenhuis operator on the hom-Lie algebra $(\mathfrak{g},[~,~],\alpha)$ if
$$[N(x),N(y)]=N\big([N(x),y]-[N(y),x]-N([x,y])\big)\quad \mbox{for all }x,y\in \mathfrak{g}.$$
\end{definition}

In the next result, we characterize $\mathcal{O}$-operators on hom-Lie algebras in terms of the Nijenhuis operators.
\begin{proposition}
A map $T:V\rightarrow \mathfrak{g}$ is an $\mathcal{O}$-operator on $(\mathfrak{g},[~,~],\alpha)$ with respect to the representation $(V,\beta,\rho)$ if and only if the operator 
$$N_T=\begin{bmatrix}
   0 & T \\
    0  & 0
\end{bmatrix}: \mathfrak{g}\oplus V\rightarrow \mathfrak{g}\oplus V $$  
is a Nijenhuis operator on the semi-direct product hom-Lie algebra $(\mathfrak{g}\oplus V,[~,~]_{\rho},\alpha+\beta)$.
\end{proposition}
\begin{proof} Let us consider the following expressions, where we use definition of the map $N_T$ and the bracket $[~,~]_{\rho}$. 
\begin{equation}\label{Char2: eq1}
[N_T(x+v),N_T(y+w)]_{\rho}= [T(v)+0,T(w)+0]_{\rho}=[T(v),T(w)], 
\end{equation}
and
\begin{align}\label{Char2: eq2}
\nonumber
&N_T\big([N_T(x+v),y+w]_{\rho}-[N_T(y+w),x+v]_{\rho}-N_T([x+v,y+w]_{\rho})\big)\\\nonumber
=&N_T \big(([T(v),y]+\{T(v),w\})-([T(w),x]+\{T(w),v\})-(0+T(\{x,w\}-\{y,v\}))\big)\\
=& T(\{T(v),w\}-\{T(w),v\}),
\end{align}
for all $x,y\in\mathfrak{g},$ $v,w\in V$. By the above equations \eqref{Char2: eq1}-\eqref{Char2: eq2}, it is clear that the condition  
$$[N_T(x+v),N_T(y+w)]_{\rho}=N_T\big([N_T(x+v),y+w]_{\rho}-[N_T(y+w),x+v]_{\rho}-N_T([x+v,y+w]_{\rho})\big)$$
is equivalent to the condition
$$[T(v),T(w)]=T(\{T(v),w\}-\{T(w),v\}),$$
for all $x,y\in\mathfrak{g},$ $v,w\in V$. Hence, the statement of the proposition holds true.
\end{proof}
\begin{definition}\label{morphism of O-operators}
Let $T:V\rightarrow \mathfrak{g}$ and $T^{\prime}:V\rightarrow \mathfrak{g}$ be two $\mathcal{O}$-operators on the hom-Lie algebra $(\mathfrak{g},[~,~],\alpha)$ with respect to the representation $(V,\beta,\rho)$. A homomorphism from $T$ to $T^{\prime}$ is given by a pair $(\phi_g,\phi_v)$, consisting of a hom-Lie algebra homomorphism $\phi_{\mathfrak{g}}:\mathfrak{g}\rightarrow \mathfrak{g}$ and a linear map $\phi_V: V\rightarrow V$ such that following conditions are satisfied
\begin{equation}\label{morphism:cond1}
T^{\prime}\circ \phi_V =\phi_{\mathfrak{g}}\circ T,\quad\quad\quad
\end{equation}
\begin{equation}\label{morphism:cond2}
\phi_V\circ \beta  =\beta\circ \phi_V,\quad\quad\quad
\end{equation}
\begin{equation}\label{morphism:cond3}
\quad\quad \quad\quad\quad\quad~~~~~~~\phi_V(\rho(x)(v)) =\rho(\phi_g(x))(\phi_V(v)), \quad\mbox{for all  }v\in V.
\end{equation}
\end{definition}

\subsection{A differential graded Lie algebra}

Let $(\mathfrak{g},[~,~],\alpha)$ be a hom-Lie algebra with a representation $(V,\beta,\rho)$. Then, we have a graded Lie algebra 
$$\textstyle{\big(\mathcal{G}^*:=C^{*-1}_{\alpha}(\mathfrak{g}\oplus V,\mathfrak{g}\oplus V),~[~,~]_N^{\alpha+\beta}\big)}$$ 
associated to the pair $(\mathfrak{g}\oplus V,\alpha+\beta)$.
 Then, by Proposition \ref{Maurer-Cartan}, the hom-Lie bracket and the representation correspond to an element $\mu+\rho\in \mathcal{G}^1$ satisfying $[\mu+\rho,\mu+\rho]_N^{\alpha+\beta}=0$. Let us define a map $d_{\mu+\rho}:\mathcal{G}^*\rightarrow \mathcal{G}^{*+1}$ by
$$d_{\mu+\rho}:=[\mu+\rho,-]_N^{\alpha+\beta}.$$
By the graded Jacobi identity for the bracket $[~,~]_N^{\alpha+\beta}$, it is clear that $(\mathcal{G}^*,[~,~]_N^{\alpha+\beta},d_{\mu+\rho})$ is a differential graded Lie algebra.

Now, we define a graded vector space 
\begin{equation*}
C^*_{\beta,\alpha}(V,\mathfrak{g})=\bigoplus_{n\geq 1}C^n_{\beta,\alpha}(V,\mathfrak{g}),
\end{equation*}
 where for each $n\geq 1$, the vector space $C^n_{\beta,\alpha}(V,\mathfrak{g})$ is a subspace of $Hom(\wedge^k V,\mathfrak{g})$, which consists of all the linear maps $P:\wedge^n V\rightarrow \mathfrak{g}$ satisfying 
$$\alpha(P(v_1,v_2,\ldots,v_{n}))=P(\beta(v_1),\beta(v_2),\ldots,\beta(v_{n})), \quad\mbox{for all }~~ v_1,v_2,\cdots, v_{n}\in V.$$

Let us define a graded Lie bracket 
$$\{\!\!\{ -,- \}\!\!\}:C^n_{\beta,\alpha}(V,\mathfrak{g})\otimes C^m_{\beta,\alpha}(V,\mathfrak{g}) \rightarrow C^{n+m}_{\beta,\alpha}(V,\mathfrak{g})$$
as follows
\begin{equation}\label{definition of bracket}
\{\!\!\{P,Q\}\!\!\}:=(-1)^n[[~\mu+\rho~,~ P]_N^{\alpha+\beta},~Q]_N^{\alpha+\beta}
\end{equation}

By definition of the graded Lie bracket $[~,~]_N^{\alpha+\beta},$ the bracket $\{\!\!\{-,-\}\!\!\}$ can be written as follows:
\begin{align}\label{derived bracket}
\nonumber
&\{\!\!\{P,Q\}\!\!\}(v_1,v_2,\ldots,v_{n+m})\\\nonumber
&=\sum_{\tau\in S_{m,1,n-1}}(-1)^{|\tau|}~ P\Big(\{Q(v_{\tau(1)},\ldots,v_{\tau(m)}),\beta^{m-1}(u_{\tau(m+1)})\},\beta^{m}(v_{\tau(n+2)}),\ldots,\beta^{m}(v_{\tau(n+m)})\Big)\\\nonumber
&+(-1)^{mn}\Bigg(\sum_{\tau\in S_{n,m}}(-1)^{|\tau|}~\big[\alpha^{m-1}P(v_{\tau(1)},\ldots,v_{\tau(n)}),\alpha^{n-1}Q(v_{\tau(n+1)},\ldots,v_{\tau(n+m)})\big]\\
&-\sum_{\tau\in S_{n,1,m-1}}(-1)^{|\tau|}~ Q\Big(\{P(v_{\tau(1)},\ldots,v_{\tau(n)}),\beta^{n-1}(u_{\tau(n+1)})\},\beta^{n}(v_{\tau(n+2)}),\ldots,\beta^{n}(v_{\tau(n+m)})\Big)\Bigg)
\end{align}

Here, $S_{r_1,r_2,\ldots,r_i}$ denotes a $(r_1,r_2,\ldots,r_i)$-shuffle in the permutation group $S_{r_1+r_2+\cdots+r_i}$. The above graded Lie bracket is obtained via the derived bracket construction, introduced by Voronov in \cite{Voronov}. Moreover, for any $T\in C^1_{\beta,\alpha}(V,\mathfrak{g})$, i.e. $T:V\rightarrow \mathfrak{g}$ is a linear map satisfying $T\circ \beta=\alpha\circ T$, we have $$\{\!\!\{T,T\}\!\!\}(v_1,v_2)=2\Big(T\{Tv_1,v_2\}-T\{Tv_1,v_2\}-[Tv_1,Tv_2]\Big),\quad \mbox{for all~~} v_1, v_2\in V.$$ 
In turn, it follows that $\{\!\!\{T,T\}\!\!\}=0$ if and only if $T:V\rightarrow \mathfrak{g}$ is an $\mathcal{O}$-operator on hom-Lie algebra $(\mathfrak{g},[~,~],\alpha)$ with respect to the representation $(V,\beta,\rho)$. 
Thus, we have the following theorem generalizing the Lie algebra case \cite{Sheng3}.

\begin{theorem}\label{Maurer Cartan element}
The graded vector space $C^*_{\beta,\alpha}(V,\mathfrak{g})$ forms a graded Lie algebra with the graded Lie bracket $\{\!\!\{-,-\}\!\!\}$. A linear map $T:V\rightarrow \mathfrak{g}$ satisfying $T\circ \beta=\alpha\circ T$ is an $\mathcal{O}$-operator on hom-Lie algebra $(\mathfrak{g},[~,~],\alpha)$ with respect to the representation $(V,\beta,\rho)$ if and only if $T\in C^1_{\beta,\alpha}(V,\mathfrak{g})$ is a Maurer-Cartan element of the graded Lie algebra $(C^*_{\beta,\alpha}(V,\mathfrak{g}),\{\!\!\{-,-\}\!\!\})$.
\end{theorem}

\begin{remark}\label{dgla}
Let $T:V\rightarrow \mathfrak{g}$ be an $\mathcal{O}$-operator on hom-Lie algebra $(\mathfrak{g},[~,~],\alpha)$ with respect to the representation $(V,\beta,\rho)$. From Theorem \ref{Maurer Cartan element}, $T\in C^1_{\beta,\alpha}(V,\mathfrak{g})$ is a Maurer-Cartan element of the graded Lie algebra $(C^*_{\beta,\alpha}(V,\mathfrak{g}),\{\!\!\{-,-\}\!\!\})$. Then, the $\mathcal{O}$-operator $T$ induces a differential $\delta_T:=\{\!\!\{T,-\}\!\!\}$ on the graded Lie algebra $(C^*_{\beta,\alpha}(V,\mathfrak{g}),\{\!\!\{-,-\}\!\!\})$, which makes it a differential graded Lie algebra.  
\end{remark}

By remark \ref{dgla}, we associate a cochain complex $(C^*_{\beta,\alpha}(V,\mathfrak{g}),\delta_T)$ to an $\mathcal{O}$-operator $T$ on a hom-Lie algebra $(\mathfrak{g},[~,~],\alpha)$ with respect to the representation $(V,\beta,\rho)$. The cohomology of this cochain complex is called the cohomology of the $\mathcal{O}$-operator $T$. 
 

\subsection{Cohomology of $\mathcal{O}$-operators in terms of hom-Lie algebra cohomology}
Now, we describe the cohomology of an $\mathcal{O}$-operator on hom-Lie algebras in terms of hom-Lie algebra cohomology of certain hom-Lie algebra with coefficients in a representation. 

%For this purpose, we first show that any $\mathcal{O}$-operator on a hom-Lie algebra $(\mathfrak{g},[~,~],\alpha)$ with respect to representation $(V,\beta,\rho)$ induces a hom-pre-Lie algebra structure $(V,\cdot_T,\beta)$. We denote by $V^c_\beta$, the sub-adjacent hom-Lie algebra of the induced hom-pre-Lie algebra. We prove that the operator $T$ also induces a representation $\rho_T$ of hom-Lie algebra $V^c_{\beta}$ on $\mathfrak{g}$ with respect to the map $\alpha$. Finally, we show that the complex $(C^*_{\beta,\alpha}(V,\mathfrak{g}),\delta_T)$ coincides with the cochain complex of hom-Lie algebra with coefficients in the representation $(\mathfrak{g},\alpha,\rho_T)$.   

Let us recall that a hom-pre-Lie algebra is a triplet $(V,\cdot,\beta)$, where $V$ is a vector space equipped with a bilinear map $\cdot: V\otimes V\rightarrow V$ and a linear map $\beta: V\rightarrow V$ such that 
\begin{align*}
\beta(u\cdot v)&=\beta(u)\cdot \beta(v), \quad \mbox{and}\\
(u\cdot v)\cdot  \beta(w) - \beta(u)\cdot  (v\cdot  w) &= (v\cdot u)\cdot \beta(w) - \beta(v)\cdot (u\cdot w),\quad\mbox{for all } u,v,w\in V.
\end{align*}

An $\mathcal{O}$-operator on a hom-Lie algebra induces a hom-pre-Lie algebra. In particular, we have the following straightforward proposition.

\begin{proposition}\label{induced hom-pre-Lie algebra}
Let $T:V\rightarrow \mathfrak{g}$ be an $\mathcal{O}$-operator on the hom-Lie algebra $(\mathfrak{g},[~,~],\alpha)$ with respect to the representation $(V,\beta, \rho)$. Then, the $\mathcal{O}$-operator induces a hom-pre-Lie algebra $(V,\cdot_T,\beta )$, where $\cdot_T$ is given by
$$v\cdot_T w= \{Tv,w\}, \quad \mbox{for all  }v, w\in V. $$
\end{proposition}



If $(V,\cdot,\beta)$ is a hom-pre-Lie algebra, then the commutator bracket $[v,w]^c=v\cdot w-w\cdot v$ gives a hom-Lie algebra structure $V^c_{\beta}:=(V,[~,~]^c,\beta)$. It is called the sub-adjacent hom-Lie algebra of the hom-pre-Lie algebra $(V,\cdot,\beta)$.

\begin{proposition}\label{rep associated to O-operator}
Let $T:V\rightarrow \mathfrak{g}$ be an $\mathcal{O}$-operator on the hom-Lie algebra $(\mathfrak{g},[~,~],\alpha)$ with respect to the representation $(V,\beta, \rho)$. Let us define a map $\rho_T:V\rightarrow \mathsf{End}(\mathfrak{g})$ given by
$$\rho_T(v)(x):=[Tv,x]+T\{x,v\},\quad \mbox{for all  }v\in V~~\mbox{and  } x\in\mathfrak{g}.$$
Then, the triplet $(\mathfrak{g},\alpha,\rho_T)$ is a representation of the sub-adjacent hom-Lie algebra $V^c_{\beta}$.
\end{proposition}
\begin{proof}
First, let us show that $\rho_T(\beta(v))(\alpha(x))= \alpha(\rho_T(v)(x))$. The required identity holds by using the facts that 
$$T\circ \beta=\alpha\circ T~~~\mbox{ and }~~~\{\alpha(x),\beta(v)\}=\beta\{x,v\},\quad\mbox{for all } x\in\mathfrak{g},~v\in V.$$ 
In fact,
\begin{align*}
\rho_T(\beta(v))(\alpha(x))&=[T(\beta (v)),\alpha (x)]+T\{\alpha(x),\beta(v)\}\\
&=\alpha([Tv,x]+T\{x,v\})\\
&=\alpha(\rho_T(v)(x)).
\end{align*}
Next, we use the properties of an $\mathcal{O}$-operator  to obtain the following expressions:
\begin{align}\label{rep:eq1}
\rho_T([v,w]^c)(\alpha(x))&=[T\big(\{Tv,w\}-\{Tw,v\}\big),\alpha(x)]+T\{\alpha(x),\{Tv,w\}-\{Tw,v\}\}\\\nonumber
&=[[Tv,Tw],\alpha(x)]+T\{\alpha(x),\{Tv,w\}-\{Tw,v\}\}
\end{align}
\begin{align}\label{rep:eq2}
\rho_T(\beta(v))\rho_T(w)(x)&=\rho_T(\beta(v))([Tw,x]+T\{x,w\})\\\nonumber
&=[T(\beta(v)),[Tw,x]+T\{x,w\}]+T\{[Tw,x]+T\{x,w\},\beta(v)\}\\\nonumber
&=[T(\beta(v)),[Tw,x]] + T\big(\{T(\beta(v)),\{x,w\}\}\big)-T\big(\{T(\{x,w\}),\beta(v)\}\big)\\\nonumber
&\quad+T\{[Tw,x]+T\{x,w\},\beta(v)\}\\\nonumber
&=[\alpha(T(v)),[Tw,x]]+T\big(\{\alpha(T(v)),\{x,w\}\}\big)+T\{[Tw,x],\beta(v)\}
\end{align}
Similarly,
\begin{equation}\label{rep:eq3}
\rho_T(\beta(w))\rho_T(v)(x)=[\alpha(T(w)),[Tv,x]]+T\big(\{\alpha(T(w)),\{x,v\}\}\big) +T\{[Tv,x],\beta(w)\}
\end{equation}

Since the map $\rho:\mathfrak{g}\rightarrow \mathsf{End}(V)$ (denoted by $\{x,v\}:=\rho(x)(v)$) is a representation of the hom-Lie algebra $(\mathfrak{g},[~,~],\alpha)$, it follows that 
\begin{align}
\{\alpha(x),\{Tv,w\}\}-\{\alpha(T(v)),\{x,w\}\}&=\{[x,Tv],\beta(w)\},\\\label{rep:eq5}
\{\alpha(x),\{Tw,v\}\}-\{\alpha(T(w)),\{x,v\}\}&=\{[x,Tw],\beta(v)\}.
\end{align}
Hence, by using equations \eqref{rep:eq1}-\eqref{rep:eq5}, we get
$$\rho_T([v,w]^c)(\alpha(x))= \rho_T(\beta(v))\rho_T(w)(x)-\rho_T(\beta(v))\rho_T(w)(x), ~~\mbox{for all } v,w \in V~~\mbox{and } x\in \mathfrak{g}.$$
Thus, the triplet $(\mathfrak{g},\alpha,\rho_T)$ is a representation of sub-adjacent hom-Lie algebra $V^c_{\beta}$.
\end{proof}

With the above notations, let us consider the cochain complex $(C^*_{\beta,\alpha}(V,\mathfrak{g}),\delta_{\beta,\alpha})$ of the sub-adjacent hom-Lie algebra $V^c_{\beta}$ with coefficients in the representation $(\mathfrak{g},\alpha,\rho_T)$. Recall that the differential $\delta_{\beta,\alpha}$ is given by 
\begin{align}\label{coboundary:subadjacent hom-Lie algebra}
&\delta_{\beta,\alpha} P(v_1,v_2,\ldots,v_{n+1})\\\nonumber
= &\sum_{i=1}^{n+1}(-1)^{i+1}\rho_T(\beta^{n-1}(v_i))\big( P(v_1,v_2,\ldots,\hat{v_i},\ldots,v_{n+1})\big)\\\nonumber
&+\sum_{ i<j }(-1)^{i+j}P([v_i,v_j]^c,\beta(v_1),\ldots,\hat{\beta(v_i)},\ldots,\hat{\beta(v_j)},\ldots,\beta(v_{n+1}))
\end{align}

Thus, there are two different differentials $\delta_\beta$ and $\delta_T:=\{\!\!\{T,- \}\!\!\}$ on the graded vector spaces $C^*_{\beta,\alpha}(V,\mathfrak{g})$. The following proposition shows that both of these differentials yield the same cohomology.

\begin{proposition}
Let $T:V\rightarrow \mathfrak{g}$ be an $\mathcal{O}$-operator on the hom-Lie algebra $(\mathfrak{g},[~,~],\alpha)$ with respect to the representation $(V,\beta, \rho)$. Then, the differentials $\delta_{\beta}$ and $\{\!\!\{T,- \}\!\!\}$ on $C^*_{\beta,\alpha}(V,\mathfrak{g})$ are related by 
$$\delta_T(P)=(-1)^n\{\!\!\{T,P \}\!\!\},\quad \mbox{for any } ~~P\in C^n_{\beta,\alpha}(V,\mathfrak{g})\mbox{ and } n\geq 1.$$
\end{proposition}

\begin{proof} For $P \in C^n_{\beta,\alpha}(V,\mathfrak{g})$ and $n\geq 1$, we have
\begin{align*}
&\{\!\!\{T,P \}\!\!\}(v_1,v_2,\ldots,v_{n+1})\\
=&\sum_{\tau\in S_{n,1}}(-1)^{|\tau|}~ T\Big(\{P(v_{\tau(1)},\ldots,v_{\tau(n)}),\beta^{n-1}(u_{\tau(n+1)})\}\Big)\\
&+(-1)^{n}\Bigg(\sum_{\tau\in S_{1,n}}(-1)^{|\tau|}~\big[\alpha^{n-1}T(v_{\tau(1)}),P(v_{\tau(2)},\ldots,v_{\tau(n+1)})\big]\\
& -\sum_{\tau\in S_{1,1,n-1}}(-1)^{|\tau|}~ P\Big(\{T(v_{\tau(1)}),(u_{\tau(2)})\},\beta(v_{\tau(3)}),\ldots,\beta(v_{\tau(n+1)})\Big)\Bigg)\\
=&(-1)^n\bigg(\sum_{i=1}^{n+1}(-1)^{i+1}T\{P(v_1,\ldots,\hat{v_i},\ldots,v_{n+1}),\beta^{n-1}(v_i)\}\\
&+\sum_{i=1}^{n+1}(-1)^{i+1}[\alpha^{n-1}\big(T(v_i)\big), P(v_1,\ldots,\hat{v_i},\ldots,v_{n+1})]\\\nonumber
&+\sum_{ i<j }(-1)^{i+j}P\big(\{T(v_i),v_j\}-\{T(v_j),v_i\},\beta(v_1),\ldots,\hat{\beta(v_i)},\ldots,\hat{\beta(v_j)},\ldots,\beta(v_{n+1})\big)\bigg)\\
= &\sum_{i=1}^{n+1}(-1)^{i+1}\rho_T(\beta^{n-1}(v_i))\big( P(v_1,v_2,\ldots,\hat{v_i},\ldots,v_{n+1})\big)\\
&+\sum_{ i<j }(-1)^{i+j}P([v_i,v_j]^c,\beta(v_1),\ldots,\hat{\beta(v_i)},\ldots,\hat{\beta(v_j)},\ldots,\beta(v_{n+1}))\\
=&(-1)^n\delta_T(P)(v_1,v_2,\ldots,v_{n+1}).
\end{align*}

\end{proof}

%\section{Deformations of $\mathcal{O}$-operators on hom-Lie algebras} 
%In this section, we discuss one-parameter deformations of $\mathcal{O}$-operators on a hom-Lie algebra. We interpret both the linear and formal (one-parameter) deformations in terms of the cohomology associated to $\mathcal{O}$-operators. 

%\subsection{Linear deformations}
 %Let $T: V\rightarrow \mathfrak{g}$ be an $\mathcal{O}$-operator on a hom-Lie algebra $(\mathfrak{g},[~,~],\alpha)$ with respect to a representation $(V,\beta,\rho)$. A linear deformation of $T$ is given by a map 
 %$$T_t:=T+t\mathfrak{T}:V\rightarrow \mathfrak{g},\quad \mbox{for some } \mathfrak{T}\in C^1_{\beta,\alpha}(V,\mathfrak{g}),$$
 %such that $T_t $ is an $\mathcal{O}$-operator on a hom-Lie algebra $(\mathfrak{g},[~,~],\alpha)$ with respect to a representation $(V,\beta,\rho)$. 


\section{\normalfont\large\textbf{Deformation of $\mathcal{O}$-operators on regular hom-Lie algebras}}
In this section, we discuss linear and formal one-parameter deformations of $\mathcal{O}$-operators on hom-Lie algebras. In this section, we always assume hom-Lie algebras to be regular and the endomorphism in the representations to be an isomorphism. 

\subsection{Deformation complex of an $\mathcal{O}$-operator on regular hom-Lie algebras}
 We define a deformation complex of $\mathcal{O}$-operators on regular hom-Lie algebras. However, we will see that to get suitable deformation cohomology; we need to define the space of $0$-cochains. For this purpose, we consider a regular hom-Lie algebra $(\mathfrak{g},[~,~],\alpha)$ with a representation $(V,\beta,\rho)$, where $\beta:V\rightarrow V$ is a vector space isomorphism. In this case, an $\mathcal{O}$-operator $T:V\rightarrow \mathfrak{g}$ induces a regular hom-pre-Lie algebra $(V,\cdot_T,\beta)$ and hence, the sub-adjacent hom-Lie algebra $V_{\beta}^c$ is also regular.
Subsequently, from Remark \ref{regular hom-Lie algebra cohomology} that we have a modified cochain complex 
$$\bigg(\widetilde{C}^*_{\beta,\alpha}(V,\mathfrak{g}):=\bigoplus_{n\geq 0}~C^n_{\beta,\alpha}(V,\mathfrak{g}),~\delta_{\beta,\alpha}\bigg),$$
where the space of $0$-cochains are given by 
$$C^0_{\beta,\alpha}(V,\mathfrak{g}):=\{x\in \mathfrak{g}| \alpha(x)=x\},$$
and the differential $\delta_{\beta,\alpha}:C^0_{\alpha,\beta}(V,\mathfrak{g})\rightarrow C^1_{\alpha,\beta}(V,\mathfrak{g})$ is defined by 
\begin{equation}\label{diff_0:1}
\delta_{\beta,\alpha}(x)(v):=\rho_T(\beta^{-1}(v))(x)\quad\mbox{for }v\in V,~x\in C^0_{\beta,\alpha}(V,\mathfrak{g}).
\end{equation}

Note that in this case we can extend the bracket $\{\!\!\{-,- \}\!\!\}$ defined by equation \eqref{derived bracket} to $\widetilde{C}^*_{\beta,\alpha}(V,\mathfrak{g})$. In particular, for $x,y\in\mathfrak{g},$ and $P\in C^n_{\beta,\alpha}(V,\mathfrak{g})$, the bracket $\{\!\!\{-,- \}\!\!\}$ is given by 
\begin{align}
\nonumber
\{\!\!\{x,y\}\!\!\}=&[x,y],\\\nonumber
\{\!\!\{P,x\}\!\!\}(v_1,v_2,\ldots,v_{n})=&\sum_{\tau\in S_{1,n-1}}(-1)^{|\tau|}~ P\Big(\{x,\beta^{-1}(v_{\tau(1)})\},v_{\tau(2)},\ldots, v_{\tau(n)} \Big)\\\nonumber
&+[\alpha^{-1}P(v_{1},\ldots,v_{n}),\alpha^{n-1}(x)].
\end{align}
In particular, 
\begin{equation}\label{diff_0:2}
\{\!\!\{T,x\}\!\!\}(v)= T(\{x,\beta^{-1}(v)\})+[\alpha^{-1}T(v),(x)].
\end{equation}

By equations \eqref{diff_0:1} and \eqref{diff_0:1}, it is clear that $\delta_{\beta,\alpha}$ coincides with $\delta_T$ at $0$-degree elements in $\widetilde{C}^*_{\beta,\alpha}(V,\mathfrak{g})$. i.e.,  
\begin{align*}
\delta_{\beta,\alpha}(x)(v)=&\rho_T(\beta^{-1}(v))(x)\\
=&[T(\beta^{-1}(v)),x]+T\{x,\beta^{-1}(v)\}\\
=& [\alpha^{-1}T(v),(x)] + T(\{x,\beta^{-1}(v)\})\\
=&\{\!\!\{T,x\}\!\!\}(v),
\end{align*}
for all $v\in V,~x\in C^0_{\beta,\alpha}(V,\mathfrak{g})$. Therefore, the cohomologies of the complexes $(\widetilde{C}^*_{\beta,\alpha}(V,\mathfrak{g}),\delta_{\beta,\alpha})$ and $(\widetilde{C}^*_{\beta,\alpha}(V,\mathfrak{g}),\delta_T)$ are the same and denoted by $\widetilde{H}^*_{\beta,\alpha}(V,\mathfrak{g})$. In the sequel, we show that the cohomology $\widetilde{H}^*_{\beta,\alpha}(V,\mathfrak{g})$ is deformation cohomology for an $\mathcal{O}$-operator on the regular hom-Lie algebra $(\mathfrak{g},[~,~],\alpha)$ with respect to a representation $(V,\beta,\rho)$. 



\subsection{Linear deformations}
Let $T:V\rightarrow \mathfrak{g}$ be an $\mathcal{O}$-operator on a hom-Lie algebra $(\mathfrak{g},[~,~],\alpha)$  with respect to a representation $(V,\beta,\rho)$. Let us consider a linear sum $T_t:=T+t\mathfrak{T}$ for some element $\mathfrak{T}\in C^1_{\beta,\alpha}(V,\mathfrak{g})$. If $T_t$ is an $\mathcal{O}$-operator on the hom-Lie algebra $(\mathfrak{g},[~,~],\alpha)$ with respect to the representation $(V,\beta,\rho)$, then $T_t$ is called a linear deformation of $T$ generated by the element $\mathfrak{T}$. The map $T_t=T+t\mathfrak{T}$ is a linear deformation of $T$ if it satisfies the following identities
\begin{align*}
T_t\circ \beta&=\alpha\circ T_t,\\
[T_t(v),T_t(w)]&=T_t\big(\{T_t(v),w\}-\{T_t(w),v\}\big),\quad\mbox{for all }v,w\in V.
\end{align*}
Equivalently, 
\begin{equation}\label{commuting condition}
\mathfrak{T}\circ \beta =\alpha\circ \mathfrak{T}
\end{equation}
and for all $v,w \in V$, we get
\begin{equation}\label{cocycle condition for linear def}
[T(v),\mathfrak{T}(w)]+[\mathfrak{T}(v),T(w)]=T\big(\{\mathfrak{T}(v),w\}-\{\mathfrak{T}(w),v\}\big)+\mathfrak{T}\big(\{T(v),w\}-\{T(w),v\}\big),
\end{equation} 
\begin{equation}\label{O-operator condition for generator}
[\mathfrak{T}(v),\mathfrak{T}(w)]=\mathfrak{T}\big(\{\mathfrak{T}(v),w\}-\{\mathfrak{T}(w),v\}\big).
\end{equation}
Thus, $T_t$ is a linear deformation of $T$ if and only if conditions \eqref{commuting condition}-\eqref{O-operator condition for generator} hold. Observe that condition \eqref{cocycle condition for linear def} implies that $\delta_T(\mathfrak{T})=0$. Moreover, it follows from \eqref{commuting condition} and \eqref{O-operator condition for generator} that the map $\mathfrak{T}$ is an $\mathcal{O}$-operator on the hom-Lie algebra $(\mathfrak{g},[~,~],\alpha)$ with respect to the representation $(V,\beta,\rho)$.

\begin{definition}
Two linear deformations $T_t^1:=T+t\mathfrak{T}_1$ and $T_t^2:=T+t\mathfrak{T}_2$ are said to be equivalent if there exist an element $x\in \mathfrak{g}$ such that $\alpha(x)=x$ and the pair $(\mathsf{Id}_{\mathfrak{g}}+t\mathsf{ad}^\dagger_x,\mathsf{Id}_V+t\rho(x)^\dagger)$ is a homomorphism of $\mathcal{O}$-operators from $T_t^1$ to $T_t^2$. 
\end{definition}
Let us recall from Definition \ref{morphism of O-operators} that the pair $(\mathsf{Id}_{\mathfrak{g}}+t\mathsf{ad}^\dagger_x,\mathsf{Id}_V+t\rho(x)^\dagger)$ is a homomorphism of $\mathcal{O}$-operators from $T_t^1$ to $T_t^2$ if the following conditions are satisfied
\begin{enumerate}[label=(\roman*)]
\item The map $(\mathsf{Id}_{\mathfrak{g}}+t\mathsf{ad}^\dagger_x)$ is a hom-Lie algebra homomorphism,
\item $\beta\circ(\mathsf{Id}_{\mathfrak{g}}+t\rho(x)^\dagger)=(\mathsf{Id}_{\mathfrak{g}}+t\rho(x)^\dagger)\circ\beta,$ 
\item $(T+t\mathfrak{T}_2)\circ(\mathsf{Id}_V+t\rho(x)^\dagger)=(\mathsf{Id}_{\mathfrak{g}}+t\mathsf{ad}_x^\dagger)\circ(T+t\mathfrak{T}_1),$ 
\item $\rho\big((\mathsf{Id}_{\mathfrak{g}}+t\mathsf{ad}_x^\dagger)(y)\big)\big((\mathsf{Id}_{\mathfrak{g}}+t\rho(x)^\dagger)(v)\big)=(\mathsf{Id}_{\mathfrak{g}}+t\rho(x)^\dagger)\big(\rho(y)(v)\big),$ for all $y\in \mathfrak{g}$ and $v\in V$.
\end{enumerate}
From the condition (i), 
$$(\mathsf{Id}_{\mathfrak{g}}+t\mathsf{ad}_x^\dagger)[y,z]=[(\mathsf{Id}_{\mathfrak{g}}+t\mathsf{ad}_x^\dagger)(y),(\mathsf{Id}_{\mathfrak{g}}+t\mathsf{ad}_x^\dagger)(z)],\quad\mbox{for all }y,z\in \mathfrak{g}.$$
On comparing the coefficients of $t^2$ from both sides of the above identity, we get 
$$[[x,\alpha^{-1}(y)],[x,\alpha^{-1}(z)]]=0,\quad\mbox{for all }y,z\in\mathfrak{g}.$$ 
Clearly, invertibility of $\alpha$ implies that the element $x$ satisfies
\begin{equation}\label{equ-linear:cond1}
[[x,y],[x,z]]=0,\quad\mbox{for all }y,z\in\mathfrak{g}.
\end{equation}
Since $\alpha(x)=x$, the condition (ii) holds. It easily follows that the condition (iii) is equivalent to the following identities
\begin{equation}\label{equ-linear:cond2}
\mathfrak{T}_1(v)-\mathfrak{T}_2(v)=T\rho(x)(\beta^{-1}(v))+[T\beta^{-1}(v),x]=\delta_T(x)(v),
\end{equation}
\begin{equation}\label{equ-linear:cond3}
\mathfrak{T}_2\rho(x)(v)=[x,\mathfrak{T}_1(v)],\quad\mbox{for all }v\in V.
\end{equation}
The last condition (iv) implies that the element $x$ also satisfies
$$\rho([x,\alpha^{-1}(y)])\rho(x)(\beta^{-1}(v))=0, \quad\mbox{for all } y\in \mathfrak{g}, ~v\in V,$$
equivalently, by using invertibility of $\alpha$ and $\beta$, we have
\begin{equation}\label{equ-linear:cond4}
\rho([x,y])\rho(x)(v)=0, \quad\mbox{for all } y\in \mathfrak{g}, ~v\in V.
\end{equation}
\begin{theorem}
Let $T:V\rightarrow \mathfrak{g}$ be an $\mathcal{O}$-operator. Let $T_t^1:=T+t\mathfrak{T}_1$ and $T_t^1:=T+t\mathfrak{T}_1$ be two equivalent linear deformations of $T$. Then, $\mathfrak{T}_1$ and $\mathfrak{T}_2$ belongs to the same cohomology class in $H^1_{\beta,\alpha}(V,\mathfrak{g})$.
\end{theorem}
\begin{proof}
The proof follows from the equation \eqref{equ-linear:cond2}.
\end{proof}
 
\begin{definition}
Let $T: V\rightarrow \mathfrak{g}$ be an $\mathcal{O}$-operator on a hom-Lie algebra $(\mathfrak{g},[~,~],\alpha)$ with respect to a representation $(V,\beta,\rho)$. A linear deformation $T_t:T+t\mathfrak{T}$ is said to be trivial if it is equivalent to the deformation $T_0=T$.
\end{definition}

\begin{definition}
 Let $T: V\rightarrow \mathfrak{g}$ be an $\mathcal{O}$-operator on a hom-Lie algebra $(\mathfrak{g},[~,~],\alpha)$ with respect to a representation $(V,\beta,\rho)$. An element $x\in \mathfrak{g}$ is called a Nijenhuis element associated to the operator $T$ if $x$ satisfies $\alpha(x)=x$, the identities \eqref{equ-linear:cond1}, \eqref{equ-linear:cond4}, and the identity
$$[x,T\rho(x)(v)+[T(v),x]]=0,\quad \mbox{for all} v\in V.$$ 
\end{definition}
Let us denote the set of Nijenhuis elements associated to the $\mathcal{O}$-operator $T$ by $\mathsf{Nij}(T)$.

\begin{theorem}\label{trivial deformations}
Let $T: V\rightarrow \mathfrak{g}$ be an $\mathcal{O}$-operator on a hom-Lie algebra $(\mathfrak{g},[~,~],\alpha)$ with respect to a representation $(V,\beta,\rho)$. For any element $x\in \mathsf{Nij}(T),$ the linear deformation $T_t:T+t\mathfrak{T}$ generated by $\mathfrak{T}:=\delta_T(x)$ is a trivial deformation of $T$.
\end{theorem}
\begin{proof}
First, we need to show that $T_t:T+t\mathfrak{T}$ is a linear deformation generated by $\mathfrak{T}:=\delta_{\beta,\alpha}(x)$, where $x\in\mathsf{Nij}(T)$. For this purpose, we need to show that $\mathfrak{T}$ satisfies the equations \eqref{commuting condition}, \eqref{cocycle condition for linear def}, and \eqref{O-operator condition for generator}. By definition of $\delta_{\beta,\alpha}$ at $0$-cochains it is clear that 
$$\alpha\circ \mathfrak{T}(v)=\alpha\circ \delta_{\beta,\alpha}(x)(v)=\delta_{\beta,\alpha}(x)(\beta(v))=\mathfrak{T}(\beta(v)), \quad\mbox{for all }v\in V.$$
i.e., $\mathfrak{T}$ satisfies the equation \eqref{commuting condition}. Since $\mathfrak{T}=\delta_{\beta,\alpha}(x)$, the equation \eqref{cocycle condition for linear def} holds trivially. Moreover, by using the identity $\alpha(x)=x$ and a straightforward calculation similar to the Lie algebra case in \cite{Sheng3}, it follows that $\mathfrak{T}$ satisfies the equation \eqref{O-operator condition for generator}. 

Now, we need to show that the linear deformation $T_t$ is trivial. Since $x\in \mathsf{Nij}(T)$, it immediately follows that the pair $(\mathsf{Id}_\mathfrak{g}+t\mathsf{ad}_x^{\dagger},\mathsf{Id}_V+t\rho(x)^\dagger)$ is a homomorphism of $\mathcal{O}$-operators from $T_t$ to $T$. 
\end{proof}
\subsection{Formal deformations}
Let $(\mathfrak{g},[~,~],\alpha)$ be a hom-Lie algebra  with a representation $(V,\beta,\rho)$. Let $k[[t]]$ be the formal power series ring in one variable $t$ and $\mathfrak{g}[[t]]$ be the formal power series in $t$ with coefficients in $\mathfrak{g}$. Then the triplet $(\mathfrak{g}[[t]],[~,~]_t,\alpha_t)$ is a hom-Lie algebra, where the bracket $[~,~]_t$ and the structure map $\alpha_t$ are obtained by extending $[~,~]$ and the map $\alpha$ linearly over the ring $k[[t]]$. Moreover, the map $\rho:\mathfrak{g}\rightarrow \mathsf{End}(V)$ and $\beta:V\rightarrow V$ can be extended linearly over $k[[t]]$ to obtain $k[[t]]$-linear maps $\rho_t:\mathfrak{g}[[t]]\otimes V[[t]] \rightarrow V[[t]]$ and $\beta_t:V[[t]]\rightarrow V[[t]]$. Then the triplet $(V[[t]],\beta_t,\rho_t)$ is a hom-Lie algebra representation of $(\mathfrak{g}[[t]],[~,~]_t,\alpha_t)$.      

\begin{definition}
 Let $T: V\rightarrow \mathfrak{g}$ be an $\mathcal{O}$-operator on a hom-Lie algebra $(\mathfrak{g},[~,~],\alpha)$ with respect to a representation $(V,\beta,\rho)$. A formal deformation of $T$ is given by 
$$\textstyle{T_t=T_0+\sum\limits_{i\geq1}t^i T_i }, \quad\mbox{ with }  T_0=T,~T_i\in C^1_{\beta,\alpha}(V,\mathfrak{g})$$ 
such that $T_t:V[[t]]\rightarrow\mathfrak{g}[[t]]$ is an $\mathcal{O}$-operator on $(\mathfrak{g}[[t]],[~,~]_t,\alpha_t)$ with respect to the representation $(V[[t]],\beta_t,\rho_t)$. 
\end{definition}
Equivalently, 
\begin{equation}\label{fdef:eq1}
T_t(\beta(v))=\alpha (T_t(v)),
\end{equation}
\begin{equation}\label{fdef:eq2}
[T_t v,T_t w]=T_t\big(\{T_t v, w\}-\{T_t w,v\}\big),\quad\mbox{ for all }  u,v\in V.
\end{equation}
Note that condition \eqref{fdef:eq1} holds trivially since $T_i\in C^1_{\beta,\alpha}(V,\mathfrak{g}),$ for all $i\geq 0$. For $k\geq 0$, if we compare the coefficients of $t^k$ from both sides of equation \eqref{fdef:eq2}, then we obtain the following system of equations
\begin{equation}\label{fdef:system}
\sum\limits_{i+j=k}[T_i v,T_j w]=\sum\limits_{i+j=k}T_i\big(\{T_j v, w\}-\{T_j w,v\}\big),\quad\mbox{ for } k=0,1,2,\ldots.
\end{equation}

The 1-cochain $T_1\in C^1_{\beta,\alpha}(V,\mathfrak{g})$ is called the infinitesimal of the deformation $T_t$. More generally, if $T_i=0$ for $1\leq i\leq (n-1)$ and $T_n$ is a non-zero cochain, then $T_n$ is called the $n$-infinitesimal of the deformation $T_t$. From equation \eqref{fdef:system}, the case $k=1$ yields the expression
$$[T_1 v,T w]+[T v,T_1 w]=T_1\big(\{T v, w\}-\{T w,v\}\big)+T\big(\{T_1 v, w\}-\{T_1 w,v\}\big),\quad\mbox{ for all }  v,w\in V,$$
which is equivalent to the condition: $\delta_T(T_1)=0$. Therefore, we get the following proposition.

\begin{proposition}\label{infinitesimal}
The infinitesimal of the deformation $T_t$ is a $1$-cocycle in the cohomology of the $\mathcal{O}$-operator $T$. More generally, the $m$-infinitesimal is a $1$-cocycle. 
\end{proposition}

Now, we consider the equivalence of two formal deformations of an $\mathcal{O}$-operator. The definition is motivated from the Lie algebra case \cite{Sheng3}.
\begin{definition}\label{Def:equivalence}
Two deformations $T_t$ and $\overline{T}_t$ of the $\mathcal{O}$-operator $T$ are said to be equivalent if there exists an element $x\in C^{0}_{\beta,\alpha}(V,\mathfrak{g})$, $k$-linear maps $\phi^{\mathfrak{g}}_i:\mathfrak{g}\rightarrow\mathfrak{g}$ and $\phi^V_i:V\rightarrow V$, for $i\geq 2$, such that the pair $(\phi^{\mathfrak{g}}_t,\phi^V_t),$ consisting of
$$\phi^{\mathfrak{g}}_t=\mathsf{Id}_{\mathfrak{g}}+t(\mathsf{ad}^{\dagger}_x)+\sum_{i\geq 2}t^i \phi^{\mathfrak{g}}_i \quad \mbox{and}\quad \phi^V_t=\mathsf{Id}_{V}+t\rho(x)^{\dagger}+\sum_{i\geq 2}t^i \phi^V_i,$$
is a formal isomorphism from $T_t$ to ${T}^{\prime}_t$. Here, for $x\in C^{0}_{\beta,\alpha}(V,\mathfrak{g})$, the maps $\mathsf{ad}_x^{\dagger}:\mathfrak{g}\rightarrow \mathfrak{g}$ and $\rho(x)^{\dagger}:V\rightarrow V$ are given by 
$$\mathsf{ad}^{\dagger}_x(y):=\alpha^{-1}\big(\mathsf{ad}_x(y)\big)\quad\mbox{and  }\rho(x)^{\dagger}(v):=\beta^{-1}\big(\rho(x)(v)\big),\quad\mbox{for }y\in \mathfrak{g}, ~v\in V.$$  
\end{definition}

With the above notations, for all $y,z\in \mathfrak{g}$ and $v\in V$, the equivalence of two deformations $T_t$ and $\overline{T}_t$ gives the following conditions
\begin{enumerate}
\item $\phi^{\mathfrak{g}}_t\circ T_t=\overline{T}_t\circ \phi^{V}_t,$\\\vspace{-4.5mm}
\item $\phi^{\mathfrak{g}}_t[y,z]=[\phi^{\mathfrak{g}}_t(y),\phi^{\mathfrak{g}}_t(z)]$,\\\vspace{-4mm}
\item $\rho_t(\phi^{\mathfrak{g}}_t(y))(\phi^V_t(v))=\phi^V_t\big(\rho(y)(v)\big)$,\\\vspace{-4.5mm}
\item $\phi^{\mathfrak{g}}_t\circ \alpha=\alpha\circ\phi^{\mathfrak{g}}_t~~$ and $~~\phi^V_t\circ\beta=\beta\circ\phi^V_t$.
\end{enumerate}

On comparing the coefficients of $t$ from both sides of the condition $(1)$, we get 
\begin{align*}
T_1(v)-\overline{T}_1(v)=&T\beta^{-1}(\{x,v\})-\alpha^{-1}[x,Tv]\\
=&T(\{x,\beta^{-1}(v)\})+[T(\beta^{-1}(v)),x]\\
=&\rho_T(\beta^{-1}(v))(x)=(\delta_{\beta,\alpha}(x))(v)            
\end{align*}
Consequently, we obtain the following result.
\begin{proposition}
The infinitesimals of equivalent deformations belong to the same cohomology class in $\widetilde{H}^2_{\beta,\alpha}(V,\mathfrak{g})$.
\end{proposition}

\subsection{Obstructions in extending a finite order deformation to the next order}

Let a linear map $T:V\rightarrow \mathfrak{g}$ be an $\mathcal{O}$-operator on a hom-Lie algebra $(\mathfrak{g},[~,~],\alpha)$ with respect to a representation $(V,\beta,\rho)$. An order $n$ deformation of the $\mathcal{O}$-operator $T$ is given by a $k[[t]]/(t^{n+1})$-linear map 
$${T_t=T_0+\sum\limits^n_{i=1}~t^i ~T_i }, \quad\mbox{ with  }  T_0=T,~~T_i\in C^1_{\beta,\alpha}(V,\mathfrak{g})$$ 
such that 
$$[T_t(v),T_t(w)]=T_t(\{T_t(v),w\}-\{T_t(w),v\}) \quad\mbox{modulo}~~ t^{n+1}, \quad \mbox{for all }v,w\in V.$$
Equivalently,
$$\sum\limits_{\substack{i+j=k\\ ~i,j\geq 0}}\{\!\!\{T_i,T_j\}\!\!\}=0\quad\quad\mbox{for any }~~k=0,1,\ldots,n.$$

\begin{definition}
Let $\textstyle{T_t=\varphi_0+\sum^n_{i=1}t^i T_i }$ be an order $n$ deformation of the $\mathcal{O}$-operator $T$. We say that $T_t$ extends to a deformation of order $n+1$ if there exists a $1$-cochain $T_{n+1}\in C^1_{\beta,\alpha}(V,\mathfrak{g})$ such that $\widetilde{T}_t=T_t+t^{n+1} T_{n+1} $ is a deformation of order $n+1$.
\end{definition}

Let us observe that the map $\widetilde{T}_t:=T_t+t^{n+1}T_{n+1}$ is an extension of the order $n$ deformation $T_t$ if and only if 
$$\sum_{\substack{i+j=n+1\\i,j\geq 0}}\{\!\!\{T_i,T_j\}\!\!\}=0.$$


\begin{definition}
Let $T_t$ be an order $n$ deformation of the $\mathcal{O}$-operator $T$. Let us consider a $2$-cochain $\Theta_T \in C^2_{\beta,\alpha}(V,\mathfrak{g})$ defined as follows
\begin{equation}\label{Obst}
\Theta_T =-1/2\sum_{i+j=n+1;~i,j>0}\{\!\!\{T_i,T_j\}\!\!\}.
\end{equation}
The $2$-cochain $\Theta_F$ is called the obstruction cochain for extending the deformation $T_t$ of order $n$ to a deformation of order $n+1$. 
From equation \eqref{Obst} and using graded Jacobi identity of the bracket $\{\!\!\{-,-\}\!\!\}$, it follows that $\Theta_T$ is a $2$-cocycle.
\end{definition}

\begin{theorem}\label{hom-Obst}
Let $T_t$ be an order $n$ deformation of $T$. Then the deformation $T_t$ extends to a deformation of order $n+1$ if and only if the cohomology class of the $2$-cocycle $\Theta_T$ vanishes.

\begin{proof}
Let us assume that the order $n$ deformation $\textstyle{T_t=\varphi_0+\sum^n_{i=1}t^i T_i }$ extends to a deformation of order $n+1$. Then, there exists an element $T_{n+1}\in C^1_{\beta,\alpha}(V,\mathfrak{g})$ such that 
$\widetilde{T_t}=T_t+t^{n+1} T_{n+1}$ is an extension of the deformation $T_t$. Thus,  
$$\sum_{\substack{i+j=n+1\\ i,j\geq 0}}\{\!\!\{T_i,T_j\}\!\!\}=0,$$
i.e.,
$$\{\!\!\{T,T_{n+1}\}\!\!\}=-\frac{1}{2} \sum_{\substack{i+j=n+1\\ i,j> 0}}\{\!\!\{T_i,T_j\}\!\!\}.$$
It follows that $\Theta_T=\delta_T(T_{n+1})=\delta_{\beta,\alpha}(T_{n+1})$. Hence, the cohomology class of $\Theta_T$ vanishes.

Conversely, let us assume that $\Theta_T$ is a coboundary. So, there exists a $1$-cochain $T_{n+1}$ such that 
$$
\Theta_T==\delta_{\beta,\alpha}(T_{n+1}). 
$$
Define a map $\widetilde{T_t}:V\rightarrow \mathfrak{g}$ as follows
$$
\widetilde{T_t}=T_t+t^{n+1}T_{n+1}.
$$
Then,
$$\Theta_T=-1/2\sum_{\substack{i+j=n+1\\i,j>0}}\{\!\!\{T_i,T_j\}\!\!\}=\delta_T(T_{n+1}).$$
Since, $\delta_{\beta,\alpha}(T_{n+1})=\delta_T(T_{n+1})=\{\!\!\{T,T_{n+1}\}\!\!\}$, we obtain the following expression
$$\sum_{\substack{i+j=n+1\\i,j\geq 0}}\{\!\!\{T_i,T_j\}\!\!\}=0.$$ 
Therefore, the deformation $T_t$ of order $n$ extends to the deformation $\widetilde{T_t}$ of order $n+1$.
\end{proof}

\end{theorem}

\begin{corollary}
Let $T:V\rightarrow \mathfrak{g}$ be an $\mathcal{O}$-operator on a hom-Lie algebra $(\mathfrak{g},[~,~],\alpha)$ with respect to a representation $(V,\beta,\rho)$. If $H^2_{\beta,\alpha}(V,\mathfrak{g})=0$, then any $1$-cocycle in $C^1_{\beta,\alpha}(V,\mathfrak{g})$ is an infinitesimal of some formal deformation of the $\mathcal{O}$-operator $T$.
\end{corollary}

\section{Applications}
In this section, we describe deformations of $s$-Rota-Baxter operators (of weight $0$) and skew-symmetric $r$-matrices on hom-Lie algebras as particular cases of $\mathcal{O}$-operators on hom-Lie algebras. 
\subsection{Rota-Baxter Operators on hom-Lie algebras}
In this subsection, we always consider Rota Baxter operators of weight $0$. Let us recall the definition of $s$-Rota-Baxter operator on a hom-Lie algebra from definition \ref{def:Rota-Baxter operators}. In particular, for any non-negative integer $s$, a linear operator $\mathcal{R}: \mathfrak{g} \rightarrow \mathfrak{g}$ is called an $s$-Rota-Baxter operator of weight $0$ on a hom-Lie algebra $(\mathfrak{g},[~,~],\alpha)$ if $\mathcal{R}\circ \alpha= \alpha\circ \mathcal{R}$ and the following identity is satisfied
\begin{equation*} 
\quad\quad\quad\quad\quad[\mathcal{R}(x), \mathcal{R}(y)]= \mathcal{R}([\alpha^s \mathcal{R}(x), y]+ [x, \alpha^s \mathcal{R}(y)],\quad\mbox{for all }x,y\in \mathfrak{g}.
\end{equation*}

\begin{proposition}\label{induced hom-Lie algebra}
Let $\mathcal{R}$ be an $s$-Rota-Baxter operator on a hom-Lie algebra $(\mathfrak{g}, [~,~], \alpha)$. Then $\mathcal{R}$ induces a hom-Lie algebra structure $(\mathfrak{g},[~,~]_\mathcal{R},\alpha)$, where the bracket $[~,~]_\mathcal{R}$ is given by 
$$[x, y]_\mathcal{R}= [\alpha^s \mathcal{R}x, y] + [x, \alpha^s \mathcal{R}y].$$ 
\end{proposition}
\begin{proof}
From Remark \ref{5.2} and Proposition \ref{induced hom-pre-Lie algebra}, one obtains the induced hom-pre-Lie algebra structure. Then the hom-Lie algebra $(\mathfrak{g},[~,~]_\mathcal{R},\alpha)$ is the sub-adjacent hom-Lie algebra to this induced hom-pre-Lie algebra.
\end{proof}

From remark \ref{5.2}, any $s$-Rota-Baxter operator on a hom-Lie algebra $(\mathfrak{g},[~,~],\alpha)$ is simply an $\mathcal{O}$-operator on $(\mathfrak{g}, [~,~], \alpha)$ with respect to the $\alpha^s$-adjoint representation. Consequently, the results developed in Section $3$ also hold for $s$-Rota-Baxter operators on hom-Lie algebras. More precisely, the $\alpha^s$-adjoint representation of the hom-Lie algebra $(\mathfrak{g},[~,~],\alpha)$ on itself induces a graded Lie algebra structure $\big(C^*_{\alpha}(\mathfrak{g},\mathfrak{g}),\{\!\!\{-,-\}\!\!\}\big)$, as described in Subsection 3.1, and we have the following result.
\begin{theorem}
A linear map $\mathcal{R}: \mathfrak{g} \rightarrow \mathfrak{g}$ is an $s$-Rota-Baxter operator on a hom-Lie algebra $(\mathfrak{g},[~,~],\alpha)$ if and only if $\mathcal{R}$ is a Maurer-Cartan element of the graded Lie algebra $\big(C^*_{\alpha}(\mathfrak{g},\mathfrak{g}),\{\!\!\{-,-\}\!\!\}\big)$. Thus, an $s$-Rota-Baxter operator $\mathcal{R}$ on $(\mathfrak{g},[~,~],\alpha)$ induces a differential graded Lie algebra structure $\big(C^*_{\alpha}(\mathfrak{g},\mathfrak{g}),\{\!\!\{~,~\}\!\!\},\delta_\mathcal{R}:=\{\!\!\{\mathcal{R},~\}\!\!\}\big)$.  
\end{theorem}

Let us observe that it follows that for a linear map $\mathcal{R}^\prime:\mathfrak{g}\rightarrow \mathfrak{g}$, the sum $\mathcal{R}+\mathcal{R}^\prime$ is an $s$-Rota-Baxter operator on $(\mathfrak{g},[~,~],\alpha)$ if and only if 
$$\delta_\mathcal{R}(\mathcal{R}^{\prime})+\frac{1}{2}\{\!\!\{\mathcal{R}^\prime,\mathcal{R}^\prime\}\!\!\}=0,$$
i.e., $\mathcal{R}^\prime$ is a Maurer-Cartan element of the differential graded Lie algebra $\big(C^*_{\alpha}(\mathfrak{g},\mathfrak{g}),\{\!\!\{~,~\}\!\!\},\delta_\mathcal{R}\big)$.

If the hom-Lie algebra $(\mathfrak{g},[~,~],\alpha)$ is regular, then one may describe linear and formal deformations of $s$-Rota-Baxter operators on $(\mathfrak{g},[~,~],\alpha)$, for any integer $s$. This description follows from the deformation theory for $\mathcal{O}$-operators in Section $4$. In particular, let us write down some definitions and results on linear deformations of $s$-Rota-Baxter operators.

\begin{definition} Let $\mathcal{R}$ be an $s$-Rota–Baxter operator on a regular hom-Lie algebra $(\mathfrak{g}, [~,~], \alpha)$.\\
(i) Let $R: \mathfrak{g}\longrightarrow \mathfrak{g}$ be a linear operator. If a $t$-parametrized family $\mathcal{R}_t := \mathcal{R} + t R$ is an $s$-Rota–Baxter operator on $(\mathfrak{g}, [~,~], \alpha),$ for all $t \in K$, then we say that $R$ generates a linear deformation of $\mathcal{R}$.\\
(ii) Let $\mathcal{R}^1_t= \mathcal{R} + t R_1$ and $\mathcal{R}^2_t:= \mathcal{R} + t R_2$ be two linear deformations of $\mathcal{R}$ generated
by $R_1$ and $R_2$ respectively. They are said to be equivalent if there exists an $x \in \mathfrak{g}$
such that $\alpha(x)=x$ and the pair $(\mathsf{Id}_\mathfrak{g} + t \mathsf{ad}^\dagger_x , \mathsf{Id}_\mathfrak{g} + t\rho_s(x)^\dagger)$ is a homomorphism from $\mathcal{R}^2_t$ to $\mathcal{R}^1_t$, where $$\mathsf{ad}^{\dagger}_x(y):=\alpha^{-1}[x, y]~~ \mbox{and  }\rho_s(x)^{\dagger}(y):=\alpha^{-1}[\alpha^s x, y]~~\mbox{for } x, y\in \mathfrak{g}.$$ 
%(iii) A linear deformation $\mathcal{R}_t = \mathcal{R} + tR$ of $\mathcal{R}$ is said to be trivial if there exists an $x \in \mathfrak{g}$ such that $(\mathsf{Id}_\mathfrak{g} + t \mathsf{ad}^\dagger_x , \mathsf{Id}_\mathfrak{g} + t\rho_s(x)^\dagger)$ is a homomorphism from $\mathcal{R}_t$ to $\mathcal{R}$, where $ad^\dagger_x$ and $\rho_s(x)^\dagger$ are defined as above.
\end{definition}

Let us consider the hom-Lie algebra $(\mathfrak{g},[-,-]_\mathcal{R},\alpha)$ given by proposition \ref{induced hom-Lie algebra}. From Proposition \ref{rep associated to O-operator}, we obtain a representation $\rho_\mathcal{R}:\mathfrak{g} \rightarrow \mathsf{End}(\mathfrak{g})$ given by
$$\rho_\mathcal{R}(x)(y):=[\mathcal{R}x,y]+\mathcal{R}[\alpha^s(y),x],\quad \mbox{for all  }x,y \in\mathfrak{g}.$$
The triplet $(\mathfrak{g},\alpha,\rho_\mathcal{R})$ is a representation of the hom-Lie algebra $(\mathfrak{g},[~,~]_\mathcal{R},\alpha)$. Then, from Subsection $3.3$, we obtain the extended cochain complex $\big(\widetilde{C}^*_{\alpha,\alpha}(\mathfrak{g},\mathfrak{g}),\delta_{\alpha,\alpha}\big)$ (with $0$-cochains) for the hom-Lie algebra $(\mathfrak{g},[~,~]_\mathcal{R},\alpha)$ with coefficients in the representation $(\mathfrak{g},\alpha,\rho_\mathcal{R})$. This complex serves as deformation complex for an $s$-Rota-Baxter operator on a regular hom-Lie algebra $(\mathfrak{g},[~,~],\alpha)$. Moreover, it the following proposition holds true.


\begin{proposition} Let $\mathcal{R}$ be an $s$-Rota–Baxter operator on a hom-Lie algebra $\mathfrak{g}$. If
$R$ generates a linear deformation of $\mathcal{R}$, then $R$ is a $1$-cocycle. Moreover, if two linear
deformations of $\mathcal{R}$ generated by $R_1$ and $R_2$ are equivalent, then $R_1$ and $R_2$ determine the
same cohomology class.
\end{proposition}

\begin{definition}
Let $(\mathfrak{g},[~,~],\alpha)$ be a regular hom-Lie algebra and $\mathcal{R}$ be an $s$-Rota-Baxter operator on it. 
Then $x\in \mathfrak{g}$ is called a Nijenhuis
element associated to $s$-Rota-Baxter operator $\mathcal{R}$ if
\begin{align*}
\alpha(x)&=x,\\
[x, [Ry, x] + R[x, y]] &= 0, \quad\mbox{for all  } y \in \mathfrak{g} \\
[[x, y], [x, z]]& = 0,\quad\mbox{for all  }  y, z \in \mathfrak{g}.
\end{align*}
Let us denote the set of all Nijenhuis elements associated to $\mathcal{R}$ by $\mathsf{Nij}(\mathcal{R})$.
\end{definition}

Thus, it is easy to see that the Theorem \ref{trivial deformations} leads to the following result.
\begin{proposition}
 Let $\mathcal{R}$ be an $s$-Rota–Baxter operator on a regular hom-Lie algebra $(\mathfrak{g},[~,~],\alpha)$. If
$R$ generates a trivial linear deformation of $\mathcal{R}$, then it induces a Nijenhuis element. Conversely, for any Nijenhuis element $x\in \mathsf{Nij}(\mathcal{R})$, the linear sum $\mathcal{R}_t = \mathcal{R} + t R$ with $R= \delta_{\alpha}^s(x)$ is a
trivial linear deformation of $\mathcal{R}$. 
\end{proposition}

Note that one can also deduce results on formal deformations of $s$-Rota-Baxter operators (of weight $0$) on regular hom-Lie algebras following the results in Subsection $4.3$.

\subsection{Skew-symmetric r-matrices on regular hom-Lie algebras}
Let $(\mathfrak{g},[~,~],\alpha)$ be a regular hom-Lie algebra. Here, we need to take $\alpha$ invertible since we are going to use the coadjoint representation, defined in Example \ref{coadjoint rep}. It is known that $(\wedge^*\mathfrak{g},[~,~]_{g},\tilde\alpha)$ is a graded hom-Lie algebra, where the graded hom-Lie bracket is given by  
\begin{align*}
&[x_1\wedge\cdots\wedge x_n,y_1\wedge \cdots\wedge y_m]_{g}\\
=&\sum_{i=1}^n\sum_{j=1}^m (-1)^{i+j}[x_i,y_j]\wedge (\alpha(x_1)\wedge\cdots \widehat{\alpha(x_i)}\wedge\cdots\wedge \alpha(x_n)\wedge \alpha(y_1)\wedge \cdots \widehat{\alpha(y_j)}\wedge\cdots\wedge \alpha(y_m)),
\end{align*}
and the map $\tilde{\alpha}:\wedge^*\mathfrak{g}\rightarrow\wedge^*\mathfrak{g}$ is defined by 
$$\tilde{\alpha}(x_1\wedge x_2\wedge\cdots\wedge x_n)=\alpha(x_1)\wedge\alpha(x_2)\wedge\cdots\wedge \alpha(x_n),$$
for all $x_1,\cdots,x_n,y_1,\cdots,y_m\in \mathfrak{g}$.
Let us now consider a graded vector space $\mathcal{H}:=\oplus_{n\geq 1} \mathcal{H}^n,$ where 
$$\mathcal{H}^n:=\{\chi\in \wedge^n\mathfrak{g}~|~\tilde\alpha(\chi)=\chi\}.$$ 
If we restrict the bracket $[~,~]_g$ on $\mathcal{H}$, then we get a graded Lie algebra $(\mathcal{H},[~,~]_g)$.

A skew-symmetric $r$-matrix \cite{Yau1} on the hom-Lie algebra $(\mathfrak{g},[~,~],\alpha)$ is an element $r\in \mathcal{H}^2$, which satisfies the identity $[r,r]_g=0$.  In other words, $r$-matrices on hom-Lie algebra $(\mathfrak{g},[~,~],\alpha)$ are Maurer-Cartan elements of the associated graded Lie algebra $(\mathcal{H},[~,~]_g)$.

Any element $r\in \mathfrak{g}\otimes \mathfrak{g}$ corresponds to an operator $r^{\sharp}:\mathfrak{g}^*\rightarrow \mathfrak{g}$ and vice versa by the following expression
\begin{equation}\label{operator to r-matrix}
\langle\xi,r^{\sharp}(\eta)\rangle=\langle\xi\otimes \eta, r\rangle,\quad \mbox{for all }\xi,\eta\in \mathfrak{g}^*.
\end{equation}
The skew-symmetry of $r$ is equivalent to 
$$\langle\xi,r^{\sharp}(\eta)\rangle+\langle\eta,r^{\sharp}(\xi)\rangle=0.$$
Let $r\in \mathcal{H}^2$ and it is given by
\begin{equation}\label{def of r-matrix}
r=\sum_i(x_i\otimes y_i- y_i\otimes x_i),\quad\mbox{for some }x_i,y_i\in \mathfrak{g}.
\end{equation}
Then, the condition $[r,r]_{\mathfrak{g}}=0$ is equivalent to
$$[r^{12},r^{13}]+[r^{12},r^{23}]+[r^{13},r^{23}]=0,$$
where, 
\begin{align*}
[r^{12},r^{13}]&=\sum_i\sum_j[x_i,x_j]\otimes \tilde{y}_i\otimes\tilde{y}_j-[x_i,y_j]\otimes \tilde{y}_i\otimes\tilde{x}_j-[y_i,x_j]\otimes \tilde{x}_i\otimes\tilde{y}_j+[y_i,y_j]\otimes \tilde{x}_i\otimes\tilde{x}_j,\\
[r^{12},r^{13}]&=\sum_i\sum_j\tilde{x}_i\otimes[y_i,x_j]\otimes \tilde{y}_j-\tilde{x}_i\otimes[y_i,y_j]\otimes \tilde{x}_j-\tilde{y}_i\otimes[x_i,x_j]\otimes \tilde{y}_j+\tilde{y}_i\otimes[x_i,y_j]\otimes \tilde{x}_j,\\
[r^{13},r^{23}]&=\sum_i\sum_j\tilde{x}_i\otimes\tilde{x}_j\otimes [y_i,y_j]-\tilde{x}_i\otimes\tilde{y}_j\otimes [y_i,x_j]-\tilde{y}_i\otimes\tilde{x}_j\otimes[x_i,y_j] +\tilde{y}_i\otimes\tilde{y}_j\otimes [x_i,x_j],
\end{align*}
and $\tilde{x}:=\alpha(x)$. Moreover, the condition $\alpha^{\otimes 2}r=r$ implies that
\begin{align*}
\langle\xi,r^{\sharp}(\eta)\rangle=\langle\xi\otimes \eta, \alpha^{\otimes 2}r\rangle
=\langle\alpha^*(\xi)\otimes \alpha^*(\eta), r\rangle=\langle\alpha^*(\xi), r^{\sharp}(\alpha^*(\eta))\rangle=\langle\xi, \alpha\circ r^{\sharp}\circ\alpha^*(\eta)\rangle,
\end{align*}
for all $\xi,\eta\in \mathfrak{g}^*$. i.e., the operator $r^{\sharp}:\mathfrak{g}^*\rightarrow \mathfrak{g}$ satisfies 
\begin{equation}\label{cond1 for corresp}
\alpha\circ r^{\sharp}=r^{\sharp}\circ (\alpha^{-1})^*.
\end{equation} 

Next, we show that equation \eqref{operator to r-matrix} defines a bijective correspondence between $r$-matrices on a hom-Lie algebra $(\mathfrak{g},[~,~],\alpha)$  and  $\mathcal{O}$-operators on the hom-Lie algebra $(\mathfrak{g},[~,~],\alpha)$ with respect to the coadjoint representation $(\mathfrak{g}^*,(\alpha^{-1})^*,\rho^\star)$. We denote $\rho^\star(x)(\xi)$ simply by $\{x,\xi\}$, for all $x\in\mathfrak{g}$ and $\xi\in\mathfrak{g}^*$. 

For $\xi,\eta\in\mathfrak{g}^*,$ we get
\begin{align}\label{identity1}
&[r^{\sharp}(\alpha^*\xi),r^{\sharp}(\alpha^*\eta)]\\\nonumber
=&\sum_i\sum_j[\langle \alpha^*\xi,y_i\rangle x_i- \langle \alpha^*\xi,x_i\rangle y_i,\langle \alpha^*\eta,y_j\rangle x_j- \langle \alpha^*\eta,x_j\rangle y_j]\\\nonumber
=&\sum_{i}\sum_j \bigg(\langle \xi,\alpha(y_i)\rangle \langle \eta,\alpha(y_j)\rangle [x_i,x_j]-\langle \xi,\alpha(y_i)\rangle \langle \eta,\alpha(x_j)\rangle [x_i,y_j] \\\nonumber
&\quad\quad\quad\quad-\langle \xi,\alpha(x_i)\rangle \langle \eta,\alpha(y_j)\rangle [y_i,x_j]+\langle \xi,\alpha(x_i)\rangle \langle \eta,\alpha(x_j)\rangle [y_i,y_j]\bigg)\\\nonumber
=&-(\langle\xi,~\rangle\otimes\langle\eta,~\rangle\otimes \mathsf{Id}_\mathfrak{g})([r^{13},r^{23}]).
\end{align}
Note that by the equations \eqref{operator to r-matrix}, \eqref{def of r-matrix}, and the condition $\alpha^{\otimes 2}r=r$, we obtain
 $$r^{\sharp}(\eta)=\sum_i\langle \eta,y_i\rangle x_i- \langle \eta, x_i\rangle y_i=\sum_i\langle \eta,\alpha(y_i)\rangle \alpha(x_i)- \langle \eta, \alpha(x_i)\rangle \alpha(y_i).$$ 
Then, for all $\xi,\eta\in\mathfrak{g}^*$, we have the following identity    
\begin{align}\label{identity2}
&r^{\sharp}(\{r^{\sharp}(\alpha^*\xi),\alpha^*\eta\})\\\nonumber
=&r^{\sharp}\bigg(\sum_{i} \big(\langle \alpha^*\xi,y_i\rangle \{x_i,\alpha^*\eta\}- \langle \alpha^*\xi,x_i\rangle \{y_i,\alpha^*\eta\}\big)\bigg)\\\nonumber
%=&r^{\sharp}\bigg(\sum_{i} \big(\langle \xi,\alpha(y_i)\rangle \{x_i,\alpha^*\eta\}- \langle \xi,\alpha(x_i)\rangle \{y_i,\alpha^*\eta\}\big)\bigg)\\\nonumber
=&\sum_j\sum_i\bigg(\langle \xi,\alpha(y_i)\rangle \Big(\langle\{x_i,\alpha^*\eta\},\alpha(y_j)\rangle \alpha(x_j)-  \langle\{x_i,\alpha^*\eta\},\alpha(x_j)\rangle \alpha(y_j)\Big)\\\nonumber 
&\quad \quad \quad-\langle \xi,\alpha(x_i)\rangle \Big(\langle\{y_i,\alpha^*\eta\},\alpha(y_j)\rangle \alpha(x_j)-  \langle\{y_i,\alpha^*\eta\},\alpha(x_j)\rangle \alpha(y_j)\Big)\bigg)\\\nonumber
=&\sum_j\sum_i\bigg(\langle \xi,\alpha(y_i)\rangle \Big(\langle \eta,[x_i,y_j]\rangle \alpha(x_j)-  \langle\eta,[x_i,x_j]\rangle \alpha(y_j)\Big)\\\nonumber 
&\quad \quad \quad-\langle \xi,\alpha(x_i)\rangle \Big(\langle\eta,[y_i,y_j]\rangle \alpha(x_j)-  \langle\eta,[y_i,x_j]\rangle \alpha(y_j)\Big)\bigg)\\\nonumber
=&-(\langle\xi,~\rangle\otimes\langle\eta,~\rangle\otimes \mathsf{Id}_\mathfrak{g})([r^{12},r^{13}]).
\end{align}
Similarly,
\begin{align}\label{identity3}
r^{\sharp}(\{r^{\sharp}(\alpha^*\eta),\alpha^*\xi\})=(\langle\xi,~\rangle\otimes\langle\eta,~\rangle\otimes \mathsf{Id}_\mathfrak{g})([r^{12},r^{13}]).
\end{align}

Therefore, 
$$\big\langle \xi\otimes \eta\otimes \gamma,~[r,r]_{\mathfrak{g}}~\big\rangle= \Big\langle \gamma,~[r^{\sharp}(\alpha^*\xi),r^{\sharp}(\alpha^*\eta)]-r^{\sharp}(\{r^{\sharp}(\alpha^*\xi),\alpha^*\eta\}-\{r^{\sharp}(\alpha^*\eta),\alpha^*\xi\})\Big\rangle,$$
for all $\xi,\eta,\gamma\in\mathfrak{g}^*$. Hence, we have the following result.

\begin{theorem}\label{correspondence theorem}
Let $(\mathfrak{g},[~,~],\alpha)$ be a regular hom-Lie algebra. Then, an element $r\in \mathcal{H}^2$ is an $r$-matrix on the regular hom-Lie algebra $(\mathfrak{g},[~,~],\alpha)$ if and only if the associated operator $r^{\sharp}:\mathfrak{g}^*\rightarrow \mathfrak{g}$, defined by equation \eqref{operator to r-matrix}, is an $\mathcal{O}$-operator on hom-Lie algebra $(\mathfrak{g},[~,~],\alpha)$ with respect to the coadjoint representation $(\mathfrak{g}^*,(\alpha^{-1})^*,\rho^\star)$.
\end{theorem}

\begin{remark}
There is another version of $r$-matrix defined in \cite{hom-Liebi}. Here, we followed the version defined in \cite{Yau1} to derive a connection between $\mathcal{O}$-operators and $r$-matrices on regular hom-Lie algebras (Theorem \ref{correspondence theorem}). Also, we refer to \cite{Yau1} for more details on hom-Yang-Baxter equations.
\end{remark}


If $r$ is an $r$-matrix on the regular hom-Lie algebra $(\mathfrak{g},[~,~],\alpha)$, then from Theorem \ref{correspondence theorem} $r^\sharp:\mathfrak{g}^*\rightarrow \mathfrak{g}$ is an $\mathcal{O}$-operator on hom-Lie algebra $(\mathfrak{g},[~,~],\alpha)$ with respect to the coadjoint representation $(\mathfrak{g}^*,(\alpha^{-1})^*,\rho^\star)$. Thus, from Proposition \ref{induced hom-pre-Lie algebra}, the triplet $(\mathfrak{g}^*,[~,~]_r,(\alpha^{-1})^*)$ is the induced sub-adjacent hom-Lie algebra with the bracket 
$$[\xi,\eta]_{r}:=\mathsf{ad}^*_{r^{\sharp}(\xi)}\eta-\mathsf{ad}^*_{r^{\sharp}(\eta)}(\xi)=\{r^{\sharp}(\eta),\xi\}-\{r^{\sharp}(\xi),\eta\},\quad\mbox{for all }\xi,\eta\in\mathfrak{g}^*.$$

Moreover, from Proposition \ref{rep associated to O-operator}, it follows that $(\mathfrak{g},\alpha,\rho_{r})$ is a representation of the hom-Lie algebra $(\mathfrak{g}^*,[~,~]_r,(\alpha^{-1})^*)$, where the map $\rho_{r}:\mathfrak{g}^*\rightarrow\mathsf{End}(\mathfrak{g})$ is given by 
$$\rho_{r}(\xi)(x)=[r^{\sharp}(\xi),x]+r^{\sharp} \{x,\xi\}\quad\mbox{for all }\xi\in \mathfrak{g}^*,~x\in \mathfrak{g}.$$
It is also clear that the induced representation $\rho_r:\mathfrak{g}^*\rightarrow \mathsf{End}(\mathfrak{g})$ is the same as the coadjoint representation $\mathsf{ad}^*:\mathfrak{g}^*\rightarrow \mathsf{End}(\mathfrak{g})$ of the hom-Lie algebra $(\mathfrak{g}^*,[~,~]_r,(\alpha^{-1})^*)$ on the pair $(\mathfrak{g},\alpha)$. In fact,  
\begin{align*}
\langle\eta, \rho_r(\xi)(x)\rangle&=\langle\eta, r^{\sharp} \{x,\xi\}+[r^{\sharp}(\xi),x]\rangle\\&=-\langle \{x,\xi\},r^{\sharp}(\eta)\rangle - \langle\eta, [x,r^{\sharp}(\xi)]\rangle\\
&=-\langle\{x,\xi\},\alpha(r^{\sharp}(\alpha^*\eta))\rangle- \langle\eta,[x,r^{\sharp}(\xi)]\rangle\\
&=\big\langle\xi,~\alpha^{-1}[x,r^{\sharp}(\alpha^*\eta)]~\big\rangle- \langle\eta, [x,r^{\sharp}(\xi)]\rangle\\
&=\langle \{r^{\sharp}(\alpha^*\eta),\xi\}-\{r^{\sharp}(\xi),\alpha^*\eta\},\alpha(x)\rangle\\
&=\langle [(\alpha^*)^2(\eta),\alpha^*\xi]_r,x\rangle\\
&=\langle \eta,\mathsf{ad}^*_{\xi}(x)\rangle, \quad \quad \mbox{for all } x\in \mathfrak{g},~\xi,\eta\in \mathfrak{g}^*.
\end{align*}

In \cite{Sheng3}, the authors define the notion of a weak morphism of $r$-matrices. Now, we extend this notion to $r$-matrices on hom-Lie algebras.      

\begin{definition}
Let $r_1,r_2$ be skew-symmetric $r$-matrices on a hom-Lie algebra $(\mathfrak{g},[~,~],\alpha)$. A pair $(\phi,\psi)$, consisting of a hom-Lie algebra homomorphism $\phi:\mathfrak{g}\rightarrow \mathfrak{g}$ and a linear map $\psi:\mathfrak{g}\rightarrow \mathfrak{g}$, is said to be a weak homomorphism from $r_1$ to $r_2$ if the following conditions hold
\begin{enumerate}[label=\alph*)]
\item $\psi\circ \alpha=\alpha\circ \psi,$
\item $(\psi\otimes \mathsf{Id}_{\mathfrak{g}})(r_1)=(\mathsf{Id}_{\mathfrak{g}}\otimes \phi)(r_2),$
\item $\psi([\phi(x),y])=[x,\psi(y)],~~~\mbox{for all }x,y\in \mathfrak{g}$.
\end{enumerate} 
A weak homomorphism $(\phi,\psi)$ is called a `weak isomorphism' if $\phi$ and $\psi$ are linear automorphisms on $\mathfrak{g}$. 
\end{definition}
The following result establishes the relationship between weak homomorphisms between $r$-matrices on hom-Lie algebras and homomorphisms between the corresponding $\mathcal{O}$-operators on hom-Lie algebras with respect to the coadjoint representation.

\begin{proposition}
Let $r_1,r_2$ be skew-symmetric $r$-matrices on a hom-Lie algebra $(\mathfrak{g},[~,~],\alpha)$. A pair $(\phi,\psi)$ is a weak homomorphism from $r_1$ to $r_2$ if and only if $(\phi,\psi^*)$ is a homomorphism of $\mathcal{O}$-operators from $r_1^\sharp$ to $r_2^\sharp$.
\end{proposition}
\begin{proof} 
First, let us observe that any skew-symmetric $r$-matrix 
$r\in \mathcal{H}^2$, given by $$r=\sum_{i}(a_i\otimes b_i- b_i\otimes a_i)\quad\mbox{for some } a_i, b_i\in \mathfrak{g},$$ can also be written as a sum $r=\sum_{j}(a_j^\prime\otimes b_j^\prime)$ by changing the indexing. Let $r_1=\sum_i x_i\otimes y_i$ and $r_2=\sum_i x^{\prime}_i\otimes y^{\prime}_i$ be skew-symmetric $r$-matrices. The associated $\mathcal{O}$-operators are given by 
$${r_1}^\sharp=\sum\limits_{i}\langle\xi,x_i\rangle y_i \quad \mbox{and}\quad r_2^\sharp=\sum\limits_{j}\langle\xi,x^{\prime}_j\rangle y^{\prime}_j, \quad\mbox{for }\xi\in \mathfrak{g}^*.$$  

Let the pair $(\phi,\psi)$ be a weak homomorphism from $r_1$ to $r_2$. Then, the map $\phi:\mathfrak{g}\rightarrow \mathfrak{g}$ is a hom-Lie algebra homomorphism, $\psi\circ\alpha=\alpha\circ \psi$, and the following conditions are satisfied
\begin{equation}\label{hom:con1}
(\psi\otimes \mathsf{Id}_{\mathfrak{g}})(r_1)=(\mathsf{Id}_{\mathfrak{g}}\otimes \psi)(r_2), 
\end{equation}
\begin{equation}\label{hom:con2}
\psi([\phi(x),y])=[x,\psi(y)],\quad\mbox{for all }x,y\in \mathfrak{g}.
\end{equation}
Let us note that $\psi\circ\alpha=\alpha\circ \psi$ if and only if $\psi^*\circ(\alpha^{-1})^*=(\alpha^{-1})^*\circ \psi^*$. Next, we show that the conditions \eqref{hom:con1} and \eqref{hom:con2} hold if and only if 
the following conditions \eqref{hom:con3} and \eqref{hom:con4} hold true, respectively.
\begin{equation}\label{hom:con3}
r_2^\sharp\circ \psi^*=\phi\circ r_1^\sharp,
\end{equation}
\begin{equation}\label{hom:con4}
\psi^*(\{x,\xi\})=\{\phi(x),\psi^*(\xi)\}, ~~~\mbox{for all }x,y\in \mathfrak{g}.
\end{equation}

In order to show the desired result, let us consider the following expressions 
\begin{equation*}\label{condition-5.9}
\langle\xi\otimes\eta,(\mathsf{Id}_{\mathfrak{g}}\otimes \phi)(r_1)\rangle=\sum\limits_{i}\langle \xi,x_i\rangle \langle \eta,\phi(y_i)\rangle =\langle\eta,\phi(\sum\limits_{i}\langle\xi, x_i \rangle y_i\rangle=\langle\eta,\phi r_1^\sharp(\xi)\rangle
\end{equation*}
and
\begin{equation*}\label{condition-5.10}
\langle\xi\otimes \eta,(\psi\otimes \mathsf{Id}_\mathfrak{g})(r_2)\rangle=\sum\limits_{j}\langle\xi, \psi(x_j^\prime)\rangle \langle \eta,y_j^\prime\rangle =\langle\eta,\sum\limits_j\langle \psi^*\xi,x_j^\prime \rangle y_j^\prime\rangle=\langle\eta,r_2^\sharp(\psi^*\xi)\rangle.
\end{equation*}
Therefore, the condition \eqref{hom:con1} holds if and only if the condition \eqref{hom:con3} holds true. Moreover, the identities $\psi\circ \alpha=\alpha\circ \psi$ and $\phi\circ\alpha=\alpha\circ \phi$ implies that  
\begin{equation*}\label{condition-5.7}
\langle\xi,\psi[y,\phi(x)]\rangle=\Big\langle\psi^*(\xi),[y,\phi(x)]\Big\rangle=\langle\{\alpha\phi(x),\psi^*(\xi)\},\alpha^2(y)\rangle=\langle\{\phi(\alpha (x)),\psi^*(\xi)\},\alpha^2(y)\rangle,
\end{equation*}
\begin{equation*}\label{condition-5.8}
\langle \xi,[\psi(y),x]\rangle=\langle\{\alpha(x),\xi\},\alpha^2(\psi(y))\rangle=\langle\{\alpha(x),\xi\},\psi\alpha^2(y)\rangle=\langle\psi^*\{\alpha(x),\xi\},\alpha^2(y)\rangle.
\end{equation*}
Since $\alpha$ is an automorphism, it follows that the condition \eqref{hom:con2} holds if and only if the condition \eqref{hom:con4} holds true. 


In other words, we can conclude that $(\phi,\psi)$ is a weak homomorphism from $r_1$ to $r_2$ if and only if $\phi$ is a hom-Lie algebra homomorphism, $\psi^*\circ(\alpha^{-1})^*=(\alpha^{-1})^*\circ \psi^*$, and the conditions \eqref{hom:con3}, \eqref{hom:con4} are satisfied. Equivalently, from the definition \ref{morphism of O-operators}, it follows that $(\phi,\psi)$ is a weak homomorphism from $r_1$ to $r_2$ if and only if the pair $(\phi,\psi^*)$ is a morphism of $\mathcal{O}$-operators from $r_1^\sharp$ to $r_2^\sharp$.
\end{proof}

\begin{definition}
Let $r$ be a skew-symmetric $r$-matrix on a regular hom-Lie algebra $(\mathfrak{g},[~,~],\alpha)$. Let us consider a linear sum $r_t:=r+t\tau\in \mathcal{H}^2$ for some $\tau\in \mathcal{H}^2$. If $r_t$ is a skew-symmetric $r$-matrix on the hom-Lie algebra $(\mathfrak{g},[~,~],\alpha)$, then it is called a linear deformation generated by the element $\tau\in \mathcal{H}^2$.  
\end{definition}

\begin{definition}
Let $r$ be a skew-symmetric $r$-matrix on a regular hom-Lie algebra $(\mathfrak{g},[~,~],\alpha)$. Then, linear deformations $r^1_t:=r+t\tau_1$ and $r^2_t:=r+t\tau_2$, generated by elements $\tau_1$ and $\tau_2$ in $\mathcal{H}^2$ are equivalent if there exists an element $x\in \mathfrak{g}$ satisfying $\alpha(x)=x$ such that the pair $(\mathsf{Id}+t \mathsf{ad}_x^{\dagger},\mathsf{Id}-t \mathsf{ad}_x^{\dagger})$ is a weak homomorphism from $r^2_t$ to $r^1_t$.
\end{definition}

\begin{proposition}
Let $(\mathfrak{g},[~,~],\alpha)$ be a hom-Lie algebra and $r\in \mathcal{H}^2$ be a skew-symmetric $r$-matrix on $(\mathfrak{g},[~,~],\alpha)$. Then,
\begin{enumerate}
\item An element $\tau\in \mathcal{H}^2$ generates a linear deformation of $r$ if and only if the induced map $\tau^\sharp:\mathfrak{g}^*\rightarrow\mathfrak{g}$ generates a linear deformation of the $\mathcal{O}$-operator $r^{\sharp}$.
\item Linear deformations $r^1_t:=r+t\tau_1$ and $r^2_t:=r+t\tau_2$ are equivalent if and only if the linear deformations $(r^1_t)^\sharp$ and $(r^2_t)^\sharp$ of the $\mathcal{O}$-operator $r^{\sharp}$ are equivalent.
\end{enumerate}
\end{proposition}

\subsection*{Formal deformations of $r$-matrices}
Let $r\in \mathcal{H}^2$ be a skew-symmetric $r$-matrix on a hom-Lie algebra $(\mathfrak{g},[~,~],\alpha)$. A formal sum $\textstyle{r_t:=r+\sum_{i\geq 1} t^i r_i}$ is called a formal deformation of $r$ if it satisfies 
$$[[r_t,r_t]]_{\mathfrak{g}_t}=0.$$
Here, $r_t\in \wedge^2\mathfrak{g}[[t]]$ and the bracket $[[~,~]]_{\mathfrak{g}_t}$ is the Lie bracket on the exterior algebra $\wedge^*\mathfrak{g}[[t]]$. 
Next, it easily follows that 
\begin{enumerate}
\item A formal sum $\textstyle{r_t:=r+\sum_{i\geq 1} t^i r_i}$ is a formal deformation of $r$ if and only if the induced map $r_t^{\sharp}:=r^{\sharp}+\sum_{i\geq 1} t^i r_i^\sharp$ is a formal deformation of the $\mathcal{O}$-operator $r^{\sharp}$.

\item The formal deformations $r^1_t:=r+\sum_{i\geq 1} t^i r^1_i$ and $r^2_t:=r+\sum_{i\geq 1} t^i r^2_i$ are equivalent if and only if the formal deformations $(r^1_t)^{\sharp}$ and $(r^1_t)^\sharp$ of the $\mathcal{O}$-operator $r^{\sharp}$ are equivalent.

\end{enumerate}

%\subsection{Quasi-triangular hom-Lie bialgebras}
%Let us recall that a hom-Lie bialgebra is a hom-Lie algebra $(\mathfrak{g},[~,~],\alpha)$ that is also equipped with a Lie coalgebra structure $\nabla:\mathfrak{g}\rightarrow \wedge^2\mathfrak{g}$ such that $\nabla$ is a $1$-cocycle on $\mathfrak{g}$ with coefficients in $\wedge^2\mathfrak{g}$.


\begin{thebibliography}{999}
\bibitem{q1}
\newblock N. Aizawa and H.-T. Sato, 
\newblock {\it q}-deformation of the Virasoro algebra with central extension.
 {\it Phys. Lett. B} \textbf{256 (2)}: 185--190, 1999.

\bibitem{DefHLIE}
F. Ammar, A. Ejbehi and A. Makhlouf,
Cohomology and Deformations of Hom-algebras.
{\it J. Lie Theory} \textbf{21(4)}: 813--836, 2011.


\bibitem{Baxter}
G. Baxter, 
An analytic problem whose solution follows from a simple algebraic identity.
\textit{Pacific J. Math.} \textbf{10}: 731--742,  1960.

\bibitem{hom-Liebi}
L. Cai and Y. Sheng, 
Purely hom-Lie bialgebra. 
\textit{Sci. China Math.} (2018). 
https://doi.org/10.1007 /s11425-016-9102-y.


\bibitem{Cartier}
P. Cartier,
On the structure of free Baxter algebras.
\textit{Advances in Math.} \textbf{9}: 253--265, 1972.


\bibitem{q3}
\newblock M. Chaichian, A.P. Isaev, J. Lukierski, Z. Popowicz and P. Prešnajder,
\newblock  {\it q}-deformations of Virasoro algebra and
conformal dimensions. {\it Phys. Lett. B} \textbf{262(1)}: 32–-38,1991.

\bibitem{q4}
M. Chaichian, P. Kulish and J. Lukierski, {\it q}-deformed Jacobi identity,  {\it q}-oscillators and {\it q}-deformed infinite dimensional
algebras.  {\it Phys. Lett. B} \textbf{237(3–4)}: 401--406, 1990.

\bibitem{q5}
R. Chakrabarti and R. Jagannathan, A {\it(p,q)}-deformed Virasoro algebra.  {\it J. Phys. A} \textbf{25(9)}: 2607--2614, 1992.



\bibitem{Connes}
A. Connes and D. Kreimer, 
Renormalization in quantum field theory and the Riemann-Hilbert problem. I. The Hopf algebra structure of graphs and the main theorem.
\textit{ Comm. Math. Phys.} \textbf{210(1)}:
249--273, 2000.



\bibitem{Guo}
L. Guo and W. Keigher, 
Baxter algebras and shuffle products.
\textit{ Adv. Math.} \textbf{150(1)}: 117--149, 2000.

\bibitem{Survey-Guo}
L. Guo, An introduction to Rota-Baxter algebra, Surveys of Modern Mathematics, 4. International Press,
Somerville, MA; Higher Education Press, Beijing, 2012.


\bibitem{HLIE01}
J. Hartwig, D. Larsson and S. Silvestrov,
Deformations of Lie algebras using $\sigma$-derivations.
\textit{Journal of Algebra} \textbf{295}: 314--361, 2006.

\bibitem{Kupershmidt}
B. Kupershmidt,
What a classical r-matrix really is.
\textit{Journal of Nonlinear Mathematical Physics} \textbf{6(4)}: 448--488, 1999. 

\bibitem{QuasiHL1} 
D. Larsson and S. Silvestrov, Quasi-hom-Lie algebras, central extensions and 2-cocycle-like
identities, {\it Journal of Algebra} \textbf{288}: 321--344, 2005.

\bibitem{q2} D. Larsson and S. Silvestrov, Quasi-Lie algebras, {\it Contemp. Math.} \textbf{391}: 241--248, 2005.


\bibitem{q7}
A. P. Polychronakos,
 Consistency conditions and representations of a {\it q}-deformed Virasoro algebra. {\it Phys.
Lett. B} \textbf{256(1)}: 35--40, 1991.

\bibitem{Rota}
G.-C. Rota, Baxter algebras and combinatorial identities, I, II.
\textit{ Bull. Amer. Math. Soc.} \textbf{75}: 330--334,  1969.

\bibitem{q8}  
H.-T. Sato,
Realizations of {\it q}-deformed Virasoro algebra. {\it Progr. Theoret. Phys.} \textbf{89(2)}: 531--544, 1993.

\bibitem{q9} 
H.-T. Sato,
{\it q}-Virasoro operators from an analogue of the Noether currents. {\it Z. Phys. C} \textbf{70(2)}: 349--355, 1996.

\bibitem{q10}  H.-T. Sato, OPE formulae for deformed super-Virasoro algebras. {\it Nucl. Phys. B} \textbf{471}: 553--569, 1996.

\bibitem{Sheng}
Y. Sheng,
Representations of hom-Lie algebras.
\textit{Algebra and Representation Theory} \textbf{15(6)}: 1081--1098, 2012.

\bibitem{OnhLie}
Y. Sheng and Z. Xiong,
On Hom-Lie algebras.
\textit{Linear Multinear Algebra} \textbf{63(6)}: 2379-2395, 2015.

\bibitem{Sheng3}
R. Tang, C. Bai, L. Guo, and Y. Sheng,
Deformations and their controlling cohomologies of $\mathcal{O}$-operators.
\textit{Comm. Math. Phys.} \textbf{368(2)}: 665--700, 2019.

\bibitem{Voronov}
Th. Voronov, 
Higher derived brackets and homotopy algebras.
\textit{ J. Pure Appl. Algebra} \textbf{202(1-3)}:
133--153, 2005.

\bibitem{Yau2}
D. Yau,
The Hom–Yang–Baxter equation, Hom–Lie algebras,
and quasi-triangular bialgebras.
\textit{J. Phys. A: Math. Theor.} \textbf{42}: 165202 (12pp), 2009.

\bibitem{Yau1}
D. Yau,
The classical Hom-Yang-Baxter equation and
Hom-Lie bialgebras.
\textit{International Electronic Journal of Algebra}, \textbf{17}: 11--45, 2015.


\end{thebibliography}
\end{document}   
