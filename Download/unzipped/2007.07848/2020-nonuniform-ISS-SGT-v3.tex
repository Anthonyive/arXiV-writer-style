\documentclass[twocolumn]{IEEEtran} % for  journal submission


%\usepackage{epsfig}
%\ifCLASSINFOpdf
%\else
%\fi
%\usepackage[cmex10]{amsmath}


%\graphicspath{{fig/}}


%\usepackage{tikz}
%%\usepackage{geometry}
%\usetikzlibrary{matrix,positioning,decorations.pathreplacing}


%\usepackage{ifthen}

%\newtheorem{property}{Property}
%\newtheorem{nnassumption}{\bf Assumption}
%\newenvironment{assumption}{\begin{nnassumption}\it}{\end{nnassumption}}
%\newtheorem{nntheorem}{\bf Theorem}
%\newenvironment{theorem}{\begin{nntheorem}\it}{\end{nntheorem}}
%\newtheorem{nncorollary}{\bf Corollary}
%\newenvironment{corollary}{\begin{nncorollary}\it}{\end{nncorollary}}
%\newtheorem{nndefinition}{\bf Definition}
%\newenvironment{definition}{\begin{nndefinition}\it}{\end{nndefinition}}
%\newtheorem{nnproposition}{\bf Proposition}
%\newenvironment{proposition}{\begin{nnproposition}\it}{\end{nnproposition}}
%\newtheorem{nnproblem}{\bf Problem}
%\newenvironment{problem}{\begin{nnproblem}\it}{\end{nnproblem}}
%\newtheorem{nnlemma}{\bf Lemma}
%%\newenvironment{lemma}{\begin{nnlemma}\it}{\end{nnlemma}}
%\newtheorem{nnremark}{\bf Remark}
%\newenvironment{remark}{\begin{nnremark} \rm }{\hfill \hspace*{1pt}\hfill $\circ$\end{nnremark}}
%\newtheorem{nnexample}{\bf Example}
%\newenvironment{example}{\begin{nnexample} \rm}{\hfill \hspace*{1pt}\hfill $\star$\end{nnexample}}
%\newenvironment{proof}{{\bf Proof.}}{\hfill \hspace*{1pt}\hfill $\Box$}
%\newenvironment{proofof}{{\bf Proof of}}{\hfill \hspace*{1pt}\hfill $\Box$}

\usepackage{epsfig} % for postscript graphics files
\usepackage{mathptmx} % assumes new font selection scheme installed
\usepackage{times} % assumes new font selection scheme installed
\usepackage{amsmath} % assumes amsmath package installed
\usepackage{amssymb}  % assumes amsmath package installed
\usepackage{amsthm}


\usepackage{hyperref}   %Backref gives the numbers of pages, where the bibentries are cited
\hypersetup{
    colorlinks=true,
    linkcolor=blue,
    filecolor=magenta,      
    urlcolor=blue,
        citecolor=red,
    pdftitle={Sharelatex Example},
    %bookmarks=true,
    pdfpagemode=FullScreen,
}


\usepackage{url}
%\usepackage{breakurl}

\usepackage{calc}
\usepackage{pdfsync}
\usepackage{enumerate}
%\usepackage[singlelinecheck=off,font=footnotesize]{caption} 



\usepackage{amsmath,amssymb,amsfonts,amsthm}%
\usepackage{color}%
%\usepackage{a4}%
%\usepackage{enumerate}%
\usepackage{enumitem}%
%\usepackage{theorem}%
\usepackage{hyperref}%

%
%\setlength{\parindent}{0ex}%
%\setlength{\parskip}{1ex}%
%\theoremstyle{change}%
%\sloppy%



\newtheorem{assumption}{Assumption} 

\newtheorem{theorem}{Theorem}[section]
\newtheorem{lemma}[theorem]{Lemma}
\newtheorem{proposition}[theorem]{Proposition}
\newtheorem{corollary}[theorem]{Corollary}

\theoremstyle{definition}
\newtheorem{definition}[theorem]{Definition}
\newtheorem{example}[theorem]{Example}
\newtheorem{xca}[theorem]{Exercise}

%\theoremstyle{remark}
\newtheorem{remark}[theorem]{Remark}


%\newtheorem{definition}{Definition:}[section]%
%\newtheorem{proposition}[definition]{Proposition:}%
%\newtheorem{theorem}[definition]{Theorem:}%
%\newtheorem{lemma}[definition]{Lemma:}%
%\newtheorem{corollary}[definition]{Corollary:}%
%\newtheorem{assumption}[definition]{Assumption:}%
%
%{\theorembodyfont{\rmfamily}\newtheorem{remark}[definition]{Remark:}}%
%{\theorembodyfont{\rmfamily}\newtheorem{example}[definition]{Example:}}%

%\newenvironment{proof}
  %{{\bf Proof:}}
  %{\qquad \hspace*{\fill} $\Box$}%

\newcommand{\tm}{\times}%
\newcommand{\Uc}{\mathcal{U}}%
\newcommand{\R}{\mathbb{R}}%
\newcommand{\ccat}[3]{{#1\, \underset{#3}{\lozenge}\,{#2}}}%
\newcommand{\PD}{\mathcal{P}}%
\newcommand{\K}{\mathcal{K}}%
\newcommand{\Kinf}{\mathcal{K_\infty}}%
\newcommand{\KL}{\mathcal{KL}}%
\newcommand{\LL}{\mathcal{L}}%
\newcommand{\ep}{\varepsilon}%
\newcommand{\N}{\mathbb{N}}%
\newcommand{\PC}{\mathrm{PC}}%
\newcommand{\UGS}{\mathrm{UGS}}%
\newcommand{\Z}{\mathbb{Z}}%
\newcommand{\id}{\mathrm{id}}%
\newcommand{\dist}{\mathrm{dist}}%

\newcommand{\T}{\ensuremath{\mathcal{T}}}  % E.g. for semigroups

\newcounter{syscounter}
\newenvironment{sysnum}{\begin{list}{($\Sigma{\arabic{syscounter}}$)}%
{\settowidth{\labelwidth}{($\Sigma4$)}
\settowidth{\leftmargin}{($\Sigma4$)~}%
\usecounter{syscounter}}}
{\end{list}}


\let\ol=\overline%
\let\ul=\underline%

\newcommand \qrq   {\quad\Rightarrow\quad}
\newcommand \qiq   {\quad\Iff\quad}
\newcommand \srs   {\ \ \Rightarrow\ \ }
\newcommand \srsd   {\ \Rightarrow\ }
\newcommand \Iff   {\Leftrightarrow}
\newcommand \eps {\varepsilon}


\usepackage{csquotes}  % For \enquotes
\newcommand\q{\enquote}


\newcommand{\todo}[1]{{\color{red}\bf TO DO: #1}}
\newcommand{\mir}[1]{{\color{red}\bf AM: #1}}
\newcommand{\cp}[1]{{\color{red}\bf CP: #1}}
\newcommand{\jg}[1]{{\color{darkpastelgreen}\bf JG: #1}}
\newcommand{\amc}[1]{{\color{blue} #1}}
\newcommand{\NN}[1]{{\color{red}\bf NN: #1}}
%\def\red{\textcolor{red}}
%\def\blue{\textcolor{blue}}
\newcommand{\red}[1]{{\color{red} #1}}
\newcommand{\blue}[1]{{\color{blue} #1}}

\newcommand{\tocheck}[1]{{\color{magenta} #1}}      %Parts, which have been checked during partial re-read.
																									%The changes in these parts are highly probable.




\begin{document}
\title{Non-uniform ISS small-gain theorem for infinite networks}

\author{Andrii Mironchenko

\thanks{A. Mironchenko is with 
Faculty of Computer Science and Mathematics, University of Passau,
94030 Passau, Germany.
Email: andrii.mironchenko@uni-passau.de. Corresponding author.
}
\thanks{
A. Mironchenko has been supported by DFG, grant Nr. MI 1886/2-1.
}
}


% make the title area
\maketitle



%\IEEEpeerreviewmaketitle


\begin{abstract}
%ISS small-gain theorems for infinite networks state that if all the subsystems 
%are ISS with a uniform transient bound and with the gain operator satisfying the
 %so-called monotone limit property, then the network is ISS.
We introduce the concept of non-uniform input-to-state stability for networks,
which combines the uniform global stability together with the uniform attractivity of any finite number of modes of the system, but which does not guarantee the uniform convergence of all modes.
We show that given an infinite network of ISS subsystems, which do not have a uniform $\KL$-bound on the transient behavior, and if the gain operator satisfies the  bounded monotone invertibility property, then the whole network is non-uniformly ISS and its any finite subnetwork is uniformly ISS. 
%Furthermore, we show that in this case any finite subnetwork has uniform ISS property.
%As for finite networks the monotone bounded invertibility property for the gain operator is equivalent to strong small-gain condition, and since  
%for finite networks non-uniform ISS is equivalent to ISS, our non-uniform small-gain theorem can be seen as an alternative way to extend the uniform ISS small-gain theorem to infinite networks.
\end{abstract}

%\textbf{Keywords}: input-to-state stability, nonlinear systems, infinite-dimensional systems.

%\begin{keywords}
\begin{IEEEkeywords}
input-to-state stability, small-gain theorem, stability of networks, nonlinear systems, infinite-dimensional systems.
%PDE control, reaction-diffusion equation, saturated control, stabilization, attraction region.
\end{IEEEkeywords}
%\end{keywords}




\section{Introduction}


We are living in the worlds of networks, which grow steadily in size and in the number of couplings between individual agents.
Smart grids, connected vehicles, swarm robotics, and smart cities are particular examples of such networks, in which the participating agents may be plugged in and out from the network at any time.
Natural generalizations of such large-scale networks are infinite networks, which overapproximate and capture the essence of the original interconnections.
The complexity of such networks motivates to use the bottom-up approach and to establish stability for a complex system based on properties of its less complex components.

During the last 2 decades, a vast literature appeared, devoted to spatially invariant systems and/or linear systems consisting of an infinite number of finite-dimensional components, interconnected with each other by means of the same pattern \cite{BPD02,BaV05,BeJ17,CIZ09,JoB05b}, etc.
Recently, a research program has been initiated which aims at developing the methods and tools to analyze and control 
\emph{infinite networks composed of nonlinear infinite-dimensional systems of different nature, which are not necessarily spatially invariant}.
This program is based on the nonlinear small-gain methods and infinite-dimensional input-to-state stability (ISS) theory.


ISS theory has been initiated in \cite{Son89}, and has quickly become one of the pillars of nonlinear control theory, including robust stabilization, nonlinear observer design and analysis of large-scale networks, see \cite{KKK95,ArK01,Son08}. For the analysis of coupled systems, the ISS paradigm is especially fruitful in combination with the small-gain approach. 
In this method, the influence of any subsystem on other subsystems of a network is characterized by so-called gain functions. The gain operator constructed from these functions characterizes the interconnection structure of the network. 
The small-gain theorems state that if the gains are small enough (i.e., the gain operator satisfies a some sort of small-gain condition), the network is stable.%

Small-gain theorems originated within the input-output theory of linear systems, for an overview see \cite{DeV09}.
The small-gain technique was extended to nonlinear feedback input-output systems in \cite{Hil91, MaH92}.
ISS paradigm allowed to extend the nonlinear small-gain theorems to the couplings of 2 nonlinear state space systems in \cite{JTP94, JMW96}, and further to finite networks of input-to-state stable ODEs \cite{DRW07,DRW10}.


A substantial progress in the infinite-dimensional input-to-state stability (ISS) theory within the last years~\cite{JNP18,JSZ17,KaK16b,KaK19,MiW18b,TPT17,ZhZ18} (see~\cite{MiP20} for a recent survey on this topic) has created the basis which allows to extend the small-gain results to the finite and infinite networks of infinite-dimensional systems.
Small-gain results for finite networks of evolution equations in Banach spaces, both in trajectory and in Lyapunov formulations, have been developed in \cite{BLJ18,Mir19b,MiI15b,TWJ12,KaK18,KaK19b} (see \cite{Mir19b} for more details and references).

Small-gain analysis of infinite (not necessarily spatially invariant) networks is especially challenging since the gain operator, collecting the information about the internal gains, acts on an infinite-dimensional space, in contrast to finite networks of arbitrary nature. This calls for a careful choice of the infinite-dimensional state space of the overall network, and motivates the use of the theory of positive operators on ordered Banach spaces for the small-gain analysis.%

For networks consisting of \emph{exponentially} ISS systems, possessing exponential ISS Lyapunov functions with linear gains, it was shown in \cite{KMS19} that the whole network is exponentially ISS and there is a coercive exponential ISS Lyapunov function for the whole network  provided that the spectral radius of the gain operator is less than one. This result is tight and provides a complete generalization of \cite[Prop.~3.3]{DIW11} from finite networks to infinite ones. In~\cite{NMK20b} this result has been extended to networks of systems which are exponentially ISS with respect to closed sets, and in this form applied to the stability analysis of infinite time-variant networks, to consensus in infinite-agent systems, as well as to the design of distributed observers for infinite networks.%

Lyapunov-based ISS small-gain theorems have been reported in \cite{DaP19,DMS19a}.
In~\cite{DaP19}, ISS was shown for an infinite network of ISS systems provided that the internal gains capturing the influence of subsystems on each other are all uniformly less than identity, which is a very conservative condition. In~\cite{DMS19a} the Lyapunov-based small-gain results for infinite networks have been shown, under assumption of the existence of a linear path of strict decay for the gain operator, which is quite strong assumption, making the result not fully nonlinear.

The main motivation for this work is the paper \cite{MKG20}, where trajectory-based small-gain theorems for infinite networks consisting of nonlinear infinite-dimensional systems have been shown. These results show that if all the subsystems 
are ISS with a uniform $\KL$-transient bound and with the gain operator satisfying the  so-called monotone limit property, then the network is ISS. 
Furthermore, in \cite{MKG20} it was shown that a network consisting of subsystems which are merely uniformly globally stable (UGS), is again UGS provided that the gain operator satisfies a uniform small-gain condition, which is implied by (and probably is weaker than) the MLIM property.



\subsection{Contribution}

In this paper we introduce the concept of \emph{non-uniform input-to-state stability} for networks, which is equivalent to ISS in case of finite networks, but is weaker than ISS  for infinite networks.
\emph{We characterize in Proposition~\ref{prop:Criterion-non-uniform-ISS} the nonuniform ISS via uniform global stability together with a uniform in initial state but non-uniform in the modes of the network version of the asymptotic gain property.}
For finite networks non-uniform ISS is equivalent to the classical ISS property, but it is weaker than ISS for infinite networks.

\emph{Our main result is the non-uniform ISS small-gain theorem (Theorem~\ref{thm:nonuniform-ISS_SGT-infinite-interconnections}), showing that an infinite interconnection of ISS systems, which possess a common $\Kinf$-bound (but not necessarily a common $\KL$-bound) on the transient behavior of subsystems, is non-uniformly ISS provided that the gain operator satisfies a uniform small-gain condition.}

\emph{We show that under the same requirement any finite subnetwork of the infinite network has an ISS property, which makes our result applicable to ISS analysis of finite networks of an unknown size.}

As for finite networks the uniform small-gain condition for the gain operator is equivalent to strong small-gain condition (as shown in \cite{MKG20}), and since  
for finite networks non-uniform ISS is equivalent to ISS, our non-uniform ISS small-gain theorem can be seen as an alternative to \cite{MKG20} way to extend the uniform ISS small-gain theorem from finite to infinite networks.

Our results are valid for a very general class of control systems, including many classes of evolution PDEs with distributed and boundary control, time-delay systems, discrete-time systems, etc.

%\subsection{Notation}
\textbf{Notation.} 
We write $\R$ for the set of real numbers and $\Z$ for the set of integers. $\R_+$ and $\Z_+$ denote the sets of nonnegative reals and integers, respectively.%

We use the following classes of comparison functions:
{\allowdisplaybreaks
\begin{equation*}
\begin{array}{ll}
{\K} &:= \left\{\gamma:\R_+\rightarrow\R_+\left|\ \gamma\mbox{ is continuous, strictly} \right. \right. \\
&\phantom{aaaaaaaaaaaaaaaaaaa}\left. \mbox{ increasing and } \gamma(0)=0 \right\}, \\
{\K_{\infty}}&:=\left\{\gamma\in\K\left|\ \gamma\mbox{ is unbounded}\right.\right\},\\
{\LL}&:=\left\{\gamma:\R_+\rightarrow\R_+\left|\ \gamma\mbox{ is continuous and strictly}\right.\right.\\
&\phantom{aaaaaaaaaaaaaaaa} \text{decreasing with } \lim\limits_{t\rightarrow\infty}\gamma(t)=0\},\\
{\KL} &:= \left\{\beta:\R_+\times\R_+\rightarrow\R_+\left|\ \beta \mbox{ is continuous,}\right.\right.\\
&\phantom{aaaaaa}\left.\beta(\cdot,t)\in{\K},\ \beta(r,\cdot)\in {\LL},\ \forall t\geq 0,\ \forall r >0\right\}. \\
\end{array}
\end{equation*}
}
For a normed linear space $(W,\|\cdot\|_W)$ and any $r>0$, we write $B_{r,W} :=\{w \in W: \|w\|_W < r\}$ (the open ball of radius $r$ around $0$ in $W$). If the space $W$ is clear from the context, we simply write $B_r$.%

Throughout the paper, all considered vector spaces are vector spaces over $\R$.%





\section{Preliminaries}


\subsection{Control systems and their properties}
\label{sec:Control systems and their properties}

In this section we introduce the concept of a control system as well as the main stability properties.
\begin{definition}
\label{def:Time-set} 
A \emph{time set} $\T$ is a subgroup of $(\R, +)$.
\end{definition}

For any time set $\T$, let $\T_+$ be the set of nonnegative elements $\{t \in \T: t \geq 0\}$. 
By notational convention, when the time set $\T$ is understood from the context, all
intervals are assumed to be restricted to $\T$. Thus, for instance,
$[a, b) = \{t \in \T , a \leq t < b\}$.


We define the concept of a (time-invariant) system in the following way:
\begin{definition}
\label{Steurungssystem}
Consider the tuple $\Sigma=(\T,X,\Uc,\phi)$ consisting of 
%\index{state space}
%\index{space of input values}
%\index{input space}
\begin{enumerate}[label = (\roman*)]  
		\item A time set $\T$.
    \item A normed vector space $(X,\|\cdot\|_X)$, called the \emph{state space}, endowed with the norm $\|\cdot\|_X$.
    %\item A \emph{set of input values} $U$, which is a nonempty subset of a certain normed vector space.
    %\item \amc{A normed vector \emph{space of input values} $U$.}
    \item A normed vector \emph{space of inputs} $\Uc \subset \{u:\T_+ \to U\}$ endowed with a norm $\|\cdot\|_{\Uc}$, 
		where $\Uc$ is a linear subspace of $U^{\T_+}$.
		%where $U$ is a normed vector \emph{space of input values}. 
		
		We assume that the following two axioms hold:
                    
\emph{The axiom of shift invariance}: for all $u \in \Uc$ and all $\tau\geq0$ the time
shift $u(\cdot + \tau)$ belongs to $\Uc$ with \mbox{$\|u\|_\Uc \geq \|u(\cdot + \tau)\|_\Uc$}.
%the latter is used in the proof of Lemma 6.

\emph{The axiom of concatenation}: for all $u_1,u_2 \in \Uc$ and for all $t>0$ the concatenation of $u_1$ and $u_2$ at time $t$, defined by
\begin{equation}
\ccat{u_1}{u_2}{t}(\tau):=
\begin{cases}
u_1(\tau), & \text{ if } \tau \in [0,t], \\ 
u_2(\tau-t),  & \text{ otherwise},
\end{cases}
\label{eq:Composed_Input}
\end{equation}
belongs to $\Uc$.


    %\item A map $\phi:D_{\phi} \to X$, $D_{\phi}\subseteq \T_+ \times X \times \Uc$ (called \emph{transition map}), so that for all $(x,u)\in X \tm \Uc$ there is a $t>0$ so that $[0,t]\tm \{(x,u)\} \subset D_{\phi}$.
    \item A map $\phi:D_{\phi} \to X$, $D_{\phi}\subseteq \T_+ \times X \times \Uc$ (called \emph{transition map}), so that for all $(x,u)\in X \tm \Uc$ it holds that $D_{\phi} \cap \T_+ \times \{(x,u)\} = [0,t_m)\tm \{(x,u)\}$, for a certain $t_m=t_m(x,u)\in (0,+\infty]$.
		
		The corresponding interval $[0,t_m)$ is called the \emph{maximal domain of definition} of $t\mapsto \phi(t,x,u)$.
		
\end{enumerate}
The triple $\Sigma$ is called a \emph{(control) system}, if the following properties hold:

\begin{sysnum}
    \item\label{axiom:Identity} \emph{The identity property:} for every $(x,u) \in X \times \Uc$
          it holds that $\phi(0, x,u)=x$.
    \item \emph{Causality:} for every $(t,x,u) \in D_\phi$, for every $\tilde{u} \in \Uc$, such that $u(s) =
          \tilde{u}(s)$ for all $s \in [0,t]$ it holds that $[0,t]\tm \{(x,\tilde{u})\} \subset D_\phi$ and $\phi(t,x,u) = \phi(t,x,\tilde{u})$.
					
    \item \label{axiom:Continuity} \emph{Continuity:} for each $(x,u) \in X \tm \Uc$ the map $t \mapsto \phi(t,x,u)$ is continuous on its maximal domain of definition.%
		%\emph{Boundedness:} for each $(x,u) \in X \times \Uc$ the map $t \mapsto \phi(t,x,u)$ is bounded on compact subintervals of its maximal domain of definition.
        \item \label{axiom:Cocycle} \emph{The cocycle property:} for all
                  $x \in X$, $u \in \Uc$, for all $t,h \geq 0$ so that $[0,t+h]\tm \{(x,u)\} \subset D_{\phi}$, we have
$\phi(h,\phi(t,x,u),u(t+\cdot))=\phi(t+h,x,u)$.
\end{sysnum}

\end{definition}


\begin{definition}\label{def_forward_completeness} 
We say that a control system $\Sigma = (X,\Uc,\phi)$ is \emph{forward complete} if $D_\phi = \T_+ \tm X \tm \Uc$, i.e., $\phi(t,x,u)$ is well-defined for all $(x,u) \in X \tm \Uc$ and $t \geq 0$.
\end{definition}


If $\T=\R$, this class of systems encompasses control systems generated by ordinary differential equations (ODEs), switched systems, time-delay systems, many classes of partial differential equations (PDEs), important classes of boundary control systems and many other systems. 

For $\T=\Z$, this class includes infinite-dimensional discrete-time systems of the form
\begin{eqnarray}
x(k+1) = A (x(k),u(k)), \quad k\in\Z_+,
\label{eq:Discrete-time-system_nonlinear-with-inputs}
\end{eqnarray}
where $A: X \times U \to X$ is a nonlinear operator and $\Uc$ is the space $\ell_\infty(\Z_+,U)$, defined as a set 
of all $u:\Z_+ \to U$, such that $\|u\|_{\infty} := \sup_{k\in\Z_+}\|u(k)\|_U < \infty$. 
For each initial condition $x \in X$ and each input $u\in\Uc$ the solution of the system \eqref{eq:Discrete-time-system_nonlinear-with-inputs} exists and is unique for all times $k\in\Z_+$. Denoting this solution at time $k$ by $\phi(k,x,u)$, we see that 
\eqref{eq:Discrete-time-system_nonlinear-with-inputs}
gives rise to a forward complete infinite-dimensional discrete-time control system $\Sigma=(\Z_+, X,\Uc,\phi)$.

%
%In this paper, we work with the following definition of a control system (which provides all the features that are necessary for a global stability analysis).%
%
%\begin{definition}\label{def_controlsystem}
%Consider a triple $\Sigma = (X,\Uc,\phi)$ consisting of the following:%
%\begin{enumerate}  
%\item[(i)] A normed vector space $(X,\|\cdot\|_X)$, called the \emph{state space}.%
%\item[(ii)] A vector space $U$ of \emph{input values} and a normed vector \emph{space of inputs} $(\Uc,\|\cdot\|_{\Uc})$, where $\Uc$ is a linear subspace of $U^{\R_+}$. We assume that the following two axioms hold:%
%\begin{itemize}
%\item \emph{The axiom of shift invariance}: for all $u \in \Uc$ and all $\tau\geq0$, the time shift $u(\cdot + \tau)$ belongs to $\Uc$ with \mbox{$\|u\|_\Uc \geq \|u(\cdot + \tau)\|_\Uc$}.%
%\item \emph{The axiom of concatenation}: for all $u_1,u_2 \in \Uc$ and for all $t>0$ the concatenation of $u_1$ and $u_2$ at time $t$, defined by%
%\begin{equation}\label{eq_Composed_Input}
  %\ccat{u_1}{u_2}{t}(\tau) :=
%\begin{cases}
%u_1(\tau) & \text{ if } \tau \in [0,t], \\ 
%u_2(\tau-t)  & \text{ otherwise}
%\end{cases}
%\end{equation}
%belongs to $\Uc$.%
%\end{itemize}
%\item[(iii)] A map $\phi:D_{\phi} \to X$, $D_{\phi}\subseteq \R_+ \tm X \tm \Uc$ (called \emph{transition map}), so that for all $(x,u)\in X \tm \Uc$ it holds that $D_{\phi} \cap \R_+ \tm \{(x,u)\} = [0,t_m)\tm \{(x,u)\}$, for a certain $t_m=t_m(x,u)\in (0,+\infty]$. The corresponding interval $[0,t_m)$ is called the \emph{maximal domain of definition} of $t\mapsto \phi(t,x,u)$.%		
%\end{enumerate}
%The triple $\Sigma$ is called a \emph{(control) system}, if it satisfies the following axioms:%
%\begin{enumerate}
%\item[($\Sigma{1}$)]\label{axiom:Identity} \emph{The identity property:} for all $(x,u) \in X \tm \Uc$, it holds that $\phi(0,x,u) = x$.%
%\item[($\Sigma{2}$)]\emph{Causality:} for every $(t,x,u) \in D_\phi$ and $\tilde{u} \in \Uc$ such that $u(s) = \tilde{u}(s)$ for all $s \in [0,t]$, it holds that $[0,t]\tm \{(x,\tilde{u})\} \subset D_\phi$ and $\phi(t,x,u) = \phi(t,x,\tilde{u})$.%
%\item[($\Sigma{3}$)]\label{axiom:Continuity} \emph{Continuity:} for each $(x,u) \in X \tm \Uc$ the map $t \mapsto \phi(t,x,u)$ is continuous on its maximal domain of definition.%
%\item[($\Sigma{4}$)]\label{axiom:Cocycle} \emph{The cocycle property:} for all $x \in X$, $u \in \Uc$ and $t,h \geq 0$ so that $[0,t+h]\tm \{(x,u)\} \subset D_{\phi}$, we have $\phi(h,\phi(t,x,u),u(t+\cdot)) = \phi(t+h,x,u)$.%
%\end{enumerate}
%\end{definition}
%
%This class of systems encompasses control systems generated by ordinary differential equations, switched systems, time-delay systems,
%many classes of partial differential equations, important classes of boundary control systems and many other systems.%

%An important property of ordinary differential equations with Lipschitz continuous right-hand sides is that if the solution stays bounded over $[0,t)$, then it can be prolonged to $[0,t+\ep)$ for some $\ep>0$. Evolution equations in Banach spaces with bounded control operators and Lipschitz continuous right-hand sides have similar properties \cite[Thm.~4.3.4]{CaH98}; the same holds for many other classes of systems \cite[Ch.~1]{KaJ11b}. The next property, adopted from \cite[Def.~1.4]{KaJ11b}, formalizes this behavior for general control systems.%

For prolongation of solutions the following property is important, which is implied by forward completeness.
\begin{definition}\label{def_BIC} 
We say that a system $\Sigma = (\T,X,\Uc,\phi)$ satisfies the \emph{boundedness-implies-continuation (BIC) property} if for each $(x,u)\in X \tm \Uc$ such that the maximal existence time $t_m = t_m(x,u)$ is finite, for any given $M>0$ there exists $t \in [0,t_m)$ with $\|\phi(t,x,u)\|_X > M$.
\end{definition}

The main concept in this paper, describing a stability property of a control system, is the following:%

\begin{definition}\label{def_ISS}
A system $\Sigma = (\T,X,\Uc,\phi)$ is called \emph{(uniformly) input-to-state stable (ISS)} if there exist $\beta \in \KL$ and $\gamma \in \K$ such that for all $(t,x,u) \in D_{\phi}$ the following  holds:%
\begin {equation}\label{eq_iss}
  \|\phi(t,x,u)\|_X \leq \beta(\|x\|_X,t) + \gamma(\|u\|_{\Uc}).%
\end{equation}
\end{definition}

We need also the following property, implied by ISS:
\begin{definition}
A system $\Sigma = (\T,X,\Uc,\phi)$ is called \emph{uniformly globally stable (UGS)} if there exist $\sigma \in\Kinf$ and $\gamma \in \Kinf \cup \{0\}$ such that for all $(t,x,u) \in D_{\phi}$ the following inequality holds:%
\begin{equation}\label{eq_UGS}
  \|\phi(t,x,u)\|_X \leq \sigma(\|x\|_X) + \gamma(\|u\|_{\Uc}).%
\end{equation}
\end{definition}

%\begin{definition}
%A forward complete system $\Sigma = (X,\Uc,\phi)$ has the
%\emph{bounded input uniform asymptotic gain (bUAG) property} if there exists a $\gamma \in \Kinf \cup \{0\}$ such that for all $ \ep,r>0$ there is a time $\tau = \tau(\ep,r) \geq 0$ such that for all $u \in \overline{B}_{r,\Uc}$ and $x \in \overline{B}_{r,X}$ the following implication holds:
%\begin{equation}
%\label{eq_AG_Absch}
  %t \geq \tau\ \quad \Rightarrow \quad \|\phi(t,x,u)\|_X \leq \ep + \gamma(\|u\|_{\Uc}).%
%\end{equation}
%\end{definition}
%
%UGS and bUAG properties are extensions of the global Lyapunov stability and uniform global attractivity to the systems with inputs.



%The following lemma (see \cite[Lem.~3.7]{Mir19b}) shows how the bUAG property can be ``upgraded'' to the ISS property.%
%
%\begin{lemma}\label{lem_UGS_and_bUAG_imply_UAG}
%%Let $\Sigma = (X,\Uc,\phi)$ be a control system with the BIC property. If $\Sigma$ is UGS and has the bUAG property, then $\Sigma$ is forward complete, has the UAG property and is ISS.
%Let $\Sigma = (X,\Uc,\phi)$ be a control system with the BIC property. If $\Sigma$ is UGS and has the bUAG property, then $\Sigma$ is forward complete and ISS.
%\end{lemma}


\subsection{Infinite interconnections}
\label{sec:Infinite interconnections}

Recall the concept of infinite interconnections, indexed by some nonempty set $I$, as developed in \cite{MKG20} and inspired by \cite[Definition 3.3]{KaJ07}.

For each $i \in I$, let $(X_i,\|\cdot\|_{X_i})$ be a normed vector space which stays for the state space of the $i$-th system $\Sigma_i$. Before we can specify the space of inputs for $\Sigma_i$, we construct the overall state space. In what follows, we denote the functions with domain $Q\subset I$ via $(x_i)_{i \in Q}$. 

Define the space
\begin{equation*}
  X_Q := \Bigl\{ (x_i)_{i \in Q} : x_i \in X_i,\ \forall i \in Q \mbox{ and } \sup_{i \in Q} \|x_i\|_{X_i} < \infty \Bigr\},%
\end{equation*}
which becomes a (real) Banach space with the norm%
\begin{equation*}
  \|x\|_{X_Q} := \sup_{i\in Q}\|x_i\|_{X_i}.%
\end{equation*}
Similarly we define the space $X_{Q\backslash\{i\}}$. We identify this space with the space
\[
X_{\neq i,Q}:=\Bigl\{ (x_i)_{i \in I\backslash\{i\}} \in X_{\neq i} : \  x_i=0,\ i \notin Q\Bigr\}.
\]
The state space for the network we define as $X:=X_I$. Also we use the shorthand notation $X_{\neq i}:=X_{\neq i,I}$.
Note that for each $Q\subset I$ the space $X_{\neq i,Q}$  is a linear subspace in $X_{\neq i}$.

%\begin{equation}
  %X := X_I:= \Bigl\{ (x_i)_{i \in I} : x_i \in X_i,\ \forall i \in I \mbox{ and } \sup_{i \in I} \|x_i\|_{X_i} < \infty \Bigr\}.
%\label{eq:State-space-interconnection}
%\end{equation}
%and becomes a normed vector space with the norm%
%\begin{equation*}
  %\|x\|_X := \sup_{i\in I}\|x_i\|_{X_i}.%
%\end{equation*}
%
%Further define for each $i \in I$ the normed vector space $X_{\neq i}$ as in \eqref{eq:State-space-interconnection}, but for the restricted index set $I \backslash \{i\}$ (i.e., $X_{\neq i}:=X_{I\backslash\{i\}}$). 
%%Then $X_{\neq i}$ can be identified with the closed linear subspace $\{ (x_j)_{j \in I} \in X : x_i = 0 \}$ of $X$.%

Consider for each $i \in I$ a control system of the form%
\begin{equation}
\label{eq:Sigma-i}
  \Sigma_i = (\T, X_i,\PC_b(\T_+,X_{\neq i}) \tm \Uc,\bar{\phi}_i),%
\end{equation}
where $\PC_b(\T_+,X_{\neq i})$ is the space of globally bounded piecewise continuous functions, with the norm $\|w\|_{\infty} = \sup_{s \geq 0}\|w(s)\|_X$. The norm on $\PC_b(\T_+,X_{\neq i}) \tm \Uc$ we define by%
\begin{equation}\label{eq_product_input_norm}
  \|(w,u)\|_{\PC_b(\T_+,X_{\neq i}) \tm \Uc} := \max\left\{ \|w\|_{\infty}, \|u\|_{\Uc} \right\}.%
\end{equation}
Here we assume that $\Uc \subset U^{\T_+}$ for some normed space $U$, and $\Uc$ satisfies the axioms of shift invariance and concatenation. Then, by the definition of $\PC_b(\T_+,X_{\neq i})$ and the norm \eqref{eq_product_input_norm}, these axioms are also satisfied for the product space $\PC_b(\T_+,X_{\neq i}) \tm \Uc$.%

\begin{definition}
\label{def_interconnection}
Given the control systems $(\Sigma_i)_{i \in I}$ as above, we call a control system of the form $\Sigma = (\T, X,\Uc,\phi)$ the \emph{(feedback) interconnection} of systems $(\Sigma_i)_{i\in I}$ if the following holds:%
\begin{enumerate}
\item[(i)] The components $\phi_i$ of the transition map $\phi:D_{\phi} \rightarrow X$ satisfy%
\begin{equation*}
  \phi_i(t,x,u) = \bar{\phi}_i(t,x_i,(\phi_{\neq i},u)) \mbox{\quad for all\ } (t,x,u) \in D_{\phi},%
\end{equation*}
where $\phi_{\neq i}(\cdot) = (\phi_j(\cdot,x,u))_{j \in I \backslash \{i\}}$.\footnote{By the causality axiom, we can assume that $\phi_{\neq i}$ is globally bounded, since $\bar{\phi}_i(t,x_i,(\phi_{\neq i},u))$ does not depend on the values $\phi_{\neq i}(s)$ with $s > t$, and on the compact interval $[0,t]$, $\phi_{\neq i}$ is bounded because it is continuous.}
\item[(ii)] $\Sigma$ has the BIC property.%
\end{enumerate}
We call $X_{\neq i}$ the space of \emph{internal input values} of system $\Sigma_i$ and $\PC_b(\T_+,X_{\neq i})$ the space of \emph{internal inputs}. Furthermore, we call $\Sigma_i$ the \emph{$i$-th subsystem} of $\Sigma$.%
\end{definition}

To measure the influence of each subsystem at any other subsystem, we define the ISS property for subsystems in the following form:
\begin{definition}\label{def_subsys_iss_semimax}
Given the spaces $(X_j,\|\cdot\|_{X_j})$, $j\in I$, and the system $\Sigma_i$ for a fixed $i \in I$, we call $\Sigma_i$  \emph{input-to-state stable (ISS) (in semimaximum formulation)} if $\Sigma_i$ if forward complete and there are $\gamma_{ij},\gamma_j \in \K \cup \{0\}$ for all $j \in I$ with $\gamma_{ii}=0$ and $\beta_i \in \KL$ such that for all initial states $x_i \in X_i$, all internal inputs $w_{\neq i} = (w_j)_{j\in I \backslash \{i\}} \in \PC_b(\T_+,X_{\neq i})$, all external inputs $u \in \Uc$ and $t \geq 0$:%
\begin{align}
  \|\bar{\phi}_i&(t,x_i,(w_{\neq i},u))\|_{X_i}\nonumber\\
	&\leq \beta_i(\|x_i\|_{X_i},t) + \sup_{j \in I}\gamma_{ij}(\|w_j\|_{[0,t]}) + \gamma_i(\|u\|_{\Uc}).%
\label{eq:ISS-subsystems}
\end{align}
%Here we assume that the functions $\gamma_{ij}$ are such that $\sup_{j \in I}\gamma_{ij}(r) < \infty$ for every $r \geq 0$ (implying that the sum on the rhs is finite) and $\gamma_{ii} = 0$.%
%Here we assume by convention that $\gamma_{ii} = 0$.%
\end{definition}

The functions $\gamma_{ij}$ are called \emph{internal gains} and $\gamma_i$ is called \emph{external gain}.

Assuming that all systems $\Sigma_i$, $i\in I$, are ISS, we can define a nonlinear monotone operator $\Gamma_{\otimes}:\ell_{\infty}(I)^+ \rightarrow \ell_{\infty}(I)^+$, called \emph{gain operator}, from the gains $\gamma_{ij}$ as follows:%
\begin{equation}
\label{eq:Gain-operator-semimax}
  \Gamma_{\otimes}(s) := \bigl(\sup_{j\in I}\gamma_{ij}(s_j)\bigr)_{i\in I},\quad s = (s_i)_{i\in I} \in \ell_{\infty}(I)^+.%
\end{equation}

$\Gamma_{\otimes}$ is well-defined provided the following assumption holds:
\begin{assumption}\label{ass_gammamax_welldef}
For every $r>0$, we have%
\begin{equation*}
  \sup_{(i,j) \in I^2}\gamma_{ij}(r) < \infty.
\end{equation*}
\end{assumption}

For small-gain analysis we need the following property of nonlinear gain operators, considered, e.g., in \cite{DRW07,MKG20}:
\begin{definition}
We say that $\id - \Gamma_{\otimes}$ has the \emph{monotone bounded invertibility (MBI) property} if there exists $\xi \in \Kinf$ such that for all $v,w \in \ell_{\infty}(I)^+$
\begin{equation*}
  (\id - \Gamma_{\otimes})(v) \leq w \quad \Rightarrow \quad \|v\|_{\ell_{\infty}(I)} \leq \xi(\|w\|_{\ell_{\infty}(I)}).%
\end{equation*}
\end{definition}

In \cite{MKG20} the following small-gain criterion for MBI property has been shown (in a more general setting):
\begin{proposition}\label{prop:criteria-MBI-without-unit}
The following conditions are equivalent:
\begin{enumerate}[label=(\roman*)]
\item\label{itm:MBI-criterion-without-unit-1} $\id - \Gamma_{\otimes}$ satisfies the MBI property.
\item\label{itm:MBI-criterion-without-unit-2} The \emph{uniform small-gain condition} holds: There is $\eta \in \Kinf$ such that
\begin{equation}
\label{eq:uSGC-dist-form}
  \hspace{-4mm}\dist(\Gamma_{\otimes}(x) - x,\ell_{\infty}(I)^+) \geq \eta(\|x\|_X), \ x \in \ell_{\infty}(I)^+.%
\end{equation}
\end{enumerate}
\end{proposition}
If the network is finite (i.e. if $I$ is of finite cardinality), then uniform small-gain condition for $\Gamma_\otimes$ holds if and only if $\Gamma_\otimes$ satisfies the strong small-gain condition, see \cite[Proposition 14]{MKG20}.

 




\section{Non-uniform ISS small-gain theorem}
\label{sec:Non-uniform ISS small-gain theorem}


In this section, we prove \emph{a small-gain theorem for the non-uniform ISS property} which constitutes another way to generalize the ISS small-gain theorem from finite to infinite networks.%

In this section $I$ is a given nonempty index set.
\begin{definition}
\label{def:Non-uniform-ISS} 
Let $\Sigma_i:=(\T, X_i,\PC_b(\T_+,X_{\neq i}) \tm \Uc,\bar{\phi}_i)$, $i\in I$ be control systems. Assume that the interconnection $\Sigma=(\T, X,\Uc,\phi)$ is well-defined and forward complete.

We call the network $\Sigma$ \emph{non-uniformly input-to-state stable (non-uniformly ISS)}, if there are $(\tilde{\beta}_i)_{i\in I } \subset \KL$, $\tilde{\sigma} \in\Kinf$ and $\gamma\in\Kinf$, such that 
\begin{eqnarray}
\tilde{\beta}_i(r,t)\leq \tilde{\sigma}(r), \quad r\in\R_+,\ t\geq 0,\ i\in I,
\label{eq:Bounds_on_beta_coupled-system}
\end{eqnarray}
and the modes of the coupled system for each $i\in I$ satisfy for all $t\geq 0$, $x\in X$ and $u\in\Uc$ the following estimate:
\begin{eqnarray}
\|\phi_i(t,x,u)\|_{X_i} \leq \tilde{\beta}_i(\|x\|_X,t) + \gamma(\|u\|_{\Uc}).
\label{eq:non-uniform-ISS}
\end{eqnarray}
\end{definition}

We need the following criterion of non-uniform ISS property for infinite networks, which is a counterpart of the criterion for the ISS property \cite[Lemma 8]{MiW18b}.
\begin{proposition}
\label{prop:Criterion-non-uniform-ISS} 
Let $\Sigma_i:=(\T, X_i,\PC_b(\T_+,X_{\neq i}) \tm \Uc,\bar{\phi}_i)$, $i\in I$ be control systems. Assume that the interconnection $\Sigma=(\T,X,\Uc,\phi)$ is well-defined.
Then $\Sigma$ is non-uniformly ISS if and only if 
$\Sigma$ is uniformly globally stable and there is $\hat{\gamma}\in\Kinf$ such that for all $\varepsilon,r>0$ and for each $i\in I$ there is $\tau_i(\ep,r)>0$, such that
\begin{align}
\|x\|_X\leq r\ &\wedge \|u\|_\Uc \leq r \ \wedge t\geq\tau_i(\ep,r)\nonumber\\
& \qrq \|\phi_i(t, x, u)\|_{X_i} \leq \ep + \hat{\gamma}(\|u\|_\Uc).
\label{eq:nonuniform-in-i-UAG-bounded-inputs}
\end{align}
\end{proposition}

\begin{proof}
\q{$\Rightarrow$}. Let $\Sigma$ be non-uniformly ISS. Then
\begin{align*}
\|\phi(t,&x,u)\|_{X} =\sup_{i\in I}\|\phi_i(t,x,u)\|_{X_i}\\
 &\leq \sup_{i\in I}\tilde{\beta}_i(\|x\|_X,t) + \gamma(\|u\|_{\Uc}) \le \hat{\sigma}(\|x\|_X) + \gamma(\|u\|_{\Uc}),
\end{align*}
which shows the UGS property of $\Sigma$.

To obtain \eqref{eq:nonuniform-in-i-UAG-bounded-inputs}, take $\hat{\gamma}:=\gamma$, and pick $\tau_i(\ep,r)$ for each $\varepsilon,r>0$ and each $i\in I$ such that $\beta_i(r,\tau_i(\ep,r))\leq \varepsilon$. As $\beta_i\in\KL$, such $\tau_i(\ep,r)$ always exists. 

\q{$\Leftarrow$}. As a well-posed interconnection, $\Sigma$ has BIC property. In combination with the UGS property, this implies forward-completeness. Furthermore, by UGS property of $\Sigma$, there are $\sigma_{UGS},\gamma_{UGS}\in\Kinf$:
\begin{align}
&\hspace{-2mm} i\in I \ \wedge \ x\in X\ \wedge u\in \Uc \ \wedge\ t\geq0\nonumber\\
& \hspace{-2mm}\srs \|\phi_i(t, x, u)\|_{X_i} \leq \sigma_{UGS}(\|x\|_X) + \gamma_{UGS}(\|u\|_\Uc).
\label{lem7:help}
\end{align}
Pick any $\varepsilon,r>0$. From the above estimate we have for all $i\in I$, $t\geq 0$ 
and any $x\in X$, $u\in \Uc$ with $\|x\|_X \leq r\leq \|u\|_\Uc$ that
\begin{eqnarray}
\|\phi_i(t, x, u)\|_{X_i} \leq \sigma_{UGS}(\|u\|_\Uc) + \gamma_{UGS}(\|u\|_\Uc).
\label{eq:UGS-for-large-u}
\end{eqnarray}
Defining $\gamma(r):=\max\{\sigma_{UGS}(r) + \gamma_{UGS}(r), \hat{\gamma}(r)\}$, for all $r\geq 0$, and combining \eqref{eq:UGS-for-large-u} with 
\eqref{eq:nonuniform-in-i-UAG-bounded-inputs}, we obtain the following convergence estimate for arbitrary inputs (where $\tau_i(\ep,r)$ is as in the assumptions of the lemma):
\begin{align}
i\in I\ \wedge\ &\|x\|_X\leq r\ \wedge u\in \Uc \ \wedge\ t\geq\tau_i(\ep,r) \nonumber\\
&\qrq \|\phi_i(t, x, u)\|_{X_i} \leq \ep + \gamma(\|u\|_\Uc).
\label{eq:nonuniform-in-i-UAG}
\end{align}

%Due to uniform global stability, there exist $\gamma,\sigma \in \Kinf$ such
%that for all $i\in I$, all $ t\geq 0$, all $x \in \overline{B_r}$ and all $ u \in \Uc$ we have
%\begin{equation}
    %\label{lem7:help}
%\|\phi_i(t,x,u)\|_{X_i} \leq \sigma(r) + \gamma(\|u\|_{\Uc}).
%\end{equation}

%Pick any $t^* \in \T_+$: $t^*>0$.

Fix arbitrary $r \in \R_+$ and define $\eps_n:= 2^{-n}  \sigma_{UGS}(r)$, for all $n \in \Z_+$. Due to \eqref{eq:nonuniform-in-i-UAG}, there exists a sequence of times
$\tau_{i,n}:=\tau_i(\eps_n,r)$, $i\in I$, $n\in\Z_+$, which we may without loss of generality assume
to be strictly increasing in $n$, such that for all $i\in I$, $x \in \overline{B_r}$, $u \in \Uc$ and $n\in\Z_+$
\[
t \geq \tau_{i,n} \qrq \|\phi_i(t,x,u)\|_{X_i} \leq \eps_n + \gamma(\|u\|_{\Uc}).
\]
From \eqref{lem7:help} we see that we may set $\tau_{i,0} := 0$ for all $i\in I$.
Define $w_i(r,\tau_{i,n}):=\eps_{n-1}$, for $i\in I$, $n \in \Z_+\setminus\{0\}$, and $w_i(r,0):=2\eps_0=2\sigma_{UGS}(r)$.

Now extend the definition of $w_i$ 
%for $t \in \R_+ \backslash \{\tau_n,n \in \N\}$ so that 
to a function
$w_i(r,\cdot) \in \LL$, for any $i\in I$. 
%All such functions satisfy the estimate \eqref{iss_sum}, because for all 
We obtain for $t \in (\tau_{i,n},\tau_{i,n+1})$, $n=0,1,\ldots$ and $x\in B_r$
%it holds 
that
\[
\|\phi_i(t,x,u)\|_{X_i} \leq \eps_n + \gamma(\|u\|_{\Uc})< w_i(r,t) + \gamma(\|u\|_{\Uc}).
\]
Doing this for all $r \in \R_+$ we obtain the definition of the functions $w_i$, $i\in I$.

Now for each $i\in I$ define $\hat \beta_i(r,t):=\sup_{0 \leq s \leq r}w_i(s,t) \geq
w_i(r,t)$ for $(r,t) \in \R_+ \times \R_+$. From this definition it follows that, 
for each $t\geq 0$, $\hat\beta_i(\cdot,t)$ is 
%continuous and $\beta(\cdot,t) \in \Kinf$.
nondecreasing in the first argument and $\hat\beta_i(r,\cdot)$ is decreasing in the second argument for each $r>0$ as
every $w_i(r,\cdot) \in \LL$.
Moreover, for each fixed $t\geq0$, $\hat \beta_i(r,t) \leq \sup_{0 \leq s \leq r}w_i(s,0)=2\sigma_{UGS}(r)$, which implies that $\hat\beta$ is continuous in the first argument at $r=0$ for any fixed $t\geq0$ and also $\hat \beta_i(0,t)=0$ for any $t\geq 0$.

By \cite[Proposition 9]{MiW19b}, $\hat\beta_i$ can be upper bounded by certain $\tilde{\beta}_i\in \KL$, and 
\eqref{eq:non-uniform-ISS} is satisfied with such $\tilde{\beta}_i$.
Furthermore, there is $\tilde{\sigma}\in\Kinf$, such that \eqref{eq:Bounds_on_beta_coupled-system} holds (choose the same function $\omega$ for all $i\in I$ in the proof of \cite[Proposition 9]{MiW19b}). 
\end{proof}

%\begin{remark}
%\label{rem:BIC-property-is-enough-for-nonuniform-ISS} 
%Forward completeness is assumed in Definition~\ref{def:Non-uniform-ISS}  and Proposition~\ref{prop:Criterion-non-uniform-ISS} 
%only for simplicity of formulation. Proposition~\ref{prop:Criterion-non-uniform-ISS} remains valid if we assume only the BIC property of the interconnection.
%\end{remark}
%

For finite networks ISS coincides with non-uniform ISS.
\begin{proposition}
\label{prop:ISS-of-finite-networks} 
Let $|I|<\infty$ and let $\Sigma_i:=(\T, X_i,\PC_b(\T_+,X_{\neq i}) \tm \Uc,\bar{\phi}_i)$, $i\in I$ be control systems. Assume that the interconnection $\Sigma=(\T, X,\Uc,\phi)$ is well-defined.
Then $\Sigma$ is ISS if and only if $\Sigma$ is non-uniformly ISS.
\end{proposition}

\begin{proof}
As $\Sigma$ has BIC property as a well-posed interconnection, ISS implies forward completeness. 
Clearly, ISS implies non-uniform ISS. The converse follows by setting $\beta(r,t):=\max_{i\in I} \tilde{\beta}_i(r,t)$. As $|I|<\infty$, $\beta \in\KL$, and since $\|\phi(t,x,u)\|_{X}=\max_{i\in I}\|\phi_i(t,x,u)\|_{X_i}$ the estimate \eqref{eq:non-uniform-ISS} implies 
for all $x\in X$,  $u\in \Uc$ and  $t\geq0$
\begin{eqnarray*}
\|\phi(t,x,u)\|_{X} \leq \beta(\|x\|_X,t) + \gamma(\|u\|_{\Uc}),
%\label{eq:non-uniform-ISS}
\end{eqnarray*}
which shows the ISS property of $\Sigma$.
\end{proof}

We recall a technical lemma, shown in \cite{Mir19b}:
\begin{lemma}\label{lem:LimSupEstimate}
Let $g:\R_+\to\R^p_+$, $p\in\Z_+$ be a globally bounded function and let $f:\R_+\to\R_+$ be an unbounded monotonically increasing function. Then%
\begin{eqnarray}
\lim_{t\to\infty} \sup_{s\geq f(t)} g(s)  =  \lim_{t\to\infty} \sup_{s\geq t} g(s). 
\label{eq:LimSupEstimate}
\end{eqnarray}
\end{lemma}


The ISS small-gain theorem in \cite[Theorem 2]{MKG20} states that if all the subsystems are ISS with a uniform transient bound (i.e. there is $\beta\in\KL$: $\beta_i \leq \beta$ for all $i\in I$ pointwise) and with the gain operator satisfying the so-called monotone limit property (which implies MBI property), then the network is ISS.
Next we show that if all the subsystems are ISS, have a uniform $\Kinf$-bound (\emph{but not necessarily a uniform $\KL$-bound}) on the transient behavior, and the gain operator satisfies merely MBI property (= uniform small-gain condition), then the network has non-uniform input-to-state stability property.

If the subsystems of the network are ISS, but do not have uniform $\KL$-bounds from above for the transient behavior, it is not possible to guarantee ISS of the network. However, if the gains satisfy the small-gain condition, the overall network will have the non-uniform ISS property.

\begin{theorem}[Nonuniform ISS Small-gain theorem]
\label{thm:nonuniform-ISS_SGT-infinite-interconnections} 
Let $\Sigma_i:=(\T, X_i,\PC_b(\T_+,X_{\neq i}) \tm \Uc,\bar{\phi}_i)$, $i\in I$ be forward complete control systems, satisfying the ISS estimates as in 
Definition~\ref{def_subsys_iss_semimax}. Let also the interconnection $\Sigma=(\T, X,\Uc,\phi)$ be well-defined and the following conditions hold:%
\begin{enumerate}[label=(\roman*)]
	\item\label{itm:nonuniform-ISS-SGT-Ass1} There exist $\gamma \in\K$ and $\sigma \in\K$ such that%
\begin{eqnarray}
\beta_i(r,t)\leq \sigma(r),\  \gamma_i(r) \leq \gamma(r),\  r\in\R_+,\ t\geq 0,\ i\in I.
\label{eq:Bounds_on_beta_and_external_gamma-nonuniform}
\end{eqnarray}
	\item\label{itm:nonuniform-ISS-SGT-Ass2} Assumption \ref{ass_gammamax_welldef} is satisfied for the operator $\Gamma_{\otimes}$ defined via the gains $\gamma_{ij}$ from (i) and $\Gamma_{\otimes}$ has the monotone bounded invertibility property.
	%, that is there is a $\xi\in \Kinf$ such that for all $w,v \in \ell_\infty^+$
  %\begin{equation}
    %\label{eq:ineqbound_ISS-nonuniform}
    %(\id-\Gamma^{ISS}_\otimes)(w)\leq v  \qrq \|w\|_{\ell_\infty}\leq \xi(\|v\|_{\ell_\infty}). 
  %\end{equation}
	\item\label{itm:nonuniform-ISS-SGT-Ass3} for each $i\in I$ only finitely many elements of $(\gamma_{ij})_{j\in I}$ are nonzero.
\end{enumerate} 
Then $\Sigma$ is non-uniformly ISS.
%
%
 %there are $(\tilde{\beta}_i)_{i\in I } \subset \KL$ and $\tilde{\sigma} \in\Kinf$ such that 
%\eqref{eq:Bounds_on_beta_coupled-system} holds and 
%the modes of the coupled system for each $i\in I$ satisfy for all $t\in\T_+$, $x\in X$ and $u\in\Uc$ the nonuniform ISS estimate
%\eqref{eq:non-uniform-ISS}.
\end{theorem}


\begin{proof}
As some of the steps are similar to those in the proof of ISS small-gain theorem in \cite{MKG20}, we refer to those parts of 
\cite{MKG20} and instead concentrate on the novelties appearing in the non-uniform case.

\noindent\textbf{UGS.} 
As all $\Sigma_i$ are ISS with corresponding $\beta_i$ and gains $\gamma_{ij}$ and $\gamma_i$, $i,j\in I$, in view of 
assumption~\ref{itm:nonuniform-ISS-SGT-Ass1} we have that 
for all initial states $x_i \in X_i$, all internal inputs $w_{\neq i} = (w_j)_{j\in I \backslash \{i\}} \in \PC_b(\T_+,X_{\neq i})$, all external inputs $u \in \Uc$ and $t \geq 0$ we have
\begin{align*}
  \|\bar{\phi}_i&(t,x_i,(w_{\neq i},u))\|_{X_i} \\
	&\leq \sigma(\|x_i\|_{X_i}) + \sup_{j \in I}\gamma_{ij}(\|w_j\|_{[0,t]}) + \gamma_i(\|u\|_{\Uc}).%
\end{align*}
As the gain operator satisfies the MBI property, we obtain forward completeness and UGS property of the network $\Sigma$ by application of the UGS small-gain theorem from \cite[Theorem 1]{MKG20}.

%
%
%We use the notation \eqref{eq:phi_neq_i} for $\phi_i$ and $\phi_{\neq i}$.
%%As in the proof of Theorem~\ref{thm:UGS_SGT-infinite-interconnections}, define $\phi_i:=\phi_i(\cdot,x,u)$ and 
%%$\phi_{\neq i} = (\phi_1,\ldots,\phi_{i-1}, \phi_{i+1},\ldots,\phi_n)$, $i=1,\ldots,n$.
%As $\Sigma$ is a well-defined and forward complete interconnection, we have $\phi_i(t,x,u) = \bar{\phi}_i(t,x_i,(\phi_{\neq i}, u))$ for all $i\in I$ and $t\geq 0$.%
%
%Pick any $r>0$, any $u\in\Uc$ with $\|u\|_\Uc\leq r$ and any $x \in B_r$. As $\Sigma$ is UGS, the estimate \eqref{GSAbschaetzung} is valid for some $\sigma_{UGS},\gamma_{UGS}\in\Kinf$ and it holds that%
%\begin{eqnarray}
%\|\phi(t,x,u)\|_X \leq \mu(r):=\sigma_{UGS}(r) + \gamma_{UGS}(r),\quad t\geq 0.%
%\label{eq:UGS_our_case-nonuniformISS}
%\end{eqnarray}
%On the other hand, for all $i\in I$ in view of the cocycle property it holds for any $t,\tau\geq 0$ that%
%\begin{align}\label{eq:Cocycle_property-nonuniformISS}
%\begin{split}
%\phi_i(t+\tau,x, u) &= \bar{\phi}_i(t+\tau,x_i,(\phi_{\neq i}, u)) \\
%&= \bar{\phi}_i\big(\tau,\bar{\phi}_i\big(t,x_i,(\phi_{\neq i}, u)\big),\big(\phi_{\neq i}(\cdot + t), u(\cdot + t)\big)\big).%
%\end{split}
%\end{align}
%Note that due to the axiom of shift invariance, it holds that $u(t+\cdot) \in\Uc$.%
%
%In view of \eqref{eq:UGS_our_case-nonuniformISS} we have for all $i\in I$ that%
%\begin{eqnarray}\label{eq:Bound_on_state-nuISS}
  %\|\bar{\phi}_i(t,x_i,(\phi_{\neq i},u))\|_{X_i} = \|\phi_i(t,x, u)\|_{X_i} \leq \|\phi(t,x, u)\|_{X} \leq \mu(r)%
%\end{eqnarray}
%and%
%\begin{eqnarray}\label{eq:Bound_on_internal_inputs-nuISS}
  %\|\phi_{\neq i}\|_\infty \leq \mu(r).%
%\end{eqnarray}
%Furthermore, as $\sigma_{UGS}(r)\geq r$ for all $r\in\R_+$ (this follows from \eqref{GSAbschaetzung} by setting $u:=0$ and $t:=0$),
%we also have%
%\begin{eqnarray}\label{eq:Bound_on_external_inputs-nuISS}
  %\|u\|_\Uc \leq r\leq \sigma_{UGS}(r)\leq \mu(r).%
%\end{eqnarray}
\noindent\textbf{The estimate \eqref{eq:non-uniform-ISS}.} 
As $\Sigma$ is the interconnection of $(\Sigma_i)_{i\in I}$ and is forward complete, we have $\phi_i(t,x,u) = \bar{\phi}_i(t,x_i,(\phi_{\neq i},u))$ for all $(t,x,u) \in \T_+ \tm X \tm \Uc$ and $i \in I$, with the notation from Definition \ref{def_interconnection}.%

Pick any $r \geq 0$, any $u \in \overline{B}_{r,\Uc}$ and any $x \in \overline{B}_{r,X}$. As $\Sigma$ is UGS, there are $\sigma^{\UGS},\gamma^{\UGS} \in \Kinf$ 
such that%
\begin{equation}
\label{eq:mu-definition}
  \|\phi(t,x,u)\|_X \leq \sigma^{\UGS}(r) + \gamma^{\UGS}(r) =: \mu(r) \quad \forall t \geq 0.
\end{equation}

Given $r\geq 0$, $\ep>0$, and $i\in I$, by the ISS property of $\Sigma_i$  choose $\tau^*_i = \tau^*_i(\ep,r) \geq 0$ (depending on $i$)  
such that $\beta_{i}(\mu(r),\tau^*_i) \leq \ep$. Using the cocycle property and arguing as in the proof of the ISS small-gain theorem in semimaximum formulation in \cite[Theorem 2]{MKG20}, we obtain:
\begin{align}\label{eq_firstest-NU-ISS-SGT-1}
%\normalsize
\begin{split}
 &x \in \overline{B}_{r,X} \ \wedge \  u \in \overline{B}_{r,\Uc} \ \wedge \ \tau \geq \tau^*_i \ \wedge \ t \geq 0 \\
	&\Rightarrow \|\phi_i(t+\tau,x,u)\|_{X_i} 
	%\leq \beta_i(\|\bar{\phi}_i(t,x_i,(\phi_{\neq i},u))\|_{X_i},\tau) \\
	                                 %&\qquad + \sup_{j \in I} \gamma_{ij}( \|\phi_j\|_{[t,t+\tau]} ) + \gamma_i(\|u(\cdot+t)\|_{\Uc}) \\
																	%&\leq \beta_{\max}(\|\phi(t,x,u)\|_X,\tau^*) + \sup_{j \in I}\gamma_{ij}(\|\phi_j\|_{[t,\infty)}) + \gamma_i(\|u\|_{\Uc}) \\
																	%&
																	\leq \ep {+} \sup_{j \in I}\gamma_{ij}(\|\phi_j\|_{[t,\infty)}) {+} \gamma_i(\|u\|_{\Uc}).%
\end{split}
\end{align}


%
%By the ISS property of $\Sigma_i$, there is a time $\tau^*_i(\ep,r)$ (depending on $i$) so that 
%\begin{align}
%\|x\|_X\leq r\ \wedge\ \|u\|_{\Uc}\leq r \ \wedge\ \tau\geq \tau^*_i &  \nonumber\\
 %\qrq   \|\phi_i(t+\tau,x , u)\|_{X_i} 
%&\leq \ep + \sup_{j\in I\backslash\{i\}}\gamma_{ij}\left(\left\|\phi_{j,[t,+\infty)}\right\|_{\infty}\right) + \gamma_i(\|u_{[t,+\infty)}\|_{\Uc})
%\nonumber\\
%&\leq \ep + \sup_{j\in I\backslash\{i\}}\gamma_{ij}\left(\left\|\phi_{j,[t,+\infty)}\right\|_{\infty}\right) + \gamma_i(\|u\|_{\Uc}).
%\label{eq:nuISS_SGT_tmp_estimate_1}
%\end{align}

Pick any $k \in \Z_+$ and define 
\begin{eqnarray}
B(r,k) := \overline{B}_{r,X} \tm \{ u \in \Uc : \|u\|_{\Uc} \in [2^{-k}r,2^{-k+1}r]\}.
\label{eq:B(r,k)-definition}
\end{eqnarray} 

Following the proof of the ISS small-gain theorem in semimaximum formulation in \cite{MKG20}, we obtain for all $t \geq 0$ that%
\begin{align}
\label{eq:nuISS_SGT_tmp_estimate_3}
  &\sup_{s \geq t + \tau^*_i}\sup_{(x,u) \in B(r,k)} \|\phi_i(s,x,u)\|_{X_i} \nonumber\\
	&\  \leq \ep {+} \sup_{j \in I} \gamma_{ij}\Bigl(\sup_{s \geq t} \sup_{(x,u) \in B(r,k)} \hspace{-3mm}\|\phi_j(s,x,u)\|_{X_j}\Bigr) {+} \gamma_i(2^{-k+1}r).%
\end{align}

%Taking the supremum of \eqref{eq:nuISS_SGT_tmp_estimate_1} over all $x\in B_r$ and $u\in\Uc$ with $\|u\|_{\Uc} \in [2^{-k}r,2^{1-k}r]$, we obtain for all $i \in  I$ and $\tau\geq \tau^*_i$ that%
%\begin{align}
%\sup_{\|u\|_{\Uc} \in [2^{-k}r,2^{1-k}r]}&\sup_{x\in B_r}\|\phi_i(t+\tau,x , u)\|_{X_i} \nonumber\\
%&\leq \ep + \sup_{j\in I\backslash\{i\}}\gamma_{ij}\left(\sup_{\|u\|_{\Uc} \in [2^{-k}r,2^{1-k}r]}\sup_{x\in B_r}\left\|\phi_{j,[t,+\infty)}\right\|_{\infty}\right) + \gamma_i(2^{1-k}r) 
%\label{eq:nuISS_SGT_tmp_estimate_2}
%\end{align}
%and thus%
%\begin{align}
%\sup_{s\geq t+\tau^*_i}& \sup_{\|u\|_{\Uc} \in [2^{-k}r,2^{1-k}r]} \sup_{x\in B_r}\|\phi_i(s, x , u)\|_{X_i} \nonumber\\
%&\leq \ep + \sup_{j\in I\backslash\{i\}}\gamma_{ij}\left(\sup_{\|u\|_{\Uc} \in [2^{-k}r,2^{1-k}r]}\sup_{x\in B_r} \sup_{s\geq t} \|\phi_j(s, x , u)\|_{X_j}\right) + \gamma_i(2^{1-k}r)\nonumber \\
%&= \ep + \sup_{j\in I\backslash\{i\}}\gamma_{ij}\left(\sup_{s\geq t} \sup_{\|u\|_{\Uc} \in [2^{-k}r,2^{1-k}r]} \sup_{x\in B_r} \|\phi_j(s, x , u)\|_{X_j}\right) + \gamma_i(2^{1-k}r).
%\label{eq:nuISS_SGT_tmp_estimate_3}
%\end{align}

Define%
\begin{align*}
  y_i(r,k) &:= \lim_{t\to +\infty}\sup_{s\geq t}\sup_{(x,u)\in B(r,k)}\|\phi_i(s,x,u)\|_{X_i}\\
           &= \limsup_{t\to +\infty}\sup_{(x,u)\in B(r,k)}\|\phi_i(t,x,u)\|_{X_i}.%
\end{align*}
By Lemma~\ref{lem:LimSupEstimate} it holds that%
\begin{equation*}
  y_i(r,k) = \lim_{t\to +\infty}\sup_{s\geq t +\tau^*_i}\sup_{(x,u)\in B(r,k)}\|\phi_i(s,x,u)\|_{X_i}.%
\end{equation*}
As each $\Sigma_i$ has a finite number of neighbors (assumption~\ref{itm:nonuniform-ISS-SGT-Ass3}), we can take the limit $t\to\infty$ in \eqref{eq:nuISS_SGT_tmp_estimate_3} and obtain%

\begin{eqnarray}
\label{eq:nuISS_SGT_tmp_estimate_4}
  y_i(r,k) \leq \ep + \sup_{j\in I}\gamma_{ij}\left(y_j(r,k)\right) + \gamma_i(2^{1-k}r).%
\end{eqnarray}
As \eqref{eq:nuISS_SGT_tmp_estimate_4} is valid for arbitrarily small $\ep>0$, we obtain by computing the limit $\ep\to +0$ that%
\begin{eqnarray}\label{eq:nuISS_SGT_tmp_estimate_5}
  y_i(r,k) \leq \sup_{j\in I\backslash\{i\}}\gamma_{ij}\left(y_j(r,k)\right) + \gamma_i(2^{1-k}r),\quad i\in I.%
\end{eqnarray}
Denote $\vec{\gamma}(u) := ( \gamma_i(\|u\|_{\Uc}) )_{i \in I}$ and $y(r,k):=(y_i(r,k))_{i\in I}$ and note that $y(r,k)  \in\ell_\infty^+(I)$, as the entries $y_i$ are uniformly bounded by $\mu(r)$, and by (i), we have $\vec{\gamma}(u) \in \ell_\infty^+(I)$.

Let us rewrite \eqref{eq:nuISS_SGT_tmp_estimate_5} in a vector form:%
\begin{eqnarray}\label{eq:nuISS_SGT_tmp_estimate_6}
  y(r,k) \leq \Gamma^{ISS}_\otimes\left(y(r,k)\right) + \vec{\gamma}(2^{1-k}r),%
\end{eqnarray}
which we reformulate as:%
\begin{eqnarray*}
  (\mathrm{id}-\Gamma^{ISS}_\otimes)(y(r,k)) \leq \vec{\gamma}(2^{1-k}r).
%\label{eq:nuISS_SGT_tmp_estimate_7}
\end{eqnarray*}
From the assumption \ref{itm:nonuniform-ISS-SGT-Ass2} of the theorem, there is $\xi\in\Kinf$ so that%
\begin{eqnarray}
  \|y(r,k)\|_{\ell_\infty} \leq \xi(\| \vec{\gamma}(2^{1-k}r)\|_{\ell_\infty}) \leq \xi(\gamma(2^{1-k}r)).%
\label{eq:nuISS_SGT_tmp_estimate_7}
\end{eqnarray}

As \eqref{eq:nuISS_SGT_tmp_estimate_7} is the same as $y_i(r,k) \leq \xi(\gamma(2^{1-k}r))$ for all $i\in I$, \eqref{eq:nuISS_SGT_tmp_estimate_7} is equivalent to
existence for any $i\in I$, any $\ep>0$, any $r>0$ and any $k\in \Z_+$ of a time $\tilde{\tau}_i=\tilde{\tau}_i(\ep,r,k)$ such that

\begin{equation} 
\label{eq:nuISS_SGT_intermediate_step}
\begin{split}
\|x\|_X\leq &r\ \wedge \ \|u\|_\Uc \in [2^{-k}r,2^{1-k}r] \ \wedge \ t\geq\tilde{\tau}_i(\ep,r,k)\\
& \qrq \|\phi_i(t, x, u)\|_{X_i}  \leq \ep + \xi(\gamma(2^{1-k}r)).
\end{split}
\end{equation}
%
%\begin{eqnarray}
%\|x\|_X\leq r\ \wedge \ \|u\|_\Uc \in [2^{-k}r,2^{1-k}r] \ \wedge \ t\geq\tilde{\tau}(\ep,r,k) \qrq \|\phi(s, x, u)\|_{X}  \leq \ep + \xi(|\gamma(2^{1-k}r)|).
%\label{eq:nuISS_SGT_intermediate_step}
%\end{eqnarray}
%Now define $k_{0} = k_{0}(\ep,r) \in \Z_+$ as the minimal $k$ so that $\xi(2\gamma_{\max}(2^{1-k}r)) \leq \ep$ and let%
%\begin{equation*}
  %\hat{\tau}(\ep,r) := \max\{ \tilde{\tau}(\ep,r,k) : 1 \leq k \leq k_0(\ep,r) \}.%
%\end{equation*}

Doing analogous steps as in the proof of the ISS small-gain theorem in \cite{MKG20}, we obtain the existence of 
$\tau_i(\varepsilon,r)>0$, s.t.
%
%Define $k_0=k_0(\ep,r) \in \Z_+$ as the minimal $k$ so that $\xi(\gamma(2^{1-k}r)) \leq \ep$ (clearly, such $k_0$ always exists and is finite) and let
%\[
%\hat{\tau}_i(\ep,r):=\max\{\tilde{\tau}_i(\ep,r,k):\ k= 1,\ldots,k_0(\ep,r)\}.
%\]
%As $k_0$ is finite, $\hat{\tau}_i(\ep,r)$ is well-defined and finite as well.
%
%Pick any $u\in\Uc$ such that $u\neq 0$ and $\|u\|_{\Uc}\leq r$. 
%Then there is $k\in \Z_+$ so that
%$\|u\|_\Uc \in (2^{-k}r,2^{1-k}r]$.
%If $k\leq k_0$ (i.e. if inputs are large enough), then for $t\geq\hat{\tau}_i(\ep,r) $ it holds that 
%\begin{eqnarray}
%\|\phi_i(t, x, u)\|_{X_i}  \leq \ep + \xi(\gamma(2^{1-k}r))  \leq \ep + \xi(\gamma(2\|u\|_{\Uc})). 
%%\leq \ep + \xi(\| \vec{\gamma}(2 \|u\|_\Uc)\|_{\ell_\infty}).
%\label{eq:Final_nonuniform-bUAG_implication_1}
%\end{eqnarray}
%It remains to consider the case when $k>k_0$, i.e. when inputs are small.
%The estimate \eqref{eq:nuISS_SGT_intermediate_step} gives convergence time, which depends on $k$ and it is not clear whether the supremum of
%$\tilde{\tau}_i(\ep,r,k)$ over all $k\geq k_0$ exists.
%To overcome this obstacle and to find the uniform time, we mimic above argument once again, namely: for any $q \in [0,r]$ one can take  
%supremum of \eqref{eq:nuISS_SGT_tmp_estimate_1} over $x\in B_r$ and over all $u\in\Uc$: $\|u\|_{\Uc} \leq q$, to obtain for all $i=1,\ldots,n$ and 
%all $\tau\geq \tau^*_i(\varepsilon,r) $ that 
%\begin{equation} 
%\label{eq:nuISS_SGT_tmp_estimate_Small_U}
%%\begin{split}
%\sup_{\|u\|_{\Uc}\leq q}\sup_{x\in B_r}\|\phi_i(t+\tau,x , u)\|_{X_i} %\\
%\leq \ep + \sup_{j\in I\backslash\{i\}}\gamma_{ij}\left(\sup_{\|u\|_{\Uc} \leq q}\sup_{x\in B_r}\left\|\phi_{j,[t,+\infty)}\right\|_{\infty}\right) + \gamma_i(q).
%%\end{split}
%\end{equation}
%Defining 
%\[
%z_i(r,q):=  \lim_{t\to +\infty}\sup_{s\geq t}\sup_{\|u\|_{\Uc} \leq q}\sup_{x\in B_r}\|\phi_i(s, x , u)\|_{X_i},
%\]
%and doing analogous steps as above, we obtain for any $i\in I$, $r>0$ and any $q\leq r$ that
%\begin{eqnarray*}
%z_i(r,q) \leq \xi(\|\vec{\gamma}(q)\|_{\ell_\infty})
  %\leq \xi(\gamma(q)).
%\end{eqnarray*}
%This means that for any $i\in I$, $\ep>0$, any $r>0$ and any $q\geq 0$ there is a time $\bar{\tau}_i=\bar{\tau}_i(\ep,r,q)$ so that
%\begin{eqnarray}
%\phantom{aaaaa}\|x\|_X\leq r\ \wedge \ \|u\|_\Uc \leq q \ \wedge \ t\geq\bar{\tau}_i(\ep,r,q) \srs \|\phi_i(t, x, u)\|_{X_i}  \leq \ep + \xi(\gamma(q)).
%\label{eq:nuISS_SGT_interediate_step_Small_U}
%\end{eqnarray}
%In particular, for $q_0:=2^{1-k_0(\ep,r)}r$ we have  
%\begin{equation} 
%\label{eq:Final_nonuniformbUAG_implication_2}
%\begin{split}
%\|x\|_X\leq r\ \wedge \ \|u\|_\Uc \leq &q_0 \ \wedge \ t\geq\bar{\tau}_i(\ep,r,q_0) \\
%& \qrq \|\phi_i(t, x, u)\|_{X_i}  \leq \ep + \xi(\gamma(2^{1-k_0(\ep,r)}r))\leq \ep + \ep.
%\end{split}
%\end{equation}
%%
%%\begin{eqnarray}
%%\|x\|_X\leq r\ \wedge \ \|u\|_\Uc \leq q_0 \ \wedge \ t\geq\bar{\tau}(\ep,r,q_0) \qrq \|\phi(s, x, u)\|_{X}  \leq \ep + \xi(|\gamma(2^{1-k_0(\ep,r)}r)|)\leq \ep + \ep.
%%\label{eq:Final_nonuniformbUAG_implication_2}
%%\end{eqnarray}
%Define 
%\[
%\tau_i(\ep,r) :=\max\{\hat{\tau}_i(\ep,r), \bar{\tau}_i(\ep,r,q_0) \}.
%\]
%Combining \eqref{eq:Final_nonuniform-bUAG_implication_1} and \eqref{eq:Final_nonuniformbUAG_implication_2}, we obtain 
%\[
%\|x\|_X\leq r\ \wedge \|u\|_\Uc \leq r \ \wedge t\geq\tau_i(\ep,r) \qrq \|\phi_i(t, x, u)\|_{X_i}  
%\leq \ep + \max\{\ep, \xi(\gamma(2 \|u\|_\Uc)) \}.
%\]
%and finally
\begin{align}
\|x\|_X&\leq r\ \wedge \ \|u\|_\Uc \leq r \ \wedge \ t\geq\tau_i(\ep,r) \nonumber\\
&\qrq \|\phi_i(t, x, u)\|_{X_i} \leq 2\ep + \xi(\gamma(2 \|u\|_\Uc)).
\label{eq:non-uniform-bUAG-est-NU-SGT}
\end{align}
where $r\mapsto \xi(\gamma(2 r))$ is a $\Kinf$-function. This shows the estimate \eqref{eq:non-uniform-ISS}.
Finally, Proposition~\ref{prop:Criterion-non-uniform-ISS} proves the claim.
\end{proof}

\begin{remark}
\label{rem:Finitely-many-systems} 
If $|I|<\infty$, then the assumptions \ref{itm:nonuniform-ISS-SGT-Ass1} and \ref{itm:nonuniform-ISS-SGT-Ass3} of 
Theorem~\ref{thm:nonuniform-ISS_SGT-infinite-interconnections} hold. Furthermore, by Proposition~\ref{prop:ISS-of-finite-networks},
non-uniform ISS is equivalent to ISS. And in view of results in \cite{MKG20}, the MBI property for $\id - \Gamma_\otimes$ is equivalent to the assertion that $\Gamma_\otimes$ satisfies the strong small-gain condition.
Hence, we recover from Theorem~\ref{thm:nonuniform-ISS_SGT-infinite-interconnections} the ISS small-gain theorem (in semimaximum formulation) for finite networks of infinite-dimensional systems, shown in \cite{Mir19b}.
\end{remark}


%\begin{remark}
%\label{rem:Changing-limsup-and-sup} 
%The assumption of a finite number of neighbors is used only to obtain \eqref{eq:nuISS_SGT_tmp_estimate_4}, as we need to change the operations of $\limsup_{t\to\infty}$ and $\sup_{j\in I\backslash\{i\}}$. If $I$ is of infinite cardinality, it is not possible in general.
%\end{remark}
%


\subsection{Nonuniform small-gain theorem vs ISS small-gain theorem}

\emph{Here we would like to explain why we were able to achieve the non-uniform ISS small-gain theorem under the validity of the MBI property of $\id-\Gamma_\otimes$, while the proof of the ISS small-gain theorem in \cite{MKG20} requires a more restrictive monotone limit property of $\Gamma_\otimes$.}
To this end let us see why the proof strategy of the non-uniform ISS small-gain theorem does not work for the (uniform) ISS small-gain theorem.

Suppose that all assumptions of the non-uniform ISS small-gain theorem hold and additionally there is $\beta\in\KL$ such that
\begin{eqnarray}
\beta_i(r,t)\leq \beta(r,t),\quad r\in\R_+,\ t\geq 0,\ i\in I.
\label{eq:Bounds_on_beta_uniform-ISS-vs-NU-ISS}
\end{eqnarray}
Without assuming the existence of the uniform $\KL$-upper bound for functions $\beta_i$, it is not possible to show the ISS property for the network $\Sigma$, even if all $\gamma_{ij}\equiv 0$ and $\gamma_i\equiv 0$.


The proofs of the ISS small-gain theorem and non-uniform ISS small-gain theorem are analogous till the estimate \eqref{eq:nuISS_SGT_tmp_estimate_3}.
The next step in the proof of the non-uniform small-gain theorem is a computation of the limit superior at both sides of \eqref{eq:nuISS_SGT_tmp_estimate_3}.
This is done in order to obtain the terms $y_j(r,k)$ at both sides of the inequality, as in \eqref{eq:nuISS_SGT_tmp_estimate_4}, which is a prerequisite for the application of the monotone bounded invertibility property of $\Gamma_{\otimes}$.
%(compare to the estimate   \eqref{eq:inequality-for-w_i} in the proof of the ISS small-gain theorem). 
In the non-uniform small-gain theorem $\tau_i(\varepsilon,r)$ are different for different $i$, and the computation of limit superior is a reasonable trick. 
However, in order to achieve ISS of the network we assume also \eqref{eq:Bounds_on_beta_uniform-ISS-vs-NU-ISS}, which ensures the existence of $\tau(\varepsilon,r)$ such that 
$\tau_i(\varepsilon,r)\leq\tau(\varepsilon,r)$ for all $i\in I$.
However, by computation of the limit superior we lose this uniformity. Following the proof of the non-uniform small-gain theorem, we obtain \eqref{eq:non-uniform-bUAG-est-NU-SGT}, but it is unclear how we can show the uniformity of $\tau_i(\ep,r)$ with respect to $i$.

%\begin{remark}
%\label{rem:ISS-not-possible-without-uniform-beta_i} 
%As there is no uniform $\KL$-upper bound for functions $\beta_i$, it is not possible to show the ISS property for the network $\Sigma$, even if all $\gamma_{ij}\equiv 0$ and $\gamma_i\equiv 0$.
%\end{remark}




\section{Applications to stability analysis of finite networks of an unknown size}
\label{sec:Analysis of finite networks of an unknown size}

We are going to show that \emph{our non-uniform ISS small-gain theorem can be used to analyze (uniform) ISS of finite networks of an unknown size}, which are ubiquitous in many real-world applications.


In this case, we treat $(\Sigma_i)_{i\in I}$ as the family of all \emph{possible} subsystems of which the network may be constructed of, where $\Sigma_i$ is defined as in \eqref{eq:Sigma-i}.
%We assume that $\Sigma:=(X,\Uc,\phi)$ is a well-defined interconnection of the systems $(\Sigma_i)_{i\in I}$, and $\Sigma$ can be called the \emph{maximal network}.

Denote by $Q \subset I$ an index set, that is fixed but usually unknown and that enumerates all components (subsystems) which are a part of the network. We assume that the systems $(\Sigma_i)_{i\in I\backslash Q}$ are not interacting with the systems $(\Sigma_i)_{i\in Q}$, which we express by saying that the inputs from $(\Sigma_i)_{i\in I\backslash Q}$ to $\Sigma_j$ for any $j\in Q$ are zero.

Our definition of interconnection (Definition~\ref{def_interconnection}) has been introduced for the interconnection of all systems in the network, and in order to use this construction for the coupling of a subset of systems, we restrict the systems $(\Sigma_i)_{i\in Q}$ to smaller input spaces.

Recall the definition of the spaces $X_Q$, $X_{Q\backslash\{i\}}$ and $X_{\neq i,Q}$, introduced in Section~\ref{sec:Infinite interconnections}. Furthermore, we identify $\PC_b(\T_+,X_{Q\backslash\{i\}})$ with the space $\PC_b(\T_+,X_{\neq i,Q})$.

Now define the system $\Sigma_i$ with inputs restricted to systems $(\Sigma_i)_{i\in Q}$:
\begin{equation}
\label{eq:Sigma-i-restricted}
  \tilde{\Sigma}_i = (\T, X_i,\PC_b(\T_+,X_{Q\backslash\{i\}}) \tm \Uc,\tilde{\phi}_i),%
\end{equation}
where $\tilde{\phi}_i$ is a restriction of $\bar{\phi}_i$ to 
\[
\{(t,x_i,v)\in D_{\bar{\phi}_i}:\quad v \in \PC_b(\T_+,X_{Q\backslash\{i\}}) \tm \Uc\}.
\]

Assume that the interconnection $\Sigma_Q$ of the systems $(\tilde{\Sigma}_i)_{i\in Q}$ is well-posed in the sense of Definition~\ref{def_interconnection} and that each system $\Sigma_i$, $i\in I$ is ISS in semimaximum formulation, as in Definition~\ref{def_subsys_iss_semimax}, with corresponding gains $\gamma_{ij}$ and the induced operator $\Gamma_{\otimes}$, defined by \eqref{eq:Gain-operator-semimax}.
The gain operator $\Gamma_{\otimes}$ characterizes the interconnection structure of the \emph{maximal network} $(\Sigma_i)_{i\in I}$.

Let us consider now the \emph{real network} $\Sigma_Q$ of subsystems $(\tilde{\Sigma}_i)_{i\in Q}$.
As the inputs from $(\Sigma_i)_{i\in I\backslash Q}$ to $\Sigma_j$ for any $j\in Q$ are zero, we can assume that 
$\gamma_{ij}\equiv 0$ if either $i\in I\backslash Q$, or $j\in I\backslash Q$.

Hence the subsystems  $(\tilde{\Sigma}_i)_{i\in Q}$ are also ISS in semimaximum formulation, with the gain operator
$\Gamma_{\otimes,Q}:\ell_{\infty}(Q)^+ \rightarrow \ell_{\infty}(Q)^+$ induced by the gains $(\gamma_{ij})_{i,j\in Q}$ as follows:%
\begin{equation}
\label{eq:Gain-operator-semimax-restricted}
  \Gamma_{\otimes,Q}(s) := \bigl(\sup_{j\in Q}\gamma_{ij}(s_j)\bigr)_{i\in Q},\quad s = (s_i)_{i\in Q} \in \ell_{\infty}(Q)^+.%
\end{equation}

We need the following lemma 
\begin{lemma}
\label{lem:Monotone-invertibility-of-smaller-operators} 
Assume that $\id - \Gamma_\otimes$ satisfies the monotone bounded invertibility property. Let $B:\ell_\infty^+(I) \rightarrow \ell_\infty^+(I)$ be another operator such that $B \leq \Gamma_\otimes$.
Then $\id - B$ satisfies the monotone bounded invertibility property.
\end{lemma}

\begin{proof}
As $B\leq \Gamma_\otimes$, for any $v\in K$ it holds that  $(\id - B)(v) = (\id - \Gamma_\otimes)(v) + (\Gamma_\otimes-B)(v)\geq (\id - \Gamma_\otimes)(v)$.
Thus, if $(\id - B)(v) \leq w$, then $(\id - \Gamma_\otimes)(v) \leq w$, and by monotone bounded invertibility property of $\Gamma_\otimes$
it holds that $\|v\|_X \leq \xi(\|w\|_X)$, for some $\xi\in\Kinf$, independent on $w,v\in K$.
\end{proof}




We have the following result:
\begin{lemma}
\label{lem:Monotone-invertibility-restricted-gain-operators} 
If $\Gamma_{\otimes}$ satisfies the monotone bounded invertibility property, then $\Gamma_{\otimes,Q}$ satisfies the monotone bounded invertibility property for any $Q \subset I$.
\end{lemma}

\begin{proof}
Let $\tilde{w},\tilde{v}\in \ell_{\infty}(Q)^+$ be such that 
\begin{eqnarray}
(\id - \Gamma_{\otimes,Q})\tilde{w} \leq \tilde{v}.
\label{eq:Premise-MBI-Gamma-Q}
\end{eqnarray}
Define $w = (w_i)_{i\in I}$ and $v = (v_i)_{i\in I} \in \ell_{\infty}(I)^+ $ by 
$w_i = \tilde{w}_i$, $v_i = \tilde{v}_i$, $i\in Q$ and $v_i = w_i = 0$, $i\in I\backslash Q$.

Also define $\Gamma_{\otimes,I}:  \ell_{\infty}(I)^+  \to  \ell_{\infty}(I)^+$ by 
\begin{equation}
\label{eq:Gain-operator-semimax-extending-restriction}
  \Gamma_{\otimes,I}(s) := \bigl(\sup_{j\in I}\tilde{\gamma}_{ij}(s_j)\bigr)_{i\in I},\quad s = (s_i)_{i\in I} \in \ell_{\infty}(I)^+,%
\end{equation}
where $\tilde{\gamma}_{ij}\equiv 0$ if either $i\in I\backslash Q$ or $j\in I\backslash Q$, and $\tilde{\gamma}_{ij} = \gamma_{ij}$ otherwise.

As $\Gamma_{\otimes,I}\leq \Gamma_{\otimes}$, $\Gamma_{\otimes,I}$ has MBI property by Lemma~\ref{lem:Monotone-invertibility-of-smaller-operators}. 

Furthermore, \eqref{eq:Premise-MBI-Gamma-Q} holds if and only if $(\id - \Gamma_{\otimes,I})w \leq v$, and by MBI property of $\Gamma_{\otimes,I}$, it holds that 
$\|w\|_{\ell_\infty(I)} \leq \xi(\|v\|_{\ell_\infty(I)})$. As $\|w\|_{\ell_\infty(I)} = \|\tilde{w}\|_{\ell_\infty(Q)}$ and $\|v\|_{\ell_\infty(I)} = \|\tilde{v}\|_{\ell_\infty(Q)}$,
we obtain that $\|\tilde{w}\|_{\ell_\infty(Q)} \leq \xi(\|\tilde{v}\|_{\ell_\infty(Q)})$ holds, with $\xi$ independent of $\tilde{w}, \tilde{v}$. This shows the claim.
\end{proof}

The following result tells that the non-uniform ISS small-gain theorem can be effectively used for ISS analysis of finite networks of an unknown size, even if there is no a-priori uniform $\KL$-bound on the transient behavior of subsystems from which the network consists of.

%As a corollary of the previous result we obtain the small-gain criterion for subnetworks of the maximal network $\Sigma$.

\begin{theorem} \textbf{(Nonuniform ISS Small-gain theorem for subnetworks)}
\label{thm:Subnetworks-stability-NU-SGT-based} 
Let $\Sigma_i:=(\T,X_i,\PC_b(\T_+,X_{\neq i}) \tm \Uc,\bar{\phi}_i)$, $i\in I$ be forward complete control systems, satisfying the ISS estimates as in Definition~\ref{def_subsys_iss_semimax}. 
Furthermore, let the conditions \ref{itm:nonuniform-ISS-SGT-Ass1}--\ref{itm:nonuniform-ISS-SGT-Ass3} of Theorem~\ref{thm:nonuniform-ISS_SGT-infinite-interconnections} hold.

Then for any subset $Q \subset I$ such that  $\Sigma_Q=(\T,X_Q,\Uc,\phi)$ (as defined in this section) is well-posed, $\Sigma_Q$ is nonuniformly ISS.
If $Q$ is a finite set, then $\Sigma_Q$ is ISS.
\end{theorem}

\begin{proof}
Non-uniform ISS of $\Sigma_Q$ follows from Theorem~\ref{thm:nonuniform-ISS_SGT-infinite-interconnections} and Proposition~\ref{lem:Monotone-invertibility-restricted-gain-operators}.
If $Q$ is a finite set, then ISS of $\Sigma_Q$ follows by Proposition~\ref{prop:ISS-of-finite-networks}.
\end{proof}

Assumption \ref{itm:nonuniform-ISS-SGT-Ass3} of Theorem~\ref{thm:nonuniform-ISS_SGT-infinite-interconnections} needs to be satisfies in 
Theorem~\ref{thm:Subnetworks-stability-NU-SGT-based} only in case of $Q$ is of infinite cardinality. For finite set $Q$ this assumption is not needed, as for the subnetwork $Q$ it will be always fulfilled.

%\mir{One can follow this path more, and get uniform in $Q$ bounds on the external gain, transient behavior. Also one can show this result based on the assumptions of the ISS small-gain theorem.}

\begin{remark}
\label{rem:Uniform-SGT-for-subnetworks} 
Theorem~\ref{thm:Subnetworks-stability-NU-SGT-based} gives a condition for ISS of any finite subnetwork, but the functions $\beta$ and $\gamma$ in the ISS definition can depend on $Q$, i.e. we do not have uniform $\beta$ and $\gamma$ for all finite $Q \subset I$.
In order to state the ISS Small-gain theorem for subnetworks, which guarantees such a uniformity, we need to require stronger conditions on the gain operator, such as a monotone limit property of $\Gamma_\otimes$, see \cite[Theorem 2]{MKG20}. 
\end{remark}



\bibliographystyle{abbrv}
\bibliography{C:/Users/Andrii/Dropbox/TEX_Data/Mir_LitList_NoMir,C:/Users/Andrii/Dropbox/TEX_Data/MyPublications}

%\bibliography{Mir_LitList_NoMir,MyPublications}
% Important: no space in the list.

%
%\bibliographystyle{abbrv}
%\bibliography{Mir_LitList_NoMir,MyPublications}

\end{document}
%