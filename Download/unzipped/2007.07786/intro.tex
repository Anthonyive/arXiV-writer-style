%\paragraph*{Electrical networks will have high number of distributed generation and storage units}



%\paragraph*{We discuss an energy management problem of such networks. In particular, solving an MPC-based economic dispatch problem with a non-centralized scheme}
\vspace{-10pt}
Considering the current trend and development \cite{molzahn2017,morstyn2016}, future electrical energy networks would have a high number of distributed generation and storage units. In this regard, in the energy management level, a non-centralized control scheme has been considered as a more suitable scheme than the conventional centralized one, due to the ability to deal with high computational requirement, flexibility, reliability and scalability of the non-centralized scheme \cite{molzahn2017,morstyn2016}. On the other hand, non-dispatchable generation units introduce additional uncertainty on top of the already uncertain loads. At the same time, storage units have slow dynamics that must be taken into account when solving the economic dispatch problem. In this regard, model predictive control (MPC) framework, accounting with the receding horizon principle, has been proposed to be implemented as a control scheme to the energy management of such energy systems \cite{baker2016,wang2015}.   As discussed in \cite{parisio2017,zhu2014}, the MPC framework allows us to handle components with dynamics, constraints (of physical and operational nature), and uncertainties better than traditional economic dispatch schemes.  

\vspace{-5pt}
In this paper, we discuss an energy management problem of such networks with distributed generation and storage units. In particular, we solve an MPC-based economic dispatch problem with a non-centralized scheme. We consider the microgrid framework \cite{schwaegerl2013} in which the energy network is partitioned into a group of interconnected microgrids \cite{arefifar2012,barani2018}. Each microgrid is a cluster of storage units, distributed generation units, and loads \cite{arefifar2012}. Furthermore, depending on the physical connection, it can also exchange power with the other microgrids and the main grid. More importantly, each microgrid is an independent entity that can manage itself, i.e., it has its own local controller. Therefore, in a non-centralized scheme, these microgrids cooperatively solve the economic dispatch problem of the network. 


A typical non-centralized approach to solving such problems is by using a distributed optimization algorithm \cite{baker2016,wang2015,kraning2014,braun2016,guo2016} (see \cite{molzahn2017,morstyn2016} for a survey).  Such algorithms are usually iterative and require high information flow, i.e., at each iteration, each local controller must exchange information with its neighbors, with the advantage of obtaining an optimal solution. In this paper, we propose an alternative non-centralized scheme with low information flow. There are two main ingredients of the approach that we propose. The first ingredient is a proper partitioning of the network and the second ingredient is the formulation of coalition-based sub-problems, which requires a coalition formation algorithm. 

In the first part of the method, we (re)-partition the network into a fixed number of microgrids. The objective of the repartitioning scheme is to obtain self-sufficient and efficient microgrids. Roughly speaking, we consider that a microgrid is self-sufficient when it can provide its loads using its local distributed generation units.  When this goal is achieved, each microgrid does not need to rely on the other microgrids, implying a local economic dispatch problem can be solved by the controller of each microgrid. In addition, the efficiency criterion is in line with the objective of the economic dispatch problem. 
Therefore, we propose a repartitioning procedure that has low computational burden and is performed in a distributed manner. %, i.e., each local controller must exchange information with its neighbors. 
The repartitioning procedure that we propose is closely related to the partitioning methods presented in \cite{ananduta2019a,ananduta2019,julian2019}. The main idea of the repartitioning procedure is to move some nodes from one partition to another in order to improve the cost that has been defined. In addition, in our scheme, we consider an event-triggered mechanism, i.e., the network is only repartitioned when the event at which at least one microgrid that is not self-sufficient occurs. To the best of our knowledge, an event-triggered  repartitioning scheme has not been proposed in the literature so far.  Note that most system partitioning methods that have been proposed, e.g., \cite{fjallstrom1998,guo2016,julian2017,ocampo2011}, are intended to be implemented offline prior to the operation of the system and in a centralized fashion, whereas this paper shows how online repartitioning can be exploited to design a non-centralized control scheme and might be performed in a distributed fashion.  \color{black}

In the second part of the method, we decompose the economic dispatch problem into coalition-based sub-problems. Since the repartitioning procedure does not guarantee that the resulting microgrids are self-sufficient, each microgrid that is not self-sufficient is grouped together with some of its neighbors to form a self-sufficient coalition. %At the worst case, all microgrids are grouped as one big coalition and solve the dispatch problem in a distributed manner. 
In this regard, we propose a coalition formation procedure, which is also carried out in a distributed manner. Furthermore, coalition-based economic dispatch sub-problems are formulated. These problems are solved by the local controllers of the microgrids in order to obtain a feasible but possibly sub-optimal solution to the centralized economic dispatch problem. 
%\paragraph*{relevances with coalitional control, and differences with existing coalitional control papers}
The coalition-based economic dispatch approach is inspired by the coalitional control framework \cite{fele2017,muros2017,fele2018}. In the coalitional control scheme, the sub-systems are clustered into several coalitions based on the relevance of the agents, e.g., the degree of coupling among the sub-systems. Furthermore, the sub-systems that are in the same coalition cooperatively compute their control inputs. 
 In our method, the coalitions formed are based on the necessity to maintain the feasibility of the economic dispatch sub-problems, which can be perceived as the relevance of the agents to the problem itself. \color{black} %Specifically related to the control problems of power networks, \cite{fele2018} provides a case study of voltage control. However, to the best of our knowledge, a coalitional scheme in the dispatch level has not been proposed. \color{black}

%\paragraph*{statement of contribution}

To summarize, the main contribution of this paper is a novel non-centralized economic dispatch method for electrical energy networks with distributed generation and storage units. The methodology has less intensive communication flows compared to typical distributed optimization methods, thus suitable with online optimization of the MPC framework. To that end, the methodology combines an event-triggered repartitioning approach with the aim of producing self-sufficient and efficient microgrids and a procedure to form self-sufficient coalitions of microgrids to solve the economic dispatch problem. This paper also provides the analysis of the proposed methodology, including the outcomes of the repartitioning and coalition formation algorithms, as well as an upper bound for the suboptimality of the proposed scheme. %Furthermore, some numerical simulations are also carried out to show the effectiveness of the proposed scheme in a benchmark case study. 
The methodology that we present in this paper is an extension of that in \cite{ananduta2019a}, where a periodical repartitioning scheme for a fully decentralized scheme is proposed. Additionally, a feasibility issue, which arises from microgrids that are not self-sufficient and can be found when using the scheme in \cite{ananduta2019a}, is solved by the coalition-based approach proposed in this paper. 


The remainder of the paper is structured as follows. In Section \ref{sec:prob_form}, we provide the mathematical formulation of the economic dispatch problem and outline the proposed method. Section \ref{sec:rep_sch} presents the proposed event-triggered repartitioning scheme. In Section \ref{sec:nc_ed}, the coalition-based economic dispatch method is explained. Afterward, Section \ref{sec:opt} provides a discussion about the trade-off between the sub-optimality and communication complexity of the method. In Section \ref{sec:case_st}, a numerical study of a well-known benchmark case is presented to show the effectiveness of the proposed scheme. Finally, Section \ref{sec:concl} provides some concluding remarks and discusses future work. %\color{blue}Note that due to space constraint, the proofs of all propositions are available in \cite{ananduta2020}.\color{black}

\vspace{-5pt}
\subsection*{Notations} 
\vspace{-10pt}
The sets of real numbers and integers are denoted by $\mathbb{R}$ and $\mathbb{Z}$, respectively. Moreover, for $a\in\mathbb{R}$, $\mathbb{R}_{\geq a}$ denotes all real numbers in the set \{$b: b\geq a, \ b \in \mathbb{R}$\}. A similar definition can be used for $\mathbb{Z}_{\geq a}$ and the strict inequality case. %For column vectors $v_i$ with $i \in \mathcal{L}=\{l_1,\dots,l_{m}\}$ and $m\geq1$, the operator  $[v_i^{\top}]^{\top}_{i \in \mathcal{L}}$ denotes the column-wise concatenation, i.e., $[v_i^{\top}]^{\top}_{i \in \mathcal{L}} = [v_{l_1}^{\top},\cdots,v_{l_{m}}^{\top}]^{\top}$.  
The set cardinality is denoted by $|\cdot|$. Finally, discrete-time instants are denoted by the subscript $k$.
