In this section, we discuss the sub-optimality and communication trade-off of the proposed scheme. 
%\subsection{Sub-optimality}
First, we show an estimation of the sub-optimality level achieved by the scheme. To that end, we state the collection of the optimization problems \eqref{eq:lo_mpc}, for all coalitions $\mathcal{C}_{p,k}, p=1,\dots,m$, as follows:
\begin{equation}
	\begin{aligned}
	&\min_{\{\{(\bm{u}_{i,\ell},\bm{v}_{i,\ell})\}_{i \in \mathcal{N}}\}_{\ell=k}^{k+h-1}} \sum_{i \in \mathcal{N}} \sum_{\ell =k}^{k+h-1}  J_{i,\ell}(\bm{u}_{i,\ell},\bm{v}_{i,\ell})\\
	&\text{s.t. \eqref{eq:co_loc_cons}, \eqref{eq:co_loc_cons2}, and \eqref{eq:coup_dec_const}, } \forall p \in \{1,\dots,m\},
	\end{aligned}
		\label{eq:lo_mpc_all}%
\end{equation}
for all $\ell \in\{k,\dots,k+h-1 \}$. Denote the optimal value of Problem \eqref{eq:lo_mpc_all} by $J^{\star}_k$. Note that $J^{\star}_k$ represents the cost function value of Problem \eqref{eq:MPC_net} computed by the proposed scheme. Furthermore, denote by $J_k^o$ the optimal value of Problem \eqref{eq:MPC_net} and define the sub-optimality measure as the difference between the cost function value computed using the proposed scheme and the optimal value of Problem \eqref{eq:MPC_net}, denoted by $\Delta J_k$, i.e., 
\begin{equation}
	\Delta J_k = J^{\star}_k-J^o_k. \label{eq:diff_J}
\end{equation}
%Therefore, an estimate of $\Delta J_k$ is shown in Proposition \ref{prop:est_difJ}.
\begin{prop}
	\label{prop:est_difJ}
	Let $J^{\star}_k$ and $J^o_k$ be the optimal values of Problems \eqref{eq:lo_mpc_all} and \eqref{eq:MPC_net} at time $k$, respectively. Furthermore, let $J^{b}_k$ denote the optimal value of the following optimization problem:
	\begin{equation}
	\begin{aligned}
	&\min_{\{\{(\bm{u}_{i,\ell},\bm{v}_{i,\ell})\}_{i \in \mathcal{N}}\}_{\ell=k}^{k+h-1}} \sum_{i \in \mathcal{N}} \sum_{\ell =k}^{k+h-1}  J_{i,\ell}(\bm{u}_{i,\ell},\bm{v}_{i,\ell})\\
	&\mathrm{s.t. } \text{ \eqref{eq:co_loc_cons} $\mathrm{and}$ \eqref{eq:co_loc_cons2} } \forall p \in \{1,\dots,m\}.
	\end{aligned}
	\label{eq:lb_mpc_all}%
	\end{equation}
	Then, the following estimate on the suboptimality measure $\Delta J_k$, defined in \eqref{eq:diff_J}, holds:
	\begin{equation}
		\Delta J_k \leq J^{\star}_k-J^b_k. \label{eq:est_diffJ}
	\end{equation}
\end{prop}
%\iffalse 
\begin{pf*}{Proof.}
	Note that Problem \eqref{eq:lb_mpc_all} can be obtained by relaxing Problem \eqref{eq:MPC_net}. In particular, the coupling constraint $v_{i,\ell}^j + v_{j,\ell}^i=0$, for each pair of nodes $i$ and $j$ that do not belong to the same coalition, is discarded in Problem \eqref{eq:lb_mpc_all}. Due to this relaxation, we can conclude that $J^b_k \leq J^o_k$. Moreover, based on Proposition \ref{prop:feas_of_orig}, the solution obtained by Problem \eqref{eq:lo_mpc}, for all coalitions $\mathcal{C}_{p,k}, \ p=1,\dots,m$, is also a feasible solution to Problem \eqref{eq:MPC_net}, implying that $J^o_k \leq J^{\star}_k$. Based on the preceding observations, the relation in \eqref{eq:est_diffJ} holds. \eod
\end{pf*}
%\fi
%\iffalse 
\begin{rem}
	\label{re:tight_bound}
	Consider the case when $\mathcal{C}_{p,k}=\mathcal{N}$, for $p=1,\dots,m$. In this case, for any $i\in \mathcal{N}$, all neighbors of node $i$, i.e., $j\in\mathcal{N}_i$, belong to the same coalition as that of node $i$. Thus, in \eqref{eq:coup_dec_const},  $\mathcal{N}_i\backslash\mathcal{C}_{p,k}=\emptyset$. This fact implies that Problem \eqref{eq:lo_mpc_all} is equivalent to Problem \eqref{eq:MPC_net} and Problem \eqref{eq:lb_mpc_all}, implying   $\Delta J_k=0$ and $J^{\star}_k-J^b_k=0$. Additionally, Problem \eqref{eq:lb_mpc_all} can be decomposed into {$m$ sub-problems, each of which can be assigned to each coalition}.\eod
\end{rem}
%\fi
%\begin{rem}
	%The optimal value of Problem \eqref{eq:lb_mpc_all} is a lower bound of Problem \eqref{eq:MPC_net}. Moreover, %Problem \eqref{eq:lo_mpc_all} is obtained from the coalition-based economic dispatch problem in \eqref{eq:lo_mpc} and Problem \eqref{eq:lb_mpc_all} is obtained from Problem \eqref{eq:lo_mpc_all}. Therefore, 
%	Problem \eqref{eq:lb_mpc_all} can also be decomposed into {$m$ sub-problems, each of which can be assigned to each coalition}.\eod %In particular, the sub-problem is as follows:
	%\begin{equation}
	%\begin{aligned}
	%&\min_{\{\{(\bm{u}_{i,\ell},\bm{v}_{i,\ell})\}_{i \in \mathcal{C}_{p,k}}\}_{\ell=k}^{k+h-1}} \sum_{i \in \mathcal{C}_{p,k}} \sum_{\ell =k}^{k+h-1}  J_{i,\ell}(\bm{u}_{i,\ell},\bm{v}_{i,\ell})\\
	%&\qquad \qquad \mathrm{s.t. } \text{ \eqref{eq:co_loc_cons} $\mathrm{and}$ \eqref{eq:co_loc_cons2}, }\forall \ell \in\{k,\dots,k+h-1 \}.
	%\end{aligned}
	%\label{eq:lb_mpc}%
%	\end{equation}
	%As mentioned in Remark \ref{re:tight_bound}, $\Delta J_k=J^{\star}_k-J^b_k=0$ when $\mathcal{C}_{p,k}=\mathcal{N}$, for $p=1,\dots,m$. Then, in this particular case, $J^b_k$ is not necessary to be computed. 
%\end{rem}

%\subsection{Communication cost}
%\color{red}
Now, we discuss the communication cost of the proposed scheme. Algorithms \ref{alg:repartitioning} and \ref{alg:clust} do require information exchange among the controllers. The total size of data exchanged throughout the process in Algorithms \ref{alg:repartitioning} and \ref{alg:clust} is $\mathcal{O}(m)$ per iteration. %It is obtained since, at each iteration, the microgrid selected to propose a node to be moved must send the information about the node, which is a scalar, to its neighbors and receives back the cost adjustment, which is also a scalar.  
%Furthermore, the total size of information exchanged in Algorithm \ref{alg:clust} is also $\mathcal{O}(m)$. %, since the information exchange process is similar to that of Algorithm \ref{alg:repartitioning}. 
Finally, we evaluate the size of data communicated when solving the coalition-based economic dispatch problem. Each coalition might need to use a distributed optimization method since there might be more than one microgrid in a coalition. As an example, we consider the dual-ascent algorithm \cite{ananduta2018} as the distributed optimization method. In this algorithm, the size of exchanged information is  $\mathcal{O}(m|\mathcal{N}|h)$ per iteration since each microgrid must exchange the coupled decision variables with each neighbor. 
In the best-case scenario, when all microgrids are self-sufficient, communication might not be necessary at all at one time instant. Furthermore, even if the repartitioning procedure is triggered, in the worst-case scenario, i.e., when the resulting coalition includes all microgrids, the extra amount of data must be exchanged to perform the repartitioning and coalition formation procedures is relatively much smaller than that of performing the distributed algorithm. In addition, for a coalition that has only one microgrid, its controller only needs to solve a local optimization problem once, which significantly reduces the computational burden as well.  
\color{black}

In comparison with existing methods that are based on distributed optimization algorithms, e.g., \cite{baker2016,wang2015,kraning2014,braun2016,guo2016}, the proposed scheme reduces the number of neighbors with which each agent must communicate since each agent only needs to communicate with a subset of neighbors that belong to the same coalition. This fact implies the reduction of communication flows. Moreover, the coalition-based problem \eqref{eq:lo_mpc} is relatively smaller than the network problem \eqref{eq:MPC_net}, thus intuitively a solution to \eqref{eq:lo_mpc} can be computed faster than a solution to \eqref{eq:MPC_net} using the same distributed iterative algorithm. 
%However, it is also important to note that when there exist at least two distinct coalitions, the solution obtained from the proposed scheme might be suboptimal. %As stated in Remark \ref{re:tight_bound}, we can only guarantee that the optimal solution is obtained when all controllers belong to the same coalition.% Additionally, distributed optimization methods, e.g. those that are proposed in \cite{baker2016,wang2015,kraning2014,braun2016,guo2016}, might also be implemented to solve the coalition-based problem \eqref{eq:lo_mpc}, for coalitions that consists of more than one agents/microgrids. 

%\color{red}
Finally, we discuss the practicality of performing the proposed scheme. As in any distributed scheme, local controllers must cooperate to perform the proposed method and a communication network must also be available. Since the partition of the electrical network is dynamic, a dynamic network, containing necessary communication links, might be required. Another possibility is by having an all-to-all network although in the process not all links will be used. Furthermore, each local controller must also be able to communicate with the active components of the network, i.e., the storage and dispatchable generation units. The second important note that we would like to mention that although in this paper we consider an MPC-based framework, where the set points are computed at each time instant, the proposed method can also be implemented for day-ahead economic dispatch without requiring any modification. In this case, the prediction horizon is set to be one day and prior to the computation of the decisions, the self-sufficiency of each microgrid is evaluated. \color{black}