\documentclass[article]{article}

\usepackage{amssymb}
\usepackage{amsmath}
\usepackage{amsfonts}
\usepackage{amsthm}
\usepackage{stmaryrd}
\usepackage[all]{xy}
\usepackage{mathrsfs}
\usepackage{graphicx}
\usepackage{hyperref}
\usepackage{color}
\usepackage{multirow}
\usepackage{extarrows}
\usepackage{amscd}
\usepackage{scalerel}
\usepackage{stackengine}
\usepackage{bbm}
\usepackage{mathtools}
\usepackage{cite}
\usepackage{tikz-cd}
\usepackage{bm}
\usepackage{gensymb}
\usepackage{verbatim}
\usepackage{enumerate}\usepackage{amssymb}
\usepackage{amsmath}
\usepackage{amsfonts}
\usepackage{amsthm}
\usepackage{stmaryrd}
\usepackage[all]{xy}
\usepackage{mathrsfs}
\usepackage{graphicx}
\usepackage{hyperref}
\usepackage{color}
\usepackage{multirow}
\usepackage{extarrows}
\usepackage{amscd}
\usepackage{scalerel}
\usepackage{stackengine}
\usepackage{bbm}
\usepackage{mathtools}
\usepackage{mathdots}
\usepackage{cite}
\usepackage{tikz-cd}
\usepackage{bm}
\usepackage{gensymb}
\usepackage{verbatim}
\usepackage{enumerate}
\usepackage{pictexwd,dcpic}



\numberwithin{equation}{section}
\newtheorem{theorem}{Theorem}[section]
\newtheorem{corollary}[theorem]{Corollary}
\newtheorem{lemma}[theorem]{Lemma}
\newtheorem{proposition}[theorem]{Proposition}
\newtheorem{desideratum}[theorem]{Desideratum}
\newtheorem{conjecture}[theorem]{Conjecture}
\newtheorem{remark}[theorem]{Remark}
\theoremstyle{definition}
\newtheorem{definition}[theorem]{Definition}


\newcommand{\ra}{\rightarrow}
\newcommand{\lra}{\longrightarrow}
\newcommand{\xra}{\xrightarrow}
\newcommand{\xlra}{\xlongrightarrow}
\newcommand{\hra}{\hookrightarrow}


\def\AAA{\mathbb{A}}
\def\BB{\mathbb{B}}
\def\CC{\mathbb{C}}
\def\DD{\mathbb{D}}
\def\EE{\mathbb{E}}
\def\FF{\mathbb{F}}
\def\GG{\mathbb{G}}
\def\HH{\mathbb{H}}
\def\II{\mathbb{I}}
\def\JJ{\mathbb{J}}
\def\KK{\mathbb{K}}
\def\LL{\mathbb{L}}
\def\MM{\mathbb{M}}
\def\NN{\mathbb{N}}
\def\OO{\mathbb{O}}
\def\PP{\mathbb{P}}
\def\QQ{\mathbb{Q}}
\def\RR{\mathbb{R}}
\def\SSS{\mathbb{S}}
\def\TT{\mathbb{T}}
\def\UU{\mathbb{U}}
\def\VV{\mathbb{V}}
\def\WW{\mathbb{W}}
\def\XX{\mathbb{X}}
\def\YY{\mathbb{Y}}
\def\ZZ{\mathbb{Z}}

\def\bfa{\mathbf{a}}
\def\bfb{\mathbf{b}}
\def\bfc{\mathbf{c}}
\def\bfd{\mathbf{d}}
\def\bfe{\mathbf{e}}
\def\bff{\mathbf{f}}
\def\bfg{\mathbf{g}}
\def\bfh{\mathbf{h}}
\def\bfi{\mathbf{i}}
\def\bfj{\mathbf{j}}
\def\bfk{\mathbf{k}}
\def\bfl{\mathbf{l}}
\def\bfm{\mathbf{m}}
\def\bfn{\mathbf{n}}
\def\bfo{\mathbf{o}}
\def\bfp{\mathbf{p}}
\def\bfq{\mathbf{q}}
\def\bfr{\mathbf{r}}
\def\bfs{\mathbf{s}}
\def\bft{\mathbf{t}}
\def\bfu{\mathbf{u}}
\def\bfv{\mathbf{v}}
\def\bfw{\mathbf{w}}
\def\bfx{\mathbf{x}}
\def\bfy{\mathbf{y}}
\def\bfz{\mathbf{z}}

\def\bfA{\mathbf{A}}
\def\bfB{\mathbf{B}}
\def\bfC{\mathbf{C}}
\def\bfD{\mathbf{D}}
\def\bfE{\mathbf{E}}
\def\bfF{\mathbf{F}}
\def\bfG{\mathbf{G}}
\def\bfH{\mathbf{H}}
\def\bfI{\mathbf{I}}
\def\bfJ{\mathbf{J}}
\def\bfK{\mathbf{K}}
\def\bfL{\mathbf{L}}
\def\bfM{\mathbf{M}}
\def\bfN{\mathbf{N}}
\def\bfO{\mathbf{O}}
\def\bfP{\mathbf{P}}
\def\bfQ{\mathbf{Q}}
\def\bfR{\mathbf{R}}
\def\bfS{\mathbf{S}}
\def\bfT{\mathbf{T}}
\def\bfU{\mathbf{U}}
\def\bfV{\mathbf{V}}
\def\bfW{\mathbf{W}}
\def\bfX{\mathbf{X}}
\def\bfY{\mathbf{Y}}
\def\bfZ{\mathbf{Z}}

\def\calA{\mathcal{A}}
\def\calB{\mathcal{B}}
\def\calC{\mathcal{C}}
\def\calD{\mathcal{D}}
\def\calE{\mathcal{E}}
\def\calF{\mathcal{F}}
\def\calG{\mathcal{G}}
\def\calH{\mathcal{H}}
\def\calI{\mathcal{I}}
\def\calJ{\mathcal{J}}
\def\calK{\mathcal{K}}
\def\calL{\mathcal{L}}
\def\calM{\mathcal{M}}
\def\calN{\mathcal{N}}
\def\calO{\mathcal{O}}
\def\calP{\mathcal{P}}
\def\calQ{\mathcal{Q}}
\def\calR{\mathcal{R}}
\def\calS{\mathcal{S}}
\def\calT{\mathcal{T}}
\def\calU{\mathcal{U}}
\def\calV{\mathcal{V}}
\def\calW{\mathcal{W}}
\def\calX{\mathcal{X}}
\def\calY{\mathcal{Y}}
\def\calZ{\mathcal{Z}}

\def\gotha{\mathfrak{a}}
\def\gothb{\mathfrak{b}}
\def\gothc{\mathfrak{c}}
\def\gothd{\mathfrak{d}}
\def\gothe{\mathfrak{e}}
\def\gothf{\mathfrak{f}}
\def\gothg{\mathfrak{g}}
\def\gothh{\mathfrak{h}}
\def\gothi{\mathfrak{i}}
\def\gothj{\mathfrak{j}}
\def\gothk{\mathfrak{k}}
\def\gothl{\mathfrak{l}}
\def\gothm{\mathfrak{m}}
\def\gothn{\mathfrak{n}}
\def\gotho{\mathfrak{o}}
\def\gothp{\mathfrak{p}}
\def\gothq{\mathfrak{q}}
\def\gothr{\mathfrak{r}}
\def\goths{\mathfrak{s}}
\def\gotht{\mathfrak{t}}
\def\gothu{\mathfrak{u}}
\def\gothv{\mathfrak{v}}
\def\gothw{\mathfrak{w}}
\def\gothx{\mathfrak{x}}
\def\gothy{\mathfrak{y}}
\def\gothz{\mathfrak{z}}

\def\gothA{\mathfrak{A}}
\def\gothB{\mathfrak{B}}
\def\gothC{\mathfrak{C}}
\def\gothD{\mathfrak{D}}
\def\gothE{\mathfrak{E}}
\def\gothF{\mathfrak{F}}
\def\gothG{\mathfrak{G}}
\def\gothH{\mathfrak{H}}
\def\gothI{\mathfrak{I}}
\def\gothJ{\mathfrak{J}}
\def\gothK{\mathfrak{K}}
\def\gothL{\mathfrak{L}}
\def\gothM{\mathfrak{M}}
\def\gothN{\mathfrak{N}}
\def\gothO{\mathfrak{O}}
\def\gothP{\mathfrak{P}}
\def\gothQ{\mathfrak{Q}}
\def\gothR{\mathfrak{R}}
\def\gothS{\mathfrak{S}}
\def\gothT{\mathfrak{T}}
\def\gothU{\mathfrak{U}}
\def\gothV{\mathfrak{V}}
\def\gothW{\mathfrak{W}}
\def\gothX{\mathfrak{X}}
\def\gothY{\mathfrak{Y}}
\def\gothZ{\mathfrak{Z}}

\def\scra{\mathscr{a}}
\def\scrb{\mathscr{b}}
\def\scrc{\mathscr{c}}
\def\scrd{\mathscr{d}}
\def\scre{\mathscr{e}}
\def\scrf{\mathscr{f}}
\def\scrg{\mathscr{g}}
\def\scrh{\mathscr{h}}
\def\scri{\mathscr{i}}
\def\scrj{\mathscr{j}}
\def\scrk{\mathscr{k}}
\def\scrl{\mathscr{l}}
\def\scrm{\mathscr{m}}
\def\scrn{\mathscr{n}}
\def\scro{\mathscr{o}}
\def\scrp{\mathscr{p}}
\def\scrq{\mathscr{q}}
\def\scrr{\mathscr{r}}
\def\scrs{\mathscr{s}}
\def\scrt{\mathscr{t}}
\def\scru{\mathscr{u}}
\def\scrv{\mathscr{v}}
\def\scrw{\mathscr{w}}
\def\scrx{\mathscr{x}}
\def\scry{\mathscr{y}}
\def\scrz{\mathscr{z}}

\def\scrA{\mathscr{A}}
\def\scrB{\mathscr{B}}
\def\scrC{\mathscr{C}}
\def\scrD{\mathscr{D}}
\def\scrE{\mathscr{E}}
\def\scrF{\mathscr{F}}
\def\scrG{\mathscr{G}}
\def\scrH{\mathscr{H}}
\def\scrI{\mathscr{I}}
\def\scrJ{\mathscr{J}}
\def\scrK{\mathscr{K}}
\def\scrL{\mathscr{L}}
\def\scrM{\mathscr{M}}
\def\scrN{\mathscr{N}}
\def\scrO{\mathscr{O}}
\def\scrP{\mathscr{P}}
\def\scrQ{\mathscr{Q}}
\def\scrR{\mathscr{R}}
\def\scrS{\mathscr{S}}
\def\scrT{\mathscr{T}}
\def\scrU{\mathscr{U}}
\def\scrV{\mathscr{V}}
\def\scrW{\mathscr{W}}
\def\scrX{\mathscr{X}}
\def\scrY{\mathscr{Y}}
\def\scrZ{\mathscr{Z}}

\def\ul{\underline}
\def\wt{\widetilde}
\def\wh{\widehat}
\def\defeq{\mathrel{\mathop=^{\rm def}}}
\def\sl{\goths\gothl}

\def\alg_k{\AAA\mathbbm{l}\mathbbm{g}_{/k}}
\def\idg{\mathbbm{1}_{\gothg(\calO)}}

\newcommand{\superb}{\mathrm{sb}}

\DeclareMathOperator{\Art}{Art}
\DeclareMathOperator{\Aut}{Aut}
\DeclareMathOperator{\Cone}{Cone}
\DeclareMathOperator{\End}{End}
\DeclareMathOperator{\Gal}{Gal}
\DeclareMathOperator{\Ker}{Ker}
\DeclareMathOperator{\im}{Im}
\DeclareMathOperator{\coker}{Coker}
\DeclareMathOperator{\Ad}{Ad}


\DeclareMathOperator{\Hdg}{Hdg}
\DeclareMathOperator{\Hom}{Hom}
\DeclareMathOperator{\Ext}{Ext}
\DeclareMathOperator{\WD}{WD}

\DeclareMathOperator{\cHom}{\mathcal{H}om}

\DeclareMathOperator{\id}{id}
\DeclareMathOperator{\Image}{Im}
\DeclareMathOperator{\Ind}{Ind}
\DeclareMathOperator{\ind}{ind}
\DeclareMathOperator{\Lie}{Lie}
\DeclareMathOperator{\rank}{rank}
\DeclareMathOperator{\Rec}{Rec}
\DeclareMathOperator{\Res}{Res}
\DeclareMathOperator{\Spec}{Spec}
\DeclareMathOperator{\Spf}{Spf}
\DeclareMathOperator{\GL}{GL}
\DeclareMathOperator{\Gaf}{\GL_2(\mathbb{A}_f)}
\newcommand{\Af}{\mathbb{A}_f}
\DeclareMathOperator{\SL}{SL}
\DeclareMathOperator{\MP}{Mp}
\DeclareMathOperator{\GSP}{GSp}
\DeclareMathOperator{\SP}{Sp}
\DeclareMathOperator{\SO}{SO}
%\DeclareMathOperator{\O}{O}
\DeclareMathOperator{\Proj}{Proj}
\DeclareMathOperator{\GS}{\Gamma }
\DeclareMathOperator{\Vol}{Vol}
\DeclareMathOperator{\Irr}{Irr}
\DeclareMathOperator{\Irrt}{Irr_{temp}}
\DeclareMathOperator{\Para}{\Phi_{temp}}
\DeclareMathOperator{\disc}{disc}
\DeclareMathOperator{\JH}{JH}
\DeclareMathOperator{\pos}{pos}% for syzygy, labeled simplicial complex
\DeclareMathOperator{\coH}{H}% homology
\DeclareMathOperator{\thetabigsptoo }{\theta_{V^{\varepsilon}_{2n},W_{2n},\psi}}
\DeclareMathOperator{\thetasmallsptoo }{\theta_{V^{\varepsilon}_{2n+2},W_{2n},\psi}}
\DeclareMathOperator{\thetabigotosp }{\theta_{W_{2n-2},V^{\varepsilon}_{2n},\psi}}
\DeclareMathOperator{\thetasmallotosp }{\theta_{W_{2n},V^{\varepsilon}_{2n},\psi}}

\newcommand{\length}{\mathrm{length}}
\newcommand{\A}{\mathbb{A}}
\newcommand{\Q}{\mathbb{Q}}
\newcommand{\Z}{\mathbb{Z}}
\newcommand{\R}{\mathbb{R}}
%\newcommand{\C}{\mathbb{C}}
%\newcommand{\G}{\mathrm{G}}
\newcommand{\cO}{\mathcal{O}}

\newcommand{\Sh}{\mathrm{Sh}}
\newcommand{\X}{\mathrm{X}}
\newcommand{\Y}{\mathrm{Y}}
\newcommand{\V}{\mathrm{V}}
\newcommand{\M}{\mathrm{M}}
\newcommand{\Div}{\mathrm{Div}}
\newcommand{\Cl}{\textrm{Cl}}
%\newcommand{\id}{\mathrm{id}}

%\renewcommand{\proofname}{Sketch of proof}

\begin{document}

\title{Local Langlands Correspondence for Even Orthogonal Groups via Theta Lifts}
\author{Rui Chen \and Jialiang Zou}
\maketitle
%\tableofcontents

\begin{abstract}
Using theta correspondence, we obtain a classification of the irreducible representations of an arbitrary even orthogonal group (i.e.\ the local Langlands correspondence) by deducing it from the local Langlands correspondence for symplectic groups due to Arthur. Moreover, we show that our classifications coincide with the local Langlands correspondence established by Arthur \cite{MR3135650} and formulated precisely by Atobe-Gan \cite{MR3708200} for quasi-split even orthogonal groups. 
\end{abstract}

\section{Introduction}
In the monumental book \cite{MR3135650}, Arthur obtained a classification of irreducible representations of symplectic and quasi-split special orthogonal groups over local fields
of characteristic 0 (the local Langlands correspondence) as well as a description of the automorphic discrete spectra of these groups over number fields (the Arthur conjecture). He proved these results by using the theory of endoscopy and the stable trace formula. Following Arthur's method, Mok established the same
results for quasi-split unitary groups \cite{MR3338302}. Later in \cite{MR3708200}, Atobe-Gan formulated precisely the local Langlands correspondence (LLC for short) for quasi-split even orthogonal groups and their pure inner forms (using Vogan $L$-packet \cite{MR1216197}). They highlighted that Arthur's results will imply the LLC for quasi-split even orthogonal groups. They also formulated the local intertwining relation (see section \ref{LIR}), which plays a key role in LLC. 

The main goal of this paper is to construct an LLC for quasi-split even orthogonal groups and their pure inner forms. We also prove several desired properties of this construction; in particular, we prove the local intertwining relation stated in \cite[\S 3.7]{MR3708200}. Moreover, we shall show that our LLC is the same as Arthur's LLC in the quasi-split case. We also write a parallel paper \cite{CZ2}, in which we deal with the unitary group case (We write it separately to avoid making notations too complicated). In a sequel to this paper, with these LLC at hand, we shall investigate the Arthur's conjecture for automorphic discrete spectra of even orthogonal groups and unitary groups. 

Next we state our results more precisely. Let $F$ be a non-archimedean local field of characteristic $0$ and $V_{2n}$ be an orthogonal space over $F$ with dimension $2n$ and discriminant character $\chi_{V}$. The LLC for even orthogonal groups provides a partition 
\begin{align}\label{1}
\Irr(\mathrm O(V_{2n})) =\bigsqcup_{\phi} \Pi_{\phi}(\mathrm O(V_{2n})),
\end{align}
where the disjoint union runs over equivalence classes of $2n$-dimensional orthogonal representations $\phi$ of $\WD_F$ with $\det \phi=\chi_{V}$. Moreover, for a fixed Whittaker datum $\mathfrak W_{c}$ of $\bigsqcup_{V_{2n}^\bullet} \mathrm O(V_{2n}^\bullet)$, where the disjoint union runs over isometry classes of $2n$-dimensional orthogonal spaces $V_{2n}^\bullet$ with discriminant character $\chi_{V}$ (see subsection \ref{whittaker}), it provides a canonical bijection 
\begin{align}\label{2}
\mathcal J_{\mathfrak W_c}: \bigsqcup_{V_{2n}^\bullet} \Pi_{\phi}(\mathrm O(V_{2n}^\bullet)) \longleftrightarrow \widehat {\mathcal {S}_{\phi}},
\end{align}
where $\mathcal {S}_{\phi}$ is the component group of the centralizer of image of $\phi$ in $\mathrm O_{2n}(\mathbb C)$ and $\widehat {\mathcal {S}_{\phi}}$ is the group of characters of $\mathcal {S}_{\phi}$; see subsection \ref{Lparmeter} below. This classification is supposed to satisfy a list of expected properties (see Theorem \ref{desideratumall}), which characterize it uniquely. 


We construct such a LLC for even orthogonal groups by transporting Arthur's LLC for symplectic groups via theta correspondence. More precisely, fix a non-trivial additive character of $F$. Let $V$ and $W$ be orthogonal and symplectic spaces over $F$ respectively. We denote the Weil representation of $\mathrm O(V)\times \SP(W)$ by $\omega=\omega_{V,W,\psi}$. For $\pi\in 
\Irr(\mathrm O(V))$, the maximal $\pi$-isotypic quotient of $\omega$ is of the form
$$\pi \boxtimes \Theta_{W,V,\psi}(\pi),$$
where $\Theta_{W,V,\psi}(\pi)$ is a finite length smooth representation of $\SP(W)$. Let $\theta_{W,V, \psi}(\pi)$ be the maximal semisimple quotient of $\Theta_{W,V, \psi}(\pi)$. The Howe duality conjecture, which was proved by Waldspurger \cite{MR1159105}, Gan-Takeda \cite{ MR3454380} and Gan-Sun \cite{MR3753911}, says that $\theta_{W, V,  \psi}(\pi)$ is irreducible (if they are
non-zero). We call $\theta_{W,V,\psi}(\pi)$ the theta lift of $\pi$. It is natural to ask when $\theta_{V,W,\psi}(\pi)$ is nonzero and determine $\theta_{V,W,\psi}(\pi)$ precisely when it is nonzero. A special case when $$|\dim V-\dim W|\leq 2$$ is considered by Prasad \cite{MR1248702}. Assuming the LLC for both even orthogonal groups and symplectic groups, the Prasad conjecture describes the theta lift of $\pi\in \Irr(\mathrm O(V))$ to $\theta_{ W,V,\psi}(\pi)\in \Irr(\SP(W))$ in terms of their $L$-parameter and characters of component group; see \cite[\S 4]{MR3708200}. The Prasad conjecture was proved by Atobe-Gan \cite{MR3708200} assuming the LLC for even orthogonal groups. In this paper, we turn the table around, i.e., we use the LLC for symplectic groups and the local theta correspondence to construct the LLC for even orthogonal groups. 

As in other instances where the LLC was shown using the theta correspondence (such as \cite{MR2999299} and \cite{MR2800725}), 
we do not show the (twisted) endoscopic character relations for the $L$-packets we constructed. To show that our $L$-packets satisfy the endoscopic character relations, one would need to appeal to the stable trace formula (or a simple form of it), as was done in \cite{MR3267112} and \cite{luo2020endoscopic}. 

 
This paper is organized as follows. First we recall some basic facts in representation theory and local theta correspondence in section 2 and 3. Then we formulate the main theorem (i.e. the desired LLC) in section 4, taking the chance to recall the results from Arthur \cite{MR3135650} and Atobe-Gan \cite{MR3708200} that we are using. In section 5 and 6, we give our construction of the LLC and prove several properties of this construction. The local intertwining relation is stated and proved in section 7 and 8. In sect 9, we prove that the LLC we construct coincides with Arthur's LLC for quasi-split even orthogonal groups. Finally we finish the proof of the main theorem in section 10. In Appendix A, we deduce the weak LLC of arbitrary even special orthogonal groups from the LLC of even orthogonal groups. In Appendix B, we recall the definition of Plancherel measures and prove a lemma on normalized intertwining operators. 

\section*{Acknowledgments} 
We would like to thank our supervisor Wee Teck Gan for many useful advices. We would also thank Hiraku Atobe, Atsushi Ichino, Wen-Wei Li, and Sug Woo Shin for helpful conversations during the conference ``Workshop on Shimura varieties, representation theory and related topics, 2019'' in Hokkaido University. We thank Hiroshi Ishimoto, Caihua Luo, Xiaolei Wan, and Chuijia Wang for helpful discussions. Both authors are supported by an MOE Graduate Research Scholarship. 
\section*{Notation}
Let $F$ be a non-archimedean local field of characteristic 0 and residue characteristic $p$. Let $W_F$ be the Weil group of $F$. Let $|\cdot|_F$ be the normalized absolute value of $F$. We fix a non-trivial additive character $\psi$ of $F$, and for $c\in F^{\times}$, we define an additive character $\psi_{c}$ of $F$ by 
$$\psi_{c}(x)=\psi(cx).$$ Note that any non-trivial additive character of $F$ is of the form $\psi_c$ for some $c\in F^{\times}$. Let $(\cdot,\cdot)_F$ be the quadratic Hilbert symbol of $F$. 

If $G$ is the $F$-rational points of a linear algebraic group, we let $\Irr(G)$ be the set of equivalence classes of irreducible smooth representations of $G$ and $\Irrt(G)$ be the subset of $\Irr(G)$ consisting of tempered representations of $G$. Let $\pi$ be a representation of $G$, the contragredient representation of $\pi$ is denote by $\pi^\vee$. 



\section{Orthogonal and Symplectic group}
\subsection{Orthogonal space}
Let $V=V_{2m}$ be an orthogonal space of dimension $2m$ over $F,$ i.e., a vector space equipped with a non-degenerate symmetric bilinear form 
$$\langle\cdot, \cdot\rangle_{V} : V \times V \rightarrow F.$$
We take a basis $\left\{e_{1}, \ldots, e_{2m}\right\}$ of $V,$ and define the discriminant of $V$ by

$$\disc(V)=(-1)^{m} \det ((\left\langle e_{i}, e_{j}\right\rangle_{V})_{i, j})  \in F^{ \times} / F^{ \times 2}.$$
Let $$\chi_{V}=(\cdot, \disc(V))_F$$ be the character of $F^{\times}$ associated to $F(\sqrt{\disc(V)})$. We call $\chi_{V}$ the discriminant character of $V$. Let $q$ be the quadratic form on $V$ defined by 
\begin{align*}
q(v)=\frac{1}{2}\langle v, v\rangle_{V}.
\end{align*}
We define the normalized Hasse-Witt invariant $\epsilon(V)\in \{\pm 1\}$ of $V$ to be the Witt-invariant associated to the quadratic form $q$; see \cite[pp.80-81]{MR770063}. The isometry class of $2m$-dimensional orthogonal spaces $V$ is uniquely determined by these two invariants $\disc(V)$ and $\epsilon(V)$. Moreover, there exists an orthogonal space of dimension $2m$, discriminant $\disc(V)\in  F^{\times}/F^{\times 2}$ and normalized Hasse-Witt invariant $\epsilon(V)\in \{\pm 1\}$, unless $m=1$,  $\disc(V)\in F^{\times 2}$ and $\epsilon(V)= -1$.

The orthogonal group $\mathrm{O}(V_{2m})$ associated to $V_{2m}$ is defined by $$\mathrm{O}(V_{2m})=\left\{g \in \GL(V_{2m}) |\left\langle g v, g v^{\prime}\right\rangle_{V}=\left\langle v, v^{\prime}\right\rangle_{V} \text { for all } v, v^{\prime} \in V_{2m}\right\},$$
and the special orthogonal group $\SO(V_{2m})$ is defined by 
\begin{align*}
\SO(V_{2m})=\{g\in \mathrm O(V_{2m})|\det g=1\}.
\end{align*}

Fix $c, d \in F^{ \times}$. Let 
$$V_{(d, c)}=F[X] /(X^{2}-d)$$
be a $2$-dimensional vector space equipped with a bilinear form
$$(\alpha, \beta) \mapsto\langle\alpha, \beta\rangle_{V_{(d, c)}} \coloneqq c \cdot \operatorname{tr}(\alpha \overline{\beta}),$$
where $\beta \mapsto \overline{\beta}$ is the involution on $F[X] /\left(X^{2}-d\right)$ induced by $a+b X \mapsto a-b X$. This involution is regarded as an element $\epsilon \in \mathrm O\left(V_{(d, c)}\right)$. The images of $1, X \in F[X]$ in $V_{(d, c)}$ are denoted by $e, e^{\prime},$ respectively.

For $m>1,$ we say that $V_{2m}$ is associated to $(d, c)$ if
\begin{align}\label{127}
V_{2m} \cong V_{(d,c)} \oplus \mathbb{H}^{m-1}
\end{align}
as orthogonal spaces, where $\mathbb{H}$ is the hyperbolic plane, i.e., $\mathbb{H}=F v_{i}+F v_{i}^{*}$ with
$$\left\langle v_{i}, v_{i}\right\rangle_{V}=\left\langle v_{i}^{*}, v_{i}^{*}\right\rangle_{V}=0\quad \mbox{and}\quad  \left\langle v_{i}, v_{i}^{*}\right\rangle_{V}= 1. $$  
Note that if $V_{2m}$ is associated to $(d,c)$, then 
\begin{align*}
\disc (V_{2m})= d \bmod F^{\times 2}\quad\mbox{and}\quad \epsilon(V)=(c,d)_{F}.
\end{align*}
Hence $$V_{(d, c)} \oplus 
\mathbb{H}^{m-1} \cong V_{\left(d^{\prime}, c^{\prime}\right)} \oplus \mathbb{H}^{n-1}$$
as orthogonal spaces if and only if 
$$d \equiv d^{\prime} \bmod F^{\times 2}\quad \mbox{and}\quad c \equiv c^{\prime} \bmod N_{E / F}\left(E^{ \times}\right),$$ 
where $E= F(\sqrt{d})=F(\sqrt{d^{\prime}})$. 

When $d \notin F^{\times 2}$, let $E=F(\sqrt{d})$ and choose a $c_0\notin N_{E/F}(E^\times)$. We define
\begin{align}\label{139}
V_{2m}^+\cong V_{(d,1)}\oplus \mathbb H^{m-1},\quad 
V_{2m}^-\cong V_{(d,c_0)}\oplus \mathbb H^{m-1}
\end{align}
to be the two different orthogonal spaces of dimension $2m$ and discriminant $d$. Note that  
\begin{align*}
\epsilon(V_{2m}^+)=1\quad \mbox{and}\quad  \epsilon(V_{2m}^-)=-1.
\end{align*}

When $d \in F^{ \times 2}$, let $D$ be the unique division algebra over $F$ of dimension $4$, which can be regarded as an orthogonal space with a bilinear form
$$(\alpha, \beta) \mapsto \tau\left(\alpha \beta^{\iota}\right),$$
where $\tau$ is the reduced trace and $\beta \mapsto \beta^{\iota}$ is the main involution. Then we define  
\begin{align}\label{140}
V_{2m}^+\cong V_{(d,1)}\oplus \mathbb H^{m-1}, \quad 
V_{2m}^-\cong D\oplus \mathbb H^{m-2} \quad \mbox{if $m\geq 2$}
\end{align} 
to be the two different orthogonal spaces of dimension $2m$ and discriminant $d$. Note that  
\begin{align*}
\epsilon(V_{2m}^+)=1\quad \mbox{and}\quad \epsilon(V_{2m}^-)=-1. 
\end{align*}


The orthogonal group $\mathrm{O}\left(V_{2 m}\right)$ is quasi-split iff $V_{2m}$ is associated to $(d,c)$ for some $d,c\in F^\times$. Note that when $\disc V=d \notin F^{\times 2}$, both $\mathrm O(V_{2m}^+)$ and $\mathrm O(V_{2m}^-)$ are quasi-split.

Let $(V_{2m},\langle\cdot,\cdot\rangle_{V_{2m}})$ be an orthogonal space associated to $(d,c)$ and $(V_{2m}^\prime,\langle\cdot,\cdot\rangle_{V_{2m}^\prime})$ be an orthogonal space associated to $(d,ac)$ for any $a\in F^{\times}$. We define a new orthogonal space $(\widetilde{V}_{2m},\langle\cdot,\cdot\rangle)_{\widetilde {V}_{2m}}$ by $\widetilde{V}_{2m}=V_{2m}$ as a vector space and 
\begin{align*}
\langle x, y\rangle_{\widetilde {V}_{2m}}= a\cdot \langle x,y\rangle_{V_{2m}}. 
\end{align*}
As subgroups of $\GL(V_{2m})=\GL(\widetilde{V}_{2m})$, we have
\begin{align*}
\mathrm O(V_{2m})=\mathrm O(\widetilde{V}_{2m}). 
\end{align*}
On the other hand, we have $\widetilde{V}_{2m}\cong V_{2m}^\prime$ as  orthogonal spaces. Hence we have an isomorphism 
\begin{align}\label{143}
\iota_a: \mathrm O(V_{2m}) \cong \mathrm O(V_{2m}^\prime). 
\end{align} 
In particular, if $d\notin F^{\times 2}$, by \ref{139}, we know that $V_{2m}^+$ is associated to $(d,1)$ and $V_{2m}^+$ is associated to $(d,c_0)$. So by \ref{143}, there is an isomorphism 
\begin{align}\label{141}
\iota_{c_0}: \mathrm O(V_{2m}^+)\cong \mathrm O(V_{2m}^-). 
\end{align}

%for both $V_{2m}^+$ and $V_{2m}^-$ having discrimiant $d$.  
	



\subsection{Symplectic space}\label{sympleticspace}
Let $W=W_{2 n}$ be a symplectic space of dimension $2n$ over $F,$ i.e., a vector space equipped with a non-degenerate symplectic form
$$\langle\cdot, \cdot\rangle_{W} : W \times W \rightarrow F.$$
The symplectic space is always split, i.e 
\begin{align}\label{128}
W_{2n}\cong \mathbb{H}^{n} ,
\end{align}
where $\mathbb{H}=F w_i+ Fw_{i}^{*}$ with $$\langle w_i,w_i\rangle_{W} =\langle w_{i}^{*},  w_{i}^{*}\rangle_{W}=0 \quad \mbox{and} \quad \langle w_i, w_{j}^{*}\rangle_{W}=-\langle w_{j}^{*}, w_{i}\rangle_{W}=\delta_{ij}.$$
The symplectic group  $\SP(W)$ associated to $W$ is defined by 
$$\operatorname{Sp}(W)=\left\{g \in \mathrm{GL}(W) |\left\langle g w, g w^{\prime}\right\rangle_{W}=\left\langle w, w^{\prime}\right\rangle_{W} \text { for all } w, w^{\prime} \in W\right\}.$$

\subsection{Parabolic subgroups}\label{sectionparabolic}
Let $r$ be the Witt index of $V$ and $V_{\mathrm{an}}$ an anisotropic kernel of $V$. Choose a basis $\{v_i,v_i^*|i=1,\cdots,r\}$ of the orthogonal complement of $V_{\mathrm{an}}$ such that  
$$
\langle v_i, v_j\rangle_V=\langle v_i^*, v_j^*\rangle_V=0 \quad \mbox{and}\quad \langle v_i, v_j^*\rangle_V=\delta_{i, j}\quad \mbox{for $1\leq i,j\leq r$}.
$$
Let $k$ be a positive integer with $k\leq r$ and set 
$$
X=X_k=Fv_1+\cdots+Fv_k \quad \mbox{and}\quad X^\vee=X_k^{\vee}=Fv_1^*+\cdots+Fv_k^*.
$$
Let $V_0$ be the orthogonal complement of $X \oplus X^{\vee}$ in $V$ so that $V_0$ is an orthogonal space of dimensional $2m_0=2m-2k$ over $F$. We shall write an element in $\mathrm{O}(V)$ as a block matrix relative to the decomposition $V=X\oplus V_0\oplus X^\vee$. 

Let $P=P_{k}=M_P U_{P}$ be the maximal parabolic subgroup of $\mathrm O(V)$ stabilizing $X$, where $M_{P}$ is the Levi component of $P$ stabilizing $X^{\vee}$. We have 
$$\begin{aligned} M_{P} &=\left\{m_{P}(a) \cdot h_0 | a \in \mathrm{GL}(X), h_0 \in \mathrm{O}(V_0)\right\} ,\\ U_{P} &=\left\{u_{P}(b) \cdot u_{P}(c) | b \in \operatorname{Hom}(V_0, X), c \in \operatorname{Sym}\left(X^{\vee}, X\right)\right\}, \end{aligned}$$ 
where 
\begin{align*}
m_{P}(a)&=\left(\begin{array}{ccc}{a} & {}& \\ {} & {\mathbf 1_{V_0}} & {} \\ {} & {} & {\left(a^{*}\right)^{-1}}\end{array}\right),\\
u_{P}(b)&=\left(\begin{array}{ccc}{\mathbf 1_{X}} & {b} & {-\frac{1}{2} b b^{*}} \\ {} & {\mathbf 1_{V_0}} & {-b^{*}} \\ {} & {} & {\mathbf 1_{X^{\vee}}}\end{array}\right),\\
u_{P}(c)&=\left(\begin{array}{ccc}{\mathbf 1_{X}} & {} & {c} \\ {} & {\mathbf 1_{V_0}} & {} \\ {} & {} & {\mathbf 1_{X^{\vee}}}\end{array}\right),
\end{align*}
and 
$$\operatorname{Sym}\left(X^{\vee}, X\right)=\left\{c \in \operatorname{Hom}\left(X^{\vee}, X\right) | c^{*}=-c\right\}.$$
Here, the elements $a^{*} \in \GL\left(X^{\vee}\right), b^{*} \in \operatorname{Hom}\left(X^{\vee}, V\right),$ and $c^{*} \in \operatorname{Hom}\left(X^{\vee}, X\right)$ are defined by requiring that 
\begin{align*}
\langle a x, x^{\prime}\rangle_{V}=\langle x, a^{*} x^{\prime}\rangle_{V},\quad \langle b v, x^{\prime}\rangle_{V}=\langle v, b^{*} x^{\prime}\rangle_{V},\quad
\langle c x^{\prime}, x^{\prime \prime}\rangle_{V}=\langle x^{\prime}, c^{*} x^{\prime \prime}\rangle_{V}
\end{align*}
for $x \in X, x^{\prime}, x^{\prime \prime} \in X^{\vee}$ and $v \in V_0$. In particular, $M_P\cong \GL(X)\times \mathrm O(V_0)$.  
Put 
$$\rho_{P}=\frac{2m_0+k-1}{2}, \quad w_{P}=\left(\begin{array}{ccc}
{} &{}& -I_{X} \\ 
{} &{\mathbf 1_{V_0}} & {}\\
{-I_{X}^{-1}}&{}& 
\end{array}\right),$$
where $I_{X} \in \operatorname{Hom}\left(X^{\vee}, X\right)$ is defined by $I_{X} v_{i}^{*}=v_{i}$ for $1 \leq i \leq k .$ Then the modulus character $\delta_{P}$ of $P$ is given by
$$\delta_{P}\left(m_{P}(a) h_0 u_{P}\right)=|\operatorname{det}(a)|_{F}^{2 \rho_{P}}$$
for $a \in \mathrm{GL}(X), h \in \mathrm{O}(V_0)$ and $u_{P} \in U_{P}$. 

Similarly, Let $W=W_{2n}$ be a $2n$ dimensional symplectic space and  $k$ be a positive integer with $k\leq n$. Write $W\cong \mathbb H^n$ as in section \ref{sympleticspace} and set 
$$
Y=Y_k=Fw_1+\cdots+Fw_k \quad \mbox{and}\quad Y^\vee=Y_k^\vee=Fw_1^*+\cdots+Fw_k^*.
$$
Let $W_0$ be the orthogonal complement of $Y \oplus Y^{\vee}$ in $W$ so that $W_0$ is a symplectic space of dimensional $2n_0=2n-2k$ over $F$. We define $Q=M_{Q} U_{Q}, m_{Q}(a), u_{Q}(b), u_{Q}(c), \operatorname{Sym}\left(Y^{\vee}, Y\right)=\left\{c \in \Hom(Y^{\vee}, Y) | c^{*}=-c\right\}$ and $I_Y$ as above. We put  
$$\rho_{Q}=\frac{2n_0+k+1}{2}, \quad w_{Q}=\left(\begin{array}{ccc}
{} &{}& -I_{Y} \\ 
{} &\mathbf 1_{W_0} & {}\\
{I_{Y}^{-1}}&{}& 
\end{array}\right).$$
Then the modulus character $\delta_{Q}$ of $Q$ is given by
$$\delta_{Q}\left(m_{Q}(a) g_0 u_{Q}\right)=|\operatorname{det}(a)|_{F}^{2 \rho_{Q}}$$
for $a \in \mathrm{GL}(Y), g_0 \in \mathrm{Sp}(W_0)$ and $u_{Q} \in U_{Q}$.




% Let $W$ with $\operatorname{dim}(W)=2 n, Q=M_{Q} U_{Q} \subset \operatorname{Sp}\left(W^{\prime}\right), m_{Q}(a),u_{Q}(b), u_{Q}(c), \operatorname{Sym}\left(Y^\vee, Y\right)=\left\{c \in \operatorname{Hom}\left(Y^\vee, Y\right) | c^{*}=-c\right\}$ and $I_{Y}$ as above. We put 
%$$\rho_{Q}=\frac{2n+k+1}{2}, \quad \omega_{Q}=\left(\begin{array}{ccc}
%{} &{}& -I_{Y} \\ 
%{} &{1_{W}} & {}\\
%{I_{Y}^{-1}}&{}& 
%\end{array}\right)$$
%Then the modulus character $\delta_{Q}$ of $Q$ is given by
%$$\delta_{Q}\left(m_{Q}\left(a^{\prime}\right) g u_{Q}\right)=|\operatorname{det}(a)|_{F}^{2 \rho_{Q}}$$
%for $a^{\prime} \in \mathrm{GL}\left(Y^{\prime}\right), g \in \mathrm{Sp}(W)$ and $u_{Q} \in U_{Q}$




\section{Theta correspondence}
In this section, we introduce the local theta correspondence for $\mathrm O(V)\times \SP(W)$ when $\dim V=2m$ and $\dim W=2n$ and recall some general results. 
\subsection{Weil representation and local theta correspondence}\label{localthteta}
We fix a non-trivial additive character $\psi$ of $F$. Let $V=V_{2m}$ and $W=W_{2n}$. We denote the Weil representation of $\mathrm O(V)\times Sp(W)$ by $\omega=\omega_{V,W,\psi}$; see \cite{MR1286835} for a detail description. For $\pi\in 
\Irr(\mathrm O(V))$, the maximal $\pi$-isotypic quotient of $\omega$ is of the form
$$\pi \boxtimes \Theta_{W,V,\psi}(\pi),$$
where $\Theta_{W,V,\psi}(\pi)$ is a finite length smooth representation of $\SP(W)$. Let $\theta_{W,V, \psi}(\pi)$ be the maximal semisimple quotient of $\Theta_{W,V, \psi}(\pi)$. The Howe duality conjecture, which was proved by Waldspurger \cite{MR1159105} when the residue characteristic is not 2 and by Gan-Takeda \cite{MR3454380}, Gan-Sun \cite{MR3753911} in general, says that 
\begin{itemize}
	\item if $\theta_{W, V,  \psi}(\pi)$ is non-zero, then it is  irreducible;
	\item If $\pi_1\ncong \pi_2$ and both $\theta_{W, V,  \psi}(\pi_1)$ and $\theta_{W, V,  \psi}(\pi_2)$ are nonzero, then $\theta_{W, V, \psi}(\pi_1)\ncong \theta_{W, V,  \psi}(\pi_2)$. 
\end{itemize}
Similarly, for $\sigma \in \Irr(\SP(W))$, We obtain smooth finite length representations $\Theta_{V,W, \psi}(\sigma)$ and $\theta_{V, W, \psi}(\sigma)$ of $\mathrm O(V)$.


We will recall some general properties of local theta correspondence in the next two subsections. 
\subsection{First occurrence and tower property}
Let $V_{2m_0}$ be an anisotropic orthogonal space over $F$, where
\begin{align*}
V_{2m_0}= \begin{cases*}
0\quad\quad\,\mbox{or}\quad D \quad &\mbox{if $\disc(V_{2m_0})=1\in F^{\times 2}$},\\
V_{(d,1)}\,\, \mbox{or}\quad  V_{(d,c_0)}\,\,( c_0\notin N_{E/F}(E^\times))\quad & \mbox{if $\disc(V_{2m_0})=d\notin  F^{\times 2}$}. 
\end{cases*}
\end{align*}
For any $r\geq 0$, we put 
\begin{align*}
V_{2m_0+2r}=V_{2m_0}\oplus \mathbb H^r,
\end{align*}
where $\mathbb H$ is the hyperbolic plane. The collection 
\begin{align*}
\mathcal V=\{V_{2m_0+2r}|r\geq 0\}
\end{align*} 
is called the Witt tower of orthogonal spaces. One can consider a tower of the theta correspondence associated to reductive dual pairs $\{(\SP(W_{2n}),\mathrm O(V_{2m}))|V_{2m}\in \mathcal V)\}$. For $\sigma \in \Irr(\SP(W_{2n}))$ and a Witt tower 
$\mathcal V=\{V_{2m_0+2r}|r\geq 0\}$, we define  
\begin{align}\label{09}
m_{\mathcal V}(\sigma)=\min \{2m |\Theta_{V_{2m},W_{2n},\psi}(\sigma)\neq 0\}. 
\end{align}
By \cite[P.67]{MR1041060}, $m_{\mathcal V}(\sigma)$ is finite. 


On the other hand, every symplectic space $W_{2n}$ of dimension $2n$ is isomorphic to $\mathbb H^n$, where $\mathbb H$ is the hyperbolic plane. The collection 
\begin{align*}
\mathcal W=\{W_{2r}|r\geq 0\}
\end{align*}
is called a Witt tower of symplectic spaces. One can also consider a tower of the theta correspondence associated to reductive dual pairs $\{(\mathrm O(V_{2m}),\SP(W_{2n}))|W_{2n}\in \mathcal W)\}$. For $\pi\in \Irr(\mathrm O(V_{2m}))$ and a Witt tower $\mathcal W=\{W_{2r}|r\geq 0\}$, we define 
\begin{align}\label{10}
m_{\mathcal W}(\pi)=\min \{2n |\Theta_{W_{2n},V_{2m},\psi}(\pi)\neq 0\}. 
\end{align}
By \cite[P.67]{MR1041060}, $m_{\mathcal W}(\pi)$ is finite.

The following proposition is often referred to as the tower property of theta correspondence; see \cite{MR818351}. 
\begin{proposition}\label{tower}
For $\sigma \in \Irr(\SP(W_{2n}))$ and a Witt tower 
$\mathcal V=\{V_{2m_0+2r}|r\geq 0\}$ (resp. $\pi\in \Irr(\mathrm O(V_{2m}))$ and a Witt tower $\mathcal W=\{W_{2r}|r\geq 0\}$), let $m_{\mathcal V}(\sigma)$ (resp. $m_{\mathcal W}(\pi)$) be defined in \ref{09} (resp. \ref{10}). Then we have 
	\begin{align*}
	\Theta_{V_{2m},W_{2n},\psi}(\sigma)\neq 0 \quad \mbox{if}\quad& 2m\geq m_{\mathcal V}(\sigma),\\
	\Theta_{W_{2n},V_{2m},\psi}(\pi)\neq 0 \quad \mbox{if}\quad &2n\geq m_{\mathcal W}(\pi).
	\end{align*}
\end{proposition}
Note that any two spaces in a same Witt tower $\mathcal V$ have the same discriminant. So we define the discriminant of a Witt tower $\disc(\mathcal V)=\disc(V_{2m})$ for any $V_{2m}$ belongs $\mathcal V$. For a fixed $d\in F^\times /F^{\times 2}$, there are two different Witt towers $\mathcal V^+=\{V^+_{2m}\}$ and $\mathcal V^-=\{V^{-}_{2m}\}$ such that $\disc(\mathcal V^+)=\disc(\mathcal V^-)=d$. More explicitly,  
\begin{align*}
\mathcal V^+&=\{V_{(d,1)}\oplus \mathbb H^{m-1}|m\geq 1\} \qquad\qquad\qquad\qquad\qquad\quad \,\mbox{if $\disc(V_{2m})=d$};\\
\mathcal V^-&=\begin{cases*}
\{D\oplus \mathbb H^{m-2}|m\geq 2\} \quad &\mbox{if $\disc(V_{2m})=1\in F^{\times 2}$};\\
\{V_{(d,c_0)}\oplus \mathbb H^{m-1}|m\geq 1\}\,\, (c_0\notin N_{E/F}(E^\times))\quad &\mbox{if $\disc(V_{2m})=d\notin F^{\times 2}$}. 
\end{cases*} 
\end{align*}
For $\sigma\in \Irr(\SP(W_{2n}))$, we have $m_{\mathcal V^+}(\sigma)$ and $m_{\mathcal V^-}(\sigma)$. 

On the other hand, there is only a single tower of symplectic spaces $\mathcal W=\{W_{2n}\}$. However, since $\pi$ is a representation of the orthogonal group $\mathrm O(V_{2m})$, we may consider its twist $\pi\otimes\det$. Thus we have two towers of theta lifts 
\begin{align*}
\Theta_{W_{2n},V_{2m},\psi}(\pi)\quad \mbox{and}\quad  \Theta_{W_{2n},V_{2m},\psi}(\pi\otimes\det)
\end{align*}
for $\pi\in \Irr(\mathrm O(V_{2m}))$. Similarly, we have $m_{\mathcal W}(\pi)$ and $m_{\mathcal W}(\pi\otimes \det)$. 

The following proposition is often referred to as the conservation relation; see \cite{MR3369906}. 
\begin{theorem}\label{con2}
	For a fixed $d\in F^\times /F^{\times 2}$ and any $\sigma\in \Irr(\SP(W_{2n}))$ (resp. $\pi\in \Irr(\mathrm O(V_{2m}))$, we have 
	\begin{align*}
	m_{\mathcal V^+}(\sigma)+m_{\mathcal V^-}(\sigma)=4n+4 ,\\
	m_{\mathcal W}(\pi)+m_{\mathcal W}(\pi\otimes \det)=4m, 
	\end{align*}
	where $\mathcal V^+$ and $\mathcal V^-$ are defined as above. 
\end{theorem}


\subsection{Collection of some results}
We collect some results from \cite{MR3166215} which will be frequently used in this paper. We emphasize that the proofs of these results are independent of the local Langlands correspondence. 



\begin{lemma}\label{temperedtotempered}
\begin{enumerate}
	\item Let $\pi\in \Irrt \left(\mathrm  O(V_{2n})\right)$.
	\begin{enumerate}[(i)]
		\item if $\Theta_{W_{2n}, V_{2n},\psi}(\pi)\neq 0$, then $\Theta_{W_{2n}, V_{2n},\psi}(\pi)=\theta_{W_{2n}, V_{2n},\psi}(\pi)$ is an irreducible tempered representation of $\SP(W_{2n})$; 
		\item if $\Theta_{W_{2n-2}, V_{2n},\psi}(\pi)\neq 0$, then $\theta_{W_{2n-2}, V_{2n},\psi}(\pi)$ is an irreducible tempered representation of $\SP(W_{2n-2})$.
	\end{enumerate}
\item Let $\sigma\in \Irrt \left(\SP(W_{2n})\right)$. 
\begin{enumerate}[(i)]
	\item if $\Theta_{V_{2n+2},W_{2n}, \psi}(\sigma)\neq 0$, then $\Theta_{V_{2n+2},W_{2n}, \psi}(\sigma)=\theta_{V_{2n+2},W_{2n}, \psi}(\sigma)$ is an irreducible tempered representation of $\mathrm O(V_{2n+2})$; 
	\item if $\Theta_{V_{2n},W_{2n}, \psi}(\sigma)\neq 0$, then $\theta_{V_{2n},W_{2n}, \psi}(\sigma)$ is an irreducible tempered representation of $\mathrm O(V_{2n})$.
\end{enumerate}
\end{enumerate}

\end{lemma}
\begin{proof}
	These follow from \cite[Proposition C.4]{MR3166215}. 
\end{proof}


\begin{lemma}\label{thetadiscrete} 
	\begin{enumerate}
		\item Let $\pi$ be an irreducible discrete series representation of $\mathrm O(V_{2n})$. 
		\begin{enumerate}[(i)]
			\item If $\sigma=\Theta_{W_{2n-2},V_{2n},\psi}(\pi)\neq 0$, then $\sigma$ is an irreducible discrete series representation of $\SP(W_{2n-2})$ and $\Theta_{W_{2n},V_{2n},\psi}(\pi)$ is an irreducible tempered  representation of $\SP(W_{2n})$ such that 
			$$
			\Theta_{W_{2n},V_{2n},\psi}(\pi)\subseteq \Ind_{Q}^{\SP(W_{2n})}\left(\chi_{V}\otimes \sigma \right),
			$$
			where $Q$ is the parabolic subgroup of $\SP(W_{2n})$ with Levi component $M_{Q}\cong \GL_1(F)\times \SP(W_{2n-2})$. 
			In this case,  $\Theta_{W_{2n},V_{2n},\psi}(\pi)=\theta_{W_{2n},V_{2n},\psi}(\pi)$ is not a discrete series representation; 
			\item If $\Theta_{W_{2n-2},V_{2n},\psi}(\pi)= 0$, then $\Theta_{W_{2n},V_{2n},\psi}(\pi)$ is either zero or an irreducible discrete series representation of $\SP(W_{2n})$. 
		\end{enumerate}
		\item Let $\sigma$ be an irreducible discrete series representation of $\SP(W_{2n})$ and $V_{2n+2}=V_{2n}\oplus \mathbb H$. 
		\begin{enumerate}[(i)]
			\item If $\pi=\Theta_{V_{2n},W_{2n},\psi}(\sigma)\neq 0$, then $\pi$ is an irreducible discrete series representation of $\mathrm O(V_{2n})$ and $\Theta_{V_{2n+2},W_{2n},\psi}(\sigma)$ is an irreducible tempered representation of $\mathrm O(V_{2n+2})$ such that 
			$$
			\Theta_{V_{2n+2},W_{2n},\psi}(\sigma)\subseteq \Ind_{P}^{\mathrm O(V_{2n+2})}\left(\mathrm{1}\otimes \pi \right),
			$$
			where $P$ is the parabolic subgroup of $\mathrm O(V_{2n+2})$ with Levi component $M_{P}\cong \GL_1(F)\times \mathrm O(V_{2n})$. In this case,  $\Theta_{V_{2n+2}^,W_{2n},\psi}(\sigma)=\theta_{V_{2n+2}^,W_{2n},\psi}(\sigma)$ is not a discrete series representation;
			\item  If $\Theta_{V_{2n},W_{2n},\psi}(\sigma)= 0$, then $\Theta_{V_{2n+2},W_{2n},\psi}(\sigma)$ is either zero or an irreducible discrete series representation of $\mathrm O(V_{2n+2})$.
		\end{enumerate}
	\end{enumerate}	
\end{lemma}
\begin{proof}
	See \cite[Corollary C.3]{MR3166215}
\end{proof}




Next we give a generalization of \cite[Lemma C.4]{MR3166215} here. Put
$$V_{2n}=X_k\oplus V_{2n_0}\oplus X_k^{\vee}, \quad W_{2n}=Y_k\oplus W_{2n_0}\oplus Y_k^{\vee}$$ 
as in section \ref{sectionparabolic}. Let $P=P_{k}=M_{P} U_{P}$ and $Q=Q_{k}=M_{Q} U_{Q}$ be the parabolic subgroups defined in section \ref{sectionparabolic} such that 
$$M_{P} \cong \GL(X_k) \times \mathrm O(V_{2n_0}), \quad M_{Q} \cong \GL(Y_k) \times \SP(W_{2n_0}).$$
Then we have 
\begin{lemma}\label{12}
	Let $\pi$ be a irreducible constituent of $\Ind_{P}^{\mathrm O(V_{2n})}(\tau\otimes \pi_0)$, where $\tau$ is an irreducible discrete series representation of $\GL(X_k)$ and $\pi_0$ is an irreducible tempered representation of $\mathrm O(V_{2n_0})$. Then we have 
	$$
	\theta_{W_{2n},V_{2n},\psi}(\pi) \subseteq \Ind_{Q}^{\SP(W_{2n})}(\tau\chi_{V}\otimes \theta_{W_{2n_0},V_{2n_0},\psi}(\pi_0)).
	$$
 Hence $\theta_{W_{2n},V_{2n},\psi}(\pi)$ is either zero or an irreducible tempered representation of $\SP(W_{2n})$. In particular, $\theta_{W_{2n},V_{2n},\psi}(\pi)$ is zero if $\theta_{W_{2n_0},V_{2n_0},\psi}(\pi_0))$ is zero. 
\end{lemma}
\begin{proof}
	In \cite{MR3166215}, they prove this when $\pi_0$ is a discrete series representation. Their proof can be easily extended to our case. We recall it here. 
	
	We denote $\omega$ and $\omega_{00}$ to be the Weil representations for $\mathrm O(V_{2n})\times \SP(W_{2n})$ and $\mathrm O(V_{2n_0})\times \SP(W_{2n_0})$. Since $\pi\subseteq \Ind_{P}^{\mathrm O(V_{2n})}(\tau\otimes \pi_0)$, we have 
	\begin{align*}
	\Theta_{W_{2n},V_{2n},\psi}(\pi)^\vee &\cong \Hom_{\mathrm O(V_{2n})}(\omega, \pi)\\
	&\hookrightarrow \Hom_{\mathrm O(V_{2n})}(\omega, \Ind_{P}^{\mathrm O(V_{2n})}(\tau\otimes \pi_0))\\
	& \hookrightarrow \Hom_{\GL(X_k)\times \mathrm O(V_{2n_0})}(R_{P}(\omega), \tau\otimes \pi_0),
	\end{align*} 
	where $R_{P}(\omega)$ is the normalized Jacquet module of $\omega$ with respect to the parabolic $P$ of $\mathrm O(V_{2n})$. The normalized Jacquet module has been computed by Kudla \cite{MR818351}. More precisely, there is a $M_P\times \SP(W_{2n})$-invariant filtration 
	\begin{align*}
	R_{P}(\omega)= R^0 \supset R^1\supset \cdots \supset R^t \supset R^{k+1}
	\end{align*}
	with successive quotients $J^a\coloneqq R^a/R^{a+1}$ ($0\leq a\leq k$). The reader can consult \cite{MR818351} and \cite[Lemma 5.1]{MR3714507} for an explicit formula for $J_a$ ($0\leq a\leq k$) and here we only recall the formula for $J_k$:  
	$$J^k\cong \Ind_{\GL(X_k)\times \mathrm O(V_{2n_0})\times Q}^{\GL(X_k)\times \mathrm O(V_{2n_0})\times \SP(W_{2n})}(\mathscr S(\mathrm{Isom}(Y_k,X_k))\otimes \omega_{00}),$$
	where 
	\begin{itemize}
		\item $\mathrm{Isom}(Y_k,X_k)$ is the set of invertible linear maps from $Y_k$ to $X_k$ and $\mathscr S(\mathrm{Isom}(Y_k,X_k))$ is the space of locally constant, compactly supported functions on $\mathrm{Isom}(Y_k,X_k)$;
		\item the action of $\GL(X_k)\times \mathrm O(V_{2n_0})\times \GL(Y_k)\times \SP(W_{2n_0})$ on $\mathscr S(\mathrm{Isom}(Y_k,X_k))\otimes \omega_{00}$ is given by 
		\begin{itemize}
			\item $\GL(X_k)\times \GL(Y_k)$ acts on $\mathscr S(\mathrm{Isom}(Y_k,X_k))$ by 
			\begin{align*}
			\left((g,h)\cdot f\right)(x)=\chi_{V}(h)\cdot f(g^{-1}\circ x\circ h)
			\end{align*} 
			for $(g,h)\in \GL(X_k)\times \GL(Y_k)$, $f\in \mathscr S(\mathrm{Isom}(Y_k,X_k))$ and $x\in \mathrm{Isom}(Y_k,X_k)$. 
			\item $\mathrm O(V_{2n_0})\times \SP(W_{2n_0})$ acts by the Weil representation $\omega_{00}$. 
		\end{itemize}
	\end{itemize}
	By the same argument in \cite[Lemma C.4]{MR3166215}, we deduce 
	\begin{align*}
	\Hom_{\GL(X_k)\times \mathrm O(V_{2n_0})}(J^a, \tau\otimes \pi_0)=0\quad \mbox{for $0\leq a<k$}. 
	\end{align*}
	This implies 
	\begin{align*}
	\Theta_{W_{2n},V_{2n},\psi}(\pi)^\vee & \hookrightarrow \Hom_{\GL(X_k)\times \mathrm O(V_{2n_0})}(R_{P}(\omega), \tau\otimes \pi_0)\\
	& \hookrightarrow \Hom_{\GL(X_k)\times \mathrm O(V_{2n_0})}(J^k, \tau\otimes \pi_0)\\
	&\cong \left(\Ind_{Q}^{\SP(W_{2n})} (\tau^\vee\chi_{V}\otimes \Theta_{V_{2n_0},W_{2n_0},\psi}(\pi_{0}))\right)^\vee. 
	\end{align*}
	Take the contragredient functor, we get an epimorphism 
	\begin{align*}
	\Ind_{Q}^{\SP(W_{2n})} (\tau^\vee\chi_{V}\otimes \Theta_{V_{2n_0},W_{2n_0},\psi}(\pi_{0}))\twoheadrightarrow \Theta_{V_{2n},W_{2n},\psi}(\pi). 
	\end{align*} 
	It follows from Lemma \ref{temperedtotempered} that $$\Theta_{W_{2n},V_{2n},\psi}(\pi)=\theta_{W_{2n},V_{2n},\psi}(\pi)\quad\mbox{and}\quad    \Theta_{W_{2n_0},V_{2n_0},\psi}(\pi_0)=\theta_{W_{2n_0},V_{2n_0},\psi}(\pi_0).$$ 
	Apply both the contragredient funtor and the MVW functor \cite[Lemma 2.2]{MR3714507}, we get 
	\begin{align*}
	\theta_{W_{2n},V_{2n},\psi}(\pi)\subseteq \Ind_{Q}^{\SP(W_{2n})} (\tau\chi_{V}\otimes \theta_{W_{2n_0}, V_{2n_0},\psi}(\pi_0)). 
	\end{align*}
	This finishes the proof. 
\end{proof}


For $\pi\in \Irr(\mathrm O(V_{2n})$, $\sigma\in \Irr(\SP(W_{2n}))$, and $\chi$ be a character of $F^\times$, let $\gamma(s,\pi,\chi,\psi)$ and $\gamma(s,\sigma,\chi,\psi)$ be the standard $\gamma$-factor defined by Lapid-Rallis \cite{MR2192828} using doubling method; see also \cite[\S 10,\S 11]{MR3166215} for its properties. The following Lemma describes how the standard $\gamma$-factors behave under the theta correspondence.
\begin{lemma}\label{comgamma}
	Let $V_{2n}$ be a $2n$-dimensional orthogonal space and $\chi_V$ be the discriminant character of $V_{2n}$. Let $\pi \in \Irrt \mathrm O(V_{2n})$. 
	\begin{enumerate}[(i)]
		\item If $\sigma= \theta_{W_{2n}, V_{2n},  \psi}(\pi)\neq 0$, then 	
		$$
		\frac{\gamma(s,\sigma,\chi,\psi)}{\gamma(s,\pi,\chi\chi_V,\psi)}= \gamma(s,\chi\chi_V,\psi).
		$$
		\item If $\sigma= \theta_{W_{2n-2}, V_{2n},  ,\psi}(\pi)\neq 0$, then 
		$$
		\frac{\gamma(s,\pi,\chi\chi_V,\psi)}{\gamma(s,\sigma,\chi,\psi)}= \gamma(s,\chi\chi_V,\psi).
		$$
	\end{enumerate}
\end{lemma}
\begin{proof}
See \cite[Theorem 11.5]{MR3166215}. 
\end{proof}


\begin{lemma}\label{pole}
	Let $V_{2n}$ be a $2n$-dimensional orthogonal space and $\chi_V$ be the discriminant character of $V_{2n}$. For any $\sigma\in \Irr(\SP(W_{2n}))$, if  
	$$	 
	\Theta_{V_{2n},W_{2n},\psi}(\sigma) \neq 0 
	$$
	then $\gamma(s,\sigma,\chi_{V},\psi)$ has a pole at $s=1$
\end{lemma}
\begin{proof}
	See \cite[Proposition 11.2]{MR3166215}. 
\end{proof}

Let $\pi\in \Irr(\mathrm O(V_{2n}))$ (resp. $\sigma\in \Irr(\SP(W_{2n}))$) and $\tau\in \Irr(\GL_{k}(F))$, the Plancherel measure $\mu_{\psi}(\tau\otimes\pi)$ (resp. $\mu_{\psi}(\tau\otimes\sigma)$) is defined in Appendix \ref{plancherelmeasure}. The following Lemma describes how the Plancherel measures behaves under the theta correspondence.
\begin{lemma}\label{complan}
	\begin{enumerate}[(i)]
		\item  Let $\pi$ and $\sigma$ be irreducible smooth representations of $\mathrm O(V_{2n})$ and $\SP(W_{2n})$ respectively, such that $\sigma=\theta_{W_{2n},V_{2n},\psi}(\pi)$. Let $\tau$ be an irreducible smooth representations of $\GL_{k}(F)$ and put $\tau_s=\tau|\cdot|_{F}^{s}$ for $s\in \mathbb{C}$. Then we have:
		$$
		\frac{\mu_{\psi}(\tau_s\chi_{V}\otimes\sigma)}{\mu_{\psi}(\tau_s\otimes\pi)}=\gamma(s,\tau,\psi).\gamma(-s,\tau^{\vee},\psi_{-1}).
		$$ 
		\item 
		Let $\pi$ and $\sigma$ be irreducible smooth representations of $\mathrm O(V_{2n})$ and $\SP(W_{2n-2})$ respectively, such that $\sigma=\theta_{W_{2n-2},V_{2n},\psi}(\pi)$. Let $\tau$ be an irreducible smooth representations of $\GL_{k}(F)$ and put $\tau_s=\tau|\cdot|_{F}^{s}$ for $s\in \mathbb{C}$. Then we have:
		$$
		\frac{\mu_{\psi}(\tau_s\otimes\pi)}{\mu_{\psi}(\tau_s\chi_{V}\otimes\sigma)}=\gamma(s,\tau,\psi).\gamma(-s,\tau^{\vee},\psi_{-1}).
		$$ 
	\end{enumerate}
\end{lemma}
\begin{proof}
	See \cite[Theorem 12.1]{MR3166215}. 
\end{proof}






\section{Local Langlands correspondence}
In this section, we state the local Langlands correspondence for symplectic groups and even orthogonal groups. 
\subsection{$L$-parameters}\label{Lparmeter}
Let $W_F$ be the Weil group of $F$ and $WD_F=W_F\times \SL_2(\mathbb C)$ be the Weil-Deligne group of $F$. 

We say that a homomorphism
$\phi : WD_{F} \rightarrow \GL_n(\mathbb C)$ is a representation of $WD_F$ if
\begin{itemize}
	\item $\phi\left(\mathrm{Frob}_{F}\right)$ is semi-simple, where $
	\mathrm{Frob}_{F}$ is a geometric Frobenius element in
	$W_{F}$;
	\item the restriction of $\phi$ to $W_{F}$ is smooth;
	\item  the restriction of $\phi$ to $\mathrm{SL}_{2}(\mathbb{C})$ is algebraic.
\end{itemize}
We call $\phi$ tempered if the image of $W_{F}$ is bounded.

We say that $\phi$ is orthogonal if there exists a non-degenerate
bilinear form $B : \mathbb C^n \times \mathbb C^n \rightarrow \mathbb{C}$ such that
$$\left\{\begin{array}{l}{B(\phi(w) x, \phi(w) y)=B(x, y)}, \\ {B(y, x)= B(x, y)}\end{array}\right.$$
for $x, y \in \mathbb C^n$ and $w \in WD_{F},$. In this case, $\phi$ is self-dual, i.e., $\phi$ is equivalent to its contragredient $\phi^{\vee}$. 
%More precisely, see \cite{MR3202556}[\S 3]

Suppose that $\phi$ is orthogonal. We may decompose
\begin{align}\label{111}
\phi=m_{1} \phi_{1}+\cdots+m_{l} \phi_{l}+\varphi+\varphi^{\vee},
\end{align}
where $\phi_{1}, \ldots, \phi_{l}$ are pairwise distinct irreducible orthogonal representations of $WD_{F}$ and $\varphi$ is a sum of irreducible representations of $WD_{F}$ which are not orthogonal. We say that a representation $\phi$ is discrete if $m_{i}=1$ for any $i=1, \ldots, l$ and $\varphi=0$.


By \cite[\S 8]{MR3202556} and \cite[\S 3]{MR3708200}, an $L$-parameter for $\mathrm O(V_{2n})$ (resp. $\SP(W_{2n})$) is an $2n$-dimensional (resp. $(2n+1)$-dimensional) orthogonal representation $\phi$ of $\WD_F$ with 
$\det \phi=\chi_{V}$ (resp. $\det \phi=1$). For $G=\mathrm O(V_{2n})$ or  $\SP(W_{2n})$, let $\Phi(G)$ be the equivalent classes of $L$-parameter for $G$. More precisely: 
\begin{align*}
\Phi(\mathrm O(V_{2n}))&=\{\phi:\WD_{F} \rightarrow \mathrm{O}(2n,\mathbb{C}) |\det(\phi)=\chi_{V}\} /(\mathrm{O}(2n, \mathbb{C})\mbox{-conjugacy}); \\ 
\Phi(\SP(W_{2n})))&=\{\phi : \WD_{F} \rightarrow \mathrm{SO}(2 n+1, \mathbb{C})\} /(\mathrm{SO}(2 n+1, \mathbb{C})\mbox{-conjugacy}). 
\end{align*}
We also denote the subset of $\Phi(G)$ consisting of tempered (resp. discrete)
representations by $\Phi_{\mathrm{temp}}(G)$ (resp. $\Phi_{\mathrm{disc}}(G)$). Then we have a sequence 
\begin{align*}
\Phi_{\disc}(G)\subseteq \Phi_{\mathrm{ temp }}(G)\subseteq \Phi(G).
\end{align*}
When $G=\mathrm O(V_{2n})$, we define $\Phi^{\epsilon}(G)$ to be the subset of $\Phi(G)$ consisting of $\phi$ which contains an irreducible orthogonal representation of $WD_F$ of odd dimension. 
%We put $\Phi^{\epsilon}_{*}(G)=\Phi^\epsilon(G)\cap \Phi_{*}(G)$ for $*\in \{\disc, \mathrm{ temp }\}$. 


Let $\phi\in \Phi(G)$ for $G=\mathrm O(V_{2n})$ or $\SP(W_{2n})$. We denote the space of $\phi$ by $M$. Let $\Aut(\phi,B)$ be the group of elements in $\GL(M)$ which centralize the image of $\phi$ and preserve $B$. Also put $\Aut(\phi,B)^+= \Aut(\phi,B)\cap \SL(M)$. We define the component groups $\mathcal {S}_{\phi}$ and $\mathcal {S}_{\phi}^+$ of $\phi$ by 
$$
\mathcal {S}_{\phi}=\Aut(\phi,B)/\Aut(\phi,B)^{\circ}\quad \mbox{and}\quad \mathcal {S}_{\phi}^+=\Aut(\phi,B)^+/\left(\Aut(\phi,B)^{\circ}\cap \Aut(\phi,B)^{+}\right),
$$
where $\Aut(\phi,B)^{\circ}$ is the identity component of $\Aut(\phi,B)$. More explicitly, write $\phi=m_{1} \phi_{1}\oplus\cdots+m_{l} \phi_{l}\oplus \varphi\oplus \varphi^{\vee}$ as \ref{111}, then 
$$\mathcal {S}_{\phi}\cong \bigoplus_{i=1}^{l}(\mathbb{Z} / 2 \mathbb{Z}) a_{i} \cong (\mathbb{Z} / 2 \mathbb{Z})^l,$$
where $a_i$ corresponds to $\phi_i\subseteq \phi$. The determinant map $\det: \GL(M) \rightarrow \mathbb{C}^{ \times}$ induces a homomorphism
\begin{equation}\label{detmap}
\begin{aligned}
\det : \mathcal {S}_{\phi} \rightarrow \mathbb{Z} / 2 \mathbb{Z},  \quad \sum_{i=1}^{l} \varepsilon_{i} a_{i} \mapsto \sum_{i=1}^{l} \varepsilon_{i} \cdot \dim\left(\phi_{i}\right),
\end{aligned}
\end{equation}
where $\varepsilon_{i} \in\{0,1\}=\mathbb Z/2\mathbb Z$ and $\mathcal {S}_{\phi}^+=\ker(\det)$. Put
$$z_{\phi} \coloneqq\sum_{i=1}^{l} m_i \cdot a_{i} \in \mathcal {S}_{\phi},$$ 
which is the image of $-\mathbf 1$ in $\mathcal S_\phi$. We call it the central element in $\mathcal {S}_{\phi}$. Also put $\bar {\mathcal {S}}_{\phi}=\mathcal {S}_{\phi}/\langle z_{\phi}\rangle$. Note that when $\phi\in \Phi(\SP(W))$, we have $\det(z_\phi)=-1$, so $z_\phi\notin \mathcal {S}_{\phi}^+$ and 
\begin{align*}
\mathcal S_\phi =\mathcal S_\phi^+\oplus (\mathbb Z/2\mathbb Z)z_\phi,\quad \mathcal S_\phi^+\cong \bar {\mathcal S}_\phi.
\end{align*}
In this case, we have  
\begin{align}\label{componentgroupsmall}
\widehat{\mathcal S^+_\phi}\cong \widehat{\bar {\mathcal {S}}_{\phi}}, 
\end{align}
where we denote by $\widehat{A}$ the pontryagin dual of an abelian group $A$. For $a=a_{i_{1}}+\cdots+a_{i_{k}} \in \mathcal {S}_{\phi}$ with $1 \leq i_{1}<\cdots<i_{k} \leq l$, we put
$$\phi^{a}=\phi_{i_{1}} \oplus \cdots \oplus \phi_{i_{k}}.$$
By \cite[\S 4]{MR3202556}, for each $c \in F^{ \times}$,we define a character $\eta_{\phi,c}$ of $\mathcal {S}_{\phi}$ by
\begin{align}\label{eta}
\eta_{\phi,c}(a)=\det(\phi^{a})(c).
\end{align}
Note that $\eta_{\phi, c}(z_{\phi})=1$ when $\phi\in \Phi(\SP(W))$, hence $\eta_{\phi,c}\in \widehat{\bar{\mathcal S}_\phi}$ in this case. Let \begin{align*}
\upsilon: \mathbb Z/2\mathbb Z \hookrightarrow \mathbb C^{\times}
\end{align*} 
be the natural embedding of $\mathbb Z/2\mathbb Z$ to $\mathbb C^\times$. We define a character $\kappa_{\phi}\in \widehat {\mathcal {S}_{\phi}}$ by composing the map $\det$ in \ref{detmap} with $\upsilon$, i.e. 
\begin{align}\label{kappa}
\kappa_{\phi}(a)=\upsilon\circ \det(a)\quad \mbox{for}\,\,a\in \mathcal S_{\phi}. 
\end{align}
 

For a representation $\phi$ of $\WD_F$, we define $L(s,\phi)$, $\varepsilon(s,\phi,\psi)$ and $\gamma(s, \phi, \psi)$ as in \cite{MR546607}. Note that the $\varepsilon$-factor and $\gamma$-factor depend on the choice of the additive character $\psi$ while the $L$-factor does not. 
% If $(\phi, M)$ is an orthogonal (resp. symplectic) representation with a $W D_{F}$ -invariant
%symmetric (resp. alternate) bilinear form $B,$ then we define the adjoint $L$ -function $L(s, \phi, \text { Ad })$ associated to $\phi$ to be the $L$ -function associated to 
% $$\mathrm{Ad}\circ \phi : W D_{F} \rightarrow \operatorname{GL}(\operatorname{Lie}(\operatorname{Aut}(M, B)))$$
% We say that $\phi$ is generic if $L(s, \phi, \mathrm{Ad})$ is regular at $s=1 .$ since $B$ is symmetric (resp. alternate), we have $\mathrm{Ad}\circ \phi \cong \wedge^{2} \phi\left(\text {resp} \mathrm{Ad} \circ \phi \cong \operatorname{Sym}^{2} \phi\right)$. Hence the adjoint L-function $L(s, \phi, \mathrm{Ad})$ is equal to the exterior square $L$ -function $L\left(s, \phi, \wedge^{2}\right)=L\left(s, \wedge^{2} \phi\right)$ (resp. the symmetric square $L$ -function $L\left(s, \phi, \mathrm{Sym}^{2}\right)=L\left(s, \operatorname{Sym}^{2} \phi\right) )$ 

The following lemma in \cite[Lemma 12.3]{MR2999299} and \cite[Lemma A.6]{MR3573972} will be used later.  
\begin{lemma}\label{gammadetermine}
	Let $\phi_{1}$ and $\phi_{2}$ be two tempered orthogonal representations of $\WD_F$ of the same dimension. Assume that 
	$$\gamma\left(s, \phi_{1} \otimes \phi_{\rho}, \psi\right) \cdot \gamma\left(-s, \phi_{1} \otimes \phi_{\rho}^{\vee}, \psi_{-1}\right)=\gamma\left(s, \phi_{2} \otimes \phi_{\rho}, \psi\right) \cdot \gamma\left(-s, \phi_{2} \otimes \phi_{\rho}^{\vee}, \psi_{-1}\right)$$
	for every irreducible representation $\phi_{\rho}$ of $\WD_F$. Then
	$$\phi_{1} \cong \phi_{2}$$
	as representations of $\WD_F$. 
\end{lemma} 
\begin{comment}
\begin{proof}
	The proof is similar to \cite[Lemma 12.3]{MR2999299}. Note that the Lemma 12.3 there is for discrete paramters, but the proof works for tempered parameter as well. 
\end{proof} 
\end{comment}


 


% In the case, when $G=\SP_{2n}$ and $\phi\in \Phi(G)$, we see that $z_\phi\notin \mathcal {S}_{\phi}^{+}$ and $\mathcal {S}_{\phi}^{+}\cong \mathcal {S}_{\phi}/\langle z_\phi\rangle$. Hence we may identify $\hat \mathcal {S}_{\phi}^+$ with $\hat{\bar S}_{\phi}$. 

%For a representation $\phi$ of $\WD_F,$ the $L$-factor, $\varepsilon$-factor and $\gamma$-factor  associated to $\phi$
%which are defined in \cite{MR546607} are denoted by $L(s, \phi), \varepsilon(s,\phi,\psi)$ and $\gamma(s, \phi, \psi)$. Note that, the $\varepsilon$-factor and $\gamma$-factor depends on the choice of the additive character $\psi$ while the $L$-factor does not. 
% If $(\phi, M)$ is an orthogonal (resp. symplectic) representation with a $W D_{F}$ -invariant
%symmetric (resp. alternate) bilinear form $B,$ then we define the adjoint $L$ -function $L(s, \phi, \text { Ad })$ associated to $\phi$ to be the $L$ -function associated to 
% $$\mathrm{Ad}\circ \phi : W D_{F} \rightarrow \operatorname{GL}(\operatorname{Lie}(\operatorname{Aut}(M, B)))$$
% We say that $\phi$ is generic if $L(s, \phi, \mathrm{Ad})$ is regular at $s=1 .$ since $B$ is symmetric (resp. alternate), we have $\mathrm{Ad}\circ \phi \cong \wedge^{2} \phi\left(\text {resp} \mathrm{Ad} \circ \phi \cong \operatorname{Sym}^{2} \phi\right)$. Hence the adjoint L-function $L(s, \phi, \mathrm{Ad})$ is equal to the exterior square $L$ -function $L\left(s, \phi, \wedge^{2}\right)=L\left(s, \wedge^{2} \phi\right)$ (resp. the symmetric square $L$ -function $L\left(s, \phi, \mathrm{Sym}^{2}\right)=L\left(s, \operatorname{Sym}^{2} \phi\right) )$ 

\subsection{Whittaker data}\label{whittaker}
To describe the local Langlands correspondence, it is necessary to choose a Whittaker datum. Let $G$ be a quasi-split reductive group over $F$. A Whittaker datum of $G$ is a conjugacy class of a pair $\mathfrak W=(B,\mu)$, where 
\begin{itemize}
	\item $B=TU$ is a rational Borel subgroup of $G$. 
	\item $\mu$ is a generic character of $U(F)$.
\end{itemize}
\begin{comment}
Note that when $V_{2n}$ is associated to $(d,c)$, $\mathrm O(V_{2n})$ is quasi-spit but not connected, in this case, we define the Whittaker datum of $\mathrm O(V_{2n})$ to be the Whittaker datum of $\SO(V_{2n})$. 
\end{comment}

Next we describe the Whittaker datum for $\mathrm O(V_{2n})$ and $\SP(W_{2n})$ explicitly. Let $V_{2n}$ be the orthogonal space associate to $(d,c)$. Write $V_{2n}\cong V_{(d,c)}\oplus \mathbb H^{n-1}$ as in \ref{127}. Let $B=TU$ be the $F$-rational Borel subgroup of $\mathrm O (V_{2n})$ stabilizing the complete flag
$$0 \subset\left\langle v_{1}\right\rangle \subset\left\langle v_{1}, v_{2}\right\rangle \subset \cdots \subset\left\langle v_{1}, \ldots, v_{n-1}\right\rangle= X_{n-1},$$
where $T$ is the $F$-rational torus in $B$ stabilizing the lines $F v_{i}$ for $i=1, \ldots, n-1$. Define a generic character $\mu_{c}$ of $U$ by
$$\mu_{c}(u)=\psi\left(\left\langle u v_{2}, v_{1}^{*}\right\rangle_{V}+\cdots+\left\langle u v_{n-1}, v_{n-2}^{*}\right\rangle_{V}+\left\langle u e, v_{n-1}^{*}\right\rangle_{V}\right).$$
Note that $V_{2n}$ is also associated to $(d,c^\prime)$ iff $c^\prime \in c N_{E/F}(E^{\times})/F^{\times 2}$. By a similar argument in \cite[\S 12]{MR3202556}, the map $c \mapsto \mu_{c}$ gives a bijection (not depending on $\psi$)
$$c N_{E/F}(E^{\times})/F^{\times 2} \longleftrightarrow 
\{\mbox{$T$-orbits of generic characters of $U$}\},$$
where $E=F(\sqrt{d})$; see also \cite[\S 2.2]{MR3708200} for a detailed description. We define a Whittaker datum of $\mathrm O(V_{2n})$ by $\mathfrak W_{c}=(B,\mu_c)$. Note that $\mathfrak W_{c}$ does not depend on the choice of $\psi$. We then consider two cases separately:  
\begin{itemize}
	\item If $\disc(V)=d\in F^{\times 2}$, then $E=F$ and $\mathrm O(V_{2n}^+)$ is quasi-split while $\mathrm O(V_{2n}^-)$ is not. The set of Whittaker data for $\mathrm O(V_{2n}^+)$ is parameterized by $F^\times/F^{\times 2}$.  
	\item If $\disc(V)=d\notin F^{\times}$, both $\mathrm O(V_{2n}^+)$ and $\mathrm O(V_{2n}^-)$ are quasi-split. The set of Whittaker data for $\mathrm O(V_{2n}^+)$ is parameterized by $N_{E/F}(E^\times)/F^{\times 2}$ and the set of Whittaker data for $\mathrm O(V_{2n}^-)$ is parameterized by $c_0N_{E/F}(E^\times)/F^{\times 2}$ for a $c_0\notin N_{E/F}(E^\times)$. 
\end{itemize}
In both case, we have a bijection 
\begin{align*}
\{F^{\times} /F^{\times 2}\}&\longleftrightarrow  \bigsqcup_{V_{2n}^{\bullet}} \{\mbox{Whittaker data of $\mathrm O(V_{2n}^{\bullet})$}\}\\
c &\mapsto \mathfrak W_{c}= (B,\mu_c) ,
\end{align*}
where $V_{2n}^{\bullet}$ runs over all orthogonal space of dimension $2n$ and discriminant $d\in F^\times /(F^{\times 2})$. 

Next we define the notion of generic representation for $\mathrm O(V_{2n})$. We identify $\mathrm{O}\left(V_{(d, c)}\right)$ as the subgroup of $\mathrm{O}\left(V_{2 n}\right)$ which fixes $\mathbb{H}^{n-1} $. Via the canonical embedding $\mathrm{O}\left(V_{(d, c)}\right) \hookrightarrow \mathrm{O}\left(V_{2 n}\right),$ we regard $\epsilon$ as an element in $\mathrm{O}\left(V_{2 n}\right)$. Note that $\epsilon$ normalizes $U$ and fixes $\mu_{c}$, we can extend $\mu_{c}$ to $\widetilde{U}=U \rtimes\langle\epsilon\rangle$. There are exactly two such extensions $\mu_{c}^{\pm} : \widetilde{U} \rightarrow \mathbb{C}^{*}$ which are determined by
 $$\mu_{c}^{\pm}\left(\epsilon\right)=\pm 1.$$
 As in \cite[\S 2.2]{MR3708200}, we say that $\pi\in \Irr(\mathrm{O}(V_{2n}))$ is $\mathfrak W_c^{\pm}$-generic if
 $$\Hom_{\widetilde{U}}(\pi, \mu_{c}^{\pm}) \neq 0.$$
 It is easy to check that $\pi$ is $\mathfrak W_c^+$ generic iff $\pi\otimes\det$ is $\mathfrak W_c^-$-generic. 


Similarly, let $W_{2n}$ be a symplectic space of dimension $2n$. Write $W_{2n}=\mathbb H^n$ as in \ref{128}. Let $B^{\prime}=T^{\prime} U^{\prime}$ be the $F$-rational Borel subgroup of $\SP(W_{2n})$ stabilizing the complete flag
$$0 \subset\left\langle w_{1}\right\rangle \subset\left\langle w_{1}, w_{2}\right\rangle \subset \cdots \subset\left\langle w_{1}, \ldots, w_{n}\right\rangle= Y_{n},$$
where $T^{\prime}$ is the $F$-split torus stabilizing the lines $F w_{i}$ for $i=1, \ldots, n $. For $c \in F^{\times}$, we define a generic
character $\mu_{c}^{\prime}$ of $U^{\prime}$ by
$$\mu_{c}^{\prime}(u)=\psi\left(\left\langle uw_{2}, w_{1}^{*}\right\rangle_{W}+\cdots+\left\langle u w_{n}, w_{n-1}^{*}\right\rangle_{W}+c\left\langle u w_{n}^{*}, w_{n}^{*}\right\rangle_{W}\right).$$
By \cite{MR3202556} $[\S 12]$ the map $c \mapsto \mu_{c}^{\prime}$ gives a bijection (depending on $\psi )$
$$F^{\times}/F^{\times 2} \longleftrightarrow \{\mbox{$T^{\prime}$-orbits of generic characters of $U^{\prime}$}\}.$$
We define a Whittaker datum of $\SP(W_{2n})$ by $\mathfrak W^\prime_{\psi,c}=(B^\prime,\mu^\prime_c)$. Then we have a bijection 
\begin{align*}
\{F^{\times} /F^{\times 2}\}&\longleftrightarrow  \{\mbox{Whittaker data of $\SP(W_{2n})$}\}\\
c &\mapsto \mathfrak W^\prime_{\psi,c}= (B^\prime,\mu^\prime_c).
\end{align*}
Note that $\mathfrak W^\prime_{\psi,c}$ does depend on the choice of $\psi$. More precisely, for any $a\in F^\times $, let $\psi_a$ be the character $\psi_a(x)=\psi(ax)$, then it is easy to check that 
\begin{align}\label{121}
\mathfrak W^{\prime}_{(\psi_a,c)}= \mathfrak W^{\prime}_{(\psi,ac)}. 
\end{align}
We say that $\sigma\in \Irr(\SP(W_{2n}))$ is $\mathfrak W_{\psi,c}^\prime$-generic if 
\begin{align*}
\Hom_{U^\prime}(\sigma, \mu^\prime_c)\neq 0.
\end{align*}

\subsection{LLC for symplectic groups}
We first describe the local Langlands correspondence for symplectic groups. This was proved by Arthur\cite{MR3135650}, with supplements by many others. 
 
\begin{theorem}\label{llcsympletic}
Let $W_{2n}$ be a $2n$-dimensional symplectic space.  
	\begin{enumerate}[(1).]
		\item There exists a surjection
		$$
		\mathcal L:  \Irr \left(\SP(W_{2n})\right) \longrightarrow \Phi(\SP(W_{2n})),$$
		which is a finite-to-one map. For any $\phi \in \Phi(\SP(W_{2n}))$, we denote $\mathcal L^{-1}(\phi)$ by $\Pi_{\phi}$ and call it the $L$-packet of $\phi$. 
		\item For each Whittaker datum $\mathfrak W^\prime_{\psi,c}$ of $\SP(W_{2n})$, there is a canonical map
		\begin{align}
		\mathcal J_{\mathfrak W^\prime_{\psi,c}} : \Pi_{\phi} \longrightarrow \widehat{\bar {\mathcal {S}}_{\phi}}. 
		\end{align}
		We write $\sigma=\sigma_{\mathfrak W^\prime_{\psi,c}}(\phi,\eta)$ if $\sigma \in \Pi_{\phi}$ corresponding to $\eta \in \widehat{\bar {\mathcal S}_{\phi}}$ under $\mathcal J_{\mathfrak W^\prime_{\psi,c}}$. 
		\item Assume that $\phi$ is a tempered $L$-parameter for $\SP(W_{2n})$, then for each Whittaker datum $\mathfrak W^\prime_{\psi,c}$ of $\SP(W_{2n})$, there is an unique $\mathfrak W^\prime_{\psi,c}$-generic representation $\sigma$ in $\Pi_{\phi}$, which corresponds to the trivial character under $\mathcal J_{\mathfrak W^\prime_{\psi,c}}$. 
		\item 
		For $\sigma\in \Pi_{\phi}$ and $\mathfrak W^\prime_{\psi,c_1},\mathfrak W^\prime_{\psi,c_2}$ be two Whittaker data of $\SP(W_{2n})$, we have 
		$$
		\mathcal J_{\mathfrak W^\prime_{\psi,c_2}}(\sigma)=\mathcal J_{\mathfrak W^\prime_{\psi,c_1}}(\sigma)\otimes \eta_{\phi,c_2/c_1},
		$$
		where $\eta_{\phi,c_2/c_1}$ is defined in \ref{eta}. 
		\item
		$\sigma\in \Pi_{\phi}$ is a tempered representation iff $\phi$ is a tempered parameter and $\sigma\in \Pi_{\phi}$ is a discrete series representation iff $\phi$ is a discrete parameter.
		\item
		\textbf{(Local intertwining relation)}
		Assume that $\phi=\phi_{\tau}+\phi_{0}+\phi_{\tau}^{\vee}$, where $\phi_{0}$ is an element in $\Phi_{\mathrm {temp}}\left(\SP(W_{2n_0})\right)$
		and $\phi_{\tau}$ is an irreducible tempered representation of $\WD_{F}$ corresponding to $\tau \in \Irr(\GL_{k}(F))$, so there is a natural embedding $\mathcal S_{\phi_{0}}\hookrightarrow \mathcal {S}_{\phi}$. Let $Q$ be a parabolic subgroup of $\SP(W_{2n})$ with Levi subgroup $$M_{Q}\cong \GL_{k}(F) \times \SP(W_{2n_0})$$ and $\sigma_{0}$ be the irreducible tempered representation of $\SP(W_{2n_0})$ corresponding to $(\phi_{0},\eta_0)$ under $\mathcal L$ and $\mathcal J_{\mathfrak W^\prime_{\psi,1}}$. Then the induced representation $\Ind_{Q}^{\SP(W_{2n})}\left(\tau \otimes \sigma_{0}\right)$ has a decomposition 
		$$
		\Ind_{Q}^{\SP(W_{2n})}(\tau \otimes \sigma_{0})=\bigoplus_{\eta}\sigma_{\mathfrak W^\prime_{\psi,1}}(\phi,\eta),
		$$
		where the sum runs over all $\widehat{\bar{ \mathcal S}_{\phi}}$ such that $\eta|_{ \bar{\mathcal S}_{\phi_0}}=\eta_0$. Moreover if $\phi_{\tau}$ is self-dual and of orthogonal type. Let 
		$$
		R_{\mathfrak W^\prime_{\psi,1}}(w, \tau \otimes \sigma_0)
		\in \End_{\SP(W_{2n})}\left(
		\Ind_{Q}^{\SP(W_{2n})}(\tau \otimes \sigma_{0})\right)$$ be the normalized intertwining operator associated to the Whittaker datum $\mathfrak W^\prime_{\psi,1}$ (see section \ref{normalizingintertwing}), where $w$ is the unique non-trivial element in the relative Wely group for $M_Q$. Then
		$$R_{\mathfrak W^\prime_{\psi,1}}(w, \tau \otimes \sigma_0) | _{\sigma}=\mathcal J_{\mathfrak W^\prime_{\psi,1}}(\sigma)(a),$$
		where $a \in \mathcal {S}_{\phi}$ corresponding to $\phi_{\tau}$.
		\item \textbf{(Compatibility with Langlands quotients)}
		Assume that
		$$\phi=\phi_1|\cdot|_{F}^{s_{1}}\oplus \cdots\oplus \phi_{r}|\cdot|_{F}^{s_{r}}+\phi_{0}\oplus \phi_{r}^\vee |\cdot|_{F}^{-s_{r}}\oplus \cdots\oplus \phi_1^\vee|\cdot|_{F}^{-s_{1}},$$ 
		where 
		\begin{itemize}
			\item $\phi_i$ is an irreducible tempered representation of $\WD_F$ of dimension $d_i$; 
			\item $\phi_0\in \Phi_{\mathrm{temp}}(\SP(W_{2n_0}))$;
			\item $s_1\geq \cdots \geq s_r>0$;
			\item $d_1+\cdots +d_r+n_0=n$. 
		\end{itemize}
		We denote by $\tau_{i}$ be the irreducible tempered representation of $\GL_{d_i}(F)$ corresponding to $\phi_i$, then the $L$-packet $\Pi_{\phi}$ consists of the unique irreducible quotients $\sigma$ of the standard modules $$\Ind_{Q}^{\SP(W_{2n})}\left(\tau_{1}|\cdot|_{F}^{s_{1}} \otimes \cdots \otimes \tau_{r}|\cdot|_{F}^{s_{r}} \otimes \sigma_{0}\right),$$
		where $Q$ is a parabolic subgroup of $\SP(W_{2n})$ with Levi subgroup $M_{Q}\cong \GL_{k_{1}}(F) \times \cdots \times \GL_{k_{r}}(F) \times \SP(W_{2n_0})$ and $\sigma_{0}$ runs over elements of $\Pi_{\phi_{0}}(\SP(W_{2n_0}))$. Moreover, the natural embedding $\mathcal S_{\phi_0}\hookrightarrow  \mathcal S_{\phi}$ is an isomorphism and $$\mathcal J_{\mathfrak W^\prime_{\psi,c}}(\sigma)=\mathcal J_{\mathfrak W^\prime_{\psi,c}}(\sigma_0)$$ if we identify $\mathcal S_{\phi_0}$ with $\mathcal S_{\phi}$ via the above isomorphism. 
		\item
		The map $\mathcal L$ respects standard $\gamma$-factor. Namely, we have 
		$$
		\gamma(s,\sigma,\chi,\psi)=\gamma(s,\phi\otimes\chi,\psi)
		$$
		for $\sigma\in \Pi_{\phi}$ and any character $\chi$ of $F^{\times}$.
		\item 
		The map $\mathcal L$ respects Plancherel measures. Namely, we have 
		\begin{align*}
		\mu_{\psi}(\tau_s\otimes \sigma)&=\gamma(s,\phi_{\tau}\otimes \phi^{ \vee},\psi)\cdot \gamma(-s,\phi_{\tau}^{\vee}\otimes\phi, \psi_{-1} )\\
		& \times \gamma (2s, \wedge^2\circ \phi_{\tau}, \psi)\cdot \gamma(-2s, \wedge^{2} \circ \phi_{\tau}^{\vee},\psi_{-1})
		\end{align*}
		for any $\sigma\in \Pi_{\phi}$ and $\tau\in \Irr (\GL_k(F))$ with $L$-parameter $\phi_{\tau}$. 
	\end{enumerate}
\end{theorem}
\begin{remark}
	The local intertwining relation we used here is the same as in \cite[\S 6.6]{MR3788848}, which is a little bit different from the local intertwining relation formulated by Arthur \cite[\S 2.4]{MR3135650}. In \cite{MR3801418}, Atobe proved that the local intertwining relation we used here is a consequence of the local intertwining relation formulated by Arthur. 
\end{remark}

%$$
%\iota_{c_2}(\sigma).\iota_{c_1}(\sigma)^{-1}=\eta_{\phi,c_2/c_1}
%$$

%(2) $\phi$ is generic, i.e., $L(s,\phi,Ad)$ is regular at $s=1$ if and only if $\Pi_{\phi}$ contains a $\mu-$ generic representation $\sigma$ for each generic character $\mu$. 

%(3) Suppose $\phi$ is generic, then for each $c\in F^{\times}$, $\sigma\in \Pi_{\phi}$ is $\mu_{c}$ generic if and only if $l_c(\sigma)$ is the trivial representation of $\mathcal {S}_{\phi}^{+}$.

%(4) If $\phi$ are unramified, then $\Pi_{\phi}$ contains a unique unramified representation $\sigma$, and it corresponds to the trivial representation of $\mathcal {S}_{\phi}^{+}$ under $\iota_{1}$. 



%The adjoint group $\SP_{ad}(W)$ acts on $\SP(W)$ by conjugation, and hence acts on $\Irr(\SP(W))$. This action factor through $\SP_{ad}(W)/\im \SP(W)$. This quotient is isomorphic to the cohomology group 
%$$
%E=\SP_{ad}(W)/\im \SP(W)=\ker (H^1(F,Z)\rightarrow H^1(F,\SP(W)))=F^{\times}/F^{\times 2}
%$$
%Here $Z\cong \mu_2$ is the center of $\SP(W)$ and the last equliaty comes from 	
%$$
%H^1(F,\mu_2)= F^{\times}/F^{\times 2} \quad H^1(F,\SP(W)) =1
%$$
%For $d\in F^{\times}/F^{\times 2}=E$, we denote the action of $d$ on $\sigma\in Irr(\SP(W))$ by $\sigma^{d}$. We have the following proposition describing the adjoint action on $\Irr(\SP(W))$ in terms of $L$-parameter and character. 
 
For any $a\in F^\times$, let $\delta_a\in \GL(W_{2n})$ such that 
\begin{align}\label{112}
\langle \delta_a w, \delta_a w^\prime\rangle_W=a \langle w,w^\prime\rangle_W \,\,\mbox{for all $w,w^\prime\in W_{2n}$}. 
\end{align}
For any $\sigma\in \Irr(\SP(W_{2n}))$, we define a new representation $\sigma^{\delta_a}$ of $\SP(W_{2n})$ by 
\begin{align}\label{113}
\sigma^{\delta_a}(g)=\sigma(\delta_a^{-1}g\delta_a). 
\end{align}
Note that different choices of $\delta_a$ differ by an element in $\SP(W_{2n})$, hence the isomorphic class of $\sigma^{\delta_a}$ is independent of the choice of $\delta_a$. The map  
\begin{align*}
\SP(W_{2n})&\rightarrow \SP(W_{2n})\\
g &\mapsto \delta_a^{-1}g\delta_a
\end{align*}
transfers $\mathfrak W^\prime_{\psi,c}$ to $\mathfrak W^\prime_{\psi,ac}$. By \cite[Theorem 4.3]{MR3194648}, we have 
\begin{align}\label{103}
\mathcal L(\sigma^{\delta_a})=\mathcal L(\sigma),\quad \mathcal J_{\mathfrak W^\prime_{\psi,c}}(\sigma^{\delta_a})=\mathcal J_{\mathfrak W^\prime_{\psi,ac}}(\sigma)=\mathcal J_{\mathfrak W^\prime_{\psi,c}}(\sigma)\otimes \eta_{\phi,a}. 
\end{align}
for any $\sigma \in \Irr(\SP(W_{2n}))$, where $\phi$ is the $L$-parameter of $\sigma$. In particular, it follows from \cite[Chapter 4. II.1]{MR1041060} that 
$\sigma^\vee\cong \sigma^{\delta_{-1}}$ for $\sigma\in \Irr(\SP(W_{2n}))$, hence by \ref{103}, we have  
\begin{align}\label{104}
\mathcal L(\sigma^\vee)=\mathcal L(\sigma),\quad \mathcal J_{\mathfrak W^\prime_{\psi,c}}(\sigma^\vee)=\mathcal J_{\mathfrak W^\prime_{\psi,c}}(\sigma)\otimes \eta_{\phi,-1}. 
\end{align}






\subsection{LLC for even orthogonal groups}\label{LLCeven}
Next we state the local Langlands correspondence for even orthogonal groups, which is the main theorem of this paper.  

\begin{theorem}\label{desideratumall}
	Let $V_{2n}$ be a $2n$-dimensional orthogonal space and $\chi_V$ be the discriminant character of $V$.
	\begin{enumerate}[(1).]
		\item There exists a surjection
		$$
		\mathcal L: \bigsqcup_{V_{2n}^{\bullet}} \Irr \left(\mathrm O(V_{2n}^\bullet)\right) \longrightarrow \Phi(\mathrm O(V_{2n})),$$
		which is a finite-to-one map, where $V_{2n}^{\bullet}$ runs over the $2n$-dimensional orthogonal spaces with discriminant character $\chi_V.$ For any $\phi \in \Phi(\mathrm O(V_{2n})),$ we denote $\mathcal L^{-1}(\phi)$ by $\Pi_{\phi}$ and call it the $L$-packet of $\phi$. We also write $\Pi_{\phi}(\mathrm O(V_{2n}))=\Pi_{\phi}\cap \Irr(\mathrm O(V_{2n}))$. 
		\item For each Whittaker datum $\mathfrak W_{c}$ of $\bigsqcup_{V_{2n}^{\bullet}}\mathrm O(V_{2n}^{\bullet})$, there exist a canonical bijection
		\begin{align}\label{51}
		\mathcal J_{\mathfrak W_{c}} : \Pi_{\phi} \longrightarrow \widehat {\mathcal {S}_{\phi}}. 
		\end{align}
		We write $\pi=\pi_{\mathfrak W_{c}}(\phi,\eta)$ if $\pi\in  \Pi_{\phi}$ corresponding to $\eta \in \widehat{\mathcal {S}_{\phi}}$ under $\mathcal J_{\mathfrak W_{c}}$.
		\item Assume that $\phi$ is a tempered $L$-parameter for $\mathrm O(V_{2n})$, then for each Whittaker datum $\mathfrak W_{c}$ of $\bigsqcup_{V_{2n}^{\bullet}}\mathrm O(V_{2n}^{\bullet})$, 
		\begin{itemize}
			\item there is an unique $\mathfrak W^+_{c}$-generic representation $\pi$ in $\Pi_{\phi}$, which corresponds to  the trivial character under $\mathcal J_{\mathfrak W_{c}}(\pi)$; 
			\item there is an unique $\mathfrak W^{-}_{c}$-generic representation $\pi$ in $\Pi_{\phi}$, which corresponds to  $\kappa_{\phi}$ under $\mathcal J_{\mathfrak W_{c}}(\pi)$, where $\kappa_{\phi}$ is defined in \ref{kappa}. 
		\end{itemize}
		\item 
		For $\pi\in \Pi_{\phi}$ and $\mathfrak W_{c_1},\mathfrak W_{c_2}$ be two Whittaker data of $\bigsqcup_{V_{2n}^{\bullet}}\mathrm O(V_{2n}^{\bullet})$, we have 
		$$
		\mathcal J_{\mathfrak W_{c_2}}(\pi)=\mathcal J_{\mathfrak W_{c_1}}(\pi)\otimes \eta_{\phi\chi_{V},c_2/c_1},
		$$
		where $\phi\chi_{V}=\phi\otimes\chi_{V}$ and $\eta_{\phi\chi_{V},c_2/c_1}$ is defined in \ref{eta}. 
		\item
		$\pi\in \Pi_{\phi}$ is a tempered representation iff $\phi$ is a tempered parameter and $\pi\in \Pi_{\phi}$ is a discrete series representation iff $\phi$ is a discrete parameter.
		\item
		Let $\mathfrak W_c$ be a Whittaker datum of $\bigsqcup_{V_{2n}^{\bullet}}\mathrm O(V_{2n}^\bullet)$. Then $\pi$ is a representation of $\mathrm O(V_{2n}^+)$ if and only if
		$$\mathcal J_{\mathfrak W_{c}}(\pi)(z_\phi)=  \chi_{V}(c).$$
		\item
		The following are equivalent:
		\begin{itemize}
			\item 	$\phi \in \Phi^{\epsilon}\left(\mathrm{O}\left(V_{2 n}\right)\right)$;
			\item some $\pi\in \Pi_{\phi}$ satisfies $\pi \otimes \det \neq \pi$;
			\item all $\pi\in \Pi_{\phi}$ satisfy $\pi \otimes\det 
			\neq \pi$. 
		\end{itemize}
	    \item
		 For $\pi\in \Pi_{\phi}$, the determinant twist $\pi\otimes \det$ also belongs to $\Pi_{\phi}$, and 
		$$\mathcal J_{\mathfrak W_{c}}(\pi \otimes \det)=\mathcal J_{\mathfrak W_{c}}(\pi)\otimes  \kappa_{\phi}.$$
		\item
		\textbf{(Local intertwining relation)}
		Assume that $\phi=\phi_{\tau}+\phi_{0}+\phi_{\tau}^{\vee}$, where $\phi_{0}$ is an element in $\Phi_{\mathrm {temp}}\left(\mathrm O(V_{2n_0})\right)$
		and $\phi_{\tau}$ is an irreducible tempered representation of $WD_{F}$ corresponding to $\tau \in \Irr(\GL_{k}(F))$, so there is a natural embedding $S_{\phi_{0}}\hookrightarrow \mathcal {S}_{\phi}$. Let $P$ be a parabolic subgroup of $\mathrm O(V_{2n}^\bullet)$ with Levi subgroup $$M_{P}\cong \GL_{k}(F) \times \mathrm O(V_{2n_0}^\bullet)$$ and $\pi_{0}$ be the irreducible tempered representation of $\mathrm O(V_{2n_0}^\bullet)$ corresponding to $(\phi_{0},\eta_0)$ under $\mathcal L$ and $\mathcal J_{\mathfrak W_{\psi,c}}$. Then the induced representation $\Ind_{P}^{\mathrm O(V_{2n}^\bullet)}\left(\tau \otimes \pi_{0}\right)$ has a decomposition 
		$$
		\Ind_{P}^{\mathrm O(V_{2n}^\bullet)}\left(\tau \otimes \pi_{0}\right)=\bigoplus_{\eta}\pi_{\mathfrak W_{c}}(\phi,\eta),
		$$
		where the sum runs over all $\widehat{\mathcal {S}_{\phi}}$ such that $\eta|_{\mathcal S_{\phi_0}}=\eta_0$. Moreover if $\phi_{\tau}$ is self-dual and of orthogonal type. Let 
		$$
		R_{\mathfrak W_c}(w, \tau \otimes \pi_0)
		\in \End_{\mathrm O(V_{2n}^\bullet)}
		\left(\Ind_{P}^{\mathrm O(V_{2n}^\bullet)}(\tau\otimes\pi_0)\right)$$ be the normalized intertwining operator associated to the Whittaker datum $\mathfrak W_{c}$ (see section \ref{normalizingintertwing}), where $w$ is the unique non-trivial element in the relative Wely group for $M_P$. Then
		$$R_{\mathfrak W_c}(w, \tau \otimes \pi_0) |_{\pi}=\mathcal J_{\mathfrak W_{c}}(\pi)(a),$$
		where $a \in \mathcal {S}_{\phi}$ corresponding $\phi_{\tau}$. 
	\item \textbf{(Compatibility with Langlands quotients)}
	Assume that
	$$\phi=\phi_1|\cdot|_{F}^{s_{1}}\oplus \cdots\oplus \phi_{r}|\cdot|_{F}^{s_{r}}+\phi_{0}\oplus \phi_{r}^\vee |\cdot|_{F}^{-s_{r}}\oplus \cdots\oplus \phi_1^\vee|\cdot|_{F}^{-s_{1}},$$ 
	where 
	\begin{itemize}
		\item $\phi_i$ is an irreducible tempered representation of $\WD_F$ of dimension $d_i$; 
		\item $\phi_0\in \Phi_{\mathrm{temp}}(\mathrm O(V_{2n_0}))$;
		\item $s_1\geq \cdots \geq s_r>0$;
		\item $d_1+\cdots +d_r+n_0=n$. 
	\end{itemize}
	We write $\tau_{i}$ to be the irreducible tempered representation of $\GL_{d_i}(F)$ corresponding to $\phi_i$, then the $L$-packet $\Pi_{\phi}$ consists of the unique irreducible quotients $\pi$ of the standard modules $$\Ind_{P}^{\mathrm O(V_{2n}^\bullet)}\left(\tau_{1}|\cdot|_{F}^{s_{1}} \otimes \cdots \otimes \tau_{r}|\cdot|_{F}^{s_{r}} \otimes \pi_{0}\right),$$
	where $P$ is a parabolic subgroup of $\mathrm O(V_{2n}^\bullet)$ with Levi subgroup $M_{P}\cong \GL_{k_{1}}(F) \times \cdots \times \GL_{k_{r}}(F) \times \mathrm O(V_{2n_0}^\bullet)$ and $\pi_{0}$ runs over elements of $\Pi_{\phi_{0}}$. Moreover, the natural embedding $\mathcal S_{\phi_0}\hookrightarrow  \mathcal {S}_{\phi}$ is an isomorphism and $$\mathcal J_{\mathfrak W_{c}}(\pi)=\mathcal J_{\mathfrak W_{c}}(\pi_0)$$ if we identify $\mathcal S_{\phi_0}$ with $\mathcal {S}_{\phi}$ via the above isomorphism. 
	\item
		The map respects standard $\gamma$-factor. Namely, we have 
		$$
		\gamma(s,\pi,\chi,\psi)=\gamma(s,\phi\otimes\chi,\psi)
		$$
		for $\pi\in \Pi_{\phi}$ and any character $\chi$ of $F^{\times}$. 
		\item 
		The map respects Plancherel measures. Namely, we have 
		\begin{align*}
		\mu_{\psi}(\tau_s\otimes \pi)&=\gamma(s,\phi_{\tau}\otimes \phi^{ \vee},\psi)\cdot \gamma(-s,\phi_{\tau}^{\vee}\otimes\phi, \psi_{-1} )\\
		& \times \gamma (2s, \wedge^2\circ \phi_{\tau}, \psi)\cdot \gamma(-2s, \wedge^{2} \circ \phi_{\tau}^{\vee},\psi_{-1})
		\end{align*}
		for any $\pi\in \Pi_{\phi}$ and $\tau\in \Irr (\GL_k(F))$ with $L$-parameter $\phi_{\tau}$. 
	\end{enumerate}
\end{theorem}

The LLC for quasi-split even orthogonal groups has been proved by Arthur\cite{MR3135650} and was explicated in Atobe-Gan\cite{MR3708200}. More precisely, they proved
\begin{theorem}\label{Arthurorth}
	Let $V^+_{2n}=V_{2n}$ be the orthogonal space associated to $(d,1)$. 
	\begin{enumerate}[(1).]
		\item There exists a surjective 
		$$
		\mathcal L^{A}: \Irr\left(\mathrm O(V_{2n}^{+})\right) \longrightarrow \Phi\left(\mathrm O(V_{2n})\right),$$
		which is a finite-to-one map. For $\phi \in \Phi(\mathrm O(V_{2n})),$ we denote the inverse image of $\phi$ by $\Pi_{\phi}(\mathrm O(V^+_{2n}))$. 
		\item For a Whittaker datum $\mathfrak W_{c}$ of $\mathrm O(V_{2n}^+)$, there exists a canonical bijection
		$$\mathcal J_{\mathfrak W_{c}}^A: \Pi_{\phi}(\mathrm O(V_{2n}^+)) \longrightarrow \widehat{\bar {\mathcal {S}}_{\phi}}.$$
	\end{enumerate}  
	 Moreover, the map $\mathcal L^A$ and $\mathcal J_{\mathfrak W_{c}}^A$ satisfies all the other properties in Theorem  \ref{desideratumall}. 
\end{theorem}
\begin{remark}
\begin{enumerate}
	\item M\oe glin \cite[\S 1.4 Theorem 1.4.1]{MR2767522} and M\oe glin-Renard \cite{MR3839702} have partially extended Theorem \ref{Arthurorth} to pure inner forms as well, though we are not sure if all the statements in Theorem \ref{desideratumall} were verified in their work.
	\item When $\disc(V_{2n}^+)=d\notin F^{\times 2}$, then as in \ref{139}, we have 
	\begin{align*}
	V_{2n}^+\cong V_{(d,1)}\oplus \mathbb H^{m-1},\quad 
	V_{2n}^-\cong V_{(d,c_0)}\oplus \mathbb H^{m-1} 
	\end{align*} 
	for a $c_0\notin N_{E/F}(E^\times)$. In particular, both $\mathrm O(V_{2n}^+)$ and $\mathrm O(V_{2n}^-)$ are quasi-split. Moreover, as in \ref{141}, we have an isomorphism  $$\iota_{c_0}: \mathrm O(V_{2n}^+)\cong \mathrm O(V_{2n}^{-}).$$
	For any $\pi\in \Irr(\mathrm O(V_{2n}^+))$, we may view $\pi$ as an reprensentation of $\mathrm O(V_{2n}^-)$ via the isomorphism $\iota_{c_0}$, to dinstingush it with representation $\pi$ of $\mathrm O(V_{2n}^+)$, we denote it by $\widetilde{\pi}$. We may extend the map $\mathcal L^A$ to $\Irr\left(\mathrm O(V_{2n}^{-})\right)$ by defining 
	\begin{align*}
	\mathcal L^A: \Irr\left(\mathrm O(V_{2n}^{-})\right)&\longrightarrow \Phi(\mathrm O(V_{2n}))\\
	\widetilde{\pi}&\mapsto\mathcal L^A(\widetilde{\pi})\coloneqq \mathcal L^A(\pi). 
	\end{align*}
	Let $\Pi_{\phi}(\mathrm O(V_{2n}^-))$ be the the inverse image of $\phi$ under $\mathcal L^A$. 
	\begin{itemize}
		\item If $c\in N_{E/F}(E^\times)$, then $\mathfrak W_{\psi,c}$ is a Whittaker datum of $\mathrm O(V_{2n}^+)$, we may extend the map $\mathcal J_{\mathfrak W_{c}}^A$ to $\Pi_{\phi}(\mathrm O(V_{2n}^-))$ be defining 
		$$\mathcal J_{\mathfrak W_{c}}^A(\widetilde{\pi})= \mathcal J_{\mathfrak W_{c}}^A(\pi)\otimes \eta_{\phi\chi_{V},c_0}.$$
		\item If $c\notin N_{E/F}(E^\times)$, then $\mathfrak W_{\psi,c}$ is a Whittaker datum of $\mathrm O(V_{2n}^-)$, we define the map $\mathcal J_{\mathfrak W_{c}}^A$ to be 
		\begin{align*}
		\mathcal J_{\mathfrak W_{c}}^A(\pi)\coloneqq \mathcal J_{\mathfrak W_{1}}^A(\pi)\otimes \eta_{\phi\chi_{V},c}(\pi)\quad \mbox{for}\,\, \pi\in \bigsqcup_{V_{2n}^\pm} \Irr\left(\mathrm O(V_{2n}^{\pm})\right).
		\end{align*} 
	\end{itemize}
	One can check that these extensions satisfy all the properties in Theorem \ref{desideratumall}. Hence when $d\notin F^{\times 2}$, Theorem \ref{desideratumall} follows from Theorem \ref{Arthurorth} and the constructions above. But our proof for Theorem \ref{desideratumall} in later sections will not use these constructions. In particular, it provides another way to prove Theorem \ref{desideratumall} for the case when $d\notin F^{\times 2}$.
\end{enumerate}	
\end{remark}

We will use the local theta correspondence to construct the map $\mathcal L$ and $\mathcal J_{\mathfrak W_{c}}$, and then prove they satisfy all the properties in Theorem \ref{desideratumall}. We will also prove our classification coincides with Arthur's in the quasi-split case, i.e.,  
\begin{align*}
\mathcal L|_{\Irr\left(\mathrm O(V_{2n}^{+})\right)} &= \mathcal L^A|_{\Irr\left(\mathrm O(V_{2n}^{+})\right)},\\
\mathcal J_{\mathfrak W_{c}}|_{\Pi_{\phi}(\mathrm O(V_{2n}^+))}&=\mathcal J_{\mathfrak W_{c}}^A|_{\Pi_{\phi}(\mathrm O(V_{2n}^+))} \,\, \mbox{for}\,\,c\in  N_{E / F}(E^{ \times}).
\end{align*}
We emphasize that our proof for Theorem \ref{desideratumall} relies on Theorem \ref{llcsympletic} and Theorem \ref{Arthurorth}. 


The following sections are devoted to proving Theorem \ref{desideratumall}. 

%$$
%\iota_{c_2}(\sigma).\iota_{c_1}(\sigma)^{-1}=\eta_{\phi,c_2/c_1}
%$$

%(2) $\phi$ is generic, i.e., $L(s,\phi,Ad)$ is regular at $s=1$ if and only if $\Pi_{\phi}$ contains a $\mu-$ generic representation $\sigma$ for each generic character $\mu$. 

%(3) Suppose $\phi$ is generic, then for each $c\in F^{\times}$, $\sigma\in \Pi_{\phi}$ is $\mu_{c}$ generic if and only if $l_c(\sigma)$ is the trivial representation of $\mathcal {S}_{\phi}^{+}$.

%(4) If $\phi$ are unramified, then $\Pi_{\phi}$ contains a unique unramified representation $\sigma$, and it corresponds to the trivial representation of $\mathcal {S}_{\phi}^{+}$ under $\iota_{1}$. 



%The adjoint group $\SP_{ad}(W)$ acts on $\SP(W)$ by conjugation, and hence acts on $\Irr(\SP(W))$. This action factor through $\SP_{ad}(W)/\im \SP(W)$. This quotient is isomorphic to the cohomology group 
%$$
%E=\SP_{ad}(W)/\im \SP(W)=\ker (H^1(F,Z)\rightarrow H^1(F,\SP(W)))=F^{\times}/F^{\times 2}
%$$
%Here $Z\cong \mu_2$ is the center of $\SP(W)$ and the last equliaty comes from 	
%$$
%H^1(F,\mu_2)= F^{\times}/F^{\times 2} \quad H^1(F,\SP(W)) =1
%$$
%For $d\in F^{\times}/F^{\times 2}=E$, we denote the action of $d$ on $\sigma\in Irr(\SP(W))$ by $\sigma^{d}$. We have the following proposition describing the adjoint action on $\Irr(\SP(W))$ in terms of $L$-parameter and character. 



%We can describe the action more explicitly, for $a\in F^{\times}$ and $\pi\in \Irr(\SP(W))$, let $\delta_{a}\in \GSP(W)$  be sympletic similitude pf scale $a$, define $\pi^{\delta_a}(g)=\pi(\delta_{a}g\delta_{a}^{-1})$, then this action does not depend on the choice of $\delta_a$ and factors through $F^{\times}/F^{\times 2}$. 




%Change of whittake data  Desideratum 3.12 in Atobe and Gan ??



\section{Construction}\label{construction}
We construct the LLC for even orthogonal groups in this section. We first construct the correspondence for tempered representations and then extend to non-tempered representations by using the Langlands classification for $p$-adic groups. Several properties in Theorem \ref{desideratumall} will be proved along the way. 
\subsection{Construction of $\mathcal L_{\psi}$}
Fix a non-trivial character $\psi$ of $F$. Let $V_{2n}$ be a $2n$-dimensional orthogonal space with discriminant character $\chi_{V}$. We will construct a map 
$$
\mathcal L_{\psi}: \bigsqcup_{V_{2n}^{\bullet}} \Irrt \left(\mathrm O(V_{2n}^\bullet)\right) \longrightarrow \Para(\mathrm O(V_{2n})),$$
where $V_{2n}^{\bullet}$ runs over the $2n$-dimensional orthogonal spaces with discriminant character $\chi_V$. Later in section \ref{changeofpsi}, we will show that $\mathcal L_\psi$ is independent of the choice of $\psi$. So we get our desired $\mathcal L$.  

For any $\pi\in \Irrt \mathrm O(V^\bullet_{2n})$, Consider the following two representations

\begin{equation}\label{con}
\begin{cases}
\sigma_1\coloneqq\theta_{W_{2n}, V^\bullet_{2n} ,\psi}(\pi)\quad &\mbox{of}\,\, \SP(W_{2n});\\
\sigma_2\coloneqq\theta_{W_{2n-2}, V^\bullet_{2n}, \psi}(\pi\otimes \det)\quad &\mbox{of}\,\, \SP(W_{2n-2}).
\end{cases}
\end{equation}
By the conservation relation \ref{con2}, exactly one of $\sigma_{i}$ is non-vanishing. We shall attach a $L$-parameter to $\pi$ in terms of the $L$-parameter of $\sigma_{i}$. 


$\bullet$ \underline{Case I}: If $\sigma_1=\theta_{W_{2n}, V^\bullet_{2n}, \psi}(\pi)\neq 0$, then we have: 
	\begin{lemma}
		Let $\phi^+$ be the $L$-parameter of $\sigma_1=\theta_{W_{2n}, V^\bullet_{2n}, \psi}(\pi)$. Then we have $\chi_{V}\subseteq \phi^+$. 
	\end{lemma}
	\begin{proof}
		By Theorem \ref{pole}, we know that $\gamma(s,\sigma, \chi_V, \psi)$ has a pole at $s=1$. On the other hand, it follows from the Theorem \ref{llcsympletic} (8) that 
		$$\gamma(s,\sigma_1, \chi_V, \psi)= \gamma(s,\phi^+\otimes\chi_V,\psi).$$ 
		So $\gamma(s,\phi^+ \otimes\chi_V,\psi)$ has a pole at $s=1$. Note that  
		\begin{align*}
		\gamma(s,\phi^+ \otimes\chi_V,\psi)=&\varepsilon(s,\phi^+\otimes \chi_V,\psi)\times \frac{L(1-s,(\phi^+ \otimes\chi_V)^\vee)}{L(s,\phi^+\otimes\chi_V)}\\
		=&\varepsilon(s,\phi^+\otimes \chi_V,\psi)\times \frac{L(1-s,\phi^+ \otimes\chi_V)}{L(s,\phi^+\otimes\chi_V)}.
		\end{align*}
		Since $\varepsilon(s,\phi^+ \otimes\chi_V,\psi)$ is holomorphic at $s=1$ and $L(s,\phi^+\otimes \chi_V)$ is non-zero at $s=1$, we deduce that $L(1-s,\phi^+ \otimes\chi_V)$ has a pole at $s=1$. Since $\phi^+\otimes\chi_{V}$ is tempered, this implies that $\phi^+\otimes\chi_V$ contains the trivial representation, which is equivalent to say $\phi^+$ contains $\chi_{V}$.
	\end{proof}
	In this case, we define the $L$-parameter of $\pi$ to be 
	$$\phi=(\phi^+\otimes\chi_V) -\mathrm 1.$$
	
$\bullet$ \underline{Case II}: If $\sigma_2=\theta_{W_{2n-2}, V^\bullet_{2n}, \psi}(\pi\otimes \det)\neq 0 $, we simply define the $L$-parameter of $\pi$ to be 
	$$
	\phi=(\phi^{-} \otimes\chi_{V})\oplus \mathrm{1},
	$$
	where $\phi^{-}$ is the $L$-parameter of $\sigma_2$. 

Note that we have $\phi\in \Phi(\mathrm O(V_{2n}))$ in both cases. Moreover, it follows from Lemma \ref{temperedtotempered} and Theoreom \ref{llcsympletic} (5) that both $\phi^+$ and $\phi^-$ are tempered parameters, so $\phi\in \Para(\mathrm O(V_{2n}))$ in both cases. Combining these two cases, we defined a map
$$
\mathcal L_{\psi}: \bigsqcup_{V_{2n}^{\bullet}} \Irrt \left(\mathrm O(V_{2n}^\bullet)\right) \longrightarrow \Para(\mathrm O(V_{2n})),$$
where $V_{2n}^{\bullet}$ runs over the $2n$-dimensional orthogonal spaces with discriminant character $\chi_V$. For a parameter $\phi\in \Para(\mathrm O(V_{2n})) $, we define the packet $\Pi_{\phi,\psi}$ to be the fiber $\mathcal L_{\psi}^{-1}(\phi)$ and $\Pi_{\phi,\psi}(\mathrm O(V_{2n}^{\bullet}))= \Pi_{\phi,\psi}\cap \Irrt \mathrm O(V_{2n}^{\bullet})$.




\subsection{Local factors}
So far we have associated a tempered $L$-parameter $\phi$ to every  $\pi\in \Irrt(\mathrm O(V^\bullet_{2n}))$. We then show this assignment respects the standard $\gamma$-factors and the Plancherel measures. These prove Theorem \ref{desideratumall} (11) and (12) in the tempered case. 


\begin{lemma}\label{respectgamma}
	Let $\pi\in \Irrt \mathrm O(V^\bullet_{2n})$ and $\mathcal L_{\psi}(\pi)=\phi$. Let $\chi$ be a character of $F^{\times}$. Then we have $\gamma(s,\pi, \chi,\psi)=\gamma(s,\phi\otimes \chi, \psi)$. 
\end{lemma}
\begin{proof}
According to our construction, we need to consider two cases. 

$\bullet$ \underline{Case I}: If $\sigma_1=\theta_{W_{2n}, V^\bullet_{2n}, \psi}(\pi)\neq 0$. It follows from Lemma \ref{comgamma} that  
$$
\frac{\gamma(s,\sigma_1,\chi,\psi)}{\gamma(s,\pi,\chi\chi_V,\psi)}= \gamma(s,\chi\chi_V,\psi).
$$
By Theorem \ref{llcsympletic} (8), we have 
$$
\gamma(s,\sigma_1,\chi,\psi)=\gamma(s,\phi^+\otimes \chi ,\psi),
$$
where $\phi^+=(\phi\otimes\chi_{V})\oplus \chi_{V}$ is the $L$-parameter of $\sigma_1$. So 
\begin{align*}
\gamma(s,\pi,\chi,\psi)&=\frac{\gamma(s,\sigma_1,\chi\chi_{V}^{-1},\psi)}{\gamma(s,(\chi\chi_{V}^{-1})\chi_V,\psi)}\\
&=\frac{\gamma(s,\phi^+\otimes\chi \chi_{V}^{-1} ,\psi)}{\gamma(s,\chi,\psi)}\\
&=\gamma(s,\phi \otimes\chi ,\psi).
\end{align*}

$\bullet$ \underline{Case II}: If $\sigma_2=\theta_{W_{2n-2}, V^\bullet_{2n}, \psi}(\pi\otimes \det)\neq 0 $. This follows from a similar calculation in Case I and the fact 
\begin{align*}
\gamma(s,\pi, \chi,\psi)= \gamma(s,\pi\otimes\det, \chi,\psi).
\end{align*}
These two cases complete the proof.  
\end{proof}

\begin{lemma}\label{respectplancherel}
Let $\pi\in \Irrt \mathrm O(V^\bullet_{2n})$ and $\mathcal L_{\psi}(\pi)=\phi$. Let $\tau$ be an irreducible smooth representation of $\GL_k(F)$ and $\phi_{\tau}$ be its $L$-parameter. Then we have   
$$
\begin{aligned} \mu_{\psi}\left(\tau_{s} \otimes \pi\right) &=\gamma\left(s, \phi_{\tau} \otimes \phi^{\vee}, \psi\right) \cdot \gamma\left(-s, \phi_{\tau}^{\vee} \otimes \phi, \psi_{-1}\right) \\
 & \quad \times \gamma\left(2 s, \wedge^{2} \circ \phi_{\tau}, \psi\right) \cdot \gamma\left(-2 s, \wedge^{2} \circ \phi_{\tau}^{\vee}, \psi_{-1}\right) .\end{aligned}
$$
\end{lemma}
\begin{proof}
According to our construction, we need to consider two cases.  

$\bullet$ \underline{Case I}: If $\sigma_1=\theta_{W_{2n}, V^\bullet_{2n}, \psi}(\pi)\neq 0$. It follows from theorem \ref{complan} that 
$$
\frac{\mu_{\psi}(\tau_s\chi_{V}\otimes\sigma_1)}{\mu_{\psi}(\tau_s\otimes\pi)}=\gamma(s,\tau,\psi)\cdot \gamma(-s,\tau^{\vee},\psi_{-1}).
$$ 
Let $\phi^+$ be the $L$-parameter for $\sigma_1$. Then by Theorem \ref{llcsympletic} (9), we have 
\begin{align*}
\mu_{\psi}(\tau_s\chi_{V}\otimes \sigma)&=\gamma(s,\phi_{\tau}\chi_{V}\otimes (\phi^+)^ {\vee},\psi)\cdot \gamma(-s, \phi_{\tau}^{\vee} \chi_{V}^{-1}\otimes\phi^+, \psi_{-1} )\\
& \quad \times \gamma (2s, \wedge^2\circ (\phi_{\tau}\chi_{V}), \psi). \gamma(-2s, \wedge^{2} \circ (\phi_{\tau}^{\vee} \chi_{V}^{-1}),\psi_{-1}).
\end{align*}
So 
\begin{align*}
\mu_{\psi}(\tau_s\otimes \pi)&= \mu_{\psi}(\tau_s\chi_{V}\otimes \sigma)\cdot \gamma(s,\tau,\psi)^{-1}.\gamma(-s,
\tau^{\vee},\psi_{-1})^{-1}\\ 
&=\gamma(s,\phi_{\tau}\chi_{V} \otimes (\phi^+)^{\vee},\psi)\cdot \gamma(-s, \phi_{\tau}^{\vee}\chi_{V}^{-1} \otimes \phi^+, \psi_{-1} )\\
& \quad \times \gamma (2s, \wedge^2\circ (\phi_{\tau}\chi_{V}), \psi)\cdot  \gamma(-2s, \wedge^{2} \circ (\phi_{\tau}^{\vee} \chi_{V}^{-1}),\psi_{-1})\\
&\quad \times \gamma(s,\phi_{\tau},\psi)^{-1}\cdot \gamma(-s,
\phi_{\tau}^{\vee},\psi_{-1})^{-1} \\ 
&= \gamma(s,\phi_{\tau}\otimes\phi^{\vee},\psi).\gamma(-s, \phi_{\tau}^{\vee}\otimes \phi, \psi_{-1} )\\
& \quad \times \gamma (2s, \wedge^2\circ \phi_{\tau}, \psi). \gamma(-2s, \wedge^{2} \circ \phi_{\tau}^{\vee},\psi_{-1}).
\end{align*}
Here we use $\phi^+=\phi\otimes\chi_{V}\oplus\chi_{V}$. 

$\bullet$ \underline{Case II}: If $\sigma_2=\theta_{W_{2n-2}, V^\bullet_{2n}, \psi}(\pi\otimes \det)\neq 0 $. This follows by a similar calculation as Case I and the fact 
$$\mu_{\psi}(\tau_s \otimes \pi)= \mu_{\psi}(\tau_s \otimes (\pi\otimes \det))$$
in \cite[Lemma B.1]{MR3166215}. 
\end{proof}



\subsection{Counting sizes of packets}
Our next goal is to attach a character of component group to each $\pi\in \Irrt(\mathrm O(V_{2n}^\bullet))$. To do this, we need some
preparations. In this subsection we consider the behaviour of $L$-parameters under local theta correspondence and count the sizes of $L$-packets.

The following lemma will be used later. 
\begin{lemma}\label{plancherel}
	Let $G=\mathrm O(V^\bullet_{2n})$ (resp. $G=\SP(W_{2n})$) and $\pi_1$ and $\pi_2$ (resp. $\sigma_1$ and $\sigma_2$) be two irreducible tempered representations of $G$ with $\mathcal L_{\psi}(\pi_1)=\phi_1, \mathcal L_{\psi}(\pi_2)=\phi_2$ (resp. $\mathcal L(\sigma_1)=\phi_1, \mathcal L(\sigma_2)=\phi_2$). Assume that 
	\begin{align*}
	\mu_{\psi}(\tau_{s}\otimes \pi_1)&=\mu_{\psi}(\tau_s\otimes \pi_2),\\
    \mu_{\psi}(\tau_{s}\otimes \sigma_1)&=\mu_{\psi}(\tau_s\otimes \sigma_2) 
	\end{align*}
	for all $k\geq 1$ and all irreducible smooth representation $\tau$ of $\GL_{k}(F)$. Then we have 
	$$
	\phi_1=\phi_2. 
	$$
\end{lemma}
\begin{proof}
	It follows Theorem \ref{llcsympletic} (9) and Lemma \ref{respectplancherel} that the map $\mathcal L_\psi$ and $\mathcal L$ respect the Plancherel measures, so we have  
	$$
    \gamma\left(s, \phi_{\tau} \otimes \phi_1^{\vee}, \psi\right) \cdot \gamma\left(-s, \phi_{\tau}^{\vee} \otimes \phi_1, \psi_{-1}\right) =
     \gamma\left(s, \phi_{\tau} \otimes \phi_2^{\vee}, \psi\right) \cdot \gamma\left(-s, \phi_{\tau}^{\vee} \otimes \phi_2, \psi_{-1}\right) 
	$$
	for any irreducible smooth representations $\tau$ of $\GL_k(F)$ with $L$-parameter $\phi_\tau$, then it follows from Lemma \ref{gammadetermine} that $\phi_1=\phi_2$.  
\end{proof}

Next we prove the $L$-parameter of $\pi$ and $\pi\otimes \det$ are the same, which proves the first part of Theorem \ref{desideratumall} (8) for tempered representations. 
\begin{proposition}\label{det}
Let $\pi\in \Irrt(\mathrm O(V^\bullet_{2n}))$. Then we have  
\begin{align*}
\mathcal L_{\psi}(\pi)=\mathcal L_{\psi}(\pi\otimes\det). 
\end{align*}
\end{proposition}
\begin{proof}
	By \cite[Lemma B.1]{MR3166215}, we know that  
	$$\mu_{\psi}(\tau_s \otimes \pi)= \mu_{\psi}(\tau_s \otimes (\pi\otimes \det))$$
	for any $k\geq 1$ and any irreducible smooth representations $\tau$ of $\GL_k(F)$. Then this proposition follows from Lemma \ref{plancherel}.
\end{proof}

The next proposition describes the behaviour of $L$-parameters under the local theta correspondence. 
\begin{proposition}\label{prasad} 
\begin{enumerate}[(i)]
	\item Let $\pi \in \Irrt \left( \mathrm O(V^\bullet_{2n})\right)$ and $\mathcal L_{\psi}(\pi)=\phi$. 
	\begin{itemize}
		\item If $\sigma\coloneqq\theta_{W_{2n}, V_{2n},  \psi}(\pi)\neq 0$, then 
		$$\mathcal L(\sigma)=(\phi\otimes \chi_V)\oplus \chi_{V}.$$
		\item If $\sigma\coloneqq\theta_{W_{2n}, V_{2n},  \psi}(\pi\otimes\det )\neq 0$, then 
		$$\mathcal L(\sigma)=(\phi\otimes \chi_V)\oplus \chi_{V}.$$
	\end{itemize}
	\item Let $\sigma\in \Irrt \left( \SP(W_{2n-2})\right)$ and $\mathcal L(\sigma)=\phi^-$. If $\pi\coloneqq\theta_{V^\bullet_{2n},W_{2n-2},\psi}(\sigma)\neq 0$, then  $$\mathcal L_{\psi}(\pi)=(\phi^-\otimes\chi_{V})\oplus \mathrm{1}.$$
\end{enumerate}
\end{proposition}
\begin{proof}
We first prove (1). By Proposition \ref{det}, we have 
$$\mathcal L_{\psi}(\pi\otimes\det)=\mathcal L_\psi(\pi)=\phi.$$ Then (i) follows from our construction of $\mathcal L_{\psi}$. The proof for (ii) is similar.
\end{proof}

The following corollary is a consequence of Proposition \ref{prasad}.  
\begin{corollary}\label{1inphi}
	\begin{enumerate}[(1)]
		\item Let $\pi \in \Irrt \left(\mathrm O(V^\bullet_{2n})\right)$ and $\mathcal L_{\psi}(\pi)=\phi$. If $\mathrm{1} \nsubseteq \phi$. Then we have  
		\begin{align*}
			\theta_{W_{2n-2}, V_{2n}^{\bullet} ,\psi}(\pi)= \theta_{W_{2n-2}, V_{2n}^{\bullet} ,\psi}(\pi\otimes\det)=0.
		\end{align*}
			So by the conservation relation \ref{con2}, both $\theta_{W_{2n}, V_{2n}^{\bullet},\psi}(\pi)$ and $\theta_{W_{2n}, V_{2n}^{\bullet}, \psi}(\pi\otimes \det)$ are non-zero. 
	\item Let $\sigma\in \Irrt \SP(W_{2n})$ and $\mathcal L(\sigma)=\phi^+$. If $\chi_{V}\nsubseteq \phi^+$, then
		\begin{align*}
	\theta_{V^{+}_{2n},W_{2n},\psi}(\sigma)=\theta_{V^{-}_{2n},W_{2n},\psi}(\sigma)=0.
	\end{align*}
	So by the conservation relation \ref{con2}, both $\theta_{V^{+}_{2n+2},W_{2n},\psi}(\sigma)$ and $\theta_{V^{-}_{2n+2},W_{2n},\psi}(\sigma)$ are non-zero. 
	\end{enumerate}
\end{corollary}


\begin{proposition}\label{1notinphi}
	\begin{enumerate}[(i)]
		\item Let $\pi \in \Irrt \left(\mathrm O(V^\bullet_{2n})\right)$ and $\mathcal L_{\psi}(\pi)=\phi$. If $\mathrm{1} \subseteq \phi$, then exactly one of $\theta_{W_{2n-2}, V_{2n}^{\bullet} ,\psi}(\pi)$ and $ \theta_{W_{2n-2}, V_{2n}^{\bullet} ,\psi}(\pi\otimes\det)$ is non-zero. In particular, we have $\pi\ncong \pi\otimes\det$ in this case. 
		\item Let $\sigma\in \Irrt \left(\SP(W_{2n})\right)$ and $\mathcal L(\sigma)=\phi^+$. If $\chi_{V}\subseteq \phi^+$, then exactly one of $\theta_{V^{+}_{2n},W_{2n},\psi}(\sigma)$ and $\theta_{V^{-}_{2n},W_{2n},\psi}(\sigma)$ is non-zero.  
	\end{enumerate}
\end{proposition}
\begin{proof}
We first prove (i). Put $$\phi^{+}=(\phi\otimes\chi_{V})\oplus\chi_{V},\quad  \phi^{-}=(\phi\otimes\chi_{V})-\chi_{V}.$$ 
 We define a map:
	$$
	\theta_{\psi,2n}: \bigsqcup_{V_{2n}^\bullet} \Irrt \left(\mathrm O(V_{2n}^{\bullet})\right)\longrightarrow \Irrt \left(\SP(W_{2n})\right) 
	$$
	by 
	$$
	\theta_{\psi,2n}(\pi)=
	\begin{cases}
	\theta_{W_{2n}, V_{2n}^\bullet, \psi}(\pi) \quad &\mbox{if}\,\,\theta_{W_{2n}, V_{2n}^\bullet,  \psi}(\pi)\neq 0, \\
	\theta_{W_{2n}, V_{2n}^\bullet,  \psi}(\pi\otimes\det) &\mbox{otherwise}.
	\end{cases}
	$$
	Then $\theta_{\psi,2n}(\pi)$ is non-zero by the conservation relation \ref{con2}. By Proposition \ref{prasad}, the restriction of $\theta_{\psi,2n}$ to the packet $\Pi_{\phi,\psi}$ gives 
	\begin{equation}\label{theta2n}
	\theta_{\psi,2n}: \Pi_{\phi,\psi} \longrightarrow \Pi_{\phi^{+}}.
	\end{equation}
	It follows from the Howe duality that
	\begin{align}\label{15}
	\theta_{\psi,2n}(\pi_1)\neq \theta_{\psi,2n}(\pi_2)\quad  \mbox{if}\,\, \pi_1\neq \pi_2\,\, \mbox{and}\,\, \pi_1 \neq \pi_2\otimes \det.
	\end{align}
	So for every $\sigma \in \Pi_{\phi^{+}}$, 
	the fibre $\theta_{\psi,2n}^{-1}(\sigma)$ contains at most two elements. 
	
	Next we shall separate into two cases according to $\mathrm{1}\oplus\mathrm{1}
	\subseteq \phi$ or not.  

$\bullet$ \underline{Case I}: If $\mathrm{1}\oplus\mathrm{1} \subseteq \phi$, then $\chi_{V}\subseteq \phi^{-}$. Define
		\begin{align*}
		\theta_{\psi,2n-2}: \Irrt \left(\SP(W_{2n-2})\right) \longrightarrow  \bigsqcup_{V_{2n}^\bullet} \Irrt \left(\mathrm O(V_{2n}^{\bullet})\right)
		\end{align*} 
		by 
		$$
		\theta_{\psi,2n-2}(\sigma)= 
		\begin{cases}
		\theta_{V^{+}_{2n},W_{2n-2},\psi}(\sigma) \quad &\mbox{if}\,\, \theta_{V^{+}_{2n},W_{2n-2},\psi}(\sigma)\neq 0,\\
		\theta_{V^{-}_{2n},W_{2n-2},\psi}(\sigma) &\mbox{otherwise}.
		\end{cases}
		$$
		Then $\theta_{\psi,2n-2}(\sigma)$ is non-zero by the conservation relation \ref{con2}. By Proposition \ref{prasad}, the restriction of $\theta_{\psi,2n-2}$ to the packet $\Pi_{\phi^{-}}$ gives  
		\begin{equation}\label{theta2n-2}
		\theta_{\psi,2n-2}: \Pi_{\phi^{-}}\longrightarrow \Pi_{\phi,\psi}.
		\end{equation}
		Moreover, it follows from the Howe duality and the conservation relation \ref{con2} that
		\begin{equation}\label{14}
		\begin{aligned}
		\theta_{\psi,2n-2}(\sigma_1)&\neq \theta_{\psi,2n-2}(\sigma_2) \quad &\mbox{if $\sigma_1\neq\sigma_2$} ,\\
		\theta_{\psi,2n-2}(\sigma_1)&\neq \theta_{\psi,2n-2}(\sigma_2)\otimes \det\quad &\mbox{for all  $\sigma_1,\sigma_2$}.
		\end{aligned}
		\end{equation} 
		So the image of $\Pi_{\phi^{-}}$ under $\theta_{\psi,2n-2}$ is at most half size of $\Pi_{\phi,\psi}$. Composing \ref{theta2n} and \ref{theta2n-2}, we get 
		$$
		\theta_{\psi,2n}\circ \theta_{\psi,2n-2}: \Pi_{\phi^{-}} \longrightarrow \Pi_{\phi^{+}}. 
		$$
		It follows from \ref{15} and \ref{14} that $\theta_{\psi,2n}\circ \theta_{\psi,2n-2}$ is injective. On the other hand, since $\mathrm 1\oplus\mathrm 1\subseteq \phi$, the inclusion map $\mathcal S_{\phi^{-}}\hookrightarrow \mathcal S_{\phi^+}$ is an isomorphism, then it follows from Theorem \ref{llcsympletic} (2) that 
		\begin{align}\label{135}
		|\Pi_{\phi^{-}}|=|\widehat{\bar {\mathcal S}_{\phi^-}}|=|\widehat{\bar {\mathcal S}_{\phi^+}}|=|\Pi_{\phi^{+}}|. 
		\end{align}
		Hence $\theta_{\psi,2n}\circ \theta_{\psi,2n-2}$ is a bijection. From the surjectivity of $\theta_{\psi,2n}$ and \ref{15}, we deduce 
		\begin{align*}
		|\Pi_{\phi,\psi}|\leq 2|\Pi_{\phi^{+}}|. 
		\end{align*}
		On the other hand, it follows from \ref{14} that 
		\begin{align*}
		|\Pi_{\phi,\psi}|\geq 2|\Pi_{\phi^{-}}|. 
		\end{align*}
		Combining \ref{135} and these inequalities, we deduce  
		\begin{align*}
		|\Pi_{\phi,\psi}|=2|\Pi_{\phi^{+}}|=  2|\Pi_{\phi^{-}}|=2|\widehat{\bar {\mathcal S}_{\phi^{-}}}|=|\widehat{\mathcal {S}_{\phi}}|. 
		\end{align*}
		Hence for $\pi\in \Pi_{\phi,\psi}$, exactly one of $\pi$ and $\pi\otimes\det$ is in the image of $\theta_{\psi,2n-2}$, this proves Case I. 
	
	$\bullet$ \underline{Case II}: If $\mathrm{1}\oplus\mathrm{1}\nsubseteq \phi$, then $\chi_{V}\nsubseteq \phi^{-}$. We define a map 
		\begin{equation}\label{theta2n-22}
		\theta_{\psi,2n-2}^{+}\sqcup\theta_{\psi,2n-2}^{-}: \Pi_{\phi^{-}}\sqcup \Pi_{\phi^{-}} \longrightarrow \Pi_{\phi,\psi}
		\end{equation}
		by 
		$$
		\begin{cases*}
		\sigma\mapsto  \theta_{V^{+}_{2n},W_{2n-2},\psi}(\sigma)\quad \mbox{for $\sigma$ in the first copy of $\Pi_{\phi^{-}}$},\\
		\sigma\mapsto \theta_{V^{-}_{2n},W_{2n-2},\psi}(\sigma) \quad \mbox{for $\sigma$ in the second copy of $\Pi_{\phi^{-}}$}. 
		\end{cases*}
		$$
		Since $\chi_{V}\nsubseteq \phi^{-}$, both $\theta_{V^{+}_{2n},W_{2n-2},\psi}(\sigma)$ and $\theta_{V^{-}_{2n},W_{2n-2},\psi}(\sigma)$ are non-zero by Corollary \ref{1inphi}, hence the map $\theta_{\psi,2n-2}^{+}\sqcup\theta_{\psi,2n-2}^{-}$ is well defined. Again, it follows from the Howe duality and the conservation relation \ref{con2} that 
		\begin{equation}\label{16}
	\begin{aligned}
	\left(\theta_{\psi,2n-2}^{+}\sqcup\theta_{\psi,2n-2}^{-}\right)(\sigma_1)\neq& \left(\theta_{\psi,2n-2}^{+}\sqcup\theta_{\psi,2n-2}^{-}\right)(\sigma_2)\quad &	\mbox{if}\,\, \sigma_1\neq \sigma_2,\\
	\left(\theta_{\psi,2n-2}^{+}\sqcup\theta_{\psi,2n-2}^{-}\right)(\sigma_1)\neq& \left(\theta_{\psi,2n-2}^{+}\sqcup\theta_{\psi,2n-2}^{-}\right)(\sigma_2)\otimes \det\quad &
	\mbox{for all}\,\, \sigma_1, \sigma_2. 
	\end{aligned}
		\end{equation}
		Composing \ref{theta2n}and \ref{theta2n-22}, we get 
		\begin{align*}
		\theta_{\psi,2n}\circ(\theta_{\psi,2n-2}^{+}\sqcup\theta_{\psi,2n-2}^{-}): \Pi_{\phi^{-}}\sqcup \Pi_{\phi^{-}} \longrightarrow \Pi_{\phi^{+}},
		\end{align*}
		and it follows from \ref{15} and \ref{16} that $\theta_{\psi,2n}\circ (\theta_{\psi,2n-2}^{+}\sqcup\theta_{\psi,2n-2}^{-})$ is injective. On the other hand, we have $|\mathcal S_{\phi^{+}}|=2|\mathcal S_{\phi^{-}}|$ in this case, then it follows from Theorem \ref{llcsympletic} (2) that 
		\begin{align}
		2|\Pi_{\phi^{-}}|=2|\widehat{\bar {\mathcal S}_{\phi^-}}|=|\widehat{\bar {\mathcal S}_{\phi^+}}|=|\Pi_{\phi^{+}}|. 
		\end{align}
		So $\theta_{\psi,2n}\circ (\theta_{\psi,2n-2}^{+}\sqcup\theta_{\psi,2n-2}^{-})$ is an bijection. Similar to Case I, together with \ref{15} and \ref{16}, we deduce  
		\begin{align*}
		|\Pi_{\phi,\psi}|=|\Pi_{\phi^{+}}|=  2|\Pi_{\phi^{-}}|=2|\widehat{\bar {\mathcal S}_{\phi^{-}}}|=|\widehat{\mathcal {S}_{\phi}}|. 
		\end{align*}
	Hence for $\pi\in \Pi_{\phi,\psi}$, exactly one of $\pi$ and $\pi\otimes\det$ is in the image of $(\theta_{\psi,2n-2}^{+}\sqcup\theta_{\psi,2n-2}^{-})$, this proves Case II.

	The proof for (ii) is similar.
\end{proof}

As a consequence of the Corollary \ref{1inphi} and Proposition \ref{1notinphi}, we can count the sizes of the fibers of $\mathcal L_{\psi}$. 
\begin{corollary}\label{size}
	Let $\phi\in \Phi_{\mathrm {temp}}(\mathrm O(V_{2n}))$. Then the size of the packet $\Pi_{\phi,\psi}$ is exactly the same with the size of $\widehat{\mathcal {S}_{\phi}}$. 
\end{corollary}
\begin{proof}
The case when $1\subseteq \phi$ follows directly from the proof of Proposition \ref{1notinphi}. So it sufficient to prove the case when 
$1\nsubseteq \phi$. We define 
	\begin{align*}
	\theta_{\psi,2n}: \bigsqcup_{V_{2n}^{\bullet}}\Irrt \left(\mathrm O(V_{2n}^{\bullet})\right)&\longrightarrow \Irrt \left(\SP(W_{2n})\right)\\
	\pi &\mapsto \theta_{W_{2n}, V_{2n}^{\bullet} ,\psi}(\pi). 
	\end{align*}
	By Propsition \ref{prasad}, Proposition \ref{1notinphi} and the Howe duality we deduce that the restriction of $\theta_{\psi,2n}$ to $\Pi_{\phi,\psi}$ gives a bijection 
	\begin{equation}\label{theta2n1}
	\theta_{\psi,2n}: \Pi_{\phi,\psi} \rightarrow \Pi_{\phi^+},
	\end{equation}
	where $\phi^+=(\phi\otimes\chi_{V})\oplus \chi_{V}$.
	It follows from Theorem \ref{llcsympletic} (2) that 
	\begin{align}\label{11}
	|\Pi_{\phi^+}|=|\widehat{\bar {\mathcal S}_{\phi^+}}|=\frac{1}{2}|\widehat{ \mathcal S_{\phi^+}}|.
	\end{align}
	On the other hand, since $\mathrm{1}\nsubseteq \phi$, we deduce  
	\begin{align}\label{13}
	\mathcal S_{\phi^+}\cong \mathcal {S}_{\phi}\oplus (\mathbb{Z}/2\mathbb{Z}) e \quad \mbox{and}\quad |\widehat{\mathcal {S}_{\phi}}|=\frac{1}{2}|\widehat{\mathcal S_{\phi^+}}|,
	\end{align}
	where $e$ correspond to $\chi_{V} \subseteq \phi^+$. Combining \ref{theta2n1},\ref{11} and \ref{13}, we deduce $$|\Pi_{\phi,\psi}|=|\Pi_{\phi^+}|=\frac{1}{2}|\widehat{\mathcal S_{\phi^+}}|=|\widehat{\mathcal {S}_{\phi}}|.$$ This finishes the proof. 
\end{proof}


%The following corollary follows from the lemma 
%\begin{corollary}
%When $\mathrm{1}\nsubseteq \phi$, the size of the packet $\Phi_{\phi}$ is exactly the same with the size of $\widehat{A}_{\phi}$. 
%\end{corollary}

%Combine corollary and ?? , we have


\subsection{Construction of $\mathcal J^\psi_{\mathfrak W_{c}}$}\label{constructJc}
Given a tempered parameter $\phi \in \Para \mathrm O(V_{2n})$, we have shown that the size of the packet $\Pi_{\phi,\psi}$ equals the size of $\widehat{\mathcal {S}_{\phi}}$. For each Whittaker datum $\mathfrak W_{c}$ of $\bigsqcup_{V_{2n}^{\bullet}}\mathrm O(V_{2n}^\bullet)$, we are going to define a bijection
\begin{align*}
\mathcal J^{\psi}_{\mathfrak W_{c}}: \Pi_{\phi,\psi}\longrightarrow \widehat{\mathcal {S}_{\phi}}
\end{align*} 
in this subsection. We will prove $\mathcal J^{\psi}_{\mathfrak W_{c}}$ is independent of the choice of $\psi$ in section \ref{changeofpsi}, hence get our desired $\mathcal J_{\mathfrak W_{c}}$.
 
Fix a Whittaker datum $\mathfrak W_{\psi,c}$ of $\bigsqcup_{V_{2n}^{\bullet}}\mathrm O(V_{2n}^\bullet)$, we shall construct $\mathcal J^{\psi}_{\mathfrak W_{c}}$ according to $\mathrm{1} \subseteq \phi$ or not. 

$\bullet$ \underline{Case I}: If $\mathrm{1} \nsubseteq \phi$, then by \ref{theta2n1}, we know that   
    \begin{equation}
    \begin{aligned}\label{52}
    \theta_{\psi,2n}: \Pi_{\phi,\psi} &\rightarrow \Pi_{\phi^+}\\
    \pi &\mapsto \sigma= \theta_{W_{2n}, V_{2n}^{\bullet} ,\psi}(\pi)
    \end{aligned}
    \end{equation}
  is a bijection, where $\phi^+=\left(\phi\otimes \chi_{V}\right)\oplus \chi_{V}$. On the other hand, we have  
	$$
	\mathcal S_{\phi^+}\cong \mathcal {S}_{\phi}\oplus (\mathbb Z/2\mathbb Z)e,
	$$
	where $e$ is the element corresponding to $\chi_{V}\subseteq \phi^+$. This induces an isomorphism 
	\begin{equation}\label{55}
	\ell:\mathcal S_{\phi}\hookrightarrow \mathcal S_{\phi^+}\twoheadrightarrow\bar{ \mathcal S}_{\phi^+}. 
	\end{equation}
	For $\pi\in \Pi_{\phi,\psi}$, we define 
	$$
	\mathcal J^{\psi}_{\mathfrak W_{c}}(\pi)\coloneqq  \ell^*(\mathcal J_{\mathfrak W^\prime_{\psi,c}}(\sigma)), 
	$$
	where $\sigma= \theta_{\psi,2n}(\pi)$ in \ref{52}. 
	
	Note that in this case, the following diagram  
	\[
	\begindc{\commdiag}[500]
	\obj(-1,1)[aa]{$\Pi_{\phi^+}$}
	\obj(1,1)[bb]{$\widehat{\bar{\mathcal{S}}_{\phi^+}}$}
	\obj(-2,0)[cc]{$\Pi_{\phi,\psi}$}
	\obj(2,0)[dd]{$\widehat{\mathcal{S}_\phi}$}
	\mor{aa}{bb}{$\mathcal{J}_{\mathfrak{W}'_{\psi,c}}$}
	\mor{cc}{aa}{$\theta_{\psi,2n}$}
	\mor{bb}{dd}{$\ell^*$}
	\mor{cc}{dd}{$\mathcal{J}_{\mathfrak{W}_{c}}^\psi$}
	\enddc
	\]
	is commutative and every arrow in this diagram is a bijection. 
	
$\bullet$ \underline{Case II}: If $\mathrm{1}\subseteq \phi$. Let $\phi^+=(\phi\otimes\chi_V)\oplus \chi_V$. Then the map 
\begin{equation}\label{501}
\ell:\mathcal S_{\phi}\hookrightarrow \mathcal S_{\phi^+}\twoheadrightarrow\bar{ \mathcal S}_{\phi^+}
\end{equation}
is surjective with kernel isomorphic to $\mathbb Z/2\mathbb Z$. Let $\Pi_{\phi,\psi}^{+}$ be the subset of all representations $\pi\in \Pi_{\phi,\psi}$ such that $\theta_{W_{2n}, V_{2n}^{\bullet} ,\psi}(\pi)\neq 0$. By Proposition \ref{1notinphi}, 
$$\pi\ncong \pi\otimes\det\quad \mbox{for $\pi\in\Pi_{\phi,\psi}$}
$$
and exactly one of $\pi$ and $\pi\otimes \det$ lies in $\Pi_{\phi,\psi}^{+}$, so $\Pi_{\phi,\psi}^{+}$ is half the size of $ \Pi_{\phi,\psi}$. It follows from Proposition \ref{prasad} and Proposition \ref{1notinphi} that the map 
	\begin{equation}\label{56}
	\begin{aligned}
	\theta_{\psi,2n}: \Pi_{\phi,\psi}^{+} &\longrightarrow \Pi_{\phi^+}\\
	\pi &\mapsto \sigma=\theta_{W_{2n}, V_{2n}^{\bullet} ,\psi}(\pi)
	\end{aligned}
	\end{equation}
	is a bijection. For each $\pi\in \Pi_{\phi,\psi}^{+}$, we define 
	\begin{align*}
 \mathcal J^{\psi}_{\mathfrak W_{c}}(\pi)\coloneqq \ell^*(\mathcal J_{\mathfrak W^\prime_{\psi,c}}(\sigma)),
	\end{align*}
	where $\sigma=\theta_{\psi,2n}(\pi)$ in \ref{56}. 

Next we define $\mathcal J^{\psi}_{\mathfrak W_{c}}$ on the other half of $\Pi_{\phi,\psi}$. If $\pi \notin \Pi_{\phi,\psi}^{+}$, then $\pi\otimes \det\in \Pi^+_{\phi,\psi}$ by the conservation relation \ref{con2}. We define 
	\begin{equation*}
	\mathcal J^{\psi}_{\mathfrak W_{c}}(\pi)\coloneqq \mathcal J^{\psi}_{\mathfrak W_{c}}(\pi\otimes\det)\otimes\kappa_{\phi},
	\end{equation*}
	where $\kappa_{\phi}$ is defined by \ref{detmap}. It is easy to check that the map 
	\begin{align*}
	\mathcal J^{\psi}_{\mathfrak W_{c}}: \Pi_{\phi,\psi} \longrightarrow \widehat {\mathcal {S}_{\phi}}
	\end{align*} 
	we construct is an bijection in this case. 

Combining these two cases, we deduce that our construction satisfies Theorem \ref{desideratumall} (2) for tempered representations:  
\begin{proposition}\label{bijection}
Let $\phi\in \Para(\mathrm O(V_{2n}))$. For each Whittaker datum $\mathfrak W_{c}$ of $\bigsqcup_{V_{2n}^{\bullet}}\mathrm O(V_{2n}^\bullet)$, the map 
	\begin{align*}
	\mathcal J^{\psi}_{\mathfrak W_{c}}: \Pi_{\phi,\psi}\longrightarrow \widehat{\mathcal {S}_{\phi}}
	\end{align*} 
is a bijection. 
\end{proposition}
\subsection{From tempered to non-tempered}\label{nontempered}
So far, we have attached an $L$-parameter and a character of component group for $\pi\in \Irrt \mathrm O(V_{2n})$. In this subsection, we will extend this construction to non-tempered representations, using the Langlands classification.  

Let $\pi\in \Irr(\mathrm O(V_{2n}))$, by Langlands classification for $p$-adic groups \cite{MR507262}, \cite{MR2050093}. We know that $\pi$ is the unique irreducible quotient of the standard module 
\begin{equation*}
\Ind_{P}^{\mathrm O(V_{2n})}\left(\tau_{1}|\cdot|_F^{s_1}\otimes \cdots\otimes \tau_{r}|\cdot|_F^{s_r}\otimes\pi_{0} \right),
\end{equation*}
where 
\begin{itemize}
	\item $P$ is a parabolic subgroup of $\mathrm O(V_{2n})$ with Levi component $\GL_{k_{1}}(F)\times \cdots \times \GL_{k_{r}}(F)\times \mathrm O(V_{2n_0})$;
	\item $\tau_{i}$ is an irreducible tempered representation of $\GL_{k_{i}}(F)$ for $i=1,\cdots,r$;
	\item $\pi_0$ is an irreducible tempered representation of $\mathrm O(V_{2n_0})$;
	\item $n=k_1+\cdots k_r +n_0 $ and $s_1>\cdots >s_r>0$. 
\end{itemize}
Then we define
$$
\mathcal L_{\psi}(\pi)\coloneqq \phi_{1}|\cdot|^{s_1}\oplus\cdots \oplus \phi_{r}|\cdot|^{s_r}\oplus \phi_0\oplus \phi_{r}^{\vee}|\cdot|^{-s_r}\oplus \cdots \oplus  \phi_{1}^{\vee}|\cdot|^{-s_1},
$$
where $\phi_{i}$ is the $L$-parameter of $\tau_{i}$ and $\phi_0=\mathcal L_{\psi}(\pi_0)$. The natural embedding $\mathcal S_{\phi_0}\hookrightarrow \mathcal {S}_{\phi}$ is an isomorphism in this case and we identify $\mathcal {S}_{\phi}$ with $\mathcal S_{\phi_{0}}$ via this isomorphism. For a Whittaker datum $\mathfrak W_{c}$ of $\bigsqcup_{V_{2n}^\bullet}\mathrm O(V_{2n}^\bullet)$, we define  
$$
\mathcal J^\psi_{\mathfrak W_{c}}(\pi)\coloneqq \mathcal J^\psi_{\mathfrak W_{c}}(\pi_0).
$$
Since the standard module is unique up to Weyl group conjugation, the maps $\mathcal L_{\psi}$ and $\mathcal J_{\mathfrak W_c}^\psi$ are well defined. It then follows from this construction that 
\begin{proposition}\label{Langlandsclasification}
For each Whittaker datum $\mathfrak W_{c}$ of $\bigsqcup_{V_{2n}^\bullet}\mathrm O(V_{2n}^\bullet)$, the maps $\mathcal L_{\psi}$ and $\mathcal J^\psi_{\mathfrak W_{c}}$ we constructed are compatible with Langlands quotients, i.e., satisfy Theorem \ref{desideratumall} (10). 
\end{proposition}

We first extend Proposition \ref{bijection} from tempered representations to general cases.  
\begin{proposition}\label{bijection2}
	For each Whittaker datum $\mathfrak W_{c}$ of $\bigsqcup_{V_{2n}^{\bullet}}\mathrm O(V_{2n}^\bullet)$, the map 
	\begin{align*}
	\mathcal J^{\psi}_{\mathfrak W_{c}}: \Pi_{\phi,\psi}\longrightarrow \widehat{\mathcal {S}_{\phi}}
	\end{align*} 
	is a bijection. Hence Theorem \ref{desideratumall} (2) holds for our construction.  
\end{proposition}
\begin{proof}
	We have proved this for $\phi\in \Para(\mathrm O(V_{2n}))$ in Proposition \ref{bijection}. The general cases follows from the tempered cases and our construction in this subsection.   
\end{proof}
 We then extend Lemma \ref{respectgamma} and Lemma \ref{respectplancherel} to general cases.  
\begin{proposition}\label{localfactor}
	The map $\mathcal L_{\psi}$ we constructed respect the standard $\gamma$-factor and the Plancherel measures, i.e., it satisfies Theorem \ref{desideratumall} (11) and (12).   
\end{proposition}
\begin{proof}
	We have proved this for tempered representations in Lemma \ref{respectgamma} and Lemma \ref{respectplancherel}. The general case follows from the tempered case and multiplicativity of the standard $\gamma$-factors \cite{MR2192828} \& Plancherel measures \cite[Section 10.2, Appendix B.5]{MR3166215}. 
\end{proof}



\subsection{Some Properties}
We shall prove the LLC we constructed satisfy Theorem \ref{desideratumall} (4) and (5) in this subsection. 


\begin{proposition}\label{changeofwhittakerdatum}
	Let $\pi\in \Pi_{\phi,\psi}$ and $\mathfrak W_{c_1},\mathfrak W_{c_2}$ be two Whittaker data of $\bigsqcup_{V_{2n}^{\bullet}}(\mathrm O(V_{2n}^\bullet))$. We have 
	$$
	\mathcal J^\psi_{\mathfrak W_{c_2}}(\pi)=\mathcal J^\psi_{\mathfrak W_{c_1}}(\pi)\otimes \eta_{\phi\chi_{V},c_2/c_1},
	$$
	where $\phi\chi_{V}=\phi\otimes\chi_{V}$ and $\eta_{\phi\chi_{V},c_2/c_1}$ is defined in \ref{eta}. Hence Theorem \ref{desideratumall} (4) holds for our construction.
\end{proposition}
\begin{proof}
By Proposition \ref{Langlandsclasification}, it is enough to prove the case when  $\phi\in \Para(\mathrm O(V_{2n}))$. Let $\pi\in \Pi_{\phi,\psi}(\mathrm O(V_{2n}^\bullet))$. We divided it into two cases: 
	
	$\bullet$ \underline{Case I}: If $\theta_{W_{2n},V_{2n}^\bullet, \psi}(\pi)\neq 0$. Let $\sigma=\theta_{W_{2n},V_{2n}^\bullet, \psi}(\pi)$ and $\phi^+$ be the $L$-parameter of $\sigma$. Then by Proposition \ref{prasad}, 
	\begin{align*}
	\phi^+=(\phi\otimes\chi_{V})\oplus \chi_{V}.
	\end{align*}
	Moreover, by our construction of $\mathcal J^\psi_{\mathfrak W_{c}}$, we have 
	\begin{align*}
	\mathcal J^\psi_{\mathfrak W_{c_1}}(\pi)=\ell^*(\mathcal J_{\mathfrak W^\prime_{\psi,c_1}}(\sigma)),\\
	\mathcal J^\psi_{\mathfrak W_{c_2}}(\pi)=\ell^*(\mathcal J_{\mathfrak W^\prime_{\psi,c_2}}(\sigma)), 
	\end{align*}
	where $\ell: \mathcal {S}_{\phi}\rightarrow \bar S_{\phi^+}$ is the isomorphism  in \ref{55}. It follows from Theorem \ref{llcsympletic} (4) that 
	\begin{align*}
	\mathcal J_{\mathfrak W^\prime_{\psi,c_2}}(\sigma)= \mathcal J_{\mathfrak W^\prime_{\psi,c_1}}(\sigma)\otimes \eta_{\phi^+,c_2/c_1}. 
	\end{align*}
	Hence 
	\begin{align*}
	\mathcal J^\psi_{\mathfrak W_{c_2}}(\pi)=&\ell^*(\mathcal J_{\mathfrak W^\prime_{\psi,c_2}}(\sigma))=\ell^*(\mathcal J_{\mathfrak W^\prime_{\psi,c_1}}(\sigma)\otimes \eta_{\phi^+,c_2/c_1})
	=\mathcal J^\psi_{\mathfrak W_{c_1}}(\pi)\otimes\eta_{\phi\chi_{V},c_2/c_1}. 
	\end{align*}
	
	$\bullet$ \underline{Case II}: If $\theta_{W_{2n},V_{2n}^\bullet, \psi}(\pi)=0$, then $\theta_{W_{2n},V_{2n}^\bullet, \psi}(\pi\otimes\det)\neq 0$ by the conservation relation \ref{con2}. Hence by Case I, 
	\begin{align}\label{137}
	\mathcal J^\psi_{\mathfrak W_{c_2}}(\pi\otimes\det)=\mathcal J^\psi_{\mathfrak W_{c_1}}(\pi\otimes
	\det)\otimes\eta_{\phi\chi_{V},c_2/c_1}. 
	\end{align}
	On the other hand, by our construction of $\mathcal J^\psi_{\mathfrak W_{c}}$, we have 
	\begin{equation}\label{136}
	\begin{aligned}
	\mathcal J^\psi_{\mathfrak W_{c_1}}(\pi)=\mathcal J^\psi_{\mathfrak W_{c_1}}(\pi\otimes\det)\otimes\kappa_{\phi},\\
	\mathcal J^\psi_{\mathfrak W_{c_2}}(\pi)=\mathcal J^\psi_{\mathfrak W_{c_2}}(\pi\otimes\det)\otimes\kappa_{\phi}. 
	\end{aligned}
	\end{equation} 
	 So by \ref{137} and \ref{136}, we deduce  
	\begin{align*}
	\mathcal J^\psi_{\mathfrak W_{c_2}}(\pi)=\mathcal J^\psi_{\mathfrak W_{c_1}}(\pi)\otimes\eta_{\phi\chi_{V},c_2/c_1}
	\end{align*}
	as desired. 
\end{proof}
\begin{proposition}\label{tempered}
	Let $\pi\in \Irr(\mathrm O(V_{2n}))$. Then we have  
	\begin{enumerate}[(i)]
		\item $\pi\in \Irrt(\mathrm O(V_{2n}))$ iff $\phi=\mathcal L_\psi(\pi)\in \Para(\mathrm O(V_{2n}))$; 
		\item $\pi$ is a discrete series representation iff $\phi=\mathcal L_\psi(\pi)\in \Phi_{\disc}(\mathrm O(V_{2n}))$. 
	\end{enumerate} 
Hence Theorem \ref{desideratumall} (5) holds for our construction.
\end{proposition}
\begin{proof}
	(i) automatically follows from our construction. We then prove (ii). First, we prove $\pi\in \Pi_{\phi,\psi}$ is a discrete series representation if $\phi$ is a discrete parameter. We prove this according to  $\mathrm 1\subseteq \phi$ or not.  

$\bullet$ \underline{Case I}: If $\mathrm 1\nsubseteq \phi$, then by Proposition \ref{prasad} and Corollary \ref{1inphi}, we have  $$\sigma=\theta_{W_{2n},V_{2n},\psi}(\pi)\neq 0$$ 
and the $L$-parameter of $\sigma$ is $$\phi^+=(\phi\otimes\chi_{V})\oplus\chi_{V}.$$
Since $\phi$ is a discrete parameter and $1\nsubseteq \phi$, we deduce that $\phi^+$ is also a discrete parameter, then $\sigma$ is a discrete series representation of $\SP(W_{2n})$ by Theorem \ref{llcsympletic} (5). It then follows from Lemma \ref{thetadiscrete} that  $\pi=\theta_{V_{2n},W_{2n},\psi}(\sigma)$ is a discrete series representation of $\mathrm O(V_{2n})$. 
	
$\bullet$ \underline{Case II}: If $\mathrm 1\subseteq \phi$, then by Proposition \ref{1notinphi}, exactly one of 
	$$\theta_{W_{2n-2},V_{2n},\psi}(\pi) \quad \mbox{and}\quad \theta_{W_{2n-2},V_{2n},\psi}(\pi\otimes\det)$$
is nonzero. Since $\pi$ is a discrete series representation iff $\pi\otimes\det$ is a discrete series representation, without loss of generality, we may assume that  $$\theta_{W_{2n-2},V_{2n},\psi}(\pi)\neq 0 .$$ 
Let $\sigma=\theta_{W_{2n-2},V_{2n},\psi}(\pi)$. Then by Proposition \ref{prasad}, the $L$-parameter of $\sigma$ is 
$$\phi^-=(\phi\otimes\chi_{V})-\chi_{V}.$$
Since $\phi$ is a discrete parameter, so is $\phi^-$. So it follows from Theorem \ref{llcsympletic} (5) that $\sigma$ is a discrete series representation of $\SP(W_{2n-2})$. Note that $\mathrm{1}\oplus \mathrm{1} \nsubseteq \phi$ since $\phi$ is a discrete parameter, hence $\chi_{V}\nsubseteq \phi^+$. It follows from Corollary \ref{1inphi} that $$\theta_{V_{2n-2},W_{2n-2},\psi} (\sigma)=0.$$
Then by Lemma \ref{thetadiscrete}, $\pi=\theta_{V_{2n},W_{2n-2},\psi}(\sigma)\neq 0$ is a discrete series representation of $\mathrm O(V_{2n})$. 

Next, for $\pi\in \Irrt(\mathrm O(V_{2n}))$, we prove $\phi=\mathcal L_\psi(\pi)$ is a discrete parameter if $\pi$ is a discrete series representation. We also prove this according to $\mathrm 1\subseteq \phi$ or not: 
	
$\bullet$ \underline{Case I}: If $\mathrm 1\nsubseteq \phi$, then by Corollary \ref{1inphi}, we have 
$$\theta_{W_{2n-2},V_{2n},\psi} (\pi)=0\quad \mbox{and}\quad \theta_{W_{2n},V_{2n},\psi}(\pi)\neq 0.$$
Let $\sigma=\theta_{W_{2n},V_{2n},\psi}(\pi)$ and $\phi^+$ be the $L$-parameter of $\sigma$. Then $\sigma$ is a discrete series representation of $\SP(W_{2n})$ by Lemma \ref{thetadiscrete}. Hence $\phi^+$ is a discrete parameter by Theorem \ref{llcsympletic} (5). On the other hand, it follows from Proposition \ref{prasad} that 
$$\phi=(\phi^+\otimes\chi_{V})-\chi_{V}.$$
So $\phi$ is also a discrete parameter. 
	
	
$\bullet$ \underline{Case II}: If $\mathrm 1\subseteq \phi$, then by Proposition \ref{1notinphi}, exactly one of  $$\theta_{W_{2n-2},V_{2n},\psi}(\pi)\quad\mbox{and}\quad \theta_{W_{2n-2},V_{2n},\psi}(\pi\otimes\det)$$
is non-zero. Since 
$$
\mathcal L_\psi(\pi)=\mathcal L_\psi(\pi\otimes\det),
$$
without loss of generality, we may assume
\begin{align}\label{1231}
\theta_{W_{2n-2},V_{2n},\psi}(\pi\otimes\det)= 0\quad \mbox{and}\quad \theta_{W_{2n-2},V_{2n},\psi}(\pi)\neq 0.
\end{align} 
Let $\sigma=\theta_{W_{2n-2},V_{2n},\psi}(\pi)$ and $\phi^-$ be the $L$-parameter of $\sigma$. It follows from Lemma \ref{thetadiscrete} that $\sigma$ is a discrete series representation of $\SP(W_{2n-2})$. Then by Theorem \ref{llcsympletic} (5), $\phi^-$ is a discrete  parameter. Note that by Proposition \ref{prasad}, we have
\begin{align}\label{1234}
\phi=(\phi^-\otimes\chi_{V})\oplus \mathrm{1}.
\end{align}
We shall prove that $\chi_{V}\nsubseteq \phi^-$, which will imply $\phi$ is a discrete parameter by \ref{1234}. We prove it by contradiction. Suppose $\chi_{V}\subseteq \phi^-$, then by the conservation relation \ref{con2}, Corollary \ref{1notinphi} and \ref{1231}, we have 
$$\theta_{V_{2n-2},W_{2n-2},\psi} (\sigma)\neq 0.$$
But by Lemma \ref{thetadiscrete}, this will imply $\pi=\theta_{V_{2n},W_{2n-2},\psi} (\sigma)$ is a tempered but not discrete series representation, which contradicts with our assumption. This completes the proof. 
\end{proof}


\section{Variation of $\psi$}\label{changeofpsi}
 We shall prove that $\mathcal L_\psi$ and $\mathcal J^\psi_{\mathfrak W_c}$ are independent of the choice of $\psi$ in this section. Note that every non-trivial additive character of $F$ is of the form $\psi_a$ for some $a\in F^\times$. 
 
 The method is to study the behaviour of theta correspondence under the change of $\psi$. More precisely, we would like to compare $\theta_{W_{2n}, V_{2n}, \psi_a}(\pi)$ and $\theta_{W_{2n},V_{2n},\psi}(\pi)$ for $\pi\in \Irr(\mathrm O(V_{2n}))$ and $a\in F^\times$. For any $\sigma\in \Irr(\SP(W_{2n}))$, let $\sigma^{\delta_a}\in \Irr(\SP(W_{2n}))$ be defined in \ref{113}.   
\begin{lemma}\label{scaling}
	Let $\pi\in \Irr \left(\mathrm O(V_{2n})\right)$ and $a\in F^{\times}$. We have  
	$$
	\theta_{W_{2n}, V_{2n}, \psi_a}(\pi)=
	\left( \theta_{W_{2n},V_{2n},\psi}(\pi)\right)^{\delta_a}. 
	$$
\end{lemma}
\begin{proof}
	See \cite[II Corollary 6.2]{kudla1996notes} and \cite[IV Proposition 1.9]{kudla1996notes}
\end{proof} 

 
\begin{proposition}\label{notdependonpsi}
	For $\pi\in \Irr\left( \mathrm O(V_{2n})\right)$ and $a\in F^{\times}$, we have 
	\begin{enumerate}[(i)]
		\item $\mathcal L_{\psi}(\pi)=\mathcal L_{\psi_a}(\pi);$
		\item $\mathcal J^\psi_{\mathfrak W_{c}}(\pi)= \mathcal J^{\psi_a}_{\mathfrak W_{c}}(\pi)$. 	
	\end{enumerate}	
\end{proposition}
\begin{proof}
Since the LLC we construct is compatible with the Langlands classifications, it is enough to prove these for tempered representations. Hence we may assume that $\pi\in \Irrt \left(\mathrm O(V_{2n})\right)$.  
We divided into two cases: 

$\bullet$ \underline{Case I}: If $\theta_{W_{2n},V_{2n},\psi}(\pi)\neq 0$. Let $\sigma=\theta_{W_{2n},V_{2n},\psi}(\pi)$. Then by Lemma \ref{scaling}, we have 
\begin{equation*}
\begin{aligned}
\theta_{W_{2n}, V_{2n},  \psi_a}(\pi)=\left(\theta_{W_{2n}, V_{2n}, \psi}(\pi)\right)^{\delta_a}=\sigma^{\delta_a}. 
\end{aligned}
\end{equation*}
It follows from \ref{103} that the $L$-parameters for $\sigma$ and $\sigma^{\delta_a}$ are the same. Then by our constructions of $\mathcal L_\psi$ and $\mathcal L_{\psi_a}$, we have 
$$\mathcal L_{\psi}(\pi)=\mathcal L_{\psi_a}(\pi).$$ Next we consider the map $\mathcal J^{\psi}_{\mathfrak W_{c}}$. By our constructions of $\mathcal J^{\psi}_{\mathfrak W_{c}}$ and $\mathcal J^{\psi_a}_{\mathfrak W_{c}}$, we have  
		\begin{equation}\label{126}
		\begin{aligned}
		\mathcal J^\psi_{\mathfrak W_{c}}(\pi)=&\ell^*\left(\mathcal J_{\mathfrak W^\prime_{\psi,c}}(\sigma)\right)\\
		\mathcal J^{\psi_a}_{\mathfrak W_{c}}(\pi)=&\ell^*\left(\mathcal J_{\mathfrak W^\prime_{\psi_a,c}}(\sigma^{\delta_a})\right),
		\end{aligned}
		\end{equation}
		where $\ell: \mathcal {S}_{\phi}\rightarrow \bar {\mathcal S}_{\phi^+}$ is defined in \ref{55}. By \ref{103}, we have 
		\begin{equation}\label{124}
		\begin{aligned}
		\mathcal J_{\mathfrak W^\prime_{\psi_a,c}}(\sigma^{\delta_a})=\mathcal J_{\mathfrak W^\prime_{\psi,c}}(\sigma).
		\end{aligned}
		\end{equation}
		So $\mathcal J^\psi_{\mathfrak W_{c}}(\pi)= \mathcal J^{\psi_a}_{\mathfrak W_{c}}(\pi)$ by \ref{126} and \ref{124}. 
		
		
$\bullet$ \underline{Case II}:  If $\theta_{W_{2n}, V_{2n}, \psi}(\pi)=0$, then $\theta_{W_{2n}, V_{2n}, \psi}(\pi\otimes\det)\neq 0$ by the conservation relation \ref{con2}. Hence, by Case I, we have 
	\begin{align*}
	\mathcal L_{\psi}(\pi\otimes\det)=&\mathcal L_{\psi_a}(\pi\otimes\det),\\
	\mathcal J^{\psi}_{\mathfrak W_{c}}(\pi\otimes\det)=&\mathcal J^{\psi_a}_{\mathfrak W_{c}}(\pi\otimes\det) .
	\end{align*}
On the other hand, it follows from Proposition \ref{det} and our construction of $\mathcal J^{\psi}_{\mathfrak W_{c}}$ that 
\begin{align*}
	\mathcal L_\psi(\pi\otimes\det)=&\mathcal L_\psi(\pi),\\
	\mathcal J^\psi_{\mathfrak W_{c}}(\pi\otimes\det)=& \mathcal J^\psi_{\mathfrak W_c}(\pi)\otimes\kappa_{\phi}. 	
\end{align*}	
Hence (i) and (ii) also hold for Case II. 	
\end{proof}

Since the map $\mathcal L_\psi$ and $\mathcal J^\psi_{\mathfrak W_{c}}$ does not depend on the choice of $\psi$, we may drop the symbol $\psi$ and denote them by $\mathcal L$ and $\mathcal J_{\mathfrak W_{c}}$. These give the desired LLC for even orthogonal groups. 


So far we have constructed the map $\mathcal L$ and $\mathcal J_{\mathfrak W_{c}}$ for each Whittaker datum $\mathfrak W_{c}$ of $\bigsqcup_{V_{2n}^{\bullet}}\mathrm O(V_{2n})$. We also proved this construction satisfies Theorem \ref{desideratumall} (1), (2), (4), (5), (10), (11), (12). 
 








\section{Preparations for local intertwining relation }\label{LIR}
Our next goal is to prove our construction satisfies the local intertwining relation. In this section, we do some preparations. We first recall the definition of normalized intertwining operators, and then construct an important equivariant map, which is key to the proof.  
 

\subsection{Normalized intertwining operators}\label{normalizingintertwing}
In this subsection, we define the normalized intertwining operators. We mainly follow \cite{MR3135650}, \cite{MR3708200} and \cite{MR3788848}.

Let $V=V_{2n}$ be a $2n$-dimensional orthogonal space with discriminant $d\in F^{\times}/F^{\times 2}$ and discriminant character $\chi_{V}$. Let $W=W_{2n}$ be a $2n$-dimensional symplectic space. Let $r$ be the Witt index of $V$ and $k$ be a positive integer with $k\leq r$. As in section \ref{sectionparabolic}, we put 
$$V=X\oplus V_0\oplus X^{\vee}, \quad W=Y\oplus W_0\oplus Y^{\vee}$$
with $X=X_{k}, X^{\vee}=X_{k}^{\vee}$ and $Y=Y_{k}, Y^{\vee}=Y_{k}^{\vee}$. Hence $\dim(V_0)=\dim(W_0)=2n-2k$. Let $P=P_{k}=M_{P} U_{P}$ and $Q=Q_{k}=M_{Q} U_{Q}$ be the parabolic subgroups defined in section \ref{sectionparabolic} such that 
$$M_{P} \cong \GL(X) \times \mathrm O(V_0), \quad M_{Q} \cong \GL(Y) \times \SP(W_0).$$
Using the basis $\left\{v_{1}, \ldots, v_{k}\right\}$ of $X$ (resp. $\left\{w_{1}, \ldots, w_{k}\right\}$ of $Y$), we identify $\GL(X)$ (resp. $\GL(Y)$) with $\GL_k(F)$. Hence we can define an isomorphism $i: \GL(X) \rightarrow \GL(Y) $ via these identifications.  

Let $\tau$ be an irreducible tempered representation of $\GL_{k}(F)$ on a space $\mathscr{V}_{\tau}$ with a central character $\omega_{\tau}$. We may regard $\tau$ as a representation of $\GL(X)$ or $\GL(Y)$ via the above identifications. For any $s \in \mathbb{C}$, we realize the representation $\tau_{s} \coloneqq \tau \otimes|\det|_{F}^{s}$ on $\mathscr{V}_{\tau}$ by setting $\tau_{s}(a) v \coloneqq |\det(a)|_{F}^{s} \tau(a) v$ for all $v \in \mathscr{V}_{\tau}$ and $a \in \GL_{k}(F)$. Let $\pi_0$ (resp. $\sigma_0$) be an irreducible tempered representation of $\mathrm{O}(V_0)$ (resp. $\SP(W_0)$) on a space $\mathscr{V}_{\pi_0}$ (resp. $\mathscr{V}_{\sigma_0}$). We consider the induced representations 
$$\Ind_{P}^{\mathrm O\left(V\right)}\left(\tau_{s} \otimes \pi_0 \right) \quad \mbox{and} \quad \Ind_{Q}^{\SP\left(W\right)}\left(\tau_{s} \otimes \sigma_0 \right)$$
of $\mathrm O(V)$ and $\SP(W)$. They are realized on the spaces of smooth functions 
$$\Psi_{s} : \mathrm O(V) \rightarrow \mathscr{V}_{\tau} \otimes \mathscr{V}_{\pi_0}\quad  \mbox{and} \quad \Phi_{s}: \operatorname{Sp}\left(W\right) \rightarrow \mathscr{V}_{\tau} \otimes \mathscr{V}_{\sigma_0}$$
such that
\begin{align*} \Psi_{s}\left(u_{P} m_{P}(a) h_0 h\right) &=|\det(a)|_{F}^{s+\rho_{P}} \tau(a) \pi_0(h_0) \Psi_{s}\left(h\right), \\ 
\Phi_{s}\left(u_{Q} m_{Q}\left(a^{\prime}\right) g_0 g\right) &=\left|\operatorname{det}\left(a^{\prime}\right)\right|_{F}^{s+\rho_{Q}} \tau\left(a^{\prime}\right) \sigma_0(g_0) \Phi_{s}\left(g\right)
\end{align*}
for any $u_{P} \in U_{P}, a\in \GL(X), h_0\in \mathrm{O}(V_0), h \in \mathrm{O}(V)$ (resp. $u_{Q} \in U_{Q}, a^{\prime} \in \mathrm{GL}(Y), g_0 \in \SP(W_0),g \in \SP(W)$). Let $A_{P}$ (resp. $A_{Q}$) be the split component of the center of $M_{P}$ (resp. $M_{Q}$) and $W\left(M_{P}\right)=N_{\mathrm O(V)}(A_P)/M_P$
(resp. $W(M_{Q})=N_{\SP(W)}(A_Q)/M_Q$) be the relative Weyl group for $M_{P}$ (resp. $M_{Q}$). Note that 
$$W(M_{P}) \cong W(M_{Q}) \cong \mathbb{Z} / 2 \mathbb{Z}.$$ 
We denote by $w$ (resp. $ w^{\prime}$) the non-trivial element in $W(M_{P})$ (resp. $W(M_{Q}) $). For any representative $\widetilde{w} \in \mathrm{O}(V)$ of $w$ (resp. $\widetilde{w}^{\prime}\in \SP(W)$ of $w^{\prime}$), we define an unnormalized intertwining operator 
\begin{align*}
\mathcal{M}\left(\widetilde{w}, \tau_{s} \otimes \pi_0\right) : \Ind_{P}^{\mathrm{O}(V)}\left(\tau_{s} \otimes \pi_0 \right) &\longrightarrow \operatorname{Ind}_{P}^{\mathrm{O}(V)}\left(w\left(\tau_{s} \otimes \pi_0 \right)\right),\\
\mathcal{M}\left(\widetilde{w}^{\prime}, \tau_{s} \otimes \sigma \right) : \operatorname{Ind}_{Q}^{\mathrm{Sp}(W)}\left(\tau_{s} \otimes \sigma_0 \right) &\longrightarrow \operatorname{Ind}_{Q}^{\mathrm{Sp}(W)}\left(w^{\prime}\left(\tau_{s} \otimes \sigma_0 \right)\right)
\end{align*}
by (the meromorphic continuations of) the integrals
\begin{align*}
 \mathcal{M}\left(\widetilde{w}, \tau_{s} \otimes \pi_0 \right) \Psi_{s}\left(h\right) &= \int_{U_{P}} \Psi_{s}\left(\widetilde{w}^{-1} u_{P} h\right) d u_{P}, \\ 
\mathcal{M}\left(\widetilde{w}^{\prime}, \tau_{s} \otimes \sigma_0 \right) \Phi_{s}\left(g\right) &=\int_{U_{Q}} \Phi_{s}\left(\widetilde{w}^{\prime-1} u_{Q} g\right) d u_{Q}, \end{align*}
where $du_P$ and $du_Q$ are the Haar measures given in \cite[\S 6.3]{MR3788848}) and $w\left(\tau_{s} \otimes \pi_0 \right)$ (resp. $w^{\prime}\left(\tau_{s} \otimes \sigma_0 \right) $) is the representation of $M_{P}$ on $\mathscr{V}_{\tau} \otimes \mathscr{V}_{\pi_0}$ (resp. $M_{Q}$ on $\mathscr{V}_{\tau} \otimes \mathscr{V}_{\sigma_0} $) given by 
\begin{align*}
w\left(\tau_{s} \otimes \pi_0 \right)\left(m_{P}\right)&=\left(\tau_{s} \otimes \pi_0 \right)\left(\widetilde{w}^{-1} m_{P} \widetilde{w}\right),\\
w^{\prime}\left(\tau_{s} \otimes \sigma_0 \right)\left(m_{Q}\right) &=\left(\tau_{s} \otimes \sigma_0 \right)\left(\widetilde{w}^{\prime-1} m_{Q} \widetilde{w}^{\prime}\right)
\end{align*}
for $m_{P} \in M_{P}$ (resp. $m_{Q}\in M_Q$). 

We shall normalize the intertwining operators $\mathcal{M}\left(\widetilde{w}, \tau_{s} \otimes \pi_0\right)$ and $\mathcal{M}\left(\widetilde{w}^{\prime}, \tau_{s} \otimes \sigma \right)$ depending on the choice of Whittaker data. Having fixed the additive character $\psi$ of F, for any $c\in F^\times /F^{\times 2}$, we use the Whittaker datum $\mathfrak W_{c}$ of $\bigsqcup_{V_{2n}^{\bullet}}\mathrm O(V_{2n}^\bullet)$ and $\mathfrak W^\prime_{\psi,1}$ of $\SP(W_{2n})$ as in subsection \ref{whittaker}.  The normalized intertwining operators 
\begin{align*}
\mathcal R_{\mathfrak W_{c}}(w, \tau_s \otimes \pi_0) &: \Ind_{P}^{\mathrm{O}(V)}\left(\tau_s\otimes \pi_0\right)\rightarrow  \operatorname{Ind}_{P}^{\mathrm{O}(V)}\left(w(\tau_s \otimes \pi_0)\right),\\
\mathcal R_{\mathfrak W^\prime_{\psi,1}}\left(w^{\prime}, \tau_s \otimes \sigma_0\right) &: \operatorname{Ind}_{Q}^{\SP(W)}\left(\tau_s \otimes \sigma_0\right) \rightarrow \operatorname{Ind}_{Q}^{\SP(W)}\left(w^\prime(\tau_s \otimes \sigma_0)\right)
\end{align*}
will be defined as follows: 
\begin{align*}
\mathcal R_{\mathfrak W_{c}}(w, \tau_s \otimes \pi_0) \Psi_s(h) &=\epsilon(V)^{k} \cdot \chi_{V}(c)^{k}\cdot |c|_F^{k\rho_P}\cdot r(w,\tau_s\otimes
\pi_0)^{-1}\cdot \mathcal{M}(\widetilde{w}_{c}, \tau_s \otimes \pi_0) \Psi_s(h), \\ 
\mathcal R_{\mathfrak W^\prime_{\psi,1}}(w^\prime, \tau_s \otimes \sigma_0) \Phi_s(g) &=r(w^\prime,\tau_s\otimes
\sigma_0)^{-1}\cdot \mathcal{M}(\widetilde{w}_1^\prime, \tau_s \otimes \sigma_0) \Phi_s(g),
\end{align*}
where 
\begin{itemize}
 \item $\epsilon(V)$ is the normalized Hasse-Witt invariant of $V=V_{2n}$.  
\item $\widetilde{w}_{c}$ and $\widetilde{w}_1^{\prime}$ are defined by
\begin{equation}\label{131}
\begin{aligned}
\widetilde{w}_{c}&=w_{P} \cdot m_{P}\left(c \cdot a\right) \cdot\left((-1)^{k} \mathbf{1}_{V_0}\right),\\
\widetilde{w}_1^{\prime}&=w_{Q} \cdot m_{Q}\left( a^{\prime}\right) \cdot\left((-1)^k \mathbf{1}_{W_0}\right),
\end{aligned}
\end{equation}
where $a \in \mathrm{GL}_{k}(F) \cong \mathrm{GL}(X)$ (resp. $a^{\prime} \in \mathrm{GL}_{k}(F) \cong \mathrm{GL}(Y)$) is given by
$$
a=\left(\begin{array}{ccc}
{} &{}& (-1)^{n-k+1} \\ 
{} &{\iddots} & {}\\
{(-1)^n}&{}& {}
\end{array}\right), \quad 
a^{\prime}=\left(\begin{array}{ccc}
{} &{}& (-1)^{n-k+1} \\ 
{} &{\iddots} & {}\\
{(-1)^n}&{}& {}
\end{array}\right) .
$$
Note that  
\begin{equation}\label{determinantminus1}
\det(\widetilde{w}_{c})=\det(w_{P})\cdot \det\left(m_{P}(c \cdot a)\right)\cdot \det\left((-1)^{k} \mathbf{1}_{V_0}\right))=(-1)^{k}. 
\end{equation}
 \item  Following \cite{MR3135650}, \cite[\S 6.2]{MR3788848} and \cite[\S 3.7]{MR3708200}, $r\left(w, \tau_{s} \otimes \pi_0\right)$ and $r\left(w^{\prime}, \tau_{s} \otimes \sigma_0\right)$ are 
    defined as 
    \begin{equation}\label{normalizationfactor}
	\begin{aligned}
	r\left(w, \tau_{s} \otimes \pi_0\right)&=\lambda(E / F, \psi)^{k}\times \frac{L\left(s, \phi_{\tau} \otimes \phi_{\pi_0}\right)}{\varepsilon\left(s, \phi_{\tau} \otimes \phi_{\pi_0}, \psi\right) L\left(1+s, \phi_{\tau} \otimes \phi_{\pi_0}\right)}\\ 
	&\quad \times \frac{L\left(2 s, \wedge^{2} \circ \phi_{\tau}\right)}{\varepsilon\left(2 s, \wedge^{2} \circ \phi_{\tau}, \psi\right) L\left(1+2 s, \wedge^{2} \circ \phi_{\tau}\right)},\\
	r\left(w^\prime, \tau_{s} \otimes \sigma_0\right)
	&= \frac{L\left(s, \phi_{\tau} \otimes \phi_{\sigma_0}\right)}{\varepsilon\left(s, \phi_{\tau} \otimes \phi_{\sigma_0}, \psi\right) L\left(1+s, \phi_{\tau} \otimes \phi_{\sigma_0}\right)} \\
	& \quad \times \frac{L\left(2 s, \wedge^{2} \circ \phi_{\tau}\right)}{\varepsilon\left(2 s, \wedge^{2} \circ \phi_{\tau}, \psi\right) L\left(1+2 s, \wedge^{2} \circ \phi_{\tau}\right)},&
	\end{aligned}
	\end{equation}
	where $\lambda(E/F, \psi)$ is the Langlands $\lambda$-factor associated to $E=F(\sqrt{d})$, $\phi_{\tau}, \phi_{\sigma_0}$ and $\phi_{\pi_0}$ are the $L$-parameters for $\tau,\sigma_0$ and $\pi_0$ respectively. Note that 
	\begin{align}\label{12345}
	\lambda(E/F, \psi)^2=\chi_{V}(-1).
	\end{align}  
\end{itemize}
\begin{remark}
\begin{enumerate}
	\item The representatives $\widetilde{w}_c$ and $\widetilde{w}^\prime_1$ are choosen according to an $F$-splittings of $\mathrm O(V)$ and $\SP(W)$; see \cite[\S 6.2]{MR3788848} for details. 
	\item In Arthur \cite{MR3135650}, the Haar measures of $u_P$ and $u_Q$ are choosen with respect to an $F$-splittings of $\mathrm O(V)$ and $Sp(W)$. Readers can consult \cite[\S 6.3]{MR3788848} for details. The factors $|c|^{k\rho_{P}}$ appears because of the different choices of Haar measures. 
\end{enumerate}
\end{remark}
\begin{lemma}\label{homomorphicatsequal0}
	$ \mathcal{R}_{\mathfrak W_{c}}\left(w, \tau_{s} \otimes \pi_0\right) $ and  $\mathcal{R}_{\mathfrak W^\prime_{\psi,1}}\left(w^{\prime}, \tau_{s} \otimes \sigma_0 \right)$ are holomorphic at $s=0$ and 
	\begin{align*}
	\mathcal{R}_{\mathfrak W_{c}}\left(w, w(\tau_{s} \otimes \pi_0)\right)&\circ \mathcal{R}_{\mathfrak W_{c}}\left(w, \tau_{s} \otimes \pi_0\right)=1, \\
	\mathcal{R}_{\mathfrak W^\prime_{\psi,1}}\left(w^\prime, w^\prime(\tau_{s} \otimes \sigma_0)\right)&\circ \mathcal{R}_{\mathfrak W^\prime_{\psi,1}}\left(w^\prime, \tau_{s} \otimes \sigma_0\right)=1.
	\end{align*}
\end{lemma}
\begin{proof}
	The case when $G$ is symplectic group or quasi-split even orthogonal group is proved in \cite[Proposition 2.3.1]{MR3135650}. When $G$ is even orthogonal group, we will give a proof in Appendix \ref{appendixa1} based on the explicit formula for Plancherel measures. 
\end{proof}

Now assume that $w(\tau \otimes \pi_0)\cong \tau\otimes \pi_0$ and $w^\prime(\tau \otimes \sigma_0)\cong \tau\otimes \sigma_0$, both of which are equivalent to $\tau^\vee \cong \tau$. We take  
\begin{align*}
\mathcal{A}_{w}=\mathcal A_{\tau,w}\otimes\mathbf 1_{\mathscr V_{\pi_0}} : \mathscr{V}_{\tau} \otimes \mathscr{V}_{\pi_0} \rightarrow \mathscr{V}_{\tau} \otimes \mathscr{V}_{\pi_0},\\
\mathcal{A}_{w^\prime}=\mathcal A_{\tau,w^\prime}\otimes\mathbf 1_{\mathscr V_{\sigma_0}} : \mathscr{V}_{\tau} \otimes \mathscr{V}_{\sigma_0} \rightarrow \mathscr{V}_{\tau} \otimes \mathscr{V}_{\sigma_0}
\end{align*}
be the unique intertwining isomorphism such that 
\begin{itemize}
	\item for any $m_P \in M_{P}, m_Q\in M_Q$, 
	\begin{align*}
	\mathcal{A}_{w} \circ w(\tau \otimes \pi_0)(m_P)&=(\tau \otimes \pi_0)(m_P) \circ \mathcal{A}_{w}\\ 
	\mathcal{A}_{w^\prime} \circ w^\prime(\tau \otimes \sigma_0)(m_Q)&=(\tau \otimes \sigma_0)(m_Q) \circ \mathcal{A}_{w^\prime};
	\end{align*}
\item $\Lambda \circ \mathcal{A}_{\tau,w}=\Lambda$ and $\Lambda \circ \mathcal{A}_{\tau,w^\prime}=\Lambda$ for a fixed  non-zero Whittaker functional $\Lambda$ on $\mathscr V_{\tau}$ with respect to the Whittaker datum $\left(B_{k}, \psi_{U_{k}}\right),$ where $B_{k}$ is the Borel subgroup consisting of upper triangular matrices in $\mathrm{GL}_{k}(F)$ and $\psi_{U_{k}}$ is the generic character of the unipotent radical $U_{k}$ of $B_{k}$ given by $\psi_{U_{k}}(x)=\psi\left(x_{1,2}+\cdots+x_{k-1, k}\right)$. Here we identify $\GL(X)$ and $\GL(Y)$ with $\GL_k(F)$ via the identification in the beginning of this subsection. 
\end{itemize}
Note that $\mathcal{A}_{w}^2=\mathbf 1_{\mathscr V_{\tau}\otimes \mathscr V_{\pi_0}}$ and $\mathcal{A}_{w^\prime}^2=\mathbf 1_{\mathscr V_{\tau}\otimes \mathscr V_{\sigma_0}}$. We define the self-intertiwining operators 
\begin{align*}
R_{\mathfrak W_{c}}(w, \tau \otimes \pi_0) &: \Ind_{P}^{\mathrm{O}(V)}(\tau\otimes \pi_0)\rightarrow  \Ind_{P}^{\mathrm{O}(V)}(\tau \otimes \pi_0),\\
R_{\mathfrak W^\prime_{\psi,1}}(w^{\prime}, \tau \otimes \sigma_0) &: \Ind_{Q}^{\SP(W)}(\tau \otimes \sigma_0) \rightarrow \Ind_{Q}^{\SP(W)}(\tau \otimes \sigma_0)
\end{align*}
by 
\begin{equation}\label{100}
\begin{aligned} 
R_{\mathfrak W_{c}}(w, \tau \otimes \pi_0) \Psi(h) &=\mathcal{A}_{w}\left(\mathcal{R}_{\mathfrak W_{c}}(w, \tau_s \otimes \pi_0) \Psi_s(h)|_{s=0}\right),\\ 
R_{\mathfrak W^\prime_{\psi,1}}(w^{\prime}, \tau \otimes \sigma_0) \Phi(g) &=\mathcal{A}_{w^{\prime}}\left(\mathcal{R}_{\mathfrak W^\prime_{\psi,1}}(w^{\prime}, \tau_s \otimes \sigma_0 ) \Phi_s(g)|_{s=0}\right),
\end{aligned}
\end{equation}
where $\Psi_s\in \Ind_{P}^{\mathrm{O}(V)}\left(\tau_s\otimes \pi_0\right)$ and $\Phi_s\in \Ind_{Q}^{\SP(W)}\left(\tau_s\otimes \sigma_0\right)$ are holomorphic sections satisfying $\Psi_s|_{s=0}=\Psi$ and $\Phi_s|_{s=0}=\Phi$ respectively. By Lemma \ref{homomorphicatsequal0} and our construction, we have  
\begin{equation}\label{twiceequalone}
\begin{aligned}
R_{\mathfrak W_{c}}(w, \tau \otimes \pi_0)^2&=1,\\
R_{\mathfrak W^\prime_{\psi,1}}(w^{\prime}, \tau \otimes \sigma_0)^2&=1.
\end{aligned}
\end{equation}

\begin{remark}
\begin{enumerate}
	\item The definition of the self-intertwining operator $R_{\mathfrak W_{c}}(w, \tau \otimes \pi_0)$ involves the additive character $\psi$, but an easy computation shows that different choices of  $\psi$ give the same $R_{\mathfrak W_{c}}(w, \tau \otimes \pi_0)$, so  $R_{\mathfrak W_{c}}(w, \tau \otimes \pi_0)$ only depends on the choice of Whittaker datum $\mathfrak W_{c}$ of $\bigsqcup_{V_{2n}^{\bullet}} \mathrm O(V_{2n}^\bullet)$. Similarly, one can show 
	\begin{align*}
	R_{\mathfrak W^\prime_{\psi_a,1}}(w^{\prime}, \tau \otimes \sigma_0)=R_{\mathfrak W^\prime_{\psi,1}}(w^{\prime}, \tau \otimes \sigma_0)\cdot \omega_{\tau}(a)
	\end{align*}
	for any $a\in F^\times$. In particular, 
	\begin{align*}
	R_{\mathfrak W^\prime_{\psi_a,1}}(w^{\prime}, \tau \otimes \sigma_0)=R_{\mathfrak W^\prime_{\psi,1}}(w^{\prime}, \tau \otimes \sigma_0) \quad \mbox{if $a\in F^{\times 2}$}, 
	\end{align*}
	so $R_{\mathfrak W^\prime_{\psi,1}}(w^{\prime}, \tau \otimes \sigma_0)$ depends only on the choice of the Whittaker datum $\mathfrak W^\prime_{\psi,1}$ of $\SP(W_{2n})$. 
	\item Following \cite[\S 7.3]{MR3573972}, we could also use the normalizing factors $r\left(w, \tau_{s} \otimes \pi_0\right)$ and $r\left(w^\prime, \tau_{s} \otimes \sigma_0\right)$ defined by
	\begin{equation}
	\begin{aligned}
	r\left(w, \tau_{s} \otimes \pi_0\right)&=\lambda(E / F, \psi)^{k}\cdot \gamma(s,\phi_{\tau}\otimes\phi_{\pi_0},\psi)^{-1}\cdot \gamma(2s,\wedge^{2} \circ \phi_{\tau},\psi)^{-1},\\
	r\left(w^\prime, \tau_{s} \otimes \sigma_0\right)&= \gamma(s,\phi_{\tau}\otimes\phi_{\sigma_0},\psi)^{-1}\cdot \gamma(2s,\wedge^{2} \circ \phi_{\tau},\psi)^{-1}. 
	\end{aligned}
	\end{equation}
	These are not exactly the same with  \ref{normalizationfactor}, but they share the same analytic behaviours near $s=0$. So the final self-intertwining operators $R_{\mathfrak W_{c}}(w, \tau \otimes \pi_0)$ and $R_{\mathfrak W^\prime_{\psi,1}}(w^{\prime}, \tau \otimes \sigma_0)$ will not change if we use these normalizing factors.  
\end{enumerate}
\end{remark}

We end up this subsection by comparing 
$R_{\mathfrak W_{c}}(w,\tau\otimes(\pi_0\otimes\det))$ and $R_{\mathfrak W_{c}}(w,\tau\otimes\pi_0)$. Note that there is an isomorphism as $\mathrm O(V)$-representations: 
\begin{equation}
\begin{aligned}\label{59}
\mathcal P_s:\Ind_{P}^{\mathrm O(V)}(\tau_s\otimes \pi_0)\otimes
\det  &\cong  \Ind_{P}^{\mathrm O(V)}(\tau_s\otimes (\pi_0\otimes\det))\\
\Psi_s &\mapsto \mathcal P_s(\Psi_s)
\end{aligned}
\end{equation}
for $\Psi_s\in \Ind_{P}^{\mathrm O(V)}(\tau_s\otimes \pi_0)$, where 
\begin{align*}
\mathcal P_s(\Psi_s)(h)=\Psi_s(h)\det(h)\quad \mbox{for $h\in \mathrm O(V)$}. 
\end{align*}
Here we realize $\Ind_{P}^{\mathrm O(V)}(\tau_s\otimes \pi_0)\otimes
\det$ on the same space with $\Ind_{P}^{\mathrm O(V)}(\tau_s\otimes \pi_0)$, but with the action twisted by $\det$. We identify these two representations via this isomorphism, then 
\begin{align*}
\mathcal{M}\left(\widetilde{w}_c, \tau_{s} \otimes (\pi_0\otimes\det) \right) (\Psi_{s})\left(h\right) &=\int_{U_{P}} \Psi_{s}\left(\widetilde{w}_c^{-1} u_{P} h\right)\det\left(\widetilde{w}_c^{-1} u_{P} h\right) d u_{P}\\
&=\det(\widetilde{w}_c^{-1})\cdot \det(h)\cdot \int_{U_{P}} \Psi_{s}\left(\widetilde{w}_c^{-1} u_{P} h\right) d u_{P}\\
&=(-1)^{\dim \tau}\cdot \det(h)\cdot \int_{U_{P}} \Psi_{s}\left(\widetilde{w}_c^{-1} u_{P} h\right) d u_{P}\\
&=(-1)^{\dim \tau}\cdot  \mathcal{M}\left(\widetilde{w}_c, \tau_{s} \otimes \pi\right) \Psi_{s}(h)\cdot \det(h). 
\end{align*}
Here we use the equation \ref{determinantminus1}. This implies
\begin{equation}\label{compareunnormlize}
\mathcal{M}\left(\widetilde{w}_c, \tau_{s} \otimes (\pi\otimes\det) \right) =(-1)^{\dim \tau} \mathcal{M}\left(\widetilde{w}_c, \tau_{s} \otimes \pi \right) 
\end{equation}
via the isomorphism $\mathcal P_s$. Since the Langlands paramter for $\pi_0$ and $\pi_0\otimes\det$ are the same, $\mathcal{M}\left(\widetilde{w}_c, \tau_{s} \otimes \pi_0) \right) $ and $\mathcal{M}\left(\widetilde{w}_c, \tau_{s} \otimes (\pi_0\otimes\det) \right) $ share the same normalizing factors. So we deduce from \ref{compareunnormlize} that  
\begin{equation}\label{comparenormalize}
R_{\mathfrak W_c}(w,\tau\otimes(\pi_0\otimes\det))=(-1)^{\dim \tau}R_{\mathfrak W_c}(w,\tau\otimes\pi_0) 
\end{equation}
via the isomorphism $\mathcal P_s$. 

\subsection{Weil representations and mixed models}\label{mixed model}
In this section, we recall some explicit formulas for the Weil representations. 
\begin{comment}
Let $\mathbb W$ be a finite dimensional vector space equipped with a non-degenerate symplectic form $\langle\cdot, \cdot\rangle _{\mathbb W}: \mathbb W\times \mathbb W \rightarrow F$. Let  $\mathcal{H}(\mathbb W)=\mathbb W \oplus F$ be the associated Heisenberg group, i.e., the multiplication law is given by
$$(w, t) \cdot\left(w^{\prime}, t^{\prime}\right)=\left(w+w^{\prime}, t+t^{\prime}+\frac{1}{2}\left\langle w, w^{\prime}\right\rangle_{\mathbb W}\right)$$
for $w, w^{\prime} \in W$ and $t, t^{\prime} \in F$. Fix a maximal totally isotropic subspaces $\mathbb X$ and $\mathbb X^\vee$ such that $\mathbb W=\mathbb X\oplus \mathbb X^\vee$. Let $\rho$ be the Heisenberg representation of $\mathcal H(\mathbb W)$ on $\mathscr S(\mathbb X^\vee)$ with central character $\psi$. Namely,
$$\rho\left(x+x^{\prime}, t\right) \varphi\left(x_0^\prime\right)=\psi\left(t+\left\langle x_0, x\right\rangle_{\mathbb W}+\frac{1}{2}\left\langle x^{\prime}, x\right\rangle_{\mathbb W}\right) \varphi\left(x_0^\prime+x^{\prime}\right)$$
for $\varphi \in \mathscr{S}(\mathbb X^\vee), x \in \mathbb X, x^{\prime}, x_0^\prime \in \mathbb X^\vee$ and $t \in F$. 
\end{comment}
We retain the notations in subsection \ref{normalizingintertwing}.  For simplicity, we write:
\begin{itemize}
	\item  $\omega_{00}$ for the Weil representation $\omega_{\psi, V_0, W_0}$ of $\mathrm{Sp}(W_0) \times \mathrm{O}(V_0)$ on a space $\mathscr{S}_{00}$;
	\item $\omega_{0}$ for the Weil representation $\omega_{\psi, V_0, W}$ of $\operatorname{Sp}(W) \times \mathrm{O}\left(V_0\right)$ on a space $\mathscr{S}_{0}$;
	\item $\omega$ for the Weil representation $\omega_{\psi, V, W}$ of $\operatorname{Sp}\left(W\right) \times \mathrm{O}\left(V\right)$ on a space $\mathscr{S}$.
\end{itemize}
We take a mixed model 
$$\mathscr{S}_0=\mathscr{S}\left(V_0\otimes Y^\vee\right)\otimes\mathscr{S}_{00} $$
of $\omega_0$, where we regard $\mathscr{S}_0$ as a space of functions on $V_0 \otimes Y^\vee$ with values in $\mathscr{S}_{00}$. Similarly, we take a mixed model 
$$\mathscr{S}=\mathscr{S}\left(X^\vee \otimes W\right) \otimes \mathscr{S}_0$$
of $\omega$, where we regard $\mathscr{S}$ as a space of functions on $X^\vee\otimes W$ with values in $\mathscr{S}_0$. Also, we write 

$\bullet \rho_{00}$ for the Heisenberg representation of $\mathcal{H}(V_0 \otimes W_0)$ on $\mathscr{S}_{00}$ with the central character $\psi$;

$\bullet \rho_0$ for the Heisenberg representation of $\mathcal{H}\left(V_0 \otimes W\right)$ on $\mathscr{S}_0$ with the central character $\psi$.
\begin{comment}
$\bullet \rho$ for the Heisenberg representation of $\mathcal{H}\left(V \otimes W\right)$ on $\mathcal{S}$ with the central character $\psi$ .
\end{comment}

Similar to those of \cite[Lemma 6.2,6.3,6.4]{MR3788848} and \cite[\S 7.4]{MR3573972}, we obtain some explicit formulas for these Weil representations. 
\begin{comment}
For $\varphi_{00} \in \mathcal{S}_{00}$ and $x \in V_0 \otimes Y_{n_0}^{*}$
$$\begin{array}{rlr}{[\omega_{00}(h_0) \varphi_{00}](x)} & {=\varphi_{00}\left(h_{0}^{-1} x\right),} & {h_0 \in \mathrm{O}(V_0)} \\
{\left[\omega_{00}\left(m(a)\right) \varphi_{00}\right](x)} & {=\chi_{V}\left(\operatorname{det}(a)\right)\left|\operatorname{det}(a)\right|_{F}^{n_0} \varphi_{00}\left(a^{*} x\right),} & {a \in \operatorname{GL}\left(Y_{n_0}\right)} \\ {\left[\omega_{00}\left(u(c)\right) \varphi_{00}\right](x)} & {=\psi\left(\frac{1}{2}\left\langle c x, x\right\rangle\right) \varphi_{00}(x),} & {c \in \operatorname{Sym}\left(Y_{n_0}^{*}, Y_{n_0}\right)} \\ 
{[\omega_{00}(w) \varphi_{00}](x)} & {=\gamma_{V}^{-n} \int_{V_0 \otimes Y_{n_0}} \varphi_{00}\left(I^{-1} z\right) \psi(-\langle z, x\rangle) d z}\end{array}$$
Here 
$$
w=\left (\begin{array}{cc}
0& -I_{Y_{n_0}}\\
I_{Y_{n_0}}^{-1}& 0
\end{array}\right)
$$
and  $\gamma_{V}$ is a 4 th root of unity satisfying $\gamma_{V}^{2}=\chi_{V}(-1)$. 
%$dz$ is the self-dual measure on $V_0\otimes Y_{n_0}$ with respect to the pairing $(x,y)\mapsto \psi(\langle y, I^{-1}x \rangle )$, and
\end{comment}

For $\varphi_0 \in \mathscr{S}_0$ and $x\in V_0\otimes Y^\vee$, we have 
\begin{align*}
\left[\omega_0(g_0) \varphi_0\right]\left(x\right) & =\omega_{00}(g_0)\varphi_0(x), & g \in \mathrm{Sp}(W_0), \\
\left[\omega_0(h_0) \varphi_0\right](x) & =\omega_{00}(h_0)\varphi_0(h_{0}^{-1}x),  &  h_0 \in  \mathrm{O}(V_0), \\
\left[\omega_0\left( m_{Q}(a)\right) \varphi_0\right]\left(x\right) & =\chi_{V}(\det a)|\operatorname{det}(a)|_{F}^{n_0} \varphi_0(a^{*} x), & {a \in \mathrm{GL}(Y)} ,\\ 
\left[\omega_0\left( u_{Q}(b)\right) \varphi_0\right]\left(x\right) & =\rho_{00}\left(b^{*} x, 0\right)\varphi_0(x) , & {b \in \operatorname{Hom}(W_0, Y)},\\
\left[\omega_0\left( u_{Q}(c)\right) \varphi_0\right]\left(x\right) &=\psi\left(\frac{1}{2}\left\langle c x, x\right\rangle\right) \varphi_0\left(x\right),  & c \in \operatorname{Sym}\left(Y^\vee, Y\right) ,\\
\left[\omega_0\left(w_{Q}\right) \varphi_0\right]\left(x\right) &=\gamma_{V}^{-k} \int_{V_0 \otimes Y} \varphi_0\left(I_{Y}^{-1} z\right) \psi\left(-\left\langle z, x\right\rangle\right) d z, \end{align*}
where $\gamma_{V}$ is a $4$-th root of unity satisfying $\gamma_{V}^{2}=\chi_{V}(-1)$.

For $\varphi\in  \mathscr{S}$ and $x\in X^\vee\otimes W$, we have 
\begin{align*}
\left[\omega(g) \varphi\right]\left(x \right)&= \omega_0(g)\varphi(g^{-1}x),  & g \in \mathrm{Sp}(W) , \\
\left[\omega(h_0) \varphi\right](x) & =\omega_0(h_0)\varphi(x), & {h \in \mathrm{O}\left(V_0\right)}, \\ 
{\left[\omega\left(m_{P}\left(a\right)\right) \varphi\right]\left(x\right)} & =\left|\operatorname{det}\left(a\right)\right|_{F}^{n} \varphi(a^{ *}x), & {a \in \mathrm{GL}(X)},\\
{\left[\omega\left(u_{P}\left(b\right)\right) \varphi\right]\left(x \right)}& =\rho_{0}\left(b^{ *} x, 0\right) \varphi\left(x\right),  & {b \in \operatorname{Hom}(V_0, X)}, \\ \left[\omega\left(u_{P}\left(c\right)\right) \varphi\right]\left(x\right) & =\psi\left(\frac{1}{2}\left\langle c x, x\right\rangle\right) \varphi\left(x\right) , & c \in \operatorname{Sym}\left(X^\vee, X\right),\\
\left[\omega\left(w_{P}\right) \varphi\right]\left(x \right)&=\ \int_{X\otimes W } \varphi\left(I_{X}^{-1} z\right) \psi\left(-\left\langle z, x\right\rangle\right) d z. 
 \end{align*}
%and 
%$$\begin{aligned}\left[\rho_{0}\left(\left(y+y^{\prime}, 0\right)\right) \varphi_0\right](x)&=\psi\left(\langle x, y\rangle+\frac{1}{2}\left\langle y^{\prime}, y\right\rangle\right) \varphi_0\left(x+y^{\prime}\right), & y \in V_0 \otimes Y \\
%&{} &y^{\prime} \in V_0 \otimes Y^\vee \\
%\left[\rho_0\left(\left(y_{0}, 0\right) \right)\varphi_0\right](x),&=\rho_{00}\left(\left(y_{0}, 0\right)\right) \varphi_0(x) & y_{0} \in V_0 \otimes W_0 \end{aligned}$$
%For $\varphi_0\in S_0$ and $x\in V_0\otimes Y^\vee$.

%$$\left[\rho^{\prime}\left(v+v_{0}+v^{*}, 0\right) \varphi^{\prime}\right]\left(x_{2}, x_{3}\right)=\psi\left(\left\langle x_{3}, v\right\rangle+\frac{1}{2}\left\langle v^{*}, v\right\rangle\right)\left[\rho\left(v_{0}, 0\right) \varphi_{2}\right]\left(x_{2}\right) \cdot \varphi_{3}\left(x_{3}+v^{*}\right)$$




%Combine the above explicit formulas, we get a formula for the mixed model $\mathcal{S}^{\prime \prime}=\mathcal{S}\left(V^{\prime} \otimes Y^\vee\right) \otimes \mathcal{S}\left(V \otimes Y_{n}^{*}\right) \otimes \mathcal{S}\left(X^\vee \otimes W\right)$ of $\omega^{''}$


\subsection{Construction of equivariant maps}\label{constructionof}
Recall that we put 
$$V=X\oplus V_0\oplus X^{\vee}, \quad W=Y\oplus W_0\oplus Y^{\vee}$$
with $\dim(X)=\dim(Y)=k $ in subsection \ref{normalizingintertwing}. Using the basis $\left\{v_{1}, \ldots, v_{k}\right\}$ of $X$ (resp. $\left\{w_{1}, \ldots, w_{k}\right\}$ of $Y$ ), we identify $\GL(X)$ (resp. $\GL(Y)$) with $\GL_k(F)$. Hence we can define an isomorphism $i: \GL(X) \rightarrow \GL(Y) $ via these identifications. Put 
$$e=v_{1} \otimes w_{1}^{*}+\cdots+v_{k} \otimes w_{k}^{*} \in X \otimes Y^{\vee}, \quad e^{*}=v_{1}^{*} \otimes w_{1}+\cdots+v_{k}^{*} \otimes w_{k} \in X^{\vee} \otimes Y.$$
%Then $i(a)^{c} e=a^{*} e$ and $\left(i(a)^{c}\right)^{*} e^{*}=a e^{*}$ for $a \in \mathrm{GL}(X)$

For $\varphi\in \mathscr S =\mathscr S(X^{\vee}\otimes W)\otimes \mathscr S_0$, we define functions $\mathfrak{f}(\varphi), \hat{\mathfrak{f}}(\varphi)$ on $\SP(W)\times \mathrm{O}(V)$ with values in $\mathscr S_0$ by 
\begin{align*}
\mathfrak{f}(\varphi)(g ,h)&=\left(\omega(g,h) \varphi\right)\left(
\begin{array}{c}{\hspace{0.4em}e^*}\\ {0} \\ {0}\end{array}\right),\\
\hat{\mathfrak{f}}(\varphi)(g, h)&=\int_{X^{\vee} \otimes Y}\left(\omega(g ,h) \varphi\right)\left(\begin{array}{c}{x} \\ {0} \\ {0}\end{array}\right) \psi\left(\left\langle x, e\right\rangle\right) d x
\end{align*}
for $g\in \SP(W)$ and $h \in \mathrm{O}(V) $. Here, we write an element in $X^{\vee}\otimes W$ as a block
matrix
$$\left(\begin{array}{l}{y_{1}} \\ {y_{2}} \\ {y_{3}}\end{array}\right)$$
with $y_{1} \in X^{\vee} \otimes Y, y_{2} \in X^{\vee} \otimes W_0$ and $y_{3} \in X^{\vee} \otimes Y^{\vee}$. We also define functions $f(\varphi), \hat f(\varphi)$ on $\SP(W)\times \mathrm {O}(V)$ with values in $\mathscr S_{00}$ by 
$$\begin{aligned} 
f(\varphi)(g,h) &=\operatorname{ev}(\mathfrak{f}\left(\varphi(g, h)\right), \\ \hat{f}(\varphi)(g ,h) &=\operatorname{ev}\left (\hat{\mathfrak{f}}(\varphi)(g, h)\right), \end{aligned}$$
where
$$\operatorname{ev}: \mathscr S_0=\mathscr S(V_0\otimes Y^{\vee})\otimes \mathscr S_{00}\rightarrow \mathscr S_{00}$$
 is the evaluation at $0\in V_0\otimes Y^{\vee}$. If $f=f(\varphi )$ or $\hat f(\varphi)$, then 
 \begin{align*}
 f(u_{Q} g, u_{P} h)=&f(g, h),  &{u_{P} \in U_{P}, u_{Q} \in U_{Q}} ,\\ 
f(g_0g,h_0h) =&\omega_{00}(g_0,h_0) f(g,h),  h_0 \in \mathrm{O}(V_0), &g \in \mathrm{Sp}(W_0), \\ 
f(m_{Q}(i(a)) g, m_{P}(a) h)  =&\chi_{V}(\operatorname{det}(a))|\det(a)|_{F}^{\rho_P+\rho_Q} f\left(g,h\right),   &a \in \GL(X). 
 \end{align*}


Let $\tau$ be an irreducible (unitary) tempered representation of $\GL_{k}(F)$ on a space $\mathscr{V}_{\tau}$. We may regard $\tau$ as a representation of $\GL(X)$ or $\GL(Y)$ via the above identifications. Let $\sigma_0$ and $\pi_0$ be irreducible tempered representations of $\SP(W_0)$ and $\mathrm{O}(V_0)$ on spaces $\mathscr{V}_{\sigma_0}$ and $\mathscr{V}_{\pi_0}$, respectively. Fix nonzero invariant non-degenerate bilinear forms $\langle\cdot, \cdot\rangle$ on $\mathscr{V}_{\tau} \times \mathscr{V}_{\tau^ {\vee}}, \mathscr{V}_{\sigma_0} \times \mathscr{V}_{\sigma_{0}^{\vee}}$ and $\mathscr{V}_{\pi_0} \times \mathscr{V}_{\pi_{0}^{\vee}}$. Let
\begin{align*}
\langle\cdot, \cdot\rangle :\left(\mathscr{V}_{\tau} \otimes \mathscr{V}_{\sigma_{0}^{\vee}}\right) \times \mathscr{V}_{\tau^{\vee}} &\rightarrow \mathscr{V}_{\sigma_{0}^{\vee}},\\
\langle\cdot, \cdot\rangle :\left(\mathscr{V}_{\tau} \otimes \mathscr{V}_{\pi_{0}^{\vee}}\right) \times \mathscr{V}_{\tau^{\vee}} &\rightarrow \mathscr{V}_{\pi_{0}^{\vee}}
\end{align*}
be the induced maps. 

Now assume that 
$$
\sigma_0=\theta_{V_0,W_0,\psi}(\pi_0)\neq 0. 
$$
We fix a nonzero $\SP(W_0)\times \mathrm O(V_0)$-equivariant map 
$$
\mathcal{T}_{00} : \omega_{00} \otimes \sigma_{0}^{\vee} \rightarrow \pi_0. 
$$
For $\varphi\in \mathscr S, \Phi_{s}\in \Ind_{Q}^{\SP(W)}(\tau_s\chi_V\otimes \sigma_{0}^{\vee}), h\in \mathrm O(V),\check{v}\in \mathscr{V}_{\tau^{\vee}}$ and $\check{v}_{0}\in \mathscr{V}_{\pi_{0}^{\vee}}$, put 
\begin{align*}
&\left\langle\mathcal{T}_{s}\left(\varphi\otimes \Phi_{s}\right)(h), \check{v} \otimes \check{v}_{0}\right\rangle\\
&\qquad=L(s, \tau)^{-1} \times \int_{U_{Q} \SP(W_0) \backslash \SP\left(W\right)}\left\langle\mathcal{T}_{00}(\hat{f}(\varphi)\left(g, h\right)\otimes \left\langle\Phi_{s}\left(g\right), \check{v}\right\rangle), \check{v}_{0}\right\rangle d g.
\end{align*}
Note that $\left\langle\Phi_{s}\left(g\right), \check{v}\right\rangle \in \mathscr{V}_{\sigma_{0}^{\vee}}$. 

\begin{proposition}\label{intertwingmap}
	We have :
	\begin{enumerate}
		\item The integral $\left\langle\mathcal{T}_{s}\left(\varphi\otimes \Phi_{s}\right)\left(h\right), \check{v} \otimes \check{v}_{0}\right\rangle$ is absolutely convergent for $\Re s>0$ and admits a holomorphic continuation to $\mathbb C$. 
	    \item For $\Re(s)<1$, we have 
	    \begin{align*}
	    &\left\langle\mathcal{T}_{s}\left(\varphi\otimes \Phi_{s}\right)\left(h\right), \check{v} \otimes \check{v}_{0}\right\rangle\\
	    &\qquad=L(s, \tau)^{-1} \gamma(s, \tau, \psi)^{-1}\\
	    &\qquad\qquad \times\int_{U_{Q} \mathbb{\SP}(W_0) \backslash \SP\left(W\right)}\left\langle\mathcal{T}_{00}\left(f(\varphi )\left(g ,h\right)\otimes\left\langle\Phi_{s}\left(g\right), \check{v}\right\rangle\right), \check{v}_{0}\right\rangle d g.
	    \end{align*}
	    \item  The map 
	    $$\mathcal{T}_{s} : \omega\otimes \operatorname{Ind}_{Q}^{\SP\left(W\right)}\left(\tau_{s}\chi_{V} \otimes \sigma_{0}^{\vee}\right) \rightarrow \operatorname{Ind}_{P}^{\mathrm{O}\left(V\right)}\left(\tau_{s}  \otimes \pi_0\right)$$
	    is $\SP(W)\times \mathrm O(V)$-equivariant. 
	    \item For $\Phi \in \operatorname{Ind}_{Q}^{\mathrm{\SP}\left(W\right)}\left(\tau\chi_{V} \otimes \sigma_{0}^{\vee}\right)$ with $\Phi \neq 0,$ there exists $\varphi \in \mathscr S$ such that
	    $$\mathcal{T}_{0}(\varphi, \Phi) \neq 0.$$
	\end{enumerate}
\end{proposition}
\begin{proof}
	The proof is similar to that of \cite[Proposition 7.2]{MR3788848}. 
\end{proof}

\subsection{Compatibilities with intertwining operators}
Now we shall prove a key property of the equivariant map we have constructed. Having fixed $\tau,\pi_0$ and $\sigma_0 =\theta_{\psi, W_0, V_0}(\pi_0) \neq 0$, we let  
$$\mathcal{M}(\widetilde{w}_c, s)=\mathcal{M}\left(\widetilde{w}_c, \tau_{s} \otimes \pi_0\right) \quad \mbox{and} \quad \mathcal{M}\left(\widetilde{w}_1^{\prime}, s\right)=\mathcal{M}\left(\widetilde{w}_1^\prime, \tau_{s}\chi_{V} \otimes \sigma_{0}^{\vee}\right)$$
be the unnormalized intertwining operators defined in subsection \ref{normalizingintertwing}. By the Howe duality, the diagram 
$$
\begin{CD}
\omega \otimes \operatorname{Ind}_{Q}^{\mathrm{SP}\left(W\right)}\left(\tau_{s}\chi_{V} \otimes \sigma_{0}^{\vee}\right) @>\mathcal T_{s}>> \operatorname{Ind}_{P}^{\mathrm{O}\left(V\right)}\left(\tau_{s}  \otimes \pi_0\right)\\
@VV1\otimes\mathcal{M}(\widetilde{w}_1^{\prime}, s) V @VV\mathcal{M}(\widetilde{w}_c, s)V\\
\omega \otimes \operatorname{Ind}_{Q}^{\mathrm{SP}\left(W\right)}\left(w^{\prime}(\tau_{s}\chi_{V} \otimes \sigma_{0}^{\vee})\right) @>\mathcal T_{-s}>> \operatorname{Ind}_{P}^{\mathrm{O}\left(V\right)}\left(w(\tau_{s} \otimes \pi_0)\right) 
\end{CD}
$$
commutes up to a scalar. The following proposition determines this constant of proportionality explicitly.
\begin{proposition}\label{commuteunnormalize}
	For $\varphi\in \mathscr S $ and $\Phi_{s}\in \Ind_{Q}^{\SP(W)}(\tau_s\chi_{V}\otimes \sigma_{0}^{\vee})$, we have 
	\begin{align*}
	\mathcal{M}\left(\widetilde{w}_c, s\right) \mathcal{T}_{s}\left(\varphi\otimes \Phi_{s}\right)=& \omega_{\tau}(-1/c) \cdot | c_{F}|^{-k(\rho_P+s)} \cdot \gamma_{V}^{k} \cdot\chi_{V}(-1)^{k}\cdot L(s, \tau)^{-1}\\& \times L\left(-s, \tau^{\vee}\right)\cdot\gamma\left(-s, \tau^{\vee}, \psi\right) 
	\cdot \mathcal{T}_{-s}\left(\varphi\otimes \mathcal{M}(\widetilde{w}_1^{\prime}, s) \Phi_{s}\right). 
	\end{align*}
 \end{proposition}
\begin{proof}
	The proof is similar to that of \cite[Proposition 8.4]{MR3573972}. 
\end{proof}	

As a consequence of Proposition \ref{commuteunnormalize}, we deduce:
\begin{corollary}\label{123}
	For $\varphi\in \mathscr S $ and $\Phi_{s}\in \Ind_{Q}^{\SP(W)}(\tau_s\chi_{V}\otimes \sigma_{0}^{\vee})$, we have
	\begin{align*}
	\mathcal{R}_{\mathfrak W_{c}}\left(w, \tau_{s}\otimes \pi_0\right) \mathcal{T}_{s}\left(\varphi\otimes \Phi_{s}\right)=&\omega_{\tau}(-1/c)\cdot \chi_{V}(-c)^k\cdot  \alpha(s)\\
	&\times\mathcal{T}_{-s}\left(\varphi\otimes \mathcal{R}_{\mathfrak W^\prime_{\psi,1}}\left(w^{\prime}, \tau_{s}\chi_{V} \otimes \sigma_{0}^{\vee} \right)\Phi_{s}\right), 
	\end{align*}
	where
	$$
	\alpha(s)=|c|_{F}^{-ks} \cdot\frac{\varepsilon(-s,\tau^{\vee},\psi)}{\varepsilon(s,\tau,\psi)}.
	$$
	In particular, if $\tau\cong \tau^{\vee}$, then $\alpha(0)=1$ and 
$$
R_{\mathfrak W_{c}}\left(w, \tau\otimes \pi_0\right) \mathcal{T}_{0}\left(\varphi\otimes \Phi\right)=\omega_{\tau}(-1/c)\cdot \chi_{V}(-c)^k\cdot \mathcal{T}_{0}\left(\varphi\otimes R_{\mathfrak W^\prime_{\psi,1}}\left(w^{\prime}, \tau\chi_{V} \otimes \sigma_{0}^{\vee} \right)\Phi\right)
$$
for  $\Phi\in \Ind_{Q}^{\SP(W)}(\tau
\chi_{V}\otimes \sigma_{0}^{\vee})$. 
\end{corollary}
\begin{proof}
The corollary immediately follows from Proposition \ref{commuteunnormalize} and the fact that   
$$
\lambda(E/F,\psi)=\epsilon(V)\cdot \gamma_V.
$$
where $\gamma_V$ is the Weil constant associated to $V$ which appears on the explicit formula for the Weil representation, and $\lambda(E/F,\psi)$ is the Langlands constant which appears in the normalizing factors. 
\end{proof}



\section{The proof of local intertwining relation}
We begin to prove the Theorem \ref{desideratumall} (9), i.e., local intertwining relation, for our construction. We retain the notations in subsection \ref{normalizingintertwing}. Let $\phi\in \Para(\mathrm O(V))$ such that    
$$\phi=\phi_{\tau}\oplus\phi_0\oplus \phi_{\tau}^{\vee},$$
where $\phi_{\tau}$ is an irreducible tempered representation of $\WD(F)$ corresponding to $\tau\in \Irr(\GL_k(F))$ and $\phi_0\in \Para(\mathrm O(V_0))$. In this case, we have a natural embedding $\mathcal S_{\phi_0}\hookrightarrow \mathcal {S}_{\phi}$. Let $\pi_0\in \Pi_{\phi_0}$ be an irreducible tempered representation of $\mathrm O(V_0)$. Our goal is to analyze the induced representation $\Ind_{P}^{\mathrm O(V)}(\tau\otimes \pi_0)$. 

We divide our proof into two part. In the first part, we analyze the $L$-parameter for each irreducible constituent $\pi$ of $\Ind_{P}^{\mathrm O(V)}(\tau\otimes \pi_0)$, and as a corollary, we get some information on the reducibility of $\Ind_{P}^{\mathrm O(V)}(\tau\otimes \pi_0)$. In the second part, we analyze the charaters $\mathcal J_{\mathfrak W_c}(\pi)$ for each Whittaker datum $\mathfrak W_c$ of $\bigsqcup_{V_{2n}^{\bullet}} \mathrm O(V_{2n}^\bullet)$.

\subsection{$L$-parameter and reducibility}
We first determine the $L$-parameter of $\pi\subseteq \Ind_{P}^{\mathrm O(V)}(\tau\otimes \pi_0)$.
\begin{lemma}\label{parameter}
	Let $\pi$ be a irreducible constituent of $\Ind_{P}^{\mathrm O(V)}(\tau\otimes \pi_0)$. Then the $L$-parameter of $\pi$ is $\phi$. 
\end{lemma}
\begin{proof}
	We divided it into two cases: 
	
	$\bullet$ \underline{Case I}: if $\theta_{W,V,\psi}(\pi)\neq 0 $, let $\sigma=\theta_{W,V,\psi}(\pi)$, then by Lemma \ref{12}, we have 
	$$
	\sigma \subseteq \Ind_{Q}^{\SP(W)}\left(\tau\chi_{V}\otimes \sigma_{0}\right),
	$$
	where  
	$$\sigma_{0}=
	\theta_{W_0,V_0,\psi}(\pi_0). 
	$$
	 Let $\phi^+$ and $\phi_{0}^+$ be the $L$-parameter of $\sigma$ and $\sigma_{0}$. Then by Theorem \ref{llcsympletic} (6), we have
	$$
	\phi^+=(\phi_{\tau}\otimes \chi_{V})\oplus \phi^+_{0} \oplus (\phi_{\tau}^{\vee}\otimes \chi_{V}).
	$$
	On the other hand, it follows from Proposition \ref{prasad} that 
	\begin{align*}
	\phi^+&=(\phi_{\pi}\otimes \chi_{V})\oplus \chi_{V} ,\\
	\phi^+_{0}&=(\phi_{0}\otimes\chi_{V})\oplus \chi_{V},
	\end{align*}
	where $\phi_{\pi}$ is the $L$-parameter of $\pi$. We deduce $\phi_{\pi}=\phi$ from these equalities. 
	
	$\bullet$ \underline{Case II}: if $\theta_{W,V,\psi}(\pi)= 0 $, then $\theta_{W,V,\psi}(\pi\otimes\det)\neq 0$ by the conservation relation \ref{con2}. It follows from \ref{59} that 
	\begin{align*}
	\pi\otimes\det \subseteq \Ind_{P}^{\mathrm O(V)}(\tau\otimes (\pi_0\otimes\det)). 
	\end{align*} 
	 Since by Proposition \ref{det},
	\begin{align*}
	\mathcal L(\pi_0\otimes\det)=\mathcal L(\pi_0)=\phi_0, 
	\end{align*}
	we deduce from Case I that $\mathcal L(\pi\otimes\det)=\phi$. Then again by Proposition \ref{det}, we have 
	\begin{align*}
	\mathcal L(\pi)=\mathcal L(\pi\otimes\det)=\phi. 
	\end{align*}
	This completes the proof. 
\end{proof}

Next we analyze the reducibility of $\Ind_{P}^{\mathrm O(V)}(\tau\otimes \pi_0)$.
\begin{lemma}\label{multiplicityfree}
	The representation $\Ind_{P}^{\mathrm O(V)}(\tau\otimes \pi_0)$ is semisimple and multiplicity free. 
\end{lemma}
\begin{proof}
	Since $\tau$ and $\pi_{0}$ are unitary representations, so is $\Ind_{P}^{\mathrm O(V)}(\tau\otimes \pi_0)$, hence it is semisimple. 
	
	Next we prove $\Ind_{P}^{\mathrm O(V)}(\tau\otimes \pi_0)$ is multiplicity free. Let $\pi\in \Irr(\mathrm O(V))$ such that 
	\begin{align*}
	m(\pi)=\dim \Hom_{\mathrm O(V)}(\pi, \Ind_{P}^{\mathrm O(V)}(\tau\otimes \pi_0))\geq 1 .
	\end{align*}
	We prove $m(\pi)=1$ according to $\theta_{W,V,\psi}(\pi)=0$ or not: 

$\bullet$ \underline{Case I}: If $\theta_{W,V,\psi}(\pi)\neq 0$, let $\sigma=\theta_{ W,V,\psi}(\pi)$. As in the proof of Lemma \ref{12}, there is an injective map 
		\begin{align*}
		\Hom_{\mathrm O(V)}(\omega, \Ind_{P}^{\mathrm O(V)}(\tau\otimes \pi_0)) \hookrightarrow \left(\Ind_{Q}^{\SP(W)}(\tau^\vee\chi_{V}\otimes \sigma_{0})\right)^{\vee},
		\end{align*}
		then it is easy to see that 
		$$
		m(\pi) \leq m(\sigma^\vee) 
		$$
		where 
		$$
		m(\sigma^\vee) =  \dim \Hom_{\SP(W)}\left(\sigma^\vee, \left(\Ind_{Q}^{\SP(W)}(\tau^\vee\chi_{V}\otimes \sigma_{0})\right)^{\vee}\right). 
		$$
		By Theorem \ref{llcsympletic} , we have $m(\sigma^\vee) \leq 1$. Hence $m(\pi)=1$. 
		
		
		
		\begin{comment}
		Let $\Ind_{P}^{\mathrm O(V)}(\tau\otimes \pi_0)=\pi_1\oplus \pi_2\cdots \oplus \pi_l$, since $\theta_{V,W,\psi}(\pi)\neq 0$, $\Hom_{\mathrm O(V)}(\omega, \pi)$ is non-zero and isomorphic to $\sigma^\vee$ 
		So 
		\begin{align*}
		\oplus_{i} \Hom_{\mathrm O(V)}(\omega, m(\pi)\pi)_{\infty} &\hookrightarrow \Hom_{\mathrm O(V)}(\omega, \Ind_{P}^{\mathrm O(V)}(\tau\otimes \pi_0))_{\infty}\\
		&\hookrightarrow \Ind_{Q}^{\SP(W)}(\chi_{V}\tau\otimes \sigma_{0})^{\vee}
		\end{align*}
		By LLC (add) for symplectic groups, the multiplicity of $\sigma^\vee$ in $\Ind_{Q}^{\SP(W)}(\chi_{V}\tau\otimes \sigma_{0})^{\vee}$ is one, hence the multiplicity $m(\pi)$ is at most one. 
		\end{comment}
		
$\bullet$ \underline{Case II}: If $\theta_{ W,V,\psi}(\pi)=0$, then by the conservation relation \ref{con2}, $\theta_{ W,V,\psi}(\pi\otimes\det)\neq 0$. Note that by \ref{59}, we have 
		\begin{align*}
		\pi\otimes\det \subseteq \Ind_{P}^{\mathrm O(V)}(\tau\otimes (\pi_0\otimes\det)). 
		\end{align*} 
		 It then follows from Case I that  
		\begin{align*}
		m(\pi\otimes\det)=\dim \Hom_{\mathrm O(V)}(\pi\otimes\det, \Ind_{P}^{\mathrm O(V)}(\tau\otimes (\pi_0\otimes\det)))=1. 
		\end{align*}
		On the other hand, by \ref{59}, we have $m(\pi)=m(\pi\otimes \det=1$. This prove Case II. 
\end{proof}
\begin{remark}
	This lemma will also follows from the abelianess of the $R$-group and the induction in stages. The abelianess of the $R$-group was proved by Goldberg \cite{MR1296726} for the quasi-split special orthogonal groups, and by Choiy-Goldberg \cite{MR3430367} for pure inner forms of quasi-split special orthogonal groups. But the proof in \cite{MR3430367} is based on the conjectural LLC for pure inner forms of quasi-split special even orthogonal groups. To avoid a circlar reasoning, we give a proof of the lemma here, which is independent of the results for $R$-groups. 
\end{remark}

Now we analyze the reducibility of $\Ind_{P}^{\mathrm O(V)}(\tau\otimes \pi_0)$. Recall that there is a natural embedding $\mathcal S_{\phi_{0}}\hookrightarrow \mathcal {S}_{\phi}$ of component groups. We divide into two cases depending on the relative size of $\mathcal S_{\phi_0}$ and $\mathcal {S}_{\phi}$.  
\begin{corollary}\label{embeddingisisomorphism}
	Assume that $\phi_{\tau}\subseteq \phi_0$, so the natural embedding $\mathcal S_{\phi_{0}}\hookrightarrow \mathcal {S}_{\phi}$ is an isomorphism. Then the induced representation $\Ind_{P}^{\mathrm O(V)}(\tau\otimes \pi_0)$ is irreducible. 
\end{corollary}
\begin{proof}
We denote by $$\JH(\Ind_{P}^{\mathrm O(V)}(\tau\otimes \pi_0))$$ 
the set of irreducible constituents of $\Ind_{P}^{\mathrm O(V)}(\tau\otimes \pi_0)$. Consider the set 
\begin{align*}
\bigsqcup_{\pi_0\in \Pi_{\phi_0}} \JH(\Ind_{P}^{\mathrm O(V)}(\tau\otimes \pi_0)) .
\end{align*}
By the Howe duality, Lemma \ref{parameter} and Lemma \ref{multiplicityfree}, this set is a subset of $\Pi_{\phi}$. Hence 
\begin{align*}
|\Pi_{\phi}|\geq \left|\bigsqcup_{\pi_0\in \Pi_{\phi_0}} \JH(\Ind_{P}^{\mathrm O(V)}(\tau\otimes \pi_0)) \right|\geq |\Pi_{\phi_{0}}|. 
\end{align*}
On the other hand, by Proposition \ref{bijection}, we have 
\begin{align*}
|\Pi_{\phi}|=|\widehat{\mathcal {S}_{\phi}}|=|\widehat{\mathcal S_{\phi_0}}|=|\Pi_{\phi_{0}}|. 
\end{align*}
Hence we must have 
\begin{align*}
\left| \JH(\Ind_{P}^{\mathrm O(V)}(\tau\otimes \pi_0)) \right|=1
\end{align*}
for all $\pi_0\in \Pi_{\phi_{0}}$. This implies $\Ind_{P}^{\mathrm O(V)}(\tau\otimes \pi_0)$ is irreducible. 
\end{proof}



\begin{corollary}\label{notanisomorphism}
	If $\phi_\tau\not \subseteq \phi_0$, so the image of $\mathcal S_{\phi_{0}}$ inside $\mathcal {S}_{\phi}$ is an index $2$ subgroup. Then $\Ind_{P}^{\mathrm O(V)}(\tau\otimes \pi_0)$ is sum of two non-isomorphic irreducible representations. 
\end{corollary}
\begin{proof}
	It follows from Lemma \ref{multiplicityfree} that $\Ind_{P}^{\mathrm O(V)}(\tau\otimes \pi_0)$ is multiplicity free, so we only need to prove $\Ind_{P}^{\mathrm O(V)}(\tau\otimes \pi_0)$ is reducible of and of length two.  
	
	We first prove 
	\begin{align}\label{1236}
	\left|\JH(\Ind_{P}^{\mathrm O(V)}(\tau\otimes \pi_0))\right|\geq 2,
	\end{align}
	in other words, $\Ind_{P}^{\mathrm O(V)}(\tau\otimes \pi_0)$ is reducible. Let  
	\begin{align*}
	\phi^+&=(\phi\otimes \chi_{V})\oplus \chi_{V}, \\
	\phi^+_{0}&=(\phi_{0}\otimes\chi_{V})\oplus \chi_{V}.
	\end{align*}
	Depending on the relative size of $\mathcal S_{\phi^+_{0}}$ and $\mathcal S_{\phi^+}$, there are two cases: 
	
	$\bullet$ \underline{Case I}: If $\phi_\tau\neq \mathrm 1$, then $\mathcal S_{\phi^+_{0}}$ is a index $2$ subgroup of $\mathcal S_{\phi^+}$. It follows from the conservation relation \ref{con2} that at least one of  
	$$\theta_{W_0,V_0,\psi}(\pi_0)\quad \mbox{and}\quad \theta_{W_0,V_0,\psi}(\pi_0\otimes\det)$$ 
	is non-zero. Note that we have the isomorphism 
	\begin{align*}
	\Ind_{P}^{\mathrm O(V)}\left(\tau \otimes (\pi_0\otimes\det)\right)\cong \Ind_{P}^{\mathrm O(V)}\left(\tau \otimes \pi_0\right)\otimes\det 
	\end{align*}
	in \ref{59}, so the reducibility of $\Ind_{P}^{\mathrm O(V)}\left(\tau \otimes (\pi_0\otimes\det)\right)$ is the same as the reducibility of $\Ind_{P}^{\mathrm O(V)}\left(\tau \otimes \pi_0\right)$. Without loss of generality, we may assume that
	$$\theta_{W_0,V_0,\psi}(\pi_0)\neq 0.$$ 
	Put $$\sigma_{0}=\theta_{W_0,V_0,\psi}(\pi_0),$$ then the $L$-parameter of $\sigma_0$ is $\phi_{0}^+$ by Proposition \ref{prasad}, so is $\sigma_{0}^\vee$ by \ref{104}. It then follows from Theorem \ref{llcsympletic} (6) that 
	$$
	\Ind_{Q}^{\SP(W)}(\tau\chi_{V}\otimes\sigma_{0}^{\vee})
	$$
	is reducible and has two non-isomorphic irreducible constituents. Put 
	\begin{align*}
	\Ind_{Q}^{\SP(W)}(\tau\chi_{V}\otimes\sigma_{0}^{\vee})\cong \sigma_1^\vee\oplus \sigma_2^\vee. 
	\end{align*}
	Recall that we have constructed a $\SP(W)\times\mathrm O(V)$-equivariant map
	$$\mathcal{T}_{0} : \omega \otimes \Ind_{Q}^{\SP(W)}\left(\tau\chi_{V} \otimes \sigma_{0}^{\vee}\right) \rightarrow \Ind_{P}^{\mathrm O(V)}\left(\tau \otimes \pi_0\right)$$ 
	in subsection \ref{constructionof}. By the non-vanishing result in Proposition\ref{intertwingmap}, we know that the restriction of $\mathcal T_0$ on $\omega\otimes\sigma_1^\vee$ and $\omega\otimes\sigma_2^\vee$ are both non-zero. Moreover, their images are irreducible and non-isomorphic to each other by the Howe duality. Hence $$\Ind_{P}^{\mathrm{O}\left(V\right)}\left(\tau  \otimes \pi_0\right)$$ is reducible.
	

	
	$\bullet$ \underline{Case II}: If $\phi_\tau= \mathrm 1$, then the natural embedding $\mathcal S_{\phi^+_{0}}\hookrightarrow \mathcal S_{\phi^+}$ is an isomorphism. Our assumptions imply that $\mathrm 1 \nsubseteq \phi_0$. It then follows from Lemma \ref{1notinphi} that both  $$\theta_{W_0,V_0,\psi}(\pi_{0})\quad \mbox{and}\quad \theta_{W_0,V_0,\psi}(\pi_{0}\otimes\det)$$ are non-zero. Put 
	\begin{align*}
	\sigma_0^+=\theta_{W_0,V_0,\psi}(\pi_{0})\quad\mbox{and}\quad  \sigma_0^-=\theta_{W_0,V_0,\psi}(\pi_{0}\otimes\det). 
	\end{align*} 
	Then $\sigma_0^{\pm}$ both have $L$-parameter $\phi_{0}^+$ by Proposition \ref{prasad}, so are $(\sigma_0^{\pm})^{\vee}$ by \ref{104}. By Theorem \ref{llcsympletic} (6), both 
	$$\Ind_{Q}^{\SP(W)}\left(\tau\chi_{V}\otimes \left(\sigma_{0}^{+}\right)^{\vee}\right)\quad \mbox{and}\quad \Ind_{Q}^{\SP(W)}\left(\tau\chi_{V}\otimes \left(\sigma_{0}^{-}\right)^{\vee}\right)$$ are irreducible. Put 
	\begin{align*}
	\left(\sigma^{+}\right)^{\vee}&= \Ind_{Q}^{\SP(W)}\left(\tau\chi_{V}\otimes \left(\sigma_{0}^{+}\right)^{\vee}\right),\\
	\left(\sigma^{-}\right)^{\vee}&= \Ind_{Q}^{\SP(W)}\left(\tau\chi_{V}\otimes \left(\sigma_{0}^{-}\right)^{\vee}\right).
	\end{align*}
  Recall that we have constructed non-zero $\mathrm O(V)\times \SP(W)$-equvariant maps 
	\begin{align*}
	\mathcal{T}_0^{+} : \omega \otimes \Ind_{Q}^{\SP(W)}\left(\tau\chi_{V}\otimes \left(\sigma_{0}^{+}\right)^{\vee}\right)  &\rightarrow \Ind_{P}^{\mathrm O\left(V\right)}\left(\tau  \otimes \pi_0\right),\\
	\mathcal{T}_0^{-} : \omega \otimes \Ind_{Q}^{\SP(W)}\left(\tau\chi_{V}\otimes \left(\sigma_{0}^{-}\right)^{\vee}\right)  &\rightarrow \Ind_{P}^{\mathrm O(V)}\left(\tau  \otimes (\pi_0\otimes\det)\right).
	\end{align*}
	in subsection \ref{constructionof}. Let 
	\begin{align*}
	\pi^{+} \coloneqq \mbox{Im}( \mathcal{T}_0^+)&\subseteq \Ind_{P}^{\mathrm O\left(V\right)}\left(\tau\otimes \pi_0\right),\\
	\pi^{-} \coloneqq \mbox{Im}( \mathcal{T}_0^-)&\subseteq \Ind_{P}^{\mathrm O\left(V\right)}\left(\tau\otimes (\pi_0\otimes\det)\right)=\Ind_{P}^{\mathrm O\left(V\right)}\left(\tau\otimes \pi_0\right)\otimes\det.
	\end{align*}
	Then $\pi^{\pm}$ are irreducible by the Howe duality and the $L$-parameter of $\pi^{\pm}$ are $\phi$ by Lemma \ref{parameter}.
	\begin{comment}
	\begin{align*}
	\theta_{V,W,\psi}(\sigma^+)&=\pi^+ \\
	\theta_{V,W,\psi}(\sigma^-)&=\pi^
	\end{align*}
	\end{comment}
	Since $\mathrm 1\subseteq \phi$, it follows from Proposition \ref{1notinphi} that $\pi^+\not \cong \pi^-\otimes \det$. Note that both $\pi^+$ and $\pi^-\otimes\det$ lies in $\Ind_{P}^{\mathrm O\left(V\right)}\left(\tau\otimes \pi_0\right)$, this implies $\Ind_{P}^{\mathrm O\left(V\right)}\left(\tau \otimes \pi_0\right)$ is reducible.
	
	It remains to show that 
	$$\left|\JH(\Ind_{P}^{\mathrm O(V)}(\tau\otimes \pi_0))\right|=2.$$ 
	Again we consider the set 
	\begin{align*}
	\bigsqcup_{\pi_0\in \Pi_{\phi_0}} \JH(\Ind_{P}^{\mathrm O(V)}(\tau\otimes \pi_0)). 
	\end{align*}
	By the Howe duality, Lemma \ref{parameter} and Lemma \ref{multiplicityfree}, this set is a subset of $\Pi_{\phi}$. Hence by \ref{1236}, we have 
	\begin{align*}
	|\Pi_{\phi}|\geq \left|\bigsqcup_{\pi_0\in \Pi_{\phi_0}} \JH(\Ind_{P}^{\mathrm O(V)}(\tau\otimes \pi_0)) \right|\geq 2|\Pi_{\phi_{0}}|. 
	\end{align*}
	On the other hand, by Proposition \ref{bijection}, we have 
	\begin{align*}
	|\Pi_{\phi}|=|\widehat{\mathcal {S}_{\phi}}|=2|\widehat{S_{\phi_0}}|=2|\Pi_{\phi_{0}}|. 
	\end{align*}
	Hence we must have 
	\begin{align*}
	\left| \JH(\Ind_{P}^{\mathrm O(V)}(\tau\otimes \pi_0)) \right|=2
	\end{align*}
	for all $\pi_0\in \Pi_{\phi_{0}}$. This completes the proof. 
\end{proof}


\subsection{Charater of component group}
For any irreducible representation $\pi\subseteq \Ind_{P}^{\mathrm O(V)}(\tau\otimes \pi_0)$, we have shown that the $L$-parameter of $\pi$ is $\phi$ in the previous subsection. In this subsection, we study the map $\mathcal J_{\mathfrak W_{c}}$ for a fixed Whittaker datum $\mathfrak W_{c}$ of $\bigsqcup_{V_{2n}^{\bullet}}\mathrm O(V_{2n}^\bullet)$. We divide it into three cases: 
\begin{itemize}
	\item  \underline{Case A}: $1\not\subseteq \phi_{0}$ and $\phi_{\tau}\neq \mathrm 1$;
	\item \underline{Case B}: $1\subseteq \phi_{0}$;
	\item \underline{Case C}: $1\not\subseteq \phi_{0}$ and $\phi_{\tau}= \mathrm 1$.
\end{itemize}
Put 
\begin{align*}
\phi^+&=(\phi\otimes \chi_{V})\oplus \chi_{V} ,\\
\phi^+_{0}&=(\phi_{0}\otimes\chi_{V})\oplus \chi_{V}.
\end{align*}
Then the following diagram
\begin{align}\label{101}
\begin{CD}
\mathcal S_{\phi_0} @>>> \mathcal S_{\phi^+_{0}} \\
@VVV @VVV \\
\mathcal {S}_{\phi} @>>> \mathcal S_{\phi^+}
\end{CD}
\end{align}
is commutative. 

\begin{proposition}\label{localintertwing}
Assume that we are in Case A or Case B. 
\begin{enumerate}[(i)]
	\item Put $\mathcal J_{\mathfrak W_{c}}(\pi_0)=\eta_0$ and $\mathcal J_{\mathfrak W_{c}}(\pi)=\eta$. Then 
	\begin{align*}
	\eta|_{\mathcal S_{\phi_0}}=\eta_0. 
	\end{align*}
	\item If we further assume that $\phi_{\tau}$ is self-dual and of orthogonal type, then the restriction of the normalized intertwining operator $R_{\mathfrak W_{c}}(w,\tau\otimes \pi_0)$ to $\pi$ is the scalar multiplication by $\eta(a)$, where $a\in \mathcal {S}_{\phi}$ corresponding to $\phi_{\tau}$. 
\end{enumerate} 
\end{proposition}
\begin{proof}
We first prove statement (i). 

In \underline{Case A}, it follows from Lemma \ref{parameter} that the $L$-parameter for $\pi$ is $\phi$. Since $\mathrm 1\not \subseteq \phi$, by Corollary \ref{1inphi}, we have $$\theta_{W,V,\psi}(\pi)\neq 0.$$ 
Then Lemma \ref{12} implies that  $$\theta_{W_0,V_0,\psi}(\pi_0)\neq 0.$$ 
Let  
\begin{align*}
\sigma=\theta_{W,V,\psi}(\pi), \quad \sigma_0=\theta_{W_0,V_0,\psi}(\pi_0). 
\end{align*}
It follows from Proposition \ref{prasad} that 
\begin{align*}
\mathcal L(\sigma)=\phi^+\quad \mbox{and}\quad \mathcal L(\sigma_0)=\phi_{0}^+. 
\end{align*}
Put $\mathcal J_{\mathfrak W^\prime_{\psi,c}}(\sigma)=\eta^+$ and $\mathcal J_{\mathfrak W^\prime_{\psi,c}}(\sigma_0)=\eta^+_0$, we have  
	\begin{align*}
	\eta|_{\mathcal S_{\phi_0}}&= \left(\eta^+|_{ \mathcal {S}_{\phi}}\right)|_{\mathcal S_{\phi_0}} & \quad \quad \mbox{(by our construction of $\eta$)} \\
	&= \left(\eta^+|_{ \mathcal S_{\phi_0^+}}\right)|_{ \mathcal S_{\phi_0}} & \quad \quad  \mbox{(by the commutative diagram \ref{101})}\\
	&=\eta_{0}^+|_{ \mathcal S_{\phi_0}} & \mbox {(by Theorem \ref{llcsympletic} (6))} \\
	&=\eta_0.  & \mbox{(by our construction of $\eta_0$)} 
	\end{align*}
	
In \underline{Case B}, it follows from Lemma \ref{parameter} that the $L$-parameter for $\pi$ is $\phi$. Since $\mathrm 1 \subseteq \phi$, by Proposition \ref{1notinphi}, exactly one of 
\begin{align*}
\theta_{W,V,\psi}(\pi) \quad \mbox{and}\quad \theta_{W,V,\psi}(\pi\otimes\det)
\end{align*}
is non-zero. If $\theta_{W,V,\psi}(\pi)\neq 0$, then we may apply the same argument in Case A to $\pi$. So we may assume that $$\theta_{W,V,\psi}(\pi)= 0 \quad \mbox{and}\quad  \theta_{W,V,\psi}(\pi\otimes\det)\neq 0.$$ 
Then by Lemma \ref{12}, we have 
\begin{align*}
\theta_{W_0,V_0,\psi}(\pi_0)= 0 \quad \mbox{and}\quad  \theta_{W_0,V_0,\psi}(\pi_0\otimes\det)\neq 0.
\end{align*}
By a similar argument to Case $A$, we have 
\begin{align}\label{53}
\mathcal J_{\mathfrak W_{c}}(\pi\otimes\det)|_{\mathcal S_{\phi_0}}= \mathcal J_{\mathfrak W_{c}}(\pi_0\otimes\det).
\end{align}
On the other hand, since $1\subseteq \phi_0$, it follows from our construction of $\mathcal J_{\mathfrak W_{\psi}}$ that 
\begin{equation}
\begin{aligned}\label{54}
\eta=\mathcal J_{\mathfrak W_{c}}(\pi)&= \mathcal J_{\mathfrak W_{c}}(\pi\otimes\det)\otimes\kappa_{\phi},\\
\eta_0=\mathcal J_{\mathfrak W_{c}}(\pi_0)&= \mathcal J_{\mathfrak W_{c}}(\pi_0\otimes\det)\otimes\kappa_{\phi_0}.
\end{aligned}
\end{equation}
Hence by \ref{53} and \ref{54}, we have $\eta|_{\mathcal S_{\phi_0}}=\eta_{0}$. Here we use the fact that $\kappa_{\phi}|_{\mathcal S_{\phi_{0}}}=\kappa_{\phi_{0}}$. 

We then prove the statement (ii). It follows from the Lemma \ref{multiplicityfree} and Schur's lemma that  $R_{\mathfrak W_{c}}(\omega,\tau\otimes \pi_0)$ acts on $\pi$ by scalar multiplication. 

In \underline{Case A}, we have 
$$\sigma=\theta_{W,V,\psi}(\pi)\neq 0\quad \mbox{and} \quad \sigma_0=\theta_{W_0,V_0,\psi}(\pi_0)\neq 0.$$ 
Recall that we have construct the $\mathrm O(V)\times \SP(W)$-equivariant map 
$$\mathcal{T}_{0} : \omega\otimes \operatorname{Ind}_{Q}^{\SP(W)}\left(\tau\chi_{V} \otimes \sigma_{0}^{\vee}\right) \rightarrow \operatorname{Ind}_{P}^{\mathrm{O}\left(V\right)}\left(\tau  \otimes \pi_0\right)$$
in subsection \ref{constructionof}. By Lemma \ref{12}, we have 
\begin{align*}
\sigma \subseteq \Ind_{Q}^{\SP(W)}(\tau\chi_{V}\otimes \sigma_{0}),
\end{align*}
 then 
$$\sigma^{\vee}\subseteq \Ind_{Q}^{\SP(W)}(\tau\chi_{V}\otimes \sigma_{0}^{\vee}).$$
Here we use the assumption that $\tau$ is self-dual and the fact that $\Ind_{Q}^{\SP(W)}(\tau\chi_{V}\otimes \sigma_{0})$ is semi-simple. It follows from the Howe duality and Proposition \ref{intertwingmap} that the restriction of $\mathcal T_0$ to $\omega\otimes \sigma^{\vee}$ gives an epimorphism 
	$$
	\mathcal T_0 :\omega\otimes \sigma^{\vee} \rightarrow \pi. 
	$$ 
Then by Corollary \ref{123}, we have 
	\begin{align}\label{81}
	R_{\mathfrak W_{c}}(w,\tau\otimes \pi_0)|_\pi=\omega_{\tau}(-c)\cdot \chi_{V}(-c)^k\cdot  R_{\mathfrak W^\prime_{\psi,1}}(w^{\prime},\tau\chi_{V}\otimes \sigma_0) |_{\sigma^{\vee}}. 
	\end{align}
 It follows from Theorem \ref{llcsympletic} (6) and \ref{104} that 
 \begin{equation}\label{82}
 \begin{aligned}
 	R_{\mathfrak W^\prime_{\psi,1}}(w^{\prime},\tau\chi_{V}\otimes \sigma_0) |_{\sigma^{\vee}}=&\mathcal J_{\mathfrak W^\prime_{\psi,1}}(\sigma^\vee)(a^\prime)\\
 	=&\mathcal J_{\mathfrak W^\prime_{\psi,1}}(\sigma)(a^\prime)\cdot \eta_{\phi^+,-1}(a^\prime)\\
 =&\mathcal J_{\mathfrak W^\prime_{\psi,1}}(\sigma)(a^\prime)\cdot \omega_{\tau}(-1)\cdot \chi_{V}(-1)^k,
 \end{aligned}
 \end{equation}
	where $a^{\prime}\in \mathcal S_{\phi^+}$ corresponding to $\phi_{\tau}\otimes \chi_{V}$. On the other hand, by Theorem \ref{llcsympletic} (4), we have 
	\begin{align}\label{84}
	\mathcal J_{\mathfrak W^\prime_{\psi,c}}(\sigma)(a^{\prime})=\mathcal J_{\mathfrak W^\prime_{\psi,1}}(\sigma)(a^{\prime})\cdot \omega_{\tau}(c)\cdot \chi_{V}(c)^{ k}. 
	\end{align}
	Recall that by our construction of $\mathcal J_{\mathfrak W_{c}}$, we also have 
	\begin{align}\label{83}
	\mathcal J_{\mathfrak W_{c}}(\pi)(a)=\mathcal J_{\mathfrak W^\prime_{c}}(\sigma)(a^{\prime}).
	\end{align}
	Combining the equalities \ref{81}, \ref{82}, \ref{84} and \ref{83}, we deduce  
	\begin{comment}
	\begin{align*}
	R_{c^{\prime}}(\omega,\tau\otimes \pi_0)|\pi=\omega_{\tau}(1/c^{\prime})\cdot \chi_{V}(c^{\prime})^k\cdot \eta(\sigma)(a^{\prime})\\
	=\eta_{c^{\prime}}(\sigma)(a^{\prime})= \eta_{c^{\prime}}(\pi)(a)
	\end{align*}
	\end{comment}
	\begin{align*}
	R_{\mathfrak W_{c}}(\omega,\tau\otimes \pi_0)|_\pi= \mathcal J_{\mathfrak W_c}(\pi)(a). 
	\end{align*}
	
	In \underline{Case B}, if $\theta_{W,V,\psi}(\pi)\neq 0$, then we may apply the same argument in Case A to $\pi$. So we may assume that $$\theta_{W,V,\psi}(\pi)=0\quad \mbox{and}\quad \theta_{W,V,\psi}(\pi\otimes\det)\neq 0.$$ 
	By a similar argument to Case $A$, we have 
	\begin{align}\label{60}
	R_{\mathfrak W_{c}}(w,\tau\otimes (\pi_0\otimes\det))|_{\pi\otimes\det}= \mathcal J_{\mathfrak W_c}(\pi\otimes
	\det)(a).
	\end{align}
  Since $1\subseteq\phi$, it follows from our construction for $\mathcal J_{\mathfrak W_{\psi}}$ in section \ref{constructJc} that 
	\begin{equation}\label{555}
	\begin{aligned}
	\eta=\mathcal J_{\mathfrak W_{c}}(\pi)&= \mathcal J_{\mathfrak W_{c}}(\pi\otimes\det)\otimes\kappa_{\phi}.
	\end{aligned}
	\end{equation}
	On the other hand, by \ref{comparenormalize}, we have 
	\begin{align}\label{61}
	R_{\mathfrak W_{c}}(w,\tau\otimes (\pi_0\otimes\det))|_{\pi\otimes\det}=(-1)^{\dim\tau}\cdot R_{\mathfrak W_{c}}(w,\tau\otimes \pi_0)|_\pi .
	\end{align}
	So by \ref{60}, \ref{555} and \ref{61}, we deduce 
	\begin{align*}
   \eta(a)=&\mathcal J_{\mathfrak W_c}(\pi)(a)\\
   =&\mathcal J_{\mathfrak W_c}(\pi\otimes
   \det)(a)\times \kappa_{\phi}(a)\\ 
   =& R_{\mathfrak W_{c}}(w,\tau\otimes (\pi_0\otimes\det))|_{\pi\otimes\det}\times (-1)^{\dim \tau}\\
   =&R_{\mathfrak W_{c}}(w,\tau\otimes \pi_0)|_\pi. 
   \end{align*}	
   This finishes the proof.
\end{proof}
\begin{remark}
	To apply the similar argument to Case C, we need the formula 
	\begin{align*}
	\mathcal J_{\mathfrak W_c}(\pi_0\otimes\det)=\mathcal J_{\mathfrak W_c}(\pi_0)\otimes \kappa_{\phi_{0}}\quad \mbox{for}\,\, \pi_{0}\in \Pi_{\phi_0}
	\end{align*}
	in the case when $1\not\subseteq \phi_{0}$. But this does not directly follow from our construction. We will prove this formula in the next proposition.
\end{remark}



Let $\pi\in \Pi_{\phi}$, then $\pi\otimes\det \in \Pi_{\phi}$ by Proposition \ref{det}. We compare $\mathcal J_{\mathfrak W_c}(\pi)$ with $\mathcal J_{\mathfrak W_c}(\pi\otimes\det)$: 
\begin{proposition}\label{determint}
Let  $\phi\in \Para(\mathrm O(V_{2n}))$ and $\pi\in \Pi_{\phi}$. Then we have 
\begin{align*}
\mathcal J_{\mathfrak W_c}(\pi\otimes\det)=\mathcal J_{\mathfrak W_c}(\pi)\otimes\kappa_{\phi},
\end{align*}
where $\kappa_{\phi}$ is defined in \ref{kappa}. 
\end{proposition}
\begin{proof}
 The proof follows an idea in \cite{MR3788848}. If $1\subseteq \phi$, then this follows from our construction of $\mathcal J_{\mathfrak W_{c}}$ in section \ref{constructJc}. So we assume that $1\not\subseteq \phi$, write 
 \begin{align*}
 \phi=m_1\phi_1\oplus \cdots \oplus m_l \phi_l\oplus \varphi\oplus \varphi^\vee,
 \end{align*}
 where $\phi_i$ are pairwise distinct irreducible $k_i$-dimensional orthogonal representations of $\WD_{F}$ and $\varphi$ is a sum of irreducible tempered representations of $\WD_F$ which are not orthogonal. Then  
 \begin{align*}
 \mathcal {S}_{\phi}=\bigoplus_{i=1}^l (\mathbb Z/2\mathbb Z)a_i,
 \end{align*} 
 where $a_i$ corresponding to $\phi_i$.
 
 For any $i\in \{1,2\cdots l\}$, let $\tau_{i}$ be the irreducible unitary representation of $\GL_{k_{i}}(F)$ corresponding to  $\phi_{i}$. We consider the induced representation $$\Ind_{P^\prime}^{\mathrm O(V^\prime)}(\tau_i\otimes\pi)\quad \mbox{and}\quad \Ind_{P^\prime}^{\mathrm O(V^\prime)}(\tau_i\otimes(\pi\otimes \det)),$$
 where $V^\prime= V\oplus \mathbb H^k$  and $P^\prime$ is a parabolic subgroup of $\mathrm O(V^\prime)$ with Levi component $M_{P^\prime}\cong \mathrm O(V)\times \GL_{k_{i}}(F)$. Put
	\begin{align*}
	\phi^\prime =\phi_i \oplus \phi \oplus \phi_i^{\vee}.
	\end{align*}
	Since $\phi_i\subseteq \phi$, the inclusion $\mathcal {S}_{\phi}\hookrightarrow \mathcal S_{\phi^\prime}$ is an isomorphism. It then follows from Proposition \ref{det}, Lemma \ref{parameter} and Corollary \ref{embeddingisisomorphism} that both $\Ind_{P^\prime}^{\mathrm O(V^\prime)}(\tau_i\otimes\pi)$ and  $\Ind_{P^\prime}^{\mathrm O(V^\prime)}(\tau_i\otimes(\pi\otimes\det))$ are irreducible and have $L$-parameter $\phi^\prime$. Write $$\pi^\prime = \Ind_{P^\prime}^{\mathrm O(V^\prime)}(\tau_i\otimes\pi),$$
	then by \ref{59}, we have 
  $$\pi^\prime\otimes\det \cong \Ind_{P^\prime}^{\mathrm O(V^\prime)}(\tau_i\otimes(\pi\otimes\det)).$$ If we identify $\mathcal S_{\phi^\prime}$ with $\mathcal {S}_{\phi}$ through the natural isomorphism, then it follows from Proposition \ref{localintertwing} (Case A) that 
\begin{equation}\label{57}
\begin{aligned}
\mathcal J_{\mathfrak W_c}(\pi)(a_i)=\mathcal J_{\mathfrak W_c}(\pi^\prime)(a_i)=&R_{\mathfrak W_c}(w,\tau_{i}\otimes\pi)|_{\pi^\prime},\\
\mathcal J_{\mathfrak W_c}(\pi\otimes\det)(a_i)=\mathcal J_{\mathfrak W_c}(\pi^\prime\otimes\det)(a_i)=&R_{\mathfrak W_c}(w,\tau_{i}\otimes(\pi\otimes\det))|_{\pi^\prime\otimes\det}.
\end{aligned}
\end{equation}
On the other hand, by \ref{comparenormalize}, we know that 
	\begin{align}\label{58}
	 R_{\mathfrak W_c}(w,\tau_{i}\otimes(\pi\otimes\det))|_{\pi^\prime\otimes\det}=(-1)^{k_i} R_{\mathfrak W_c}(w,\tau_{i}\otimes\pi)|_{\pi^\prime}.
	\end{align}
Hence by \ref{kappa}, \ref{57} and \ref{58}, we have 
	\begin{align*}
	\mathcal J_{\mathfrak W_c}(\pi)(a_i)= \mathcal J_{\mathfrak W_c}(\pi\otimes\det)(a_i)\cdot (-1)^{k_i}=  \mathcal J_{\mathfrak W_c}(\pi\otimes\det)(a_i)\cdot \kappa_{\phi}(a_i). 
	\end{align*}
This finishes the proof. 	
\end{proof}



Now we can prove Proposition \ref{localintertwing} for Case C.
\begin{corollary}\label{CaseC}
	Assume that we are in Case C, i.e., $1\not\subseteq \phi_0$ and $\phi_\tau=1$. 
	\begin{enumerate}[(i)]
		\item Put $\mathcal J_{\mathfrak W_c}(\pi_0)=\eta_0$ and $\mathcal J_{\mathfrak W_c}(\pi)=\eta$, then 
		\begin{align*}
		\eta|_{\mathcal S_{\phi_0}}=\eta_0.
		\end{align*}
		\item If we further assume $\phi_{\tau}$ is self-dual and of orthogonal type, then the restriction of the normalized intertwining operator $R_{\mathfrak W_c}(\omega,\tau\otimes \pi_0)$ to $\pi$ is the scalar multiplication by $\eta(a)$, where $a\in \mathcal {S}_{\phi}$ corrresponds to $\phi_{\tau}$. 
	\end{enumerate} 
\end{corollary}
\begin{proof}
	It follows by Proposition \ref{determint} that 
	$$\mathcal J_{\mathfrak W_c}(\pi_0\otimes\det)=\mathcal J_{\mathfrak W_c}(\pi_0)\otimes \kappa_{\phi_{0}}.$$
	So now we can apply the same argument in Proposition \ref{localintertwing} for Case B to this case.
\end{proof}



Combining Lemma \ref{parameter}, Corollary \ref{embeddingisisomorphism}, Corollary \ref{notanisomorphism}, Proposition\ref{localintertwing} and Corollary \ref{CaseC}, we deduce the local intertwing relation  for even orthogonal groups: 
\begin{corollary}\label{LIRfinal}
	The map $\mathcal L$ and $\mathcal J_{\mathfrak W_c}$ we constructed satisfy the local intertwining relation in Theorem \ref{desideratumall} (9).  
\end{corollary}
\begin{comment}
\begin{proposition}\label{induceddecomposition}
There is a decomposition of the induced representation 
$$
\Ind_{P}^{\mathrm O(V)}(\tau\otimes \pi_0)=\bigoplus_{\eta}\pi_{c^{\prime}}(\phi,\eta)
$$
where the sum runs for all $\eta\in \widehat{\mathcal {S}_{\phi}^{+}}$ such that $\eta|_{S_{\phi_{0}}^{+}}=\eta_0$
\end{proposition}
This Lemma together with Lemma \ref{localintertwing} gives us the local intertwing relation for Even orthogonal groups, which is  Desideratum \ref{desideratumorthgonal}(7). 
\end{comment}



\section{Comparison with LLC à la Arthur}
Let $\mathfrak W_c$ be a Whittaker datum of $\mathrm O(V_{2n}^+)$. In this section, we shall prove that our parameterization maps $\mathcal L$ and $\mathcal J_{\mathfrak W_c}$ equal to Arthur's parameterization maps $\mathcal L^A$ and $\mathcal J^A_{\mathfrak W_c}$ for $\mathrm O(V_{2n}^+)$.  

Let $\pi\in \Irr(\mathrm O(V_{2n}^+))$. Recall that in section \ref{construction}, we associated a pair
\begin{align*}
\left(\phi= \mathcal L(\pi), \quad \eta=\mathcal J_{\mathfrak W_c}(\pi)\right)
\end{align*}
to $\pi$. Also, in Theorem \ref{Arthurorth}, Arthur and Atobe-Gan also associated a pair 
\begin{align*}
\left(\phi^A= \mathcal L^A(\pi), \quad \eta^A=\mathcal J^A_{\mathfrak W_c}(\pi)\right)
\end{align*}
to $\pi$. 
\begin{theorem}\label{comparearthur}
	We have 
	\begin{align*}
	\phi=\phi^A \quad \mbox{and}\quad \eta=\eta^A. 
	\end{align*}
\end{theorem}
\begin{proof}
By Proposition \ref{Langlandsclasification} and Theorem \ref{Arthurorth}, both two LLC are compatible with Langlands quotients. Without loss of generality, we may assume that $\pi$ is tempered. It follows by Lemma \ref{respectplancherel} and Theorem \ref{Arthurorth} that 
\begin{align*}
\mu_{\psi}(\tau_s\otimes \pi)&=\gamma(s,\phi_{\tau}\otimes \phi^{ \vee},\psi)\cdot \gamma(-s,\phi_{\tau}^{\vee}\otimes\phi, \psi_{-1} )\\
&\quad  \times \gamma (2s, \wedge^2\circ \phi_{\tau}, \psi)\cdot \gamma(-2s, \wedge^{2} \circ \phi_{\tau}^{\vee},\psi_{-1})
\end{align*}
and 
\begin{align*}
\mu_{\psi}(\tau_s\otimes \pi)&=\gamma(s,\phi_{\tau}\otimes (\phi^{ A})^\vee,\psi)\cdot \gamma(-s,\phi_{\tau}^{\vee}\otimes\phi^A, \psi_{-1} )\\
&\quad  \times \gamma (2s, \wedge^2\circ \phi_{\tau}, \psi)\cdot \gamma(-2s, \wedge^{2} \circ \phi_{\tau}^{\vee},\psi_{-1})
\end{align*}
for any $\tau\in \Irr (\GL_k(F))$, where $\phi_{\tau}$ is the $L$-parameter of $\tau$. Hence
\begin{align*}
\gamma(s,\phi_{\tau}\otimes \phi^{ \vee},\psi)\cdot \gamma(-s,\phi_{\tau}^{\vee}\otimes\phi, \psi_{-1} )=\gamma(s,\phi_{\tau}\otimes (\phi^{ A})^\vee,\psi)\cdot \gamma(-s,\phi_{\tau}^{\vee}\otimes\phi^A, \psi_{-1} ).
\end{align*}
Then by Lemma \ref{gammadetermine}, we deduce that $\phi=\phi^A$. 

Next we prove $\eta=\eta^A$. Write 
	\begin{align*}
	\phi=m_1\phi_1\oplus \cdots \oplus m_l \phi_l\oplus \varphi\oplus \varphi^\vee,
	\end{align*}
	where $\phi_i$ are pairwise distinct irreducible orthogonal representation of $\WD_{F}$ and $\varphi$ is a sum of irreducible tempered representations of $\WD_F$ which are not orthogonal. Then 
	\begin{align*}
	\mathcal {S}_{\phi}=\bigoplus_{i=1}^l (\mathbb Z/2\mathbb Z)a_i,
	\end{align*} 
	where $a_i$ corresponds to $\phi_i$. 
	
For any $i\in \{1,2,\cdots, l\}$, let $\tau_{i}$ be the irreducible unitary representation of $\GL_{k_{i}}(F)$ corresponding to $\phi_{i}$. We consider the induced representation $\widetilde{\pi}_i=\Ind_{P^\prime}^{\mathrm O(V^\prime)}(\tau_i\otimes\pi)$, where $V^\prime= V\oplus \mathbb H^k$ and $P^\prime$ is a parabolic subgroup of $\mathrm O(V^\prime)$ with Levi component $$M_{P^\prime}\cong \mathrm O(V)\times \GL_{k_{i}}(F).$$
It follows from Corollary \ref{LIRfinal} and Theorem \ref{Arthurorth} that $\widetilde{\pi}_i$ is irreducible, with $L$-parameter 
\begin{align*}
\phi^\prime =\phi_i \oplus \phi \oplus \phi_i^{\vee}.
\end{align*}
and corresponds to 
\begin{align*}
\begin{cases*}
\eta \quad &\mbox{under $\mathcal J_{\mathfrak W_c}$},\\
\eta^A \quad &\mbox{under $\mathcal J^A_{\mathfrak W_c}$}. 
\end{cases*}
\end{align*}
Here we identify $\mathcal S_{\phi^\prime}$ with $\mathcal {S}_{\phi}$ through the natural isomorphism $\mathcal S_{\phi^\prime}\cong \mathcal {S}_{\phi}$. Let $R_{\mathfrak W_c}(w,\tau_{i}\otimes\pi)$ be the normalized intertwining operator defined in subsection \ref{normalizingintertwing}. Apply Corollary \ref{LIRfinal} and Theorem \ref{Arthurorth} again, we  have 
\begin{equation}\label{102}
\eta(a_i)=R_{\mathfrak W_c}(w,\tau_{i}\otimes\pi)|_{\widetilde{\pi}_i}=\eta^A(a_i). 
\end{equation}
Hence 	
\begin{align*}
\eta=\eta^A.
\end{align*}
This completes the proof. 
\end{proof}


We then prove Theorem \ref{desideratumall} (6) holds for our construction: 
\begin{corollary}\label{plus}
	Let $\mathfrak W_c$ be a Whittaker datum of $\bigsqcup_{V_{2n}^{\bullet}}\mathrm O(V_{2n}^\bullet)$. Then $\pi$ is a representation of $\mathrm O(V_{2n}^+)$ iff 
	\begin{align*}
	\mathcal J_{\mathfrak W_c}(\pi)(z_{\phi})=\chi_{V}(c).
	\end{align*}
\end{corollary}
\begin{proof}
	We divided the proof into two cases: 
	
	$\bullet$ \underline{Case I}: If $\mathfrak W_c$ is a Whittaker datum of $\mathrm O(V_{2n}^+)$, i.e., $c\in N_{E/F}(E^\times)$, we have $\chi_{V}(c)=1$. Then this case follows from Theorem \ref{Arthurorth} and Theorem \ref{comparearthur}. 
	
	
	$\bullet$ \underline{Case II}: If $\mathfrak W_c$ is a Whittaker datum of $\mathrm O(V_{2n}^-)$, i.e., $c\notin N_{E/F}(E^\times)$. Then this case follows from Case I and Proposition \ref{changeofwhittakerdatum}. 
\end{proof}

\section{Completion of the proof}
So far, we have proved that our construction of LLC satisfies Theorem \ref{desideratumall} (1), (2), (4), (5), (6), (9), (10), (11), (12). In this section, we shall prove our construction of LLC satisfies (3), (7) and (8) in Theorem \ref{desideratumall}, hence complete the proof of Theorem \ref{desideratumall}. 


We first prove that our construction satisfies Theorem \ref{desideratumall} (8): 
\begin{proposition}\label{dettwistfinal}
Let $\phi\in \Phi(\mathrm O(V_{2n}))$ and $\pi\in \Pi_{\phi}$. Then the determinant twist $\pi\otimes \det$ also belongs to $\Pi_{\phi}$, and 
$$\mathcal J_{\mathfrak W_{c}}(\pi \otimes \det)=\mathcal J_{\mathfrak W_{c}}(\pi)\otimes  \kappa_{\phi}.$$
\end{proposition}
\begin{proof}
	By our construction in subsection \ref{nontempered}, it is enough to prove this when $\phi\in \Para(\mathrm O(V_{2n}))$. Then this follows from Proposition \ref{det} and Proposition \ref{determint}. 
\end{proof}

Before proving Theorem \ref{desideratumall} (3), we do some preparations. Let $\mathrm O(V_{2n})$ be a quasi-split even orthogonal group and $\mathfrak W_c$ be a Whittaker datum of $\mathrm O(V_{2n})$. Let $U^\prime, \widetilde{U}$ and $\mu^\prime_c, \mu_c^+$ be those defined in subsection \ref{whittaker}. The following lemma computes the Whittaker module of the Weil representation $\omega=\omega_{V_{2n},W_{2n},\psi}$.  
\begin{lemma}\label{Jacquetmodule}
	 We have 
	\begin{align*}
	\omega_{(U^\prime,\mu^\prime_c)}\cong \ind_{\widetilde{U}}^{\mathrm O(V_{2n})}\mu_c^+. 
	\end{align*}
\end{lemma}
\begin{proof}
	See \cite{MR1454260} and \cite{MR1738175} for an analogous computation. We omit the details. 
\end{proof}
Let $\pi\in \Irrt(\mathrm O(V_{2n}))$. We calculate $$\Hom_{\mathrm O(V_{2n})\times U^\prime}(\omega, \pi\boxtimes \mu_c^\prime)$$ in two different ways. On one hand, we have 
\begin{align*}
\Hom_{\mathrm O(V_{2n})\times U^\prime}(\omega, \pi\boxtimes \mu_c^\prime)&\cong \Hom_{U^\prime}(\Theta_{W_{2n},V_{2n}, \psi}(\pi), \mu_c^\prime)\\
&\cong  \Hom_{U^\prime}(\theta_{W_{2n},V_{2n}, \psi}(\pi), \mu_c^\prime),
\end{align*}
where the last equality follows from Lemma \ref{temperedtotempered}. On the other hand, by Lemma \ref{Jacquetmodule}, we have 
\begin{align*}
\Hom_{\mathrm O(V_{2n})\times U^\prime}(\omega, \pi\boxtimes \mu_c^\prime)&\cong \Hom_{\mathrm O(V_{2n})}\left(\ind_{\widetilde{U}}^{\mathrm O(V_{2n})}\mu_c^+, \pi\right)\\
&\cong \Hom_{\mathrm O(V_{2n})}\left(\pi^\vee, \Ind_{\widetilde{U}}^{\mathrm O(V_{2n})}(\mu_c^+)^\vee\right)\\
&\cong \Hom_{\widetilde{U}}\left(\pi^\vee, (\mu_c^+)^\vee\right)\\
&\cong \Hom_{\widetilde{U}}\left(\pi, \mu_c^+\right).
\end{align*}
Here the last equality follows from the fact that both $\pi$ and $\mu_{c}^+$ are unitary. So we deduce the following proposition:  
\begin{proposition}\label{generic to generic}
	Let $\mathrm O(V_{2n})$ be a quasi-split even orthogonal group and $\mathfrak W_c$ be a Whittaker datum of $\mathrm O(V_{2n})$. Then $\pi\in \Irrt(\mathrm O(V_{2n}))$ is $\mathfrak W_c^+$-generic iff $\sigma=\theta_{W_{2n},V_{2n},\psi}(\pi)$ is $\mathfrak W_{\psi,c}$-generic.   	
\end{proposition}

 

We then prove Theorem \ref{desideratumall} (3): 
\begin{proposition}\label{generic}
	Assume that $\phi\in \Para(\mathrm O(V_{2n}))$, then for each Whittaker datum $\mathfrak W_{c}$ of $\bigsqcup_{V_{2n}^{\bullet}}\mathrm O(V_{2n}^\bullet)$,
	\begin{itemize}
		\item there is an unique $\mathfrak W^+_{c}$-generic representation $\pi$ in $\Pi_{\phi}$ and $\mathcal J_{\mathfrak W_{c}}(\pi)$ is the trivial representation of $\mathcal {S}_{\phi}$; 
		\item there is an unique $\mathfrak W^-_{c}$-generic representation $\pi$ in $\Pi_{\phi}$ and $\mathcal J_{\mathfrak W_{c}}(\pi)=\kappa_{\phi}$.  
	\end{itemize}
\end{proposition}
\begin{proof}
	The first statement follows from Theorem \ref{llcsympletic} (3),  Proposition \ref{generic to generic} and our construction of $\mathcal J_{\mathfrak W_{c}}$. Note that $\pi$ is $\mathfrak W^+_{c}$-generic iff $\pi\otimes\det$ is $\mathfrak W^{-}_{c}$-generic, so the second statement follows from the first statement and Proposition \ref{determint}.    
\end{proof}


Finally we prove Theorem \ref{desideratumall} (7): 
\begin{proposition}
Under the LLC we constructed, the following are equivalent: 
	\begin{itemize}
		\item 	$\phi \in \Phi^{\epsilon}\left(\mathrm{O}\left(V_{2 n}\right)\right)$;
		\item some $\pi\in \Pi_{\phi}$ satisfies $\pi \otimes \det \neq \pi$;
		\item 	
		all $\pi\in \Pi_{\phi}$ satisfy $\pi \otimes\det 
		\neq \pi$. 
	\end{itemize} 
\end{proposition}
\begin{proof}
	Note that for $\phi\in \Phi(\mathrm O(V_{2n}))$, we have 
	\begin{align*}
	\kappa_{\phi}\neq 1 \quad \mbox{iff $\phi\in \Phi^{\epsilon}(\mathrm O(V_{2n}))$}.
	\end{align*}
	Then this Proposition follows from Theorem \ref{desideratumall} (2) and (8), which we have proved in Proposition \ref{bijection2} and Proposition \ref{dettwistfinal}. 
\end{proof}

So we have completed the proof of Theorem \ref{desideratumall}. 


\appendix
\section{LLC for special even orthogonal groups}
In \cite{MR3135650}, Arthur established a weaker version LLC for quasi-split special even orthogonal groups from the LLC for quasi-split even orthogonal groups. This was explicated by Atobe-Gan \cite{MR3708200}. Since we now construct the LLC for even orthogonal groups, following Arthur's idea, we can deduce a weaker version LLC for special even orthogonal groups. We shall do it in this appendix. 

Let $V=V_{2n}$ be an orthogonal space and $\chi_V$ be the discriminant character of $V$. By \cite[\S 8]{MR3202556} and \cite[\S 3]{MR3708200}, an $L$-parameter for the special orthogonal group $\SO(V_{2n})$ is a $2n$-dimensional orthogonal representation $\phi$ of $\WD_{F}$. We define
\begin{align*}
\Phi(\SO(V_{2n}))=\{\phi: \WD_{F} \rightarrow \mathrm{O}(2n, \mathbb{C}) | \det(\phi)=\chi_{V}\} /(\SO(2n, \mathbb{C})\mbox{-conjugacy}).
\end{align*} 
Note that $\Phi(\SO(V_{2n}))$ is different from $\Phi(\mathrm O(V_{2n}))$ since we modulo the $\SO(2n,\mathbb C)$-conjugacy rather than $\mathrm O(2n,\mathbb C)$-conjugacy here. There is a natural surjective map 
\begin{align}\label{523}
\Phi(\SO(V_{2n}))\twoheadrightarrow \Phi(\mathrm O(V_{2n})),  
\end{align} 
we define $\Phi^\epsilon(\SO(V_{2n}))$ to be preimage of $\Phi^\epsilon(\mathrm O(V_{2n}))$. It is easy to check that the map \ref{523} is bijective on the subset $\Phi^\epsilon(\SO(V_{2n}))$ and is a two-to-one map on $\Phi(\SO(V_{2n}))\setminus \Phi^\epsilon(\SO(V_{2n}))$. 

As in subsection \ref{whittaker}, for every $c\in F^\times/F^{\times 2}$, we have a Whittaker datum $\mathfrak W_{c}$ of $\bigsqcup_{V_{2n}^{\bullet}}\SO(V_{2n}^{\bullet})$. Next we state the local Langlands correspondence for special even orthogonal groups, the reader can consult \cite[\S 3.3]{MR3708200} for a detailed description. 
\begin{desideratum}\label{desideratumallspecial1}
	Let $V=V_{2n}$ be an orthogonal space and $\chi_V$ be the discriminant character of $V$.
	\begin{enumerate}[(1).]
		\item There exists a surjection
		$$
		\mathcal L: \bigsqcup_{V_{2n}^{\bullet}} \Irr \left(\SO(V_{2n}^\bullet)\right) \longrightarrow \Phi(\SO(V_{2n}))$$
		which is a finite-to-one map, where $V_{2n}^{\bullet}$ runs over the $2n$-dimensional orthogonal spaces of discriminant character $\chi_V$. For any $\phi \in \Phi(\SO(V_{2n}))$, we denote $\mathcal L^{-1}(\phi)$ by $\Pi_{\phi}$ and call it the $L$-packet of $\phi$. We also write $\Pi_{\phi}(\SO(V_{2n}))=\Pi_{\phi}\cap \Irr\left(\SO(V_{2n})\right)$.
		\item For each Whittaker datum $\mathfrak W_{c}$ of $\bigsqcup_{V_{2n}^{\bullet}}\SO(V_{2n}^{\bullet})$, there exist a canonical bijection
		\begin{align}\label{522}
		\mathcal J_{\mathfrak W_{c}} : \Pi_{\phi} \longrightarrow \widehat {\mathcal S^+_{\phi}}.
		\end{align}
		\item The $L$-packet $\Pi_{\phi}$ and the bijections $\mathcal J_{\mathfrak W_{c}}$ satisfy analogues of Theorem \ref{desideratumall} (3)-(12). 
	\end{enumerate}
\end{desideratum}
Desideratum \ref{desideratumallspecial1} has not been established. However, following what Arthur did in \cite{MR3135650}, we can deduce a weaker version of Desideratum \ref{desideratumallspecial1} as follows. 


We introduce an equivalence relation $\sim_{\epsilon}$ on $\Irr(\SO (V_{2n}^\bullet))$. Choose an element $\epsilon\in \mathrm O(V_{2n}^\bullet)$ such that $\det(\epsilon)=-1$. For $\pi_0\in \Irr(\SO (V_{2n}^\bullet))$, we define its conjugate $\pi_0^\epsilon$ by $$\pi_0^\epsilon(h)=\pi_0(\epsilon^{-1}h\epsilon) \quad \mbox{for $h\in \mathrm O(V_{2n}^\bullet)$}.$$ 
Then the equivalence relation $\sim_{\epsilon}$ on $\Irr(\SO (V_{2n}^\bullet))$ is defined by 
\begin{align*}
\pi_0 \sim_{\epsilon} \pi_0^\epsilon.
\end{align*}
The canonical map $\Irr(\SO (V_{2n}^\bullet))\rightarrow \Irr(\SO (V_{2n}^\bullet))/\sim_{\epsilon}$ is denoted by $\pi\mapsto [\pi_0]$. We say that $[\pi_0]\in \Irr(\SO (V_{2n}^\bullet))/\sim_{\epsilon}$ is tempered (resp. discrete) if so is some (and hence any) representative $\pi_0$. We also define an equivalence relation $\sim_{\det}$ on $\Irr(\mathrm  O(V_{2n}^\bullet))$ by 
\begin{align*}
\pi\sim_{\det} \pi\otimes\det \quad \mbox{for $\pi\in \Irr(\mathrm O(V_{2n}^\bullet))$}.
\end{align*}
Then the restriction and the induction gives a canonical bijection 
\begin{align*}
\Irr(\mathrm O(V_{2n}^\bullet))/\sim_{\det} \longleftrightarrow \Irr(\SO  (V_{2n}^\bullet))/\sim_{\epsilon}. 
\end{align*}
\begin{comment}
On the other hand, we introduce an equivalence relation $\sim_{\epsilon}$ on $\Phi(\SO(V_{2n}))$. For $\phi,\phi^\prime\in \Phi(\SO(V_{2n}))$, we write $\phi\sim_{\epsilon}\phi^\prime$ if $\phi$ is $\mathrm O(2n,\mathbb C)$-conjugate to $\phi^\prime$. The equivalence class of $\phi$ is also denoted by $\phi$.
\end{comment}
We state the weaker version of LLC for $\SO(V_{2n})$ as follows: 
\begin{theorem}[Weak LLC for special even orthogonal groups]\label{desideratumallspecial}
	Let $V=V_{2n}$ be the orthogonal space and $\chi_V$ be the discriminant character of $V$.
	\begin{enumerate}[(1).]
		\item There exists a surjection
		$$
		\mathcal L: \bigsqcup_{V_{2n}^{\bullet}} \left(\Irr \left(\SO(V_{2n}^\bullet)\right)/\sim_{\epsilon}\right) \longrightarrow \Phi(\mathrm O(V_{2n}))$$
		which is a finite-to-one map, where $V_{2n}^{\bullet}$ runs over the $2n$-dimensional orthogonal spaces of discriminant character $\chi_V$. For $\phi \in \Phi(\mathrm O(V_{2n}))$, we denote $\mathcal L^{-1}(\phi)$ by $\Pi^0_{\phi}$ and call it the $L$-packet of $\phi$. We also write $\Pi^0_{\phi}(\SO(V_{2n}))=\Pi^0_{\phi}\cap \Irr\left(\SO(V_{2n})\right)$.
		\item For each Whittaker datum $\mathfrak W_{c}$ of $\bigsqcup_{V_{2n}^{\bullet}}\SO(V_{2n}^{\bullet})$, there exist a canonical bijection
		\begin{align}\label{524}
		\mathcal J^0_{\mathfrak W_{c}} : \Pi^0_{\phi} \longrightarrow \widehat {\mathcal S^+_{\phi}}.
		\end{align}
		\item The $L$-packet $\Pi^0_{\phi}$ and the bijection $\mathcal J^0_{\mathfrak W_{c}}$ satisfy analogues of Theorem \ref{desideratumall} (3)-(12). 
		\item For $\phi\in \Phi(\mathrm O(V_{2n})),$ let $\Pi_{\phi}$ be the $L$-packet defined in Theorem \ref{desideratumall}. Then the image of $\Pi_{\phi}$ under the map 
		\begin{align*}
		\Irr(\mathrm O(V_{2n}^\bullet))\longrightarrow \left(\Irr(\mathrm O(V_{2n}^\bullet))/\sim_{\det} \right) \longrightarrow \left(\Irr(\SO(V_{2n}^\bullet))/\sim_{\epsilon}\right)
		\end{align*}
		is the packet $\Pi_{\phi}^0$ and the diagram 
		$$
		\begin{CD}
		\Pi_{\phi} @>\mathcal J_{\mathfrak W_c}>> \widehat{\mathcal {S}_{\phi}} \\
		@VVV  @VVV \\
		\Pi_{\phi}^0 @>\mathcal J^0_{\mathfrak W_{c}}>>  \widehat{\mathcal S^+_{\phi}}   
		\end{CD}
		$$
		is commutative for any Whittaker datum $\mathfrak W_{c}$ of $\bigsqcup_{V_{2n}^{\bullet}}\SO(V_{2n}^{\bullet})$.  
	%	\item The following are equivalent: 
	%	\begin{itemize}
	%		\item $\phi\in \Phi^\epsilon(\mathrm O(V_{2n}))$;
	%		\item some $[\pi_0]\in \Pi_{\phi}^0$ satisfies     $\pi_0^\epsilon \cong \pi_0$;
	%		\item all $[\pi_0]\in \Pi_{\phi}^0$ satisfy $\pi_0^\epsilon \cong \pi_0$. 
	%	\end{itemize}
	\end{enumerate}
\end{theorem}
\begin{proof}
	This follows from Theorem \ref{desideratumall}; see also Atobe-Gan \cite[\S 3.5]{MR3708200} for an explication. 
\end{proof}



\section{The Plancherel measures and normalized intertwining operators}\label{plancherelmeasure}
We recall the definition of Plancherel measures and prove Lemma \ref{homomorphicatsequal0} in this appendix. 
 
We retain the notations in subsection \ref{normalizingintertwing}. Let $\overline{P}=M_{P} U_{\overline P}$ be the opposite parabolic subgroup to $P$ of $\mathrm O(V)$ and consider the induced representation
$$\Ind_{\overline{P}}^{\mathrm O\left(V\right)}\left(\tau_{s} \otimes \pi_0 \right).$$ 
Similar to those of subsection \ref{normalizingintertwing}. They are realized on the space of smooth functions 
$$\overline{\Psi}_{s} : \mathrm O(V) \rightarrow \mathscr{V}_{\tau} \otimes \mathscr{V}_{\pi_0}$$
such that
\begin{align*} \overline{\Psi}_{s}\left( u_{\overline P} m_{P}(a) h_0 h\right) &=|\det(a)|_{F}^{s+\rho_{\overline P}} \tau(a) \pi_0(h_0)  \overline{\Psi}_{s}\left(h\right) 
\end{align*}
for any $ u_{\overline P} \in U_{\overline P}, a\in \GL(X), h_0\in \mathrm{O}(V_0), h \in \mathrm{O}(V)$.

As in \cite[\S 12]{MR3166215}, we define the 
standard intertwining operator
\begin{align*}
J_{\overline{P}|P}(\tau_{s}\otimes \pi_0): \Ind_{P}^{\mathrm O(V)}(\tau_{s}\otimes \pi_0) &\longrightarrow \Ind_{\overline{P}}^{\mathrm O(V)}(\tau_{s}\otimes \pi_0)
\end{align*}
by (the meromorphic continuations of) the integrals
\begin{align*}
J_{\overline{P}|P}(\tau_{s}\otimes \pi_0)\Psi_s(h)&= \int_{U_{\overline P}} \Psi_s(\bar uh)d\bar u 
\end{align*}
for $\Psi_s\in \Ind_{P}^{\mathrm O(V)}(\tau_{s}\otimes \pi_0)$. Similarly, we have the 
standard intertwining operator
\begin{align*}
J_{P|\overline P}(\tau_{s}\otimes \pi_0): \Ind_{\overline P}^{\mathrm O(V)}(\tau_{s}\otimes \pi_0) &\longrightarrow \Ind_{P}^{\mathrm O(V)}(\tau_{s}\otimes \pi_0). 
\end{align*}
By \cite[\S 12]{MR3166215}, the Plancherel measures $\mu(\tau_{s}\otimes\pi_0)$ are defined as 
\begin{align*}\label{planch}
J_{P|\overline{P}}(\tau_{s}\otimes \pi_0)\circ J_{\overline{P}|P}(\tau_{s}\otimes \pi_0)= \mu(\tau_{s}\otimes\pi_0)^{-1}. 
\end{align*}
The definition of $\mu$ depends on the choice of Haar measures on $U_P$ and $U_{\overline P}$. We refer to \cite[Appendix B]{MR3166215} for the choice of these Haar measures. 

\begin{comment}
Put
\begin{align*}
\rho_{\bar P|P}(\tau_{s}\otimes\pi_0)&= \gamma(s,\phi_{\tau}\otimes \phi_{\pi_0}^{\vee},\psi)^{-1}\gamma(2s,\wedge^{2}\circ \phi_{\tau},\psi)^{-1}\\
\rho_{P|\bar P}(\tau_{s}\otimes\pi_0)&= \gamma(-s,\phi_{\tau}^{\vee}\otimes \phi_{\pi_0},\psi_{-1})^{-1}\gamma(-2s,\wedge^{2}\circ \phi_{\tau}^{\vee},\psi_{-1})^{-1}\\
\end{align*}
Here $\phi_{\tau},\phi_{\pi_0},\phi_{\sigma_0}$  is the $L$-parameter of $\tau,\pi_0$ and $\sigma_0$ respectively. 
\end{comment} 
Fix a Whittaker datum $\mathfrak W_{c}$ of $\bigsqcup_{V_{2n}^\bullet}\mathrm O(V_{2n}^\bullet)$. Let $\widetilde{w}_c$ be the lift of $w$ in \ref{131}. Then there is a intertwining isomorphism 
\begin{align*}
\ell(w_c,\tau_s\otimes \pi_0): \Ind_{\overline P}^{\mathrm O(V)}(\tau_{s}\otimes \pi_0)\rightarrow \Ind_P^{\mathrm O(V)}(w(\tau_{s}\otimes \pi_0))
\end{align*}
given by left translation 
\begin{align*}
\left(\ell(w_c,\tau_s\otimes \pi_0)\overline{\Psi}_{s}\right)(h) =\overline{\Psi}_s(\widetilde{w}_{c}^{-1}h)
\end{align*}
for $\overline{\Psi}_s\in \Ind_{\overline P}^{\mathrm O(V)}(\tau_{s}\otimes \pi_0)$ and $h\in \mathrm O(V)$. It is easy to check that the following diagram 
\begin{equation}\label{120}
\begin{CD}
\Ind_{P}^{\mathrm O(V)}(\tau_{s}\otimes \pi_0) @>J_{\overline P| P}(\tau_{s}\otimes \pi_0)>> \Ind_{\overline P}^{\mathrm O(V)}(\tau_{s}\otimes \pi_0) \\
@| @VV\ell(w_c,\tau_s\otimes \pi_0)V\\
\Ind_{P}^{\mathrm O(V)}(\tau_{s}\otimes \pi_0) @>\mathcal M(\widetilde{w}_{c},\tau_{s}\otimes \pi_0)>> \Ind_{P}^{\mathrm O(V)}(w(\tau_{s}\otimes \pi_0))
\end{CD}
\end{equation}
is commutative.  
\begin{comment}
 It is easy to check that 
$$
\ell(w_c,\tau_s\otimes \pi_0)\circ J_{\bar{P}|P}(\tau_{s}\otimes \pi_0)= \mathcal M(\widetilde{w}_{c},\tau_{s}\otimes \pi_0)
$$


Moreover, the following diagram 
$$
\begin{tikzcd}
\Ind_{P}^{\mathrm O(V)}(\tau_{s}\otimes \pi_0) \arrow[r,"J_{\bar P|P}"] \arrow[d, "\mathrm 1"] & \Ind_{\bar P}^{\mathrm O(V)}(\tau_{s}\otimes \pi_0) \arrow[r, "J_{P|\bar P}"] \arrow[d,"\ell(w)"] &\Ind_{P}^{\mathrm O(V)}(\tau_{s}\otimes \pi_0)  \arrow[d,"\ell(w^2)"] \\
\Ind_{P}^{\mathrm O(V)}(\tau_{s}\otimes \pi_0) \arrow[r,"\mathcal M(\widetilde{w}_{c^\prime})"] & \Ind_{P}^{\mathrm O(V)}(w(\tau_{s}\otimes \pi_0)) \arrow[r]&\Ind_{P}^{\mathrm O(V)}(\tau_{s}\otimes \pi_0) 
\end{tikzcd}
$$
is commutative (rewrite this commutative diagram). 
It is easy to check that 
$$
\ell(w^2_c,\tau_s\otimes \pi_0)\circ J_{P|\bar P}(\tau_{s}\otimes \pi_0)= \mathcal M(\widetilde{w}_{c},w(\tau_{s}\otimes \pi_0))\circ \ell(w_c,\tau_s\otimes \pi_0)
$$
\end{comment}
Note that 
\begin{align*}
\widetilde{w}_{c}^2= 
m_P((-\mathbf 1)^{k-1})\cdot \mathbf 1_{V_0}.
\end{align*}
Hence $\widetilde{w}_{c}^2$ lies in the center of $M_P$. We have an intertwining isomorphism  
\begin{align*}
\ell(w^2_c,\tau_s\otimes \pi_0): \Ind_{P}^{\mathrm O(V)}(\tau_{s}\otimes \pi_0)\rightarrow \Ind_P^{\mathrm O(V)}(w^2(\tau_{s}\otimes \pi_0))=\Ind_P^{\mathrm O(V)}(\tau_{s}\otimes \pi_0)
\end{align*}
given by left translation 
\begin{align}\label{129}
\left(\ell(w_c^2,\tau_s\otimes \pi_0)\Psi_{s}\right)(h) = \Psi_s((\widetilde{w}_{c})^{-2}h)=\omega_{\tau}(-1)^{k-1}\Psi_s(h).
\end{align}
for $\Psi_s\in \Ind_{P}^{\mathrm O(V)}(\tau_{s}\otimes \pi_0)$ and $h\in \mathrm O(V)$. Here $\omega_\tau$ is the central character of $\tau$. Then we have the following commutative diagram 
\begin{equation}\label{122}
\begin{CD}
\Ind_{\overline P}^{\mathrm O(V)}(\tau_{s}\otimes \pi_0) @>J_{P|\overline P}(\tau_{s}\otimes \pi_0)>> \Ind_{P}^{\mathrm O(V)}(\tau_{s}\otimes \pi_0) \\
@VV\ell(w_c,\tau_s\otimes \pi_0)V @VV\ell(w^2_c,\tau_s\otimes \pi_0)V\\
\Ind_{P}^{\mathrm O(V)}(w(\tau_{s}\otimes \pi_0)) @>\mathcal M(\widetilde{w}_{c},w(\tau_{s}\otimes \pi_0))>> \Ind_{P}^{\mathrm O(V)}(\tau_{s}\otimes \pi_0).
\end{CD}
\end{equation}
Combining the \ref{120}, \ref{129} and \ref{122}, we have 
\begin{equation}\label{1235}
\begin{aligned}
&\mathcal M(\widetilde{w}_{c},w(\tau_{s}\otimes \pi_0))\circ \mathcal M(\widetilde{w}_{c},\tau_{s}\otimes \pi_0)\\
=& \ell(w^2_c,\tau_s\otimes \pi_0)\circ J_{P|\overline P}(\tau_{s}\otimes \pi_0)\circ J_{\overline P|P}(\tau_{s}\otimes \pi_0)\\
=& \omega_{\tau}(-1)^{k-1}\times \mu(\tau_{s}\otimes\pi_0)^{-1}.
\end{aligned}
\end{equation}

Now we begin to prove Lemma \ref{homomorphicatsequal0} for even orthogonal groups. 
\begin{proposition}\label{appendixa1}
Let $\mathcal R_{\mathfrak W_c}\left(w, \tau_{s} \otimes \pi_0\right)$ and be the normalized intertwining operator defined in subsection \ref{normalizingintertwing}. Then we have  
\begin{enumerate}[(i)]
	\item $	\mathcal R_{\mathfrak W_c}\left(w, w(\tau_{s} \otimes \pi_0)\right)\circ \mathcal R_{\mathfrak W_c}\left(w, \tau_{s} \otimes \pi_0\right)=1 $;
	\item $\mathcal R_{\mathfrak W_c}\left(w, w(\tau_{-\bar s} \otimes \pi_0)\right)^*=\mathcal R_{\mathfrak W_c}\left(w,\tau_{s} \otimes \pi_0 \right)$. 
\end{enumerate}
In particular, $\mathcal R_{\mathfrak W_c}\left(w,\tau_{s} \otimes \pi_0 \right)$ is unitary when $s$ is purely imaginary. Hence $\mathcal R_{\mathfrak W_c}\left(w,\tau_{s} \otimes \pi_0 \right)$ is holomorphic at $s=0$. 
\end{proposition}
\begin{proof}
	We first prove (i): Let $\phi_\tau$ and $\phi_{0}$ be the $L$-parameter of $\tau$ and $\pi_0$. Then by \ref{1235} and Corollary \ref{LIRfinal}, we have 
	\begin{align*}
	 &\quad\mathcal R_{\mathfrak W_c}\left(w, w(\tau_{s} \otimes \pi_0)\right)\circ \mathcal R_{\mathfrak W_c}\left(w, \tau_{s} \otimes \pi_0\right)\\
	&=\epsilon(V)^{2k}\cdot \chi_{V}(c)^{2k}\cdot  r(w,w(\tau_s\otimes\pi_{0}))^{-1}\cdot r(w,\tau_s\otimes\pi_{0})^{-1}\\
	& \quad \times|c|_F^{k\rho_P}\cdot |c|_F^{k\rho_{\overline P}}\cdot \mathcal M(\widetilde{w}_{c^\prime},w(\tau_{s}\otimes \pi_0))\circ \mathcal M(\widetilde{w}_{c^\prime},\tau_{s}\otimes \pi_0)\\
	&=r(w,w(\tau_s\otimes\pi_{0}))^{-1}\cdot r(w,\tau_s\otimes\pi_{0})^{-1}\cdot \omega_{\tau}(- 1)^{k-1}\times \mu(\tau_{s}\otimes\pi_0)^{-1}\\
	&=r(w,w(\tau_s\otimes\pi_{0}))^{-1}\cdot r(w,\tau_s\otimes\pi_{0})^{-1}\cdot \omega_{\tau}(- 1)^{k-1}\cdot \gamma(s,\phi_{\tau}\otimes \phi_{0}^{\vee},\psi)^{-1}\\
	&\quad \times\gamma(-s,\phi_{\tau}^{\vee}\otimes\phi_0, \psi_{-1} )\cdot
   \gamma (2s, \wedge^2\circ \phi_{\tau}, \psi)^{-1}\cdot \gamma(-2s, \wedge^{2} \circ \phi_{\tau}^{\vee},\psi_{-1})^{-1}\\
	&=1,
\end{align*}
where the last equality follows from \ref{normalizationfactor}, \ref{12345} and the following formulas 
	\begin{align*}
	\gamma(s,\phi,\psi)&= \frac{\varepsilon(s,\phi,\psi)\cdot L(1-s,\phi^\vee)}{L(s,\phi)} ,\\
	\varepsilon(s,\phi,\psi_{-1})&=\det(\phi)(-1)\cdot  \varepsilon(s,\phi,\psi)
	\end{align*}
	for any representation $\phi$ of $\WD_F$. 
	 
	The second statement follows from a similar argument in \cite[Proposition 2.3.1]{MR3135650}, we omit the details here. 
\end{proof}


%\cdot &\gamma(s,\phi_0^\vee\otimes\phi_\tau,\psi)^{-1}\cdot \gamma(-s,\phi_0\otimes\phi_\tau^\vee,\psi_{-1})^{-1}\\
%\cdot &\gamma(2s,\wedge^2\circ \phi_\tau,\psi)^{-1}\cdot \gamma(-2s,\wedge^2\circ \phi_\tau^\vee,\psi_{-1})^{-1}\\

\begin{comment}


\begin{lemma}\label{intertwingidentity}
We have 
\begin{enumerate}
\item \begin{align*}
R_{P| \bar P}\left(\tau_{s} \otimes \pi_0\right)\cdot R_{\bar P | P}\left(\tau_{s} \otimes \pi_0\right)=1
\end{align*}
\item 	\begin{align*}
R_{P|\bar P}\left(\tau_{s} \otimes \pi_0\right)^*= R_{\bar P| P}\left(\tau_{-\bar s} \otimes \pi_0\right)
\end{align*}
\end{enumerate}
\end{lemma}
\begin{proof}
By Lemma\ref{respectplancherel}, we have 
$$
\mu(\tau_{s}\otimes\pi_0)=\gamma(s,\phi_{\tau}\otimes \phi_{\phi_0}^{\vee},\psi)\cdot \gamma(2s,\wedge^{2}\circ \phi_{\tau},\psi)\cdot \gamma(-s,\phi_{\tau}^{\vee}\otimes\phi_{\pi_0},\psi_{-1})\cdot \gamma(-2s,\wedge^{2}\circ\phi_{\tau}^{\vee},\psi_{-1})
$$
So 
\begin{align*}
R_{P|\bar P}\left(\tau_{s} \otimes \pi_0\right)R_{\bar P |P}\left(\tau_{s} \otimes \pi_0\right)&=\rho_{\bar P| P}\left(\tau_{s} \otimes \pi_0\right)^{-1} \rho_{P|\bar P}\left(\tau_{s} \otimes \pi_0\right)^{-1}
J_{\bar P| P}\left(\tau_{s} \otimes \pi_0\right) \\J_{P |\bar P}\left(\tau_{s} \otimes \pi_0\right)
&=\gamma(s,\phi_{\tau}\otimes \phi_{\pi_0}^{\vee},\psi)\gamma(2s,\wedge^{2}\circ \phi_{\tau},\psi)\gamma(-s,\phi_{\tau}^{\vee}\otimes\phi_{\pi_0},\psi_{-1})\\
&\gamma(-2s,\wedge^{2}\circ\phi_{\tau}^{\vee},\psi_{-1})\mu(\tau_{s}\otimes\pi_0)^{-1}=1\\
\end{align*}
The case for symplectic group is similar. 
\end{proof}

We also have 
\begin{lemma}\label{intertwingadjoint} 
\begin{align*}
R_{P|\bar P}\left(\tau_{s} \otimes \pi_0\right)^*= R_{\bar P| P}\left(\tau_{-\bar s} \otimes \pi_0\right)
\end{align*}
\end{lemma}
\begin{proof}
%	This is in \cite{MR3135650}[Proposition 2.3.1] 
prove $\rho_{\bar P|P}(\tau_{s}\otimes\pi_0) =\overline{\rho_{\bar P| P}(\tau_{-\bar s}\otimes\pi_0)}$
\end{proof}

\begin{corollary}
The operator $R_{P|\bar P}\left(\tau_{s} \otimes \pi_0\right)$ and $R_{Q|\bar Q}\left(\tau_{s} \otimes \sigma_0\right)$ is unitary if $s$ is purely imaginary. So the operator 
\begin{align*}
R_{P|\bar P}\left(\tau \otimes \pi_0\right)=R_{P|\bar P}\left(\tau_{s} \otimes \pi_0\right)|_{s=0}
\end{align*}
is therefore defined. 
\end{corollary}
\begin{proof}
See lit for metaplectic Lemma 8.7 
\end{proof}
We have constructed normalized intertwing operators $R_{P|\bar P}\left(\tau \otimes \pi_0\right)$ (resp. $R_{Q|\bar Q}\left(\tau \otimes \sigma_0\right)$)between 
$\Ind_{P}(\tau\otimes \pi_0)$(resp. $\Ind_{Q}(\tau\otimes \sigma_0)$) and $\Ind_{\bar P}(\tau\otimes \pi_0)$(resp. $\Ind_{\bar Q}(\tau\otimes \sigma_0)$). We will convert these objects to self-intertwining operators of $\Ind_{P}(\tau\otimes \pi_0)$(resp. $\Ind_{Q}(\tau\otimes \sigma_0)$), which will attached to an elements $w\in W\left(M_{P}\right)$(resp.$w\in W\left(M_{Q}\right)$). 


Finally, it is easy to check that 
\begin{align*}
\mathcal R(aaa)= \ell(\tilde w)\circ R_{\bar P| P}\left(\tau_{s} \otimes \pi_0\right)
\end{align*}
So $\mathcal{R}\left(w, \tau_{s} \otimes \pi_0\right)$ is holomorphic at $s=0$. 
\end{comment}




\bibliographystyle{alpha}
\nocite{*}
\bibliography{LLC4ort}


\end{document}

