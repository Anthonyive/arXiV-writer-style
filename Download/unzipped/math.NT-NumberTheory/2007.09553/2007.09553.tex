\documentclass[11pt]{article}
%\usepackage{amsmath,amssymb,amsbsy,amsfonts,amsthm,latexsym,   amsopn,amstext,amsxtra,euscript,amscd,color}

\usepackage{amsmath,amssymb,amsbsy,amsfonts,amsthm,latexsym,
            amsopn,amstext,amsxtra,euscript,amscd,color,breqn,mathrsfs}

            
\PassOptionsToPackage{hyphens}{url}\usepackage{hyperref}
    
\usepackage[capbesideposition=outside,capbesidesep=quad]{floatrow}

%\usepackage{tabularx}

\captionsetup{labelfont=bf, justification=justified, singlelinecheck=false, position=above}

\restylefloat{table}
            
\usepackage{multirow,caption}

\newfont{\teneufm}{eufm10}
\newfont{\seveneufm}{eufm7}
\newfont{\fiveeufm}{eufm5}

\usepackage{systeme}

%%%these are needed for latexdraw            
\usepackage{xcolor}
\usepackage{tikz}
\usetikzlibrary{shapes}
\usetikzlibrary{decorations.markings}
%\input{tikzstyledefs.tex}

\usepackage{pstricks}
\usetikzlibrary{shapes,arrows}
 \usepackage{pstricks-add}
%\usepackage{pst-all}
\usepackage{epsfig}
% \usepackage{pst-grad} % For gradients
% \usepackage{pst-plot} % For axes
 \usepackage[space]{grffile} % For spaces in paths
 \usepackage{etoolbox} % For spaces in paths
 \makeatletter % For spaces in paths
 \patchcmd\Gread@eps{\@inputcheck#1 }{\@inputcheck"#1"\relax}{}{}
 \makeatother
%%%the above are needed for latexdraw 
            
\newtheorem{definition}{Definition}
\newtheorem{theorem}{Theorem}
\newtheorem{corollary}{Corollary}
\newtheorem{lemma}{Lemma}
\newtheorem{proposition}{Proposition}
\newtheorem{example}{Example}
\newtheorem{remark}{Remark}

\newtheorem{thm}{Theorem}
\newtheorem{lem}[thm]{Lemma}
\newtheorem{cor}[thm]{Corollary}
\newtheorem{notation}[thm]{Notation}
\newtheorem{prop}[thm]{Proposition}
\newtheorem{rem}[thm]{Remark}
\newtheorem{defn}[thm]{Definition}
\newtheorem{conjecture}[thm]{Conjecture}


\newcommand{\Tr}{{\rm Tr}}
\newcommand{\Trn}{{\rm Tr}_n}
\newcommand{\Trm}{{\rm Tr}_m}
\newcommand{\cda}{{_cD_a}}

\newcommand{\cB}{\mathscr{B}}

\def\+{\oplus}
\def\E{{\mathbf E}}
\def\fA{{\mathfrak A}}
\def\fB{{\mathfrak B}}
\def\fC{{\mathfrakC}}
\def\fD{{\mathfrak D}}

\def\cA{{\mathcal A}}
\def\cB{{\mathcal B}}
\def\cC{{\mathcal C}}
\def\cD{{\mathcal D}}
\def\cE{{\mathcal E}}
\def\cF{{\mathcal F}}
\def\cG{{\mathcal G}}
\def\cH{{\mathcal H}}
\def\cI{{\mathcal I}}
\def\cJ{{\mathcal J}}
\def\cK{{\mathcal K}}
\def\cL{{\mathcal L}}
\def\cM{{\mathcal M}}
\def\cN{{\mathcal N}}
\def\cO{{\mathcal O}}
\def\cP{{\mathcal P}}
\def\cQ{{\mathcal Q}}
\def\cR{{\mathcal R}}
\def\cS{{\mathcal S}}
\def\cT{{\mathcal T}}
\def\cU{{\mathcal U}}
\def\cV{{\mathcal V}}
\def\cW{{\mathcal W}}
\def\cX{{\mathcal X}}
\def\cY{{\mathcal Y}}
\def\cZ{{\mathcal Z}}

\def\C{{\mathbb C}}
\def\R{{\mathbb R}}
\def\E{{\mathbb E}}
\def\D{{\mathbb D}}
\def\F{{\mathbb F}}
\def\Z{{\mathbb Z}}
\def\N{{\mathbb N}}
\def\Q{{\mathbb Q}}
 
\def\K{{\mathbb K}}
\def\L{{\mathbb L}}
\def\M{{\mathbb M}}
\def\Z{{\mathbb Z}}
\def\A{{\mathbb A}}
\def\B{{\mathbb B}}
\def\U{{\mathbb U}}
\def\W{{\mathbb W}}
\def\V{{\mathbb V}}
\def\e{{\mathbf e}}
\def\cGB{\mathcal{GB}}
\def\wt{{\rm wt}}
\def\wtp{{\rm wt}^{(p)}}
\def\Wfp{W_f^{(p)}}
\def\Fn{{\mathbb{F}_{p^n}}}

\newcommand{\zetak}[1][2^k]{\zeta_{{#1}}}

\def\ba{{\bf a}}
\def\bb{{\bf b}}
\def\cc{{\bf c}}
\def\dd{{\bf d}}
\def\uu{{\bf u}}
\def\vv{{\bf v}}
\def\ww{{\bf w}}
\def\xx{{\mathbf x}}
\def\kk{{\mathbf k}}
\def\yy{{\mathbf y}}
\def\zz{{\mathbf z}}
\def\XX{{\mathbf X}}
\def\KK{{\mathbf K}}
\def\YY{{\mathbf Y}}
\def\ZZ{{\mathbf Z}}
\def\bl{{\bf \lambda}}
\def\ee{{\mathbf e}}
\def\00{{\bf 0}}
\def\11{{\bf 1}}
\def\+{\oplus}

\def\\{\cr}
\def\({\left(}
\def\){\right)}
\def\lcm{{\rm lcm\/}}
\def\fl#1{\left\lfloor#1\right\rfloor}
\def\rf#1{\left\lceil#1\right\rceil}

\newcommand{\BBZ}{\mathbb{Z}}
\newcommand{\BBR}{\mathbb{R}}
\newcommand{\BBN}{\mathbb{N}}
\newcommand{\BBC}{\mathbb{C}}
\newcommand{\BBD}{\mathbb{D}}
\newcommand{\BBQ}{\mathbb{Q}}
\newcommand{\BBF}{\mathbb{F}}
\newcommand{\BBV}{\mathbb{V}}

\newcommand{\bwht}[2]{\mathcal{W}_{#1}(#2)}
\newcommand{\vwht}[3]{\mathcal{W}_{#1}(#2,#3)}
\newcommand{\vwhtp}[4]{\mathcal{W}^{#4}_{#1}(#2,#3)}
\newcommand{\cardinality}[1]{\# #1}
\newcommand{\absval}[1]{|#1|}

\newcommand{\calP}{{\cal P}}
\newcommand{\calB}{{\cal B}}
\newcommand\dsum{\displaystyle \sum}
\def\wt{{\rm wt}}
\def\inv{{\rm inv}}
\renewcommand\ss{{\bf s}}

\newcommand{\comm}[1]{\marginpar{
\vskip-\baselineskip \raggedright\footnotesize
\itshape\hrule\smallskip#1\par\smallskip\hrule}}

\setlength{\marginparwidth}{2.2cm}
\usepackage[colorinlistoftodos,prependcaption,textsize=tiny]{todonotes}
\newcommand{\pante}[1]{\todo[linecolor=blue, color=blue!40]{\scriptsize{#1}}}

\providecommand{\newoperator}[3]{%
  \newcommand*{#1}{\mathop{#2}#3}}

\newoperator{\FD}{\mathrm{FD}}{\nolimits}

\newcommand{\x}{{\tt X}}
\newcommand{\f}{\Bbb F} 
\newcommand{\sA}{\mathscr{A}}
\newcommand{\sB}{\mathscr{B}}
\newcommand{\sX}{\mathscr{X}}
\newcommand{\sY}{\mathscr{Y}}
\newcommand{\sZ}{\mathscr{Z}}
\newcommand{\sC}{\mathscr{C}}

\newcommand{\ar}[2]{\arrow <4pt> [0.3, 0.67] from #1 to #2}
\newcommand{\ari}[1]{{{\color{red}#1}}}
\newcommand{\ner}[1]{{{\color{blue}#1}}}
%\usepackage{graphicx}



\begin{document}
\title{\bf Using double Weil sums in finding the Boomerang and the $c$-Boomerang Connectivity Table for monomial functions on finite fields}
\author{Pantelimon~St\u anic\u a  \\ %\thanks{$^*$Corresponding author}
%\thanks{P. St\u anic\u a is with 
Applied Mathematics Department, \\
Naval Postgraduate School, Monterey, USA. \\
%\protect\\ 
E-mail: pstanica@nps.edu}
%\markboth{IEEE Transactions on Information Theory,~Vol.~XX, No.~Y, Month~2020}
%{ P. St\u anic\u a:  $c$-Boomerang Uniformity}

 \maketitle 

\begin{abstract}
In this paper we characterize the  $c$-Boomerang Connectivity Table ($c\neq 0$), for all monomial function $x^d$ in terms of characters and Weil sums on the finite field~$\F_{p^n}$. We further simplify these expressions for the Gold functions $x^{p^k+1}$ for all $1\leq k<n$, $p$ odd. It is the first such complete description  for the classical BCT and its relative $c$-BCT, for all parameters involved.
\end{abstract}

{\bf Keywords:} 
Finite fields,
$p$-ary functions, 
$c$-differentials,  
differential uniformity,
boomerang uniformity,
double Weil sums
\newline
%\end{IEEEkeywords}
{\bf MSC 2000}: 11L15, 11T06, 11T24, 94A60.
 

\section{Introduction and basic definitions}


For the first time, in this paper we find a characterization of the boomerang connectivity table and its uncle, $c$-boomerang connectivity table ($c$-BCT)~\cite{S20}, for all monomial functions in terms of characters of the relevant finite field (all characteristics). We further detail that characterization for the Gold functions $x^{p^k+1}$ for all $1\leq k<n$, where $p$ is an odd prime.  Since our method mostly relies on finding some double Weil sums, it may have an interest beyond its applicability in the computation of the $c$-BCT.
   
For a positive integer $n$ and $p$ a prime number, we let $\F_{p^n}$ be the  finite field with $p^n$ elements, and $\F_{p^n}^*=\F_{p^n}\setminus\{0\}$ be the multiplicative group (for $a\neq 0$, we often write $\frac{1}{a}$ to mean the multiplicative inverse of $a$).  We use $|{S}|$ to denote the cardinality of a set $S$ and $\bar z$, for the complex conjugate.
We call a function from $\F_{p^n}$   to $\F_p$  a {\em $p$-ary  function} on $n$ variables. For positive integers $n$ and $m$, any map $F:\F_{p^n}\to\F_{p^m}$   is called a {\em vectorial $p$-ary  function}, or {\em $(n,m)$-function}. When $m=n$, $F$ can be uniquely represented as a univariate polynomial over $\F_{p^n}$ (using some identification, via a basis, of the finite field with the vector space) of the form
$
F(x)=\sum_{i=0}^{p^n-1} a_i x^i,\ a_i\in\F_{p^n}.
$
For $f:\F_{p^n}\to \F_p$ we define the absolute trace $\Trn:\F_{p^n}\to \F_p$, given by $\displaystyle \Trn(x)=\sum_{i=0}^{n-1} x^{p^i}$ (we will denote it by $\Tr$, if the dimension is clear from the context).  The reader can consult~\cite{LN97} for more on this and related notions in finite fields.
  
Given a vectorial $p$-ary  function $F$, the derivative of $f$ with respect to~$a \in \F_{p^n}$ is the $p$-ary  function
$
 D_{a}F(x) =  F(x + a)- F(x), \mbox{ for  all }  x \in \F_{p^n}.
$ 
For an $(n,n)$-function $F$, and $a,b\in\F_{p^n}$, we let $\Delta_F(a,b)=\cardinality{\{x\in\F_{p^n} : D_{a}f(x)=b\}}$. We call the quantity
$\delta_F=\max\{\Delta_F(a,b)\,:\, a,b\in \F_{p^n}, a\neq 0 \}$ the {\em differential uniformity} of $F$. If $\delta_F= \delta$, then we say that $F$ is differentially $\delta$-uniform. 
%If $\delta=1$, then $F$ is called a {\em perfect nonlinear} ({\em PN}) function, or {\em planar} function. If $\delta=2$, then $F$ is called an {\em almost perfect nonlinear} ({\em APN}) function. It is well known that PN functions do not exist if $p=2$.

For the interested reader, we point to~\cite{Bud14,CH1,CH2,CS17,MesnagerBook,Tok15} for a proper background on  Boolean and $p$-ary functions.

  
As a follow up to Wagner's work~\cite{Wag99} on the boomerang attack against block ciphers (see also~\cite{BK99,KKS00,BDK02,Kim12}) and  Cid et al.~\cite{Cid18} who introduced the theoretical tool called  the Boomerang Connectivity Table (BCT) and Boomerang Uniformity, we defined in~\cite{S20} the $c$-Boomerang Connectivity Table ($c$-BCT) and analyzed some known perfect nonlinear as well as the inverse function in even and odd characteristics.  

Let $F$ be a permutation on $\F_{p^n}$  and $(a,b)\in\F_{p^n}\times \F_{p^n}$. We define the entries of the {\em Boomerang Connectivity Table} \textup{(}{\em BCT}\textup{)} by
\[
\cB_F(a,b)=\cardinality \{x\in\F_{2^n}|F^{-1} (F(x)+b)-F^{-1}(F(x+a)+b)=a  \},
\]
where $F^{-1}$ is the compositional inverse of $F$, and the {\em boomerang uniformity} of $F$  is
$\displaystyle
\beta_F=\max_{a,b\in\Fn*} \cB_F(a,b).
$
We also say that $F$ is a $\beta_F$-uniform BCT function. Surely, $\Delta_F(a,b)=0,2^n$ and $\cB_F(a,b)=p^n$  whenever $ab=0$. We know that $\delta_F=\delta_{F^{-1}}$, $\beta_F=\beta_{F^{-1}}$, and  for permutations, $\beta_F\geq \delta_F$ and they are equal for APN permutations.    
 We mention here that this concept became an object of study for many recent papers~\cite{BC18, BPT19, CV19,Li19, LiHu20, Mes19,TX20}, to mention just a few.  
 
  Li et al.~\cite{Li19} (see also~\cite{Mes19}) observed that
\begin{equation*}
\begin{split}
\label{eq:boom-diff}
\cB_F(a,b)&=\cardinality \left\{ (x,y)\in\Fn\times \Fn\,\Large\big|\, \substack{F(x)+F(y)=b \\  F(x+a)+F(y+a)=b } \right\}, \\
% &\stackrel{y=x+\gamma}{=}\cardinality \left\{ (x,\gamma)\in\Fn\times \Fn\,\Large\big|\, \substack{F(x+\gamma)+F(x)=b \\ F(x+\gamma+a)+F(x+a)=b }   \right\}\\
& =\sum_{\gamma\in\Fn} \cardinality  \left\{ x \in\Fn \,\big|\, D_\gamma F(x)=b \text{ and }  D_\gamma F(x+a)=b   \right\},
\end{split}
\end{equation*}
and therefore, the concept can be extended to non-permutations, since it avoids the inverse of~$F$.

 
Based upon our prior $c$-differential concept~\cite{EFRST20} (see also~\cite{HMRS20,RS20,SG20,YZ20} for very recent work on that topic), we extended this notion recently in~\cite{S20} to the $c$-Boomerang Connectivity Table.
For a $p$-ary $(n,m)$-function   $F:\F_{p^n}\to \F_{p^m}$, and $c\in\F_{p^m}$, the ({\em multiplicative}) {\em $c$-derivative} of $F$ with respect to~$a \in \F_{p^n}$ is the  function
\[
 _cD_{a}F(x) =  F(x + a)- cF(x), \mbox{ for  all }  x \in \F_{p^n}.
\]


For an $(n,n)$-function $F$, and $a,b\in\F_{p^n}$, we let the entries of the $c$-Difference Distribution Table ($c$-DDT) be defined by ${_c\Delta}_F(a,b)=\cardinality{\{x\in\F_{p^n} : F(x+a)-cF(x)=b\}}$. We call the quantity
\[
\delta_{F,c}=\max\left\{{_c\Delta}_F(a,b)\,|\, a,b\in \F_{p^n}, \text{ and } a\neq 0 \text{ if $c=1$} \right\}\]
the {\em $c$-differential uniformity} of~$F$. If $\delta_{F,c}=\delta$, then we say that $F$ is differentially $(c,\delta)$-uniform (or that $F$ has $c$-uniformity $\delta$, or for short, {\em $F$ is $\delta$-uniform $c$-DDT}). We can recover all the classical perfect and almost perfect nonlinear functions, taking $c=1$.
%If $\delta=1$, then $F$ is called a {\em perfect $c$-nonlinear} ({\em PcN}) function (certainly, for $c=1$, they only exist for odd characteristic $p$; however, as proven in~\cite{EFRST20}, there exist PcN functions for $p=2$, for all  $c\neq1$). If $\delta=2$, then $F$ is called an {\em almost perfect $c$-nonlinear} ({\em APcN}) function. 
%When we need to specify the constant $c$ for which the function is PcN or APcN, then we may use the notation $c$-PN, or $c$-APN.
It is easy to see that if $F$ is an $(n,n)$-function, that is, $F:\F_{p^n}\to\F_{p^n}$, then $F$ is PcN if and only if $_cD_a F$ is a permutation polynomial.

Further, 
for an $(n,n)$-function $F$, $c\neq 0$, and $(a,b)\in\Fn\times \Fn$,  we define~\cite{S20} the {\em $c$-Boomerang Connectivity Table}  \textup{(}$c$-BCT\textup{)} entry at $(a,b)$ to be
%\begin{align*}
%_c\cB_F(a,b)&=\cardinality \left\{ (x,\gamma)\in\Fn\times \Fn\,\Large\big|\, \substack{F(x+\gamma)+cF(x)=_cD_\gamma F(x)=b \\ F(x+\gamma+a)+cF(x+a)=_cD_\gamma F(x+a)=b } \right\}\\
%& =\sum_{\gamma\in\Fn} \cardinality  \left\{ x \in\Fn \,\big|\, _cD_\gamma F(x)=b \text{ and }  _cD_\gamma F(x+a)=b   \right\}.
%\end{align*}
{\small
\begin{equation*}
\label{eq:originalBCT}
  _c\cB_F(a,b)=\cardinality \left\{ x\in\Fn\,\Big|\,  F^{-1}(c^{-1} F(x+a)+b) -F^{-1}(cF(x)+b)=a \right\}.
\end{equation*}
}
and  the {\em $c$-boomerang uniformity} of $F$ is 
$\displaystyle
\beta_{F,c}=\max_{a,b\in\Fn*} {_c}\cB_F(a,b).
$
If $\beta_{F,c}=\beta$, we also say that $F$ is a $\beta$-uniform $c$-BCT function.
We showed in~\cite{S20} that we can avoid inverses, thus   allowing the definition to be extended to all $(n,m)$-function, not only permutations. Precisely,   the entries of the $c$-Boomerang Connectivity Table at $(a,b)\in\F_{p^n}\times\F_{p^n}$  can be given by
\begin{align*}
_c\cB_F(a,b)&=\cardinality \left\{ (x,y)\in\Fn\times \Fn\,\Big|\, \Large\substack{F(y)-cF(x) =b \\  F(y+a)- c^{-1} F(x+a)=b }  \right\}\\
& =\sum_{\gamma\in\Fn} \cardinality  \left\{ x \in\Fn \,\big|\, \Large \substack{_cD_\gamma F(x)=b \text{ and }  _{c^{-1}}D_\gamma F(x+a)=b \\
\text{the $c$-boomerang system}}  \right\}.
\end{align*}
%\textup{(}We shall call the system above, the $c$-boomerang system, for easy referral.\textup{)}
  
  
%  Besides various connections between $c$-DDT and $c$-BCT and characterizations via Walsh transforms, we investigated some of the known perfect nonlinear and the inverse function in all characteristics, in~\cite{S20}. For example, we showed that in general, if $F(x)=x^{2^n-2}$ on $\F_{2^n}$, ${_c}\cB_F(a,b)\leq 3$, and if $F(x)=x^{p^n-2}$ on $\F_{p^n}$ ($p$ odd), ${_c}\cB_F(a,b)\leq 4$ and gave complete conditions when the upper bound happens.
  
  The exact computation of the differential and/or boomerang uniformity and its relative with respect to $c$ seems to be quite difficult, even for monomials.
  It is the purpose of this paper to characterize the $c$-BCT ($c\neq 0$) for all monomials $x^d$ in terms of characters on the finite field~$\F_{p^n}$, where $p$ is any prime number. We use that characterization to further describe the $c$-BCT for all Gold functions $x^{p^k+1}$, $1\leq k<n$, $p$ odd, and $c\neq 0$.  In particular, our result can be seen as a significant generalization of the known results, where $p=2$, $\gcd(n,k)=1,2$, in which case the boomerang uniformity is 2, respectively, 4.
%Section~\ref{sec2} contains some background on differential and boomerang uniformity.  Section~\ref{sec3}  characterizes the $c$-boomerang uniformity for all monomials in terms of characters and Section~\ref{sec4}  details that characterization for all the Gold functions. Section~\ref{sec5}   concludes the paper.  



%\section{Some lemmas}
%\label{sec3}
%
% We will be using  throughout Hilbert's Theorem 90 (see~\cite{Bo90}), which states that if $\mathbb{F}\hookrightarrow \mathbb{K}$  is a cyclic Galois extension and $\sigma$ is a generator of the Galois group ${\rm Gal}(\mathbb{K}/\mathbb{F})$, then for $x\in \mathbb{K}$, the relative trace $\Tr_{\mathbb{K}/\mathbb{F}}(x)=0$ if and only if $x=\sigma(y)-y$, for some $y\in\mathbb{K}$.
%We also need the following two lemmas.  
%\begin{lem} 
%\label{lem10} 
%Let $n$ be a positive integer. We have:
%\begin{enumerate}
% \item[$(i)$] The equation
%$x^2 + ax + b = 0$, with $a,b\in\F_{2^n}$, $a\neq 0$,
%has two solutions in $\F_{2^n}$ if  $\Tr\left(
%\frac{b}{a^2}\right)=0$, and zero solutions otherwise \textup{(}see~\textup{\cite{BRS67}}\textup{)}.
%\item[$(ii)$]  The equation
%$x^2 + ax + b = 0$, with $a,b\in\F_{p^n}$, $p$ odd,
%has (two, respectively, one) solutions in $\F_{p^n}$ if and only if the discriminant $a^2-4b$ is a (nonzero, respectively, zero) square in $\F_{p^n}$.
%%\item[$(2)$] The equation
%%$x^3 + ax + b = 0$, with $a,b\in\F_{2^n}$, $b\neq 0$, has (denoting by $t_1,t_2$ the roots of $t^2+bt+a^3=0$):
%%\begin{itemize}
%%\item[$(i)$] three solutions in $\F_{2^n}$ if and only if $\Tr(a^3/b^2)=\Tr(1)$ and $t_1,t_2$ are cubes in $\F_{2^n}$ for $n$ even, and in $\F_{2^{2n}}$ for $n$ odd;
%%\item[$(ii)$] a unique solution in $\F_{2^n}$ if and only if $\Tr(a^3/b^2)\neq \Tr(1)$;
%%\item[$(iii)$] no solutions in $\F_{2^n}$ if and only if $\Tr(a^3/b^2)=\Tr(1)$ and $t_1,t_2$ are not cubes in $\F_{2^n}$ ($n$ even), $\F_{2^{2n}}$ ($n$ odd).
%%\end{itemize}
%\end{enumerate}
%\end{lem}
%
%%\begin{lem}[\textup{\cite{EFRST20}}]
%%\label{lem:gcd}
%%Let $p,k,n$ be integers greater than or equal to $1$ (we take $k\leq n$, though the result can be shown in general). Then
%%\begin{align*}
%%&  \gcd(2^{k}+1,2^n-1)=\frac{2^{\gcd(2k,n)}-1}{2^{\gcd(k,n)}-1},  \text{ and if  $p>2$, then}, \\
%%& \gcd(p^{k}+1,p^n-1)=2,   \text{ if $\frac{n}{\gcd(n,k)}$  is odd},\\
%%& \gcd(p^{k}+1,p^n-1)=p^{\gcd(k,n)}+1,\text{ if $\frac{n}{\gcd(n,k)}$ is even}.\end{align*}
%%Consequently, if either $n$ is odd, or $n\equiv 2\pmod 4$ and $k$ is even,   then $\gcd(2^k+1,2^n-1)=1$ and $\gcd(p^k+1,p^n-1)=2$, if $p>2$.
%%\end{lem}
 
  

 

\section{A description of the $c$-BCT of the power map $x^d$ in terms of characters on $\F_{p^n}$} 
\label{sec3}


We concentrate here on the $c$-boomerang uniformity of the power maps $x^d$
over finite field $\mathbb{F}_{p^n}$. Let $G$ be the Gauss' sum $\displaystyle G(\psi,\chi)=\sum_{z\in\F_q^*} \psi(z)\chi(z)$, where $\chi,\psi$, are additive, respectively, multiplicative characters of $\F_q$, $q=p^n$. Below, we let $\chi_1(a)=\exp\left(\frac{2\pi \imath \Tr(a)}{q}\right)$ be the principal additive character, and $\psi_k\left(g^\ell\right)=\exp\left(\frac{2\pi ik\ell}{q-1}\right)$ be the $k$-th multiplicative character of $\F_q$, $0\leq k\leq q-2$. We let $\psi_1$ be the generator of the cyclic group of multiplicative characters.
\begin{thm}
\label{thm:cBU_Char}
Let $F(x)=x^d$ be a monomial function $\F_q$, $q=p^n$, $p$ a prime number. Let $c\in\F_q^*$ and $b\in\F_q$. Then, the $c$-Boomerang Connectivity Table entry  $_c\cB_F(a,ab)$ at $(a,ab)$, $a\neq 0$, is given by 
{\small
\[
\frac{1}{q}\left(\Delta_{F,c}(1,b)+\Delta_{F,c^{-1}}(1,b)\right)-1 +\frac1{q^2}\sum_{\alpha,\beta\in\F_q,\alpha\beta\neq 0} \chi_1(-b(\alpha+\beta))\, S_{\alpha,\beta}\, S_{-\alpha c,-\beta c^{-1}},
\]
}
with
\begin{align*}
S_{\alpha,\beta}&=\sum_{x\in \F_q} \chi_1\left(\alpha x^d\right)\chi_1\left(\beta(x+1)^d\right)\\
&= \frac{1}{(q-1)^2} \sum_{j,k=0}^{q-2}   G(\bar\psi_j,\chi_1) G(\bar\psi_k,\chi_1) \sum_{x\in \F_q} \psi_1\left((\alpha x^d)^j (\beta(x+1)^d)^k\right).
\end{align*}
\end{thm}
\begin{proof}
For $b \neq 0$ and fixed $c\neq 1$, the $c$-boomerang uniformity of $x^d$ is given by $\displaystyle \max_{b \in \mathbb{F}_{p^n}^*}~  _cB_F(1,b) $, where $_cB_F(1,b)$ is the number of solutions in $\mathbb{F}_{q}\times \mathbb{F}_{q}$, $q=p^n$, of the following system
\begin{equation}
\label{eq:eq3.1}
\begin{cases}
 x^d-cy^d=b \\
(x+1)^d-c^{-1} (y+1)^d=b.
\end{cases}
\end{equation}
%%(Surely, the first equation will have solutions if $b$ is of the form $\alpha^d-c\beta^d$.)
%The number of solutions of~\eqref{eq:eq3.1} is less than or equal to the number of solutions for the system\begin{equation}
%\label{eq:eq3.2}
%\begin{cases}
%w^d-x^d+cy^d-c^{-1} z^d=0\\
%w-x-z+y=0.
%\end{cases}
%\end{equation}
%Observe that any solution $(x,y,w,z)$ of Equation~\eqref{eq:eq3.1} is a solution of Equation~\eqref{eq:eq3.2} since then $w-x=z-y=a'$ and the second  equation is automatically satisfied. Vice-versa, for any solution  $(x,y,w,z)$  of Equation~\eqref{eq:eq3.2}, there exists an $a$ such that  $(x,y,w,z)$ is a solution corresponding to $a_{xywz}$ (be aware that the $a_{xyzw}$ will possibly change if one changes $(x,y,w,z)$).
We know  that the number  $N(b)$  of of solutions $(x_1,\ldots,x_n)\in\F_q^n$, for $b$ fixed of an equation $f(x_1,\ldots,x_n)=b$ is
\begin{align*}
\cN(b)
&= \frac{1}{q}\sum_{x_1,\ldots,x_n\in \F_q}\sum_{\alpha\in\F_q} \chi_1\left(\alpha \left( f(x_1,\ldots,x_n)- b\right)\right)\\
&=\frac{1}{q}\sum_{x_1,\ldots,x_n\in \F_q}\sum_{\chi\in \widehat{\F_q}}\chi(f(x_1,\ldots,x_n))\overline{\chi(b)},
\end{align*}
where $\widehat{\F_q}$ is the set of all additive characters of $\F_q$, and $\chi_1$ is the principal additive character of  $\F_q$. Next, note  that the number of solutions  $(x_1,\ldots,x_n)\in\F_q^n$ of a system $f_1(x_1,\ldots,x_n)=b_1$,  $f_2(x_1,\ldots,x_n)=b_2$ is exactly
\[
\frac{1}{q^2}\sum_{x_1,\ldots,x_n\in \F_q}\sum_{\alpha,\beta\in\F_q} \chi_1\left(\alpha\left(f_1(x_1,\ldots,x_n)-b_1 \right) \right) \chi_1\left(\alpha\left(f_2(x_1,\ldots,x_n)-b_2 \right) \right).
\]
For our system~\eqref{eq:eq3.1}, we see that the number of solutions for some $a,b$ fixed is therefore
\allowdisplaybreaks
\begin{align*}
&\cN_{b;c}
=\frac{1}{q^2}\sum_{x,y\in \F_q}\sum_{\alpha,\beta\in\F_q} \chi_1\left(\alpha\left(x^d-cy^d-b \right) \right) \chi_1\left(\beta\left((x+1)^d-c^{-1} (y+1)^d-b \right) \right)\\
&=\frac{1}{q^2}\sum_{x,y\in \F_q}\sum_{\alpha,\beta\in\F_q} \chi_1(-b(\alpha+\beta)) \chi_1\left(\alpha x^d+\beta(x+1)^d\right) \overline{ \chi_1\left(\alpha cy^d+ \beta c^{-1} (y+1)^d  \right)}\\
&= \frac{1}{q^2}\sum_{\alpha,\beta\in\F_q} \chi_1(-b(\alpha+\beta))
\sum_{x\in \F_q}
\chi_1\left(\alpha x^d+\beta(x+1)^d\right) \sum_{y\in \F_q} \overline{ \chi_1\left(\alpha cy^d+ \beta c^{-1} (y+1)^d  \right)}.
\end{align*}
We now, rewrite the above expression as (the term $q^2$ comes from $\alpha=\beta=0$)
\allowdisplaybreaks
\begin{align*}
&q^2\cN_{b;c}+q^2= \sum_{\alpha\in\F_q,\beta=0} \chi_1(-b \alpha)\sum_{x\in \F_q}
\chi_1\left(\alpha x^d\right) \sum_{y\in \F_q}   \chi_1\left(-\alpha cy^d \right)\\
&\qquad +\sum_{\alpha=0,\beta\in\F_q} \chi_1(-b \beta)
\sum_{x\in \F_q}
\chi_1\left(\beta(x+1)^d\right) \sum_{y\in \F_q}  \chi_1\left(-\beta c^{-1} (y+1)^d  \right)\\
&\qquad +\sum_{\alpha,\beta\in\F_q,\alpha\beta\neq 0} \chi_1(-b(\alpha+\beta))
\sum_{x\in \F_q}
\chi_1\left(\alpha x^d+\beta(x+1)^d\right) \\
&\qquad\qquad\qquad\qquad\qquad\qquad\cdot  \sum_{y\in \F_q} \overline{ \chi_1\left(\alpha cy^d+ \beta c^{-1} (y+1)^d  \right)}\\
&=\sum_{x,y\in \F_q} \sum_{\alpha\in\F_q}\chi_1\left(\alpha\left( x^d-cy^d-b  \right) \right)\\
&\qquad +\sum_{x,y\in \F_q} \sum_{\beta\in\F_q}\chi_1\left(\beta\left( (x+1)^d-c^{-1}(y+1)^d-b  \right) \right)\\ 
&\qquad +\sum_{\alpha,\beta\in\F_q,\alpha\beta\neq 0} \chi_1(-b(\alpha+\beta))
\sum_{x\in \F_q}
\chi_1\left(\alpha x^d+\beta(x+1)^d\right)\\
&\qquad\qquad\qquad\qquad\qquad\qquad\cdot  \sum_{y\in \F_q} \overline{ \chi_1\left(\alpha cy^d+ \beta c^{-1} (y+1)^d  \right)}\\
&=\sum_{x,y\in \F_q} \sum_{\alpha\in\F_q}\chi_1\left(\alpha\left( x^d-cy^d-b  \right) \right)+\sum_{x,y\in \F_q} \sum_{\beta\in\F_q}\chi_1\left(\beta\left( x^d-c^{-1}y^d-b  \right) \right)\\ 
&\qquad +\sum_{\alpha,\beta\in\F_q,\alpha\beta\neq 0} \chi_1(-b(\alpha+\beta))
\sum_{x\in \F_q}
\chi_1\left(\alpha x^d+\beta(x+1)^d\right)\\
&\qquad\qquad\qquad\qquad\qquad\qquad\cdot  \sum_{y\in \F_q} \overline{ \chi_1\left(\alpha cy^d+ \beta c^{-1} (y+1)^d  \right)}\\
&=q\Delta_{F,c}(1,b)+q\Delta_{F,c^{-1}}(1,b) +\sum_{\alpha,\beta\in\F_q,\alpha\beta\neq 0} \chi_1(-b(\alpha+\beta))\, S_{\alpha,\beta}\, S_{-\alpha c,-\beta c^{-1}},
\end{align*}
where $S_{\alpha,\beta} =\sum_{x\in \F_q} \chi_1\left(\alpha x^d+\beta(x+1)^d\right)$.
Further,~\cite[Equation (5.17)]{LN97} relates the additive character $\chi$ to the cyclic group (of cardinality $q-1$) of all multiplicative characters $\psi$ of $\F_q$, via
\begin{align*}
\chi_1(w)&=\frac{1}{q-1} \sum_{z\in\F_q^*} \chi_1(z)\sum_{j=0}^{q-2} \psi_j(w) \overline{\psi_j(z)}
=\frac{1}{q-1} \sum_{j=0}^{q-2} G(\bar\psi_j,\chi_1) \psi_j(w),
\end{align*}
where $G$ is the Gauss' sum $\displaystyle G(\psi,\chi)=\sum_{z\in\F_q^*} \psi(z)\chi(z)$.
Using this, we get
\allowdisplaybreaks
\begin{align*}
S_{\alpha,\beta}&=\sum_{x\in \F_q} \chi_1\left(\alpha x^d\right)\chi_1\left(\beta(x+1)^d\right)\\
&=\frac{1}{(q-1)^2}  \sum_{x\in \F_q} \sum_{j,k=0}^{q-2} G(\bar\psi_j,\chi_1) G(\bar\psi_k,\chi_1)\psi_j\left(\alpha x^d\right)  \psi_k\left(\beta(x+1)^d\right),
%&=\frac{1}{(q-1)^2} \sum_{x\in \F_q}  \sum_{j,k=0}^{q-2} \psi_j(\alpha)\psi_k(\beta) G(\bar\psi_j,\chi_1) G(\bar\psi_k,\chi_1)\psi_j\left(x^d\right)  \psi_k\left((x+1)^d\right)\\
%&= \frac{1}{(q-1)^2} \sum_{j,k=0}^{q-2} \psi_j(\alpha)\psi_k(\beta) G(\bar\psi_j,\chi_1) G(\bar\psi_k,\chi_1) \sum_{x\in \F_q} \psi_j\left(x^d\right)  \psi_k\left((x+1)^d\right),
\end{align*}
from which we infer our claim.
\end{proof}

%If $c=-1$, then 
%\allowdisplaybreaks
%\begin{align*}
%\cN_{b;-1}&=\frac{1}{q^2}\sum_{x,y\in \F_q}\sum_{\alpha,\beta\in\F_q} \chi_1(-b(\alpha+\beta)) \chi_1\left(\alpha x^d+\beta(x+1)^d\right) \chi_1\left(\alpha y^d+ \beta  (y+1)^d  \right)\\
%&=\frac{1}{q^2}\sum_{\alpha,\beta\in\F_q} \chi_1(-b(\alpha+\beta)) \left(\sum_{x\in \F_q}\chi_1\left(\alpha x^d+\beta(x+1)^d\right) \right)^2\\
%&=\frac{1}{q^2}\sum_{\alpha,\beta\in\F_q}   \left(\sum_{x\in \F_q}\chi_1\left(\alpha \left(x^d-\frac{b}{2}\right)\right)\chi_1\left(\beta\left((x+1)^d-\frac{b}{2}\right)\right) \right)^2\\
%&=\frac{1}{q^2}\sum_{\alpha,\beta\in\F_q}   \sum_{x,y\in \F_q}\left[\chi_1\left(\alpha \left(x^d-\frac{b}{2}\right)\right)\chi_1\left(\beta\left((x+1)^d-\frac{b}{2}\right)\right) \right.\\
%&\qquad\qquad \quad\cdot 
%\left.\chi_1\left(\alpha \left(y^d-\frac{b}{2}\right)\right)\chi_1\left(\beta\left((y+1)^d-\frac{b}{2}\right)\right)\right] \\
%&=\frac{1}{q^2}   \sum_{x,y\in \F_q} \sum_{\alpha,\beta\in\F_q}\left[\chi_1\left(\alpha \left(x^d-\frac{b}{2}\right)\right)\chi_1\left(\beta\left((x+1)^d-\frac{b}{2}\right)\right) \right.\\
%&\qquad\qquad \quad\cdot 
%\left.\chi_1\left(\alpha \left(y^d-\frac{b}{2}\right)\right)\chi_1\left(\beta\left((y+1)^d-\frac{b}{2}\right)\right)\right] \\
%&=\frac{1}{q^2}  \sum_{x,y\in \F_q} \left(\sum_{\alpha\in\F_q} \chi_1\left(\alpha \left(x^d+y^d-b\right)\right)\right) \left( \sum_{\beta\in\F_q} \chi_1\left(\beta\left((x+1)^d+(y+1)^d-b\right)\right) \right) \\
%&\qquad  
% \sum_{y\in \F_q} \left(\sum_{\alpha\in\F_q} \chi_1\left(\alpha \left(y^d-\frac{b}{2}\right)\right)\sum_{\beta\in\F_q} \chi_1\left(\beta\left((y+1)^d-\frac{b}{2}\right)\right) \right) 
%\end{align*}

\begin{remark}
In the previous theorem and the next ones, we could have embedded the differential entries $\Delta_{F,c}(1,b),\Delta_{F,c^{-1}}(1,b)$ into the character sums, but we wanted to point out how the $c$-BCT entries depend upon the $c$-DUT entries.
\end{remark}
 \begin{rem}
 Surely, we could have written the previous theorem for any function $F$, but we simply wanted it for the Gold functions from the next section. We may come back to that idea for other functions.
 \end{rem}

\section{The $c$-BCT for all Gold functions $x^{p^k+1}$, $p$ odd}
\label{sec4}
We will now use this approach to push  even further the above result for the Gold function. It is perhaps the first result of this type that computes the boomerang uniformity and its relative, the $c$-boomerang uniformity for all functions in this class (we gave a lower bound in~\cite{S20}
the $c$-boomerang uniformity). 
We will, in fact, find all entries in the $c$-BCT, including $c=1$, as well,  for all $c\neq 0$.

We shall make use of the following result from~\cite{Co98} (we make slight changes in notations and combine various results). Let $1\leq k<n$, $d=\gcd(n,k)$, and $\mathscr{S}_k(A,B)=\sum_{x\in \F_q} \chi_1\left(A x^{p^k+1}+ B x\right)$. We let $\eta=\psi_{(q-1)/2}$ be the quadratic character of $\F_q$.
\begin{thm}[\textup{\cite{Co98}}]
\label{thm:Co98}
Let $q=p^n$, $n\geq 2$, $p$ an  odd prime, $1\leq k<n$, $e=\gcd(n,k)$. Let $f(x)=A^{p^k} x^{p^{2k}}+Ax$, for some nonzero $A$. The following statements hold:
\begin{enumerate}
\item[$(1)$]
If $f$ is a permutation polynomial over $\F_q$, and  $x_0$ is the unique element such that $f(x_0)=-B^{p^k},B\neq 0$, then:
\begin{itemize}
\item[$(i)$] If $\frac{n}{e}$ is odd, then 
\[
\mathscr{S}_k(A,B)=
\begin{cases}
(-1)^{n-1} \sqrt{q}\,\eta(-A)\,\overline{\chi_1(Ax_0^{p^k+1})} &\text{ if } p\equiv 1\pmod 4\\
(-1)^{n-1} \imath^{3n} \sqrt{q}\,\eta(-A)\,\overline{\chi_1(Ax_0^{p^k+1})} &\text{ if } p\equiv 3\pmod 4.
\end{cases}
\]
(the solution $x_0=-\frac{1}{2}\sum_{j=0}^{\frac{n}{e}-1} (-1)^j A^{-\frac{p^{(2j+1)k}+1}{p^k+1}} B^{p^{(2j+1)k}}$).
\item[$(ii)$] If $\frac{n}{e}$ is even, then $n=2m$, $A^{\frac{q-1}{p^e+1}}\neq (-1)^{\frac{m}{e}}$ and
\[
\mathscr{S}_k(A,B)=(-1)^{\frac{m}{e}} p^m\, \overline{\chi_1(Ax_0^{p^k+1})}.
\]
\end{itemize}
\item[$(2)$] If $f$ is not a permutation polynomial, then, for $B\neq 0$, $\mathscr{S}_k(A,B)=0$, unless, $f(x)=-B^{p^k}$ has a solution $x_0$ (this can only happen if $\frac{n}{e}$ is even with $n = 2m$, and $A^{\frac{q-1}{p^e+1}}= (-1)^{\frac{m}{e}}$), in which case
\[
\mathscr{S}_k(A,B)=(-1)^{\frac{m}{e}+1}p^{m+e} \overline{\chi_1(Ax_0^{p^k+1})}.
\]
\end{enumerate}
\end{thm}


The proof of our results are long and complicated, so we will split the analysis into several cases $c=1,c=-1$, etc., and record each case in a separate theorem. The goal in each case is to make more explicit the expressions of Theorem~\ref{thm:cBU_Char} for the Gold functions.

We  need some notations below.  For $1\leq k<n$, let $\alpha,\beta\in\F_{p^n}$, $L_{\alpha,\beta}(x)=(\alpha+\beta)x^{p^{2k}}+(\beta^{p^{n-k}}+\beta)x$ and $\sY_1$, $\sY_2$ be the set of $(\alpha,\beta)\in\F_{p^n}^{*2}$ where $L_{\alpha,\beta}$, respectively,   $L_{-\alpha c,-\beta c^{-1}}$   are not permutations.

Further, let $\mathscr{A}_1$ be the set of all $(\alpha,\beta)\in\F_{p^n}^{*2}$ such that $L_{\alpha,\beta}(x)=-(\beta+\beta^{p^k})$ has a root $x_{\alpha,\beta}$,  and
$\alpha,\beta$ satisfy (with $d=\gcd(2k,n)$)
\[
(-1)^{\frac{n}{d}}\left(\frac{\beta^{p^{n-k}}+\beta}{\alpha+\beta} \right)^{\frac{p^n-1}{p^d-1}}=1,
\]
(hence, $L_{\alpha,\beta}$ is not a permutation polynomial~\cite{ZWW20}). Observe that the left hand side expression is just the relative norm from $\F_{p^n}$ to $\F_{p^d}$ of the argument.
Similarly, let $\sA_2$ be the set of all $(\alpha,\beta)\in\F_{p^n}^{*2}$ such that $L_{-\alpha c,-\beta c^{-1}}(x)=(\beta c^{-1}+(\beta c^{-1})^{p^k})$ has a root $x_{-\alpha c,-\beta c^{-1}}$,  and
$\alpha,\beta$ satisfy  
\[
(-1)^{\frac{n}{d}}\left(\frac{(c^{-1}\beta)^{p^{n-k}}+c^{-1}\beta}{\alpha c+\beta c^{-1}} \right)^{\frac{p^n-1}{p^d-1}}=1
\]
(hence, $L_{-\alpha c,-\beta c^{-1}} $ is not a permutation polynomial).


We showed in~Theorem~\ref{thm:cBU_Char} that ${_c}B_F(a,ab)$ equals (we let $q=p^n$)
\begin{equation}
\label{eq:eq32}
\frac{1}{q}\left(\Delta_{F,c}(1,b)+q\Delta_{F,c^{-1}}(1,b)\right)-1 +\frac1{q^2}\sum_{\alpha,\beta\in\F_q,\alpha\beta\neq 0} \chi_1(-b(\alpha+\beta))\, S_{\alpha,\beta}\, S_{-\alpha c,-\beta c^{-1}},
\end{equation}
where
$
S_{\alpha,\beta}=\sum_{x\in \F_q} \chi_1\left(\alpha x^d\right)\chi_1\left(\beta(x+1)^d\right)$, and $d=p^k+1$. We let $\displaystyle T_b=\sum_{\alpha,\beta\in\F_q,\alpha\beta\neq 0} \chi_1(-b(\alpha+\beta))\, S_{\alpha,\beta}\, S_{-\alpha c,-\beta c^{-1}}$.

We now  concentrate on $S_{\alpha,\beta}$, for $\alpha\beta\neq 0$. Using the fact that $\chi_1(u^p)=\chi_1(u)$ for $u\in\F_q$, we compute
\allowdisplaybreaks
\begin{align*}
S_{\alpha,\beta}
&=\sum_{x\in \F_q} \chi_1\left(\alpha x^{p^k+1}+\beta(x+1)^{p^k+1}\right)\\
&=\sum_{x\in \F_q} \chi_1\left((\alpha+\beta)  x^{p^k+1}+ \beta x^{p^k} +\beta x+\beta)\right)\\
&=\sum_{x\in \F_q} \chi_1\left((\alpha+\beta)  x^{p^k+1}\right) \chi_1\left( (\beta^{p^{n-k}} x)^{p^k}\right)\chi_1( \beta x+\beta)\\
&=\sum_{x\in \F_q} \chi_1\left((\alpha+\beta)  x^{p^k+1}\right) \chi_1\left( \beta^{p^{n-k}} x\right)\chi_1( \beta x+\beta)\\
&=\sum_{x\in \F_q} \chi_1\left((\alpha+\beta)  x^{p^k+1}\right) \chi_1\left( (\beta^{p^{n-k}} +\beta) x+\beta\right)\\
&=\chi_1(\beta) \sum_{x\in \F_q} \chi_1\left((\alpha+\beta)  x^{p^k+1}+ (\beta^{p^{n-k}} +\beta) x\right).
\end{align*}
Let  $A:=\alpha+\beta, B=\beta^{p^{n-k}}+\beta$ (recall $\alpha\beta\neq 0$). 
If $\alpha=-\beta=\beta^{p^{n-k}}$ (the last identity can only happen for  $\frac{n}{e}$ even, where $e=\gcd(n,k)$), then  $S_{\alpha,\beta}=q\chi_1(\beta)$ (there are $p^e-1$ such nonzero $\beta$'s, since $\beta\neq 0,\beta^{p^{n-k}}+\beta=0$ is equivalent to $\beta^{p^k-1}+1=0$). If $\alpha=-\beta\neq \beta^{p^{n-k}}$, then $S_{\alpha,\beta}=0$. If   $\alpha\neq -\beta= \beta^{p^{n-k}}$, then we  use~\cite[Theorem 1 and 2]{Co98_1} (observe that the case $\frac{n}{e}$ odd does not happen), obtaining that for  $\frac{n}{e}$ is even (thus, $n=2m$), then  (with $A=\alpha+\beta$; we simplify a bit the original statement)
 \[
S_{\alpha,\beta}=
\begin{cases}
 (-1)^{\frac{m}{e}}\, p^m\,\chi_1(\beta) &\text{ if } A^{\frac{q-1}{p^e+1}}\neq (-1)^{\frac{m}{e}}\\ 
 (-1)^{\frac{m}{e}+1}\, p^{m+e}\,\chi_1(\beta) &\text{ if } A^{\frac{q-1}{p^e+1}}= (-1)^{\frac{m}{e}}. 
\end{cases}
\]


We next assume that $\alpha\neq -\beta\neq \beta^{p^{n-k}}$.
We shall now be using Theorem~\ref{thm:Co98},
which gives explicitly the sum $\mathscr{S}_k(A,B)=\sum_{x\in \F_q} \chi_1\left(A x^{p^k+1}+ B x\right)$ depending upon whether $L_{\alpha,\beta}(x)=(\alpha^{p^k}+\beta^{p^k}) x^{p^{2k}}+(\alpha+\beta)x$  is a permutation polynomial or not. 

Now, it is known that a linearized polynomial of the form $L_r(x)=x^{p^r}+\gamma x\in\F_{p^n}$ is a permutation polynomial if and only if the relative norm $N_{\F_{p^n}/\F_{p^d}}(\gamma)\neq 1$, that is, $(-1)^{n/d} \gamma^{(p^n-1)/(p^d-1)}\neq 1$, where $d=\gcd(n,r)$. For our polynomial $L_{\alpha,\beta}=(\alpha+\beta)x^{p^{2k}}+(\beta^{p^{n-k}}+\beta)x$, (dividing by $\alpha+\beta\neq 0$) the previous nonpermutability condition becomes (with $d=\gcd(2k,n)$)
\begin{equation}
\label{eq:pp_eq}
(-1)^{\frac{n}{d}}\left(\frac{\beta^{p^{n-k}}+\beta}{\alpha+\beta} \right)^{\frac{p^n-1}{p^d-1}}=1.
\end{equation}
Surely, there are  $\frac{p^n-1}{p^d-1}$ roots for the equation $x^{\frac{p^n-1}{p^d-1}}=(-1)^{\frac{n}{d}}$, forming a set $\sX_1$ of cardinality $\frac{p^n-1}{p^d-1}$. We then see that for an arbitrary $\gamma\in \sX_1$, and any $\beta\in\F_{p^n}$, then there is a unique $\alpha\in\F_{p^n}$ such that $\frac{\beta^{p^{n-k}}+\beta}{\alpha+\beta}=\gamma$, namely $\alpha=\frac{\beta^{p^{n-k}}+\beta-\beta \gamma}{\gamma}$. Therefore, there are $\leq \frac{p^n(p^n-1)}{p^d-1}$ pairs $\alpha,\beta$ forming a set $\sY_1$ (with the restrictions $\alpha\neq -\beta\neq \beta^{p^{n-k}}$)  such that $L_{\alpha,\beta}$ is not a permutation (the reason that the number of pairs is not precisely $ \frac{p^n(p^n-1)}{p^d-1}$ is because more than one $\beta$ may generate the same $\alpha$ if the linearized $x^{p^{n-k}}+(1-\gamma)x$ is not a permutation polynomial).   Let $\mathscr{A}_1\subseteq \sY_1$ be the subset of all $(\alpha, \beta)\in\sY_1$ such that $L_{\alpha,\beta}(x)=-(\beta+\beta^{p^k})$ has at least a root $x_{\alpha,\beta}$. 
Surely, if $L_{\alpha,\beta}$ is not a permutation on $\F_q$ and $L_{\alpha,\beta}=-(\beta+\beta^{p^k})$ has no root, then  $S_{\alpha,\beta}=0$.

If $L_{\alpha,\beta}$ is not a PP (abbreviation of ``permutation polynomial''), but the  linearized equation has a root (hence, by Theorem~\ref{thm:Co98},   $\frac{n}{e}$ is even), then for all  $(\alpha,\beta)\in \sA_1$ (note that the cardinality of $|\sA_1|\leq \frac{p^n(p^n-1)}{p^d-1}$), then 
\[
S_{\alpha,\beta}=(-1)^{\frac{n}{2e}+1}p^{\frac{n}{2}+e} \chi_1(\beta) \overline{\chi_1\left((\alpha+\beta) x_{\alpha,\beta}^{p^k+1}\right)}.
\]

In addition, again, by Theorem~\ref{thm:Co98}, when $\frac{n}{e}$ is even   and $L_{\alpha,\beta}$ is a permutation on $\F_q$ and $x_{\alpha,\beta}$ is the root of $L_{\alpha,\beta}=-(\beta+\beta^{p^k})$ then 
\[
S_{\alpha,\beta}=(-1)^{\frac{n}{2e}}p^{\frac{n}{2}}  \chi_1(\beta) \overline{\chi_1\left((\alpha+\beta) x_{\alpha,\beta}^{p^k+1}\right)}.
\]


 Finally, if  $\frac{n}{e}$ is odd and $L_{\alpha,\beta}$ is a permutation on $\F_q$ and $x_{\alpha,\beta}$ is the root of $L_{\alpha,\beta}=-(\beta+\beta^{p^k})$, then $S_{\alpha,\beta}$ equals
\begin{align*}
(-1)^{n-1} \sqrt{q}\,\chi_1(\beta)\eta(-\alpha-\beta)\,\overline{\chi_1((\alpha+\beta)x_{\alpha,\beta}^{p^k+1})},\ &\text{ if } p\equiv 1\pmod 4\\
(-1)^{n-1} \imath^{3n} \sqrt{q}\,\chi_1(\beta)\eta(-\alpha-\beta)\,\overline{\chi_1((\alpha+\beta)x_{\alpha,\beta}^{p^k+1})},\ &\text{ if } p\equiv 3\pmod 4.
\end{align*}


We  take $\sX_2,\sY_2,\sA_2$ to be the corresponding sets as above, where $\alpha,\beta$ are replaced by $-\alpha c,-\beta c^{-1}$, respectively. 

\subsection{The case $c=1$}

%We now go back to the double sum of Equation~\eqref{eq:eq32} (recall that $\alpha\beta\neq 0$). 
 
 
Observe that the conditions $\alpha=-\beta$, $-\beta=\beta^{p^{n-k}}$, and $-c\alpha=c\beta$, $c\beta=(-c\beta)^{p^{n-k}}$ are the equivalent, when $c=1$. We here use the notation 
\[
\displaystyle \Sigma_1= \sum_{A, A^{\frac{q-1}{p^e+1}}= (-1)^{\frac{m}{e}}} \chi_1(-bA).
\]

First, if  $\frac{n}{e}$ is odd, and $p\equiv 1\pmod 4$,  $\alpha\neq -\beta\neq \beta^{p^{n-k}}$ and $L_{\alpha,\beta}$ is  a permutation (as well as, $L_{-\alpha,-\beta}$) (recall that $(\alpha,\beta)\in \overline{\sY_1}$), then 
\allowdisplaybreaks
\begin{align*}
S_{\alpha,\beta}S_{-\alpha,-\beta}=&
 q\,\chi_1(\beta)\eta(-\alpha-\beta)\,\overline{\chi_1((\alpha+\beta)x_{\alpha,\beta}^{p^k+1})}\\
&\qquad \cdot \chi_1(-\beta)\eta(\alpha+\beta) \,\overline{\chi_1(-(\alpha+\beta)x_{\alpha,\beta}^{p^k+1})}= q (-1)^n,
\end{align*}
and so,
\[
T_{b}^{(p)}=(-1)^n p^{n}\sum_{(\alpha,\beta)\in\bar\sY_1} \chi_1(-b(\alpha+\beta)).
\]
Similarly, in the same case, if $p\equiv 3\pmod 4$, then  (there are two extra copies of $\imath^{3n}$ rendering a factor of $(-1)^n$)
\begin{align*}
S_{\alpha,\beta}S_{-\alpha,-\beta}&= (-1)^n  q\,\eta(-1)=q,
\end{align*}
therefore,
\[
T_{b}^{(p)}=p^{n}\sum_{(\alpha,\beta)\in\bar\sY_1} \chi_1(-b(\alpha+\beta)).
\]

Therefore, when $\frac{n}{e}$ is odd, we can uniquely write this as 
\begin{align*}
T_b&=(-1)^{n\frac{p+1}{2}} p^n \sum_{(\alpha,\beta)\in\bar\sY_1} \chi_1(-b(\alpha+\beta)).
\end{align*}

If  $0\neq \alpha=-\beta= \beta^{p^{n-k}}$ (there are $p^e-1$ such roots $\beta$), so, $\frac{n}{e}$ is even, then $S_{\alpha,\beta}=q\chi_1(\beta)$, $S_{-\alpha,-\beta}=q\chi_1(-\beta)$, and so,
\allowdisplaybreaks
\begin{align*}
T_{b,1}&=\sum_{0\neq \alpha=-\beta= \beta^{p^{n-k}}}\chi_1(-b(\alpha+\beta)) S_{\alpha,\beta}S_{-\alpha,-\beta}\\
& = p^{2n} \sum_{0\neq \alpha=-\beta= \beta^{p^{n-k}}} \chi_1(-b(\alpha+\beta))\chi_1(\beta)\chi_1(-\beta) =p^{2n}(p^e-1).
\end{align*}

Assume $\alpha\neq-\beta= \beta^{p^{n-k}}$, so $\frac{n}{e}$ is even. We observe that either both $\alpha+\beta,-(\alpha+\beta)$ satisfy  $X^{\frac{q-1}{p^e+1}}=(-1)^{\frac{m}{e}}$, or none will do. Furthermore,  
\begin{align*}
S_{\alpha,\beta}S_{-\alpha,-\beta}= 
\begin{cases} 
p^{n}  &\text{ if } A^{\frac{q-1}{p^e+1}}\neq (-1)^{\frac{m}{e}}\\
p^{n+2e}&\text{ if } A^{\frac{q-1}{p^e+1}}=(-1)^{\frac{m}{e}}.
\end{cases}
\end{align*}
Thus, if $A^{\frac{q-1}{p^e+1}}= (-1)^{\frac{m}{e}}$,  then 
\begin{align*}
T_{b,2}&=p^{n+2e} \sum_{\substack{\alpha,\beta\in\F_q^*\\ A^{\frac{q-1}{p^e+1}}= (-1)^{\frac{m}{e}}}} \chi_1(-b(\alpha+\beta))  \\
&=p^{n+2e}\sum_{A ,A^{\frac{q-1}{p^e+1}}=(-1)^{\frac{m}{e}}}\sum_{\beta\in\F_q^*}  \chi_1(-bA )\\
&=p^{n+2e}(p^n-1) \Sigma_1.
\end{align*}
If $A^{\frac{q-1}{p^e+1}}\neq  (-1)^{\frac{m}{e}}$, $A\neq 0$ (since $\alpha\neq -\beta$), then
\begin{align*}
T_{b,3}&=p^{n} \sum_{\substack{\alpha,\beta\in\F_q^*\\ A^{\frac{q-1}{p^e+1}}\neq (-1)^{\frac{m}{e}}}} \chi_1(-b(\alpha+\beta)) =p^n(p^n-1) \sum_{\substack{A\neq 0\\ A^{\frac{q-1}{p^e+1}}\neq (-1)^{\frac{m}{e}}}} \chi_1(-bA)\\
&=  p^n(p^n-1) \left(\sum_{A\in\F_q}\chi_1(-bA)-  \sum_{A,A^{\frac{q-1}{p^e+1}}= (-1)^{\frac{m}{e}}} \chi_1(-bA)-1\right)\\
& =- p^n(p^n-1)\left(1+ \Sigma_1\right).
\end{align*}

Now, let  $\alpha\neq-\beta\neq \beta^{p^{n-k}}$ and $\frac{n}{e}$ even. Then (we assume that $x_{\alpha,\beta}$ is a root of $L_{\alpha,\beta}(x)=-(\beta+\beta^{p^k})$, if it exists; observe also that that root happens for $L_{-\alpha,-\beta}(x)=(-\beta)+(-\beta)^{p^k}$, as well),
\begin{align*}
S_{\alpha,\beta}S_{-\alpha,-\beta}= 
\begin{cases} 
p^{n}\chi_1\left(-2Ax_{\alpha,\beta}^{p^k+1}\right) &\text{ if   $L_{\alpha,\beta}$ is PP}\\
p^{n+2e}\chi_1\left(-2Ax_{\alpha,\beta}^{p^k+1}\right) &\text{ if   $L_{\alpha,\beta}$ is not PP}.
\end{cases}
\end{align*}
In this case, then,
\begin{align*}
T_{b,4}&=p^{n+2e} \sum_{(\alpha,\beta)\in\sA_1}   \overline{\chi_1\left((\alpha+\beta)\left(2x_{\alpha,\beta}^{p^k+1}+b\right)\right)}\\
&\quad +p^n\sum_{(\alpha,\beta)\in\bar \sY_1} \overline{\chi_1\left((\alpha+\beta)\left(2x_{\alpha,\beta}^{p^k+1}+b\right)\right)}.
\end{align*}
Thus, when $\frac{n}{e}$ is even, 
\[
T_b=T_{b,1}+T_{b,2}+T_{b,3}+T_{b,4}.
\]
We record what we have shown below.
\begin{thm}
\label{thm:c=1}
Let $F(x)=x^{p^k+1}$, $1\leq k<n$, be the Gold function on $\F_{p^n}$, $p$ and odd prime,  $n\geq 2$, and $c=1$. The Boomerang Connectivity Table entry of $F$ at $(a,ab$) is 
$\displaystyle _c\cB_F(a,ab)=\frac{q+1}{q}\Delta_{F}-1 +\frac1{q^2}T_b$, where:
\begin{itemize}
\item[$(i)$] If $\frac{n}{e}$ is odd,   then  
\begin{align*}
T_b&=(-1)^{n\frac{p+1}{2}} p^n \sum_{(\alpha,\beta)\in\bar\sY_1} \chi_1(-b(\alpha+\beta)).
\end{align*}
\item[$(ii)$] If  $\frac{n}{e}$ even, then, with $A=\alpha+\beta$ and $\Sigma_1=\displaystyle \sum_{\substack{\alpha,\beta\in\F_q^*\\ A^{\frac{q-1}{p^e+1}}= (-1)^{\frac{m}{e}}}} \chi_1(-bA)$,
\allowdisplaybreaks
\begin{align*}
T_b=&\left(p^{2n+e}-2p^{2n}+p^n \right)
 +\left(p^{2n+2e}-p^{n+2e}-p^{2n}+p^n \right) \Sigma_1\\
& +p^{n+2e} \sum_{(\alpha,\beta)\in\sA_1}  \overline{\chi_1\left((\alpha+\beta)\left(2x_{\alpha,\beta}^{p^k+1}+b\right)\right)}\\
& +p^n\sum_{(\alpha,\beta)\in\bar \sY_1} \overline{\chi_1\left((\alpha+\beta)\left(2x_{\alpha,\beta}^{p^k+1}+b\right)\right)}.
\end{align*}
\end{itemize}
\end{thm}
 

\subsection{The case $c=-1$}

 In this case, $S_{\alpha,\beta}=S_{-\alpha c,-\beta c^{-1}}$. 
In addition to $\Sigma_1$ from Theorem~\ref{thm:c=1}, we let $\displaystyle \Sigma_2= \sum_{\beta^{p^k-1}+1=0} \chi_1(2\beta)$. 
 
 If $\frac{n}{e}$ is odd and $\alpha\neq -\beta\neq \beta^{p^{n-k}}$, then 
\begin{align*}
\left(S_{\alpha,\beta}\right)^2&=p^n \chi_1\left(2\beta-2(\alpha+\beta)x_{\alpha,\beta}^{p^k+1}\right) \text{ if  $L_{\alpha,\beta}$ is   PP,  $p\equiv 1\pmod 4$},\\
\left(S_{\alpha,\beta}\right)^2&=(-1)^n p^n \chi_1\left(2\beta-2(\alpha+\beta)x_{\alpha,\beta}^{p^k+1}\right) \text{ if  $L_{\alpha,\beta}$ is  PP, $p\equiv 3\pmod 4$}.
\end{align*}
We then get, if $L_{\alpha,\beta}$ is a PP, then
\begin{align*}
T_{b}^{(p)}&=p^{n} \sum_{(\alpha,\beta)\in\bar \sY_1}  \chi_1\left(2\beta-A\left(b+2x_{\alpha,\beta}^{p^k+1}\right)\right)\text{ if } p\equiv 1\pmod 4\\
T_{b}^{(p)}&=(-1)^n p^{n} \sum_{(\alpha,\beta)\in\bar \sY_1}  \chi_1\left(2\beta-A\left(b+2x_{\alpha,\beta}^{p^k+1}\right)\right) \text{ if } p\equiv 3\pmod 4.
\end{align*}
Therefore, if $\frac{n}{e}$ is odd, then, we can write this uniquely as
\[
T_b=(-1)^{n\frac{p-1}{2}} p^{n} \sum_{(\alpha,\beta)\in\bar \sY_1}  \chi_1\left(2\beta-A\left(b+2x_{\alpha,\beta}^{p^k+1}\right)\right).
\]

If   $\frac{n}{e}$ is even and $0\neq \alpha=-\beta= \beta^{p^{n-k}}$, then
\begin{align*}
T_{b,1}&= \sum_{0\neq \alpha=-\beta= \beta^{p^{n-k}}}\chi_1(-b(\alpha+\beta)) \left(S_{\alpha,\beta}\right)^2  \\ 
&=p^{2n} \sum_{\beta,\beta^{p^k-1}+1=0} \chi_1(2\beta)=p^{2n}  \Sigma_2.
\end{align*}
If $\frac{n}{e}$ is even ($n=2m$) and $0\neq \alpha\neq -\beta= \beta^{p^{n-k}}$, then
\begin{align*}
\left(S_{\alpha,\beta}\right)^2  = 
\begin{cases} 
p^{n+2e}\chi_1(2\beta) &\text{ if } A^{\frac{q-1}{p^e+1}}=(-1)^{\frac{m}{e}}\\
p^{n}\chi_1(2\beta)  &\text{ if } A^{\frac{q-1}{p^e+1}}\neq (-1)^{\frac{m}{e}}
\end{cases}
\end{align*}
Thus, if $\frac{n}{e}$ is even, $0\neq \alpha\neq -\beta= \beta^{p^{n-k}}$ and $A^{\frac{q-1}{p^e+1}}= (-1)^{\frac{m}{e}}$,  then 
\begin{align*}
T_{b,2}&=p^{n+2e} \sum_{\substack{\alpha,\beta\in\F_q^*,\beta^{p^k-1}+1=0 \\ A^{\frac{q-1}{p^e+1}}= (-1)^{\frac{m}{e}}}} \chi_1(-b(\alpha+\beta)) \chi_1(2\beta)\\
&=p^{n+2e} \sum_{A,A^{\frac{q-1}{p^e+1}}=(-1)^{\frac{m}{e}}} \chi_1(-bA)  \sum_{\beta,\beta^{p^k-1}+1=0} \chi_1(2\beta)=p^{n+2e}\Sigma_1\Sigma_2.
\end{align*}
If $ \frac{n}{e}$ is even, $0\neq \alpha\neq -\beta= \beta^{p^{n-k}}$ and $A^{\frac{q-1}{p^e+1}}\neq  (-1)^{\frac{m}{e}}$  (note that $A\neq 0$), then
\begin{align*}
T_{b,3}&=p^{n} \sum_{\substack{\alpha,\beta\in\F_q^*,\beta^{p^k-1}+1=0 \\ A\neq 0,A^{\frac{q-1}{p^e+1}}\neq (-1)^{\frac{m}{e}}}}  \chi_1(2\beta-b(\alpha+\beta)) \\
&=p^n \sum_{\beta,\beta^{p^k-1}+1=0} \chi_1(2\beta) \sum_{\substack{A\neq 0\\ A^{\frac{q-1}{p^e+1}}\neq (-1)^{\frac{m}{e}}}} \chi_1(-bA)\\
&=  -p^n\Sigma_1  \left(\sum_{A,A^{\frac{q-1}{p^e+1}}=(-1)^{\frac{m}{e}}} \chi_1(-bA)+1 \right)\\
&=-p^n\Sigma_2\left(\Sigma_1+1 \right).
\end{align*}
Now, we let   $\frac{n}{e}$ be even and $\alpha\neq-\beta\neq \beta^{p^{n-k}}$. Then (we assume that $x_{\alpha,\beta}$ is a root of $L_{\alpha,\beta}(x)=-(\beta+\beta^{p^k})$, if it exists),
\begin{align*}
\left(S_{\alpha,\beta}\right)^2  = 
\begin{cases} 
p^{n+2e} \chi_1(2\beta) \overline{ \chi_1\left(2Ax_{\alpha,\beta}^{p^k+1}\right)} &\text{ if   $L_{\alpha,\beta}$ is not PP}\\
p^{n}\chi_1(2\beta) \overline{\chi_1\left(2Ax_{\alpha,\beta}^{p^k+1}\right)} &\text{ if   $L_{\alpha,\beta}$ is PP}.
\end{cases}
\end{align*}
In this case, then,
\begin{align*}
T_{b,4}&=p^{n+2e} \sum_{(\alpha,\beta)\in\sA_1}   \chi_1\left(2\beta-(\alpha+\beta)\left(2x_{\alpha,\beta}^{p^k+1}+b\right)\right)\\
&\qquad\qquad\qquad  +p^n\sum_{(\alpha,\beta)\in\bar \sY_1}    \chi_1\left(2\beta-(\alpha+\beta)\left(2x_{\alpha,\beta}^{p^k+1}+b\right)\right).
\end{align*}


Therefore, if $\frac{n}{e}$ is even, then
\allowdisplaybreaks
\begin{align*}
T_b&=T_{b,1}+T_{b,2}+T_{b,3}+T_{b,4}\\
&=p^{2n}\Sigma_1+p^{n+2e}\Sigma_1\Sigma_2-p^n\Sigma_1(\Sigma_2+1)\\
&\quad +p^{n+2e} \sum_{(\alpha,\beta)\in\sA_1}    \chi_1\left(2\beta-(\alpha+\beta)\left(2x_{\alpha,\beta}^{p^k+1}+b\right)\right)\\
&\quad   +p^n\sum_{(\alpha,\beta)\in\bar \sY_1}    \chi_1\left(2\beta-(\alpha+\beta)\left(2x_{\alpha,\beta}^{p^k+1}+b\right)\right)\\
&=p^n(p^n-1)\Sigma_2+p^n(p^{2e-1}-1)\Sigma_1\Sigma_2\\
&\quad +p^{n+2e} \sum_{(\alpha,\beta)\in\sA_1}    \chi_1\left(2\beta-(\alpha+\beta)\left(2x_{\alpha,\beta}^{p^k+1}+b\right)\right)\\
&\quad   +p^n\sum_{(\alpha,\beta)\in\bar \sY_1}   \chi_1\left(2\beta-(\alpha+\beta)\left(2x_{\alpha,\beta}^{p^k+1}+b\right)\right).
\end{align*}
We record this result next: recall that below, $x_{\alpha,\beta}$ is the root of $L_{\alpha,\beta}(x)=-(\beta^{p^k}+\beta)$, which always exists if $(\alpha,\beta)\in \cA_1\cup\bar\sY_1$, and 
\[
  \Sigma_1=\sum_{A,A^{\frac{q-1}{p^e+1}}=(-1)^{\frac{m}{e}}} \chi_1(-bA),\quad   \Sigma_2= \sum_{\beta^{p^k-1}+1=0} \chi_1(2\beta).
\]
\begin{thm}
Let $F(x)=x^{p^k+1}$, $1\leq k<n$, be the Gold function on $\F_{p^n}$, $p$ and odd prime,  $n\geq 2$, and $c=-1$. The $(-1)$-Boomerang Connectivity Table entry of $F$ at $(a,ab$) is 
$\displaystyle _c\cB_F(a,ab)=\frac{q+1}{q}\Delta_{F,-1}(1,b)-1 +\frac1{q^2}T_b$, where:
\begin{itemize}
\item[$(i)$] If $\frac{n}{e}$ is odd, then 
\begin{align*}
T_b&=(-1)^{n\frac{p-1}{2}} p^{n} \sum_{(\alpha,\beta)\in\bar \sY_1}  \chi_1\left(2\beta-A\left(b+2x_{\alpha,\beta}^{p^k+1}\right)\right).
\end{align*}
\item[$(ii)$]  If $\frac{n}{e}$ is even,
\allowdisplaybreaks
\begin{align*}
T_b&=p^n(p^n-1)\Sigma_1+p^n(p^{2e}-1)\Sigma_1\Sigma_2\\
&\quad +p^{n+2e} \sum_{(\alpha,\beta)\in\sA_1}    \chi_1\left(2\beta-(\alpha+\beta)\left(2x_{\alpha,\beta}^{p^k+1}+b\right)\right)\\
&\quad   +p^n\sum_{(\alpha,\beta)\in\bar \sY_1}   \chi_1\left(2\beta-(\alpha+\beta)\left(2x_{\alpha,\beta}^{p^k+1}+b\right)\right).
\end{align*}
\end{itemize}
\end{thm}
 

\subsection{The case $c^{p^k-1}=1$}

  Since they were treated earlier, we assume here that $c\neq \pm 1$. In this case, the conditions $-\beta=\beta^{p^{n-k}}$, and  $c^{-1}\beta=(-c^{-1} \beta)^{p^{n-k}}$ are equivalent. We will be using below that if  $-\beta=\beta^{p^{n-k}}$ (which is the same as $\beta^{p^k-1}=-1$, or, even further, $\beta^{p^e-1}=-1$); this can happen only if $\frac{n}{e}$ is even), then $\beta^{\frac{q-1}{p^e+1}}=(-1)^{\frac{m}{e}}$. This follows from the following computation (let $n=2m=2et$, so $t=\frac{m}{e}$):
\begin{align*}
\beta^{\frac{q-1}{p^e+1}}&=\beta^{(p^e-1)\left(p^{2e(t-1)}+\cdots+p^{2e}+1\right)}=(-1)^{t}=(-1)^{\frac{m}{e}}.
\end{align*}

We first assume that  $\frac{n}{e}$ is odd, $n=2m$, $e=\gcd(n,k)$, and $\alpha\neq -\beta\neq \beta^{p^{n-k}}$.  As before,
the only relevant case is   $-\alpha c\neq  \beta c^{-1}\neq (-\beta c^{-1})^{p^{n-k}}$ and both $L_{\alpha,\beta},L_{-\alpha c,-\beta c^{-1}}$ are permutations, surely on $\bar \sY_1\cap\bar\sY_2$. Then,
\[
\begin{cases}
p^n \eta((\alpha+\beta)(-\alpha c-\beta c^{-1})) \chi_1(\beta(1-c^{-1}))&\text{ if } p\equiv 1\pmod 4\\
p^n (-1)^n \eta(\alpha+\beta)\eta(-\alpha c-\beta c^{-1}) \chi_1(\beta(1-c^{-1}))&\text{ if } p\equiv 3\pmod 4,
\end{cases}
\]
and so,
\[
T_b=p^n (-1)^{n\frac{p-1}{2}} \sum_{(\alpha,\beta)\in\bar \sY_1\cap\bar\sY_2} \chi_1(-b(\alpha+\beta)+\beta(1-c^{-1}))
\eta((\alpha+\beta)(-\alpha c-\beta c^{-1})).
\]

We continue now with $\frac{n}{e}$ being even.
If $\alpha=-\beta=\beta^{p^{n-k}}$, then surely $-\alpha c\neq \beta c^{-1}=(-c^{-1} \beta)^{p^{n-k}}$.  From here on until the end of the subsection, we let $A=\alpha+\beta, A'=-c\alpha-c^{-1}\beta$. Observe that if $\alpha=-\beta=\beta^{p^{n-k}}$, then ${A'}^{\frac{q-1}{p^e+1}}=(1-c^{-1})^{\frac{q-1}{p^e+1}} (-1)^{\frac{m}{e}}=(c-1)^{\frac{q-1}{p^e+1}} (-1)^{\frac{m}{e}}$. We  label in this section $\displaystyle \Sigma_3=\sum_{-\beta=\beta^{p^{n-k}}}  \chi_1\left(\beta(1-c^{-1})\right)$. 
Thus, if $\frac{n}{e}$ is even and $\alpha=-\beta=\beta^{p^{n-k}}$ (and $-\alpha c\neq \beta c^{-1}=(-c^{-1} \beta)^{p^{n-k}}$), then
\[
S_{\alpha,\beta}S_{-\alpha c,\beta c^{-1}} 
=\begin{cases}
p^{n+m}(-1)^{\frac{m}{e}} \chi_1\left(\beta(1-c^{-1})\right) &\text{ if }  {A'}^{\frac{q-1}{p^e+1}}=(-1)^{\frac{m}{e}}\\
p^{n+m+e}(-1)^{\frac{m}{e}+1} \chi_1\left(\beta(1-c^{-1})\right) &\text{ if }  {A'}^{\frac{q-1}{p^e+1}}\neq (-1)^{\frac{m}{e}},
\end{cases}
\] 
and so, if ${A'}^{\frac{q-1}{p^e+1}}=(-1)^{\frac{m}{e}}$, that is $(c-1)^{\frac{q-1}{p^e+1}} =1$,
\allowdisplaybreaks
\begin{align*}
T_{b,1}&=p^{n+m}(-1)^{\frac{m}{e}} \sum_{\substack{0\neq \alpha=-\beta\\
-\beta=\beta^{p^{n-k}}}} \chi_1(-b(\alpha+\beta))  \chi_1\left(\beta(1-c^{-1})\right)  \\
&= p^{n+m}(-1)^{\frac{m}{e}} \sum_{0\neq -\beta=\beta^{p^{n-k}}}  \chi_1\left(\beta(1-c^{-1})\right) =p^{n+m}(-1)^{\frac{m}{e}}  \Sigma_3.
\end{align*}
If ${A'}^{\frac{q-1}{p^e+1}}\neq (-1)^{\frac{m}{e}}$, that is, $(c-1)^{\frac{q-1}{p^e+1}} \neq 1$,  \begin{align*}
T_{b,2}&= p^{n+m+e}(-1)^{\frac{m}{e}+1} \sum_{-\beta\neq \beta^{p^{n-k}}}  \chi_1\left(\beta(1-c^{-1})\right) \\
&=p^{n+m+e}(-1)^{\frac{m}{e}+1}\left(-1-\sum_{0\neq -\beta=\beta^{p^{n-k}}}  \chi_1\left(\beta(1-c^{-1})\right) \right)\\
&=p^{n+m+e}(-1)^{\frac{m}{e}+1}  (-1-\Sigma_3).
\end{align*}

If $\frac{n}{e}$ is even and $\alpha\neq -\beta=\beta^{p^{n-k}}$, there are two subcases: $-\alpha c\neq \beta c^{-1}=(-\beta c^{-1})^{p^{n-k}}$, and $-\alpha c= \beta c^{-1}=(-\beta c^{-1})^{p^{n-k}}$ (this can only happen for $\alpha=-\beta c^{-2}$). Here, for easy writing, we let $\cC,\cC'$ be the set of $(\alpha,\beta)$ satisfying the conditions
$A^{\frac{q-1}{p^e+1}}=(-1)^{\frac{m}{e}}$, respectively, ${A'}^{\frac{q-1}{p^e+1}}=(-1)^{\frac{m}{e}}$.
We write $\cC\triangle\cC'= (\cC\setminus\cC')\cup (\cC'\setminus\cC)$ for the symmetric difference.

In the first subcase ($\alpha\neq -\beta c^{-2}$), we get
\[
S_{\alpha,\beta}S_{-\alpha c,\beta c^{-1}} = 
\begin{cases}
p^{n+2e}  \chi_1(\beta(1-c^{-1})) &\text{ on } \cC\cap \cC'\\
-p^{n+e}  \chi_1(\beta(1-c^{-1})) &\text{ on } \cC\triangle\cC'\\
p^{n}  \chi_1(\beta(1-c^{-1})) &\text{ on } \bar\cC\cap\bar\cC'
\end{cases}
\]
We now look at the second subcase ($\alpha=-\beta c^{-2}$) (note that in this case,   $A^{\frac{q-1}{p^e+1}}=(-1)^{\frac{m}{e}}$ reduces to $(c^2-1)^{\frac{q-1}{p^e+1}}=1$).
Using the fact that $c^{p^e}=c$, then (with $m=te$)
\begin{align*}
(c^2-1)^{\frac{q-1}{p^e+1}}&=(c^2-1)^{(p^e-1)(p^{2e(t-1)}+\cdots+p^{2e}+1}\\
&= \left((c^2-1)^{p^{2e(t-1)}+\cdots+p^{2e}+1}\right)^{p^e-1}= \left(\prod_{i=1}^t (c^2-1)^{p^{2e(t-i)}}\right)^{p^e-1}\\
&= \left(\prod_{i=1}^t (c^{2p^{2e(t-i)}}-1)\right)^{p^e-1}= \left(\prod_{i=1}^t (c^{2}-1)\right)^{p^e-1}\\
&=(c^2-1)^{t(p^e-1)}=\left(\frac{(c^2-1)^{p^e}}{c^2-1} \right)^t=(c^2-1)^{t(p^e-1)}\\
&=\left(\frac{\left(c^{p^e}\right)^2-1}{c^2-1} \right)^t=1,
\end{align*}
and so, $A^{\frac{q-1}{p^e+1}}=(-1)^{\frac{m}{e}}$ holds automatically.

Thus,
\allowdisplaybreaks
\begin{align*}
S_{\alpha,\beta}S_{-\alpha c,\beta c^{-1}} =p^{n}\chi_1\left(-\beta c^{-1}\right)
(-1)^{\frac{m}{e}} p^m \chi_1(\beta) 
= 
(-1)^{\frac{m}{e}} p^{n+m}\chi_1(\beta(1-c^{-1})),
\end{align*}
and so,
\allowdisplaybreaks
\begin{align*}
T_{b,3}&=p^{n+2e} \sum_{\substack{\alpha,\beta\in\cC\cap \cC'\\ \alpha\neq -\beta=\beta^{p^{n-k}}\\ 0\neq \alpha \neq -\beta c^{-2}\neq 0 }}   \chi_1(-b(\alpha+\beta)) \chi_1(\beta(1-c^{-1}))\\
&\quad -p^{n+e}\sum_{\substack{\alpha,\beta\in\cC\triangle \cC'\\ \alpha\neq -\beta=\beta^{p^{n-k}}\\ 0\neq \alpha \neq -\beta c^{-2}\neq 0 }}  \chi_1(-b(\alpha+\beta)) \chi_1(\beta(1-c^{-1}))\\
&\quad+p^n\sum_{\substack{\alpha,\beta\in\bar \cC\cap \bar\cC'\\ \alpha\neq -\beta=\beta^{p^{n-k}}\\ 0\neq \alpha \neq -\beta c^{-2}\neq 0 }}   \chi_1(-b(\alpha+\beta)) \chi_1(\beta(1-c^{-1}))
\\
&\quad +p^{n+m}(-1)^{\frac{m}{e}}  \sum_{\substack{\beta^{p^k-1}=-1\\ 
\alpha=-\beta c^{-2} }} \chi_1(\beta(1-c^{-1}))\chi_1(-b(\beta-\beta c^{-2})).
\end{align*}

Next, when $\alpha\neq -\beta\neq \beta^{p^{n-k}}$, the case $-\alpha c=\beta c^{-1}\neq (-\beta c^{-1})^{p^{n-k}}$ renders $S_{-\alpha c,\beta c^{-1}} =0$. Thus, it is sufficient to assume next, when $\frac{n}{e}$ is even, that $\alpha\neq -\beta\neq \beta^{p^{n-k}}$ and $-\alpha c\neq \beta c^{-1}\neq (-\beta c^{-1})^{p^{n-k}}$.
We first investigate the condition from Equation~\eqref{eq:pp_eq} when $L_{\alpha,\beta}, L_{-\alpha c,\beta c^{-1}}$ are PP, under $c^{p^k-1}=1$, that is, $c^{p^e-1}=1$.
We compute ($n=dt=2et$, since $\frac{n}{e}$ is even), using $\frac{p^n-1}{p^d-1}=p^{2e(t-1)}+\cdots+p^{2e}+1$,
\begin{align*}
&\left(\frac{(\beta c^{-1})^{p^{n-k}} +\beta c^{-1}}{\alpha c+\beta c^{-1}}  \right)^{\frac{p^n-1}{p^d-1}}
=\left(\frac{ c^{-p^{n-k}} \beta^{p^{n-k}} +\beta c^{-1}}{\alpha c+\beta c^{-1}}  \right)^{\frac{p^n-1}{p^d-1}}\\
&=\left(\frac{ c^{1-p^{n-k}} \beta^{p^{n-k}} +\beta }{\alpha c^2+\beta }  \right)^{\frac{p^n-1}{p^d-1}}
= \left(\frac{ \beta^{p^{n-k}} +\beta }{\alpha c^2+\beta }  \right)^{\frac{p^n-1}{p^d-1}},\text{ since } p^e-1\,|\,p^{n-k}-1.
%&=   \frac{ \left(\beta^{p^{n-k}} +\beta\right)^{\frac{p^n-1}{p^d-1}}}{(\alpha c^2+\beta)^{p^{2e(t-1)}+\cdots+p^{2e}+1}}=\frac{ \left(\beta^{p^{n-k}} +\beta\right)^{\frac{p^n-1}{p^d-1}}}{\prod_{i=1}^t (\alpha c^2+\beta)^{p^{2e(t-i)}}}\\
%&=\frac{ \left(\beta^{p^{n-k}} +\beta\right)^{\frac{p^n-1}{p^d-1}}}{\prod_{i=1}^t \left(\alpha^{p^{2e(t-i)}} (c^2)^{p^{2e(t-i)}}+\beta^{p^{2e(t-i)}}\right)}\\
\end{align*}
Summarizing, $L_{\alpha,\beta},L_{-\alpha c,\beta c^{-1}}$ are not PP if and only if  $\left(\frac{ \beta^{p^{n-k}} +\beta }{\alpha +\beta }  \right)^{\frac{p^n-1}{p^d-1}}=(-1)^{\frac{n}{d}}$ (condition $(L_1)$), respectively,  $\left(\frac{ \beta^{p^{n-k}} +\beta }{\alpha c^2+\beta }  \right)^{\frac{p^n-1}{p^d-1}}=(-1)^{\frac{n}{d}}$ (condition $(L_2)$). 
Surely, $L_{\alpha,\beta}, L_{-\alpha c,\beta c^{-1}}$ are not PP if and only if $\left(\frac{\alpha c^2+\beta}{\alpha+\beta} \right)^{\frac{p^n-1}{p^d-1}}=1$.

 We are now ready to find the relevant products. Given the prior definition of the sets $\sA_i,\bar \sY_i$, $i=1,2$, we modify them to impose also $\alpha\neq \beta c^{-2}$, and write them as $ \sA_i', \tilde \sY_i'$, $i=1,2$.
 First, 
\[
S_{\alpha,\beta}S_{-\alpha c,\beta c^{-1}}=p^{n+2e} \chi_1\left(\beta+(\alpha+\beta) x_{\alpha,\beta}^{p^k+1}\right) \chi_1\left(-\beta c^{-1}-(\alpha c+\beta c^{-1}) x_{-\alpha c,-\beta c^{-1}}^{p^k+1}\right),
\]
if neither $L_{\alpha,\beta}, L_{-\alpha c,\beta c^{-1}}$ is  PP (thus, $(\alpha,\beta)\in\sA_1'\cap \sA_2')$. Secondly,
\[
S_{\alpha,\beta}S_{-\alpha c,\beta c^{-1}}=p^{n+e} \chi_1\left(\beta+(\alpha+\beta) x_{\alpha,\beta}^{p^k+1}\right) \chi_1\left(-\beta c^{-1}-(\alpha c+\beta c^{-1}) x_{-\alpha c,-\beta c^{-1}}^{p^k+1}\right),
\]
if exactly one of $L_{\alpha,\beta}, L_{-\alpha c,\beta c^{-1}}$ is   PP (thus, $(\alpha,\beta)\in(\sA_1'\cap \tilde\sY_2)\cup (\sA_2'\cap\tilde \sY_1)$). Lastly, 
\[
S_{\alpha,\beta}S_{-\alpha c,\beta c^{-1}}=p^{n} \chi_1\left(\beta+(\alpha+\beta) x_{\alpha,\beta}^{p^k+1}\right) \chi_1\left(-\beta c^{-1}-(\alpha c+\beta c^{-1}) x_{-\alpha c,-\beta c^{-1}}^{p^k+1}\right),
\]
if both $L_{\alpha,\beta}, L_{-\alpha c,\beta c^{-1}}$ are   PP (thus, $(\alpha,\beta)\in\tilde\sY_1\cap \tilde \sA_2$). 


With the notation  
\allowdisplaybreaks
\begin{align*}
\Sigma(L)&=\sum_{(\alpha,\beta)\in L} \chi_1\left(\beta+(\alpha+\beta) \left(x_{\alpha,\beta}^{p^k+1}-b\right)\right)  \\
&\qquad\qquad\qquad\qquad\quad \cdot \chi_1\left(-\beta c^{-1}-(\alpha c+\beta c^{-1}) x_{-\alpha c,-\beta c^{-1}}^{p^k+1}\right),
\end{align*}
we obtain
\allowdisplaybreaks
\begin{align*}
T_{b,4}&=p^{n+2e}  \Sigma\left(\sA_1'\cap \sA_2'\right)+p^{n+e}\Sigma\left((\sA_1'\cap \tilde\sY_2)\cup (\sA_2'\cap\tilde \sY_1)\right)+p^n\Sigma\left(\tilde\sY_1\cap \tilde \sA_2\right).
\end{align*}
Therefore, when $\frac{n}{e}$ is even, then 
\[
T_b=T_{b,1}+T_{b,2}+T_{b,3}+T_{b,4}.
\]


We now record what we have shown in the next theorem. We recall that $\displaystyle \Sigma_3=\sum_{-\beta=\beta^{p^{n-k}}}  \chi_1\left(\beta(1-c^{-1})\right)$. 
Further, $\cC,\cC'$ is the set of $(\alpha,\beta)$ satisfying the conditions
$A^{\frac{q-1}{p^e+1}}=(-1)^{\frac{m}{e}}$, respectively, ${A'}^{\frac{q-1}{p^e+1}}=(-1)^{\frac{m}{e}}$, where $A=\alpha+\beta, A'=-\alpha c-\beta c^{-1}$.

\begin{thm}
Let $F(x)=x^{p^k+1}$, $1\leq k<n$, be the Gold function on $\F_{p^n}$, $p$ and odd prime,  $n\geq 2$, and $c^{p^k-1}=1,c\neq \pm 1$. The $(-1)$-Boomerang Connectivity Table entry of $F$ at $(a,ab$) is 
$\displaystyle _c\cB_F(a,ab)=\frac{1}{q}\left(\Delta_{F,c}(1,b)+q\Delta_{F,c^{-1}}(1,b)\right)-1 +\frac1{q^2}T_b$, where:
\begin{itemize}
\item[$(i)$] If $\frac{n}{e}$ is odd, then 
{\small
\begin{align*}
T_b&=p^n (-1)^{n\frac{p-1}{2}} \sum_{(\alpha,\beta)\in\bar \sY_1\cap\bar\sY_2} \chi_1(-b(\alpha+\beta)+\beta(1-c^{-1}))
\eta((\alpha+\beta)(-\alpha c-\beta c^{-1})).
\end{align*}
}
\item[$(ii)$]  If $\frac{n}{e}$ is even $(n=2m)$,
\allowdisplaybreaks
\begin{align*}
T_b&=p^{n+m}(-1)^{\frac{m}{e}}  \Sigma_3+p^{n+m+e}(-1)^{\frac{m}{e}+1}  (-1-\Sigma_3)\\
&\quad + p^{n+2e} \sum_{\substack{\alpha,\beta\in\cC\cap \cC'\\ \alpha\neq -\beta=\beta^{p^{n-k}}\\ 0\neq \alpha \neq -\beta c^{-2}\neq 0 }}   \chi_1(-b(\alpha+\beta)) \chi_1(\beta(1-c^{-1}))\\
&\quad -p^{n+e}\sum_{\substack{\alpha,\beta\in\cC\triangle \cC'\\ \alpha\neq -\beta=\beta^{p^{n-k}}\\ 0\neq \alpha \neq -\beta c^{-2}\neq 0 }}  \chi_1(-b(\alpha+\beta)) \chi_1(\beta(1-c^{-1}))\\
&\quad+p^n\sum_{\substack{\alpha,\beta\in\bar \cC\cap \bar\cC'\\ \alpha\neq -\beta=\beta^{p^{n-k}}\\ 0\neq \alpha \neq -\beta c^{-2}\neq 0 }}   \chi_1(-b(\alpha+\beta)) \chi_1(\beta(1-c^{-1}))
\\
&\quad +p^{n+m}(-1)^{\frac{m}{e}}  \sum_{\substack{\beta^{p^k-1}=-1\\ 
\alpha=-\beta c^{-2} }} \chi_1\left(\beta(1-c^{-1})(1-b(1+c^{-1})\right)\\
&+p^{n+2e}  \Sigma\left(\sA_1'\cap \sA_2'\right)+p^{n+e}\Sigma\left((\sA_1'\cap \tilde\sY_2)\cup (\sA_2'\cap\tilde \sY_1)\right)+p^n\Sigma\left(\tilde\sY_1\cap \tilde \sA_2\right).
\end{align*}
\end{itemize}
\end{thm}

\subsection{The general case} 
 
  We can surely find an expression for the $c$-BCT for $c^{p^k-1}\neq 1$, but it is going to be slightly complicated, although, as we mentioned, computing the boomerang uniformity is a difficult endeavor.
  
  As in the previous results, for $c\in\F_{p^n}$, $c^{p^k-1}\neq 0$, the $c$-Boomerang Connectivity Table entry of $F(x)=x^{p^k+1}$ at $(a,ab$) is 
\[
\displaystyle _c\cB_F(a,ab)=\frac{1}{q}\left(\Delta_{F,c}(1,b)+q\Delta_{F,c^{-1}}(1,b)\right)-1 +\frac1{q^2}T_b,
\]
 where 
$\displaystyle T_b=\sum_{\alpha,\beta\in\F_q,\alpha\beta\neq 0} \chi_1(-b(\alpha+\beta))\, S_{\alpha,\beta}\, S_{-\alpha c,-\beta c^{-1}}$
  
When $\frac{n}{e}$ is odd, then $T_b$ is the same as for the case of $c^{p^k-1}=1$, $c\neq \pm 1$, namely,
{\small
\begin{align*}
T_b&=p^n (-1)^{n\frac{p-1}{2}} \sum_{(\alpha,\beta)\in\bar \sY_1\cap\bar\sY_2} \chi_1(-b(\alpha+\beta)+\beta(1-c^{-1}))
\eta((\alpha+\beta)(-\alpha c-\beta c^{-1})).
\end{align*}
}

 For even $\frac{n}{e}$, we will not write the $T_b$ expressions, rather we will find just the products $S_{\alpha,\beta} S_{-\alpha c,-\beta c^{-1}}$. 
When $\frac{n}{e}$ is even, and $\alpha=-\beta=\beta^{p^{n-k}}$, then either $-\alpha c=\beta c^{-1}=(-\beta c^{-1})^{p^{n-k}}$ (in which case $S_{-\alpha c,-\beta c^{-1}}=0$), or $-\alpha c\neq \beta c^{-1}\neq (-\beta c^{-1})^{p^{n-k}}$, in which case $S_{\alpha,\beta} S_{-\alpha c,-\beta c^{-1}}$ equals
{\small
\[
\begin{cases}
p^{n+m+e} (-1)^{\frac{m}{e}+1}\chi_1\left(\beta(1- c^{-1})+(\alpha c+\beta c^{-1})x_{-\alpha c, -\beta c^{-1}}^{p^{k+1}}\right), & (\alpha,\beta)\in\sA_2\\
p^{n+m}(-1)^{\frac{m}{e}} \chi_1\left(\beta(1- c^{-1})+(\alpha c+\beta c^{-1})x_{-\alpha c, -\beta c^{-1}}^{p^{k+1}}\right), & (\alpha,\beta)\in\bar \sY_2.
\end{cases}
\]
}
Similarly, when $\frac{n}{e}$ is even, and $-\alpha c= \beta c^{-1}= (-\beta c^{-1})^{p^{n-k}}$,  $\alpha\neq -\beta\neq \beta^{p^{n-k}}$, and   $S_{\alpha,\beta} S_{-\alpha c,-\beta c^{-1}}$ equals
{\small
\[
\begin{cases}
p^{n+m+e} (-1)^{\frac{m}{e}+1} \chi_1\left(\beta(1- c^{-1})-(\alpha +\beta )x_{\alpha , \beta}^{p^{k+1}}\right), & (\alpha,\beta)\in\sA_1\\
p^{n+m}(-1)^{\frac{m}{e}} \chi_1\left(\beta(1- c^{-1})-(\alpha +\beta )x_{\alpha , \beta}^{p^{k+1}}\right), & (\alpha,\beta)\in\bar \sY_1.
\end{cases}
\]
}
If $\frac{n}{e}$ is even,  $\alpha\neq -\beta=\beta^{p^{n-k}}$ and  $-\alpha c\neq \beta c^{-1}\neq (-\beta c^{-1})^{p^{n-k}}$ (we let here and below $\cC_1,\cC_2$ be the sets of $(\alpha,\beta)$ such that $A^{\frac{q-1}{p^e+1}}=(-1)^{\frac{m}{e}}$, respectively, ${A'}^{\frac{q-1}{p^e+1}}=(-1)^{\frac{m}{e}}$), then $S_{\alpha,\beta} S_{-\alpha c,-\beta c^{-1}}$  equals
{\small
\[
\begin{cases}
-p^{n+e}   \chi_1\left(\beta(1- c^{-1})+(\alpha c+\beta c^{-1})x_{-\alpha c, -\beta c^{-1}}^{p^{k+1}}\right), & (\alpha,\beta)\in(\bar\cC_1\cap \sA_2)\cup(\cC_1\cap\bar \sY_2)\\
p^{n}   \chi_1\left(\beta(1- c^{-1})+(\alpha c+\beta c^{-1})x_{-\alpha c, -\beta c^{-1}}^{p^{k+1}}\right), & (\alpha,\beta)\in\cC_1\cap \bar\sY_2\\
p^{n+e}   \chi_1\left(\beta(1- c^{-1})+(\alpha c+\beta c^{-1})x_{-\alpha c, -\beta c^{-1}}^{p^{k+1}}\right), & (\alpha,\beta)\in\cC_1\cap \sA_2.
\end{cases} 
\]
}
Similarly, when $\frac{n}{e}$ is even, $\alpha\neq -\beta\neq \beta^{p^{n-k}}$ and  $-\alpha c\neq \beta c^{-1}= (-\beta c^{-1})^{p^{n-k}}$, then $S_{\alpha,\beta} S_{-\alpha c,-\beta c^{-1}}$  equals
{\small
\[
\begin{cases}
-p^{n+e}   \chi_1\left(\beta(1- c^{-1})-(\alpha +\beta)x_{\alpha, \beta}^{p^{k+1}}\right), & (\alpha,\beta)\in(\bar\cC_2\cap \sA_1)\cup(\cC_2\cap\bar \sY_1)\\
p^{n}   \chi_1\left(\beta(1- c^{-1})-(\alpha +\beta)x_{\alpha, \beta}^{p^{k+1}}\right), & (\alpha,\beta)\in\cC_2\cap \bar\sY_1\\
p^{n+e}  \chi_1\left(\beta(1- c^{-1})-(\alpha +\beta)x_{\alpha, \beta}^{p^{k+1}}\right), & (\alpha,\beta)\in\cC_2\cap \sA_1.
\end{cases} 
\]
}
Finally, if $\frac{n}{e}$ is even, $\alpha\neq -\beta\neq \beta^{p^{n-k}}$ and  $-\alpha c\neq \beta c^{-1}\neq  (-\beta c^{-1})^{p^{n-k}}$, then $S_{\alpha,\beta} S_{-\alpha c,-\beta c^{-1}}$  equals
 {\footnotesize
\[
\begin{cases}
-p^{n+e}   \chi_1\left(\beta(1- c^{-1})-(\alpha +\beta)\left(x_{\alpha, \beta}^{p^{k+1}}-x_{-\alpha c, -\beta c^{-1}}^{p^{k+1}}\right)\right), & (\alpha,\beta)\in (\sA_1\cap\bar \sY_2)\cup(\bar \sY_1\cap \sA_2)\\
p^{n+2e}  \chi_1\left(\beta(1- c^{-1})-(\alpha +\beta)\left(x_{\alpha, \beta}^{p^{k+1}}-x_{-\alpha c, -\beta c^{-1}}^{p^{k+1}}\right)\right), & (\alpha,\beta)\in\sA_1\cap \sA_2\\
p^{n} \chi_1\left(\beta(1- c^{-1})-(\alpha +\beta)\left(x_{\alpha, \beta}^{p^{k+1}}-x_{-\alpha c, -\beta c^{-1}}^{p^{k+1}}\right)\right), & (\alpha,\beta)\in\bar\sY_1\cap\bar \sY_2.
\end{cases} 
\]
} 
 
 \section{Concluding Remarks}
 \label{sec5}
 
It would be interesting to see what the entries of the $c$-BCT are for other functions of interest, like the known PcN  or APcN (for all $c\neq 0$). We hope to see other applications and refinements of our methods, as well as continued progress in computing the $c$-differential and $c$-boomerang uniformity for other classes of functions.
 
 
 \begin{thebibliography}{99}
 
%\bibitem{Berger06}
%T. Berger, A. Canteaut, P. Charpin, Y. Laigle-Chapuy, {\em On almost perfect nonlinear functions} IEEE Trans. Inf. Theory 52:9 (2006), 4160--4170.

%\bibitem{BRS67}
%E. R. Berlekamp, H. Rumsey, G. Solomon,
%{\em On the solutions of algebraic equations over finite fields}, Information and Control 10 (1967), 553--564.

\bibitem{BDK02}
E. Biham, O. Dunkelman, N. Keller, {\em New results on boomerang and rectangle attacks}, In: Daemen J., Rijmen V. (eds.), Fast Software Encryption, FSE 2002, LNCS 2365, 2002, pp. 1--16, Springer, Berlin, Heidelberg.

\bibitem{BK99}
A. Biryukov, D. Khovratovich, {\em Related-key cryptanalysis of the full AES-$192$ and AES-$256$},  In: Matsui M. (ed.),  Adv. in Crypt. -- ASIACRYPT 2009,  LNCS 5912, 2009, pp. 1--18, Springer, Berlin, Heidelberg.
 
%\bibitem{Bluher04}
%A. W. Bluher, {\em On $x^{q+1} + ax + b$}, Finite Fields Appl. 10 (3) (2004), 285--305.

%\bibitem{BCJW02} N. Borisov, M. Chew, R. Johnson, D. Wagner, {\em Multiplicative Differentials}, In: Daemen J., Rijmen V. (eds.), Fast Software Encryption, FSE 2002, LNCS 2365, pp. 17--33, Springer, Berlin, Heidelberg, 2002.

\bibitem{BC18}
C. Boura, A. Canteaut, {\em On the boomerang uniformity of cryptographic Sboxes}, IACR Trans. Symmetric Cryptol. 3 (2018),  290--310.

\bibitem{BPT19}
C. Boura, L. Perrin, S. Tian, {\em Boomerang Uniformity of Popular S-box
Constructions},  Workshop on Cryptography and Coding 2019, Paper 15,
\url{https://www.lebesgue.fr/sites/default/files/proceedings\_WCC/WCC\_2019\_paper\_15.pdf.}

%\bibitem{Bo90}
%N. Bourbaki, Elements of Mathematics, Algebra II (translated by P. M. Cohn and J. Howie), Springer, Berlin, 1990.

\bibitem{Bud14}
L. Budaghyan, Construction and Analysis of Cryptographic Functions, Springer-Verlag, 2014.

%\bibitem{BC08}
%L. Budaghyan, C. Carlet, {\em Classes of quadratic APN trinomials and hexanomials and related structures}, IEEE Trans. Inform.Theory 54:5 (2008), 2354--2357.

\bibitem{CV19}
M. Calderini, I.Villa,
{\em On the Boomerang Uniformity of some Permutation Polynomials}, 
\url{https://eprint.iacr.org/2019/881.pdf.}

 \bibitem{CH1} C.~Carlet, {\em Boolean functions for cryptography and error correcting codes}, In: Y. Crama, P. Hammer  (eds.), Boolean Methods and Models,
Cambridge Univ. Press, Cambridge, pp. 257--397, 2010.

\bibitem{CH2}
C. Carlet, {\em Vectorial Boolean Functions for Cryptography},
In: Y. Crama, P. Hammer  (eds.), Boolean Methods and Models,
Cambridge Univ. Press, Cambridge, pp. 398--472, 2010.

%\bibitem{Car18}
%C. Carlet, {\em Characterizations of the Differential Uniformity of
%Vectorial Functions by the Walsh Transform}, IEEE Trans. Inf. Theory 64:9 (2018), 6443--6453.

%\bibitem{Carlet19}
%C. Carlet, {\em  On APN exponents, characterizations of differentially uniform functions by the Walsh transform, and related cyclic-difference-set-like structures}, Des. Codes Cryptogr. 87:2-3 (2019), 203--224. 

%\bibitem{CV95}
%F. Chabaud, S. Vaudenay, {\em Links between differential and linear cryptanalysis}, In: A. De Santis (ed.), Adv. in Crypt -- EUROCRYPT '94, LNCS 950, pp. 356--365, 1995, Springer.

\bibitem{Cid18}
C. Cid, T. Huang, T. Peyrin, Y. Sasaki, L. Song, {\em Boomerang Connectivity Table:
A new cryptanalysis tool}, In: J. B. Nielsen and V. Rijmen (eds.), 
Adv. in Crypt. -- EUROCRYPT '18, pp. 683--714, Cham, 2018. Springer.

\bibitem{Co98_1}
R. S. Coulter, {\em Explicit evaluations of some Weil sums}, Acta Arithmetica 83 (1998), 241--251.

\bibitem{Co98}
R. S. Coulter, {\em Further evaluations of Weil sums}, Acta Arithmetica 86 (1998), 217--226.

%\bibitem{CGM88} W. Chou, J. Gomez-Calderon, G. L.Mullen, {\em Value sets of Dickson polynomials over finite fields}, J. Number Theory 30:3 (1988),  334--344.
 
% \bibitem{CM04}
%R. S. Coulter, M. Henderson, {\em A note on the roots of trinomials over a finite field}, Bull. Austral. Math. Soc. 69 (2004), 429--432.
 
% \bibitem{CS97}
%R. S. Coulter, R. W. Matthews,
%{\em Planar functions and planes of Lenz-Barlotti class} II, Des. Codes Cryptogr. 10 (1997), 167--184.

\bibitem{CS17} T. W.~Cusick, P.~St\u anic\u a,
{Cryptographic Boolean Functions and Applications} (Ed. 2), Academic Press, San Diego, CA,  2017.
 
% \bibitem{DY06}
%C. Ding, J. Yuan, {\em A new family of skew Paley-Hadamard difference sets}, J.
%Comb. Theory Ser. A 113 (2006), 1526--1535.

%\bibitem{Dob06}
%H. Dobbertin, P. Felke, T. Helleseth, P. Rosendahl, {\em Niho type cross-correlation functions via Dickson polynomials and Kloosterman sums}, IEEE Trans. Inf. Theory 52 (2) (2006) 613--627.

%\bibitem{Dob03}
%H. Dobbertin, D. Mills, E. N. Muller, A. Pott, and W. Willems, {\em APN
%functions in odd characteristic}, Discr. Math. 267 (1-3) (2003), 95--112.

\bibitem{EFRST20}
P. Ellingsen, P. Felke, C. Riera P. St\u anic\u a, A. Tkachenko,
{\em $C$-differentials, multiplicative uniformity and (almost) perfect $c$-nonlinearity}, to appear in IEEE Trans. Inf. Theory, 2020, \url{https://doi.org/10.1109/TIT.2020.2971988}.

\bibitem{HMRS20}
S. U. Hasan, M. Pal, C. Riera, P. St\u anic\u a,
{\em On the $c$-differential uniformity of certain maps over finite fields}, 
\url{https://arxiv.org/abs/2004.09436}.

%\bibitem{HK08}
%T. Helleseth, A. Kholosha, {\em On the equation $x^{2^\ell+1}+x+a=0$ over $GF(2^k)$}, Finite Fields Appl. 14 (2008), 159--176.

%\bibitem{HRS99}
%T. Helleseth, C. Rong,  D. Sandberg, {\em New families of almost perfect
%nonlinear power mappings}, IEEE Trans. Inf. Theory 45 (1999), 475--485.

%\bibitem{HS97}
%T. Helleseth, D. Sandberg, {\em Some power mappings with low
%differential uniformity}, Appl. Algebra Eng. Commun. Comput. 8 (1997),
%363--370.

\bibitem{KKS00}
J. Kelsey, T. Kohno and B. Schneier, {\em Amplified boomerang attacks against reduced-round MARS and Serpent},  In: Goos G., Hartmanis J., van Leeuwen J., Schneier B. (eds,), 
Fast Software Encryption, FSE 2000. LNCS 1978. Springer, Berlin, Heidelberg.

\bibitem{Kim12}
J. Kim, S. Hong, B. Preneel, E. Biham, O. Dunkelman,
{\em Related-Key Boomerang and Rectangle Attacks: Theory and Experimental Analysis}, IEEE Trans. Inf. Theory 58(7) (2012),  4948--4966.

\bibitem{Li19}
K. Li, L. Qu, B. Sun, C. Li, {\em New results about the boomerang uniformity of permutation polynomials}, IEEE Trans. Inf. Theory 65(11) (2019), 7542--7553.

\bibitem{LiHu20}
N. Li, Z. Hu, M. Xiong, X. Zeng, {\em $4$-uniform BCT permutations from generalized
butterfly structure}, \url{https://arxiv.org/abs/2001.00464}.

%\bibitem{Li13}
%Y. Li, M. Wang,  Y. Yu, {\em Constructing Differentially $4$-Uniform
%Permutations Over $GF(2^{2k})$ From the Inverse Function Revisited},
%\url{https://eprint.iacr.org/2013/731.}

%\bibitem{Li78}
%J. Liang, {\em On the solutions of trinomial equations over finite fields},
%Bull. Cal. Math. Soc. 70 (1978), 379--382.

\bibitem{LN97}
R. Lidl, H. Niederreiter, FiniteFields (Ed. 2), Encycl. Math. Appl., vol.20, Cambridge Univ. Press, Cambridge, 1997.

\bibitem{MesnagerBook} S. Mesnager, { Bent functions: fundamentals and results}, Springer Verlag, 2016.

\bibitem{Mes19}
S. Mesnager, C. Tang, M. Xiong, {\em On the boomerang uniformity of quadratic permutations}, \url{https://eprint.iacr.org/2019/277.pdf}.

%\bibitem{No68}
%W. N\"obauer, {\em \"Uber eine Klasse von Permutationspolynomen und die dadurch dargestellten Gruppen}, J. Reine Angew. Math. 231 (1968), 215--219.

%\bibitem{PT17}
%J. Peng, C. H. Tan, {\em New differentially $4$-uniform permutations by
%modifying the inverse function on subfields}, Cryptogr. Commun.  9 (2017), 363--378.

%\bibitem{QT13}
% L. Qu, Y. Tan, C. H. Tan,  C. Li, {\em Constructing differentially
%$4$-uniform permutations over $\F_{2^{2k}}$  via the switching method}, IEEE
%Trans. Inf. Theory 59:4 (2013),  4675--4686.
 
%\bibitem{QT16}
%  L. Qu, Y. Tan, C. Li, and G. Gong, {\em More constructions of differentially
%$4$-uniform permutations on $\F_{2^{2k}}$}, Des., Codes Cryptogr. 78 (2016), 391--408.

\bibitem{RS20}
C. Riera, P. St\u anic\u a,
 {\em Investigations on $c$-(almost) perfect nonlinear functions}, 
 \url{https://arxiv.org/abs/2004.02245}.
 
  \bibitem{S20}
 P. St\u anic\u a, {\em Investigations on $c$-boomerang uniformity and perfect nonlinearity}, \url{https://arxiv.org/abs/2004.11859}, 2020.
 
 \bibitem{SG20}
 P. St\u anic\u a, A. Geary,
 {\em The $c$-differential behavior of the inverse function under the $EA$-equivalence},
 \url{https://arxiv.org/abs/2006.00355}.
 
% \bibitem{TCT14}
% D. Tang, C. Carlet, X.  Tang, {\em Differentially $4$-uniform bijections by permuting the inverse function}, Des. Codes. Cryptogr. 77 (2014), 117--141.
 
\bibitem{Tok15} N. Tokareva, { Bent Functions, Results and Applications to Cryptography}, Academic Press, San Diego, CA,  2015.

\bibitem{TX20}
Z. Tu, N. Li, X. Zeng, J. Zhou, {\em A class of quadrinomial permutation with boomerang uniformity four},  IEEE Trans. Inf. Theory, 
\url{https://doi.org/10.1109/TIT.2020.2969578}.

%\bibitem{Yu13}
% Y. Yu, M. Wang, Y. Li, {\em Constructing differentially $4$ uniform permutations from known ones},  Chin. J. Electron. 22(3) (2013), 495--499.

\bibitem{YZ20}
H. Yan, S. Mesnager, Z. Zhou,
{\em Power Functions over Finite Fields with Low $c$-Differential Uniformity}, \url{https://arxiv.org/pdf/2003.13019.pdf}.

%\bibitem{Wu13}
%B. Wu, {\em The compositional inverses of linearized permutation binomials over finite fields}, arxiv.org, \url{https://arxiv.org/pdf/1311.2154.pdf}.
 
\bibitem{Wag99}
D. Wagner, {\em The boomerang attack}, In: L. R. Knudsen (ed.), Fast Software Encryption, FSE '99, LNCS 1636, pages 156--170, 1999, Springer, Heidelberg.
 
%\bibitem{Za14}
%Z. Zha, L. Hu, S. Sun, {\em Constructing new differentially $4$-uniform permutations from the inverse function}, Finite Fields Appl. 25 (2014), 64--78.

\bibitem{ZWW20}
Y. Zheng, Q. Wang, W.  Wei,
{\em On Inverses of Permutation Polynomials of Small Degree Over Finite Fields}, IEEE Trans. Inf. Theory 66:2 (2020),  914--922.

\end{thebibliography}


\end{document}
