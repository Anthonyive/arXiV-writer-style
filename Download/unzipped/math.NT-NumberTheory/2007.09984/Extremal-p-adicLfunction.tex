
%\documentclass[a4paper,twoside,11pt]{book}
%\usepackage{amsmath,amssymb,amsthm}
%\usepackage[]{latexsym}
%\usepackage[catalan]{babel}
%\usepackage[latin1]{inputenc}
\documentclass{amsart}
\usepackage[latin1]{inputenc}
%\usepackage[catalan]{babel}
\usepackage{amssymb}
\usepackage{latexsym}
\usepackage{amscd}
\usepackage{longtable}
\usepackage{amsmath}
\usepackage{mathrsfs}
\usepackage{array}
\usepackage[all]{xy}
\usepackage{a4}



\newtheorem{defn0}{Definition}[section]
\newtheorem{prop0}[defn0]{Proposition}
\newtheorem{thm0}[defn0]{Theorem}
\newtheorem{lemma0}[defn0]{Lemma}
\newtheorem{claim0}[defn0]{Claim}
\newtheorem{corollary0}[defn0]{Corollary}
\newtheorem{example0}[defn0]{Example}
\newtheorem{remark0}[defn0]{Remark}
\newtheorem{assumption0}[defn0]{Assumption}
\newtheorem{conjecture0}[defn0]{Conjecture}
\newtheorem{notation0}[defn0]{Notation}
\newtheorem{question0}[defn0]{Question}

\renewcommand{\theequation}{\thesection.\arabic{equation}}

\newenvironment{definition}{\begin{defn0}\rm}{\end{defn0}}
\newenvironment{proposition}{\begin{prop0}}{\end{prop0}}
\newenvironment{theorem}{\begin{thm0}}{\end{thm0}}
\newenvironment{lemma}{\begin{lemma0}}{\end{lemma0}}
\newenvironment{claim}{\begin{claim0}}{\end{claim0}}
\newenvironment{corollary}{\begin{corollary0}}{\end{corollary0}}
\newenvironment{example}{\begin{example0}\rm}{\end{example0}}
\newenvironment{remark}{\begin{remark0}\rm}{\end{remark0}}
\newenvironment{assumption}{\begin{assumption0}\rm}{\end{assumption0}}
\newenvironment{conjecture}{\begin{conjecture0}}{\end{conjecture0}}
\newenvironment{notation}{\begin{notation0}}{\end{notation0}}
\newenvironment{question}{\begin{question0}\rm}{\end{question0}}





\newcommand{\Gal}{{\mathrm {Gal}}}
\newcommand{\Fr}{{\mathrm {Fr}}}
\newcommand{\disc}{{\mathrm {disc }}}
\newcommand{\ord}{\mathrm{ord}}
\newcommand{\Pic}{\mathrm{Pic}}
\newcommand{\Norm}{\mathrm{N}}
\newcommand{\Trace}{{\mathrm{Tr}}}
\newcommand{\M}{\mathrm{M}}
\newcommand{\Aut}{\mathrm{Aut}}
\newcommand{\Jac}{\mathrm{Jac}}
\newcommand{\Br}{\mathrm{Br}}
\newcommand{\sign}{\mathrm{sign}}
\newcommand{\Ind}{{\mathrm{Ind}}}
\newcommand{\PGL}{{\mathrm{PGL}}}
\newcommand{\Frob}{\mathrm{Frob}}
\newcommand{\GL}{{\mathrm{GL}}}
\newcommand{\End}{{\mathrm{End}}}
\newcommand{\Supp}{{\mathrm{Supp}}}
\newcommand{\PSL}{{\mathrm {PSL}}}
\newcommand{\SL}{{\mathrm {SL}}}
\newcommand{\Irr}{{\mathrm {Irr}}}
\newcommand{\Alb}{{\mathrm {Alb}}}
\newcommand{\Img}{{\mathrm {Im}}}

\newcommand{\Z}{{\mathbb Z}}
\newcommand{\A}{{\mathbb A}}
\newcommand{\I}{{\mathbb I}}
\newcommand{\Ha}{{\mathbb H}}
\newcommand{\Q}{{\mathbb Q}}
\newcommand{\C}{{\mathbb C}}
\newcommand{\R}{{\mathbb R}}
\newcommand{\F}{{\mathbb F}}
\newcommand{\N}{{\mathbb N}}
\newcommand{\G}{{\mathbb G}}
\newcommand{\T}{{\mathbb T}}
\newcommand{\PP}{{\mathbb P}}


\newcommand{\cA}{{\mathcal A}}
\newcommand{\cQ}{{\mathcal Q}}
\newcommand{\cR}{{\mathcal R}}
\newcommand{\cN}{{\mathcal N}}
\newcommand{\cS}{{\mathcal S}}
\newcommand{\cD}{{\mathcal D}}
\newcommand{\cC}{{\mathcal C}}
\newcommand{\cF}{{\mathcal F}}
\newcommand{\cB}{{\mathcal B}}
\newcommand{\cX}{{\mathcal X}}
\newcommand{\cH}{{\mathcal H}}
\newcommand{\cK}{{\mathcal K}}
\newcommand{\cG}{{\mathcal G}}
\newcommand{\cGL}{{\mathcal GL}}
\newcommand{\cV}{{\mathcal V}}
\newcommand{\cT}{{\mathcal T}}
\newcommand{\cU}{{\mathcal U}}
\newcommand{\cE}{{\mathcal E}}
\newcommand{\cI}{{\mathcal I}}
\newcommand{\cL}{{\mathcal L}}
\newcommand{\cM}{{\mathcal M}}
\newcommand{\cJ}{{\mathcal J}}
\newcommand{\cP}{{\mathcal P}}
\newcommand{\cO}{{\mathcal O}}
\newcommand{\cW}{{\mathcal W}}
\newcommand{\co}{{\mathfrak o}}
\newcommand{\dP}{{\mathfrak P}}
\newcommand{\dH}{{\mathfrak H}}
\newcommand{\dc}{{\mathfrak c}}
\newcommand{\tendsto}[1]{%
  \xrightarrow{\smash{\raisebox{-2ex}{$\scriptstyle#1$}}}}

\newcommand{\gagu}{{\gamma^{\overline{\sigma}}}}
\newcommand{\ts}{{t_{\sigma}}}
\newcommand{\bgu}{{b_{\overline{\sigma}}}}
\newcommand{\rN}{{\mathrm N}}
\newcommand{\SO}{{\mathrm SO}}
\newcommand{\Div}{{\mathrm {Div}}}
\newcommand{\cInd}{\mbox{c-Ind}}
\newcommand{\Res}{{\mathrm {Res}}}
\newcommand{\Sym}{{\mathrm {Sym}}}
\newcommand{\Spec}{{\mathrm {Spec}}}
\newcommand{\wq}{{\mathfrak{q}}}
\newcommand{\rank}{{\mathrm rank}}
\newcommand{\qbar}{{\bar\Q}}
\newcommand{\kp}{{\overline{k_p}}}
\newcommand{\ra}{{\rightarrow}}
\newcommand{\lra}{\longrightarrow}
\newcommand{\Hom}{{\mathrm {Hom}}}
\newcommand{\Dist}{{\mathrm {Dist}}}
\newcommand{\Meas}{{\mathrm {Meas}}}
\newcommand{\Ker}{{\mathrm {Ker}}}
\newcommand{\Cmin}{{\mathfrak C_{min}/R}}
\newcommand{\Cstr}{{\widetilde{\mathfrak C}}}
\newcommand{\Cst}{{\mathfrak C}}
\newcommand{\Zst}{{\mathfrak Z}}
\newcommand{\Zstr}{{\widetilde{\mathfrak Z}}}
\newcommand{\Cminp}{{\mathfrak C_{min, \wp}}}
\newcommand{\car}{{\mathrm{char}}}
\newcommand{\vol}{{\mathrm{vol}}}
\newcommand{\gd}{\mathfrak d}


\title{Extremal $p$-adic $L$-functions}
\author{Santiago Molina Blanco}


\begin{document}
%%%%%%%%%%%%%%%%%%%%%%%%%%%%%%%%%%%%%%%%%%%%%


\maketitle
%%%%%%%%%%%%%%%%%%%%%%%%%%%%%%%%%%%%%%%%%%%%%
%\vskip 1cm

\begin{abstract}
In this short note we give a reinterpretation of the classical construction of cyclotomic $p$-adic L-functions attached to modular cuspforms. We are able to provide a genuinely new construction under the unlikely hypothesis that the Hecke polynomial has a double root. Although the fulfillment of this hypothesis contradicts Tate's conjecture in this classical setting, we focus our attention on these \emph{extremal $p$-adic L-functions} because that scenario should not be ruled out as long as the conjecture remains open. Moreover, there are examples of Hilbert modular forms where these extremal hypothesis is satisfied and our work will provide new explicit $p$-adic L-functions in the Hilbert case.
We study the admissibility and the interpolation properties of these extremal $p$-adic L-functions. 
\end{abstract}

\tableofcontents

\section{Introduction}

Let $f\in S_{k+2}(\Gamma_1(N),\epsilon)$ be a modular cuspform for $\Gamma_1(N)$ with nebentypus $\epsilon$ and weight $k+2$. A very important topic in modern Number Theory is the study of the L-function $L(s,\pi)$ attached to the automorphic representation $\pi$ of $\GL_2(\A)$ generated by $f$. Understanding this complex valued analytic function is the key point for some of the most important problems in mathematics such as the \emph{Birch and Swinnerton-Dyer conjecture}.% or the \emph{Langlands program}.

Back in the middle of the seventies, Vishik \cite{Vis} and Amice-V\'elu \cite{A-V} defined a $p$-adic measure $\mu_{f,p}$ of $\Z_p^\times$ associated with $f$, under the hypothesis that $p$ does not divide $N$. 
The construction of this measure was the starting point for the theory of $p$-adic L-functions attached to modular cuspforms. 
The $p$-adic L-function $L_p(f,s)$ %attached to $f$ 
is a $\C_p$-valued analytic function related with the classical L-function $L(s,\pi)$ by means of the so-called \emph{interpolation properties}. The function $L_p(f,s)$ is defined by means of $\mu_{f,p}$ as
\[
L_p(f,s):=\int_{\Z_p^\times}{\rm exp}(s\cdot{\rm log(x)})d\mu_{f,p}(x),
\]
where ${\rm exp}$ and ${\rm log}$ are respectively the $p$-adic exponential and $p$-adic logarithm functions.



Mazur, Tate and Teitelbaum extended in \cite{MTT86} the definition of $\mu_{f,p}$ to more general situations and provided a $p$-adic analogue of the Birch and Swinnerton-Dyer conjecture, with the classical L-function $L(\pi,s)$ replaced by $L_p(f,s)$.
It has been shown that $L_p(f,s)$ is directly related with the ($p$-adic, or eventually $l$-adic) cohomology of modular curves, and this makes such $p$-adic Birch and Swinnerton-Dyer conjectures become more tractable. 
In fact, the theory of $p$-adic L-functions has grown tremendously during the last years. Many results, whose complex counterparts are inaccessible with current techniques, have been proven in this $p$-adic setting. 


In this short note we give first a reinterpretation of the construction of the $p$-adic measures $\mu_{f,p}$. Our construction exploits the theory of automorphic representations and, in that sense, it is similar to the construction provided in \cite{Spi14} but for weights greater that 2. This opens the door to possible generalizations of $p$-adic measures attached to automorphic representations of $\GL_2(\A_F)$ of any weight, for any number field $F$.

If we go back to the classical setting treated in this note, we are able to construct $\mu_{f,p}$ in every possible situation except when the local automorphic representation $\pi_p$ attached to $f$ is \emph{supercuspidal}.
We hope that our work clarifies why it is not expected to find good $p$-adic measures in such supercuspidal case. 
Nevertheless, although all cases where we construct $\mu_{f,p}$ were not considered in \cite{MTT86}, almost all of them can be obtained as twists of the $p$-adic measures described there.
However, we obtain a genuinely new construction that we proceed to describe: Assume that $f$ is an eigenvector for the Hecke operator $T_p$ with eigenvalue $a_p$, in this situation Theorem \ref{mainthm} can be expressed as:
\begin{theorem}
Let $f\in S_{k+2}(\Gamma_1(N),\epsilon)$ be a cuspform, and assume that Hecke polynomial $P(X):=X^2-a_pX+\epsilon(p)p^{k+1}$ has a double root $\alpha$. Then there exists a locally analytic $p$-adic measure $\mu_{f,p}^{\rm ext}$ of $\Z_p^\times$ such that, for any locally polynomial character $\chi=\chi_0(x)x^m$ with $m\leq k$,
\begin{equation}\label{IntForInt}
\int_{\Z_p^\times}\chi d\mu_{f,p}^{\rm ext}=\frac{4\pi }{\Omega_f^+i^m}\cdot e_p^{\rm ext}(\pi_p,\chi_0)\cdot L\left(m-k+\frac{1}{2},\pi,\chi_0\right),
\end{equation}
where $L\left(s,\pi,\chi_0\right)$ is the L-function twisted by $\chi_0$, 
\[
e_p^{\rm ext}(\pi_p,\chi_0)=\left\{\begin{array}{ll}
(1-p^{-1})^{-1}\left(p^{k-m}\alpha^{-1}+p^{m-k-1}\alpha-2p^{-1}\right);&\chi_0\mid_{\Z_p^\times}=1;\\
-(1-p^{-1})^{-1}rp^{r(m-k-1)}\alpha^{r}\tau(\chi_0);&{\rm cond}(\chi_0)=r>0,
\end{array}\right.
\]
and $\tau(\chi_0)$ is the Gauss sum of Definition \ref{defGS}.
\end{theorem}

We call $\mu_{f,p}^{\rm ext}$ \emph{the extremal $p$-adic measure}.
Coleman and Edixhoven showed in \cite{ColEd} that $P(X)$ never has double roots if the weight is 2, namely, $k = 0$. Moreover, they showed that assuming \emph{Tate's conjecture} the polynomial $P(X)$ can never be a square for general weights $k+2$. Since we believe in Tate's conjecture, we expect this situation never occur, hence surely the hypothesis of the theorem is never fulfilled and %it is expected that 
$\mu_{f,p}^{\rm ext}$ can never be constructed. Since Tate's conjecture is open, we believe that it is still interesting to study this phenomena. Notice that in this unlikely situation, two $p$-adic measures $\mu_{f,p}$ and $\mu_{f,p}^{\rm ext}$ coexist. In fact we can define an alternative $p$-adic L-function
\[
L_p^{\rm ext}(f,s):=\int_{\Z_p^\times}{\rm exp}(s\cdot{\rm log(x)})d\mu_{f,p}^{\rm ext}(x),
\]
called \emph{the extremal $p$-adic L-function}, that coexists with $L_p(f,s)$, satisfying interpolation property \eqref{IntForInt} with completely different Euler factors $e_p^{\rm ext}(\pi_p,\chi_0)$. One can think that maybe the existence of $L_p^{\rm ext}(f,s)$ could provide a contradiction that would imply that the Hecke polynomial never has multiple roots, thus providing unconditional proof of this fact studied in \cite{ColEd}.

Although the extremal situation of $P(x)$ being a square is discarded by Tate's conjecture in this classical setting, there are examples of Hilbert modular forms satisfying this hypothesis (see \cite[\S 3.3.1]{Chi15}). Since our techniques can be extended to the Hilbert setting following the work of Spiess in \cite{Spi14}, new constructions of concrete extremal $p$-adic L-functions could be provided in these cases. 


\subsubsection*{Acknowledgements.}
The author would like to thank David Loeffler and V\'ictor Rotger for their comments and discussions throughout the development of this paper.

The author is supported in part by DGICYT Grant MTM2015-63829-P.
This project has received funding from the European Research Council
(ERC) under the European Union's Horizon 2020 research and innovation
programme (grant agreement No. 682152).



\subsection{Notation}

For any ring $R$,
we denote by 
$V(k)_R:=\Sym^{k}(R^2)$ the $R$-module of homogeneous polynomials in two variables with coefficients in $R$, endowed with an action of $\GL_2(R)$:
\begin{equation}\label{actpol}
\left( \left(\begin{array}{cc}a&b\\c&d\end{array}\right)\ast P\right)(x,y):=%(ad-bc)^{-\frac{k}{2}}\cdot 
P\left((x,y)\left(\begin{array}{cc}a&b\\c&d\end{array}\right)\right).
\end{equation}
We denote by $V(k):=V(k)_\C$.
Similarly, we define the (right-) action of $A\in\GL_2(\R)^+$ on the set of modular forms of weight $k+2$
\[
(f\mid A)(z):=\rho(A,z)^{k+2}\cdot f(Az);%\quad\left(\begin{array}{cc}a&b\\c&d\end{array}\right)z:=\frac{az+b}{cz+d},
\qquad \rho\left(\left(\begin{array}{cc}a&b\\c&d\end{array}\right),z\right):=\frac{(ad-bc)}{cz+d}.
\]
%If $f\in M_{k+2}(N)$ and $p$ does not define $N$, we define the Hecke operator $T_p$ as
%\[
%T_pf:=\frac{1}{p}\left(f\mid\left(\begin{array}{cc}p&\\&1\end{array}\right)+\sum_{c\in \Z/p\Z}f\mid\left(\begin{array}{cc}1&c\\&p\end{array}\right)\right).
%\]
%Similarly, if $p\parallel N$ we define
%\[
%U_pf:=\frac{1}{p}\left(\sum_{c\in \Z/p\Z}f\mid\left(\begin{array}{cc}1&c\\&p\end{array}\right)\right).
%\]

We will denote by $dx$ the Haar measure of $\Q_p$ so that ${\rm vol}(\Z_p)=1$. Similarly, we write $d^\times x$ for the Haar measure of $\Q_p^\times$ so that ${\rm vol}(\Z_p^\times)=1$. By abuse of notation, will will also denote by $d^\times x$ the corresponding Haar measure of the group of ideles $\A^\times$.   

For any local character $\chi:\Q_p^\times\rightarrow\C^\times$, write
\[
L(s,\chi)=\left\{\begin{array}{lc}(1-\chi(p)p^{-s})^{-1},&\chi\mbox{ unramified}\\
1,&\mbox{otherwise.}\end{array}\right.
\]


\section{Local integrals}

\subsection{Gauss sums}

In this section $\psi:\Q_p\rightarrow\C^\times$ will be a non-trivial additive character such that $\ker(\psi)=\Z_p$.

\begin{lemma}\label{intpsi}
For all $s\in\Q_p^\times$ and $n>0$, we have
\[
\int_{s+p^n\Z_p}\psi(ax)dx=p^{-n}\psi(sa)\cdot1_{\Z_p}(p^na).
\]
In particular,
\[
\int_{\Z_p^\times}\psi(ax)dx=\left\{\begin{array}{ll}(1-p^{-1}),&a\in\Z_p\\-p^{-1},&a\in p^{-1}\Z_p^\times\\0,&\mbox{otherwise}\end{array}\right.
\]
\end{lemma}
\begin{proof}
We compute
\begin{eqnarray*}
\int_{s+p^n\Z_p}\psi(xa)d x&=&\int_{p^n\Z_p}\psi((s+x)a)d x=\psi(sa)\int_{\Z_p}|xp^n|\psi(xp^na)d^\times x\\
&=&p^{-n}\psi(sa)\int_{\Z_p}\psi(xp^na)d x=p^{-n}\psi(sa)\cdot1_{\Z_p}(p^na).
\end{eqnarray*}
To deduce the second part, notice that
\[
\int_{\Z_p^\times}\psi(ax)dx=\sum_{s\in(\Z/p\Z)^\times}\int_{s+p\Z_p}\psi(ax)dx=p^{-1}\sum_{s\in(\Z/p\Z)^\times}\psi(sa)1_{\Z_p}(pa),
\]
hence the result follows.
\end{proof}
\begin{lemma}\label{psichi}
For all $\chi:\Z_p^\times\rightarrow\C^\times$ be a character of conductor $n\geq 1$. Let $1+p^n\Z_p\subset U\subseteq \Z_p^\times$ be a open subgroup. We have
\[
\int_{U}\chi(x)\psi(ax)d^\times x=0,\qquad\mbox{unless $|a|=p^n$.}
\]
\end{lemma}
\begin{proof}
We compute
\begin{eqnarray*}
\int_{U}\chi(x)\psi(ax)d^\times x&=&\sum_{s\in U/(1+p^n\Z_p)}\chi(s)\int_{s+p^n\Z_p}\psi(ax)d x\\
&=&p^{-n}1_{\Z_p}(p^na)\sum_{s\in U/(1+p^n\Z_p)}\chi(s)\psi(sa).
\end{eqnarray*}
Hence the integral $I:=\int_{U}\chi(x)\psi(ax)d^\times x$ must be zero if $a\not\in p^{-n}\Z_p$. Moreover, if $a\in p^{-n+1}\Z_p$,
\[
I=\int_{U}\chi(x(1+p^{n-1}))\psi(ax(1+p^{n-1}))d^\times x=\chi(1+p^{n-1})I=0,
\]
and the result follows.
\end{proof}
We now define the Gauss sum:
\begin{definition}\label{defGS}
For any character $\chi$ of conductor $n\geq 0$, 
\[
\tau(\chi)=\tau(\chi,\psi)=p^n\int_{\Z_p^\times}\chi(x)\psi(-p^{-n}x)dx.
\]
\end{definition}





\section{Classical cyclotomic $p$-adic $L$-function}\label{Classical}

\subsection{Classical Modular symbols}

Let $f\in S_{k+2}(N,\epsilon)$ be a modular cuspidal newform of weight $(k+2)$ level $\Gamma_1(N)$ and nebentypus $\epsilon$. 

By definition, we have
\[
(f\mid A)(z)\cdot (A^{-1}P)(1,-z)\cdot dz=\det(A)\cdot f(Az)\cdot P(1,-Az)\cdot d(Az), \quad A\in \GL_2(\R)^+,
\]
for any $P\in V(k)$. Hence, if we denote by $\Delta_0$ the group of degree zero divisors of $\PP^1(\Q)$ with the natural action of $\GL_2(\Q)$, we obtain the \emph{Modular Symbol}:
\begin{eqnarray*}
&&\phi_f^{\pm}\in \Hom_{\Gamma_1(N)}(\Delta_0,V(k)^\vee);\\ 
&&\phi_f^{\pm}(s-t)(P):=2\pi i\left(\int_t^s f(z)P(1,-z)dz\pm\int_t^s f(-\bar z)P(1,\bar z)d(-\bar z)\right).
\end{eqnarray*}
Notice that $\Gamma_1(N)$-equivariance follows from the fact that the above equality implies
\begin{equation}\label{eqint}
\phi_{f\mid A}^\pm(D)=\det(A)\cdot A^{-1}\left(\phi_f^\pm(AD)\right),\qquad A\in\GL_2(\R)^+.
\end{equation}
The following result is well known and classical:
\begin{proposition}\label{ratMS}
There exists periods $\Omega_\pm$ such that
\[
\phi_f^\pm=\Omega_{\pm}\cdot\varphi_f^\pm,
\]
for some $\varphi_f^\pm\in  \Hom_{\Gamma_1(N)}(\Delta_0,V(k)_{R_f}^\vee)$, where $R_f$ is the ring of coefficients of $f$.
\end{proposition}


\subsection{Classical $p$-adic distributions}\label{classicdist}

Given $f\in S_{k+2}(N,\epsilon)$, we will assume that $f$ is an eigenvector for the Hecke operator $T_p$ with eigenvalue $a_p$. % and $\varepsilon_p:=\varepsilon\mid_{(\Z/p\Z)^\times}$ is trivial. 
%In case that $p\nmid N$, 
Let $\alpha$ be a non zero root of the Hecke polynomial $X^2-a_pX+\epsilon(p)p^{k+1}$%. In case $p\mid N$, let $\alpha$ be the eigenvalue of the Hecke operator $U_p$. %(this value can be $\pm 1$). 

We will construct a distribution $\mu_{f,\alpha}$ of locally polynomial functions of $\Z_p^\times$ of degree less that $k$ attached to $f$ (and $\alpha$ in case $p\nmid N$). Since the open sets $U(a,n)=a+p^n\Z_p$ ($a\in \Z_p^\times$ and $n\in \N$) form a basis of $\Z_p^\times$, it is enough to define the image of $P\left(1,\frac{x-a}{p^n}\right)1_{U(a,n)}(x)$, for any $P\in V(k)$ with integer coefficients:
\begin{equation}\label{eqclassmu}
\int_{U(a,n)}P\left(1,\frac{x-a}{p^n}\right)d\mu^\pm_{f,\alpha}(x):=\frac{1}{\alpha^n}\varphi^\pm_{f_\alpha}\left(\frac{a}{p^n}-\infty\right)(P),
\end{equation}
where $f_{\alpha}(z):=f(z)-\beta\cdot f(pz)$, where $\beta=\frac{\epsilon(p)p^{k+1}}{\alpha}$.
It defines a distribution because $\mu^\pm_{f,\alpha}$ satisfies \emph{additivity}, namely, since 
\[
P\left(1,\frac{x-a}{p^n}\right)1_{U(a,n)}(x)=%p^{\frac{k}{2}}
\sum_{b\equiv a\;{\rm mod} \;p^{n}} (\gamma_{a,b}P)\left(1,\frac{x-b}{p^{n+1}}\right)1_{U(b,n+1)}(x),\quad\gamma_{a,b}:=\mbox{\tiny$\left(\begin{array}{cc}1&\frac{b-a}{p^n}\\0&p\end{array}\right)$},
\]
it can be shown using that $U_p f_\alpha=\alpha f_\alpha$ that 
\[
\int_{U(a,n)}P\left(1,\frac{x-a}{p^n}\right)d\mu^\pm_{f,\alpha}(x)=%p^{\frac{k}{2}}
\sum_{b\equiv a\;{\rm mod} \;p^{n}}\int_{U(b,n+1)} (\gamma_{a,b}P)\left(1,\frac{x-b}{p^{n+1}}\right)d\mu^\pm_{f,\alpha}(x).
\]
%Indeed, we have that $f_{\alpha_p}\in S_{k+2}({\rm lcm}(N,p))$ is a form that satisfies $U_p f_{\alpha_p}=\alpha_p\cdot f_{\alpha_p}$, thus, 
%\begin{eqnarray*}
%\int_{U(a,n)}P\left(1,\frac{x-a}{p^n}\right)d\mu^\pm_{f,\alpha_p}(x)&=&\frac{1}{p\alpha_p}\sum_{\gamma_{a,b}}\frac{1}{\alpha_p^n}\varphi^\pm_{f_{\alpha_p}\mid \gamma_{a,b}}\left(\frac{a}{p^n}-\infty\right)(P)\\
%&=&\sum_{b\equiv a\;{\rm mod} \;p^{n}}\frac{1}{\alpha_p^{n+1}}\varphi^\pm_{f_{\alpha_p}}\left(\gamma_{a,b}\frac{a}{p^n}-\gamma_{a,b}\infty\right)(\gamma_{a,b}P)\\
%&=&\sum_{b\equiv a\;{\rm mod} \;p^{n}}\frac{1}{\alpha_p^{n+1}}\varphi^\pm_{f_{\alpha_p}}\left(\frac{b}{p^{n+1}}-\infty\right)(\gamma_{a,b}P)
%\end{eqnarray*}
%where the second equality follows from \eqref{eqint}.

The following result shows that, under certain hypothesis, we can extend $\mu^\pm_{f,\alpha}$ to a locally analytic measure.
\begin{theorem}[Visnik, Amice-V\'elu]\label{ThmVAV}
Fix an integer $h$ such that $1\leq h\leq k+1$. Suppose that %the polynomial $X^2-a_pX+p^{k+1}$ has a root 
 $\alpha$ satisfies ${\rm ord}_p\alpha<h$. Then there exists a locally analytic measure $\mu_{f,\alpha}^\pm$ satifying:
\begin{itemize}
\item $\int_{U(a,n)}P\left(1,\frac{x-a}{p^n}\right)d\mu^\pm_{f,\alpha}(x):=\frac{1}{\alpha^n}\varphi^\pm_{f_{\alpha}}\left(\frac{a}{p^n}-\infty\right)(P)$, for any locally polynomial function $P\left(1,\frac{x-a}{p^n}\right)1_{U(a,n)}(x)$ of degree strictly less than $h$.

\item For any $m\geq 0$,
\[
\int_{U(a,n)}(x-a)^md\mu^\pm_{f,\alpha}(x)\in \left(\frac{p^m}{\alpha}\right)^n\alpha^{-1}.
\]

\item If $F(x)=\sum_{m\geq 0}c_m(x-a)^m$ is convergent on $U(a,n)$, then
\[
\int_{U(a,n)}F(x)d\mu^\pm_{f,\alpha}(x)=\sum_{m\geq 0}c_m\int_{U(a,n)}(x-a)^md\mu^\pm_{f,\alpha}(x).
\]
\end{itemize}
\end{theorem}
If we assume that there exists such a root $\alpha$ with ${\rm ord}_p\alpha<k+1$, then we define the (\emph{cyclotomic}) \emph{$p$-adic $L$-function}:
\[
L_p(f,\alpha,s):=\int_{\Z_p^\times}{\rm exp}(s\cdot{\rm log(x)})d\mu_{f,\alpha}^+(x).
\]







\section{$p$-adic $L$-functions}\label{padicLfunct}

In this section we provide a reinterpretation of the distributions $\mu^\pm_{f,\alpha_p}$. 
Let $f\in S_{k+2}(\Gamma_1(N),\epsilon)$ be a cuspidal newform as above and let $p$ be any prime. Fix the embedding
\begin{equation}\label{actQ_p}
\Z_p^\times\hookrightarrow\Q_p^\times\hookrightarrow\GL_2(\Q_p);\qquad x\longmapsto\left(\begin{array}{cc}x&\\&1\end{array}\right).
\end{equation}


\begin{assumption}\label{mainassumption}
Assume that there exists a $\Z_p^\times$-equivariant morphisms 
\[
\delta:C(\Z_p^\times,L)\longrightarrow V,
\]
where $L$ is certain finite extension of the coefficient field $\Q(\{a_n\}_n)$,  and $V$ is certain model over $L$ of the local automorphic representation $\pi_p$ generated by $f$. Assume also that, for big enough $n$,
\begin{equation}\label{keyprop}
\left(\begin{array}{cc}1&s\\&p^n\end{array}\right)\delta(1_{U(s,n)})=\frac{1}{\gamma^n}\sum_{i=0}^mc_i(s,n)V_i,
\end{equation}
where $m$ is fixed, $V_i\in V$ do not depend neither $s$ nor $n$, and $c_i(s,n)\in \cO_L$.
\end{assumption}


\subsection{$p$-adic distributions}

Let us consider the subgroup
\[
\hat K_1(N)=\left\{g\in \GL_2(\hat \Z):\;g\equiv(\begin{smallmatrix}\ast&\ast\\0&1\end{smallmatrix})\;{\rm mod}\;N\right\}.
\]
Again by strong approximation we have that $\GL_2(\A_f)=\GL_2(\Q)^+\hat K_1(N)$.
Thus, for any $\GL_2(\A_f)\ni g=h_gk_g$, where $h_g\in \GL_2(\Q)^+$, $k_g\in \hat K_1(N)$ are well defined up to multiplication by $\Gamma_1(N)=\GL_2(\Q)^+\cap\hat K_1(N)$.
Write $K:=\hat K_1(N)\cap\GL_2(\Z_p)$. By strong multiplicity one $\pi_p^K$ is one dimensional. Therefore $V^K=Lw_0$ and $V=L[\GL_2(\Q_p)]w_0$.
Notice that we have a natural morphism 
\[
\varphi_{f,p}^\pm:V\longrightarrow\Hom(\Delta_0,V(k)_L^\vee);\qquad \varphi_{f,p}^\pm(gw_0)=\det(h_g)\cdot
\varphi_{f\mid h_g^{-1}}^\pm.
\] 
\begin{remark}\label{rmkacthom}
Notice that if $g\in\GL_2(\Q_p)$ then $h_g\in \hat K_1(N)^p:=\hat K_1(N)\cap\prod_{\ell\neq p}\GL_2(\Q_\ell)$. This implies that, for any $h\in \GL_2(\Q)^+\cap\hat K_1(N)^p$, we have $h_{hg}=h\cdot h_g$, for all $g\in\GL_2(\Q_p)$.
By \eqref{eqint}, this implies that $\varphi_{f,p}^\pm(hv)=h\ast\varphi_{f,p}^\pm(v)$, for all $v\in V\subset\pi_p$, where the action of $h\in \GL_2(\Q)^+\cap\hat K_1(N)^p$ is given by 
\[
(h\ast\varphi)(D):=h(\varphi(h^{-1}D)),\qquad\varphi\in \Hom(\Delta_0,V(k)_L^\vee).
\]
\end{remark}
\begin{remark}\label{rmkchar}
By definition, for any $\left(\begin{smallmatrix}a&b\\c&d\end{smallmatrix}\right)\in\Gamma_0(N)$, we have 
\[
f\left(\frac{az+b}{cz+d}\right)=\epsilon(d)\cdot(cz+d)^{k+2}f(z),\qquad f\mid\left(\begin{smallmatrix}a&b\\c&d\end{smallmatrix}\right)=\epsilon(d)\cdot  f.
\]
For any $z\in \Q_p^\times$ such that $z=p^nu$ where $u\in\Z_p^\times$, we can choose $d\in\Z$ such that $d\equiv u^{-1}\;{\rm mod}\;N\Z_p$ and $d\equiv p^{n}\;{\rm mod}\;N\Z_\ell$, for $\ell\neq p$. Let us choose $A=(\begin{smallmatrix}a&b\\c&d\end{smallmatrix})\in\Gamma_0(N)$, and we have  
\[
(z,1)=p^nA^{-1}(uA,p^{-n}A)\in\GL_2(\A_f),\qquad (uA,p^{-n}A)\in\hat K_1(N).
\]
This implies that, if $\varepsilon_p$ is the central character of $\pi_p$, 
\[
\varepsilon_p(z)\varphi_{f,p}^\pm(w_0)=\varphi_{f,p}^\pm(zw_0)=\det(p^nA^{-1})\cdot\varphi_{f\mid p^{-n}A}^\pm=p^{-nk}\epsilon(d)\cdot\varphi_{f}^\pm
\]
Hence $\varepsilon_p=\epsilon_p^{-1}|\cdot|^k$, where $\epsilon_p=\epsilon\mid_{\Z_p^\times}$.
\end{remark}

Again let $C_k(\Z_p^\times,\C_p)$ be the space of locally polynomial functions of $\Z_p^\times$ of degree less that $k$. Recall the $\Z_p^\times$-equivariant isomorphism
\begin{equation}\label{imathpol}
\imath:C(\Z_p^\times,\Z)\otimes_\Z V(k)_{\C_p}(-k)\longrightarrow C_k(\Z_p^\times,\C_p) ;\qquad h\otimes P\longmapsto P(1,x)\cdot h(x).
\end{equation}
Fixing $L\hookrightarrow\C_p$, we define the distributions $\mu^\pm_{f,p}$ attached to $f$ and $\delta$:
\begin{equation}\label{defmugen}
\int_{\Z_p^\times}\imath(h\otimes P)(x)d\mu^\pm_{f,p}(x):=\varphi_{f,p}^\pm(\delta(h))(0-\infty)(P).
\end{equation}
%\begin{remark}
%Notice that all our choices are compatible, in the sense that, for all $\gamma\in \Q^\times\cap\Z_p^\times$,
%\begin{eqnarray*}
%\int_{\Z_p^\times}\gamma\imath(h\otimes P)(x)d\mu^\pm_{f,p}(x)
%\end{eqnarray*}
%\end{remark}



\subsection{Admissible Distributions}

We have just constructed a distribution 
\[
\mu_{f,p}^\pm:C_k(\Z_p^\times,\C_p)\longrightarrow \C_p.
\]
This section is devoted to extend this distribution to a locally analytic measure $\mu_{f,p}^\pm\in\Hom\left(C_{\rm loc-an}(\Z_p^\times,\C_p),\C_p\right)$.

\begin{definition}
Write $v_p:\C_p\rightarrow\Q\cup\{-\infty\}$ the usual normalized $p$-adic valuation. 
For any $h\in \R^+$,
a distribution $\mu\in \Hom(C_k(\Z_p^\times,\C_p),\C_p)$ is \emph{$h$-admissible} if 
\[
v_p\left(\int_{U(a,n)}g d\mu\right)\geq v_p(A)-n\cdot h,
\]
for some fixed $A\in \C_p$, and any $g\in C_k(\Z_p^\times,\cO_{\C_p})$ which is polynomical in a small enough $U(a,n)\subseteq\Z_p^\times$. We will denote previous relation by
\[
\int_{U(a,n)}g d\mu\in A\cdot p^{-n h}\cO_{\C_p}.
\]
\end{definition}


\begin{proposition}\label{propext}
If $h<k+1$, a $h$-admissible the distribution $\mu$ can be extended to a locally analytic measure such that
\[
\int_{U(a,n)}g d\mu\in A\cdot p^{-n h}\cO_{\C_p}, 
\]
for any $g\in C(\Z_p^\times,\cO_{\C_p})$ which is analytic in $U(a,n)$.
\end{proposition}
\begin{proof}
Notice that any locally analytic function is topologically generated by functions of the form $P_m^{a,N}(x):=\left(\frac{x-a}{p^N}\right)^m1_{U(a,N)}(x)$, where $m\in\N$. By definition,  we have defined the values $\mu(P^{a,N}_m)$ when $m\leq k$. If $m>h$, we define $\mu(P_m^{a,N})=\lim_{n\rightarrow\infty}a_n$, where
\[
a_n=\sum_{b\;{\rm mod}\; p^{n};\;b\equiv a\;{\rm mod}\; p^N}\sum_{j\leq h}\left(\frac{b-a}{p^N}\right)^{m-j}\binom{m}{j}p^{j(n-N)}\mu(P_j^{b,n})%\in Ap^{j(n-N)-nh}\cO_{\C_p}.
\]
and the definition agrees with $\mu$ when $h<m\leq k$ because $p^{j(n-N)}\mu(P_j^{b,n})\stackrel{n}{\rightarrow}0$ when $j>h$, hence
\[
\lim_{n\rightarrow\infty}a_n=\sum_{b\;{\rm mod}\; p^{n};\;b\equiv a\;{\rm mod}\; p^N}\sum_{j=0}^m\left(\frac{b-a}{p^N}\right)^{m-j}\binom{m}{j}p^{j(n-N)}\mu(P_j^{b,n})=\mu(P_m^{a,N})
\]
The limit converge because $\{a_n\}_n$ is Cauchy, indeed by additivity
\[
a_{n_2}-a_{n_1}=\sum_{j\leq h}\sum_{b\equiv a\;(p^{n_2})}\sum_{b'\equiv b\;(p^{n_1})}\sum_{k=h+1}^{m}r(k)\binom{k}{j}\left(\frac{b'-b}{p^N}\right)^{k-j}p^{(n_2-N)j}\mu(P_{j}^{b',n_2}),
\]
where $r(k)=\binom{m}{k}\left(\frac{b'-a}{p^N}\right)^{m-k}$.
Since 
\[
\left(\frac{b'-b}{p^N}\right)^{k-j}p^{(n_2-N)j}\mu(P_{j}^{b',n_2})\in A\cdot p^{-Nk}p^{(n_1-n_2)(k-j)}p^{(k-h)n_2}\cO_{\C_p},
\]
we have that $a_{n+1}-a_{n}\stackrel{n}{\rightarrow} 0$. 

It is clear by the definition that $\mu(P_m^{a,N})\in A\cdot p^{-N h}\cO_{\C_p}$ for all $m, a$ and $N$. Moreover, it extends to a locally analytic measure by continuity which is determined by the image of locally polynomial functions of degree at most $h$.
\end{proof}

Notice that, for all $m\leq k$,
\[
P_m^{a,n}(x)=\left(\frac{x-a}{p^n}\right)^m1_{U(a,n)}(x)=\imath\left(1_{U(a,n)}\otimes\left(\frac{Y-aX}{p^n}\right)^mX^{k-m}\right)
\]
Using property \eqref{keyprop} and Remarks \ref{rmkacthom} and Remark \ref{rmkchar}, we compute that 
\begin{eqnarray*}
\int_{\Z_p^\times}P_m^{a,n} d\mu^\pm_{f,p}&=&\varphi_{f,p}^\pm(\delta(1_{U(a,n)}))(0-\infty)\left(\left(\frac{Y-aX}{p^n}\right)^mX^{k-m}\right)\\
&=&\sum_{i=0}^m\frac{c_i(a,n)}{\gamma^n}\cdot\varphi_{f,p}^\pm\left(p^{-n}\left(\begin{smallmatrix}p^n&-a\\&1\end{smallmatrix}\right)V_i\right)(0-\infty)\left(\left(\frac{Y-aX}{p^n}\right)^mX^{k-m}\right)\\
&=&\sum_{i=0}^m\frac{c_i(a,n)}{\varepsilon_p(p)^{n}\gamma^n}\cdot\varphi^\pm_{f,p}(V_i)\left(\frac{a}{p^n}-\infty\right)\left((p^{-n}Y)^m(p^{-n}X)^{k-m})\right)\\
&=&\sum_{i=0}^m\frac{c_i(a,n)}{\gamma^n}\cdot\varphi^\pm_{f,p}(V_i)\left(\frac{a}{p^n}-\infty\right)\left(Y^mX^{k-m}\right).
\end{eqnarray*}
Notice that $\varphi^\pm_{f,p}(V_i)\in \Hom(\Delta_0,V(k)_L^\vee)^{\Gamma_1(Np^r)}_\epsilon:=\Hom_{\Gamma_1(Np^r)}(\Delta_0,V(k)_L^\vee)_\epsilon$ for some big enough $r\in\N$, where the subindex $\epsilon$ indicates that the action of $\Gamma_1(Np^r)/\Gamma_0(Np^r)$ is given by the character $\epsilon$. By Manin's trick we have that 
\[
\Hom_{\Gamma_1(Np^r)}(\Delta_0,V(k)_L^\vee)_\epsilon\simeq\Hom_{\Gamma_1(Np^r)}(\Delta_0,V(k)_{\cO_L}^\vee)_\epsilon\otimes_{\cO_L} L.
\]
Since $Y^mX^{k-m}\in V(k)_{\cO_L}$, $c(a,n)\in\cO_L$ and the functions $P_m^{a,n}$ generate $C_k(\Z_p^\times,\cO_{\C_p})$, we deduce that 
\begin{equation}\label{calcint}
\int_{U(a,n)}gd\mu^\pm_{f,p}\in\frac{A}{\gamma^n}\cO_{\C_p},\qquad\mbox{for all }g\in C_k(\Z_p^\times,\cO_{\C_p}),
\end{equation}
and some fixed $A\in L$.
We deduce the following result.
\begin{theorem}\label{thmadm}
Fix an embedding $L\hookrightarrow\C_p$. We have that $\mu^\pm_{f,p}$ is $v_p(\gamma)$-admissible. %, where $v:\C_p^\times\rightarrow\Q$ is the valuation such that $v(p)=1$.
\end{theorem}

%\begin{remark}
%If $\pi(\chi,\epsilon_p)$ is spherical, by Remark \ref{rmkcharpol} the algebraic number $\alpha_p=\chi(p)^{-1}$ is a root of the polynomial $X^2-a_pX+p^{k+1}$. Moreover, $a_p$ is an algebraic integer. If $a_p$ belongs to the maximal ideal of $\cO_L$, this implies that $h=\frac{k+1}{2}$ and if it does not we can always choose an embedding $L\hookrightarrow\C_p$ so that $h<k+1$. In any of the cases we have that $h<k+1$, thus by Proposition \ref{propext} we can extend  $\mu^\pm_{f,p}$ to a locally analytic measure. 
%\end{remark}


\subsection{Interpolation properties}



%\subsection{Modular forms vs Automorphic forms}

Given the modular form $f\in S_{k+2}(\Gamma_1(N))$, let us consider the automorphic form $\phi:\GL_2(\Q)\backslash\GL_2(\A)\rightarrow\C$, characterized by its restriction to $\GL_2(\R)^+\times\GL_2(\A_f)$:
\[
\phi(g_\infty,g_f)=\frac{\det\left(\gamma\right)}{\det(g_\infty)}\cdot f\mid\gamma^{-1}g_\infty \left( i\right), \quad g_f=\gamma k\in \GL_2(\Q)^+\hat K_1(N), \quad g_\infty=\left(\begin{smallmatrix}a&b\\c&d\end{smallmatrix}\right).
\]
Given $\varphi_{f,p}=\varphi_{f,p}^+$ and $g\in\GL_2(\Q_p)$, we compute 
\begin{eqnarray*}
\varphi_{f,p}(gw_0)(0-\infty)(P)&=&\det(h_g)\cdot\varphi_{f\mid h_g^{-1}}(0-\infty)(P)\\
&=&\frac{-4\pi\det(h_g)}{\Omega_f^+}\cdot\int_\infty^0 f\mid h_g^{-1}(ix)P(1,-ix)dx\\
&=&\frac{4\pi }{\Omega_f^+}\cdot\int_{\R^+}P(x^{-1},-i)\cdot\phi\left(\left(\begin{smallmatrix}x&\\&1\end{smallmatrix}\right),g\right)d^{\times}x.
\end{eqnarray*}
This implies that, if we consider the automorphic representation $\pi$ generated by $\phi$, and the $\GL_2(\Q_p)$-equivariant morphism
\[
\phi_f:\pi_p\longrightarrow\pi:\qquad gw_0\longmapsto g\phi,
\]
we have that 
\[
\varphi_{f,p}(\delta(h))(0-\infty)(P)=\frac{4\pi }{\Omega_f^+}\cdot\int_{\R^+}P(x^{-1},-i)\cdot\phi_f\left(\delta(h)\right)\left(\left(\begin{smallmatrix}x&\\&1\end{smallmatrix}\right),1\right)d^{\times}x.
\]
Let $H$ be the maximum subgroup of $\Z_p^\times$ such that $h\mid_{sH}$ is constant, for all $sH\in\Z_p^\times/H$. Notice that $h=\sum_{s\in\Z_p^\times/H}h(s)1_{sH}$. Moreover, for all $v\in\pi_p$, the automorphic form $\phi_f(v)$ is $U^p:=\prod_{\ell\neq p}\Z_\ell^\times$-invariant when embedded in $\GL_2(\A_f)$  by means of \eqref{actQ_p}. Hence we have $\varphi_{f,p}(\delta(h))(0-\infty)(P)=$
\begin{eqnarray*}
&=&\sum_{sH\in\Z_p^\times/H}\frac{4\pi h(s)}{\Omega_f^+}\cdot\int_{\R^+}\int_{U^p}P(x^{-1},-i)\phi_f\left(\delta(1_{sH})\right)\left(\left(\begin{smallmatrix}x&\\&1\end{smallmatrix}\right),1,\left(\begin{smallmatrix}t&\\&1\end{smallmatrix}\right)\right)d^{\times}xd^\times t\\
&=&\sum_{sH\in\Z_p^\times/H}\frac{4\pi h(s)}{\Omega_f^+}\cdot\int_{\R^+}\int_{U^p}P(x^{-1},-i)\phi_f\left(\delta(1_{H})\right)\left(\left(\begin{smallmatrix}x&\\&1\end{smallmatrix}\right),\left(\begin{smallmatrix}s&\\&1\end{smallmatrix}\right),\left(\begin{smallmatrix}t&\\&1\end{smallmatrix}\right)\right)d^{\times}xd^\times t\\
&=&\frac{4\pi}{\Omega_f^+{\rm vol}(H)}\cdot\int_{\A^\times/\Q^\times}\tilde h(y)\cdot\phi_f\left(\delta(1_{H})\right)\left(\begin{smallmatrix}y&\\&1\end{smallmatrix}\right)d^{\times}y,
\end{eqnarray*}
where $\tilde h(y)=P(|y|^{-1},-i)\cdot h(y_p|y|y_\infty^{-1})$, for all $y=(y_v)_v\in\A^\times$.



Let $\chi\in C_k(\Z_p^\times,\C_p)$ be a locally polynomial character. This implies that $\chi(x)=\chi_0(x)x^m$, for some natural $m\leq k$ and some locally constant character $\chi_0$. This implies that $\chi=\imath(\chi_0\otimes Y^{m}X^{k-m})$. We deduce that
\[
\int_{\Z_p^\times}\chi(x)d\mu_{f,p}(x):=\frac{4\pi }{\Omega_f^+i^m{\rm vol}(H)}\cdot\int_{\A^\times/\Q^\times}\tilde\chi_0(y)|y|^{m-k}\phi_f\left(\delta(1_H)\right)\left(\begin{smallmatrix}y&\\&1\end{smallmatrix}\right)d^{\times}y,
\]
where $\tilde\chi_0(y):=\chi_0(y_p|y|y_\infty^{-1})$.

Let $\psi:\A/\Q\rightarrow\C^\times$ be a global additive character and we define the Whittaker model element
\[
W^H:\GL_2(\A)\longrightarrow\C;\qquad W^H(g):=\int_{\A/\Q}\phi_f(\delta(1_H))\left(\left(\begin{array}{cc}1&x\\&1\end{array}\right)g\right)\psi(-x)dx.
\]
This element admits a expression $W^H(g)=\prod_vW^H_v(g_v)$, if $g=(g_v)\in\GL_2(\A)$. Moreover by \cite[Theorem 3.5.5]{Bump}, it provides the \emph{Fourier expansion} 
\[
\phi_f(\delta(1_H))(g)=\sum_{a\in\Q^\times}W^H\left(\left(\begin{array}{cc}a&\\&1\end{array}\right)g\right).
\]
We compute
\begin{eqnarray*}
\int_{\A^\times/\Q^\times}\tilde\chi_0(y)|y|^{m-k}\phi_f\left(\delta(1_H)\right)\left(\begin{smallmatrix}y&\\&1\end{smallmatrix}\right)d^{\times}y&=&\int_{\A^\times}\tilde\chi_0(y)|y|^{m-k}W^H\left(\begin{smallmatrix}y&\\&1\end{smallmatrix}\right)d^{\times}y\\
&=&\prod_v \int_{\Q_v^\times}\tilde\chi_0(y_v)|y_v|^{m-k}W_v^H\left(\begin{smallmatrix}y_v&\\&1\end{smallmatrix}\right)d^{\times}y_v.
\end{eqnarray*}
By definition of $\delta$, when $v\neq p$ the element $W_v^H$ correspond to the new-vector, thus by \cite[Proposition 3.5.3]{Bump} 
\[
\int_{\Q_v^\times}\tilde\chi_0(y_v)|y_v|^{m-k}W_v^H\left(\begin{smallmatrix}y_v&\\&1\end{smallmatrix}\right)d^{\times}y_v=L_v\left(m-k+\frac{1}{2},\pi_v,\tilde\chi_0\right),\qquad v\neq p.
\]
We conclude using the results explained in \cite[\S 3.5]{Bump} 
\[
\int_{\Z_p^\times}\chi(x)d\mu_{f,p}(x)=\frac{4\pi }{\Omega_f^+i^m}\cdot e_p(\pi_p,\chi_0)\cdot L\left(m-k+\frac{1}{2},\pi,\tilde\chi_0\right),
\]
where the \emph{Euler factor}
\[
e_p(\pi_p,\chi_0)=\frac{L_p\left(m-k+\frac{1}{2},\pi_p,\tilde\chi_0\right)^{-1}}{{\rm vol}(H)}\int_{\Q_p^\times}\tilde\chi_0(y_p)|y_p|^{m-k}W_p^H\left(\begin{smallmatrix}y_p&\\&1\end{smallmatrix}\right)d^{\times}y_p.
\]


\subsection{The morphisms $\delta$}

In this section we will construct morphisms $\delta$ satisfying Assumption \ref{mainassumption}. The only case that will be left is the case when $\pi_p$ is \emph{supercuspidal}, in this situation we will not be able to construct admissible $p$-adic distributions.

Let $\pi_p$ be the local representation. Let $W:\pi_p\rightarrow\C$ be the Whittaker functional, and let us consider the Kirillov model $\mathcal{K}$ given by the embedding
\[
\lambda:\pi_p\hookrightarrow \mathcal{K};\qquad \lambda(v)(y)=W\left(\left(\begin{array}{cc}y&\\&1\end{array}\right)v\right).
\]
Recall that the Kirillov model lies in the space of locally constant functions $\phi:\Q_p^\times\rightarrow\C$ endowed with the action
\begin{equation}\label{eqKir}
\left(\begin{array}{cc}1&x\\&1\end{array}\right)\phi(y)=\psi(xy)\phi(y),\qquad \left(\begin{array}{cc}a&\\&1\end{array}\right)\phi(y)=\phi(ay).
\end{equation}

We construct the $\Z_p^\times$-equivariant morphism
\[
\delta:C(\Z_p^\times,\C)\longrightarrow \mathcal{K};\qquad \delta(h)(y)=\int_{\Z_p^\times}\Psi(zy)h(z)\psi(-zy)d^\times z,
\]
for a well chosen locally constant function $\Psi$. Notice that, if $h=1_{H}$ for $H$ small enough 
\[
\delta(h)(y)=\Psi(y)\int_{H}\psi(-zy)d^\times z={\rm vol}(H)\Psi(y),\qquad \mbox{if $|y|<<0$.}
\]
This implies that, in order to choose $\Psi$, we need to control how $\mathcal{K}$ looks like:
\begin{itemize}
\item By \cite[Theorem 4.7.2]{Bump}, if $\pi_p=\pi(\chi_1,\chi_2)$ principal series then $\mathcal{K}$ consists on functions $\phi$ such that vanish for values $y$ with large absolute value and if $|y|$ is small there exists constants $C_1$ and $C_2$ such that
\[
\phi(y)=\left\{\begin{array}{ll}C_1|y|^{1/2}\chi_1(y)+C_2|y|^{1/2}\chi_2(y),&\chi_1\neq\chi_2,\\C_1|y|^{1/2}\chi_1(y)+C_2v(y)|y|^{1/2}\chi_1(y),&\chi_1=\chi_2,\end{array}\right.
\]  
where $v:\Q_p^\times\rightarrow\Z$ is the valuation.

\item By \cite[Theorem 4.7.3]{Bump}, if $\pi_p=\sigma(\chi_1,\chi_2)$ a special representation such that $\chi_1\chi_2^{-1}=|\cdot|^{-1}$ then $\mathcal{K}$ consists on functions $\phi$ such that vanish for values $y$ with large absolute value and if $|y|$ is small there exists constants $C$ such that
\[
\phi(y)=C|y|^{1/2}\chi_2(y).
\]  

\item By \cite[Theorem 4.7.1]{Bump} If $\pi_p$ is supercuspidal then $\mathcal{K}=C_c(\Q_p^\times,\C)$.

\end{itemize}

By Lemma \ref{intpsi} and Lemma \ref{psichi} we have that $\delta(h)(y)=0$ for $y$ with big absolute value. This implies that 
\begin{itemize}
\item In case $\pi_p=\pi(\chi_1,\chi_2)$ with $\chi_1\neq\chi_2$, we can choose 
\[
\Psi=|\cdot|^{1/2}\chi_1\qquad\mbox{or}\qquad\Psi=|\cdot|^{1/2}\chi_2.
\]

\item In case $\pi_p=\pi(\chi_1,\chi_2)$ with $\chi_1=\chi_2$, we can choose 
\[
\Psi=|\cdot|^{1/2}\chi_1\qquad\mbox{or}\qquad\Psi=v\cdot|\cdot|^{1/2}\chi_1.
\]

\item In case $\pi_p=\sigma(\chi_1,\chi_2)$ we have 
\[
\Psi=|\cdot|^{1/2}\chi_2.
\]

\item In case $\pi_p$ supercuspidal it is not possible.
\end{itemize}

We have to prove whether $\delta$ satisfies the property \eqref{keyprop}:
If $\Psi$ is invariant under the action of $1+p^n\Z_p$,
\begin{eqnarray*}
\left(\begin{smallmatrix}1&a\\&p^n\end{smallmatrix}\right)\delta(1_{U(a,n)})(y)=&=&\left(\begin{smallmatrix}p^{n}&\\&p^n\end{smallmatrix}\right)\left(\begin{smallmatrix}p^{-n}&\\&1\end{smallmatrix}\right)\left(\begin{smallmatrix}1&a\\&1\end{smallmatrix}\right)\delta(1_{U(a,n)})(y)\\
&=&\varepsilon_p(p^{n})\cdot \psi(ap^{-n}y)\cdot\delta(1_{U(a,n)})(p^{-n}y)\\
&=&\varepsilon_p(p)^{n}\cdot \int_{U(a,n)}\Psi(p^{-n}yz)\psi(p^{-n}y(a-z))d^\times z\\
&=&\frac{\varepsilon_p(p)^{n}\cdot\Psi(p^{-n}ya)\cdot |p|^n}{1-p^{-1}}\cdot\int_{\Z_p}\psi(yz)d z\\
&=&\frac{\varepsilon_p(p)^{n}\cdot |p|^n}{1-p^{-1}}\cdot\Psi(p^{-n}ya)\cdot 1_{\Z_p}(y),
\end{eqnarray*}
since $d^\times x=(1-p^{-1})^{-1}|x|^{-1}dx$.
\begin{itemize}
\item If $\Psi$ is a character we deduce the property \eqref{keyprop} with $m=0$, $\gamma=\Psi(p)p\varepsilon_p(p)^{-1}$, $c_0(a,n)=\Psi(a)$ and $V_0=(1-p^{-1})^{-1}\Psi(y) 1_{\Z_p}(y)$. 

\item If $\Psi=v\cdot\chi$, with $\chi$ a character, it also satisfies property \eqref{keyprop} with $m=1$, $\gamma=\chi(p)p\varepsilon_p(p)^{-1}$, $c_0(a,n)=-n\chi(a)$, $c_1(a,n)=\chi(a)$, $V_0=(1-p^{-1})^{-1}\chi(y) 1_{\Z_p}(y)$ and $V_1=(1-p^{-1})^{-1}v(y)\chi(y) 1_{\Z_p}(y)$.
\end{itemize}


\subsection{Computation Euler systems}

The following result describes the Euler factors in each of the situations:
\begin{proposition}\label{EulerFactors}
We have the following cases:
\begin{itemize}
\item[$(i)$] If $\Psi=|\cdot|^{1/2}\chi_i$ we have that
\[
e_p(\pi_p,\chi_0)=\left\{\begin{array}{lc}
\frac{(1-p^{-1})^{-1}p^{r(m-k-\frac{1}{2})}\chi_i(p)^{-r}\tau(\chi_0\chi_i,\psi)}{L(m-k+1/2,\tilde\chi_0\chi_j)L(k-m+1/2,\tilde\chi_0\chi_i^{-1})},&\pi_p=\pi(\chi_i,\chi_j);\\
\frac{(1-p^{-1})^{-1}p^{r(m-k-\frac{1}{2})}\chi_i(p)^{-r}\tau(\chi_0\chi_i,\psi)}{L(k-m+1/2,\tilde\chi_0\chi_i^{-1})},&\pi_p=\sigma(\chi_i,\chi_j),
\end{array}
\right.
\]
where $r$ is the conductor of $\chi_i\chi_0$.

\item[$(ii)$] If $\Psi=v\cdot|\cdot|^{1/2}\chi_i$ we have that
\[
e_p(\pi_p,\chi_0)=\left\{\begin{array}{ll}
\frac{p^{k-m-\frac{1}{2}}\chi_i(p)+p^{m-k-\frac{1}{2}}\chi_i(p)^{-1}-2p^{-1}}{1-p^{-1}};&\chi_0\chi_i\mid_{\Z_p^\times}=1;\\
\frac{-rp^{r(m-k-\frac{1}{2})}\chi_i(p)^{-r}\tau(\chi_0\chi_i,\psi)}{1-p^{-1}};&{\rm cond}(\chi_0\chi_i)=r>0.
\end{array}\right.
\]
\end{itemize}
\end{proposition}
\begin{proof}
In order to compute the Euler factors $e_p(\pi_p,\chi_0)$, we have to compute the local periods
\[
I_\delta:=\frac{1}{{\rm vol}(H)}\int_{\Q_p^\times}\tilde\chi_0(y)|y|^{m-k}W_p^H\left(\begin{smallmatrix}y&\\&1\end{smallmatrix}\right)d^\times y=\frac{1}{{\rm vol}(H)}\int_{\Q_p^\times}\tilde\chi_0(y)|y|^{m-k}\delta(1_H)(y)d^\times y.
\]
Recalling that $\tilde\chi_0$ is $H$-invariant, we obtain
\[
I_\delta=\frac{1}{{\rm vol}(H)}\int_{\Q_p^\times}\tilde\chi_0(y)|y|^{m-k}\int_{H}\Psi(zy)\psi(-zy)d^\times zd^\times y=\int_{\Q_p^\times}\tilde\chi_0(x)|x|^{m-k}\Psi(x)\psi(-x)d^\times x.
\]


In case $(i)$ we have that $\Psi=|\cdot|^{1/2}\chi_i$,
hence by Lemma \ref{intpsi} and Lemma \ref{psichi}
\begin{eqnarray*}
I_\delta&=&\sum_np^{n(k-m-\frac{1}{2})}\chi_i(p)^n\int_{\Z_p^\times}\chi_0(x)\chi_i(x)\psi(-p^nx)d^\times x\\
&=&\left\{\begin{array}{ll}
\sum_{n\geq0}p^{n(k-m-\frac{1}{2})}\chi_i(p)^n-(1-p^{-1})^{-1}p^{m-k-\frac{1}{2}}\chi_i(p)^{-1};&\chi_0\chi_i\mid_{\Z_p^\times}=1;\\
(1-p^{-1})^{-1}p^{r(m-k-\frac{1}{2})}\chi_i(p)^{-r}\tau(\chi_0\chi_i,\psi);&{\rm cond}(\chi_0\chi_i)=r>0
\end{array}\right.\\
&=&\left\{\begin{array}{ll}
(1-p^{-1})^{-1}(1-p^{m-k-\frac{1}{2}}\chi_i(p)^{-1})(1-p^{k-m-\frac{1}{2}}\chi_i(p))^{-1};&\chi_0\chi_i\mid_{\Z_p^\times}=1;\\
(1-p^{-1})^{-1}p^{r(m-k-\frac{1}{2})}\chi_i(p)^{-r}\tau(\chi_0\chi_i,\psi);&{\rm cond}(\chi_0\chi_i)=r>0
\end{array}\right.
\end{eqnarray*}
Since $e_p(\pi_p,\chi_0)=L_p(m-k+1/2,\pi_p,\tilde\chi_0)^{-1}\cdot I_\delta$ and 
\begin{eqnarray*}
L_p(s,\pi_p,\tilde\chi_0)&=&\left\{\begin{array}{lc}L(s,\tilde\chi_0\chi_i)\cdot L(s,\tilde\chi_0\chi_j),&\pi_p=\pi(\chi_i,\chi_j),\\
L(s,\tilde\chi_0\chi_i),&\pi_p=\sigma(\chi_i,\chi_j),\end{array}\right.
\end{eqnarray*}
part $(i)$ follows.



In case $(ii)$ we have that $\Psi=v\cdot|\cdot|^{1/2}\chi_i$, hence
we compute 
\begin{eqnarray*}
I_\delta&=&\sum_nnp^{n(k-m-\frac{1}{2})}\chi_i(p)^n\int_{\Z_p^\times}\chi_0(x)\chi_i(x)\psi(-p^nx)d^\times x\\
&=&\left\{\begin{array}{ll}
\sum_{n\geq0}np^{n(k-m-\frac{1}{2})}\chi_i(p)^n+(1-p^{-1})^{-1}p^{m-k-\frac{1}{2}}\chi_i(p)^{-1};&\chi_0\chi_i\mid_{\Z_p^\times}=1;\\
-r(1-p^{-1})^{-1}p^{r(m-k-\frac{1}{2})}\chi_i(p)^{-r}\tau(\chi_0\chi_i,\psi);&{\rm cond}(\chi_0\chi_i)=r>0
\end{array}\right.\\
&=&\left\{\begin{array}{ll}
\frac{p^{k-m-\frac{1}{2}}\chi_i(p)+p^{m-k-\frac{1}{2}}\chi_i(p)^{-1}-2p^{-1}}{(1-p^{-1})(1-p^{k-m-\frac{1}{2}}\chi_i(p))^2};&\chi_0\chi_i\mid_{\Z_p^\times}=1;\\
-r(1-p^{-1})^{-1}p^{r(m-k-\frac{1}{2})}\chi_i(p)^{-r}\tau(\chi_0\chi_i,\psi);&{\rm cond}(\chi_0\chi_i)=r>0,
\end{array}\right.
\end{eqnarray*}
where the second equality follows from the identity $\sum_{n>0}nx^n=x(1-x)^{-2}$.
The result then follows.
\end{proof}




\section{Extremal $p$-adic L-functions}


If $\pi_p=\pi(\chi_1,\chi_2)$ or $\sigma(\chi_1,\chi_2)$ with $\chi_1$ unramified, then the Hecke polynomial $X^2-a_pX+\epsilon(p)p^{k+1}=(x-\alpha)(x-\beta)$, where $\alpha=p^{1/2}\chi_1(p)^{-1}$. This implies that if $\gamma=\alpha$ has small enough valuation, we can always construct a $v(\alpha)$-admissible distribution $\mu_\alpha$. In fact, if $\pi_p=\pi(\chi_1,\chi_2)$ and $\chi_2$ is also unramified, we can sometimes construct a second $v_p(\beta)$-admissible distribution $\mu_\beta$. 

By previous computations, the interpolation property implies that, for any locally polynomial character $\chi=\chi_0(x)x^m\in C_k(\Z_p^\times,\C_p)$,
\[
\int_{\Z_p^\times}\chi d\mu_\alpha=\frac{4\pi }{\Omega_f^+i^m}\cdot e_p(\pi_p,\chi_0)\cdot L\left(m-k+\frac{1}{2},\pi,\chi_0\right),
\]
with
\[
e_p(\pi_p,\chi_0)=\left\{\begin{array}{ll}
(1-p^{-1})^{-1}(1-\epsilon(p)\alpha^{-1} p^{m})(1-\alpha^{-1}p^{k-m});&\chi_0\chi_2\mid_{\Z_p^\times}=1;\\
(1-p^{-1})^{-1}p^{rm}\alpha^{-r}\tau(\chi_0\chi_2,\psi);&{\rm cond}(\chi_0\chi_2)=r>0.
\end{array}\right.
\]
This interpolation formula coincides (up to constant) with the classical interpolation formula of the distribution $\mu_{f,\alpha}^+$ defined in \S \ref{classicdist}. Indeed, it is easy to prove that $\varphi^+_{f_\alpha}$ is proportional to $\varphi_{f,p}^+(V_0)$ (see equation \eqref{UpV0}), hence the fact that $\mu_{f,\alpha}^+$ is proportional to $\mu_{\alpha}$ follows from \eqref{eqclassmu}, \eqref{defmugen} and property \eqref{keyprop}. 
%In case that $\pi_p=\sigma(\chi_1,\chi_2)$ with $\chi_1$ unramified, one can similarly relate $\mu_{\alpha}$ with $\mu_{f,p}^+$. 
In fact, if $\Psi$ is a character, all the the admissible $p$-adic distributions constructed in this paper are twists of the $p$-adic distributions described in \S \ref{classicdist} (also in \cite{MTT86}), hence for those situations we only provide a new interpretation of classical constructions.

The only genuine new construction is for the case $\Psi=v\cdot|\cdot|^{1/2}\chi$ and $\pi_p=\pi(\chi,\chi)$.
\begin{theorem}\label{mainthm}
Let $f\in S_{k+2}(\Gamma_1(N),\epsilon)$ be a newform, and assume that $\pi_p=\pi(\chi,\chi)$. Then there exists a $(k+1)/2$-admissible distribution $\mu_{f,p}^{\rm ext}$ of $\Z_p^\times$ such that, for any locally polynomial character $\chi=\chi_0(x)x^m\in C_k(\Z_p^\times,\C_p)$,
\[
\int_{\Z_p^\times}\chi d\mu_{f,p}^{\rm ext}=\frac{4\pi }{\Omega_f^+i^m}\cdot e_p^{\rm ext}(\pi_p,\chi_0)\cdot L\left(m-k+\frac{1}{2},\pi,\chi_0\right),
\]
with
\[
e_p^{\rm ext}(\pi_p,\chi_0)=\left\{\begin{array}{ll}
\frac{p^{k-m-\frac{1}{2}}\chi(p)+p^{m-k-\frac{1}{2}}\chi(p)^{-1}-2p^{-1}}{1-p^{-1}};&\chi_0\chi\mid_{\Z_p^\times}=1;\\
\frac{-rp^{r(m-k-\frac{1}{2})}\chi(p)^{-r}\tau(\chi_0\chi,\psi)}{1-p^{-1}};&{\rm cond}(\chi_0\chi)=r>0.
\end{array}\right.
\]
\end{theorem}
\begin{proof}
The only thing that is left to prove is that $\mu_{f,p}^{\rm ext}$ is $(k+1)/2$-admissible, but this follows directly from Theorem \ref{thmadm} and the fact that 
\[
\varepsilon_p=\epsilon_p^{-1}|\cdot|^k=\chi^2,\qquad\gamma=\chi(p)p|p|^{\frac{1}{2}}\varepsilon_p(p)^{-1}=\chi(p)p^{\frac{1}{2}+k}\epsilon_p(p).
\]
Hence $v_p(\gamma)=\frac{1}{2}+k+v_p(\chi(p))=\frac{k+1}{2}$.
\end{proof}


%Coleman and Edixhoven showed in \cite{ColEd} that it is impossible to have $\pi_p=\pi(\chi,\chi)$ if $k = 0$. Moreover, they showed that the property $\pi_p=\pi(\chi,\chi)$ for $k>0$ contradicts Tate's conjecture. Since we believe in Tate's conjecture, we expect $\pi_p=\pi(\chi,\chi)$ never occur and that is the reason why 


\begin{definition}
We call $\mu_{f,p}^{\rm ext}$ \emph{extremal $p$-adic measure}.
Since $(k+1)/2<k+1$, by Proposition \ref{propext} we can extend $\mu_p^{\rm ext}$ to a locally analytic measure. Hence we define the \emph{extremal $p$-adic L-function}
\[
L_p^{\rm ext}(f,s):=\int_{\Z_p^\times}{\rm exp}(s\cdot{\rm log(x)})d\mu_{f,p}^{\rm ext}(x).
\]
\end{definition}

%In the situation $\epsilon=1$, we have $\alpha=p^{1/2}\chi(p)^{-1}=p^{1/2+k}\chi(p)$. Therefore 
%\[
%e_p^{\rm imp}(\pi_p,\chi_0)=\left\{\begin{array}{ll}
%\frac{p^{k-m}+p^{m-k-1}-2 p^{(k-1)/2}}{p^{(k+1)/2}(1-p^{-1})};&\chi_0\chi\mid_{\Z_p^\times}=1;\\
%\frac{-rp^{r(m-k-1)}\alpha^{r}\tau(\chi_0\chi,\psi)}{1-p^{-1}};&{\rm cond}(\chi_0\chi)=r>0.
%\end{array}\right.
%\]


Hence, we conclude that in the conjecturally impossible situation that $\pi_p=\pi(\chi,\chi)$, two $p$-adic L-functions coexist
\[
L_p(f,s),\qquad L_p^{\rm ext}(f,s).
\] 
their corresponding interpolation properties look similar but they have completely different Euler factors. 

\subsection{Alternative description}

In the classical setting described in \S \ref{Classical} ($\chi$ unramified), the $p$-adic distribution $\mu_\alpha$ is given by Equation \eqref{eqclassmu}, while the extremal $p$-adic distribution satisfies
\begin{eqnarray*}
\int_{U(a,n)}P\left(1,\frac{x-a}{p^n}\right)d\mu^{\rm ext}_{f,p}(x)&=&\varphi_{f,p}^+(\delta(1_{U(a,n)}))(0-\infty)\left(P\left(X,\frac{Y-aX}{p^n}\right)\right)\\
&=&\frac{1}{\alpha^n}\cdot\varphi^+_{f,p}(V_1-nV_0)\left(\frac{a}{p^n}-\infty\right)\left(P\right),%\\
%&=&\frac{1}{\alpha_p^n}\varphi^+_{f_\alpha}\left(\frac{a}{p^n}-\infty\right)(P)
\end{eqnarray*}
where $V_0=(1-p^{-1})^{-1}|y|^{1/2}\chi(y)1_{\Z_p}(y)$ and $V_1=(1-p^{-1})^{-1}v(y)|y|^{1/2}\chi(y)1_{p\Z_p}(y)$. Using the relations \eqref{eqKir}, we  compute the action of the Hecke operator $T_p$ on $V_0+V_1$:
\begin{eqnarray*}
T_p (V_0+V_1)&=&\left(\begin{array}{cc}p^{-1}&\\&1\end{array}\right)(V_0+V_1)+\sum_{c\in\Z/p\Z}\left(\begin{array}{cc}1&p^{-1}c\\&p^{-1}\end{array}\right)(V_0+V_1)\\
%&=&(V_0+V_1)(p^{-1}y)+\frac{1}{\varepsilon_p(p)}\sum_{c\in\Z/p\Z}\left(\begin{array}{cc}p&\\&1\end{array}\right)\left(\begin{array}{cc}1&p^{-1}c\\&1\end{array}\right)(V_0+V_1)(y)\\
%&=&(V_0+V_1)(p^{-1}y)+\frac{1}{\varepsilon_p(p)}\sum_{c\in\Z/p\Z}\left(\begin{array}{cc}1&p^{-1}c\\&1\end{array}\right)(V_0+V_1)(py)\\
&=&(V_0+V_1)(p^{-1}y)+\frac{1}{\varepsilon_p(p)}(V_0+V_1)(py)\sum_{c\in\Z/p\Z}\psi(cy)\\
&=&\frac{\alpha|y|^{1/2}\chi(y)}{(1-p^{-1})}\left(v(y)1_{\Z_p}(p^{-1}y)+\frac{1+v(py)}{p}\sum_{c\in\Z/p\Z}\psi(cy)1_{\Z_p}(py)\right)\\
&=&\frac{|y|^{1/2}\chi(y)}{(1-p^{-1})}2\alpha\left(1+v(y)\right)1_{\Z_p}(y)=2\alpha(V_0+V_1)
\end{eqnarray*}
since $\alpha=\gamma=p^{1/2}\chi(p)^{-1}=\varepsilon_p(p)^{-1}p^{1/2}\chi(p)$.
Similarly,
\begin{equation}\label{UpV0}
U_p V_0=\sum_{c\in\Z/p\Z}\left(\begin{array}{cc}1&p^{-1}c\\&p^{-1}\end{array}\right)V_0=\frac{1}{\varepsilon_p(p)}V_0(py)\sum_{c\in\Z/p\Z}\psi(cy)=\alpha V_0.%\\
%T_p V_1&=&p\left(\begin{array}{cc}p^{-1}&\\&1\end{array}\right)V_1+p\sum_{c\in\Z/p\Z}\left(\begin{array}{cc}1&p^{-1}c\\&p^{-1}\end{array}\right)V_1\\
%&=&pV_1(p^{-1}y)+\frac{p}{\varepsilon_p(p)}V_1(py)\sum_{c\in\Z/p\Z}\psi(cy)=\alpha (V_0+ V_1)
\end{equation}
Hence, $V_0$ and $V_1$ are basis of the generalized eigenspace of $U_p$, in which $V_0$ is the eigenvector and $V_0+V_1$ is the newform. This implies that (up to constant) $\varphi^+_{f,p}(V_0)\stackrel{\cdot}{=}\varphi^+_{f_\alpha}$, where $f_\alpha$ is the p-specialization defined in \S \ref{classicdist}, while we have that $\varphi^+_{f,p}(V_0+V_1)\stackrel{\cdot}{=}\varphi^+_{f}$. We conclude that, in terms of the classical definitions given in \S \ref{classicdist}, the extremal distribution can be described as
\[
\int_{U(a,n)}P\left(1,\frac{x-a}{p^n}\right)d\mu^{\rm ext}_{f,p}(x)=\frac{1}{\alpha^n}\cdot\varphi^+_{f-(n+1)f_\alpha}\left(\frac{a}{p^n}-\infty\right)\left(P\right).
\]




%\begin{eqnarray*}
%U_p V_0&=&p\sum_{c\in\Z/p\Z}\left(\begin{array}{cc}1&p^{-1}c\\&p^{-1}\end{array}\right)V_0=\frac{p}{\varepsilon_p(p)}V_0(py)\sum_{c\in\Z/p\Z}\psi(cy)=\alpha V_0,\\
%U_p V_1&=&\frac{p}{\varepsilon_p(p)}V_1(py)\sum_{c\in\Z/p\Z}\psi(cy)=\alpha (V_0+ V_1)
%\end{eqnarray*}

%\begin{eqnarray*}
%U_p V_0&=&p\sum_{c\in\Z/p\Z}\left(\begin{array}{cc}1&p^{-1}c\\&p^{-1}\end{array}\right)V_0=\frac{p}{\varepsilon_p(p)}\sum_{c\in\Z/p\Z}\left(\begin{array}{cc}p&\\&1\end{array}\right)\left(\begin{array}{cc}1&p^{-1}c\\&1\end{array}\right)V_0(y)\\
%&=&\frac{p}{\varepsilon_p(p)}\sum_{c\in\Z/p\Z}\left(\begin{array}{cc}1&p^{-1}c\\&1\end{array}\right)V_0(py)=\frac{p}{\varepsilon_p(p)}V_0(py)\sum_{c\in\Z/p\Z}\psi(cy)\\
%&=&\varepsilon_p(p)^{-1}p^{1/2}\chi(p)(1-p^{-1})^{-1}|y|^{1/2}\chi(y)\sum_{c\in\Z/p\Z}\psi(cy)1_{\Z_p}(py)=\alpha V_0,
%\end{eqnarray*}
%since $\alpha=\gamma=\varepsilon_p(p)^{-1}p^{1/2}\chi(p)$. Similarly,
%\begin{eqnarray*}
%U_p V_1&=&\frac{p}{\varepsilon_p(p)}V_1(py)\sum_{c\in\Z/p\Z}\psi(cy)\\
%&=&\varepsilon_p(p)^{-1}p^{1/2}\chi(p)(1-p^{-1})^{-1}|y|^{1/2}\chi(y)(1+v(y))\sum_{c\in\Z/p\Z}\psi(cy)1_{\Z_p}(py)\\
%&=&\alpha (V_0+ V_1)
%\end{eqnarray*}


\bibliographystyle{plain}
\bibliography{biblio}
\end{document}
