	\documentclass[12pt]{article}
%\usepackage[portuguese]{babel}
\usepackage[latin1]{inputenc}
%\usepackage{indentfirst}
%\usepackage[active]{srcltx}
\usepackage{amsfonts}
%\usepackage{graphicx}
\usepackage{amsmath}
\usepackage{amssymb}
\usepackage{latexsym}
%\usepackage{amscd}
%\usepackage[all]{xy}


\usepackage[pagebackref]{hyperref}

\newtheorem{theorem}{Theorem}[section]
\newtheorem{lemma}[theorem]{Lemma}
\newtheorem{proposition}[theorem]{Proposition}
%\newtheorem{corollary}[theorem]{Corollary}
\newtheorem{conjecture}[theorem]{Conjecture}
\newtheorem{problem}[theorem]{Problem}
\newtheorem{construction}[theorem]{Construction} 
\newtheorem{claim}[theorem]{Claim}
\newtheorem{addendum}[theorem]{Addendum}
\newtheorem{defin}[theorem]{Definition}

\newenvironment{proof}{\noindent \textbf{Proof: }}{\hfill
$\Box$  \vspace{1ex}}
\newenvironment{definition}{\begin{defin}\em}{\end{defin}}
\newtheorem{defins}[theorem]{Definitions}
\newenvironment{definitions}{\begin{defins}\em}{\end{defins}}
\newtheorem{exs}[theorem]{Examples}
\newenvironment{examples}{\begin{exs}\em}{\end{exs}}
\newtheorem{ex}[theorem]{Example}
\newenvironment{example}{\begin{ex}\em}{\end{ex}}
\newtheorem{rem}[theorem]{Remark}
\newenvironment{remark}{\begin{rem}\em}{\end{rem}}
\newtheorem{rems}[theorem]{Remarks}
\newenvironment{remarks}{\begin{rems}\em}{\end{rems}}
\newtheorem{corollary}[theorem]{Corollary}
\newtheorem{esc}[theorem]{Escolium}




\setlength{\textwidth}{15cm} \setlength{\textheight}{20cm}
\setlength{\topmargin}{-0.5cm} \setlength{\oddsidemargin}{1cm}
\setlength{\evensidemargin}{1cm}
\renewcommand{\baselinestretch}{1.3}    %DIST\^{A}NCIA ENTRE LINHAS






%\newcommand{\cO}{\mathcal O}
%\newcommand{\cX}{\mathcal X}
%\newcommand{\cJ}{\mathcal J}
%\newcommand{\cD}{\mathcal D}
%\newcommand{\cK}{\mathcal K}
%
%\newcommand{\bE}{\mathbf {E}}
%
\newcommand{\F}{\mathbb{F}}
%\newcommand{\Z}{\mathbb {Z}}
%\newcommand{\D}{\mathbb {D}}
%\newcommand{\ele}{\mathbb {l}}
%\newcommand{\bP}{\mathbb {P}}
\newcommand{\N}{\mathbb{N}}

\def\Bb {\mathcal{B}}

\def\C{\mathbb{C}}

\def\Z{\mathbb{Z}}

\def\bchi{\boldsymbol{\chi}}

%\setcounter{page}{1}

%\newtheorem{theorem}{Theorem}[section]
%\newtheorem{definition}[theorem]{Definition}

%\newtheorem{lemma}[theorem]{Lemma}
%\newtheorem{corollary}[theorem]{Corollary}
%\newtheorem{proposition}[theorem]{Proposition}
%\newtheorem{remark}[theorem]{Remark}
%\newtheorem{example}[theorem]{Example}
%\newtheorem{notation}[theorem]{Notation}
%\newenvironment{Proof}{\noindent{\bf Proof} }{\hfill $\square$}

\def \disc {\rm{disc}\,}
\def \mdc {\rm{mdc}\,}
\def \ord {\rm{ord}\,}
\def \Ord {\rm{Ord}\,}


\begin{document}

\begin{center}
{\Large\textbf{On the existence of pairs of primitive and normal 
elements over finite fields}}
\end{center}
\vspace{3ex}

\noindent\begin{center} 
\textsc{C. Carvalho, J.P. Guardieiro, V.G.L. Neumann and G. Tizziotti}\\ 
\vspace{1ex}
\small{Faculdade de Matem\'{a}tica, Universidade Federal de Uberl\^{a}ndia, 
Av. 
J. N. 
\'{A}vila  2121, 38.408-902 Uberl\^{a}ndia -- MG, Brazil }
\end{center}
%
%\title{On existence of some special pair of primitive and normal elements over 
%finite fields}
%
%\author{C. Carvalho, J.P. Guardieiro Sousa, V. Neumann and G. Tizziotti}
%
%%\maketitle
%
%\begin{abstract}
%
%\end{abstract}



\vspace{8ex}
\noindent
\textbf{Keywords:} Primitive element, normal element, normal basis, finite 
fields.\\
\noindent
\textbf{MSC:} 12E20, 11T23



\vspace{4ex}
\begin{small}
\begin{center}
\textbf{Abstract}
\end{center}
Let $\F_{q^n}$ be a finite field with $q^n$ elements, and let $m_1$ and $m_2$ 
be 
positive integers. Given polynomials  $f_1(x), f_2(x) \in \F_q[x]$ with 
$\deg(f_i(x)) \leq m_i$, for $i = 1, 2$, and such that the rational function 
$f_1(x)/f_2(x)$ 
belongs to a certain set which we define, we present a sufficient condition for 
the  existence of a primitive element $\alpha \in \F_{q^n}$, normal over 
$\F_q$, 
such that 
$f_1(\alpha)/f_2(\alpha)$ is also primitive.  
\end{small}


\section{Introduction}
Let $\F_q$ be a finite field with $q$ elements. 
An element $\alpha\in\mathbb{F}_q$ is called \textit{primitive} if it is a 
generator of the multiplicative group $\mathbb{F}_q^*$. Let $n$ be a positive 
integer, and element $\beta \in \mathbb{F}_{q^n}$ is called \textit{normal over 
$\F_q$} if the set $\{\beta, \beta^q, \ldots, \beta^{q^{n - 1}} \}$ 
is a basis for 
$\mathbb{F}_{q^n}$ as an  $\mathbb{F}_q$-vector space. The primitive normal 
basis theorem states that for any $q$ and $n$ there exists an element in 
$\mathbb{F}_{q^n}$ which is simultaneously primitive and normal over $\F_q$. 

In their proof of this theorem (see \cite{CH}), Cohen and Huczynska developed a 
technique which 
has, since then, been adapted to treat other problems involving primitive and 
normal elements. For example, these same authors used a modified version of 
their technique to prove the strong normal basis theorem (see \cite{CH2}), 
which states that, 
except for a few pairs $(q,n)$, one can find an element $\alpha \in      
\mathbb{F}_{q^n}$  such that $\alpha$ and $\alpha^{-1}$ are primitive and 
normal over $\F_q$. Later, using the same line of reasoning, Kapetanakis (see 
\cite{kapeta})  proved 
that there exists 
an element $\alpha \in      
\mathbb{F}_{q^n}$  such that $\alpha$ and $(a \alpha + b)/(c \alpha + d)$, with 
$a,b,c,d \in  \F_q$, are 
primitive and 
normal over $\F_q$, except for a few combinations of $q$, $n$ and $a,b,c,d$. 
In 2017 Anju and Sharma, also following ideas from \cite{CH} and assuming that 
$q$ has characteristic two, proved that given polynomials $f(x), g(x) \in 
\F_{q^n}[x]$,  being $f(x)$ of degree at most 2 and $g(x)$ of degree at most 1, 
then there exists $\alpha \in \F_{q^n}$, primitive and normal over $\F_q$, such 
that $f(\alpha)/g(\alpha)$ is also primitive, except for a few combinations of 
$q = 2^k$, $m$, $f(x)$ and 
$g(x)$ (see \cite{mersenne}).  
More recently, Hazarika, Basnet and Cohen (\cite{HBC}) studied this problem 
working with a field of characteristic three and considering polynomials    
of degree at most two instead of rational functions. Hazarika and Basnet 
(\cite{HB})  also 
considered the related problem of finding pairs of elements $(\alpha, 
f(\alpha))$, both in $\F_{q^n}$, and both being primitive and normal over 
$\F_q$, where $f$ is a quotient of a polynomial of degree two by a polynomial 
of degree at most one, and $q$ has characteristic two.

In this paper we work with a finite field of any characteristic, and given 
polynomials 
$f_1(x), f_2(x) \in  \F_{q^n}$, of any degree,  we study, like Anju and Sharma, 
the 
existence of $\alpha \in \F_{q^n}$, primitive and normal over $\F_q$, such that
$f_1(\alpha)/f_2(\alpha)$ 
is also primitive. More specifically, given positive 
integers $m_1$ and $m_2$ we determine a set $\Upsilon_q(m_1, m_2)$ (see 
Definition \ref{upsilonset}) comprising 
certain rational functions $f(x)/g(x)$, where $\deg(f_i(x)) \leq m_i$, with $i 
= 1,2$,  and we determine conditions which assure, for each $f(x)/g(x) \in 
\Upsilon_q(m_1, m_2)$,  the existence of an element   
$\alpha \in \F_{q^n}$, primitive and normal over $\F_q$, such 
that $f(\alpha)/g(\alpha)$ is also primitive (see Corollary \ref{mainresult}).

In the following section we list the definitions and results which will be used 
in the proof of the main result, which is the content of Section 3. In Section 
4 we 
present several numerical examples  illustrating our main result.


\section{Preliminaries}


%\subsection{Primitive and normal elements} An element $\alpha\in\mathbb{F}_q$ 
%is called \textit{primitive} if $\alpha$ is a generator of the multiplicative 
%group
%$\mathbb{F}_q^*$, or equivalently, if the multiplicative order of $\alpha$ is 
%$q-1$. A pair $(\alpha,\beta) \in \mathbb{F}_q^2$ is a \textit{primitive pair} 
%in $\mathbb{F}_q$ if $\alpha$ and $\beta$ are primitive elements. It is clear 
%that $(\alpha,\beta) \in \mathbb{F}_q^2$ is a primitive pair if and only if 
%$(\alpha,\beta^{-1}) \in \mathbb{F}_q^2$ is a primitive pair. An element 
%$\alpha\in \F_{q^n}$ is called a \textit{normal element} of $\F_{q^n}$ over 
%$\F_q$ if
%$\{ \alpha, \alpha^q, \ldots , \alpha^{q^{n-1}}\}$ is a basis of $\F_{q^n}$ 
%over $\F_q$. In this case
%a basis of this form is called  a \textit{normal basis}.

 We define sets that will play an important  role in what 
follows.
Throughout this paper $p$ is a prime, $k$ is a positive integer, 
$\mathbb{F}_q$ will denote a finite field with $q=p^k$ elements and we denote
by $\mathbb{N}$ the set of positive integers.


\begin{definition}  \label{upsilonset}
Let $f_1, f_2 \in \mathbb{F}_q [x]$ and $m_1,m_2 \in \mathbb{N}$.  We define  
$\Upsilon_{q} (m_1,m_2)$ as the set of rational functions 
$\frac{f_1}{f_2} \in \mathbb{F}_q(x)$
such that:
\begin{enumerate}  
\item[i)] 
$\deg (f_1) \leq m_1$,   $\deg (f_2) \leq m_2$; 
\item[ii)]
$\gcd(f_1,f_2)=1$;  
\item[iii)]
 there exists $n \in \mathbb{N}$ and an irreducible monic polynomial $g \in 
\F_q[x]$ such that $\gcd(n, q - 1)=1$, $g^n  \mid  f_1 f_2$ and 
$g^{n+1} \nmid f_1 f_2$. 
\end{enumerate}
%b) We define $\Bb_p(m_1,m_2)$ as the set of pairs $(k,n) \in \mathbb{N}^2$ 
%such 
%that 
%$\mathbb{F}_{q^n}$ contains an element 
%$\alpha$ normal over $\mathbb{F}_q$ with $\alpha$ and $f(\alpha)$ 
%primitive elements for all $ f \in \Upsilon_{q^n} (m_1,m_2)$.
\end{definition}
%
%
%i) We define  $\Lambda_q(f_1,f_2)$ as the set of pairs $(n,g) \in \mathbb{N} 
%\times (\mathbb{F}_q [x]
%\setminus \{x\})$ such that $\gcd(n,q-1) = 1$, $g$ is monic, irreducible, $g^n 
%\mid f_1 f_2$ and  $g^{n+1} \nmid f_1 f_2$. \\
%iii)We define $\Gamma_p(m_1,m_2)$ as the set of positive integers $k$ such that
%$\mathbb{F}_{p^k}$  contains an element $\alpha$ with $(\alpha, f(\alpha))$  a 
%primitive pair  for all $f\in \Upsilon_{p^k} (m_1,m_2)$.\\

%$$
%\Bb_p(m_1,m_2)=
%\left\{
%(k,n) \mid
%\begin{array}{l}
%\exists \alpha \in \F_{q^n} \text{ such that }
%\alpha \text{ is normal over } \F_q  \\
%\text{and }
%\left(\alpha , f(\alpha) \right)
%\text{ is a primitive pair}  \\
%\text{for all } f \in \Upsilon_{q^n} (m_1,m_2), \text{ where } q=p^k
%\end{array}
%\right\}.
%$$
%\end{definition}

%
%In \cite{mersenne}, the authors studied a set similar to $\Bb_2(2,2)$.
	

\begin{definition}
Let  $s$ by a divisor of $q-1$, an element $\alpha\in\mathbb{F}_q^*$
is called \textit{$s$-free} if, for any $d \in \mathbb{N}$ such that $d\, |\, s$ and $d \neq 1$, there is no $\beta\in\mathbb{F}_q$ satisfying $\beta^d = \alpha$.
\end{definition}

%
%
%Thus, if $\alpha \in\mathbb{F}_q^*$ is $s$-free then  $\alpha$ cannot be a 
%$d$-th power of an element, where $d\neq1$ and $d|s$. Next, we present some 
%easy remarks that we will use in what follows.
%
%
%\begin{remark} \label{obs s_free}
%Let $\alpha\in\mathbb{F}_q^*$, then:
%
%\begin{enumerate}
%
%\item $\alpha$ is primitive if and only if $\alpha$ is $(q-1)$-free;
%
%\item if $\alpha$ is $s$-free for some integer $s$ then $\alpha$ is $e$-free 
%for any $e \, | \, s$;
%
%\item if $\alpha$ is $s_1$-free and $s_2$-free, then $\alpha$ is 
%$\mathrm{lcm}(s_1,s_2)$-free;
%
%\item let $p_1, \ldots, p_n$ be primes, then $\alpha$ is $(p_1 .\cdots. 
%p_n)$-free if and only if $\alpha$ is  $(p_{1}^{\alpha_1} .\cdots. 
%p_{n}^{\alpha_n})$-free, where $\alpha_i > 0$ for all $i$.
%
%
%\end{enumerate}
%
%\end{remark}
%

%\subsection{Characters} 

For $\beta \in \mathbb{F}_{q^n}$ and $\displaystyle 
f(x) = \sum_{i=1}^{t} f_i x^i\in \F_q[x]$, we define an action of 
$\mathbb{F}_q[x]$ over $\mathbb{F}_{q^n}$ by $\displaystyle f \circ \beta = 
\sum_{i=0}^{t} f_i \beta^{q^i}$. 
Through this action $\mathbb{F}_{q^n}$ may be viewed as an $\mathbb{F}_q[x]$-module, and the
annihilator of $\beta$ is an ideal of  $\mathbb{F}_q[x]$.
 The unique monic generator  $g$ of this ideal is the order of $\beta$, denoted by $\mbox{Ord}[\beta]$.
 Observe that clearly $g$ is a factor of $x^n - 1$.
One may prove that if $\mbox{Ord}[\beta]$ is $g$, then $\beta = h \circ \lambda$ for some $\lambda \in \mathbb{F}_{q^n}$, where $h=\frac{x^n-1}{g}$. 

Similarly  to the concept of being an $s$-free element for any $s|q^n-1$, we 
can also define what is to be a $g$-free element for any $g \in 
\mathbb{F}_q[x]$ that divides $x^n-1$.

\begin{definition} \label{g-free}
Let $g \in \mathbb{F}_q[x]$ be such that $g|x^n-1$. An element $\alpha \in 
\mathbb{F}_{q^n}$ is said to be {\em $g$-free} if for any $h \in 
\mathbb{F}_q[x]$ such that $h|g$ and $\lambda \in \mathbb{F}_{q^n}$ we have 
that   $\alpha = h \circ \lambda$ implies $h=1$.
\end{definition}

%\begin{remark}\marginpar{retirar esse e os proximos 2 paragrafos} \label{obs 
%normal} 
%It is easy to check that 
%$\alpha \in \F_q$ is primitive if and only if $\alpha$ is $(q-1)$-free, and 
%that $\beta \in \mathbb{F}_{q^n}$ is normal over $\mathbb{F}_ q$ if and only 
%if $\beta$ is $(x^n - 1)$-free.
%\end{remark}
%
%Let $\mathbb{F}_{q^n}^{\wedge}$ be the set of multiplicative characters of 
%$\mathbb{F}_{q^n}^*$. The \textit{order of a character} $\chi \in 
%\mathbb{F}_{q^n}^{\wedge}$, denoted by $ord(\chi)$, is the smallest positive 
%integer $d$ such that $\chi^d = \chi_0$, where $\chi_0$ is the trivial 
%character.
%
%\begin{remark} \label{obs ordem}
%	By \cite[Th. 1.15]{LN} we know that there are $\phi(d)$ characters of 
%$\mathbb{F}_{q^n}^*$ of order $d$, where $\phi$ is the Euler's function.
%\end{remark}

%For any character $\chi \in \mathbb{F}_{q^n}^{\wedge}$, $f \in 
%\mathbb{F}_q[x]$, and $\beta \in \mathbb{F}_{q^n}$ we define an action of 
%$\mathbb{F}_q[x]$ over $\mathbb{F}_{q^n}^{\wedge}$ by $\chi \circ f(\beta) = 
%\chi(f \circ \beta)$. The \textit{$\mathbb{F}_q$-order of a character $\chi 
%\in 
%\mathbb{F}_{q^n}^{\wedge}$}, denoted by $\Ord{(\chi)}$, is defined to be a 
%unique monic polynomial $g \in \mathbb{F}_q [x]$ of the least degree dividing 
%$x^n-1$ such that $\chi \circ g$ is the trivial character in 
%$\mathbb{F}_{q^n}$.
%There are $\Phi(g)$ characters $\chi \in \mathbb{F}_{q^n}^{\wedge}$ of 
%$\mathbb{F}_q$-order $g$, where $\Phi(g):= |(\mathbb{F}_q[x] / g 
%\mathbb{F}_q[x])|^{\ast}$ is the analogue Euler's function on 
%$\mathbb{F}_q[x]$ 
%(see \cite{CH}). % pg 44

%
%For more details about characters of finite fields see \cite[Chapter 5]{LN}. 
%(** Pra que essa frase?)
%


From \cite[Section 3]{CH} we know that the characteristic function of the 
set $\alpha \in \F_{q^n}$ of $s$-free elements, with $s \mid q^n - 1$, is given 
by
\begin{equation} \label{funcao caracteristica}
\rho_s(\alpha) = 
\theta(s)\sum_{d|s}\frac{\mu(d)}{\phi(d)}\sum_{\chi_d}\chi_d(\alpha),
\end{equation}
where $\theta(s):=\frac{\phi(s)}{s}$, $\mu$ is the Moebius's function and 
$\chi_d$ runs through the set of $\phi(d)$ multiplicative characters of 
$\mathbb{F}_{q^n}^*$ of order 
$d$.


We endow the group of additive characters of 
$\mathbb{F}_{q^n}$ with an structure of  $\mathbb{F}_q[x]$-module by means of  
the 
operation which combines a polynomial $f$ and a character $\psi$ to produce the 
character $\psi \circ f$ defined by $\psi \circ f(\beta) = 
\chi(f \circ \beta)$ for all  $\beta \in \mathbb{F}_{q^n}$.
The \textit{$\mathbb{F}_q$-order of an additive character $\psi$}, denoted by 
$\Ord{(\chi)}$, is defined to be a 
unique monic polynomial $g \in \mathbb{F}_q [x]$ of least degree dividing 
$x^n-1$ such that $\chi \circ g$ is the trivial character in $\mathbb{F}_{q^n}$.
There are $\Phi(g)$ additive characters of 
$\mathbb{F}_q$-order $g$, where $\Phi(g):= |(\mathbb{F}_q[x] / g 
\mathbb{F}_q[x])|^{\ast}$ is the analogue Euler's function on 
$\mathbb{F}_q[x]$. 
%(see \cite{CH}). % pg 44
Also in \cite[Section 3]{CH} we find the expression for the characteristic  
function 
for the set of $g$-free elements $\alpha \in \mathbb{F}_{q^n}$.
	For any $g \in \F_{q}[x]$ such that $g \mid x^n -1$,
	this characteristic function $\kappa_{g}$
	is given by 
	\begin{equation} \label{char-g-free}
	\kappa_g(\alpha) = \frac{\Phi(g)}{N(g)}
	\sum_{h \mid g} \frac{\mu'(h)}{\Phi(h)}
	\sum_{\Ord (\psi)=h} \psi(\alpha),
	\end{equation}
	where
%	$\Phi(g) = \left|   \left(  \F_{q}[x] / g\F_{q}[x] \right)^* \right|$
%	is the analogue of Euler's function on $\F_{q}[x]$,
	$N(g) = \left|   \left(  \F_{q}[x] / g\F_{q}[x] \right) 
	\right|=q^{\deg(g)}$,
	the last sum runs over all additive characters $\psi$ of $\mathbb{F}_q^n$ 
	which have   
	$\F_{q}$-order $h$, and $\mu'$ is the M\"obius function on $\F_{q}[x]$
$$
\mu'(h) =
\left\{
\begin{array}{ll}
(-1)^s & {\textrm if} \,\, h \mbox{ is is a product of } s \mbox{ disctinct 
monic 
irredutible polynomials;} \\
0 & \rm{ otherwise. }
\end{array}
\right.
$$

The next result is a combination of \cite[Theorem 5.5]{Fu} and a special case 
of \cite[Theorem 5.6]{Fu}, which we will need in what follows.

\begin{lemma} \label{lema cota}
Let $v(x),u(x) \in \F_{q^n}(x)$ be rational functions. Write
	$v(x)=\prod_{j=1}^k s_j(x)^{n_j}$, where
	$s_j(x) \in \F_{q^n}[x]$ are irreducible polynomials, pairwise 
	non-associated,  and
	$n_j$ are non-zero integers. Let $D_1=\sum_{j=1}^k \deg (s_j)$,
	let $D_2=\max (\deg (u), 0)$, let $D_3$ be the degree of the denominator of 
	$u(x)$
	and let $D_4$ be the sum of degrees of those irreducible polynomials 
	dividing the denominator of $u$,
	but distinct from $s_j(x)$ ($j=1,\ldots, k$). Let $\chi$ and $\psi$ be, 
	respectively, a multiplicative character and 
	a non-trivial additive character of $\F_{q^n}$.\\
a) Assume that $v(x)$ is not of the form $r(x)^{ord(\chi)}$
	in $\mathbb{F}(x)$, where $\mathbb{F}$ is an algebraic closure of 
	$\mathbb{F}_{q^n}$. Then 
$$
	\displaystyle \left|
	\sum_{\substack{\alpha \in \mathbb{F}_{q^n}  \\ v(\alpha)\neq 0 , 
	v(\alpha)\neq
			\infty}} \chi(v(\alpha)) \right|
	\leq
	(D_1-1 ) q^{\frac{n}{2}}.
$$\\
b) Assume that
	$u(x)$ is not of the form $r(x)^{q^n}-r(x)$ in $\F(x)$, where $\mathbb{F}$ 
	is an algebraic closure of $\mathbb{F}_{q^n}$. Then
$$
	\displaystyle \left|
	\sum_{\substack{\alpha \in \mathbb{F}_{q^n}  \\ v(\alpha)\neq 0, 
	v(\alpha)\neq \infty , \\
	u(\alpha)\neq \infty}} \chi(v(\alpha)) \psi(u(\alpha)) \right|
	\leq
	\left( D_1 + D_2 + D_3 + D_4 - 1  \right) q^{\frac{n}{2}}.
$$	
\end{lemma}	


\section{Main results}\label{section-main}

Let $m_1$ and $m_2$ be positive integers, we want to determine 
conditions on $q$ and $n$ such that
for each $f \in \Upsilon_{q^n} (m_1,m_2)$
there exists $\alpha \in \mathbb{F}_{q^n}$, primitive and normal
over $\mathbb{F}_q$, such that $f(\alpha) \in \mathbb{F}_{q^n}$  is also a 
primitive element.	
For this we will need the following concept.
\begin{definition} 	\label{upsilon}
Let $q=p^k$, let
$e_1$ and $e_2$ be divisors of $q^n-1$ and let $g$ be a divisor of $x^n-1$.
Given $f \in \Upsilon_{q^n}(m_1,m_2)$
we will denote by $N_{f}(e_1,e_2,g)$ the number of $\alpha \in \F_{q^n}$ such that
$\alpha$ is $e_1$-free, $f(\alpha)$ is $e_2$-free and
$\alpha$ is $g$-free.
\end{definition}

It is easy to check that 
$\alpha \in \F_{q^n}$ is primitive if and only if $\alpha$ is $(q^n-1)$-free, 
and 
that $\beta \in \mathbb{F}_{q^n}$ is normal over $\mathbb{F}_q$ if and only 
if $\beta$ is $(x^n - 1)$-free. We want to find conditions which assure that
$N_{f}(q^n-1,q^n-1,x^n-1)>0$ for all
$f \in \Upsilon_{q^n} (m_1,m_2)$, yet in the next result we deal with a 
slightly more general situation. Before stating it, we observe that when $n = 
1$ or $n = 2$ then every primitive element in $\mathbb{F}_{q^n}$ is normal over 
$\mathbb{F}_q$, so we may ignore the ``normal''  requirement and  the problem 
was already solved in \cite{CSS}. Thus we assume from now on that $n \geq 
3$. 



For $\ell \in \mathbb{N}$ we denote by $W(\ell)$ the number of 
distinct square-free  divisors of $\ell$, and
for a polynomial $g \in \F_q[x]$ we denote by
$W_q(g)$ the number
of monic square free factors of $g$ in $\F_{q}[x]$. 


\begin{theorem}\label{principal}
	Let  $e_1$ and $e_2$ be divisors of   $q^n-1$, let 
	 $g \in \F_q[x]$ be a factor of $x^n -1$ and let $f \in \Upsilon_{q^n} 
	 (m_1,m_2)$. Then
	$$
	N_{f}(e_1,e_2,g) > \frac{\phi(e_1)\phi(e_2) \Phi(g)}{e_1 e_2 N(g)}
	\left[
	q^n -1 - (m_1+m_2+1) q^{n/2} \left(W(e_1)W(e_2)W_q(g) - 1 \right)
	\right] ,
	$$
	and a fortiori  
	if $q^{n/2} \geq (m_1 + m_2 + 1) W(e_1)W(e_2)W_q(g)$ then 
	$N_{f}(e_1,e_2,g)>0$.
\end{theorem}
\begin{proof}
Let $f = \dfrac{f_1}{f_2} \in \Upsilon_{q^n}(m_1,m_2)$
%, with
%	$\deg f_1 = \widetilde{m}_1 \leq m_1$ and $\deg f_2=\widetilde{m}_2 \leq m_2$, and let %$e_1,e_2 \mid q^n-1$
%	and $g \in \F_q[x]$ with $g \mid x^n -1$.
%Define
and let
$$
S_{f}:=
\left\{ \alpha \in \F_{q^n} \mid f_1(\alpha)=0 \text{ or } f_2(\alpha)=0  \right\} \cup \{ 0 \}.
$$

From the definition of $N_{f}(e_1,e_2,g)$ and equations \eqref{funcao 
caracteristica} and (\ref{char-g-free})	 we have
\begin{eqnarray*}
N_{f}(e_1,e_2,g) & = &
\sum_{\alpha \in \F_{q^n} \backslash S_{f}}
\rho_{e_1}(\alpha) \rho_{e_2} (f(\alpha)) \kappa_g(\alpha) \nonumber \\
& = &
\frac{\phi(e_1)\phi(e_2) \Phi(g)}{e_1 e_2 N(g)}
\sum_{\substack{d_1\mid e_1 , d_2\mid e_2 \\ h\mid g}}
\frac{\mu(d_1)\mu(d_2) \mu'(h)}{\phi(d_1)\phi(d_2) \Phi(h)}
\sum_{\substack{\ord(\chi_1)=d_1 \\ \ord(\chi_2)=d_2 \\ \Ord(\psi)=h}}
\tilde{\bchi}_f (\chi_1,\chi_2,\psi), %\label{number}
\end{eqnarray*}
where
$$
\tilde{\bchi}_f (\chi_1,\chi_2,\psi)
=
\sum_{\alpha \in \F_{q^n} \backslash S_{f}}
\chi_1(\alpha) \chi_2(f(\alpha)) \psi(\alpha) .
$$



To find a bound for $N_f(e_1,e_2,g)$ we will
bound $| \tilde{\bchi}_f(\chi_{1},\chi_{2},\psi) |$,
and we consider five cases.
\begin{enumerate}
\item[(i)] We first consider the case where $\chi_1$, $\chi_2$ and
$\psi$ are trivial characters, so that
$$
\tilde{\bchi}_f(\chi_{1},\chi_{2},\psi) = \sharp (\mathbb{F}_{q^n} \setminus S_f) \geq
q^n - (m_1+m_2+1).
$$

\item[(ii)] Now we deal with the case where $\chi_1$ and $\chi_2$ are trivial 
multiplicative  characters, while $\psi$ is not a trivial additive character. 
It is well known that 
$\displaystyle \sum_{\alpha \in \mathbb{F}_{q^n}}\psi(\alpha)= 0$, so that
$$
|\tilde{\bchi}_f(\chi_{1},\chi_{2},\psi) | =
\left| \sum_{\alpha\in\F_{q^n}\backslash S_f} \psi (\alpha)
\right| =
\left| - \sum_{\alpha\in S_f} \psi (\alpha)
\right| \leq m_1 + m_2+1.
$$

\item[(iii)] We treat the case where $\chi_1$ is not a trivial character, while 
$\chi_2$ and $\psi$ are trivial characters. It is well known that 
$\sum_{\alpha \in \mathbb{F}_q^*}\chi_1(\alpha)= 0$, so we have 
\begin{eqnarray}
|\tilde{\bchi}_f(\chi_{1},\chi_{2},\psi) | & =&
\left| \sum_{\alpha \in \mathbb{F}_q^*}\chi_1(\alpha)
-
\sum_{\alpha\in\mathbb{F}_{q}\setminus S_f}\chi_1(\alpha)
\right| =
\left| \sum_{\alpha \in S_f\setminus\{0\}}\chi_1(\alpha) \right|
\nonumber \\
& \leq &(m_1+m_2) < (m_1+m_2)q^{\frac{n}{2}}.
\nonumber
\end{eqnarray}
\end{enumerate}


Before proceeding to treat the cases where we assume at most one trivial 
character, we will rewrite the expression for 
$\tilde{\bchi}_f(\chi_{1},\chi_{2},\psi)$. 

Let $\chi_1$ and $\chi_2$ be multiplicative characters of orders $d_1$ and
$d_2$, respectively, where $d_1 \, |\, e_1$ and $d_2 \, | \, e_2$
and let $\psi$ be an additive character of $\F_q$-order $h$.
Let $i \in \{1, 2\}$,
it is
well-known (see e.g. \cite[Thm.\ 5.8]{LN}) that there exists a character
$\chi$  of order $q^n -1$ and and integer
$n_i \in\{0,1,...,q-2\}$ such that $\chi_{i}(\alpha)=\chi(\alpha^{n_i})$
for all $\alpha \in \mathbb{F}_{q^n}^*$, and observe that
$n_i=0$ if and only if $\chi_i$ is a trivial character. Hence,
\begin{eqnarray}
\tilde{\bchi}_f(\chi_{1},\chi_{2},\psi) & = &
\sum_{\alpha\in\F_{q^n}\backslash S_f}
\chi(\alpha^{n_1}f_1(\alpha)^{n_2}f_2(\alpha)^{-n_2})\psi(\alpha)
 \nonumber \\
& = &
\sum_{\alpha\in\F_{q^n}\backslash S_f}\chi(v(\alpha))\psi(\alpha), 
\nonumber
%\label{number2}
\end{eqnarray}
where $v(x)=x^{n_1}f_1(x)^{n_2}f_2(x)^{-n_2}$.

\begin{enumerate}
\item[(iv)] Now we assume that $\psi$ is a trivial additive character, while 
$\chi_2$ is not a trivial multiplicative character, so that $n_2 \neq 0$, and 
we make no assumptions 
on $\chi_1$.
To bound $\tilde{\bchi}_f(\chi_{1},\chi_{2},\psi) $
we want to use Lemma \ref{lema cota} (a), and we
start by showing that indeed we can use it. So we suppose by means of absurd
that $v(x) = \left( \frac{v_1(x)}{v_2(x)} \right)^{q^n-1}$
for some
$v_1(x),v_2(x)\in\mathbb{F}[x]$, with 
%$\textrm{deg}(v_1)=r_1$,
%$\textrm{deg}(v_2)=r_2$, and
$\mbox{gcd}(v_1,v_2)=1$, then
\begin{equation*}
x^{n_1}f_1(x)^{n_2}v_2(x)^{q^n-1}=f_2(x)^{n_2}v_1^{q^n-1}(x).
\end{equation*}
Since $\frac{f_1(x)}{f_2(x)} \in \Upsilon_{q^n}
(m_1,m_2)$, there exists an
irreducible monic polynomial $t(x) \in \mathbb{F}_{q^n}[x]$, $t(x)\neq x$ and a
positive integer $a$
with $\mbox{gdc}\, (a,q^n-1)=1$ such that
$t(x)^a$ is the larges power of $t(x)$ which appears in the factorization of 
either $f_1(x)$ or $f_2(x)$. Let's
suppose that $t(x)^a$ appears in the factorization of $f_2(x)$, and let
$\tilde{t}(x)$ be an irreducible factor of $t(x)$ in $\mathbb{F}[x]$. Clearly
$\tilde{t}(x)$ has degree one, $\tilde{t}(x) \neq x$ and since $\mathbb{F}_{q^n}$
is a
perfect field we
know that $\tilde{t}(x)$ appears with multiplicity one in the factorization of
$t(x)$ in $\mathbb{F}[x]$.
Since $f_1(x)$ and $f_2(x)$ are coprime in $\mathbb{F}_{q^n}[x]$ they are also
coprime in  $\mathbb{F}[x]$ so
$\tilde{t}(x)^{a n_2}$ is the largest power of $\tilde{t}(x)$ which appears in 
the factorization of $v_2(x)^{q^n-1}$. From
this one may conclude that
$q^n-1 \, | \, a n_2$, and from $\gcd(a,q^n-1)=1$ we get $q^n-1 \, | \, n_2$, a
contradiction. So we must have that
$t(x)^a$ appears in the factorization of $f_1(x)$, and reasoning as above again
we
conclude again that $q^n-1\, | \, n_2$, which is impossible. Thus, if 
$n_2 \neq 0$ we get that $v(x)$ is not of the form $\left( 
\frac{v_1(x)}{v_2(x)} \right)^{q^n-1}$
in $\mathbb{F}(x)$.

Let $T_v$ be the set of $\beta \in \mathbb{F}_{q^n}$ such that $v(\beta) = 0$ or
$v(\beta)$ is not defined. If $0 \in T_v$ then $T_v = S_f$ and from Lemma
\ref{lema cota} we have
$$
\left| \tilde{\bchi}_f(\chi_{1},\chi_{2},\psi) \right|
=
\left| \sum_{\alpha\in\F_{q^n}\setminus S_f}\chi(v(\alpha))
\right| =
\left| \sum_{\alpha\in\F_{q^n}\setminus T_v}\chi(v(\alpha))
\right|
\leq (m_1 + m_2) q^{\frac{n}{2}}.
$$
If $0 \notin T_h$ then %$n_1 = 0$ and
$$
\left| \tilde{\bchi}_f(\chi_{1},\chi_{2},\psi)\right|
=
\left| \sum_{\alpha\in\F_{q^n}\setminus S_f}\chi(v(\alpha))
\right| =
\left| \sum_{\alpha\in\F_{q^n}\setminus T_v}\chi(v(\alpha)) - \chi(v(0))
\right|,
$$
so
$\left| \tilde{\bchi}_f(\chi_{1},\chi_{2},\psi)\right|
\leq (m_1 + m_2 - 1) q^{\frac{n}{2}} + 1$ and
anyway we get $| \tilde{\bchi}_f(\chi_{1},\chi_{2},\psi) | \leq (m_1 + m_2)
q^{\frac{n}{2}}$.


%Besides that $v(x)\neq \left( \frac{v_1(x)}{v_2(x)} \right)^{q^n-1}$ and $n_2 \neq 1$. From Lemma \ref{cotaadmult} it follows that
%$$
%\left| \tilde{\bchi}_f(\chi_{1},\chi_{2},\psi) \right|
%=
%\left| \sum_{\alpha\in\F_{q^n}\backslash S_v}\chi(v(\alpha))
%-
%\sum_{\alpha\in S_f\backslash S_v}\chi(v(\alpha)) \right|
%\leq M q^{\frac{n}{2}} + |S_f \backslash S_v|,
%$$
%where $M$ is the constant of Lemma \ref{cotaadmult}.
%
%Now, if $d_1=1$, then
%$n_1=0$ and so $v(\alpha) =f_1(\alpha )^{n_2}f_2(\alpha )^{-n_2}$.
%In this case $M=m_1 + m_2 - 1$ and
%$|S_f \backslash S_v| \leq 1$.
%If $d_1\neq 1$, then
%$v(\alpha) =\alpha^{n_1 }f_1(\alpha )^{n_2}f_2(\alpha )^{-n_2}$.
%In this case $M=m_1 + m_2$ and
%$S_f = S_v$. In both cases we get
%$$
%\left| \tilde{\bchi}_f(\chi_{1},\chi_{2},\psi) \right|
%<
%(m_1+m_2+1) q^{\frac{n}{2}}.
%$$

\item[(v)] 
Lastly we consider the case where $\psi$ is not a trivial character, and either 
$\chi_1$ or $\chi_2$ is not a trivial character, so that  $d_1\neq 1$ or 
$d_2\neq 1$.
Obviously 
$x$ is not of the form $r(x)^{q^n}-r(x)$ in $\F(x)$, so we may use
Lemma \ref{lema cota} (b).

As in the above case
let $T_v$ be the set of $\beta \in \mathbb{F}_{q^n}$ such that $v(\beta) = 0$ or
$v(\beta)$ is not defined. If $0 \in T_v$ then $T_v = S_f$ and from Lemma
\ref{lema cota} we have
$$
\left| \tilde{\bchi}_f(\chi_{1},\chi_{2},\psi) \right|
=
\left| \sum_{\alpha\in\F_{q^n}\setminus T_v}\chi(v(\alpha)) \psi(\alpha)
\right| 
\leq (m_1 + m_2+1) q^{\frac{n}{2}}.
$$
If $0 \notin T_h$ then %$n_1 = 0$ and
$$
\left| \tilde{\bchi}_f(\chi_{1},\chi_{2},\psi)\right|
 =
\left| \sum_{\alpha\in\F_{q^n}\setminus T_v}\chi(h(\alpha))\psi(\alpha) -
\chi(h(0)) \psi(\alpha)
\right| \leq (m_1 + m_2) q^{\frac{n}{2}} + 1 ,
$$
and
anyway we get $| \tilde{\bchi}_f(\chi_{1},\chi_{2},\psi) | \leq (m_1 + m_2 + 1)
q^{\frac{n}{2}}$.
\end{enumerate}


This finishes the analysis of the possibilities for the characters $\chi_1$, 
$\chi_2$ and $\psi$, and 
now we use the above estimates to bound $N_{f}(e_1,e_2,g)$. 
Let $\chi_0$ be the trivial multiplicative character and let
$\psi_0$ be the trivial additive character.
Write
$$
N_{f}(e_1,e_2,g)
=
\frac{\phi(e_1)\phi(e_2) \Phi(g)}{e_1 e_2 N(g)}
(S_1 + S_2 + S_3 + S_4 + S_5),
$$
where
$$S_1=\tilde{\bchi}_f (\chi_0,\chi_0,\psi_0),$$ % \geq q^n - (m_1+m_2+1)$,
$$
S_2=
\sum_{\substack{h\mid g \\ h \neq 1}}
\frac{\mu'(h)}{\Phi(h)}
\sum_{\substack{\Ord(\psi)=h}}
\tilde{\bchi}_f (\chi_0,\chi_0,\psi),
$$
$$
S_3=
\sum_{\substack{d_1\mid e_1  \\
		                   d_1 \neq 1}}
\frac{\mu(d_1)}{\phi(d_1)}
\sum_{\ord(\chi_1)=d_1}
\tilde{\bchi}_f (\chi_1,\chi_0,\psi_0),
$$
$$
S_4=
\sum_{\substack{d_1\mid e_1 , d_2\mid e_2 \\ 
		d_2 \neq 1}}
\frac{\mu(d_1)\mu(d_2)}{\phi(d_1)\phi(d_2)}
\sum_{\substack{\ord(\chi_1)=d_1 \\ \ord(\chi_2)=d_2}}
\tilde{\bchi}_f (\chi_1,\chi_2,\psi_0)
$$
and
$$
S_5=
\sum_{\substack{d_1\mid e_1 , d_2\mid e_2 \\ 
		   d_1 \neq 1 \text{ or } d_2 \neq 1 \\
	        1 \neq h\mid g}}
\frac{\mu(d_1)\mu(d_2) \mu'(h)}{\phi(d_1)\phi(d_2) \Phi(h)}
\sum_{\substack{\ord(\chi_1)=d_1 \\ \ord(\chi_2)=d_2 \\ \Ord(\psi)=h}}
\tilde{\bchi}_f (\chi_1,\chi_2,\psi).
$$
From what we did above and using that
there are $\phi(d_1)$ multiplicative characters of order $d_1$, $\phi(d_2)$
multiplicative characters of order $d_2$
and $\Phi(h)$ additive characters of $\F_q$-order $h$
 we get
\begin{eqnarray}
|S_2+S_3+S_4+S_5| & < &
(m_1+m_2+1)q^{\frac{n}{2}}
\left(
\sum_{\substack{d_1 \mid e_1, d_2\mid e_2, h \mid g \\ (d_1,d_2,h) \neq (1,1,1)}}
|\mu(d_1)| |\mu(d_2)| |\mu'(h)|
\right) \nonumber \\
& = & 
(m_1+m_2+1)q^{\frac{n}{2}}
\left(
W(e_1)W(e_2)W(g) - 1
\right). \nonumber
\end{eqnarray}
Therefore, we conclude that
\begin{eqnarray}
\displaystyle N_{f}(e_1,e_2,g) & > &\frac{\phi(e_1)\phi(e_2) \Phi(g)}{e_1 e_2 
N(g)}
\left(
q^n-(m_1+m_2+1)- \right. \nonumber \\
& & \left. (m_1+m_2+1)q^{\frac{n}{2}}(W(e_1)W(e_2)W_q(g)-1)
\right). \nonumber
\end{eqnarray}
Thus, if
\begin{eqnarray}
q^n & \geq & (m_1+m_2+1)q^{\frac{n}{2}}(W(e_1)W(e_2)W_q(g))
\nonumber \\
      & > & (m_1+m_2+1)+(m_1+m_2+1)q^{\frac{n}{2}}(W(e_1)W(e_2)W_q(g)-1) ,
\nonumber
\end{eqnarray}
then
$N_f(e_1,e_2,g)>0$.
\end{proof}

\begin{corollary}\label{mainresult}
If
$q^{\frac{n}{2}} \geq (m_1 + m_2+1) W(q^n-1)^2W_q(x^n-1)$
then 
for each $ f \in \Upsilon_{q^n} (m_1,m_2)$
there exists $\alpha \in \mathbb{F}_{q^n}$, primitive and normal
over $\mathbb{F}_q$, such that $f(\alpha) \in \mathbb{F}_{q^n}$  is also a 
primitive element.	
\end{corollary}


The next two results, and their proofs, are similar, respectively, to \cite[Lemma 3.4]{CGJV} and
\cite[Lemma 3.5]{CGJV}, so we don't present the proofs.

\begin{lemma}
	Let $\ell$ be a divisor of $q^n-1$ and let $\{p_1,...,p_r\}$ be the set of all 
	primes which divide $q^n-1$,
	but do not divide $\ell$.
	Also let $g \in \F_q[x]$ be a divisor of $x^n -1$ and $\{P_1,...,P_s\} 
	\subset \F_q[x]$
	be the set of all monic irreducible polynomials which
	divides $x^n -1$, but do not divide $g$. Then
\begin{eqnarray}
N_f(q^n-1,q^n-1,x^n-1) & \geq &
\sum_{i=1}^{r}N_f(p_i \ell,\ell,g)
+
\sum_{i=1}^{r}N_f(\ell,p_i \ell,g) \nonumber \\
& &
+\sum_{i=1}^{s}N_f( \ell,\ell, P_i g)
-
(2r+s-1)N_f(\ell,\ell,g). \nonumber
\end{eqnarray}
\end{lemma}


\begin{lemma} \label{divisores}
Let $\ell$ be a divisor of $q^n-1$ and let $\{p_1,...,p_r\}$ be the set of all 
primes which divide $q^n-1$,
but do not divide $\ell$.
Also let $g \in \F_q[x]$ be a divisor of $x^n -1$ and $\{P_1,...,P_s\} \subset 
\F_q[x]$
be the set of all monic irreducible polynomials which
divide $x^n -1$, but do not divide $g$.
Suppose that
$$
\delta=1-2\sum_{i=1}^{r}\frac{1}{p_i} - \sum_{i=1}^{s}\frac{1}{q^{\deg P_i}} >0
$$
and let  $\Delta=\frac{2r+s-1}{\delta}+2$. If 
$q^{\frac{n}{2}} \geq 
(m_1 + m_2+1)
W(\ell)^2W_q(g) \Delta$, then 
for each $ f \in \Upsilon_{q^n} (m_1,m_2)$
there exists $\alpha \in \mathbb{F}_{q^n}$, primitive and normal
over $\mathbb{F}_q$, such that $f(\alpha) \in \mathbb{F}_{q^n}$  is also a 
primitive element.	
\end{lemma}

For  positive integers
$m_1$ and $m_2$ let  $\Bb_p(m_1,m_2)$
	be the set of pairs $(q,n) \in \mathbb{N}^2$, with $q$ a power of $p$,
	such that for each $ f \in \Upsilon_{q^n} (m_1,m_2)$ there exists a 
	primitive
	element
	$\alpha \in \F_{q^n}$, normal over $\F_q$, with
	$f(\alpha)$ primitive in $\F_{q^n}$.
Let 
$$
\Bb(m_1,m_2) = \bigcup_{p \text{ is prime}} \Bb_p(m_1,m_2).
$$
We finish this section by proving that 
there exists only a finite number
of pairs $(q,n) \in \N^2$ such that $q$ is a prime power and
$(q,n) \notin \Bb(m_1,m_2)$. 
For this,
we will need the following result, which is modeled after
\cite[Lemma 3.3]{CH} and
\cite[Lemma 4.1]{Kapetanakis-Reis} and, like these results, is proved using the 
multiplicativity of the function $W(\cdot)$ and the fact that if a positive 
integer has $s$ distinct prime divisors then $W(M) = 2^s$.

\begin{lemma}\label{cota-t}
	Let $M$ be a positive integer and $t$ be a positive real number.
	Then
	$W(M)
	\leq A_t \cdot M^{\frac{1}{t}}$,
	where
	$$
	A_t=\prod_{\substack{\wp < 2^t \\ \wp \text{ is prime}
	                  \\ \wp \mid M}}
	\frac{2}{\sqrt[t]{\wp}}.
	$$
\end{lemma}
We come to the  last result in this section.

\begin{proposition}\label{Asymptotic}
	There exists only a finite number
	of pairs $(q,n) \in \N^2$ such that $q$ is a prime power and
	$(q,n) \notin \Bb(m_1,m_2)$.
	%After
	%Let $q$ be a prime power and $n$ be a natural number.
	%If $n \geq 3$ and $q \geq 3.74\cdot 10^9$
	%or $n \geq 84$ and $q \geq 23$ then
	%$(q,n) \in \Bb(3,2)$.
\end{proposition}
\begin{proof}
Clearly every $\alpha \in \F_q^*$ is  
	normal over
	$\F_q$ and it is well known that  if $\alpha \in \F_{q^2}$ is 
	primitive, then
	$\alpha$ is also normal over $\F_q$. Thus, for $n=1$ or $n=2$ we get that
	$(q,n)\in \Bb(m_1,m_2)$ if and only if
	for every $f \in \Upsilon_{q^n} (m_1,m_2)$
	there exists a primitive element $\alpha \in \F_{q^n}$ such that 
	$f(\alpha)$ is
	also primitive. This problem was solved in \cite{CSS} and from \cite[Thm. 
	3.1]{CSS} we know that a sufficient condition for the existence of 
	such an element is that
	$q^{n/2} \geq (m_1 + m_2) W(q^n-1)^2$. Using Lemma \ref{cota-t}
	and choosing a real number $t>4$
	one may check that if $q \geq \left( 
	(m_1+m_2)A_t\right)^{\frac{2t}{(t-4)n}}$
	then $(q,n)\in \Bb(m_1,m_2)$. In particular there exists only a finite 
	number of pairs
	$(q,n) \notin \Bb(m_1,m_2)$ when $n=1$ or $n=2$.
	
	We assume now that $n \geq 3$, 
	clearly $W_q(x^n-1) \leq 2^n$ and from Lemma \ref{cota-t}
	we have $W(q^n-1) \leq A_t \cdot q^{\frac{n}{t}}$ for any real number $t 
	>0$, so  
	from Theorem \ref{principal}, we get that if 
	$q^{\frac{n}{2}} \geq (m_1+m_2+1)\cdot A_t^2 \cdot q^{\frac{2n}{t}} \cdot 
	2^n$
	then $(q,n)\in \Bb(m_1,m_2)$.
%	, since
%	\begin{equation}\label{inequation}
%		(m_1+m_2+1)\cdot A_t^2 \cdot q^{\frac{2n}{t}} \cdot 2^n 
%		\geq (m_1 + m_2 + 1) W(q^n-1)^2W_q(x^n-1).
%	\end{equation}
	In particular, if we choose a real number $t>4$, 
	a condition to have $(q,n)\in \Bb(m_1,m_2)$
	for some $n \geq 3$ is
	\begin{equation}\label{condition-q}
		q \geq 
		\left(
		2^n \cdot (m_1+m_2+1)\cdot A_t^2
		\right)^{\frac{2t}{(t-4)n}}.
	\end{equation}
	In particular this means that for a given natural number $n$ there exists a 
	finite number of prime
	powers such that $(q,n)\notin \Bb(m_1,m_2)$.
	
	Inequality $q^{\frac{n}{2}} \geq (m_1+m_2+1)\cdot A_t^2 \cdot 
	q^{\frac{2n}{t}} \cdot 2^n$
	is also equivalent to
	\begin{equation}\label{condition-n}
		n \geq
		\frac{\ln \left( (m_1+m_2+1)\cdot A_t^2 \right)}%
		{(\frac{t-4}{2t}) \cdot  \ln q - \ln 2} .
	\end{equation}
	The function on the right hand side is a decreasing function of $q > 
	2^{\frac{2t}{t-4}}$.
	If we choose $t\geq 29$ then the right hand side of \eqref{condition-n}
	is a decreasing function of $q \geq 5$.
	So, if
	$N$ is a natural number such that \eqref{condition-n} is true for $q=5$,
	for some $t\geq 29$, then
	$(q,n)\in \Bb(m_1,m_2)$ for all prime powers $q\geq 5$ and all
	natural numbers
	$n \geq N$.
	
	From \cite[Lemma 2.11]{Lenstra} we have for $n \geq 16$
	$$
	W_q(x^n-1) \leq 
	\left\{
	\begin{array}{ll}
	2^{\frac{n+5}{4}} & \text{if } q=2;\\
	2^{\frac{n+4}{3}} & \text{if } q=3;\\
	2^{\frac{n}{3}+2} & \text{if } q=4,
	\end{array}
	\right.
	$$
	and for these values of $q$ we may change inequality \eqref{condition-n} for
	\begin{equation}\label{condition-n-q4}
		n \geq
		\left\{
		\begin{array}{ll}
			\frac{4t}{t-8}
			\left(
			\frac{\ln ((m_1+m_2+1)\cdot A_t^2)}{\ln 2} + \frac{5}{4}
			\right)
			& \text{if } q=2, \text{ for some } t > 8; \\
			\frac{\ln ((m_1+m_2+1)\cdot A_t^2) + \frac{4}{3} \ln 2}%
			{\left( \frac{t-4}{2t} \right) \ln 3 - \frac{1}{3} \ln 2} 
			& \text{if } q=3, \text{ for some } t \geq 7;\\
			\frac{3t}{2t-12}
			\left(
			\frac{\ln (4\cdot (m_1+m_2+1)\cdot A_t^2)}{\ln 2} + \frac{5}{4}
			\right)
			& \text{if } q=4, \text{ for some } t > 6.
		\end{array}
		\right.
	\end{equation}
	Joining \eqref{condition-n} and \eqref{condition-n-q4} we get that 
	there exists a positive integer $M$ such that if $n \geq M$ then
	$(q,n)\in \Bb(m_1,m_2)$ for every prime power $q$. Finally,
	the result follows from the fact that there exists a finite number of 
	natural numbers
	$n<M$ and for those natural numbers there exists a finite number of prime 
	powers
	such that $(q,n)\notin \Bb(m_1,m_2)$.
\end{proof}





\section{Numerical examples}

Let $(m_1,m_2)=(3,2)$ and $p=5$, 
in this section we determine pairs $(q,n)$ with $n \geq 3$ such that  $(q,n) 
\in \Bb_p(m_1,m_2)$.


%
%\subsection{Case $q=5$}



\begin{proposition}\label{case5_1}
For $q=5$ and $n \geq 3$, we have $(5,n) \in \mathcal{B}_5(3,2)$
for all $n \geq 13$ and for
$n \in \{7,9,10,11\}$.
\end{proposition}
\begin{proof}
From 	
\cite[Lemma 2.11]{Lenstra} we have 
$W_5(x^n-1) \leq 
2^{\frac{n}{3}+6}$ and from Lemma \ref{cota-t} we have
$W(5^n-1)\leq A_t \cdot 5^{\frac{n}{t}}$, with  $t > 0$ a real number.
From Theorem \ref{principal} we get that  if
$5^{n/2} \geq 6 W(5^{n}-1)^2 W_5(x^n-1)$
then $(5,n) \in \mathcal{B}_5(3,2)$. Hence 
$(5,n) \in \mathcal{B}_5(3,2)$ if $n$ satisfies
\begin{eqnarray}\label{cond5_1}
5^{n/2} \geq 6 \cdot A_t^2 \cdot 5^{\frac{2n}{t}} \cdot 2^{\frac{n}{3}+6}
\end{eqnarray}
for some real number $t > 0$. For $t=7.8$ we get $A_t=1380.449$ and
\eqref{cond5_1} is satisfied for
$n\geq 127$. 
Using SageMath (\cite{SAGE}) we test condition
$5^{n/2} \geq 6 W(5^{n}-1)^2 W_5(x^n-1)$ for $n<127$ and we get
that it holds for all
 $n \geq 25$ and for
	$
	n \in \{ 11,13,15,17,19,20,21,22,23\}.
	$
Now we use Lemma \ref{divisores} to verify that
for $n \in \{7,9,10,14,16,18,24\}$ we also have $(5,n) \in \mathcal{B}_5(3,2)$
and we present in  
Table \ref{Delta5_1} the values of $\ell$ and $g$, and the respective value of 
$n$  for which the  condition
$q^{\frac{n}{2}} \geq 6
W(\ell)^2W_q(g) \Delta$ holds.
	\begin{table}[h]
		\centering
		\begin{tabular}{ccccc}
			$n$  & $\ell$ & $g$ & $\{p_1,p_2,\ldots ,p_r\}$ &  $[\deg P_1 , 
			\ldots , \deg P_s]$\\
			\hline
			$7$ & $2$ & $x-1$
			& $\{ 19531 \}$ & $[6]$ \\
			$9$ & $2$ & $x-1$
			& $\{ 19,31,829 \}$ & $[2,6]$ \\
			$10$ & $6$ & $x - 1$
			& $\{ 11,71,521 \}$ & $[1]$ \\
			$14$ & $6$ & $x^2-1$
			& $\{ 29,449,19531 \}$ & $[6,6 ]$ \\
			$16$ & $6$ & $x^4-1$
			& $\{ 13,17,313,11489 \}$ & $[2,2,4,4]$  \\
			$18$ & $6$ & $x^2 - 1$
			& $\{ 7,19,31,829,5167 \}$ & $[2,2,6,6]$ \\
			$24$ & $6$ & $x^4-1$
			& $\{ 7,13,31,313,601,390001 \}$ & $[2,2,2,2,2, 2,2,2,2,2]$
		\end{tabular}
\caption{Values of $\ell$ an $g$ for $q=5$}
\label{Delta5_1}
	\end{table}	

\end{proof}



\begin{proposition}\label{case5_2}
For $q=5^2$ and $n \geq 3$, we have $(q,n) \in \mathcal{B}_5(3,2)$
for all $n \geq 5$.
\end{proposition}
\begin{proof} 
Let $s$ be the number of monic irreducible factors of 
$x^n-1 \in \F_q[x]$, then $W_q(x^n-1) = 2^s$ and  
from	\cite[Inequality (2.10)]{Lenstra} we have 
$$
s \leq \frac{1}{2} \left( n + \gcd(n,q-1) \right).
$$
If $n \geq 25$ we
may assume $W_q(x^n-1) \leq 2^{\frac{3n}{4}}$, since in this case
$\gcd(n,q-1) \leq n/2$, and using Lemma \ref{cota-t} we get that  if 
\begin{eqnarray}\label{cond5_2}
q^{\frac{n}{2}} \geq 6 \cdot A_t^2 \cdot q^{\frac{2n}{t}} \cdot 2^{\frac{3n}{4}}
\end{eqnarray}
for some real number $t$ then 
$(q,n) \in \mathcal{B}_5(3,2)$.
For $t=8$ we have $A_t=2760.34$ and
\eqref{cond5_2} is satisfied for
$n\geq 62$. 
Using SageMath we test condition
$q^{n/2} \geq 6 W(q^{n}-1)^2 W_q(x^n-1)$ for $n<62$  and we get
that the inequality holds for all
$n \geq 13$ and for
$n \in \{ 7,10,11\}$.

We use now  Lemma \ref{divisores}, with the values of $\ell$ and $g$ which 
appear in Table \ref{Delta5_2}, to verify that
for $n \in \{5,6,8,9,12\}$ we have $(q,n) \in \mathcal{B}_5(3,2)$.

\begin{table}[h]
\centering
\begin{tabular}{ccccc}
$n$  & $\ell$ & $g$ & $\{p_1,p_2,\ldots ,p_r\}$ &  $[\deg P_1 , \ldots , \deg 
P_s]$\\
\hline
$5$ & $6$ & $1$
			& $\{ 11 , 71 , 521 \}$ & $[1]$ \\
$6$ & $6$ & $x-1$
			& $\{ 7 , 13 , 31 , 601 \}$ & $[1,1,1,1,1]$ \\
$8$ & $6$ & $x - 1$
			& $\{ 13 , 17 , 313 , 11489 \}$ & $[1,1,1,1,1,1,1]$ \\
$9$ & $6$ & $x-1$
			& $\{ 7 , 19 , 31 , 829 , 5167 \}$ & $[1,1,3,3]$ \\
$12$ & $6$ & $x^4-1$
			& $\{ 7,13,31,313,601,390001 \}$ & $[1,1,1,1,1,1,1,1]$
\end{tabular}
\caption{Values of $\ell$ an $g$ for $q=5^2$}
\label{Delta5_2}
\end{table}	
\end{proof}


\begin{lemma}\label{case5_3}
Let $q=5^k$ be a prime power of $5$.
If $k \geq 13$ and $n \geq 3$ then $(q,n) \in \mathcal{B}_5(3,2)$.
\end{lemma}
\begin{proof}
We know that
$W_q(x^n-1)\leq 2^n$, so if we prove  that
$$
q^{\frac{n}{2}} \geq 6 \cdot (A_t \cdot q^{\frac{n}{t}})^2 \cdot 2^n
$$
holds for some real number $t > 0$, then
$(q,n) \in \mathcal{B}_5(3,2)$. Observe that the above inequality is equivalent 
to
$q \geq 
\left(
2 \cdot \sqrt[n]{6 \cdot A_t^2}
\right)^{\frac{2t}{t-4}}$.
For $t=6$ we get $A_t=12.76$ and $q \geq 5^{13}$ for $n \geq 3$.
\end{proof}


\begin{proposition}\label{case5_4}
Let $q=5^k$ be a prime power of $5$ where
$3 \leq k \leq 12$
and let $n\geq 3$. Then $(q,n) \in \mathcal{B}_5(3,2)$.
\end{proposition}
\begin{proof}
As always we want to verify 
$$
q^{n/2} \geq 6 W(q^{n}-1)^2 W_q(x^n-1)
$$
where $q=5^k$ for $3 \leq k \leq 12$. Using inequalities
$W(q^{n}-1) \leq A_t \cdot q^{\frac{n}{t}}$ and 
$W_q(x^n-1)\leq 2^n$ we may verify
$$
q^{\frac{n}{2}} \geq 6 \cdot A_t^2 \cdot q^{\frac{2n}{t}} \cdot  2^{n}
\Longleftrightarrow
n \geq \frac{\ln (6 \cdot A_t^2)}{ \left( \frac{t-4}{2t} \right)\ln q -\ln  2}.
$$
Using the above inequality and assigning values to $q$ and $t$ we find ranges 
of values of
$n$ for which $(q,n) \in \mathcal{B}_5(3,2)$. %get Table \ref{Case5_1}.
\begin{table}[h]
	\centering
	\begin{tabular}{ccc|ccc|ccc}
		$t$ & $q$ & $n$ & $t$  &$q$ & $n$ & $t$  &$q$ & $n$ \\
		\hline
		$7.5$ & $5^3$  & $n \geq 34$ & $6$& $5^7$ & $n \geq 7$  & $6$ & 
		$5^{10}$ 
		                                                       & $n \geq 5$ \\
		$7$ & $5^4$  & $n \geq 18$ & $6$ & $5^8$ & $n \geq 6$& $6$ & $5^{11}$
		& $n \geq 4$ \\
		$7$ & $5^5$  & $n \geq 12$ & $6$ & $5^9$ & $n \geq 5$  & $6$ & $5^{12}$
		& $n \geq 4$ \\
		$7$ & $5^6$ & $n \geq 9$ 
	\end{tabular}
	\caption{Values of $q$, $n$ and a real number $t$ for which
		$(q,n)\in \mathcal{B}_5(3,2)$}
	\label{Case5_1}
\end{table}

Next, using SageMath,
we verify $q^{n/2} \geq 6 W(q^{n}-1)^2 W_q(x^n-1)$  for
pairs $(q,n)$ which are not in Table \ref{Case5_1}, taking $3 \leq k \leq 12$
and $n \geq 3$.
We get $(q,n) \in \mathcal{B}_5(3,2)$ except
for the pairs $(5^3,3),(5^3,4),(5^4,3)$ and $(5^4,4)$. These
pairs also belong to $\mathcal{B}_5(3,2)$, as we may see by applying 
Lemma \ref{divisores}. The details are as follows. For $q=5^3$ and $n=3$ we take
$\ell=2$, $g=1$,
where $q^n-1=2^2 \cdot 19 \cdot 31 \cdot 829$ and
$x^n-1=(x - 1) \cdot (x^2 + x + 1)$.
For $q=5^3$ and $n=4$ we take $\ell=2 \cdot 3$ and $g=1$,
and use that $q^n-1=2^4 \cdot 3^2 \cdot 7 \cdot 13 \cdot 31 \cdot 601$ and
$x^n-1=(x - 1) \cdot (x+1) \cdot (x-2) \cdot (x+2)$.
For $q=5^4$ and $n=3$ we take $\ell=2\cdot 3$ and $g=1$,
and we use that  $q^n-1=2^4 \cdot 3^2 \cdot 7 \cdot 13 \cdot 31 \cdot 601$ and
$x^n-1$ has $3$ linear factors.
For $q=5^4$ and $n=4$ we take $\ell=2 \cdot 3$ and $g=1$ and use that 
 $q^n-1=2^6 \cdot 3 \cdot 13 \cdot 17 \cdot 313 \cdot 11489$ and
$x^n-1=(x - 1) \cdot (x+1) \cdot (x-2) \cdot (x+2)$.
\end{proof}

The following theorem summarizes the above results.

\begin{theorem}
Let $q=5^k$ be a prime power of $5$
and let $n \geq 3$.
We have $(q,n) \in \mathcal{B}_5(3,2)$ for all $k\geq 3$. If 
$k=2$ then $(q,n) \in \mathcal{B}_5(3,2)$ for $n\geq 5$ and
if $k=1$ then $(q,n) \in \mathcal{B}_5(3,2)$ 
for all $n \geq 13$ and for
$n \in \{7,9,10,11\}$.
\end{theorem}


\section*{Acknowledgements}
C\'{\i}cero Carvalho  was partially funded by FAPEMIG APQ-01645-16, Jo\~{a}o 
Paulo Guardieiro was partially funded by CAPES 88882.441370/2019-01, Victor 
G.L.\ Neumann was 
partially funded by FAPEMIG APQ-03518-18 and Guilherme Tizziotti was 
partially funded by 
CNPq 307037/2019-3.




\begin{thebibliography}{99}

\bibitem{mersenne} Anju and R.K. Sharma, {\em Existence of some special 
primitive normal elements over finite fields}. Finite Fields and Their 
Applications, v. 46, pp. 280--303, 2017.


\bibitem{CGJV} C. Carvalho, J.P. Guardieiro Sousa, V. Neumann and G. 
Tizziotti, {\em On existence of some special pair of primitive elements over 
finite fields}, preprint {\tt arXiv:2002.01867 [math.NT]}.

%\bibitem{Cohen} S.D.\ Cohen, {\em Pair of primitive elements in fields of even 
%order}, Finite Fields Appl. v. 28, pp. 22--42, 2014.

\bibitem{CH} S.D.\ Cohen and S.\ Huczynska, {\em The primitive normal basis 
theorem -- without a computer}. Journal of London Mathematical Society, v. 67, 
n. 1, pp. 41--56, 2003.

\bibitem{CH2} S. D. Cohen and S. Huczynska, {\em The strong primitive normal basis theorem}. Acta Arith., 143 (4), pp. 299--332, 2010.

\bibitem{CSS}
S.D.\ Cohen. H.\ Sharma and R.\ Sharma, {\em Primitive values of rational 
functions at primitive elements of a finite field},  preprint {\tt 
arXiv:1909.13074 [math.NT]}.

\bibitem{Fu} L. Fu and D.Q. Wan, {\em A class of incomplete character sums}, 
Quart. J. Math. 65, pp. 1195--1211, 2014.


\bibitem{HBC} H.\ Hazarika, D.K.\ Basnet and S.D.\ Cohen, {\em The existence of 
primitive normal elements of quadratic forms over finite fields}, 
preprint {\tt arXiv:2001.06977 [math.NT]}.


\bibitem{HB} H.\ Hazarika and  D.K.\ Basnet, {\em On existence of primitive 
normal elements of rational form over finite fields of even characteristic},
preprint {\tt arXiv:2005.01216 [math.NT]}.

%\bibitem{HK} H. Hazarika and D. Kumar Basnet, {\em Sufficient condition for 
%existence of special type of primitive normal elements over finite fields}, 
%preprint arxiv.org/abs/1902.04736.

\bibitem{kapeta} G. Kapetanakis, Normal bases and primitive elements over 
finite fields, Finite Fields Appl. 26, 
pp. 123--143, 2014.

\bibitem{Kapetanakis-Reis} G. Kapetanakis, L. Reis, {\em Variations of the Primitive
	Normal Basis Theorem}.
Designs, Codes and Cryptography 87 (2019) 1459--1480.

\bibitem{Lenstra} H.W. Lenstra and R.J. Schoof, {\em Primitive Normal Bases for Finite Fields}.
Mathematics of Computation, v. 48, p. 217--231, 1987.

%\bibitem{SAG} K. Sharma, A. Awasthi and A. Gupta, {\em Existence of pair of 
%primitive elements over finite fields of characteristic 2}. Journal of Number 
%Theory, v. 193, p. 386--394, 2018.
%

\bibitem{SAGE} SageMath, the Sage Mathematics Software System (Version 8.1),
   The Sage Developers, 2020, \texttt{https://www.sagemath.org}.
   
   
\bibitem{LN} R. Lidl and H. Niederreiter, {\em Finite Fields}. Cambridge 
university press, 1997.


\end{thebibliography}



\end{document}

