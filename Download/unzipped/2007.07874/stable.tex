%!TEX root = paper.tex


We introduce and analyse a randomised procedure to generate an independent set.
This procedure captures the behaviour of each colour class of the colouring procedure we analyse in Subsection~\ref{sub:list colouring}.
The idea is to assign to each vertex a random \emph{priority}
and resolve conflicting edges by removing the endpoint with lower priority.

Fix a parameter $\gamma>0$. Given a $\Delta$-regular graph $G=(V,E)$, the following procedure outputs a random independent set $\bI$ of $G$.
\begin{enumerate} 
\item\label{samplestep1} Activate each vertex of~$G$ with probability $\gamma/\Delta$,
  independently at random.
  Let $\A$ be the set of activated vertices.
\item\label{samplestep2} Assign to each activated vertex $v\in \A$ a number $\pi(v)$
  chosen uniformly at random in $[0,1]$.
\item\label{samplestep3} In order to resolve any conflict, i.e.~two neighbouring vertices in $\A$,
  remove the vertex with lower priority $\pi$.
  This yields the independent set
  \[\bI=\sst{v\in \A}{\pi(v) > \pi(u) \text{ for every }u \in N(v)\cap \A},\]
  consisting of all the local maxima of $\pi$ in $G[\A]$.
\end{enumerate}

The purpose of this section is to prove the following result.
\begin{theorem}\label{thm:independent}
  For every $\iota>0$, there are $\Delta_{\ref{thm:independent}}=\Delta_{\ref{thm:independent}}(\iota)$
  and $\gamma_{\ref{thm:independent}}=\gamma_{\ref{thm:independent}}(\iota)$ such that the following holds.
  Let $G$ be a $\sigma$-sparse $\Delta$-regular graph with $\Delta\geq\Delta_{\ref{thm:independent}}$,
  and let
  $\bI$ be a random independent set obtained by the algorithm above with some parameter
  $\gamma\geq\gamma_{\ref{thm:independent}}$.
  For every vertex $r\in V(G)$, 
\[
\left|\Prob[r\in \bI]-\frac{1-e^{-\gamma}}{\Delta}\right|\leq \frac{\iota}{\Delta}.
\]
Moreover, setting $\bI_r = N(r) \cap \bI$, it holds that
  \[
  \frac{\Prob\left[\bI_r \neq \varnothing\right]}{\E\left[|\bI_r|\right]}
 \leq 
  1-\eps_{\ref{col_result}}(\sigma)+\iota.
  \]
\end{theorem}
\noindent
The ratio in Theorem~\ref{thm:independent} can be read as the inverse of the average size of~$\bI_v$
when $\bI_v$ is non-empty.
For comparison, if $\bI$ is chosen instead as a random colour class of a proper $\chi$-colouring of~$G$, then this ratio is a lower bound for $\chi/\Delta$.
After proving Theorem~\ref{thm:independent}, we use the remainder of the section to transfer this bound to the chromatic number, showing that
$\chi(G)\leq (1-\eps_{\ref{col_result}}(\sigma)+o(1))\cdot\Delta$.


In the proof, we say that a vertex $u$ \emph{trumps} a vertex $v$ if $uv$ is an edge,
$u$ and $v$ are activated and $\pi(u)\ge\pi(v)$.
With this vocabulary, $\bI$ is the set of activated vertices that are not trumped.
\begin{proof}[Proof of Theorem~\ref{thm:independent}]
It is convenient to instead prove a slightly stronger, local version of Theorem~\ref{thm:independent},
where $\sigma$ is the local sparsity of $r$, so that $G[N(r)]$
contains exactly $(1-\sigma)\binom{\Delta}{2}$ edges.
Proving the local version is enough because
$\eps_{\ref{col_result}}(\sigma)$ is an increasing function of $\sigma$.

For convenience, we define
\[
 \I_n=\sst{\{u_i\}_{i=1}^n\subseteq N(r)}{\forall i,j\in\{1,\dots,n\}, ~u_iu_j\not\in E(G)}
\]
 as the collection of independent sets of size $n$ in $N(r)$.
For an independent set
$\{u_i\}_{i=1}^n\in \mathcal{I}_n$, we define
\[
 \Pk(\{u_i\}_{i=1}^n):=\Prob\left[\forall i\in[n],~u_i\in \bI \,\,\middle|\,\, \forall i\in[n],~u_i \in \A\right].
\]
 This is the probability that all of the considered vertices
are retained in the independent set if they were activated in Step~\ref{samplestep1}.

Let us show the first part of the lemma.
\begin{claim}\label{claim:v in I}
        For every vertex $v\in V$,
        it holds that 
\[
\Prob[v\in\bI]=
        \frac{1}{\Delta}\int_0^\gamma\left(1 -\frac{x}{\Delta}\right)^{\Delta} dx=
\frac{1-e^{-\gamma}}{\Delta}+o\left(\frac{1}{\Delta}\right).
\]
\end{claim}
\begin{proof}
Assuming that~$v$ is activated
and given $\pi(v)$, the probability that $v$ is trumped by some
other vertex $q\in N(v)$ is the probability~$q$ is activated times the probability
that $\pi(q)\geq\pi(v)$.
This latter probability is $1-\pi(v)$ because $\pi(q)$ is chosen uniformly at random in $[0,1]$.
Consequently,
\[
        \Prob[q\text{ trumps }v]=\frac{\gamma}{\Delta}(1-\pi(v)) = \frac{\gamma x}{\Delta},
\]
where we write $x=1-\pi(v)$ in order to simplify integration.
Expressing the probability that no neighbour of~$v$ trumps~$v$ as
$\left(1- \gamma x/\Delta\right)^\Delta$ and integrating over the possible values of $x$,
we get
\[
        \Prob[v\in\bI] =
        \Prob[v\in\A]\cdot\int_0^1\left(1 -\frac{\gamma x}{\Delta}\right)^{\Delta} dx
        = \frac{\gamma}{\Delta}\int_0^1\left(1 -\frac{\gamma x}{\Delta}\right)^{\Delta} dx
        = \frac{1}{\Delta}\int_0^\gamma\left(1 -\frac{x'}{\Delta}\right)^{\Delta} dx'
\]
which proves the first part of the claim.
This value is always less than the limit $(1-e^{-\gamma})/\Delta$
since
\[\int_0^\gamma\left(1-\frac{x}{\Delta}\right)^{\Delta}dx\leq\int_0^\gamma e^{-x}dx=1-e^{-\gamma}.
\]
For the lower bound, we use the fact $e^{-t}(1-t^2)\leq 1-t$ for every $t\in[0,1]$
applied to $t=x/\Delta$.
This gives
\[\int_0^\gamma\left(1-\frac{x}{\Delta}\right)^{\Delta}dx\geq \int_0^\gamma e^{-x}\left(1-\frac{x^2}{\Delta^2}\right)^{\Delta}dx
\geq \int_0^\gamma e^{-x}dx -\frac{1}{\Delta}\int_0^\gamma e^{-x}x^2 dx.
\]
Last, bounding the second integral by $\int_0^\infty e^{-x}x^2dx=2$,
we deduce that $\Prob[v\in\bI]\geq (1-e^{-\gamma})/\Delta-2/\Delta^2$. 
\end{proof}
As a consequence of Claim~\ref{claim:v in I},
the expected size of $\bI_r=\bI\cap N(r)$ is
\[
\E[|\bI_r|]=\sum_{v\in N(r)}\Prob[v\in \bI]= 1 - e^{-\gamma} + o(1).
\]
We now prove the following reduction.
\begin{claim}\label{claim:no outside p3}
We may assume that no pair of distinct vertices $u,v\in N(r)$ have a common neighbour outside of
$N[r]$.
\end{claim}
\begin{proof}
Assume otherwise that there is vertex $w\in V(G)\setminus N[r]$
with at least two neighbours in $N(r)$.
We construct a $\sigma$-sparse $\Delta$-regular graph $G'$ as the disjoint union
of $G\setminus w$ and a complete bipartite graph $K_{\Delta-1,\Delta}$ on a vertex partition $X\cup Y$ with $|X|=\Delta$ and $|Y|=\Delta-1$,
in which we further connect each vertex $N_G(u)$ to a distinct vertex of~$X$.
Here the role of the bipartite graph is only to preserve the $\Delta$-regularity.
The neighbourhood of~$r$ is the same in~$G$ and in~$G'$, so $N_{G'}(v)$ induces
exactly $(1-\sigma)\binom{\Delta}{2}$ edges in~$G'$.
Let $\bI'$ be the random independent set obtained by our procedure on~$G'$
and consider $\bI_r'=\bI \cap\{N_{G'}(r)\}$.
We know as a corollary of Claim~\ref{claim:v in I}
that $\E[|\bI_r|]=\E[|\bI_r'|]$.
Further, we claim that
\begin{equation}\label{eq:G' is worse}
\Prob[\bI_r\neq\varnothing]\leq\Prob[\bI_r'\neq\varnothing].
\end{equation}
In that case,
$\Prob[\bI_r\neq\varnothing]/\E[|\bI_r|]\leq \Prob[\bI_r'\neq\varnothing]/\E[|\bI_r'|]$,
so it is enough to prove Theorem~\ref{thm:independent} for $G'$ and $r$ to deduce it for $G$ and $r$.
Since the number of common neighbours of $N(r)$ outside $N[r]$ is strictly
lower in $G'$ than in $G$,
the iteration of this transformation terminates, which proves Claim~\ref{claim:no outside p3}.

It remains to show~\eqref{eq:G' is worse}.
To do so, we couple $\bI$ and $\bI'$
in such a way that~$\bI_r$ is empty in every outcome where~$\bI_r'$ is.
Assuming that $\A$ and $\pi$ are given, the activation set~$\A'$
and the priority function~$\pi'$ are defined as follows.
First, vertices of $V(G)\cap V(G')=V(G)\setminus\{r\}$ are activated accordingly for $\A$
and $\A'$ and get the same priority, that is
$\A'\cap V(G):= \A\setminus\{r\}$ and $\pi'=\pi$ on this set.
Now, let $U$ be the set of vertices of $\A\cap N(r)\cap N(w)$
that trump all their neighbours in $\A\setminus \{w\}$ for $G$,
i.e. the set of vertices of $N(r)\cap N(w)$ that would be in $\bI$ if we ignore $w$
in Step~\ref{samplestep3}.
If $U\neq\varnothing$,
let $v_U$ be the vertex of $U$ with the highest value $\pi(v_U)$
and let~$w'$ be the unique neighbour of~$v_U$ in~$X$ for the graph~$G'$.
We activate $w'$ for $\A'$ if $w$ is activated for $\A$,
and in this case we set $\pi'(w'):=\pi(w)$.
Next, we activate (for $\A'$) the remaining vertices of $X\cup Y$ independently at random with
probability $\gamma/\Delta$ and we give them priorities $\pi'$ chosen independently,
uniformly at random in $[0,1]$.
The set $\bI'$ is then defined from $\A'$ and $\pi'$ similarly as for $\bI$,
as the set of vertices of $\A'$ whose value by $\pi'$ is larger than all of their neighbours
in $\A'$.

It is clear that $\A'$, $\pi'$, and further $\bI$ are distributed as in our procedure
because these activations and priorities are mutually independent.
Crucially, note that the choice of~$v_U$ and~$w'$ (when $U\neq\varnothing$)
is independent from the value of $\pi(w)$.
If $U=\varnothing$, then $\bI_r=\varnothing=\bI_r'$ regardless of
the state of $w$ and the vertices of $X\cup Y$,
so assume that $U\neq\varnothing$ and $\bI_r'=\varnothing$.
Since $\bI_r$ and $\bI_r'$ coincide on $N(r)\setminus N(w)$,
we know that $\bI_r\subseteq U$.
Further, since $v_U$ is in $U$ but not in $\bI_r'$, this vertex is trumped by $w'$,
so $w'\in\A'$ and $\pi(v_U)=\pi'(v_U)\le\pi'(w')=\pi(w)$. 
Moreover, $\pi(v_U)$ is by definition of $v_U$ the highest value of $\pi(U\cap\A)$,
so $w$ trumps every activated vertex of $U$, and further $\bI_r=\varnothing$.
\end{proof}

It remains to estimate $\E[\bI_r\neq\varnothing]$.
To do so, let $P_r$ be the the number of pairs
in $N(r)\cap \bI_r$, i.e. $P_r=\binom{|\bI_r|}{2}$, and let $T_r$ be the number of triples
in $N(r)\cap \bI_r$, i.e. $T_r=\binom{|\bI_r|}{3}$.
Our bound on $\E[\bI_r\neq\varnothing]$ relies on the following variation
of the inclusion-exclusion principle:
\begin{equation}\label{eq:inclusion-exclusion} 
\Prob[\bI_r \neq \varnothing] \leq \E[|\bI_r|] - \E[P_r] + \E[T_r].
\end{equation}
To see this, fix $\bI_r$ and
apply the inclusion-exclusion principle to $|\bI_r|$ identical sets of size $1$
to get 
\[\mathbb{1}_{\bI_r \neq \varnothing}\leq |\bI_r| - \binom{|\bI_r|}{2} + \binom{|\bI_r|}{3},\]
where $\mathbb{1}_{\bI_r \neq \varnothing}$ equal $1$ if $\bI_r$ is non-empty and $0$ otherwise.
Taking the expectation of this last inequality proves~\eqref{eq:inclusion-exclusion}.

Let us now estimate $P_r$ and $T_r$ in terms of the following parameter.
Given a pair $uv\in \I_2$,
define 
\[\ell_{uv}:=\frac{1}{\Delta}|N(v) \cap N(u)|.\]
We start with $P_r$.
\begin{claim}\label{claim:pairs}
It holds that
\[
  \E[P_r] = \sigma\cdot\Eon{uv\in \I_2}\left[\frac{1}{2-\ell_{uv}}\right]
  + o_\gamma(1)
  + o_\Delta(1).
\]
\end{claim}
\begin{proof}
First note that
\[
   \E[P_r]= \sum_{uv\in \I_2}\Prob[u, v \in \bI~|~u,v\in \A]\cdot\Prob[u, v \in \A].
\]
As the activation of the vertices are independent we have
$\Prob[u, v\in \A]=\gamma^2/\Delta^2$
so let us consider $\Pk(u,v)$.

Assume first that $u$ and $v$ are activated and that $\pi(u)\geq\pi(v)$.
In that case, a vertex of $N(u)\cap N(v)$ that trumps $u$ necessarily trumps $v$,
so $u$ and $v$ are in $\I$ exactly when $v$ trumps the vertices of~$N(v)$
and $u$ trumps the vertices of $N(u)\setminus N(v)$.
As a consequence, the probability that both~$u$ and~$v$ are in $\bI$ is
\[
        \left(1-\frac{\gamma}{\Delta}(1-\pi(u))\right)^\Delta\cdot
        \left(1-\frac{\gamma}{\Delta}(1-\pi(v))\right)^{(1-\ell_{uv})\Delta}
        = e^{-\gamma(1-\pi(u))}e^{-\gamma(1-\ell_{uv})(1-\pi(v))}+o_\Delta(1),
\]
where we again write the probability that a vertex $w$ trumps
an activated neighbour~$t$ as $\frac{\gamma}{\Delta}(1-\pi(t))$.

Integrating on the values of $x=1-\pi(u)$ and $y=1-\pi(v)$, we get
\[
   \Prob[u,v\in\bI\text{ and }\pi(u)>\pi(v)| u,v\in\A]=
   \int_0^1\int_0^xe^{-\gamma x}e^{-\gamma(1-\ell_{uv})y} dydx + o_\Delta(1).
\]
Accounting for the symmetry between the case $\pi(u)>\pi(v)$ and $\pi(u)<\pi(v)$,
we deduce
\[
    \Pk(u,v)=2\int_0^1\int_0^xe^{-\gamma x}e^{-\gamma(1-\ell_{uv})y} dydx + o(1),
\]
where the dependence is as $\Delta\to\infty$.

Computing the integral, we obtain
\begin{align*}
  \int_0^1\int_0^x\mathrm{e}^{-\gamma x}\mathrm{e}^{-\gamma \left(1-\ell_{uv}\right)y} dydx &=
  \frac{1}{\gamma(1-\ell_{uv})}\int_0^1\mathrm{e}^{-\gamma x}\left(1-e^{-\gamma(1-\ell_{uv})x}\right) dx \\
  &=
  \frac{1}{\gamma^2(1-\ell_{uv})}\left(
        1-e^{-\gamma}-\frac{1-e^{-\gamma(2-\ell_{uv})}}{2-\ell_{uv}}
\right)\\
  &= \frac{1}{\gamma^2(2-\ell_{uv})} + o\left(\frac{1}{\gamma^2}\right).
\end{align*}
Recalling that $|\I_2|=(\sigma/2+o(1))\Delta^2$,
we can conclude that
\[
  \E[P_r] =
  \sum_{uv\in \I_2}\frac{\gamma^2}{\Delta^2}\cdot\frac{2}{\gamma^2(2-\ell_{uv})}
  +o_\gamma(1)+o_\Delta(1)
  =\sigma\cdot\Eon{uv\in\I_2}\left[\frac{1}{2-\ell_{uv}}\right]+o_\gamma(1)+o_\Delta(1).\qedhere
\]
\end{proof}

Let us estimate the number of triples in $\bI_r$.
For that purpose, we first derive an integral expression for $\E[T_r]$.
Set
\[
   f(\ell_1,\ell_2,\ell_3)=
   \frac{1}{(2-\ell_1)(3-\ell_1-\ell_2-\ell_3)}.
\]
We now prove the following expression.
\begin{claim}\label{claim:pkeep triples}
For every triple $uvw\in\I_3$,
\[
   \Pk(u,v,w)\leq
   \frac{2}{\gamma^3}\cdot
   \left[
   f(\luv,\luw,\lvw)+
   f(\lvw,\luv,\luw)+
   f(\luw,\luw,\lvw)
   \right]+ o_\Delta(1).
   \]
\end{claim}
\begin{proof}
Assuming that $u$, $v$ and~$w$ are activated and that
the values $x=1-\pi(u)$, $y=1-\pi(v)$ and~$z=1-\pi(w)$
are given, we aim to express the probability that $\{u,v,w\}\subseteq\bI$.
If moreover $x\ge y\ge z$, i.e.~$\pi(u)\le\pi(v)\le\pi(w)$,
this last event happens exactly when none of the  vertices of $N(u)$ trump~$u$,
none of $N(v)\setminus N(u)$ trump $v$ and
none of the vertices of $N(w)\setminus(N(u)\cup N(v))$
trump~$w$.
The sizes of these sets are estimated by respectively
$|N(u)|=\Delta$, $|N(v)\setminus N(u)|=(1-\luv)\Delta$
and $|N(w)\setminus(N(u)\cup N(v))|\geq(1-\luw-\lvw)\Delta$.
The probability that $\{u,v,w\}\subseteq\bI$ in this case is therefore at most
 \[
        \left(1-\frac{\gamma}{\Delta}x\right)^\Delta
        \left(1-\frac{\gamma}{\Delta}y\right)^{\Delta(1-\luv)}
        \left(1-\frac{\gamma}{\Delta}z\right)^{\Delta(1-\luw-\lvw)},
 \]
 which tends to the limit
 \[
   p_{xyz}=
    e^{-\gamma\left(x +
    \left(1-\luv\right)y
    + (1-\luw-\lvw)z
    \right)}
  \]
  as $\Delta\to\infty$.
%
Integrating over the possible values of $x$, $y$ and $z$
satisfying $1\geq x\geq y\geq z\geq 0$ gives
\begin{align*}
   \int_0^1\int_{z}^1\int_{y}^1p_{xyz}dxdydz
    &=\frac{1}{\gamma^3}\cdot
    \int_{0}^\gamma\int_{z'}^\gamma\int_{y'}^\gamma
    e^{-x'-(1-\luv)y'
    - (1-\luw-\lvw)z'}dx'dy'dz'\\
    &\leq \frac{1}{\gamma^3}\cdot
    \int_{0}^\infty\int_{z}^\infty\int_{y}^\infty
    e^{-x-(1-\luv)y
    - (1-\luw-\lvw)z}dxdydz\\
    &= \frac{1}{\gamma^3}\cdot\int_{0}^\infty\int_{z}^\infty e^{-(2-\luv)y- (1-\luw-\lvw)z}dydz \\
    &= \frac{1}{\gamma^3}\cdot\frac{1}{2-\luv}\cdot\int_{0}^\infty e^{- (3-\luv-\luw-\lvw)z}dz\\
    &= \frac{1}{\gamma^3}\cdot f(\luv,\luw,\lvw),
\end{align*}
where the first line comes from the change of variables $x'=\gamma x$, $y'=\gamma y$
and $z'=\gamma z$.
To summarise, we have
\[
\Prob\left[u,v,w\in\bI\text{ and }\pi(u)\le\pi(v)\le\pi(w)\,\,\middle|\,\,u,v,w\in\A\right]
\leq \frac{1}{\gamma^3}f(\luv,\luw,\lvw) + o_\Delta(1) 
\]
Taking into account the six possible orderings of $\pi(u)$, $\pi(v)$ and~$\pi(w)$
and the symmetry of $f$ between the second and the third variable yields
\begin{align*}
   \Pk(u,v,w)=&\frac{1}{\gamma^3}(
   2f(\luv,\luw,\lvw)+
   2f(\lvw,\luv,\luw)+
   2f(\luw,\lvw,\luv)
) +o_\Delta(1),
\end{align*}
which finishes the proof of the claim.
\end{proof}
The function $f$ satisfies the following bound that isolates its parameters:
\begin{equation}\label{eq:f as sum}
f(\ell_1,\ell_2,\ell_3)+f(\ell_3,\ell_1,\ell_2)+f(\ell_2,\ell_3,\ell_1) \leq
        \frac13\cdot\sum_{i=1}^3\frac{1}{(2-\ell_i)(1-\ell_i)}
\end{equation}
for every $\ell_1,\ell_2,\ell_3\in[0,1]$.
This relation can be proven as follows:
\begin{align*}
f(\ell_1,\ell_2,\ell_3)+f(\ell_2,\ell_1,\ell_3)+f(\ell_1,\ell_2,\ell_3)&=
\frac{1}{3-\ell_1-\ell_2-\ell_3}\cdot\left(\frac{1}{2-\ell_1}+\frac{1}{2-\ell_2}+\frac{1}{2-\ell_3}\right)\\
&\leq \frac13\cdot\sum_{i=1}^3\frac{1}{3-3\ell_i}\left(\frac{1}{2-\ell_1}+\frac{1}{2-\ell_2}+\frac{1}{2-\ell_3}\right)\\
&\leq \sum_{i=1}^3\frac{1}{3-3\ell_i}\cdot\frac{1}{2-\ell_i}.
\end{align*}
Here, the second line is obtained by convexity of the function $x\mapsto\frac{1}{3-x}$
applied on the left factor and the last step is an application of the rearrangement inequality.

We deduce, from Claim~\ref{claim:pkeep triples} and~\eqref{eq:f as sum},
the following bound on $\E[T_r]$.
\begin{claim}\label{claim:triples}
It holds that
\[\E[T_r]\leq \frac{|\I_3|}{\Delta^3}+
\frac\sigma6\cdot\Eon{uv\in \I_2}\left(
\frac{2}{2-\ell_{uv}}+\luv-1
\right)
.\]
\end{claim}
\begin{proof}
We compute the expected number of triples similarly as for the pairs:
\[\E[T_r]=\sum_{uvw\in\I_3}\Prob[u,v,w\in\A]\cdot\Pk(u,v,w)=\frac{\gamma^3}{\Delta^3}\cdot\sum_{uvw\in\I_3}\Pk(u,v,w).
\]
Further, applying Claim~\ref{claim:pkeep triples} and Equation~\eqref{eq:f as sum} gives
\[
   \E[T_r]
   \leq\frac{1}{3\Delta^3}\cdot\sum_{uvw\in \I_3}\sum_{ab\in\{uv,uw,vw\}}\frac{2}{(2-\ell_{ab})(1-\ell_{ab})} + o_\Delta(1).
\]
We decompose this expression into two parts:
   \begin{equation}\label{eq:Tr decomposed}
   \E[T_r]\leq\frac{|\I_3|}{\Delta^3}+
   \frac{1}{3\Delta^3}\cdot
   \sum_{uvw\in \I_3}\sum_{ab\in\{uv,uw,vw\}}\left(\frac{2}{(2-\ell_{ab})(1-\ell_{ab})}-1\right)
   + o_\Delta(1).
   \end{equation}
   Consider the term 
   \[R:=\sum_{uvw\in \I_3}\sum_{ab\in\{uv,uw,vw\}}\left(\frac{2}{(2-\ell_{ab})(1-\ell_{ab})}-1\right).\]
   A pair $ab\in \I_2$ contributes to this double sum once for each $w\in N(r)$
   such that $uvw\in\I_3$.
   Since such a vertex $w$ cannot be a neighbour of $v$ and, by Claim~\ref{claim:no outside p3},
   each of the $\luv\Delta$ common neighbours of $u$ and $v$ are in $N[r]$,
   we know that there are
   at most $(1-\luv)\Delta$ such vertices $w$. It follows that
   \[
   R\leq \sum_{uv\in\I_2}(1-\luv)\Delta\cdot\left(\frac{2}{(2-\luv)(1-\luv)}-1\right)
   =\Delta\cdot\sum_{uv\in\I_2}\left(\frac{2}{2-\luv}+\luv-1)\right).
   \]
   It remains to write the sum as an expectation using $|\I_2|=(\sigma/2+o(1))\Delta^2$:
   \begin{equation}\label{eq:R}
   \frac{R}{\Delta^3}\leq
   \frac{\sigma}{2}\cdot\Eon{uv\in\I_2}\left(\frac{2}{2-\luv}+\luv-1\right)
   + o(1).
   \end{equation}
   The claim then follows from Equations~\eqref{eq:Tr decomposed} and~\eqref{eq:R}.
\end{proof}
We are now ready to conclude the proof of the theorem.
By Claims~\ref{claim:pairs} and~\ref{claim:triples},
\begin{align*}
        \E[P_r-T_r] &\geq
\sigma\cdot\Eon{uv\in \I_2}\left[\frac{1}{2-\ell_{uv}}\right]
-\left(\frac{|\I_3|}{\Delta^3}+
\frac\sigma6\cdot\Eon{uv\in \I_2}\left[
\frac{2}{2-\ell_{uv}}+\luv-1
\right]\right)+o(1)\\
&=
\frac\sigma6\cdot\Eon{uv\in \I_2}\left[
\frac{4}{2-\luv}-\luv+1
\right]
-\frac{|\I_3|}{\Delta^3}.
\end{align*}
The function $g:x\mapsto \frac{4}{2-x}-x+1$ is increasing on $[0,1]$,
so we may bound the last expectation by $g(0)=3$.
Further, a theorem from Rivin~\cite{Riv02} shows that $|\I_3|\leq\sigma^{3/2}\binom{\Delta}{3}$.
It follows that
\[
\E[P_r-T_r] \geq \frac{\sigma}{2}-\frac{\sigma^{3/2}}{6} + o(1)
= \eps_{\ref{col_result}}(\sigma) + o(1).
\]
Recall that as a consequence to Claim~\ref{claim:v in I},
the expected size of $\bI_r$ is $1-e^{-\gamma}+o_\Delta(1)$.
Using Equation~\eqref{eq:inclusion-exclusion}, we conclude that
\[
\frac{\Prob[\bI_r\neq\varnothing]}{\E[|\bI_r|]}\leq
\frac{\E[|\bI_r|-P_r+T_r]}{\E[\bI_r]}\leq
\frac{1-e^{-\gamma}-\eps_{\ref{col_result}}(\sigma)}{1-e^{-\gamma}}+o_\Delta(1)
=1-\eps_{\ref{col_result}}(\sigma)+o_\Delta(1)+o_\gamma(1),
\]
which proves the theorem.
\end{proof}

\begin{rem}
    As observed at the beginning of the section, the value of $\eps_{\ref{col_result}}$ is what we would expect to obtain if we activated the entire graph (used a single colour) and there were no correlations (created by common neighbours). Ultimately we proved above that small lists behave like lists of size 1 and that, given there are many triples, correlations help pairs more than they help triples, thus improving the colouring. 
\end{rem}
