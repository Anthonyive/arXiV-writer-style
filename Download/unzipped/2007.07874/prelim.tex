%!TEX root = paper.tex

\subsection{Graph theoretic notation and probabilistic preliminaries}
%\subsection{Probabilistic preliminaries}
\label{sub:prelim}
%\subsection{Notational conventions}
\label{sub:notation}


Throughout the paper we have adopted the following notation. 

For $k\in \N$, let $[k]$ denote the set $\{1,2,\dots,k\}$.

Given a graph $G$ and a vertex $v$, we write $N_G(v)$ for the (open) neighbourhood $\{u\in S:uv\in E(G)\}$ of $v$ in $S$ and $N_G[v]$ for the closed neighbourhood $N_G(v)\cup \{v\}$ of $v$ in $G$. 
The degree of $v$ in $G$ is denoted by $d_G(v)=|N_G(v)|$.
We usually drop the subscript when there is no ambiguity.

Given a graph $G$ and a vertex subset $S\subseteq V(G)$, we write $G[S]$ for the subgraph of $G$ induced by $S$.

Given a graph $G$, a list-assignment for $G$ is a map $L:V(G)\to 2^{\N}$, where $2^{\N}$ by convention denotes the set of all subsets of $\N$. We call a list-assignment $L$ a $k$-list-assignment if $|L(v)|=k$ for all $v\in V(G)$, i.e.~a $k$-list-assignment is a map $L:V(G)\to \binom{\N}{k}$, where $\binom{\N}{k}$ by convention denotes the set of all subsets of $\N$ of size $k$.
We call $L(v)$ the list of the vertex $v$. 
Given a list-assignment $L$ of $G$, a partial proper $L$-colouring of $G$ is a map $c:U\to \N$, where $U\subseteq V(G)$, such that $c(v)\in L(v)$ for all $v\in V(G)$ and $c(v)\neq c(w)$ for any $vw\in E(G)$.
We write $\dom(c)$ for the domain of $c$ and drop `partial' if $\dom(c)=V(G)$.
Note that the existence of a proper $L$-colouring for any a constant $k$-list-assignment $L$ of $G$ is equivalent to the assertion $\chi(G) \le k$.

Since we will be interested in gradually building up partial proper $L$-colourings, we introduce some terminology to describe the process.
Given a list-assignment $L$ and a partial proper $L$-colouring $c$ of $G$, the {\em residual subgraph} $G_c$ of $G$ with respect to $c$ is the induced subgraph $G[V(G)\setminus \dom(c)]$ and the {\em residual list-assignment} $L_c:G_c\to 2^{\N}$ is defined by $L_c(v) = L(v)\setminus c(N(v))$ for all $v\in V(G)$.
Note that if $c'$ is a proper $L_c$-colouring of $G_c$, then the union of the colourings $c$ and $c'$ is a proper $L$-colouring of $G$.




Our proofs rely on probabilistic methods, for which we require certain probabilistic tools. We use the following form of the Lov\'asz local lemma~\cite{ErLo75}.

\begin{slll}[\cite{ErLo75}]
    Let $p\in[0,1)$, and $\mathcal{A}$ be a finite set of ``bad'' events so that for every $A\in\mathcal{A}$ 
    \begin{itemize}
    \item $\Prob[A]\leq p$, and 
    \item $A$ is mutually independent of all but at most $d$ other events in $\mathcal{A}$.
    \end{itemize}
    If $4pd\leq 1$, then the probability that none of the (``bad'') events in $\mathcal{A}$ occur is strictly positive. 
\end{slll}



To help bound the probability of ``bad'' events in our application of the local lemma, we need to prove concentration of measure.
If $\Omega$ is a product of discrete spaces, we can define smoothness as the property that if $\omega\in\Omega$ and $\omega'\in\Omega$ differ in only one coordinate then $|X(\omega)-X(\omega')|<c$. Talagrand's inequality~\cite{Tal95} tells us that such smooth random variables are highly concentrated. 
However, some random variables that arise from our colouring procedure are not smooth and it is possible for one vertex to cause many others to be uncoloured. Fortunately, such a situation is highly unlikely, one might say exceptional, and can be handled by an adaption of Talagrand's inequality due to Bruhn and Joos~\cite{BrJo18}.

We can formalise this notion as follows, let $\Omega$ be a product space of discrete probability spaces and let $\Omega^*\subseteq \Omega$ be a set of \emph{exceptional} outcomes. We say that $X$ has downward $(s,c)$-certificates if for every for  $\omega \in \Omega\backslash\Omega^*$ we have an index set $I$ of size at most $s$ that identifies all the influences on $X(\omega)$ such that for another event $\omega' \in \Omega\backslash\Omega^*$ if $\omega|_{I}$ differs from $\omega'|_{I}$ in fewer than ${t}/{c}$ coordinates, then $X(\omega')\leq X(\omega)+t$. 
In other words, for an \emph{unexceptional} $\omega$ any increase in $X(\omega)$ comes from changes in a not too large set of coordinates indexed by $I$, and none of these coordinates increase $X(\omega)$ too much. 
\begin{theorem}[Bruhn and Joos~\cite{BrJo18}, cf.~Talagrand~\cite{Tal95}]\label{thm:downward certs}
Let $((\Omega_i,\sigma_i,\Prob_i))_{i=1}^n$ be discrete probability spaces, $(\Omega,\sigma,\Prob)$ be their product space and $\Omega^*\subseteq\Omega$ a set of \emph{exceptional} outcomes. Let $X:\Omega\to \R$ be a random variable, $M=\max\{\sup|X|,1\}$, and $c\geq 1$. If  $ \Prob[\Omega^*]\leq M^{-2}$ and $X$ has downward $(s,c)$-certificates then for $t>50c\sqrt{s}$,
\begin{equation}
    \Prob[|X-\E[X]|\geq t]\leq4e^{-\frac{t^2}{16c^2s}}+4\Prob[\Omega^*].
\end{equation}
\end{theorem}



