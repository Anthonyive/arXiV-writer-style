\documentclass[12pt]{amsart}
\usepackage{amsmath,amssymb,enumerate}
\usepackage{latexsym}
\usepackage{amsfonts}
\usepackage {amsbsy}
\usepackage {amsmath}
\usepackage{amssymb}
\usepackage{enumerate}
\usepackage{ bbm}

\newtheorem{theorem}{Theorem}
\newtheorem{lemma}[theorem]{Lemma}
\newtheorem{corollary}[theorem]{Corollary}
\newtheorem{proposition}[theorem]{Proposition}

\theoremstyle{definition}
\newtheorem{definition}[theorem]{Definition}
\newtheorem{example}[theorem]{Example}
\newtheorem{remark}[theorem]{Remark}


\numberwithin{theorem}{section}
\numberwithin{equation}{section}


\newcommand{\B}{\mathbb{B}}
\newcommand{\N}{\mathbb{N}}
\newcommand{\R}{\mathbb{R}}
\newcommand{\C}{\mathbb{C}}
\newcommand{\Cc}{\mathcal{C}}
\newcommand{\e}{\varepsilon}
\newcommand{\p}{\psi}
\newcommand{\al}{\alpha}
\newcommand{\ep}{\epsilon}
\newcommand{\Om}{\Omega}
\newcommand{\Omf}{\partial\Omega}
\newcommand{\Omb}{\bar{\Omega}}
\newcommand{\fami}{\mathcal{V}(\Om,\fii,f)}
\newcommand{\fii}{\varphi}
\newcommand{\D}{\Delta_H }
\newcommand{\U}{\mathtt{U}}
\newcommand{\QED}{\null\hfill\qedsymbol}
\newcommand{\Exp}{\text{Exp}}
\newcommand{\ca}{\text{Cap}}
\DeclareMathOperator{\tr}{tr}
\DeclareMathOperator{\diam}{diam}
\DeclareMathOperator{\supp}{supp}


\makeatletter
\@namedef{subjclassname@2010}{%
  \textup{2010} Mathematics Subject Classification}
\makeatother


\pagestyle{myheadings}

\author{Mohamad Charabati}
\address{Mohamad Charabati, Al-Baath University, 
 Homs  p. 77,  Syrian Arab Republic.}
\email{charabati.mohamad@gmail.com}
\smallskip

\author{Ahmed Zeriahi}
\address{Ahmed Zeriahi, Institut de Math\'ematiques de Toulouse, 
Universit\'e de Toulouse, 
CNRS, UPS, 118 route de Narbonne, 
31062 Toulouse cedex 09, France}
\email{ahmed.zeriahi@math.univ-toulouse.fr}

%\oddsidemargin 2ex
%\evensidemargin 2ex
%\textheight 7.9in
%\textwidth 6.2in

%Opening



\begin{document}



\thanks{The second author was partially supported by the ANR project GRACK}
  

\keywords{Complex Monge-Amp\`ere equations, complex Hessian equations, Dirichlet problem, subsolution,  capacity estimates.}

\subjclass[2010]{31C45, 32U15, 32U40, 32W20, 35J96}


\title[The continuous subsolution problem]{ The  Continuous Subsolution problem  \\ for Complex Hessian Equations }


\date{\today}

%\content

\maketitle

\tableofcontents


\begin{abstract}
 Let  $\Omega \Subset \C^n$ be a bounded strongly $m$-pseudoconvex domain ($1\leq m\leq n$) and  $\mu$ a positive Borel measure on $\Omega$. 
  We   study the complex Hessian equation $(dd^c u)^m \wedge \beta^{n - m} = \mu$ on $\Omega$. 
 
 First we give a sufficient condition on the measure $\mu$ in terms of its domination by the $m$-Hessian capacity which guarantees the existence of a continuous solution to the associated Dirichlet problem with a continuous boundary datum.
 
  As an application, we prove that if the equation has a continuous $m$-subharmonic subsolution whose modulus of continuity satisfies a Dini type condition, then the equation has a continuous solution with an arbitrary continuous boundary datum. Moreover when the measure has a finite mass, we give a precise quantitative estimate on the modulus of continuity of the solution. 
 
 One of the main steps in the proofs is to establish a new capacity estimate showing that  the $m$-Hessian measure of a continuous $m$-subharmonic function on $\Omega$ with zero boundary values is dominated by  an explicit function of the $m$-Hessian capacity with respect to $\Omega$, involving the modulus of continuity of $\varphi$. Another important ingredient is a new weak stability estimate on the Hessian measure of a continuous $m$-subharmonic function.
 % This gives generalisations of previous results obtained in \cite{KN18,KN19} and more recently in \cite{BZ20}. 
\end{abstract} 

\section*{Introduction}


Complex Hessian equations are important examples of fully non-linear PDE's of second order on complex manifolds.  They interpolate between (linear) complex Poisson equations ($m = 1$) and (non linear) complex Monge-Amp\`ere equations ($m=n$).
 They arize in many geometric problems, including the $J$-flow
\cite{SW} and quaternionic geometry \cite{AV}. They have attracted the attention of many researchers these last years. An account of the most relevant papers connected to this problem have been mentionned in \cite{BZ20}. We will not repeat them here and refer to this paper and the references therein.
 
 
 \subsection{Statement of the problem}
 
Let $\Omega \Subset \C^n$ be a bounded domain and $m$  a fixed integer such that $1 \leq m \leq n$. 
We consider the following general Dirichlet problem for the complex $m$-Hessian equation : 

\smallskip
\smallskip

{\it The Dirichlet problem:} Let $g \in \mathcal{C}^{0} (\partial \Omega)$ be a continuous function (the boundary  datum) and $\mu$ a  positive Borel measure  on $\Omega$ (the right hand side). The Dirichlet problem with boundary datum $g$ and right hand side $\mu$ consists in finding a function $ U \in \mathcal{SH}_m (\Omega) \cap C^{0} (\Omega)$  satisfying the following conditions :   

\begin{equation}\label{eq:DirPb}
\left\{\begin{array}{lcl} 
 (dd^c U)^m \wedge \beta^{n - m} = \mu,  &\hbox{on}\  \Omega, \, \, \, \, \, (\dag)\\
  U_{\mid  \partial \Omega} = g, & \hbox{on}\  \partial \Omega, \, \, \, (\dag \dag)
\end{array}\right.
\end{equation}
 
The  equation $(\dag)$ must be understood in the sense  of currents  on $\Omega$ (see section 2). 

The equality $ (\dag \dag)$ means that $\lim_{z \to \zeta} U (z) = g (\zeta)$ for any $\zeta \in \partial \Omega$. 

Observe that the comparison principle implies the uniqueness of the solution to the Dirichlet problem  (\ref{eq:DirPb})  when it exists. We will denote it by $U_{g,\mu} = U^{\Omega}_{g,\mu}$.

Recall the usual notations $d = \partial + \bar{\partial}$ and $d^c := (i\slash 2)  ( \bar{\partial} - \partial)$ so  that
$dd^c = i  \partial  \bar{\partial}$.  For  a real function $u \in \mathcal{C}^2 (\Omega)$, we denote for each integer $1 \leq k \leq n$ by $\sigma_k (u)$ the continuous function defined 	at each point $z \in  \Omega$ as the $k$-th symmetric polynomial of the eigenvalues $\lambda (z) := (\lambda_1 (z), \cdots, \lambda_n(z))$ 
of the complex Hessian matrix $  \left(\frac{\partial^2 u }{\partial z_j \partial \bar{z}_k} (z)\right)$ of $u$ i.e. 
$$
\sigma_k (u) (z) := \sum_{1 \leq j_1 < \cdots < j_k \leq n} \lambda_{j_1} (z) \cdots \lambda_{j_k} (z), \, \,  \, \, z \in \Omega.
$$
A simple computation shows that  
$$
(dd^c u)^k \wedge \beta^{n - k} = \frac{(n-k)! \, k!}{n!}  \, \sigma_k (u)  \,  \beta^n, \, \, 
$$
pointwise on $\Omega$ for $1 \leq k \leq m$, where $\beta := dd^c \vert z\vert^2$ is the usual K\"hler form on $\C^n$.


We say that a real function $u \in \mathcal{C}^2 (\Omega)$ is $m$-subharmonic on $\Omega$ if for any $1 \leq k \leq m$, we have $\sigma_k (u) \geq 0$ pointwise  on $\Omega$

Observe that  the function $u$ is $1$-subharmonic  on $\Omega$ ($m= 1$) if it is  subharmonic on $\Omega$ and $\sigma_1 (u) = (1 \slash 4) \Delta u$, while $u$ is  $n$-subharmonic  on $\Omega$ ($m = n$) if  $u$ is  plurisubharmonic on $\Omega$ and $\sigma_n (u)  = \mathrm{det}  \left(\frac{\partial^2 u }{\partial z_j \partial \bar{z}_k} (z)\right)$.


It was shown by Z. B\l ocki  in \cite{Bl05}, that it is possible to define a general notion of $m$-subharmonic function using the concept of $m$-positive currents (see section 2). Moreover, identifying positive $(n,n)$-currents with positive Radon measures, it is possible to define the $k$-Hessian measure $(dd^c u)^k \wedge \beta^{n - k}$ when $1 \leq k \leq m$ for any (locally) bounded $m$-subharmonic function $u$ on $\Omega$ (see section 2). 

\smallskip

Several questions related to the Dirichlet problem  (\ref{eq:DirPb}) will be addressed.

1. The first problem is to find a necessary and sufficient condition on $\mu$ which garantees the existence of a solution to the Dirichlet problem (\ref{eq:DirPb}).

2.  The second problem is to study the regularity of the solution $U^{\Omega}_{g,\mu}$ in terms of the regularity of the data $(g,\mu)$.

%3. The third problem is to study the stability of the solution $U^{\Omega}_{g,\mu}$ in terms of small perturbations of the data $(g,\mu)$.

\smallskip

When $\mu = 0$, the Dirichlet problem (\ref{eq:DirPb}) can be solved using the Perron method as for the complex Monge-Amp\`ere equation (see \cite{Bl05}, \cite{Ch16a}).

When $g = 0$ and $\mu$ is a positive Borel measure on $\Omega$, the Dirichlet problem is much more difficult. A necessary condition for the existence of a solution to (\ref{eq:DirPb}) is  the existence of a subsolution.  

For the complex Monge-Amp\`ere equation, S. Ko\l odziej proved that if the Dirichlet problem (\ref{eq:DirPb}) has a bounded subsolution, then it has a bounded solution (\cite{Kol95}). The same result for the Hessian equation was proved by  N. C. Nguyen in \cite{N13}.

The particular case of the Dirichlet problem (\ref{eq:DirPb}) we are interested in can be formulated as follows.    

\smallskip

 
{\it The continuous subsolution problem :}
 Let $\mu$ be a positive Borel measure on $\Omega$. Assume that there exists a continuous function $\varphi \in  \mathcal{SH}_m (\Omega) \cap \mathcal{C}^{0} (\bar{\Omega})$ satisfying the following conditions :
\begin{equation} \label{eq:subsolution}
\mu \leq (dd^c \varphi)^m \wedge \beta^{n - m}, \, \, \mathrm{on} \, \, \, \, \Omega,  
\, \, \, \mathrm{and} \, \, \varphi_{\mid  \partial \Omega} \equiv 0.
\end{equation}

$(i)$  Does the  Dirichlet problem (\ref{eq:DirPb}) admit a continuous  solution $U_{\mu,g}$ for any continuous boundary datum $g$? 
 
$(ii)$ In this case, is it possible to estimate  the modulus of continuity of the solution $U_{\mu,g}$ in terms of the modulus of continuity of  $\varphi$ and $g$ and some characteristic function related to $\mu$ ?

\smallskip

The continuous subsolution problem stated above has attracted a lot of attention these last years. It was formulated by Ko\l odziej  for the complex Monge-Amp\`ere equation and  adressed in \cite{DGZ16} in the case of the existence of H\"older continuous subsolution. 

 There has been many articles on the subject.  The H\"older continuous subsolution problem was solved very recently for positive Borel measures with finite mass in \cite{BZ20} . For more details on the previous results  on this problem, we refer to \cite{BZ20} and the references therein.

Recently S. Ko\l odziej and  N.C. Nguyen gave a  Dini type sufficient condition on the modulus of continuity of the subsolution which garantees the existence of a continuous plurisubharmonic solution for the complex Monge-Amp\`ere equation (\cite{KN18}).

\subsection{Main results}
Our main goal in this paper is to give a partial answer to the  "Continuous Subsolution Problem". Namely, we will give  sufficient condition of Dini type on the modulus of  continuity of the subsolution $\varphi$ which garantees the existence of a  continuous solution to  the  Dirichlet problem (\ref{eq:DirPb}). Moreover we will  give a precise estimate of the modulus of continuity of the  solution $U_{\mu,g}$ when $\mu$ has finite mas.

We will improve and  extend the result of \cite{KN18} to the Hessian equation using an original idea from \cite{KN19}, some new ideas from \cite{BZ20} and an idea from a  former  preliminary draft of a project  which has not been completed. Moreover our main result improves the H\" older continuous subsolution theorem obtained  in \cite{BZ20}. The terminology will be defined in the next section. 

 \smallskip
 \smallskip
 Our first main result gives a sufficient condition on the Borel measure $\mu$ in terms of its diffusion with respect to the $m$-Hessian capacity which garantees the existence of a continuous solution to the Dirichlet problem (\ref{eq:DirPb}).
 
  \smallskip
 \smallskip
 
{\bf Theorem 1}. {\it  Let  $\Omega \Subset \C^n$ be a bounded strongly $m$-pseudoconvex  domain and $\mu$ be a positive Borel measure on $\Omega$ with finite mass. Assume that  $\mu$  is diffuse  with respect to the $m$-Hessian capacity i.e. there exists a constant $A > 0$ such that for any compact set $K \subset \Omega$,
$$
\mu (K) \leq A \text{c}_m (K) \gamma (\text{c}_m (K)),
$$
where $\gamma : \R^+ \longrightarrow \R^+$ is a continuous increasing function on $\R^+$ which  satisfies  the following  Dini type condition
 \begin{equation} \label{eq:DiniConditionMu}
\int_{0^+} \frac{\gamma (t)^{1 \slash m}}{t} d t < + \infty.
 \end{equation}
 Then   for any    continuous boundary datum $g \in \mathcal C^0 (\partial \Omega)$, the Dirichlet problem (\ref{eq:DirPb}) admits a unique solution  $U = U_{\mu,g} \in \mathcal{SH}_m (\Omega) \cap \mathcal C^0  (\bar{\Omega})$.}

The capacity $\text{c}_m (K) = \text{c}_m (K,\Omega)$ will be defined in the next section. 

Our second main result gives a new comparison inequality which will be applied to  positive Borel measures without restriction on their support nor on their mass. 

Let us fix $0 < r < m \slash (n-m)$ and $0 < b < 2 n$ and define the following functions for $t \in \R^+$:
\begin{equation}\label{eq:estimatefunction}
\ell_m (t) := \left\{\begin{array}{lcl} 
 t^{r}, \,  \, \, \, \, \, \, \, \, \, \,   \, \, \, \,  \, \, \, \, \, \, \, \, \, \,    \,  \, \, \, \, \, \, \hbox{if} \, \,  \, \, \, 1 \leq m < n, \\
   \exp ( -b \, t^{- 1 \slash n}), \, \, \,  \hbox{if} \, \, \, \, \,  m = n.
\end{array}\right.
\end{equation}
 
 \smallskip
 
 {\bf Theorem 2}.{ \it Let $\Omega \Subset \C^n$ be a bounded $m$-hyperconvex  domain and $\varphi \in \mathcal{SH}_m (\Omega)  \cap \mathcal{C}^0 (\bar{\Omega})$ with $\varphi = 0$ on $\partial \Omega$. 
 
 Then  there exists a constant $B = B (m,n, \varphi,\Omega) > 0$ such that for any compact set $K \subset \Omega$, 
$$
\int_{K}(dd^c\varphi)^m\wedge\beta^{n-m}  \leq  B \, \left\{\vartheta_m  (c_m (K)) + \left[\vartheta_m  (c_m (K))\right]^m\right\} \, c_m (K),
$$
where
$\vartheta_m (t) := \kappa_\varphi \circ \theta_m \circ \ell_m (t)$ , $\kappa_\varphi$ is the modulus of continuity of $\varphi$ and $\theta_m$ is the lower inverse  of the function $t \longmapsto t^{2m} \kappa_\varphi (t)^{1 - m}$ . }

 \smallskip
 \smallskip
  The constant $B$ in the theorem is explicit (see  (\ref{eq:finalConst})).
 
 Theorem 2  generalizes the estimate proved in \cite{BZ20} in the H\"older continuous case. 



 \smallskip
 \smallskip


 As a consequence of Theorem 1 and Theorem 2, we will deduce the following two results which solves the continuous subsolution problem under a Dini type condition on the modulus of continuity of the subsolution.

Since the two results are different for complex Monge-Amp\`ere equations and Hessian equations, we will state them separately.
 \smallskip
 
 \smallskip
  {\bf Theorem 3}.{ \it Let $\Omega \Subset \C^n$ be a  bounded strongly $m$-pseudoconvex  domain  with $1 \leq  m < n$ and $\mu $ a positive Borel measure on $\Omega$. 
  
  Assume that there exists $\varphi\in \mathcal{SH}_m(\Omega)\cap\mathcal{C}^{0}(\overline\Omega)$ such that  
\begin{equation} \label{eq:subsol2}
 \mu \leq (dd^c\varphi)^m\wedge\beta^{n-m}, \, \, \, \mathrm{weakly \, \,  on} \, \,  \Omega \, \, \, \mathrm{and} \, \, \varphi_{\mid  \partial \Omega} \equiv 0,
\end{equation} 
 and the modulus of continuity $\kappa_\varphi$ of $\varphi$ satisfies the following Dini type condition:
 \begin{equation} \label{eq:DC1}
 \int_{0^+} \frac{\left[\kappa_\varphi (t)\right]^{1 \slash m}}{t}  d t< + \infty, 
 \end{equation}


 Then for any continuous function $g \in \mathcal{C}^{0} (\partial \Omega)$, there exists a unique function $U = U_{g,\mu} \in \mathcal{SH}_m (\Omega) \cap \mathcal{C}^0 ({\Omega})$ such that  
 $$
 (dd^c U)^m\wedge\beta^{n-m} = \mu, \, \, \, \mathrm{weakly \, \,  on} \, \,  \Omega \, \, \, \mathrm{and} \, \, \, U_{\mid  \partial \Omega}  = g.
 $$


Moreover if $\mu (\Omega) < + \infty$, the $\widehat{\kappa}$-modulus of continuity of $U$ satisfies the following esstimate
 
$$
\widehat{\kappa}_U (\delta) \leq C  \, \widehat{\kappa}_m (\delta),
$$ 
 where  $\widehat{\kappa}_m (\delta)$ is given by the equation (\ref{eq:MC-Solution}) and $C = C(m,n, \mu, \Omega) > 0$ is a uniform constant.} 
 
  \smallskip
 \smallskip
 
 For Complex Monge-Amp\`ere equations (the case $m = n$) we obtain a much better result.
 % du to the fact that in this case we have a strong comparison inequality between volume and the Monge-Amp\`ere capacity.
   
 \smallskip
 \smallskip
 
 {\bf Theorem 4}. { \it Let $\Omega \Subset \C^n$ be a  bounded strongly pseudoconvex  domain and $\mu $ a positive Borel measure on $\Omega$. 
 
 Assume that there exists $\varphi\in \mathcal{PSH} (\Omega)\cap\mathcal{C}^{0}(\overline\Omega)$ such that  
\begin{equation} \label{eq:subsol1}
 \mu \leq (dd^c\varphi)^n\, \, \, \, \mathrm{weakly \, \, on} \, \,  \Omega \, \, \, \mathrm{and} \, \, \varphi_{\mid  \partial \Omega} \equiv 0.
\end{equation} 
 and  the modulus of continuity $\kappa_\varphi$ of $\varphi$ satisfies the following Dini type condition:
 \begin{equation} \label{eq:DC2}
 \int_{0^+} \frac{\left[\kappa_\varphi (t)\right]^{1 \slash n}}{t \vert \log t \vert} d t< + \infty. 
 \end{equation}


 Then for any continuous function $g \in \mathcal{C}^{0} (\partial \Omega)$, there exists a unique function $U = U_{g,\mu} \in \mathcal{PSH} (\Omega) \cap  \mathcal{C}^0 ({\Omega})$ such that  
 $$
 (dd^c U)^n  = \mu, \, \, \, \mathrm{weakly \, \, on} \, \,  \Omega, \, \,  \, \, \, \mathrm{and} \, \, \, U_{\mid  \partial \Omega}  = g.
 $$
 
 
 Moreover if $\mu (\Omega) < + \infty$, the $\widehat{\kappa}$-modulus of continuity of $U$ satisfies the following estimate 
$$
\widehat{\kappa}_U (\delta) \leq C  \, \widehat{\kappa}_n (\delta),
$$ 
where  $\widehat{\kappa}_n (\delta)$ is given by the equation (\ref{eq:MC-Solution}) with $m = n$ and $C = C(n, \mu,\Omega) > 0$ is a uniform constant.} 
 
  
 \smallskip
 \smallskip
 
 Here the $\widehat{\kappa}$-modulus of continuity of  a given function $\phi:\Omega \longrightarrow \R$ is defined for $0 < \delta <\delta_0$  by
 \begin{equation}
\widehat{\kappa}_{\phi} (\delta) := \sup_{z \in \Omega_{\delta}} (\widehat{\phi}_\delta (z) - \phi (z)), \, \, \, 
\end{equation}
where
\begin{equation} 
\widehat{\phi}_\delta (z) := \int_{\B} u (z + \delta \xi) d \lambda_{\B} (\xi), 
\end{equation} 
for $z \in \Omega_\delta := \{z \in \Omega ; \text{dist} (z,\partial \Omega) > \delta \}$ and $0 < \delta < \delta_0$.

 
  \smallskip
 \smallskip

 The existence of a continuous solution under the Dini type condition in Theorem 4 was proved recently in (\cite{KN18}) by a slightly different method. However our  estimate on the modulus of continuity of the solution improves the result of  \cite{KN18}  where the measure $\mu$  is supposed to have a compact support.
   
 
 \subsubsection{Organization of the paper}
 
After raising the problem of continuous subsolutions and stating the main results in the introduction, we give in section 1  the necessary definitions and preliminaries that will be needed in the sequel. 

 \smallskip
 
Section 2 contains news results which will play a crucial role in the proofs of our main resultas. We first give a new estimate on the behaviour near the boundary of a domain of the $m$-Hessian measure of a potential in terms of the $m$-Hessian capacity on the domain. Then  we prove  continuity properties of these measures acting on normalized potentials.

 \smallskip
 
 Section 3 contains the proof of Theorem 1. We first give a priori uniform estimates and then prove continuity of the Hessian potentials of  Borel measures which are diffuse with respect to the corresponding capacity (see definition 3.1). Then we  establish a new stability estimate that improves the one obtained  in \cite{BZ20} under a weaker domination condition on the measure. This estimate is inspired by  an estimate proved  in \cite{BGZ08} in the spirit of \cite{EGZ09} in the case of compact K\"ahler manifolds.

 \smallskip
 
Section 4 contains the proof of Theorem 2 as well as some consequences.  The proof of this theorem  consists in extending the analoguous result proved in \cite{BZ20} in the H\"older continuous case.

 \smallskip
 
Section 5 contains the  proofs of Theorem 3 and Theorem 4. These proofs are done at the same time  in several steps following the same scheme.   We first use the weak stabilty result Theorem \ref{thm:stability} to reduces the estimation of the modulus of continuity of the solution to the Dirichlet problem  (\ref{eq:DirPb}) to the estimate of the $L^m$-norm of the difference of two normalized potentials with respect to the measure $\mu$ using its domination by  the Hessian measure of the subsolution. Then we use results from Section 3 to estimate the  $L^m$-norm with respect to $\mu$ in terms of the $L^m$-norm with respect to the Lebesgue measure following a scheme which has become standard and which was initiated in \cite{EGZ09} and completed in  \cite{GKZ08}.

 \smallskip
 %The proof of Corollary 5 uses the same method and the additional information  specific to   the complex Monge-Amp\`ere operator, namely  the exponential domination of volumes of Borel sets by their Monge-Amp\`ere capacity proved in \cite{ACKPZ09}.

 Section 6 contains an extension of the main results by dropping the assumption on boundary values of the subsolution as well as an example of  application.

\section{Preliminary results}
 In this section, we recall the basic properties of $m-$subharmonic functions and some known results we will use  throughout the paper. 
 
\subsection{Hessian potentials}
 For a hermitian $n \times n$ matrix $a = (a_{j,\bar k})$ with complex coefficients, we denote by $\lambda_1, \cdots \lambda_n$ the eigenvalues of the matrix $a$. For any $1 \leq k \leq n$ we define the $k$-th trace of $a$ by the formula

$$
S_k (a) := \sum_{1 \leq j_1 < \cdots < j_k \leq n} \lambda_{j_1} \cdots \lambda_{j_k},
$$
which is the $k$-th elementary symetric polynomial of the eigenvalues $(\lambda_1, \cdots, \lambda_n)$ of $a$.

 Recall that $d = \partial + \bar{\partial}$ and define $d^c := i (\bar{\partial} - \partial$ so that $dd^c = 2 i \partial \bar{\partial}$ and denote by 
 $$
 \beta := dd^c \vert z\vert^2
 $$
  the standard K\"ahler form on $\C^n$.
 
 Let $\C^n_{(1,1)} $ be the space of real $(1, 1)$-forms on $\C^n$  with constant
coefficients, and define the cone of $m$-postive  $(1,1)$-forms on $\C^n$ by

$$
\Theta_m := \{\omega \in \C^n_{(1,1)}  \, ; \,  \omega \wedge  \beta^{n - 1} \geq 0, \cdots,  \omega^m \wedge  \beta^{n - m} \geq 0\}.
$$

 
\begin{definition}
1) A smooth $(1,1)$-form $\omega$ on $\Omega$ is said to be $m$-postive on $\Omega$ if for any $z \in \Omega$, $\omega (z) \in \Theta_m$.

2) A function $u:\Omega \rightarrow \mathbb{R}\cup\{-\infty\}$ is said to be  $m-$subharmonic  on $\Omega$ if it is subharmonic on $\Omega$ (not identically $-\infty$ on any component) and  for any collection of smooth $m-$positive $(1,1)-$forms  $\omega_1,...,\omega_{m-1}$ on $\Omega$, the following inequality holds in the sense of currents 
  $$
  dd^c u\wedge \omega_1\wedge...\omega_{m-1} \wedge \beta^{n-m}\geq 0,
  $$
  in the sense of currents on $\Omega$.
\end{definition}

We denote by $\mathcal{SH}_m (\Omega) $ the positive convex cone of $m$-subharmonic functions on $\Omega$ which are not identically $-\infty$ on any component of $\Omega$.  These are the $m$-Hessian potentials.

We give below the most basic properties of $m$-subharmonic functions that will be used in the sequel (see \cite{Bl05}, \cite{Lu12}).

\begin{proposition}\label{prop:basic}

\noindent 1.  If $u\in \mathcal{C}^2(\Omega)$, then $u$ is  $m$-subharmonic on $\Omega$ if and only if $(dd^c u)^k\wedge \beta^{n-k}\geq0$
pointwise on $\Omega$ for $k=1, \cdots, m$.

 \noindent 2. $\mathcal{PSH}(\Omega)=\mathcal{SH}_n(\Omega)\subsetneq \mathcal{SH}_{n-1}(\Omega)\subsetneq...\subsetneq \mathcal{SH}_1(\Omega)=\mathcal{SH}(\Omega) $.
 
\noindent 3.  $\mathcal{SH}_m(\Omega) \subset L^1_{loc} (\Omega)$ is a positive convex cone. 
  
\noindent 4.  If $u$ is $m$-subharmonic on $\Omega$ and $f: I \rightarrow\mathbb{R}$ is a  convex, increasing function on some interval containing the image of $u$, then $f \circ u$ is $m$-subharmonic on $\Omega$.
  
  
\noindent 5. The limit of a decreasing sequence of  functions in $\mathcal{SH}_m(\Omega)$ is $m$-subharmonic on $\Omega$ when it is not identically $- \infty$ on any component.
 
\noindent 6.  Let $u$ be an $m$-subharmonic function on $\Omega$. Let $v$ be an $m$-subharmonic function on a domain $\Omega'  \subset \C^n$ with $\Omega \cap \Omega' \neq \emptyset$. If $u\geq v$ on $\Omega \cap \partial\Omega'$, then the function
   $$
   z \mapsto w(z):=\left\{\begin{array}{lcl}
\max(u(z),v(z)) &\hbox{ if}\ z \in \Omega \cap \Omega'\\
 u(z)  &\hbox{if}\  z \in\Omega\setminus\Omega'\\
\end{array}\right.
$$
is $m$-subharmonic on $\Omega$.

 \end{proposition}
 
 \subsection{Approximation of Hessian potentials}
 Another ingredient which will be important is the regularization process.  Let $\chi$ be a fixed  positive radial Borel function with compact support in the unit ball $\B \subset \C^n$ and $\int_{\mathbb{C}^n}\chi (\zeta)d\lambda_{2 n}(\zeta)=1$.
   For any $ 0 < \delta < \delta_0 := \mathrm{diam} (\Omega)$, we set 
   $\chi_{\delta}(\zeta)=\frac{1}{\delta^{2n}}\chi (\frac{\zeta}{\delta})$ and $\Omega_{\delta}=\{z \in\Omega;  \mathrm{dist} (z,\partial\Omega)>\delta\}$.
 
 Let $u \in \mathcal{SH}_m (\Omega) \subset L^1_{loc} (\Omega)$ and define its standard $\delta$-regularization by the formula
      
  \begin{equation} \label{eq:reg}
   u \star {\chi}_\delta  (z) := \int_{\Omega} u (z - \zeta) \chi_{\delta} (\zeta) d \lambda_{2n} (\zeta), z \in \Omega_{\delta}.
  \end{equation}
  Then it is easy to see that $ {u}_{\delta}$ is $m$-subharmonic and smooth on $\Omega_{\delta}$ and decreases to $u$ in $\Omega$ as $\delta $ decreases to $0$.
  
  Observe that when $\chi= \chi^\B  := (1/\tau_{2n})  {\bf 1}_\B$ is the normalized characteristic function of the unit ball, then $u \star \chi = \widehat{u}_\delta$ is the mean-value function of $u$  defined on $\Omega_\delta$  by
  $$
   \widehat{u}_\delta (z) := (1\slash \tau_n) \int_\B u (z +\delta \zeta) d \lambda_{2n} (\zeta), \, \, z \in \Omega_\delta,
  $$
where $\tau_n := \lambda_{2n} (\B)$.
\begin{lemma}   \label{lem:Poisson-Jensen}
Let $u \in \mathcal{SH}_m (\Omega) \cap L^{1} (\Omega)$. Then for $0 < \delta < \delta_0$, its $\delta$-regularization extends to $\C^n$ by the formula
      
  \begin{equation} \label{eq:reg}
 u \star  {\chi}_\delta u (z)  := \int_{\Omega} u (\zeta) \chi_{\delta} (z - \zeta) d \lambda_{2n} (\zeta), z \in \C^n,
  \end{equation}
and have the following properties
 
 1)  $ { u}_{\delta}$  is $m$-subharmonic on $\Omega_\delta$, smooth on $\C^n$ provided that $\chi$ is smooth;
 
     
 2) $({u}_\delta)$ decreases to $u$ in $\Omega$ as $\delta $ decreases to $0$; 
 
 
 3) the mean-value function $\widehat{u}_\delta $ satisfies the estimate
  $$
  \int_{\Omega_\delta} \left(\widehat{u}_\delta (z) - u(z)\right) d\lambda_{2 n}(z) \leq a_n \delta^2 \int_{\Omega_\delta} dd^c u \wedge \beta^{n - 1},
  $$ 
  where $a_n > 0$ is a uniform constant which does not depend on $u$ nor on $\delta$.


\end{lemma}
\begin{proof}
The first and second property are clear. The third one follows from Poisson-Jensen formula for subharmonic functions (see \cite{GKZ08}, \cite{Ze20}).
\end{proof}

Let us introduce  the notions of $m$-pseudoconvexity that will be used in the sequel.

\begin{definition} 1.  We say that the open set $\Omega \Subset \C^n$  is  $m$-hyperconvex  if it admits a defining function $\rho : \Omega \longrightarrow ]-\infty , 0[$ which is a bounded continuous $m$-subharmonic on $\Omega$ (see \cite{Lu12,Lu15}. 

2. We say that the open set $\Omega \Subset \C^n$  is  strongly $m$-pseudoconvex if $\Omega$ admits a smooth defining function $\rho$ which is strictly $m$-subharmonic in a neighbourhood of $\bar \Omega$ such that  $\vert \nabla \rho\vert > 0$ on $\partial \Omega = \{\rho = 0\}$. In this case we can choose $\rho $ so that
\begin{equation} \label{eq:stronpconvexity}
(dd^c \rho)^k \wedge \beta^{n - k} \geq \beta^n \, \, \mathrm{for} \, \, 1 \leq k \leq m,
\end{equation}
pointwise on $\Omega$.
\end{definition}

 \begin{example}
1.  Any euclidean ball in $\C^n$ is strongly  $m$-pseudoconvex and any polydisc in $\C^n$ is $m$-hyperconvex but not strongly  $m$-pseudoconvex.

2. The domain $\{ z \in \C^n ; \sum_{1 \leq j \leq n} \vert z_j\vert < 1 \}$ is a bounded  $m$-hyperconvex domain with Lipschitz but not smooth boundary, hence it not strongly  $m$-pseudoconvex (see \cite{Ch16a} for more examples. 
\end{example}

The following lemma will be also needed. 
 For a function $g \in \mathcal{C}^0 (\partial \Omega)$, we denote by   $\mathcal{SH}^g_m (\Omega)$ the set of functions $w \in \mathcal{SH}_m (\Omega) \cap L^{\infty} (\Omega)$ such that $w =g$ on $\partial \Omega$ i.e.  for any $\zeta \in \Omega$, $\lim_{z \to \zeta} w (z) =  g (\zeta)$.
 
\begin{lemma} \label{lem:appximationwithbdv} Let $g \in \mathcal{C}^0 (\partial \Omega)$ and   $w \in \mathcal{SH}_m^g (\Omega)$. Then there exists a decreasing sequence $(w_j)$ of functions in $\mathcal{SH}_m^g  (\Omega) \cap C^0 (\bar{\Omega})$ which converges to $w$ on $\Omega$.
\end{lemma}
\begin{proof} First take any decreasing sequence  of continuous functions $(h_j)$ on $\bar{\Omega}$ which converges to $w$ on $\bar{\Omega}$. We can arrange so that $h_j = g$ on $\partial \Omega$. Indeed take the harmonic extension $G$ of $g$ to $\Omega$ and then the sequence $\min \{h_j,G\}$ satifies the requirement. 

Now  set
$$
w_j := \sup \{v \in \mathcal{SH}_m (\Omega) ; v \leq h_j\}.
$$ 
 By \cite{BZ20}, we know that the sequence $(w_j)$ satisfies all the requirements of the lemma.
\end{proof}

\subsection{Remarks on the modulus of continuity}
%Given a continuous function, there are  various modulus of continuity that are used in the literature but they were not compared until recently (see \cite{Ze20}).


 Let  $\phi : {\Omega} \longrightarrow \R$ be a continuous function.  We fix $\delta_0 > 0$ so that $\Omega_{\delta_0} \neq \emptyset$ and recall the following definition for $0 < \delta < \delta_0$ and $z \in \Omega_\delta$,
\begin{equation} \label{eq:supnorm}
\widehat{\phi}_\delta  (z) := \int_{\bar{\B}} \phi (z + \delta \zeta) d \lambda_\B (\zeta),  \, \, \, 
\end{equation}
where $\lambda_\B$ is the normalized Lebesgue measure on $\B$.

 We introduce the  modulus of (uniform)  continuity of $\phi$ on ${\Omega}$ defined for $\varepsilon> 0$ by the formula
  \begin{equation} \label{eq:fullkappa}
  \kappa_\phi (\varepsilon) := \sup \{\vert \phi(z) - \phi (z')\vert \, ; \, z, z' \in {\Omega}, \vert z-z'\vert \leq \varepsilon\}.
  \end{equation}
  Then $\phi$ extends to a uniformly continuous function on $\bar{\Omega}$ if and only if $\lim_{\varepsilon \to 0^+} \kappa_\varphi (\varepsilon) = 0$.
  
We introduce another  characteristic function defined  for $0 < \delta < \delta_0$ by the formula
\begin{equation} \label{eq:hatkappa}
\widehat{\kappa}_\phi (\delta) := \sup_{\Omega_\delta} \left(\widehat{\phi}_\delta (z) - \phi (z)\right).
\end{equation}

We see immediately that $\widehat{\kappa}_\phi (\delta) \leq  \kappa_\phi (\delta)$ for any $0< \delta < \delta_0$. 

These  characteristics quantify the continuity of $\phi$ on $\Omega$. While the (full) modulus of  continuity $ \kappa_\phi$ characterizes uniform continuity of $\phi$ on $\bar{\Omega}$, the (partial) modulus of continuity $\widehat{\kappa}_\phi$ only characterizes the continuity of $\phi$ on $\Omega$. Indeed  the condition $\lim_{\delta \to 0^+} \widehat{\kappa}_\phi (\delta) = 0$  implies that the function $\phi$ is continuous, but it does not imply the extension of the function $\phi$ by continuity to $\bar{\Omega}$ as the example of a harmonic function on $\Omega$ shows. 

We will state a result from \cite{Ze20}  which clarifies the relations between these notions of continuity in some cases.

We need some definitions. 
\begin{definition} 1.  A continuous  function $\kappa : [0,t_0] : \longrightarrow \R^+$ is a modulus of continuity if it is  increasing,  subadditive and $\kappa (0) =0$.

2. A function  $\phi : \Omega \longrightarrow \R$ is said to be $\kappa$-continuous near the boundary $\partial \Omega$ if there exists $0 < \delta_1 < \delta_0$ small enough and a constant $M_1 > 0$ such that for any $\zeta \in \partial \Omega$ and any $z \in \Omega$ with $\vert z - \zeta\vert \leq \delta \leq \delta_1$, we have $\vert u (z) - u(\zeta) \vert \leq M_1  \kappa (\delta)$.
\end{definition}
Uniform continuity on $\bar{\Omega}$ implies uniform continuity near the boundary $\partial \Omega$. However as observed above,  the condition $\lim_{\delta \to 0} \widehat{\kappa}_\phi (\delta) = 0$ implies the continuity of $\phi$ on $\Omega$ but it does not imply continuity near the boundary $\partial \Omega$.

 
We first introduce the following condition on $\kappa$.
\begin{equation} \label{eq:MCcondition}
\exists A > 1, \, \, \limsup_{t \to 0^+} \frac{\kappa (A t)}{A \kappa (t)} <  \frac{1}{2n}.
\end{equation}
Observe that  the condition (\ref{eq:MCcondition}) is satisfied by any logarithmic H\"older modulus of continuity $\kappa_{\alpha,\nu}  (t) := t^\alpha (-\log t)^{\nu}$ for $0 < t < < 1$ with $0\leq  \alpha < 1$ and $\nu \in \R$, with $\nu < 0$ when $\alpha =0$..

The following lemma  is proved in  \cite{Ze20}. 
\begin{lemma} \label{lem:sup-mean}  Let $\kappa$ be a modulus of continuity satisfying (\ref{eq:MCcondition}).  Let $\Omega \Subset \C^n$ be a bounded domain and $u \in  \mathcal{SH} (\Omega) \cap L^{\infty} ({\bar \Omega})$. Assume that $u$ is $\kappa$-continuous  near $\partial\Omega$. Then the following properties are equivalent:

$(i)$ $\exists c_1 >0$, $\exists  \, 0 < \delta_1 < \delta_0$ such that for any $0< \delta < \delta_1$
$$
\widehat{u}_\delta  (z)   \leq u  (z) +  c_1 \kappa (\delta), \, \, \, \text{for  any } \, \, \, z \in \Omega_\delta,
$$ 

$(ii)$ $\exists c_2 >0$, $\exists  \, 0 < \delta_2 < \delta_0$ such that for any $0< \delta < \delta_2$,
$$
\sup_{\bar B(z,\delta)} u \leq  u (z)+  c_2 \kappa (\delta),  \, \, \, \text{for  any } \, \, \, z \in \Omega_\delta.
$$

\smallskip

Moreover if one of these conditions is satisfied then $u$ is $\kappa$-continuous on $\bar{\Omega}$ i.e. $\kappa_u \leq C \kappa$, where $C > 0$ is a uniform constant.
\end{lemma}
 
\subsection{Complex Hessian operators}

Following \cite{Bl05}, we can define the Hessian operators acting on (locally) bounded $m$-subharmonic functions as follows. 
%Observe that if $u \in \mathcal{SH} (\Omega)$ then $dd^u \wedge \beta^{n-m}$ is a well defined $(m-1,(m-1)$-positive current on $\Omega$.  
 Given $u_1, \cdots, u_k \in \mathcal{SH}_m (\Omega) \cap L^{\infty} (\Omega)$ ($1 \leq k \leq m$), one can define inductively the following  positive $(m-k,m-k)$-current on $\Omega$
$$
dd^c u_1 \wedge \cdots \wedge dd^c u_k \wedge \beta^{n - m} := dd^c (u_1 dd^c u_2 \wedge \cdots \wedge dd^c u_k \wedge \beta^{n - m}).
$$
%When $k = m$ we obtain a positive $(n,n)$-current on $\Omega$ which can be identified to a Borel measure on $\Omega$.

In particular, if $u  \in \mathcal{SH}_m (\Omega) \cap L^{\infty}_{loc} (\Omega)$, the positive current $(dd^c u)^m \wedge \beta^{n-m}$ can be identifed to a  positive Borel measure on $\Omega$, the so called $m$-Hessian measure of $u$ denoted by:
$$
\sigma_m (u) := (dd^c u)^m \wedge \beta^{n-m},
$$

Observe that when $m= 1$,  $\sigma_1 (u) = dd^c u \wedge \beta^{n-1} $ is the Riesz measure of $u$ (up to a positive constant), while  $\sigma_n (u) = (dd^c u)^n$ is the complex   Monge-Amp\`ere measure of $u$ on $\Omega$. 

It is then possible to extend Bedford-Taylor theory to this context. 
In particular, Chern-Levine Nirenberg inequalities hold and the Hessian operators are continuous under local uniform convergence and  monotone convergence 
pointwise a.e. on $\Omega$ of sequences of functions in  $\mathcal{SH}_m (\Omega) \cap L^{\infty}_{loc} (\Omega)$ (see \cite{Bl05}, \cite{Lu12}).

We define  $\mathcal{E}_m^0 (\Omega) $ to be the positive convex cone of  negative  functions  $\phi \in \mathcal{SH}_m (\Omega) \cap L^{\infty} (\Omega)$ such that
$$
\int_{\Omega} (dd^c \phi)^m \wedge \beta^{n - m} < + \infty, \, \, \phi_{|\partial \Omega} \equiv 0.
$$
These are the "test functions" in  $m$-Hessian Potential Theory in the sense that Stokes theorem is valid for these functions (see \cite{Lu12}).

More precisely it follows from \cite{Lu12,Lu15} that the following property hlods:  if $\phi \in  \mathcal{E}_m^0 (\Omega) $ and $u , v \in \mathcal{SH}_m (\Omega) \cap L^{\infty} (\Omega)$ with $u \leq 0$, then for $0 \leq k \leq m - 1$, 

\begin{equation} \label{eq:testinequality}
\int_\Omega (-\phi)  dd^c u \wedge (dd^c v)^k \wedge \beta^{n - k-1} \leq \int_\Omega (-u)  dd^c \phi \wedge (dd^c v)^k \wedge \beta^{n - k-1}. 
\end{equation}


An important tool in the corresponding Potential Theory is the Comparison Principle.

\begin{proposition} \label{prop:Comparison Principle}
 Assume that $u,v\in \mathcal{SH}_m(\Omega)\cap L^{\infty}(\Omega)$ and for any $\zeta \in \partial \Omega$, $\liminf_{z \rightarrow \zeta }(u(z)- v(z))\geq 0$.  Then 
 $$
 \int_{\{u<v\}}(dd^c v)^m\wedge\beta^{n-m} \leq \int_{\{u<v\}}(dd^c u)^m\wedge\beta^{n-m}.
 $$
 Consequently, if $(dd^cu)^m\wedge\beta^{n-m}\leq(dd^cv)^m\wedge\beta^{n-m}$ weakly on $\Omega$, then $u \geq v$ in $\Omega$.
 
\end{proposition}
It follows from the comparison principle that if the Dirichlet problem (\ref{eq:DirPb}) admits a solution, then it is unique.



Let us recall the following estimates due to Cegrell (\cite{Ceg04}) for complex Monge-Amp\`ere operators and extended by Charabati to  complex Hessian operators (\cite{Ch16a}).


\begin{lemma} \label{lem:Cegrell} Let  $u, v, w \in\mathcal{E}_m^0(\Omega)$. Then for any $1 \leq k \leq m - 1$
   
 $$
    \begin{array}{lcl}
 \int_{\Omega}dd^cu\wedge(dd^cv)^k\wedge(dd^cw)^{m-k-1}\wedge\beta^{n-m}
     \leq  I_m (u)^{\frac{1}{m}} \, I_m (v)^{\frac{k}{m}} \,  I_m (w)^{\frac{m-k-1}{m}},
  \end{array}
 $$
 where $I_m (u) :=  \int_{\Omega}(dd^c u)^m \wedge \beta^{n-m}$.
 
 In particular, if $\Omega$ is strongly $m$-hyperconvex, then
 $$
 \int_{\Omega}dd^c u \wedge (dd^c w)^k \wedge\beta^{n-k -1}  \leq  c_{m,n} \left(I_m (u)\right)^{\frac{1}{m}} \left(I_m (w)\right)^{\frac{k}{m}},
 $$
 and 
$$ 
  \int_{\Omega}dd^c u  \wedge \beta^{n-1}  \leq  c_{m,n} \left(I_m (u)\right)^{\frac{1}{m}}
 $$
 where $c_{m,n} > 0$ is a uniform constant.
\end{lemma}


The following consequence will be useful in the sequel. This result is usually stated for plurisubharmonic functions on a bounded domain  with boundary values $0$. Let us give a more general version using Cegrell inequalities (see \cite{BZ20}).

\begin{corollary} \label{coro:Comparison Principle} Let $\Omega \Subset \C^n$ be a bounded strongly $m$-pseudoconvex domain. Assume that  $u,v\in \mathcal{SH}_m(\Omega)\cap L^{\infty}(\Omega)$ satisfy $u \leq v$ on $\Omega$ and for any $\zeta \in \partial \Omega$, $\lim_{z \rightarrow \zeta }(u(z)- v(z))= 0$.
Then
 $$
 \int_{\Omega}(dd^c v)^m\wedge\beta^{n-m} \leq \int_{\Omega}(dd^c u)^m\wedge\beta^{n-m}.
 $$
\end{corollary}
 We will need the following result which was proved by B\l ocki for the complex Monge-Amp\`ere operator (\cite{Bl93})
 \begin{lemma} \label{lem:Blocki} Let $\psi,  v, w \in \mathcal{SH}_m (\Omega) \cap L^{\infty} (\Omega)$ such that $\psi \leq 0$,  $v \leq w$ and 
$ \lim_{z \to \zeta} (w (z) - v(z)) = 0$. Then
$$
\int_\Omega (w-v)^m (dd^c \psi)^m \wedge \beta^{n-m} \leq m! \Vert \psi \Vert_{\infty}^m  \int_\Omega  (dd^c v)^m \wedge \beta^{n-m}
$$ 
 \end{lemma}
The proof in this case is the same as in \cite{Bl93} since it essentially only uses the integration by parts formula



\section{Hessian measures of continuous potentials} 


\subsection{Hessian capacities} 

An important tool in dealing with our problems is the notion of capacity. This was introduced by  Bedford and Taylor in their pionneer  work for the complex Monge-Amp\`ere operator (see \cite{BT82}). 
Let us recall the coresponding notion of capacity we will use here (see \cite{Lu12}, \cite{SA13}). Let $\Omega \Subset \C^n$  be a $m$-hyperconvex domain.  The $m$-Hessian capacity  is defined as follows. For any compact set $K \subset \Omega$,
$$
 \text{c}_m(K,\Omega) := \sup \{\int_K  (dd^c u)^m \wedge \beta^{n - m} ; u \in \mathcal{SH}_m (\Omega) , - 1 \leq u \leq 0\}.
$$

We can extend this capacity as an outer capacity on $\Omega$. Given a  set $S \subset \Omega$, we define the inner capacity of $S$ by the formula
$$
\text{c}_m(S,\Omega) := \sup \{\text{c}_m(K,\Omega) ; K \, \, \hbox{compact} \, \, K \subset S\}.
$$ 

The outer capacity of $S$ is defined by the formula 
$$
\text{c}^*_m(S,\Omega) := \inf \{\text{c}_m(U,\Omega) ; U \, \, \hbox{ is open} \, \, U \supset S\}, 
$$ 

One can show that $\text{c}^*_m(\cdot,\Omega)$ is a Choquet capacity and then  any Borel set$ B \subset \Omega$ is capacitable and
for any compact set $K \subset \Omega$, 
 \begin{equation} \label{eq:cap}
 \text{c}_m(K,\Omega)=\int_{\Omega}(dd^c u_K^*)^m\wedge\beta^{n-m},
 \end{equation}
  where $u_K$ is the relative equilibrium potential of $(K,\Omega)$ defined by the formula :
  
 $$
 u_K:=\sup\{u\in \mathcal{SH}_m(\Omega);u\leq0\ \hbox{in}\ \Omega, u\leq-{\bf 1}_K  \, \, \mathrm{on } \, \, \Omega\},
 $$
 and $u_K^*$ is its upper semi-continuous regularization on $\Omega$ (see \cite{Lu12}).
 
 It is well knwon that $u_K^*$ is $m$-subharmonic on $\Omega$, $- 1 \leq u_K^* \leq 0$, $u_K^* = - 1$ quasi-everywhere (with respect to $ \text{c}_m$) on $\Omega$ and $u_K^* \to 0$ as $z \to \partial \Omega$ (see \cite{Lu12}).
 

\subsection{Hessian mass estimates near the boundary} %  §3.2

Here we prove a comparison inequality which seems to be new even in the case of a complex  Monge-Amp\`ere measure. This will play a crucial role in the proof of Theorem 2  and may have an interest in its own. It is a generalization of an estimate proved in \cite{BZ20} for Hessian measures of H\"older continuous potentials.
\begin{lemma} \label{lem:ComparisonIneq} Let $\Omega \Subset \C^n$ be a $m$-hypercovex domain and 
 $\varphi \in \mathrm{SH}_m (\Omega) \cap L^{\infty}  (\Omega)$. Then for any compact set $K \subset \Omega$ we have
$$
\int_K (dd^c \varphi)^m \wedge \beta^{n - m}  \leq  (\text{osc}_K  \varphi)^m  \text{c}_m (K,\Omega).
$$

If moreover $\varphi$ is continuous in $\bar \Omega$ and $\varphi = 0$ on $\partial \Omega$,  then for any compact subset $K \subset \Omega$, we have
 $$
 \frac{\sigma_m (\varphi) (K)}{\left[\kappa (\delta_K (\partial \Omega))\right]^m}  \leq   \text{c}_m (K,\Omega),
 $$
 where $\kappa = \kappa_\varphi$ is the  modulus of continuity of $\varphi$ on $\bar{\Omega}$ and 
 $$\delta_K (\partial \Omega) := \sup_{z \in K}  \mathrm{dist} (z,\partial \Omega)
 $$
  is the Hausdorff distance of $K$ to the boundary $\partial \Omega$.
\end{lemma}

\begin{proof} 
1) We can assume that $\max_K \varphi = 0$ and $\varphi \not \equiv  0$. Then $ a := \text{osc}_K  \varphi = - \inf_K \varphi > 0$ and then  the function $v := a^{-1} \varphi$ is $m$-subharmonic on $\Omega$, and satisfies the inequalities  $v \leq 0$ on $\Omega$ and $v \geq - 1$ on $K$.

 Fix $\varepsilon >0$ and let $u_K$ be the relative extremal $m$-subharmonic function of $(K,\Omega)$.  Then $ K \subset \{ (1 + \varepsilon) u_K^*  < v\} \cup \{u_K < u_K^*\}$. Since the set $\{u_K < u_K^*\}$ has zero $m$-capacity (see \cite{Lu12}), it follows from  the comparison principle that
\begin{eqnarray*}
\int_K (dd^c v)^m \wedge \beta^{n - m}  &\leq &\int_{\{ (1 + \varepsilon) u_K^*  < v\}} (dd^c v)^m \wedge \beta^{n - m} \\
&\leq & (1 + \varepsilon)^m \int_{\{(1 + \varepsilon) u_K ^* < v\}} (dd^c u_K^*)^m \wedge \beta^{n - m} \\
& \leq & (1 + \varepsilon)^m \int_\Omega  (dd^c u_K^*)^m \wedge \beta^{n - m}  =  (1 + \varepsilon)^m \text{c}_m (K,\Omega).
\end{eqnarray*}
The last identity follows from \cite{Lu12}.  Letting $\varepsilon \to 0$, we obtain the first statement. 

2) Fix a compact set $K \subset \Omega$. Since $\kappa$ is the modulus of continuity of $\varphi$, we have for any $\zeta \in \partial \Omega$ and $z \in K$
$\varphi (\zeta) - \varphi (z) \leq \kappa (\vert \zeta - z\vert)$.  Since $\varphi = 0$ in $\partial \Omega$, we obtain that for any $z \in K$,
$$
- \varphi (z) \leq \kappa \left( \delta_K (\partial \Omega)\right).
$$
Hence $\hbox{osc}_K \varphi \leq   \kappa \left( \delta_K (\partial \Omega)\right)$.  Applying the first statement to $\varphi$ we obtain the required inequality. 
\end{proof}



\subsection{ Hessian measures acting on Hessian potentials} 
We will study continuity properties of the functional associated to the Hessian measure of a function  $\varphi\in \mathcal{SH}_m(\Omega)\cap \mathcal{C}^{0} (\bar{\Omega}) $, acting on the space $\mathcal{SH}_m (\Omega) \cap L^{\infty} (\Omega)$.

 Let $g \in \mathcal{C}^0 (\partial \Omega)$ be a continuous function on  $\partial \Omega$ and $R > 0$ a positive constant. We denote by $\mathcal{SH}_m^g (\Omega,R)$ the set of functions $v \in \mathcal{SH}_m (\Omega) \cap L^{\infty} (\Omega)$ such that 
 $$
 \int_\Omega (dd^c v)^m \wedge  \beta^{n-m} \leq R, \,   \, \, \text{and} \, \, \, \lim_{z \to \zeta} v (z) = g (\zeta), \, \, \forall \zeta \in \partial \Omega.
 $${
The following result improves previous estimates given in \cite{N14} and \cite{BZ20}. 
 \begin{theorem}\label{thm:ModC}
Let $\varphi\in \mathcal{SH}_m(\Omega)\cap \mathcal{C}^{0} (\bar{\Omega})$ and $g \in \mathcal{C}^0 (\partial \Omega)$ be given functions.
Then  there exists $C_m  = C (m,\Omega,g, R) >0$   such that for every $u,v\in \mathcal{SH}^g_m(\Omega, R)$,
% with  $\Vert u-v \Vert_m \leq \delta_m$, 
we have
\begin{equation} \label{eq:MocEst}
\int_{\Omega}|u-v|^m (dd^c\varphi)^m\wedge\beta^{n-m}\leq C_m \, \kappa_\varphi \circ \theta_m \left(\Vert u-v \Vert_m^{m}\right),
\end{equation}
where $\Vert u-v \Vert_m := \left(\int_{\Omega} \vert u - v\vert^m d \lambda_{2 n} \right)^{1 \slash m}$,  $\theta_m$ is the reciprocal of the function $t \longmapsto t^{2m} \kappa_{\varphi}^{1-m} (t)$.
\end{theorem}


\begin{proof}
Observe that for any $\varepsilon > 0,$ $ u_\varepsilon := \max \{u - \varepsilon, v \} \in \mathcal{SH}^g_m(\Omega)$, $u_\varepsilon \geq v$ and $u_\varepsilon = v$ near the boundary $\partial \Omega$. By the comparison principle, this implies that $u_\varepsilon \in  \mathcal{SH}^g_m(\Omega, R)$. Therefore, replacing $u$ by $u_\varepsilon$,  we can assume that $u \geq v$ on $\Omega$ and $u = v$ near the boundary, for the inequality (\ref{eq:MocEst}) will follow since $\vert u - v \vert =   ( \max\{u,v\} - u) + (\max\{u,v\}  - v)$.

On the other hand by approximation on the support $S$ of $u-v$ which is compact, we can assume that $u$ and $ v $ are smooth on a neighbourhood of $S$.

Then it remains to estimate the following integral
$$
I_m := \int_{\Omega} (u - v)^m  (dd^c\varphi)^m\wedge\beta^{n-m}.
$$

  
 Observe  first that we can extend $\varphi$ by continuity to $\C^n$ with the same modulus of continuity. Indeed, it is easy to see that the function defined  for $z \in \C^n$ by the following formula 
  $$
  \tilde{\varphi} (z) := \sup \{ \varphi (\zeta) - \kappa_\varphi (\vert z - \zeta\vert) \, ; \; \zeta \in \bar{\Omega}\}\cdot
  $$
  is the required extension.
 For simplicity, it will be denote  by $\varphi$.   
 
 Then we denote by $\varphi_{\delta}$ the smooth approximants of $\varphi $ on $\C^n$, defined by (\ref{eq:reg}).
    
  We know  that for $0 < \delta < \delta_0$, $\varphi_\delta \in  \mathcal{SH}_m (\Omega_{\delta})\cap\mathcal{C}^{\infty}(\mathbb C^n)$.
  
  
  Since $\varphi_\delta$ is not $m$-subharmonic on the whole  $\Omega$, we will consider  its $m$-subharmonic envelope defined by the formula:
  
  \begin{equation} 
  \psi_\delta (z) := \sup \{\psi (z) \, ; \, \psi \in \mathcal{SH}_m (\Omega), \psi \leq \varphi_\delta \, \, \text{on} \,\, \Omega\}, \, \, z \in \Omega.
  \end{equation} 
  We know by \cite[Theorem 3.3 ]{BZ20} that $\psi_\delta \in \mathcal{SH}_m (\Omega) \cap \mathcal{C}^0 (\bar{\Omega},$ $ \psi_\delta  \leq \varphi_\delta $ on $ \Omega$ and 
  \begin{equation} \label{eq:projection}
  (dd^c  \psi_\delta)^m \wedge \beta^{n-m} \leq (\sigma_m (\varphi_\delta))_+,
   \end{equation} 
  weakly on $\Omega$, where $(\sigma_m (\varphi_\delta))_+$ is defined pointwise on $\Omega$ by  $(\sigma_m (\varphi_\delta))_+(z) = \sigma_m (\varphi_\delta)(z)$ for $z \in  \Omega$ such that $dd^c \varphi_\delta (z) \in  \Theta_m$ and $(\sigma_m (\varphi_\delta))_+(z) = 0$ otherwise.
  
 
  To prove the required estimate, we write for $0 < \delta < \delta_0$
$$
I_m = A_m (\delta) + B_m (\delta),
$$
where
$$
A_m (\delta) := \int_{\Omega} (u - v)^m \left[ (dd^c\varphi)^m - (dd^c\psi_\delta)^m\right]\wedge\beta^{n-m}.
$$
and 
$$
B_m (\delta) := \int_{\Omega} (u - v)^m  (dd^c\psi_\delta)^m \wedge\beta^{n-m}.
$$

We estimate each term separately for fixed $0 < \delta < \delta_0$.

To estimate the first term, observe that
$$
\left( (dd^c\varphi)^m - (dd^c\psi_\delta)^m\right) \wedge  \beta^{n - m}= dd^c (\varphi - \psi_\delta) \wedge T,
$$
where $T := \sum_{j = 0}^{m -1} (dd^c \varphi)^j \wedge (dd^c \psi_\delta)^{m-j-1} \wedge\beta^{n - m}$.

Then
$$
A_m (\delta) = \int_{\Omega} (u - v)^m dd^c (\varphi - \psi_\delta - \kappa_\varphi(\delta)) \wedge T. 
$$
Integration by parts yields
$$
A_m (\delta) = \int_{\Omega}  (\psi_\delta - \varphi + \kappa_\varphi(\delta) ) \left[- dd^c (u - v)^m \right]\wedge T. 
$$
An easy computation shows that 
\begin{eqnarray} \label{eq:formalineq}
- dd^c (u - v)^m \wedge T & \leq &  m  (u - v)^{m-1} dd^c (v - u) \wedge T \\
&\leq & m  (u - v)^{m-1} dd^c v \wedge T, \nonumber
\end{eqnarray} 
in the sense of currents on $\Omega$.

 Observe that from the definition we have  $\psi_\delta \leq \varphi_\delta  \leq \varphi + \kappa_\varphi (\delta)$ on $\Omega$. On the other hand, since $ \varphi - \kappa_\varphi (\delta) \leq \varphi_\delta$ on $\Omega$, it follows
that $ \psi_\delta - \varphi  + \kappa_\varphi  (\delta) \geq 0 $ on $\Omega$.  
Combining the two estimates  we conclude that  $0 \leq \psi_\delta - \varphi  + \kappa_\varphi  (\delta) \leq 2 \kappa_\varphi (\delta)$, and then 
$$
A_m (\delta) \leq 2 m \, \kappa_\varphi  (\delta)  \int_{\Omega}  (u - v)^{m-1} dd^c v \wedge T.
$$
By definition of $T$ we have  
\begin{eqnarray*}
&& \int_{\Omega}  (u - v)^{m-1} dd^c v \wedge T \\
 &&=   \sum_{j = 0}^{m -1} \int_{\Omega}  (u - v)^{m-1} dd^c v \wedge  (dd^c \varphi)^j \wedge (dd^c \psi_\delta)^{m-j-1} \wedge\beta^{n - m}
\end{eqnarray*}

Observe that  if we write 
$$
  (dd^c \varphi)^j \wedge (dd^c \psi_\delta)^{m-j-1} \wedge\beta^{n - m} = dd^c w \wedge S_j, 
  $$ 
  where $w = \varphi$ or $w = \psi_\delta$,  then as before by integration by parts  using an inequality analogous to (\ref {eq:formalineq}) with $k$ intead of $m$ and $ dd^c v \wedge S_j$ instead of $T$, we obtain that  for $1 \leq k \leq m$, 
$$
\int_{\Omega}  (u - v)^{k} dd^c v \wedge  dd^c w \wedge S_j \leq k  \Vert w \Vert_{\Omega} \int_{\Omega}  (u - v)^{k-1} (dd^c v)^2 \wedge S_j.
$$
Repeating the integration by parts we finally get
\begin{equation} \label{eq:estimateA}
A_m (\delta) \leq 2  m! \, \Vert \varphi \Vert_{\Omega}^{m -1} \kappa_\varphi  (\delta)  \int_{\Omega} (dd^c v)^m \wedge \beta^{n - m} \leq C_1 \kappa_\varphi  (\delta),
\end{equation}
where $C_1 :=  m m! \, R \, \Vert \varphi \Vert_{\Omega}^{m -1} $.

 To estimate the second term, we need to establish the following estimate $0 < \delta < \delta_0$,

\begin{equation} \label{eq:Festimate}
dd^c \varphi_\delta \leq M_2 \frac{\kappa_\varphi (\delta)}{\delta^2}  \beta, \, \, \text{pointwise on} \, \, \,  \Omega,
\end{equation}
where $M_2 > $ is a uniform constant.

Indeed, by differentiating the integral formula $
\varphi_\delta (z) = \varphi \star \chi_\delta (z)$
and by making an obvious change of variables, we obtain for $j, k = 1, \cdots, n$
\begin{eqnarray*}
\partial_j \partial_{\bar{k}} \varphi_{\delta}  (z)
&=&  \delta^{-2} \int_{\C^n}  \varphi (z - \delta \eta)  \partial_j \partial_{\bar{k}} \chi   (\eta) d \lambda_{2 n} (\eta) \\
&=& \delta^{-2} \int_{\C^n}  [\varphi (z - \delta \eta)  - \varphi (z)] \partial_j \partial_{\bar{k}} \chi   (\eta) d \lambda_{2 n} (\eta),
\end{eqnarray*}
where the last equation follows from the fact that by Stokes formula $\int_{\C^n} \partial_j \partial_{\bar{k}} \chi   (\eta) d \lambda_{2 n} (\eta) = 0$ since $\chi$ is a smooth test function with compact support.
Thus the estimate (\ref{eq:Festimate}) follows from the last equation since the support of $\chi$ is contained in the unit ball.

 Now from the inequalities (\ref{eq:projection}) and (\ref{eq:Festimate}), it follows that
 $$
 (dd^c \psi_\delta)^m \wedge \beta^{n-m} \leq  M_2^m \frac{\kappa_\varphi^m (\delta)}{\delta^{2 m}} \beta^n,  \, \, \,  \text{weakly  on} \, \, \,  \Omega.
 $$
 Therefore we have
\begin{equation} \label{eq:estimateB}
B_m (\delta) \leq C_2 \frac{\kappa_\varphi^m (\delta)}{\delta^{2 m}} \int_{\Omega} (u - v)^m \beta^n,
\end{equation}
where $C_2 := M_2^m$.

From (\ref{eq:estimateA}) and (\ref{eq:estimateB}) we conclude that
$$
\int_{\Omega} (u - v)^m  (dd^c\varphi)^m\wedge\beta^{n-m} \leq  C_1 \kappa_\varphi  (\delta) +   C_2 \frac{\kappa^m_\varphi (\delta)}{\delta^{2 m}} \int_{\Omega} (u - v)^m \beta^n.
$$

We want to optimize the right hand side by taking $\delta> 0$ so that 
$$
 \delta^{2 m} \kappa^{1-m}_\varphi (\delta) = \int_{\Omega} (u - v)^m \beta^n = \Vert u-v\Vert_m^m,
$$
i.e.  $\delta = \theta_m (\Vert u-v\Vert_m^m)$, where $\theta_m$ is the reciprocal of the function $t \longmapsto t^{2m} \kappa_\varphi^{1-m} (t)$. This is possible if  $\Vert u - v \Vert_{m}^m \leq \theta_m^{-1} (\delta_0)$ so that  $\delta < \delta_0$. Then applying the previous estimate we obtain the  estimate of the Lemma in this case.

Now assume that $\Vert u - v \Vert_{m}^m >  \theta_m^{-1} (\delta_0)$.  By Lemma \ref{lem:Blocki}, we have
$$
\int_{\Omega}(u-v)^m (dd^c\varphi)^m\wedge\beta^{n-m}\leq m! \Vert \varphi \Vert_{\infty}^m \int_\Omega  (dd^cv)^m\wedge\beta^{n-m} \leq  m!  R \Vert \varphi \Vert_{\infty}^m.
$$
We see that we obtain the inequality of the Lemma \ref{eq:MocEst} by increasing the constant $ C$ consequently.
\end{proof}

\begin{corollary} \label{cor:Lmuestimate} Under the same assumptions as the Theorem \ref{thm:ModC}, we have
$$
  \int_\Omega \vert u - v \vert^m \sigma_m (\varphi) \leq  C (m) \,  \kappa_\varphi \circ \theta_m\left(M  \Vert u - v \Vert_1\right),
$$
  where $M := \left[\Vert u - v\Vert_{\infty}\right]^{m - 1}$.
\end{corollary}
\begin{proof} 
Apply Theorem \ref{thm:ModC} 	and observe that 
$$
\Vert u-v\Vert_m^m \leq  \Vert u-v\Vert_{\infty}^{m - 1} \, \,  \Vert u-v\Vert_1.
$$
The required inequality follows immediately, since $\kappa_\varphi $ is non decreasing.
\end{proof}
\subsection{Global approximants to the solution} %  §3.4

Let $u  \in \mathcal{SH}_m (\Omega) \cap \mathcal C (\bar \Omega)$. We define the volume mean-values of $u$  as follows:
\begin{equation} \label{eq:volumemeanvalue}
\widehat{u}_\delta (z):= \frac{1}{\tau_{2n}\delta^{2n}} \int_{|\zeta -z|\leq \delta} u(\zeta) dV_{2n}(\zeta), z \in \Omega_{\delta},
\end{equation}
where $\tau_{2n}$ is the volume of the unit ball in $\C^n$.

We need  the following lemma which was proved in \cite{Ch16b} in the H\"older continuous case.  

 \begin{lemma}\label{lem:approximation}
 Let  $u \in \mathcal{SH}_m (\Omega) \cap L^{\infty} (\Omega)$ such that there exists two functions $v , w : \bar{\Omega} \longrightarrow \R$  continuous  on $\bar \Omega$ such that $v \leq u \leq w$ on $ \Omega$ and $v = w$ on $\partial \Omega$.

 Then there exist $\delta_0>0$ small enough, depending on $\Omega$,   such that for any $  0<  \delta < \delta_0$ the function defined by 
 
\begin{equation} \label{eq:approximants}
 	\tilde {u}_{\delta}
 		= \begin{cases}
			\max\{\widehat{u}_{\delta} -  \widehat{\kappa} (\delta) , u\} & \text{ on } \; \Omega_\delta,\\
                   u  & \text{ on } \; \Om\setminus \Omega_\delta,
           \end{cases}
\end{equation}
is  a bounded $m$-subharmonic  function  on $\Omega$ which satisfies the inequalities 
$$
 0 \leq  \tilde {u}_{\delta} - u \leq \widehat{u}_{\delta}  - u \leq \tilde{u}_{\delta}  - u +  \widehat{\kappa} (\delta) , \, \, \, \mathrm{on} \, \, \, \Omega_\delta,
$$
where $\widehat{\kappa} (\delta) := \widehat{\kappa}_v (\delta) + \widehat{\kappa}_w (\delta) + \delta$ for $0 < \delta < \delta_0$.

Moreover $\tilde{u}_{\delta} = u$ in a neighbourhood of $\partial \Omega_\delta$ in $\Omega$.
\end{lemma}
\begin{proof} By the gluing property (see Proposition \ref{prop:basic}),  it is enough to prove that for $\delta > 0$ small enough, $ \widehat{u}_{\delta} -   \widehat{\kappa} (\delta) \leq  u $ on $\partial \Omega_\delta$. 


 Indeed fix $0 < \delta < \delta_0 < 1$ and  fix $z \in \partial \Omega_\delta$. Then there exists $\zeta \in \partial \Omega$ such that $\vert z - \zeta \vert = \delta$. Hence   
 
 \begin{eqnarray*}
 \widehat{u}_{\delta} (z)  \leq   \widehat{w}_\delta (z) & \leq & w (\zeta) + \widehat{\kappa}_w (\delta)  \\
 & \leq & v (\zeta)  + \widehat{\kappa}_w (\delta)  \\
 & \leq & v (z) +   \widehat{\kappa}_v (\delta) + \widehat{\kappa}_w (\delta) \\
& <  &  v  (z) +   \widehat{\kappa} (\delta) \leq u(z) +  \widehat{\kappa} (\delta)
 \end{eqnarray*} 
 which proves the required condition. Observe that, since $v$ is continuous, the set   $ \{\widehat{u}_\delta  -  \widehat{\kappa} (\delta) < v(z) \} $ is a neighbourhood of $\partial \Omega_\delta$. Hence  $\widehat{u}_\delta - \widehat{\kappa} (\delta) \leq u $ is a neighbourhood of $\partial \Omega_\delta$ and then  $\tilde{u}_{\delta} = u$ in a neighbourhood of $\partial \Omega_\delta$. 
 \end{proof}
 
 The following estimate will play a crutial role in the proof of Theorem 3 and Theorem 4.
 
\begin{corollary}\label{cor:approximation} Let $\Om$ be a bounded strongly $m$-pseudoconvex domain and  $\mu$ a positive Borel measure on $\Omega$ with finite mass. Assume there exists  $\varphi \in  \mathcal{SH}_m (\Omega) \cap \mathcal C (\bar \Omega)$ such that $\varphi = 0$ on $\partial \Omega$ and $\mu \leq \sigma_m (\varphi)$ weakly on $\Omega$.
Let $ g \in \mathcal C (\partial \Omega)$ and $u \in \mathcal{SH}_m (\Omega) \cap L^{\infty} ( \Omega)$ satisfying $\sigma_m (u) \leq \mu$ weakly on $\Omega$ and $u = g$ on $\partial \Omega$.

Then there exists two  continuous $m$-subharmonic functions $v$ and $ w$ on $\Omega$ satisfying the requirements of Lemma \ref{lem:approximation} so that the corresponding functions  $ (\tilde{u}_\delta)_{0 < \delta < \delta_0} $  defined by the formula (\ref{eq:approximants}) satisfy the following estimates: 

$$
\int_{\Omega_\delta} (\tilde{u}_\delta - u)^m d \mu  \leq C (m,\mu) \,  \kappa \circ \theta_m( d M\delta^2), \, \, 0 < \delta < \delta_0,
$$
where $C(m,\mu) = C (m,\Omega,g, \mu) >0$ and $d = d(m,n,\mu) > 0$ are uniform constants, $ M := (\text{osc}_{\Omega} u)^{m -1}$ and $\kappa(\delta) := \kappa_\varphi(\delta) + \kappa_g(\sqrt{\delta}) + \delta$.  
\end{corollary}


\begin{proof}
We want to apply Lemma \ref{lem:approximation} and  Corollary  \ref{cor:Lmuestimate}. To this end, we need to construct two functions $v$ and $w$ satifying the requirement of the Lemma \ref{lem:approximation}.  Let $w$ be the maximal $m$-subharmonic function on $\Omega$ with boundary values $g$. By \cite{Ch16b}, we have $\kappa_w  (\delta) \leq \kappa_g (\sqrt{\delta})$ and by the comparison principle we have $u \leq w$ on $\Omega$.   

 Moreover,  the function $v := \varphi + w$ is $m$-subharmonic on $\Omega$, continuous on $\bar{\Omega}$ with $\kappa_v (\delta)  \leq \kappa_\varphi (\delta)  + \kappa_g (\sqrt{\delta})$ and $v = g$ on $\partial \Omega$. Since   $ \sigma_m (v) \geq \sigma_m (\varphi) $ and $\sigma_m (u) \leq \mu \leq \sigma_m (\varphi$ weakly on $\Omega$, it follows from the comparison principle that
$v \leq u$ on $\Omega$.

 Therefore we can apply Lemma \ref{lem:approximation} to construct  global approximants $ (\tilde{u}_\delta)_{0 < \delta < \delta_0} $  given by the formula (\ref{eq:approximants}). Since  $ \tilde{u}_\delta  \geq  u$ on $\Omega_\delta$, and $\tilde{u}_\delta = u$ in a neighbourhood of $\Omega \setminus \Omega_\delta$,  it follows from Corollary \ref{coro:Comparison Principle} that 
 
 
  $$
 \int_{\Omega} (dd^c \tilde{u}_\delta)^m \wedge \beta^{n-m} \leq  \int_{\Omega} (dd^c u)^m \beta^{n-m} \leq \mu (\Omega).
 $$
 

  By Corollary  \ref{cor:Lmuestimate}, we have for $0 < \delta < \delta_0$
 $$
 \int_{\Omega } (\tilde{u}_{\delta} - u)^m d\mu \leq C_m \,  \kappa_{\varphi}  \circ \theta_m  (M \Vert \widetilde u_{\delta} - u\Vert_1),
 $$
 where $M := (\text{osc}_{\Omega} u)^{m -1}$ and  $C(m) = C (m,\Omega,g, \mu) >0$ is a uniform constant.
 
 Now observe that $\tilde{u}_{\delta} - u = 0$ on $\Omega \setminus \Omega_\delta$ and  $\tilde{u}_{\delta} - u \leq \widehat{u}_\delta - u$ on $\Omega_\delta$. This yields 
 $$
 \Vert \tilde{u}_{\delta} - u \Vert_1 \leq \int_{\Omega_\delta} (\widehat{u}_\delta - u) d \lambda_{2n}.
 $$

 
 By Lemma \ref{lem:Poisson-Jensen} , we have $\int_{\Omega_\delta} (\widehat{u}_\delta - u ) d \lambda_{2 n} \leq D_n \Vert \Delta u\Vert_{\Omega} \delta^2$, where $D_n > 0$ is a positive uniform constant. 
 By Lemma \ref{lem:Cegrell},  we have $\|\Delta u\|_{\Omega} \leq c_{m,n}  \mu (\Omega)^{1 \slash m} < + \infty$.
 Hence
 
 $$
\Vert \tilde{u}_{\delta} - u \Vert_1  \leq d (m,n) \delta^2.
 $$
 This ends the proof of the lemma.
 \end{proof}


\section{Continuity of the potential of a diffuse measure}


\subsection{Diffuse Borel measures}


We will use the following terminology from Potential theory (see \cite{Po16}). 
\begin{definition}  \label{def:cap-domination} Let $\mu$ be  a positive Borel measure on $\Omega$. 
 
 1. We say that $\mu$ is diffuse  with respect to  the capacity $  \text{c}_m = \text{c}_m (\cdot,\Omega)$ if  $\mu (K) = 0$ whenever $K \subset \Omega$ is a compact set with $ \text{c}_m (K,\Omega) = 0$. 
 
2. We associate to $\mu$ a natural one variable function as follows :
\begin{equation} \label{eq:AC}
 \Gamma _\mu (t) := \sup \{ \mu (K) ; K \subset \Omega \, \, \, \hbox{is compact },  \, \, \,   \text{c}_m (K,\Omega) \leq t\}.
\end{equation}
It follows from the definition that $\Gamma_\mu$ is non decreasing right continuous function on $\R^+$ which satisfies the following property: for any compact set $K \subset \Omega$, we have
\begin{equation} \label{eq:capdomination}
\mu (K) \leq \Gamma _\mu \left(\text{c}_m (K)\right),
\end{equation}
where $\text{c}_m (K) = \text{c}_m (K,\Omega)$. 

 Observe that by inner regularity of the measure $\mu$ and the capacity $ \text{c}_m (\cdot,\Omega)$, this inequality is satisfied for any Borel set $K \subset \Omega$.
 
3. If $\Gamma$ is a non-decreasing right continuous function on $\R^+$, we say that $\mu$ is $\Gamma$-diffuse (with respect to  the $m$-Hessian capacity)  if for any compact subset $K \subset \Omega$, with $\text{c}_m (K,) \leq 1$, 
\begin{equation} \label{eq:capdomination}
 \mu (K) \leq  \Gamma \left( \text{c}_m (K)\right). 
\end{equation}
This means that $\Gamma_\mu (t) \leq \Gamma (t),$ for any $t \in [0,1]$.
\end{definition}
 
 
 Let us mentione that S.  Ko\l odziej was the first to relate the domination  of the measure $\mu$ by the Monge-Amp\`ere capacity to the regularity of the solution to complex Monge-Amp\`ere equations (see \cite{Kol96}). 
 
The following lemma is easy to prove (see \cite{Po16}).
\begin{lemma}
A   positive Borel measure $\mu$ on $\Omega$ is  diffuse  if and only if $\lim_{t \to 0^+} \Gamma_\mu (t) = 0$.
\end{lemma}
 
 Let us give a simple example.
 \begin{example} Let $\phi \in  \mathcal{SH}_m(\Omega) \cap L^{\infty} (\Omega)$ and  $\sigma_\phi  := (dd^c \phi)^m \wedge \beta^{n-m}$ be its $m$-Hessian measure. Set $M := \hbox{osc}_{\Omega} \phi$. 
 Then from the definition of the $m$-Hessian capacity, we have for any compact subset $K \subset \Omega$,
 $$
 \sigma_\phi   (K) \leq A  \, \text{c}_m (K), \, \, \, \hbox{where} \, \, \, A := M^m.
 $$
  This implies that the measure $\sigma_\phi$ is diffuse  wrt the $m$-Hessian capacity on $\Omega$ and  $\Gamma_{\sigma_\phi}  (t) \leq A t$ for any $t \in \R^+$.  
  
 An example of Ko\l odziej shows that there exits a measure $\mu$ such that $\mu \leq \text{c}_m (\cdot)$ but $\mu$ is not the Monge-Amp\`ere of a bounded plurisubharmonic function (see \cite{Kol96}).
 \end{example} 

The following examples du to Dinew and Ko\l odziej   are more involved.
\begin{example}
1.Assume that $1 \leq m < n$. Then  Dinew and Ko\l odziej  proved in \cite{DK14} that the volume measure $\lambda_{2 n}$ is diffuse with respect to capacity. Namely for any $1 < r < \frac{n}{n - m}$, there exists a constant $N (r) > 0$ such that for any compact subset $K \subset \Omega$, 
\begin{equation} \label{eq:DK}
\lambda_{2 n} (K) \leq N (r) \text{c}_m (K)^{r}.
\end{equation}
 Observe that this estimate is sharp in terms of the exponent when $m < n$. This can be seen by taking $\Omega = \B$ the unit ball and  $K := \bar{\B_r} \subset \B$ the closed ball of radius $r \in ]0,1[$, since $\text{c}_m(\bar{\B}_r ,\B) \approx  r^{2 (n-m)}$ as $r \to 0$ (see \cite{Lu12}).

  Let $ 0 \leq f \in L^p (\Omega)$ with $p > n \slash m$. Then $ \frac{n (p-1)}{p (n - m)} > 1$. By H\"older inequality and inequality (\ref{eq:DK}) we obtain: for any  $1 < \tau < \frac{n (p-1)}{p (n - m)}$ there exists a constant $M (\tau) > 0$ such that for any compact set $K \subset \Omega$, 
\begin{equation*} 
\int_K f d \lambda_{2 n} \leq  M (\tau) \Vert f\Vert_p  \text{c}_m (K)^{\tau}.
\end{equation*}

2. When $m=n$ the domination is much more precise. It was proved in  \cite{ACKPZ09} that for any $0< b < 2 n $, there exists a constant $B > 0$ such that for any compact subset $K \subset \Omega$, 
 
 \begin{equation} \label{eq:ACKPZ}
 \lambda_{2 n} (K) \leq  \, B \,  \text{c}_n (K) \,  \exp \left( -b \left[\text{c}_n (K)\right]^{- 1 \slash n}\right).
 \end{equation}


Let  $ 0 \leq f \in L^p (\Omega)$ with $p > 1$, then by H\"older inequality and inequality (\ref{eq:ACKPZ}),for any $0< b < 2 n (p-1) \slash p$, there exists a constant $B' > 0$ such that for any compact set $K \subset \Omega$, 
\begin{equation*} 
\int_K f d \lambda_{2 n} \leq  B' \Vert f\Vert_p  \,  \text{c}_n (K) \exp \left( -b \left[\text{c}_n (K)\right]^{- 1 \slash n}\right).
\end{equation*}
\end{example}

Theorem 2 will provide us with new examples.
 
 The condition (\ref{eq:capdomination}) plays an important role in the following stability result which will be a crucial point in the proof of our theorems (see \cite{EGZ09, GKZ08, Ch16a}).
 
 
\subsection{Uniform a priori estimates}
The following lemma is elementary, but it turns out to play a crucial role.
\begin{lemma}\label{lem:Kolo}
Let $f :\mathbb{R^+}\rightarrow\mathbb{R^+}$ be a decreasing right continuous function  such that $\lim_{s \to + \infty} f (s) = 0$ and let $\eta :\mathbb{R^+}\rightarrow\mathbb{R^+}$ be a  non-decreasing function which satisfies the following Dini condition
\begin{equation} \label{eq:DC0}
\int_{0^+} \frac{\eta (t)}{t} d t< + \infty.
\end{equation}
Assume that for any $t \in [0,1]$ and any $s > 0$, we have
\begin{equation} \label{eq:Etadecay}
t \, f (s+t) \, \leq  \,  f (s) \cdot  \eta ( f (s)).
\end{equation}
Then $f (s)=0$ for all $s\geq S_{\infty}$, where 
$$
S_{\infty}:= s_0 + \int_{0}^{ e f (s_0)}  \frac{\eta (t)}{t} d t, 
$$
and $s_0 \geq 0$ satisfies the condition
$$
\eta (f (s_0)) \leq 1 \slash e < 1.
$$
\end{lemma}
Observe that the Dini condition implies that $\lim_{t \to 0^+} \eta (t) = 0$. 
This Lemma is a reformulation of a Lemma of Kolodziej \cite{Kol05} in the spirit of  \cite{EGZ09} and \cite{BGZ08}). Its proof is a variant of the proof of \cite[Lemma 2.4]{EGZ09}. We will give it here for convenience for the reader.
\begin{proof} 
Since $\lim_{s \to + \infty} f(s) = 0$, thereexits $s_0 > 0$ such that $f(s_0) < 1 \slash e$. If $f (s_0) = 0$ we are done. If $f (s_0) > 0$, there exists $s > s_0$ such $f (s)) <   f (s_0) \slash e$ since $\lim_{s \to  + \infty} f (s)  = 0$. Therefore we can set  
$$
s_1 :=  \inf \{ s > s_0  \, \, ; \, \,  f (s) <  f (s_0) \slash e \}.
$$ 
By (\ref{eq:Etadecay}) we have
$$
f (s_0 + 1) \leq f (s_0) \eta (f(s_0)) < f (s_0) \slash e,
$$
which implies that $s_0 < s_1 \leq s_0 + 1$.

By definition of $s_1$,  there exists a sequence $s'_k$ decreasing to $s_1$ such that $ f (s'_k) <  f (s_0) \slash e$ for any $k > 0$. Since $f$ is right continuous, it follows that $f (s_1) = \lim_{k \to + \infty} f (s'_k) \leq f (s_0) \slash e$.

Thus we have proved that $s_0 < s_1 \leq s_0 + 1$ and $f (s_1) \leq f (s_0) \slash e$.

 We will construct  by induction an increasing  sequence $(s_j)_{j \geq 0}$ of positive numbers such that or any $j \in \N$
 $$
 s_j < s_{j + 1} \leq s_j + 1 \, \, \, \hbox{and} \, \, \, f (s_{j + 1} )\leq f (s_j) \slash e
 $$
Indeed assume by induction that for a fixed $j \geq 1$,   $s_1, \cdots, s_j$ are constructed with the required properties . Then the number 
$$
s_{j + 1} :=  \inf \{ s > s_j \, \, ; \, \,  f (s) <  f (s_j) \slash e \}.
$$
is well defined and by the same reasoning for $s_1$ we see that it satisfies the required properties. 

On the other hand, since $f (s_{j + 1} )\leq f (s_j) \slash e$, from (\ref{eq:Etadecay}), it follows that for any $s \in ]s_j , s_{j +1}[$ we have
$$
 (s - s_j) f (s) \leq f (s_{j}) \eta (f (s_{j})) \leq  e f (s)  \eta (f (s_{j})),
$$
since $ s_j < s < s_{j+1}$ and then $f (s) \geq f (s_{j}) \slash e$.

Therefore for any $j \in \N$ and $s \in ]s_j, s_{j + 1}[$, $ s-s_j \leq e  \eta (f (s_{j}))$, hence for any $j \in \N$,

$$
s_{j + 1} - s_j \leq e  \eta (f (s_{j}))
$$
Moreover since $f (s_{j} )\leq f (s_{j-1}) \slash e$ for any $j\geq 1$, it follows that 
$f (s_{j } )\leq f (s_0) \slash e^j$ and then 
$$
s_{j + 1} - s_j \leq  e {\eta} (f (s_0)  e^{-j}),
$$
for all $j \in \N$ since $\eta$ is non decreasing.

Therefore
\begin{eqnarray*}
s_{\infty} := \lim_{j \to + \infty} s_j = s_0 + \sum_{j \geq 0} (s_{j + 1} - s_j) & \leq & s_0 + e \sum_{j =0}^{+ \infty} \eta (f (s_0)  e^{-j}) \\
&\leq & s_0 +  e \int_0^{+\infty} {\eta} \left(f (s_0)  e^{- x+ 1}\right)  d x.
\end{eqnarray*}
 A simple change of variables yields
 $$
 s_{\infty} \leq s_0 + e \int_0^{e f (s_0)} \frac{\eta (t)}{t} d t.
 $$ 
 Recall that by construction  $s_j \leq s_{\infty}$ and $f (s_{j + 1} )\leq f (s_0) \slash e^j$  for any $j \in \N$. Therefore for any $j \in \N$

$$ 
0 \leq f (s_{\infty})  \leq f (s_{j + 1} )\leq f (s_0) \slash e^j,
$$ 
which implies that  that $f (s_{\infty}) = 0$. 

Therefore if we define
$$
S_{\infty} := s_0 + e \int_0^{e f (s_0)}  \frac{\eta (t)}{t} d t,
$$
 we conclude that $f (s) = 0$ for any $s \geq S_{\infty}$.
 
 Finally observe that, since  $\eta (f (s_0)) \leq 1\slash e < 1$, we have $f (s_0) \leq \eta^{-1} (1\slash e)$.
 \end{proof}


\begin{remark} 1. Observe that if $ \eta (f (0)) \leq 1 \slash e$ then $ f (s) = 0$ for any $s \geq  S_{\infty}$, where
$$
S_{\infty} := e \int_0^{e f (0)}  \frac{\eta (t)}{t} d t.
$$

2. If $\eta$ does not have the monotonicity property, we can replace in the statement of the lemma $\eta$ by the least non decreasing majorant function of $\eta$ define by
$$
\bar{\eta} (t) := \sup \{\eta (s) \geq  0 ; s \leq t\} , \, t \geq 0.
$$
\end{remark}

\smallskip
We now deduce a uniform a priori estimate on solutions to  complex Hessian equations. 

\begin{lemma} \label{lem:CapEst} Let $u, v \in  \mathcal{SH}_m (\Omega) \cap L^{\infty} (\Omega)$ be such that 
$$
 \liminf_{z\to \partial\Omega}(u-v)(z)\geq 0.
$$
 Then for any $t > 0, s > 0$, we have


 \begin{equation}\label{eq:CapEst}
 t^m {Cap}_m(\{u< v -s-t\},\Omega)\leq \int_{\{u<v -s\}}(dd^cu)^m\wedge\beta^{n-m}.
\end{equation}
\end{lemma}
The  lemma is well known. It follows from the comparison principle (see \cite{Kol96}, \cite{EGZ09}, \cite{GKZ08}, \cite{Ch16a}, \cite{KN19}).

\begin{corollary} \label{cor:UnifEst} Let  $\mu$ be a positive Borel measure on $\Omega$ with finite mass.
Assume that $\mu$ is $\Gamma$-diffuse  with respect to  the $m$-Hessian capacity and the function  $ \gamma (t) := \Gamma (t) \slash t$ is non-decreasing and
satisfies the following Dini type condition
\begin{equation} \label{eq:DiniConditionMu}
\int_{0^+}  \frac{\gamma^{1 \slash m} (t)}{t} d t < + \infty.
\end{equation}
Assume that $u \in \mathcal{SH}_m (\Omega) \cap L^{\infty} (\Omega)$ is the solution to the Dirichlet problem (\ref{eq:DirPb}) with boundary datum $g \in \mathcal{C}^0 (\partial \Omega)$ and right hand side $\mu$.  Then we have the following uniform estimate on $\Omega$ 


\begin{equation} \label{eq:UnifEst} 
 \min_{\partial \Omega} g -  2  e \left(\mu (\Omega)\slash a\right)^{1 \slash m}  -  \int_0^{a} \frac{\gamma^{1 \slash m} (t)}{t} d t \leq  \,  u \leq \max_{\partial \Omega} g,
 \end{equation}
  where
  $$
   a := e^m \gamma^{-1} (1\slash e^m).
  $$
  
In particular
 \begin{equation} \label{eq:UnifOsc} 
 \mathrm{osc}_{\bar \Omega} u \leq \mathrm{osc}_{\partial  \Omega} g +2  e \left(\mu (\Omega)\slash a\right)^{1 \slash m}  + \int_0^{a }\frac{\gamma^{1 \slash m} (t)}{t} d t.
 \end{equation}
\end{corollary}
\begin{proof} Set $ \alpha := \min_{\partial \Omega} g$. Then $\liminf_{z \to \partial \Omega} (u - \alpha) \geq 0$.
Then we can apply Lemma \ref{lem:CapEst} with $u $ and $v = \alpha$ and get the following estimate for any $s, t > 0$
\begin{eqnarray} \label{eq:CapEst2}
t^m \hbox{c}_m  (\{u - \alpha <  - s - t\},\Omega)  &\leq & \int_{\{u - \alpha < - s\}} (dd^c u)^m \wedge \beta^{n-m}  \nonumber \\
&  \leq & \mu (\{u - \alpha < - s\} ).
\end{eqnarray}

 By definition of $\Gamma$, we deduce that for any $s, t > 0$
\begin{eqnarray*}
t^m\hbox{c}_m  (\{u - \alpha <  - s - t\},\Omega)\, \, \, \leq  \, \, \, \Gamma  (\hbox{c}_m  (\{u - \alpha <  - s \},\Omega)). 
\end{eqnarray*}
 Define $f (s) :=\hbox{c}_m  (\{u - \alpha <  - s \},\Omega)^{1 \slash m}$ for $s > 0$. Then we see that the condition of the Lemma \ref{lem:Kolo} is satified with  $\eta (t) := \gamma^{1 \slash m} (t^m)$.

 Applying  Lemma \ref{lem:Kolo} we conclude that
 $ f (s) = 0 $ for $s \geq S_{\infty}$. This means that $u \geq  \alpha - S_{\infty}$ outside a set of zero capacity. Since such set is of Lebesgue measure zero, it follows that $u \geq  \alpha - S_{\infty}$.
 
 On the other hand by the classical maximum principle we also have $u \leq M = M_{g} := \max_{\partial \Omega} g$ in $\Omega$.
 
 Therefore we have the following uniform bound on $u$ in $\Omega$
 \begin{equation} \label{eq:Ubound}
  \min_{\partial \Omega} g - S_{\infty} \leq u \leq \max_{\partial \Omega} g,
 \end{equation}
  where
  $$
   S_{\infty} \leq  s_0 + \int_0^{e \eta^{-1} (1\slash e)}   \frac{\eta (t)}{t} d t  =    s_0 + (1 \slash m) \int_0^{a} \frac{\gamma^{1 \slash m} (t)}{t} d t,
 $$
 and
$$
 a :=  \left[e \eta^{-1} (1\slash e)\right]^m = e^m \gamma^{-1} (1\slash e^m).
 $$ 
 We need to establish a uniform estimate on the initial time $s_0$. 
 Recall that $s_0$ satisfies $\eta (f (s_0)) \leq 1 \slash e.$ This condition is equivalent to the following one   
 $$
\hbox{c}_m  (\{u <  \alpha   - s_0\},\Omega)  \leq a e^{- m}. 
 $$
 Observe that by (\ref{eq:CapEst2}) we have for $t > 0$
 $$
 \hbox{c}_m  (\{u -  \alpha <  - 2 t\})  \leq \mu (\Omega) \slash t^m
 $$
 Then choosing $t_0 =  e \left(\mu (\Omega)\slash a\right)^{1 \slash m}$ and  $s_0 := 2 t_0 $, we obtain the estimate  $f (s_0) \leq a$ and then 
 $$
 S_{\infty} \leq   2  e \left(\mu (\Omega)\slash a\right)^{1 \slash m}  + \int_0^{a} \frac{\gamma^{1 \slash m} (t)}{t} d t.
  $$
 Therefore from this upper bound and (\ref{eq:Ubound}),  we obtain the uniform estimate (\ref{eq:UnifEst}).
 \end{proof}

\subsection{Existence of a continuous solution : Proof of Theorem 1}

We first prove a weak stability theorem in terms of capacity in the spirit of  a similar result of \cite{EGZ09}.
  
  
 \begin{lemma}\label{lem:CapacityStability}  
Let  $\mu$ be a positive Borel measure with finite mass on $\Omega$ and $u, v$ be two bounded $m$-subharmonic functions on $\Omega$ such that 
$$
 \liminf_{z\to\partial\Omega}(u-v)(z)\geq 0 \, \, \, { and} \, \, \, \, (dd^c u)^m \wedge \beta^{n-m} \leq \mu,
$$
 in the sense of currents on $\Omega$.
Assume that $\mu$ is $\Gamma$-diffuse with respect to the $m$-Hessian capacity and the function  $ \gamma (t) := \Gamma (t) \slash t$ is non-decreasing and
satisfies the Dini type condition (\ref{eq:DiniConditionMu}).

Then  there exists a  uniform constant $B > 0$  such that for any $\varepsilon > 0$, we have
$$
\sup_{\Omega}(v-u)_+ \leq\varepsilon+ B \int_0^{ \varsigma (\varepsilon)} \frac{\gamma^{1 \slash m}  (t)}{t} d t,
$$
where
$$
\varsigma (\varepsilon) := e^m  \left[\mathrm{c} _{m} (\{v - u>\varepsilon\}, \Omega)\right].
$$
% and $B > 0$ is a uniform constant depending only on $\gamma$ and $\mu (\Omega)$.
\end{lemma}


\begin{proof} 
 Fix  $\varepsilon > 0$ and apply Lemma \ref{lem:CapEst} with $t \in [0,1] $ and $s + \varepsilon$. Then  we obtain 
\begin{equation}\label{equa}
t^m \mathrm{c} _{m} (\{u-v<-\varepsilon-t-s\},\Omega)\leq \mu \left(\{u-v<-\varepsilon-s\}\right).
\end{equation}
 Set $ f (s) := \left[\mathrm{c} _{m} (\{u-v<-\varepsilon-s\},\Omega) )\right]^{1 \slash m}$ for $s \in \R^+$.
 Then by the domination condition  we deduce that for $t \in [0,1]$ and $s \in \R^+$
$$
t f (s + t) \leq f (s)  \gamma^{1 \slash m} (f (s)^m)
.$$
Now we can apply Lemma \ref{lem:Kolo} with $\eta (t) :=  \gamma^{1 \slash m} (t^m)$. 

There are two cases to be considered:

1) If $\eta (f (0)) \leq 1 \slash e$, then  by   Lemma \ref{lem:Kolo} we conclude that $ \mathrm{c} _{m} (\{u-v<-\varepsilon-s\},\Omega) = 0$ if $s \geq S_{\infty}$ i.e. 
$$
\sup_{\Omega} (v-u) \leq  \varepsilon + S_{\infty} = \varepsilon + \int_0^{e f (0)} \frac{\eta (t)}{t} d t.
$$
A simple change of variables leads to the estimates.
\begin{equation} \label{eq:estimation1}
\sup_{\Omega} (v-u) \leq  \varepsilon + (1 \slash m) \int_0^{\varsigma  (\varepsilon)} \frac{\gamma^{1 \slash m} (t)}{t} d t,
\end{equation}
where $\varsigma  (\varepsilon) = e^m f(0)^m = e^m  \left[\mathrm{c} _{m} (\{v - u>\varepsilon\}, \Omega)\right]$.

2) If  $\eta (f (0)) > 1 \slash e$, then 
$$ \varsigma  (\varepsilon) := e^m  \left[\mathrm{c} _{m} (\{v - u>\varepsilon\}, \Omega)\right] = e^m f (0)^m  \geq   e^m \left[\eta^{-1} (1 \slash e)\right]^m =: a.
$$
  Hence
$$
\int_0^{\varsigma  (\varepsilon)} \frac{\gamma^{1 \slash m} (t)}{t} d t \geq A := \int_0^{ a} \frac{\gamma^{1 \slash m} (t)}{t} d t.
$$


On the other hand since $u \geq v$ in $\partial \Omega$, by the uniform estimate, we have 
$v - u \leq \max_{\partial \Omega} u - u \leq \mathrm{osc}_{\Omega} u \leq M  = M (\eta,,\Omega),$ hence if we let $B_0:= M  A^{-1}$ we get

\begin{equation} \label{eq:estimation2}
\sup_{\Omega} (v-u) \leq  B_0 \int_0^{\varsigma  (\varepsilon)} \frac{\gamma^{1 \slash m} (t)}{t} d t 
\end{equation}
Comparing the estimates (\ref{eq:estimation1}) and (\ref{eq:estimation2}) we obtain the estimate required in the theorem with  
$B := \max \{B_0,1 \slash m\}$.
\end{proof}


We prove Theorem 1 on the existence of a continuous solution to the Dirichlet problem for diffuse measures satisfying the Dini condition (\ref{eq:DiniConditionMu}).



\begin{proof} The proof will be done in three steps.

1) Existence of a bounded solution. Indeed, since $\mu$ is diffuse wrt to $c_m$, it follows from the generalized Radon-Nikodym theorem that there exits a function $v \in \mathcal{SH}_m (\Omega) \cap L^{\infty} (\Omega)$ and $F \in L^1 (\Omega,\sigma_m(v))$ such that $\mu = F \sigma_m (v)$ on $\Omega$ ( see  \cite{Ceg98}, \cite{Lu12}).

  Set $F_j := \min \{F,j^m\}$  for $j \in \N$. Then we have $\mu_j := F_j \sigma_m (v) \leq j^m \sigma_m (v) = \sigma_m (j v)$ with $j v  \in  \mathcal{SH}_m (\Omega) \cap L^{\infty} (\Omega)$. By the bounded subsolution theorem, there exists 
$u_j \in \mathcal{SH}_m (\Omega) \cap L^{\infty} (\Omega)$ such that $u_j = g$ and $\sigma_m (u_j) = \mu_j = F_j \sigma_m (v)$.
By the comparison principle, the sequence $(u_j)_{j \in \N}$ is decreasing and by Corollary \ref{cor:UnifEst}, the sequence $(u_j)$ is uniformly bounded on $\Omega$. Therefore it converges to $u \in  \mathcal{SH}_m (\Omega) \cap L^{\infty} (\Omega)$. By the continuity of the Hessian operator wrt decreasing sequences, it follows that $\sigma_m (u) = \mu$ weakly on $\Omega$.

2) Boundary values of the solution. Since $\sigma_m (u_j) \leq \sigma_m (u_k) \leq \mu$ for $k \geq j$, it follows that the measures $\mu_j$ are uniformly $\Gamma$-diffuse wrt $c_m$ and then by  Lemma \ref{lem:CapacityStability}, there is a uniform consatnt $B$ such that for any $k \geq j$, 
$$
\sup_{\Omega}(u_j-u_k) \leq\varepsilon+ B \int_0^{ \varsigma_{j,k} (\varepsilon)} \frac{\gamma^{1 \slash m}  (t)}{t} d t,
$$
where
$$
\varsigma_{j,k} (\varepsilon) := e^m  \, \mathrm{c} _{m} (\{u_j - u_k>\varepsilon\}).
$$

Since the sequence $(u_j)$ is decreasing, it follows that it converges in capacity wrt to $c_m$ on each compact set in $\Omega$ (see \cite{Lu12}). 
Observe that since $\lim_{j,k \to + \infty} \varsigma_{j,k} (\varepsilon) = 0$, for $j, k$ large enough, we have $u_j - u_k \leq \varepsilon \slash 2$ near the boundary of $\Omega$. This implies that for $j, k$ large enough, the sets $\{u_j - u_k>\varepsilon\}$ are conained in a fixed compact set in $\Omega$. Therefore  for any $\varepsilon> 0$, $\lim_{j,k \to + \infty} \varsigma_{j,k} (\varepsilon) = 0$.

It follows that  the sequence $(u_j)$ converges uniformly to $u$ on $\Omega$. Hence $u = g$ on $\partial \Omega$.


3) Continuity of the solution $u$: By Lemma \ref{lem:appximationwithbdv}, there exists  a decreasing sequence  $(w_j)$ of continuous $m$-subharmonic functions in $\bar{\Omega}$ which converges to $u$ pointwise in $\Omega$ and such that $w_j = g$ on $\partial \Omega$. 

Fix $\e > 0$ and $j_0 >> 1$ large enough so that $w_j \geq g -\e$ on $\partial \Omega$ for $j \geq j_0$.
Then $\liminf_{z \to \partial \Omega} (u - w_j + 2 \e > 0$ for $j \geq j_0$.
Then applying Lemma \ref{lem:CapacityStability} with $u $ and $v = w_j - 2 \e$ we obtain  for $j \geq j_0$
$$
 w_j - 2 \e - u \leq \e + B \int_0^{\varsigma _j (\e)} \frac{\gamma^{1 \slash m} (t)}{t} d t,
$$

where
$$
\varsigma _j (\e) = e^m \, \mathrm{c}_m (\{ w_j > u + 3 \e\}).
$$
Since $(w_j)$ decreases to $u$ pointwise in $\Omega$, it follows as before that it converges in capacity over any compact subset of $\Omega$. Observe that for $j \geq j_0$, we have $\{u < w_j - 3 \e\} \subset \{u < w_{j_0} - 3 \e\} \Subset \Omega$. Hence $\lim_{j \to + \infty} \varsigma_j (\e) = 0$,and then
$$
\lim_{j \to + \infty} \max_{\bar \Omega}(w_j - u) \leq 4 \e.
$$
As $\e > 0$ is arbitrary, it follows that the sequence $(w_j)$ converges uniformly in $\bar{\Omega}$ to $u$, hence $u$ is continuous in $\bar \Omega$. 
\end{proof}

In the case of infinite mass, we can prove the following result using Theorem 1.
\begin{corollary} \label{cor:continuoussolution} Let  $\mu$ be a positive Borel measure on $\Omega$ with $\mu(\Omega) = + \infty$. Assume that the following two conditions are satisfied

$(i)$  $\mu$  is $\Gamma$-diffuse  with respect to the $m$-Hessian capacity with $\Gamma$  satisfying the  Dini type condition (\ref{eq:DiniConditionMu}), 

$(ii)$  the Dirichlet problem (\ref{eq:DirPb}) admits a bounded subsolution $\psi \in \mathcal{SH}_m(\Omega) \cap L^{\infty} (\Omega)$ i.e.  $(dd^c \psi)^m \wedge \beta^{n-m} \geq \mu$ weakly on $\Omega$ and $\psi = 0$ on $\partial \Omega$.

 Then   for any    continuous boundary datum $g \in \mathcal C^0 (\partial \Omega)$, the Dirichlet problem (\ref{eq:DirPb}) admits a unique solution  $U = U_{\mu,g} \in \mathcal{SH}_m (\Omega) \cap \mathcal C^0  (\bar{\Omega})$.
\end{corollary}
\begin{proof} Let $(K_j)_{j \in \N}$ be  an increasing sequence of relatively compact Borel subsets of  $\Omega$ such that $\Omega = \cup_j K_j$.
Set $\mu_j := {\bf 1}_{K_j}$ for $j \in \N$.  By Theorem  1,  for each $j \in \N$ there exists  $u_j \in \mathcal{SH}_m(\Omega) \cap \mathcal{C}^0 (\bar{\Omega})$ such that $(dd^c u_j)^m \beta^{n-m} = \mu_j$ and $u_j = g$ on $\partial \Omega$. By the comparison principle, $(u_j)$ is a decreasing sequence.

On the other hand let $w_g$ the maximal $m$-subharmonic function on $\Omega$ with $w_g = g$ on $\partial \Omega$. Then $\psi_g := \psi + w_g \in \mathcal{SH}_m(\Omega) \cap L^{\infty} ({\Omega})$ and $(dd^c \psi_g)^m \wedge \beta^{n-m} \geq \mu \geq \mu_j$. By by the comparison principle it follows that $w_g \geq u_j \geq \psi_g$ on  $\Omega$.
Therefore $(u_j)$ decreases to a function $u \in \mathcal{SH}_m(\Omega) \cap L^{\infty} ({\Omega})$ such that $u=g$ on $\partial \Omega$  and $(d^c u)^m \wedge \beta^{n-m} = \mu$ weakly on $\Omega$. 

We need to prove that $u$ is  continuous on $\bar{\Omega}$. Indeed  choose $K_j := \{ \psi < -\varepsilon_ j\}$, for $j \in \N$,  where $(\varepsilon_j)_{j \in \N}$ is decreasing sequence of positive numbers converging to $0$ so that $\mu (\{ \psi = -\varepsilon_ j\}) = 0$  for any $j \in \N$. Then $ \psi_j := u_j + \max \{\psi,-\varepsilon_ j\}  \in \mathcal{SH}_m(\Omega) \cap L^{\infty} ({\Omega})$, $\psi_j = g$ on $\partial \Omega$ and we have
$$
(dd^c \psi_j)^m \wedge \beta^{n-m} \geq (dd^c u_j)^m \wedge \beta^{n-m} + (dd^c  \max \{\psi,-\varepsilon_ j\})^m \wedge \beta^{n-m} = \mu,
$$
 weakly on $\Omega$.
 
By the comparison principle it follows that 
$$u_j + \max \{\psi,-\varepsilon_ j\} \leq u \leq u_j \, \, \, \text{on} \, \, \,  \bar{\Omega}.
$$ 
This proves that $(u_j)$ converges   to $u$ uniformly on $\bar{\Omega}$, hence $u$ is continuous on $\bar{\Omega}$.
\end{proof}
 
\subsection{Weak uniform stability theorem}
The role of the weak stability theorem $L^{\infty}$-$L^{1}$ was discovered  in \cite{EGZ09}, were  the H\"older continuity of the solution to the Dirichlet problem for the complex Monge-Amp\`ere equation on compact homogenous manifolds was proved. Since then, this result became the main tool in deriving estimates on the modulus of continuity of solutions to  the complex Monge-Amp\`ere and Hessian equations.

In order to estimate the modulus of continuity of the solution in this general context, we need to prove an analoguous result.

Denote by
\begin{equation} \label{eq:integral-mu}
J_\Gamma  (\tau) := \int_0^\tau \, \frac{\gamma^{ 1 \slash m} (t)}{t } d t, \, \, \tau \in \R^+.
\end{equation}

\begin{theorem} \label{thm:stability} Let  $\mu$ be a positive  Borel measure on $\Omega$ with finite mass. Assume that $\mu$ is $\Gamma$-diffuse  with respect to the $m$-Hessian capacity and  satisfies the  Dini type condition (\ref{eq:DiniConditionMu}).

Let   $u, v\in \mathcal{SH}_m (\Omega) \cap L^{\infty} (\Omega)$ be  such that $ \liminf_{z\in\partial\Omega}(u-v)(z)\geq 0 $ and
$$
 (dd^c u)^m \wedge \beta^{n-m} \leq \mu,
$$
 in the sense of currents on $\Omega$. 
 
 Then
 $$
 \sup_{\Omega} (v-u)_+ \leq B h_\Gamma  (e^m \Vert v - u)_+\Vert_{m,\mu}^m)
 $$
 where $\Vert (v - u)_+\Vert_{m,\mu}^m := \int_\Omega  (v - u)_+^m d \mu$ and $h_\Gamma$ is the reciprocal  of the  function $s \longmapsto s^{2m}  J_\Gamma^{-1} (s)$ on $\R^+$.
\end{theorem}
Observe that $h$ is a continuous increasing  function on $\R^+$ such that $h(0) = 0$. 
\begin{proof}  We fix $\varepsilon > 0$ and apply Lemma \ref{lem:CapacityStability} to obtain the estimate
$$
\sup_{\Omega}(v-u)_+ \leq\varepsilon+ B \int_0^{ \varsigma (\varepsilon)} \frac{\gamma^{1 \slash m}  (t)}{t} d t,
$$
where
$$
\varsigma (\varepsilon) := e^m   \, \mathrm{c} _{m} (\{v - u>\varepsilon\}, \Omega).
$$
To estimate $\varsigma (\varepsilon)$ we apply  Lemma \ref{lem:CapEst} with $s = t = \varepsilon \slash 2$ which yields
\begin{eqnarray}
\mathrm{c} _{m} (\{v - u>\varepsilon\}, \Omega)&\leq& 2^m \varepsilon^{-m} \int_{v - u \geq \varepsilon\slash 2} (dd^c u)^m \wedge \beta^{n-m} \nonumber \\
&\leq & 2^{2 m} \varepsilon^{-2 m} \int_\Omega (v - u)_+^m (dd^c u)^m \wedge \beta^{n-m}
\end{eqnarray}
Hence
$$
\varsigma (\varepsilon) \leq 2^{2 m}  e^m \varepsilon^{-2 m } \Vert (v - u)_+\Vert_{m,\mu}^m.
$$
Then by Lemma \ref{lem:CapacityStability}  
$$
\sup_{\Omega}(v-u)_+ \leq \varepsilon+ B J_\Gamma \left( 2^m e^m \varepsilon^{-2 m} \Vert (v - u)_+\Vert_{m,\mu}^m\right).
$$
Therefore if we choose $\varepsilon := h_\Gamma(2^m e^m  \Vert (v - u)_+\Vert_{m,\mu}^m)$ we obtain the required estimate.
\end{proof}
%\subsection{The modulus of continuity of the solution}

%\subsection{The modulus of continuity of the solution}

\begin{corollary} Let  $\mu$ be a positive  Borel measure on $\Omega$ with finite mass. Assume that $\mu$ is $\Gamma$-diffuse  wrt the $m$-Hessian capacity with $\Gamma (t) := t^{1 + a}$, where $a > 0$.
Let   $u, v\in \mathcal{SH}_m (\Omega) \cap L^{\infty} (\Omega)$ be  such that $ \liminf_{z\in\partial\Omega}(u-v)(z)\geq 0 $ and
$$
 (dd^c u)^m \wedge \beta^{n-m} \leq \mu,
$$
 in the sense of currents on $\Omega$. 
 
 Then there exists a uniform constant $A  = A (a,m) >0$ such that
 $$
 \sup_{\Omega} (v-u)_+ \leq A \left(\Vert (v - u)_+\Vert_{m,\mu}\right)^{\nu}
 $$
 where $\Vert (v - u)_+\Vert_{m,\mu} := \left[\int_\Omega  (v - u)_+^m d \mu\right]^{1 \slash m}$ and $\nu := \frac{a }{2 a  +1}$.
\end{corollary}

\begin{proof} It is a straight forwards consequence of Theorem \ref{thm:stability}. Indeed it is enough to compute $h_\Gamma$ in this case. A simple computation shows that $J_\Gamma (\tau) = \frac{m}{a} \tau^{a\slash m}$. Hence 
$h_\Gamma (t) = C(a,m)  t^{\frac{a}{2 a m +m}}$ for $t > 0$.
\end{proof}

\section{ Mass estimates for Hessian measures}

For the proof of Theorem 2, we will use the same method  as in \cite{BZ20} which was inspired by an idea in \cite{KN19}. However, since our measure has not a compact support nor a bounded mass, we need to use the control on the behaviour of the mass of the $m$-Hessian  of the subsolution close to the boundary, given by Lemma \ref{lem:ComparisonIneq}.  
 
 \subsection{Proof of Theorem 2}
Recall the volume estimate stated before.

Let us fix $0 < r < m \slash (n-m)$ and $0 < b < 2 n$ and define the following function on $\R^+$:
\begin{equation}\label{eq:estimatefunction}
\ell_m (t) := \left\{\begin{array}{lcl} 
 t^{r}, \,  \, \, \, \, \, \, \, \, \, \hbox{if} \, \,   1 \leq m < n, \\
   \exp ( -b \, t^{- 1 \slash n}), \, \, \,  \hbox{if} \, \,  m = n.
\end{array}\right.
\end{equation}
 
Then the estimates  (\ref{eq:DK}) and (\ref{eq:ACKPZ}) can be written as follows: there exists a constant $B_m > 0$  such that for any Borel set $S \subset \Omega$, 
 \begin{equation} \label{eq:volumeestimate}
 \lambda_{2 n} (S) \leq B_m  \,  \ell_m \left(\text{c}_m (S,\Omega)\right) \, \text{c}_m (S,\Omega),
  \end{equation}
  where $B_m$ depends on $m,r$ and $\Omega$ when $m<n$ and  $B_n$ depends on $n, b$ and $\Omega$

 Recall that $\varphi \in \mathcal{SH}_m (\Omega)  \cap \mathcal{C}^0 (\bar{\Omega})$ with $\varphi = 0$ on $\partial \Omega$ and we want to estimate the mass of the Hessian measure $\sigma_m (\varphi)$ on compact sets in $\Omega$.
 \begin{proof} 
We extend $\varphi$ as a continuous function in the whole of $\C^n$ with the same modulus of continuity and denote by $\varphi$ the extension.
 Then denote by $\varphi_{\delta}$ ($0 < \delta < \delta_0$) the smooth approximants of $\varphi $ in $\Omega$, defined as usual, for $z \in \bar \Omega$, 
   $$
    \varphi_{\delta}(z)=  \int_{\Omega}\varphi(\xi)\chi_{\delta}(z-\xi) d\lambda(\xi).
    $$

 Observe that for $0 < \delta < \delta_0$ and $z\in\Omega_{\delta}  := \{z \in \Omega ; \mathrm{dist} (z, \partial \Omega) > \delta\}$, 
     $$
     \varphi_{\delta}(z)=  \int_{\Omega}\varphi(z - \zeta)\chi_{\delta}(\zeta) d\lambda(\zeta),
     $$
  and then $\varphi_\delta \in  \mathcal{SH}_m (\Omega_{\delta})\cap\mathcal{C}^{\infty}(\C^n)$.
 
 Since $\varphi \in C^{0} (\bar\Omega)$, we have $\varphi_\delta \leq \varphi + \kappa_{\varphi} (\delta)$ on $\Omega$. 


 We consider the $m$-subharmonic envelope of   $\varphi_\delta$ on  $\Omega$  defined by the formula 
 $$
 \psi_\delta := \sup \{\psi \in \mathcal{SH}_m (\Omega) ; \psi \leq \varphi_\delta \, \, \, \hbox{on} \, \, \,  \Omega \}\cdot 
 $$ 
 
 It follows from  \cite[Theorem 3.3]{BZ20}  that $\psi_\delta \in \mathcal{SH}_m (\Omega)$ and $\psi_\delta \leq \varphi_\delta$ on $\Omega$. 
  
 Fix $0 < \delta < \delta_0$ and a compact set $K \subset \Omega_\delta$ and consider the following set
 
 $$
 E :=\{3 \kappa (\delta)  u_K^*+ \psi_\delta<\varphi- 2 \kappa (\delta)\} \subset \Omega.
 $$

 
 Since $ \kappa$ is the modulus of continuity of $\varphi $ on $\bar \Omega$, we have  $\varphi -  \kappa (\delta) \leq \varphi_{\delta}  \leq \varphi + \kappa (\delta)$ on $\Omega$ and then   $\varphi -  \kappa (\delta) \leq \psi_\delta \leq  \varphi_{\delta}  \leq \varphi (z) + \kappa (\delta)$  on $\Omega$. 
Therefore  $\liminf_{z \to \partial \Omega} (\psi_\delta - \varphi + \kappa (\delta)) \geq 0$, and then $E \Subset \Omega$. By the comparison principle, we conclude that
 
\begin{eqnarray} \label{eq:fundmentalestimate}
 \int_{E}(dd^c\varphi)^m\wedge\beta^{n-m} & \leq & \int_{E}(dd^c(3 \kappa (\delta)  u_K^* + \psi_{\delta}))^m\wedge\beta^{n-m} \nonumber \\
 & \leq & 3 \kappa (\delta) L \int_{E}(dd^c(u_K^* + \psi_{\delta}))^m\wedge\beta^{n-m} \\
 &+&\int_{E}(dd^c\psi_{\delta})^m\wedge\beta^{n-m}, \nonumber
 \end{eqnarray}
 where $L := \max_{0 \leq j \leq m - 1} (3 \kappa (\delta_0))^j$.
 
 Observe that $-1 + \varphi -  \kappa (\delta) \leq u_K^* + \psi_\delta  \leq \varphi + \kappa (\delta)$ on $\Omega$, hence   $\vert u_K^* + \psi_\delta\vert \leq \sup_{\Omega}  \vert \varphi\vert + 1 + \kappa (\delta_0) =: M_0$ on $\Omega$. 
 
 Therefore from inequality (\ref{eq:fundmentalestimate}), it follows that
 
\begin{equation} \label{eq:finalestimate1}
  \int_{E}(dd^c\varphi)^m\wedge\beta^{n-m} \leq 3 \kappa (\delta) L  M_0^m  \text{c}_m (E,\Omega) + \int_{E}(dd^c\psi_{\delta})^m\wedge\beta^{n-m}.
\end{equation}
 
 Moreover we have
 
 \begin{equation}\label{eq:1}
 dd^c\varphi_{\delta}\leq\frac{M_1 \kappa (\delta)}{\delta^{2}}  \beta, \, \, \, \mathrm{pointwise \, on} \, \, \, \Omega,
 \end{equation}
 where $M_1 > 0$ is a uniform constant depending only on $\Omega$.

By \cite[Theorem 3.3]{BZ20}, we have

\begin{equation}\label{eq:2}
 (dd^c\psi_{\delta})^m\wedge\beta^{n-m}  \leq  (\sigma_m (\varphi_{\delta}))_+ \leq\frac{M_1^m \kappa (\delta)^{m }}{\delta^{2 m}}  \beta^n,
 \end{equation}
 in the sense of currents on $ \Omega$.
 
 Therefore  
\begin{equation}\label{eq:3}
\int_{E}(dd^c\psi_{\delta})^m\wedge\beta^{n-m} \leq   M_1^{m} \kappa (\delta)^{m } \delta^{-2 m} \lambda_{2 n} (E).
 \end{equation}
 
 
Let us denote for simplicity $c_m (\cdot) :=  \text{c}_m (\cdot,\Omega)$. Then from (\ref{eq:finalestimate1}) and (\ref{eq:1}), we deduce that
\begin{equation}\label{eq:finalestimate2}
\int_{E}(dd^c\varphi)^m\wedge\beta^{n-m} \leq  3 \kappa (\delta) L M_0^m  \text{c}_m (E)+ M_1^{m} \kappa(\delta)^m \delta^{-2 m} \lambda_{2 n} (E) 
\end{equation} 

From (\ref{eq:finalestimate2}) and (\ref{eq:volumeestimate}), it follows that 
\begin{eqnarray} \label{eq:finalestimate3}
\int_{E}(dd^c\varphi)^m\wedge\beta^{n-m} & \leq &   3 \kappa (\delta) L M_0^m  \text{c}_m (E) \nonumber\\
&+&  B_m M_1^{m} \kappa(\delta)^m \delta^{-2 m} \ell_m (\text{c}_m (E)) \text{c}_m (E). 
\end{eqnarray} 


Since  $\varphi - \kappa (\delta) \leq \psi_\delta  \leq  \varphi_{\delta}  \leq \varphi + \kappa (\delta)$  on $\Omega$, it follows that  
 $ E \subset\{u_K^* < - 1 \slash 3\}.$ 
 
The comparison principle yields the follwoing estimate

 \begin{equation}\label{eq:4}
  \text{c}_m  (E,\Omega) \leq 3^{m}   \text{c}_m (K,\Omega).
 \end{equation}
Indeed fix $v \in SH_m (\Omega)$ with $- 1 \leq v \leq 0$. Then $ E \subset\{ 3 u_K^* < - 1 \} \subset \{3 u_K^* < v\} \Subset \Omega$ and  the comparison principle implies that
 $$
 \int_{E} (dd^c v)^m \wedge \beta^{n - m} \leq  \int_{\{ 3 u_K^* <  v\}} 3^m  (dd^c u_K^*)^m \wedge \beta^{n - m} \leq 3^m \text{c}_m (K,\Omega).
 $$
Taking the supremum over $v$ we obtain the estimate (\ref{eq:4}).

Since $K \setminus \{u_K < u_K^*\} \subset  \{ u_K^* = - 1\} \subset E$ and $K \cap \{u_K < u_K^*\}$ has zero capacity, we see that $\int_K (dd^c\varphi)^m\wedge\beta^{n-m} \leq  \int_E (dd^c\varphi)^m\wedge\beta^{n-m}$. 

Therefore  we finally deduce from (\ref{eq:finalestimate2}), (\ref{eq:volumeestimate}) and  (\ref{eq:2}) that  for a fixed $0 < \delta < \delta_0$ and any compact set $K \subset \Omega_\delta$,  we have 
\begin{eqnarray*}
\int_K (dd^c\varphi)^m\wedge\beta^{n-m} &\leq & A_0 \kappa (\delta)  c_m (K) \\
&+& A_1  \kappa (\delta)^m \delta^{-2m}   \ell_m (c_m (K)) \,  c_m (K).
\end{eqnarray*} 
where $A_0 :=  3^{m + 1}  L_0 M_0^m$ and $A_1 :=  M_1^{m}  3^{m r}  $.


By inner regularity of the capacity, we deduce that the previous estimate holds for any Borel subset $S \subset \Omega_\delta$ i.e.
\begin{eqnarray} \label{eq:estimate1}
\int_S (dd^c\varphi)^m\wedge\beta^{n-m} &\leq & A_0 \kappa (\delta)  c_m (S) \\
& +&  A_1 \kappa (\delta)^m \delta^{-2m}   \ell_m (c_m (S)) \,  c_m (S). \nonumber
\end{eqnarray} 

Let $K \subset \Omega$ be any fixed compact set and $0 < \delta < \delta_0$. Then

$$
\int_K (dd^c\varphi)^m\wedge\beta^{n-m}  = \int_{K \cap \Omega_\delta} (dd^c\varphi)^m\wedge\beta^{n-m}  + \int_{K \setminus \Omega_\delta} (dd^c\varphi)^m\wedge\beta^{n-m}.
$$

We will estimate each term separately. By (\ref{eq:estimate1}) the first term is estimated easily: for $0 < \delta < \delta_0$, we have

\begin{equation} \label{eq:estimate2}
\int_{K \cap \Omega_\delta} (dd^c\varphi)^m\wedge\beta^{n-m}  \leq A_0 \kappa (\delta) c_m (K) +  A_1  \kappa (\delta)^m \delta^{-2m}   \ell_m (c_m (K)) \,  c_m (K).
\end{equation} 

To estimate the second term we apply Lemma \ref{lem:ComparisonIneq} for the Borel set $ S := K \setminus \Omega_\delta$. Since $\delta_B (\partial \Omega) \leq \delta$ we get
$$
\int_{K \setminus \Omega_\delta} (dd^c\varphi)^m\wedge\beta^{n-m}  \leq \kappa (\delta)^{m} c_m (K).
$$

Finally we obtain for $0 < \delta < \delta_0$, 
\begin{eqnarray} \label{eq:finalestimate3}
\int_{K}(dd^c\varphi)^m\wedge\beta^{n-m}  &\leq &  3 \kappa_\varphi(\delta) L M_0^m  \text{c}_m (E,\Omega)\\
&+& M_1^{m} \kappa_\varphi (\delta)^m \delta^{-2 m} \lambda_{2 n} (E). \nonumber
\end{eqnarray} 

Therefore we finally deduce from (\ref{eq:finalestimate2}), (\ref{eq:finalestimate3}), (\ref{eq:DK}) and  (\ref{eq:2}) that  for a fixed $0 < \delta < \delta_0$ and any compact set $K \subset \Omega$,  we have 
\begin{eqnarray}\label{eq:finalestimate4}
\int_K (dd^c\varphi)^m\wedge\beta^{n-m} & \leq & A_0 \, \kappa(\delta) \,   c_m (K)\\
&  + &  A_2 \,  \kappa (\delta)^{m}  \, \delta^{-2m} \,  \ell_m (c_m (K)), \nonumber
\end{eqnarray} 
where $\kappa = \kappa_\varphi$,  $A_0 :=  2^{m + 1}  L_0 M_0^m$ and $A_2 :=  3^m N (r)  M_1^{m}   $.


We want to optimize the right hand side of (\ref{eq:finalestimate4}) by choosing  $\delta $ so that $\kappa(\delta)  = \kappa (\delta)^{m} \delta^{-2m} \ell_m (c_m (K)) $ i.e. 
$$
\kappa (\delta)^{1-m} \delta^{2 m}  = \ell_m (c_m (K)).
$$

Observe that the function $x \longmapsto \kappa (x)^{1-m}  x^{2 m}$ is continuous on $\R^+$ and takes the values $0$ at $t=0$ and $+ \infty$ at $ t = +\infty$. Therefore for any $y > 0$, the equation $y = \kappa (x)^{1-m}  x^{2 m}$ has at least one solution $x >0$ . Let us define  
the lower inverse function of the function $x \longmapsto \kappa (x)^{1-m}  x^{2 m}$ by  the following formula:
 \begin{equation} \label{eq:lowerinverse}
 \theta_m (y) :=  \inf \{x > 0 ;  \kappa (x)^{1-m}  x^{2 m} = y \}, y > 0.
 \end{equation}
 
Choose $\delta >0$ to be  $\delta = \theta_m\left[\ell_m (c_m(K)\right]$. 

Set $\vartheta_m (t) := \kappa \circ \theta_m (\ell_m (t))$ and observe that if $ \delta_K(\partial \Omega) \leq \vartheta_m  (c_m(K))$, then by Lemma \ref{lem:ComparisonIneq} we get
 \begin{eqnarray} \label{eq:finalIneq1}
   \int_{K}(dd^c\varphi)^m\wedge\beta^{n-m}  \leq  [\vartheta (c_m(K))]^m \, c_m (K).
  \end{eqnarray}
 
  Now assume that $ \vartheta_m  (c_m(K)) < \delta_K (\partial \Omega) \leq \delta_0$. Then we can take $\delta := \vartheta_m  (c_m(K))$ in the  inequality (\ref{eq:finalestimate4}) and get
  
  \begin{equation}\label{eq:finalIneq2}
   \int_{K}(dd^c\varphi)^m\wedge\beta^{n-m} \leq B \, \vartheta (c_m (K) \, c_m (K)  + [\vartheta (c_m (K))]^m \,  c_m (K).
  \end{equation}
  Combining inequalities  (\ref{eq:finalIneq1} ) and (\ref{eq:finalIneq2}), we obtain the estimate of the theorem with the constant $B$ given by the following formula:
  \begin{equation} \label{eq:finalConst}
   B := A_0  + A_1 + 1.
\end{equation}   
\end{proof}


\subsection{Some consequences}

For $m < n$ we recover the result of (\cite{BZ20}).
\begin{corollary}  Let $\Omega \Subset \C^n$ be a $m$-hyperconvex  domain and $\varphi\in \mathcal{SH}_m (\Omega)\cap \mathcal{C}^{\alpha} (\overline\Omega)$ such that $\varphi = 0$ in $\partial \Omega$.  

 Then for any  $0 < \epsilon <\alpha m \slash  [ (n-m) (2m + \alpha (1-m))]$,  there exists a constant $ A = A (m,n,\alpha, \epsilon, \Omega)>0$ such that for every compact $K\subset\Omega$, we have
$$
\int_ K (dd^c\varphi)^m \wedge \beta^{n-m} \leq A \left[\text{c}_m (K)\right] ^{1 + \epsilon}.
$$ 

\end{corollary}

\begin{proof}
Since $\kappa_\varphi (t) = \kappa_0 t^{\alpha}$, by Theorem 2, we have for any compact $K \subset \Omega$,
$$
\int_{K}(dd^c\varphi)^m\wedge\beta^{n-m}  \leq  A  \, \left\{\vartheta_m  (c_m (K)) + \left[\vartheta_m  (c_m (K))\right]^m\right\} \, c_m (K),
$$
where
$\vartheta_m (t) := \kappa \circ \theta_m \circ \ell_m (t)$ and $\theta_m^{-1} (t) := t^{2 m+ \alpha(1 - m)}$. 
On the other hand $\ell_m (t) = t^{r}$ with $0 < r < m \slash (n-m)$, hence $\vartheta_m (t) = t^{\alpha r \slash [2m + \alpha (1-m)]}$.
\end{proof}
For $m=n$ we obtain a much more precise result.

\begin{corollary}  \label{cor:MongeAmpereMass} Let $\Omega \Subset \C^n$ be a hyperconvex  domain.
 Let $\varphi\in \mathcal{PSH} (\Omega)\cap \mathcal{C}^{\alpha} (\overline\Omega)$ such that $\varphi = 0$ in $\partial \Omega$.  Then for any $0 < q <  \frac{ 2 n \alpha}{  2 n+  (1 - n) \alpha}$, there exists a constant $ Q = Q (n,q,\alpha,\Omega)>0$ such that for every compact $K\subset\Omega$, 
$$
\int_ K (dd^c\varphi)^n \leq Q \, \exp \left( - q \left[\text{c}_n (K,\Omega)\right]^{- 1 \slash n}\right).
$$ 
\end{corollary}
\begin{proof} Since $\kappa_\varphi (t) = \kappa_0 t^{\alpha}$, by Theorem 2, we have for any compact $K \subset \Omega$,
$$
\int_{K}(dd^c\varphi)^m\wedge\beta^{n-m}  \leq  A  \, \left\{\vartheta_m  (c_m (K)) + \left[\vartheta_m  (c_m (K))\right]^m\right\} \, c_m (K),
$$
where
$\vartheta_m (t) := \kappa \circ \theta_m \circ \ell_m (t)$ and $\theta_m^{-1} (t) := t^{2 m+ \alpha(1 - m)}$. 

When $m = n$ we have $\ell_n (t) = \exp (-b t^{-1 \slash n})$ with $b < 2 n$, hence $\vartheta_n (t) = \exp (- q  t^{-1 \slash n})$, where $q = \frac{\alpha b}{ 2 n+  (1 - n) \alpha}$.
\end{proof}

From this result, we deduce a global exponential  integrability theorem for plurisubharmonic functions in the Cegrell class with respect to Borel measures with  H\"older continuous Monge-Amp\`ere potentials. 

Let $\mathcal{F} (\Omega)$ defined as the class of negative plurisubharmonic functions $\psi$ on $\Omega$ such that there exists a decreasing sequence  of plurisubharmonic test functions  $(\psi_j)$ in $\mathcal{E}^0 (\Omega)$ such that $\sup_j \int_{\Omega} (dd^c \psi_j)^n < + \infty$ (see \cite{Ceg04}).

 We denote by  $\dot{\mathcal{F}} (\Omega)$ the set of functions $\psi \in \mathcal{F} (\Omega)$ normalized by the condition $\int_\Omega (dd^c \psi)^n \leq 1$.
\begin{corollary}  Let $\Omega \Subset \C^n$ be a hyperconvex  domain and $\varphi\in \mathcal{PSH} (\Omega)\cap \mathcal{C}^{\alpha} (\overline\Omega)$ such that $\varphi = 0$ in $\partial \Omega$.   Then   for any $0 < q < q_n (\alpha) := \frac{ 2 n \alpha}{   2n + (1- n) \alpha  }$, there exists a constant $ \tilde Q = \tilde Q (n,q,\alpha,\Omega)>0$ such that for  any  $\psi \in \dot{\mathcal{F}} (\Omega)$, we have
 $$
 \int_\Omega \left(e^{-q \psi} - 1\right) (dd^c \varphi)^n \leq  \tilde  Q.
 $$
\end{corollary} 
\begin{proof}  Indeed we have
$$
\int_\Omega\left(e^{-\varepsilon \psi} - 1\right) (dd^c \varphi)^n = \varepsilon \int_0^{+\infty} e^{\varepsilon t} \int_{ \{\psi < - t\}} (dd^c \varphi)^n \, d t.
$$
On the other hand, observe that for $\psi \in \dot{\mathcal{F}} (\Omega)$ we have for any $t > 0$
$$\text{c}_n  ( \{\psi < - t\}) \leq t^{- n} \int_\Omega (dd^c \psi)^n = t^{- n}.
$$ 
 Therefore applying Corollary \ref{cor:MongeAmpereMass} we obtain
$$
\int_ \Omega \left(e^{-\varepsilon \psi} - 1\right)  (dd^c \varphi)^n \leq  \varepsilon \,  Q \int_0^{+\infty} e^{\varepsilon t} e^{- q t} d t = Q \, \frac{\varepsilon}{q -\varepsilon} =: \tilde Q.
$$
\end{proof}
Local exponential integrability properties of plurisubharmonic functions with respect to
Borel measures with  H\"older continuous Monge-Amp\`ere potentials were first obtained in \cite{DNS10}.
\section{ Modulus of continuity of the solution}

Now we are ready to prove Theorem 3 and Theorem 4 from the introduction using Theorem 2, Corollary 3.10  and Theorem \ref{thm:ModC}.


\subsection{Proofs of Theorem 3 and Theorem 4}
Recall that we are given  a Borel measure $\mu$ on $\Omega$ such that there exists $\varphi\in \mathcal{SH}_m(\Omega)\cap\mathcal{C}^{0}(\overline\Omega)$ with $\varphi_{\mid \partial \Omega} \equiv 0$ and satisfying the inequality  $ (dd^c\varphi)^m\wedge\beta^{n-m} \geq \mu$ weakly on $\Omega$. 

The goal is first to prove that if modulus of continuity  $\kappa_\varphi$ of $\varphi$ satisfies the  Dini type condition (\ref{eq:DC1}) for $1 \leq m < n$  or (\ref{eq:DC2}) for $m=n$  respectively, then  for any boundary value datum $g \in \mathcal C^0 (\partial \Omega)$, the Dirichlet problem (\ref{eq:DirPb}) has a unique continuous solution.
Moreover  when $\mu (\Omega) < + \infty$, we will give an estimate on the modulus of continuity of the solution.

Recall that the function $ h_m $ is defined by its  reciprocal as follows $\tau = h_m (t)$ is the unique solution to the following equation:
\begin{equation} \label {eq:hm}
 h_m^{-1} (\tau) := \tau^{2m} J^{-1}_m (\tau), \, \, \, \, \, \, J_m (\tau) :=  \int_0^\tau \left[\kappa_\varphi \circ \theta_m \circ \ell_m (t)\right]^{1 \slash m} \frac{d t}{t},
 \end{equation}
where $\ell_m (s)$ is defined by (\ref{eq:estimatefunction}) and $\theta_m$ is the inverse of the function $t \longmapsto t^{2m} \kappa_\varphi(t)^{1 - m} $.
  
We are going to prove the two theorems at the same time since the proofs only differ in the last step.

\begin{proof}  There are two steps in the proof.

1. {\it Existence of a continuous solution.} 
 Since $\mu\leq (dd^c\varphi)^m\wedge\beta^{n-m}$ and $\varphi = 0$ on $\partial \Omega$,   it follows from Theorem 2,  that $\mu$ is $\Gamma$-diffuse with $\Gamma (t) = t \gamma_m (t)$ and $\gamma_m (t) :=  \kappa_\varphi \circ \theta_m \circ \ell_m (t)$.
 
We claim that that  the conditions (\ref{eq:DC1}) and (\ref{eq:DC2}) of Theorem 3 and Theorem 4 respectively  imply that the Dini condition (\ref{eq:DiniConditionMu}) holds for $\gamma_m$ in both cases.

Indeed assume first that $1 \leq m < n$. Then $\ell_m (t) = t^r$, where $1 < r < m \slash (n-m)$. By the change of variable $s = \ell_m(t) = t^r$ we obtain
$$
\int_{0^+}^1 \gamma (t)^{1 \slash m} \frac{d t}{t} = \frac{1}{r} \int_{0^+}^1 \kappa_\varphi^{1 \slash m} (\theta_m (s)) \frac{d s}{s}.
$$
 Then the change of variables $x = \theta_m (s)$ allows to write $s = \theta_m^{-1} (x) = x^{2 m} \kappa_\varphi^{1 - m} (x)$ which implies $\frac{d s}{s} = 2 m \frac{dx}{x} + (1-m) d \kappa_\varphi (x).$
Then an easy computation shows that 
\begin{eqnarray*}
\int_{0^+}^1 \gamma (t)^{1 \slash m} \frac{d t}{t} &= & \frac{2 m }{r} \int_{0^+}^{\theta_m(1)} \kappa_\varphi^{1 \slash m} (x) \frac{d x}{x}  \\
&+ & \frac{ m(1-m)  }{r}\kappa_\varphi^{1 \slash m}(\theta_m(1)). 
\end{eqnarray*}

This proves that the condition (\ref{eq:DC1})  of Theorem 3 is equivalent to  the Dini condition (\ref{eq:DiniConditionMu}) for the function $\gamma = \gamma_m$.

Now assume that $m = n$. Then $\ell_n (t) = e^{-b \slash t^{1\slash n}}$. We set $s = \ell_n (t)$. Then 
$\frac{d t}{t} = n \frac{d s}{s (-\log s)}$. Hence
$$
\int_{0^+} \gamma (t)^{1 \slash n} \frac{d t}{t} =  n \int_{0^+} \kappa_\varphi^{1 \slash n} (\theta_n (s)) \frac{d s}{s (-\log s)}.
$$
Now observe that $x = \theta_n (s)$ satisfies $s = \theta^{-1} (x) = x^{2 n} \kappa_\varphi^{1 - n} (x)$.

Since $\kappa_\varphi$ is increasing, it follows that  $s \geq c_1 x^{2n}$ and then $ x = \theta_n (s) \leq (s \slash c_1)^{1 \slash 2 n}$ for $s \in ]0,1]$. Therefore
$$
\int_{0^+}^1 \gamma (t)^{1 \slash n} \frac{d t}{t}  \leq  n \int_{0}^{a} \kappa_\varphi^{1 \slash n}  ((s\slash c_1)^{1 \slash 2 n})) \frac{d s}{s (-\log s)},
$$
Now the change of variable $x = (s\slash c_1)^{1 \slash 2 n}$ leads to the inequality

$$
\int_{0^+} \gamma (t)^{1 \slash n} \frac{d t}{t}   \leq  2 n^2 \int_{0^+} \kappa_\varphi^{1 \slash n}  (x) \frac{d x}{ x  (-\log (c_1 x))},
$$  
This shows that the condition (\ref{eq:DC2}) in Theorem 4 implies that the  Dini condition (\ref{eq:DiniConditionMu}) holds for the function $\gamma = \gamma_n$.
This proves our claim about $\gamma_n$. A more careful computation shows that actually the two conditions  (\ref{eq:DC2})  and  (\ref{eq:DiniConditionMu})  are equivalent, but we don't need that here.

 By Corollary \ref{cor:continuoussolution}, it follows that there is a unique function $u\in \mathcal{SH}_m(\Omega)\cap \mathcal{C}^{0}({\Omega}) $ such that 
$$ 
(dd^c u)^m\wedge\beta^{n-m}=\mu,
$$
in the weak sense on $\Omega$ and $ u=g $ on $\partial\Omega$.

\smallskip


{\it Step 2 : Estimation of the partial $\widehat{\kappa}$-modulus of continuity}. 
Assume that $\mu (\Omega) < + \infty$. We want to estimate the modulus of continuity of the solution $u$.

For  $0 < \delta < \delta_0$ and  denote as before by 
 $\widehat{u}_{\delta}(z)$ the mean value  of $u$ on the ball $B (z,\delta) \subset \Omega$.
By Lemma \ref{lem:approximation}, the global approximants  defined for $0 < \delta < \delta_0$ by  
$$
 \tilde{u}_{\delta}:= \left\{ \begin{array}{lcl}
\max\{\hat{u}_{\delta} -  \kappa (\delta),u \} &\hbox{on} & \Omega_{\delta}, \\
u  &\hbox{on} & \Omega\setminus\Omega_{\delta}
\end{array}\right.
$$
with $\kappa (\delta) :=  \kappa_\varphi (\delta) + \kappa_g (\sqrt{\delta}) + \delta$ satisfy the following properties:

\begin{itemize}
\item $\tilde{u}_{\delta}$ is  $m$-subharmonic and bounded on $\Omega$,
\item $\tilde{u}_{\delta}(z) = u (z)$ on $\Omega \setminus \Omega_\delta$
\item $0 \leq \tilde{u}_{\delta}(z) - u (z) \leq  \widehat{u}_\delta  (z) - u (z) \leq \tilde{u}_{\delta}(z) - u (z) + \kappa (\delta) $ for $z\in \Omega_{\delta}$.
\end{itemize}

Therefore we can apply Corollary \ref{cor:approximation} and get for $0<\delta<\delta_0$,
\begin{eqnarray} \label{eq3}
\int_{\Omega}(\tilde{u}_{\delta}-u)^m d\mu & = &   \int_{\Omega_\delta}(\tilde{u}_{\delta}-u)^m d\mu \nonumber \\
  &\leq& C_m \kappa_\varphi \circ \theta_m \left(d M \delta^{2}\right),
\end{eqnarray}
where    $M := \left(\mathrm{osc}_\Omega u\right)^{(m - 1)}$.
 
  By Theorem \ref{thm:stability} it follows that 
 \begin{eqnarray} \label{eq4}
\sup_{\Omega}(\tilde{u}_{\delta}-u)
 & \leq & D \,  h_{m} (2^m e^m \Vert \tilde u_\delta - u\Vert_{m,\mu}^m), 
 \end{eqnarray}
 where $D >0$ is a uniform constant,  and $h_m =  h_{\Gamma_m} $ is defined by the formula (\ref{eq:hm}).
 

 Therefore  from equation  (\ref{eq3}) and (\ref{eq4}) we deduce that
 \begin{equation*} 
\sup_{\Omega}( \tilde{u}_{\delta}-u)  \leq    D \, h_m \left[C_m \kappa_\varphi \circ \theta_m \left(d M \delta^{2}\right)\right], 
 \end{equation*}
 which implies
 \begin{equation} \label{eq:Finaleq}
 \sup_{\Omega_{\delta}}( \tilde{u}_{\delta}-u)  \leq  D  \,   h_m \left[C_m \kappa_\varphi \circ \theta_m  \left( d M \delta^{2} \right)\right].
  \end{equation}
 
% where  $M := \left(\mathrm{osc}_\Omega u\right)^{(m - 1) \slash m}$. 
 
Recall that $ \hat{u}_{\delta} - u \leq \tilde u_\delta - u + \kappa (\delta) $ on $\Omega_\delta$. Hence  for $0 < \delta < \delta_0$
 
 
  \begin{equation} \label{eq5}
   \sup_{\Omega_{\delta}}( \hat{u}_{\delta}-u) \leq \sup_{\Omega}(\tilde{u}_{\delta}-u)+ \kappa (\delta) .
  \end{equation}
  Using (\ref{eq:Finaleq}) and (\ref{eq5}) we finally get for $0 < \delta < \delta_0$
  \begin{equation} \label{eq:MC-estimate}
   \sup_{\Omega_{\delta}}( \hat{u}_{\delta}-u) \leq D \tilde{\kappa} (\delta),
 \end{equation}
 where 
\begin{equation} \label {eq:MC-Solution}
\tilde \kappa (\delta) :=   h_m \left[C_m \kappa_\varphi \circ \theta_m  \left(d M \delta^{2} \right)\right]  +  \kappa_\varphi (\delta) + \kappa_g (\sqrt{\delta}) + \delta,
 \end{equation}
 and $h_m$ is defined by (\ref{eq:hm}).
 \end{proof}

 




\subsection{Some consequences}
Let us state  corollaries of Theorem 3 and Theorem 4 to show how the  estimates obtained so far are more precise compared to previous ones (see \cite{KN19}, \cite{BZ20}).

\begin{corollary} \label{cor:HolderHess}
Let $\Omega \Subset \C^n$ be a  bounded strongly $m$-pseudoconvex  domain  with $1 \leq  m \leq n$ and $\mu $ a positive Borel measure on $\Omega$. Assume that there exists $\varphi\in \mathcal{SH}_m(\Omega)\cap\mathcal{C}^{\alpha}(\overline\Omega)$ such that  
\begin{equation} \label{eq:subsol2}
 \mu \leq (dd^c\varphi)^m\wedge\beta^{n-m}, \, \, \, \mathrm{weakly \, \, on} \, \,  \Omega \, \, \, \mathrm{and} \, \, \, \varphi_{\mid{\partial \Omega}} \equiv 0.
\end{equation} 
 

 Then for any continuous function $g \in \mathcal{C}^{2 \alpha} (\partial \Omega)$, there exists a unique function $U = U_{g,\mu} \in \mathcal{SH}_m (\Omega) \cap \mathcal{C}^{0} (\bar{\Omega})$ such that  
 $$
 (dd^c U)^m\wedge\beta^{n-m} = \mu, \, \, \, \mathrm{and} \, \, \, U = g \, \, \, \mathrm{on} \, \, \,  \partial \Omega.
 $$
 Moreover if $\mu (\Omega) < + \infty$, $U \in \mathcal{C}^{\tilde{\alpha}} (\bar{\Omega})$ for any $\tilde{\alpha} <  \tilde{\alpha}_m$,
where

 \begin{equation} \label{eq:Holderexp}
 \tilde \alpha_m := \frac{ 2 \tilde r   \alpha^2}{ m \tilde{m} \left[ \tilde m   + 2 \alpha \tilde r\right]},
 \end{equation}
and  $ \tilde r:= \frac{m}{n-m}$ and $\tilde m := 2 m + \alpha(1-m)$.
 \end{corollary}

 \begin{proof}  We want to apply  Theorem 3.  Here we have $\kappa_\varphi (t) = \kappa_0 t^\alpha$, $\ell_m (t) = t^{r}$ with $0 < r < m \slash (n-m)$. By Theorem 2, $\mu$ is $\Gamma$-diffuse with $\Gamma (t) = A t  \kappa_\varphi \circ \theta_m \left(\ell_m (t)\right)$ and $\theta_m$ is the inverse  of the function $t \longmapsto t^{2m} \kappa_\varphi (t)^{1 - m} = t^{2m + \alpha (1-m)}$. Thus $\Gamma (t) = A  t^{1 + r \alpha \slash \tilde m}$.
 
 
 Then $ J_m (\tau) = A^{1\slash m} \frac{ m \tilde m}{r \alpha} \tau^{ \alpha r \slash m \tilde m} $ and 
 $h^{-1}_m (t) =A' (m,\alpha)  \, \,  t^{2 m +  m \tilde m \slash  \alpha r}$.  
 
 Finally  by (\ref{eq:MC-Solution}) the $\widehat{\kappa}$-modulus of continuity of the solution is dominated as follows 
 
$$
\widehat{\kappa}_U (\delta) \leq  C'(\alpha,m,n,\Omega) \delta^{\frac{2  r \alpha^2 }{ m \tilde m \left[  \tilde m + 2 \alpha  r \right]}}. 
 $$ 
 As before  we apply Lemma \ref{lem:sup-mean} to concude.  \end{proof}
 
 \begin{corollary}  \label{cor:HolderMA} Under the assumption of Theorem 4, with $\varphi \in C^\alpha (\bar{\Omega}$ with $0 < \alpha \leq 1$ and $g \in \mathcal{C}^{2 \alpha} (\partial \Omega)$,  the solution $U := U_{g,\mu}$ to the Dirichlet problem is H\"older continuous and its modulus of continuity satisfies the  following  estimate
$$
\kappa_U (\delta) \leq C \delta^{\alpha \slash n \tilde n} (- \log \delta)^{1\slash 2},
$$  
where $\tilde n := ( 2-\alpha) \,  n + \alpha $ and $C > 0$ is a positive uniform constant. 
   \end{corollary}
Here $\kappa_U$ is the usual modulus of continuity of $U$ on $\bar{\Omega}$ defined as follows:
\begin{equation}
\kappa_U(\delta) := \sup \{ \vert U (z) - U(z') \vert \, ; \, z, z' \in  \bar{\Omega}, \vert z - z'\vert \leq \delta\}%\cdot
\end{equation}

The precise relationship between $\kappa_U$ and $\widehat{\kappa}_U$ will be discussed in section 2.3 (see \cite{Ze20} for more details).  
\begin{proof}
We want to apply  Theorem 4.  Here we have $\kappa_\varphi (t) = \kappa_0 t^\alpha$, $\ell_n (t) =e^{- bt^{-1\slash n}}$ with $0 < b < 2 n$. By Theorem 2, $\mu$ is $\Gamma$-diffuse with $\Gamma (t) = A_0 t  \kappa_\varphi \circ \theta_n \left(\ell_n (t)\right)$ and $\theta_n$ is the inverse  of the function 
$t \longmapsto t^{2n} \kappa_\varphi (t)^{2 - n} = t^{2n + \alpha (1-n)}$.  Thus 
$$
\Gamma (t) = A_0  t  e^{-b_1 t^{-1\slash n}},
$$
where $b_1:= \alpha b \slash [2 n + \alpha (1-n)]$.
 
 Then for $\tau > 0$, we have
 $$
 J_n (\tau) = A_0^{1\slash n}  \int_0^\tau  e^{-b_2 t^{-1\slash n}} \frac{dt}{t}
 $$
 where $b_2 := b_1\slash n$. 
 
  By the change of variable $s = t^{-1\slash n}$  we get
 $$
  J_n (\tau) = n A_0^{1\slash n}  \int_{\tau^{-1\slash n}}^{+ \infty}  e^{-b_2 s} \frac{ds}{s}.
 $$
  Fix $\tau_0 > 0$. Then for $0< \tau < \tau_0$ we have  
  
  $$
  J_n (\tau) \leq A_1  \int_{\tau^{-1\slash n}}^{+ \infty}  e^{-b_2 s} {ds} = A_2  e^{-b_2 \tau^{-1\slash n}},
  $$
  where $A_1 := n A_0^{1\slash n} \tau_0^{1\slash n}$ and $A_2 := A_1 \slash b_2$.
  
 Therefore given $\varepsilon > 0$, there exists $y_0 > 0$ and a constant $A_3 > 0$ such that  for $0 < y < y_0$ we have
  $$
h^{-1}_n (y) :=  y^{2n}  J_n^{-1} (y) \geq \frac{b_2 y^{2 n}}{\left(- \log (y\slash A_2)\right)^n}.
  $$
An easy computation shows that there exists $x_0 > 0$ small enough such that for $0< x < x_0$,  $h_n (x)  \leq A_3 x^{1\slash 2 n} \, (- \log x)^{1 \slash 2}$

By Theorem 4, the $\widehat{\kappa}$-modulus of continuity of the solution $U$ to the Dirichlet problem (\ref{eq:DirPb}) in this case satisfies $\widehat{\kappa}_U (\delta) \leq \widehat{\kappa} (\delta)$, where
$$
\widehat{\kappa} (\delta) := A_4 \delta^{\alpha \slash n  \tilde n} (- \log \delta)^{1 \slash 2}.
$$

Now we need to apply Lemma \ref{lem:sup-mean} to concude. Indeed it's clear that the modulus of continuity $\widehat{\kappa}$ obtained above satisfies the condition  (\ref{eq:fullkappa}). Moreover the function $U$ is  $\widehat{\kappa}$-continuous near the boundary by Corollary \ref{cor:approximation}. \end{proof}


  %Observe that the exponents obtained here are more precise than the ones obtained previously in \cite{N18a} and \cite{BZ20}. 
 
  
\section{Applications}


Let $\Omega \Subset \C^n$ be a  bounded strongly $m$-pseudoconvex  domain,  $\mu $ a positive Borel measure on $\Omega$  and $m$ be an integer such that  $1 \leq  m \leq n$. Here we will prove that all previous results still hold without assuming that the subsolution has boundary values $0$, but with a lost on the control of the modulus of continuity of the solution.


\begin{theorem} \label{thm:general} Let $\Omega \Subset \C^n$ be a  bounded strongly $m$-pseudoconvex  domain  with $1 \leq  m \leq n$ and $\mu $ a positive Borel measure on $\Omega$. Assume that there exists $\psi\in \mathcal{SH}_m(\Omega)\cap\mathcal{C}^{0}(\overline\Omega)$ such that  
\begin{equation} \label{eq:subsol2}
 \mu \leq (dd^c\psi)^m\wedge\beta^{n-m}, \, \, \, \mathrm{weakly \, \, on} \, \,  \Omega.
\end{equation} 

Assume that the modulus of continuity $\kappa_\psi$ of $\psi$ satisfies the following Dini type condition:
 \begin{equation} \label{eq:DC0}
 \int_{0^+} \frac{\left[\kappa_\psi (t)\right]^{1 \slash m}}{t \vert \log t\vert^{\epsilon_m} } d t< + \infty, 
 \end{equation}
 where $\epsilon_n = 1$ and $\epsilon_m = 0$ if $1 \leq m < n$.
 Then for any continuous function $g \in \mathcal{C}^{0} (\partial \Omega)$, there exists a unique function $U = U_{g,\mu} \in \mathcal{SH}_m (\Omega) \cap \mathcal{C}^{0} (\bar{\Omega})$ such that  
 $$
 (dd^c U)^m\wedge\beta^{n-m} = \mu, \, \, \, \mathrm{and} \, \, \, U = g \, \, \, \mathrm{on} \, \, \,  \partial \Omega.
 $$
 
 Moreover if $\mu (\Omega) < + \infty$,  the $\widehat{\kappa}$-modulus of continuity of $U$ satisfies the following estimate: 
$$
\widehat{\kappa}_U (\delta) \leq C  \, \widehat{\kappa}_m (\delta),
$$ 
 where  $\widehat{\kappa}_m (\delta)$ is given by the following formula.
 \begin{equation} \label {eq:generalsolution}
\widehat{\kappa}_m (\delta) :=   h_m \left[C_m \tilde{\kappa}_\psi  \circ\theta_m  \left(d M \delta^{2} \right)\right]  +  \tilde{\kappa}_\psi (\delta) + \kappa_g (\sqrt{\delta}) + \delta,
 \end{equation}
where $\tilde{\kappa}_\psi (\delta) =  \kappa_\psi (\sqrt{\delta})$ and $h_m$ is defined by (\ref{eq:hm}).
\end{theorem}
\begin{proof} We want to apply Theorem 3 and Theorem 4. 
We claim that there exists  a function $\varphi \in \mathcal{SH}_m(\Omega)\cap\mathcal{C}^{0}(\overline\Omega)$ such that  $\varphi = 0$ on $\partial \Omega$ with $\kappa_{\varphi} (\delta) \leq \kappa_\psi (\sqrt{\delta})$ and $\mu  \leq (dd^c \tilde \varphi)^m\wedge\beta^{n-m}, $ weakly on $\Omega$.

Indeed, consider the maximal $m$-subharmonic  function  on $\Omega$ with boundary values $ - \psi$  i.e.
$$
\tilde{\varphi} (z) := \sup \{v (z) \, ; \,  v \in \mathcal{SH}_m(\Omega), v \leq - \psi, \, \, \text{on} \, \, \partial \Omega \}.
$$ 
We know by \cite{Ch16b} that $\tilde{\varphi} \in \mathcal{SH}_m(\Omega)\cap\mathcal{C}^{0}(\overline\Omega)$, $\tilde{\varphi}= - \psi$ on $\partial \Omega$ and $\kappa_{\tilde{\varphi}} (\delta) \leq   \kappa_\psi (\sqrt{\delta})$.

It follows that the function $\varphi = \tilde \varphi + \psi \in \mathcal{SH}_m(\Omega)\cap\mathcal{C}^{0}(\overline\Omega)$ and  satisfies the inequality
 $$ (dd^c \varphi)^m\wedge\beta^{n-m} \geq  (dd^c \tilde{\varphi})^m\wedge\beta^{n-m} +  (dd^c\psi)^m\wedge\beta^{n-m} \geq \mu,
 $$
 weakly on $\Omega$. 
 Moreover $\varphi = 0$ on $\partial \Omega$ and $\kappa_{\varphi}  (\delta) \leq \kappa_\psi (\sqrt{\delta})$.
 It is clear that the modulus of continuity of $\varphi$ satisfies the Dini condition (\ref{eq:DC0}) so that we can apply Theorem 3 in the case $m < n$ and Theorem 4 in the case $m=n$,  which proves the theorem. 
\end{proof}

Let us give an example of application of this result.
\begin{example} Let $1 \leq m \leq n$. Let us embed $\R^m \hookrightarrow \C^m $ as the real part of $\C^m$ and consider the following Borel measure $\mu := \lambda_m \otimes \lambda_{2 n - 2 m} \sim H_{2n -m}$ on the unit ball $\B \subset  \C^n  \simeq  \C^m \times \C^{n - m}$. This is a singular positive measure on $\B$

We consider the following function : 
$$
\psi  (z) := \sum_{1 \leq j\leq m} (\Re z_j)_+  +  \sum_{m < j \leq n} \vert z_j\vert^2 - 2.
$$
 It is easy to check that $\psi$ is  plurisubharmonic and Lipschitz on any bounded  domain in $\C^n$ which satisfies the following inequality
 
 \begin{equation} \label{eq:generalsubsol}
 \mu \leq A (dd^c \psi)^m \wedge \beta^{n - m}
 \end{equation}
in the sense of currents on $\C^n$,  where $A > 0$ is a positive constant.
 
 Moreover $\int_\B (dd^c \psi)^m \wedge \beta^{n - m} < + \infty$. 
 Therefore by Theorem  \ref{thm:general},  there exists $u \in \mathcal{SH}_m (\B) \cap C^{0} (\bar \B)$ such that $(dd^c u)^m \wedge \beta^{n - m} = \mu$ on $\B$ and $u = 0$ on $\partial \B$. Moreover $ u \in {C}^{\alpha'} (\bar \B)$ for any  where $\alpha' < \widehat{\alpha}_m$, where $\widetilde \alpha_m$  is given by the formula
  $$
  \widehat{\alpha}_m := \frac{ \widetilde{r}}{ 2 \tilde{m} \left[ m \tilde{m} + \widetilde{r} \right]}, \, \, \mathrm{with} \, \, \tilde{m} := 2 m + (1-m)\slash 2.
  $$

  When $m =n$ we can improve this exponent since by Corollary \ref{cor:HolderMA}, the modulus of continuity satisfies  $\kappa_u (\delta) \leq Const \,  \delta^{1 \slash 2 n \tilde{n}} (- \log \delta)^{1 \slash 2}$.
  
    Observe that the unit ball $\B$ can be replaced by any bounded strongly $m$-pseudoconvex domain $\Omega \Subset \C^n$.
  
\end{example}

 
 
\smallskip

\smallskip

 {\bf Acknowledgements:} This project grew up  when the first author was staying at the Institute of Mathematics of Toulouse (IMT) during the summer $2016$ and  a first draft with partial results was available in December 2017. This project has been completed after the recent works of  
 S\l awomir Ko\l odziej and Ngoc Cuong Nguyen \cite{KN18,KN19}  which were a source of fruitful inspiration.  We are grateful to them. 
This work is a natural continuation of the work done by  Amel Benali and the second author on the same subject. Many results and ideas from the latter  has been used here.  
We are indebted to Chinh Hoang Lu for a careful cheking of the first version of this paper and for useful comments. We also thank Eleonora Di Nezza and Vincent Guedj  for useful discussions and comments.                                                                                                                             


 \begin{thebibliography}{X-XX}
 
 \bibitem[ACKPZ09]{ACKPZ09}  P. {\AA}hag, U. Cegrell, S. Ko\l odziej, H.H. Pham, A. Zeriahi: {\it Partial Energy and Integrability Exponents}. Adv. Math. 222 (2009), no. 6, 2036-2058. 
 
 \bibitem[AV]{AV} S. Alekser, and M. Verbitsky : {\it Quaternionic Monge-Amp\`ere equations and Calabi problem
for HKT-manifolds}. Israel J. Math. 176 (2010), 109-138.

 \bibitem[BT76]{BT76} E. Bedford, B.A.  Taylor : {\it  The Dirichlet problem for a complex Monge-Amp\`ere equation.} Invent. Math. 37 (1976), no. 1, 1-44.
 
 \bibitem[BT82]{BT82} E. Bedford, B.A.  Taylor : {\it  A new capacity for plurisubharmonic functions.} Acta Math. 149 (1982), no. 1-2, 1-40. 
 
 \bibitem[BZ20]{BZ20} A. Benali, A. Zeriahi : {\it  The H\"older continuous subsolution theorem.} Preprint,  arXiv:2004.06952 (to appear in Journal de l'\'Ecole Plolytechnique, Math\'ematiques). 
 
  \bibitem[BGZ08]{BGZ08} S. Benelkourchi, V. Guedj,  A. Zeriahi : {\it  A priori estimates for the complex Monge-Amp\`ere equation.} Ann.  Scul. Norm. Super. Pisa Cl. Sci. (5) 7 (2008), no. 1, 81–96. 
  
%\bibitem[Ber19]{Ber19} R. Berman : {\it  From Monge-Amp\`ere equations to envelopes and geodesic rays in the zero temperature limit.} Math. Z. 291 (2019), no. 1-2, 365-394.
  
% \bibitem[BD12]{BD12} R. Berman, J.-P. Demailly : {\it  Regularity of plurisubharmonic upper envelopes in big cohomology classes.} Perspectives in analysis, geometry, and topology, 39-66, Progr. Math., 296, Birkh\"auser/Springer, New York, 2012.
 
 \bibitem[Bl93]{Bl93} {Z. B\l ocki} : {\it Estimates for the complex Monge-Amp\`ere operator}. Bull. Polish Acad. Sc. Math., 43, no 2 (1993), 151-157.
 
\bibitem[Bl05]{Bl05} {Z. B\l ocki} : {\it  Weak solutions to the complex Hessian equation.}  Ann. Inst. Fourier (Grenoble) 55(5), (2005) 1735-1756.

%\bibitem[Brem59]{Brem59} {H. J. Bremermann} : {\it On a generalized Dirichlet problem for plurisubharmonic functions and pseudo-convex domains. Characterization of Šilov boundaries.} Trans. Amer. Math. Soc. 91 (1959), 246-276.

\bibitem[Ceg98]{Ceg98} {U. Cegrell} : {\it Pluricomplex energy.} Acta Math. 180 (1998), no. 2, 187–217. 

\bibitem[Ceg04]{Ceg04} {U. Cegrell} : {\it  The general definition of the complex Monge-Amp\`ere operator.} Ann. Inst. Fourier (Grenoble) 54, no. 1, (2004) 159-179.
 
 
 \bibitem[Ch16a]{Ch16a}{M. Charabati} : {\it Le probl\`eme de Dirichlet pour l'\'equation de Monge-Amp\`ere complexe.} Th\`ese de Doctorat de l'Universit\'e de Toulouse (UT3 Paul-Sabatier), 2016.


\bibitem[Ch16b]{Ch16b} {M. Charabati} :  {\it  Modulus of continuity of solutions to complex Hessian equations.} Internat. J. Math. 27 (2016), no. 1, 1650003, 24 pp.

 %\bibitem[CZ03]{CZ03} U. Cegrell,  A. Zeriahi: {\it Subextension of plurisubharmonic functions with bounded Monge-Amp\`ere mass.} C. R. Math. Acad. Sci. Paris 336 (2003), no. 4, 305–308. 
 
%\bibitem[CKZ05]{CKZ05} U. Cegrell, S. Kolodziej, A. Zeriahi: {\it  Subextension of plurisubharmonic functions with weak singularities}. Math. Z. 250 (2005), no. 1, 7-22. 


\bibitem[DNS10]{DNS10} T.C. Dinh, Viet A. Nguyen, N. Sibony: {\it Exponential estimates for plurisubharmonic functions.} J. Differential Geometry 84 (2010) 465-488.

\bibitem[DDGKPZ15]{DDGKPZ15} J.-P. Demailly, S. Dinew, V. Guedj, S. Kolodziej, H.H. Pham, A. Zeriahi: {\it H\"older Continuous Solutions to Monge-Amp\`ere Equations} J. Eur. Math. Soc. (JEMS) 16 (2014), no. 4, 619-647. 

\bibitem[DGZ16]{DGZ16} {S. Dinew, V. Guedj, A. Zeriahi} : {\it Open problems in pluripotential theory. }  Complex Var. Elliptic Equ. 61 (2016), no. 7, 902-930. 


\bibitem[DK14]{DK14} S. Dinew, S. Ko\l dziej :  {\it  A priori estimates for the complex Hessian equation.}  Anal. PDE 7 (2014), no. 1, 227-244. 

\bibitem[EGZ09]{EGZ09} {P. Eyssidieux, V. Guedj, A. Zeriahi} : {\it Singular K\"ahler- Einstein metrics.}  J. Amer. Math. Soc.  22 (2009) 607-639.


\bibitem[GKZ08]{GKZ08} V.  Guedj, S. Ko\l dziej, A. Zeriahi : {\it  A. H\"older continuous solutions to Monge-Amp\`ere equations.} Bull. Lond. Math. Soc. 40 (2008), no. 6, 1070-1080. 
 
 
 %\bibitem[GLZ19]{GLZ19} V.  Guedj, C.H. Lu, A. Zeriahi : {\it Plurisubharmonic envelopes and supersolutions}. J. Differential Geom. 113 (2019), no. 2, 273-313.
 
 
\bibitem[Kol95]{Kol95}  S.  Ko\l dziej, {\it The range of the complex Monge-Amp\`ere operator. II.}  Indiana Univ. Math. J., 44 (1995), no. 3, 765-782.


 \bibitem[Kol96]{Kol96} S.  Ko\l dziej, {\it   Some sufficient conditions for solvability of the Dirichlet problem for the
complex Monge-Amp\`ere operator.}  Ann. Polon. Math. 65 (1996), 11-21.

\bibitem[Kol05]{Kol05} S.  Ko\l dziej, {\it The complex Monge-Amp\`ere equation and Pluripotential Theory.} Mem. Amer. Math. Soc. Vol 178, n° 840 (2005), 64 pp.
 
 
 \bibitem[KN20a]{KN18} S.  Ko\l odziej,  N.C. Nguyen : {\it   A remark on the continuous subsolution problem for the complex Monge-Amp\`ere equation}.   Acta Math. Vietnam. 45 (2020), no. 1, 83–91.

 \bibitem[KN20b]{KN19} S.  Ko\l odziej,  N.C. Nguyen : {\it An inequality between complex hessian measures of H\"older continuous $m$-subharmonic functions and capacity}. Geometric Analysis, p. 157-166, Part of the Progress in Mathematics book series (PM, volume 333)(2020).


\bibitem[Lu12]{Lu12}  C.H. Lu : {\it \'Equations Hessiennes Complexes}. Th\`ese de Doctorat de l'Universit\'e de Toulouse (UT3 Paul-Sabatier), 2012.

 
%\bibitem[Lu13]{Lu13}  C.H. Lu : {\it Viscosity solutions to complex Hessian equations.} J. Funct. Anal. 264 (2013), no. 6, 1355-1379. 
    
\bibitem[Lu15]{Lu15}  C.H. Lu : {\it A variational approach to complex Hessian equations in $\C^n$}. J. Math. Anal. Appl. 431 (2015), no. 1, 228-259.
  
  
\bibitem[LPT20]{LPT20}  C.H. Lu, T.T. Phung, T.D. T\^o  : {\it Stability and H\"loder continuity of solutions to complex Monge-Amp\`ere equations on compact hermitian manifolds}. Preprint (2020).  arXiv:2003.08417. 

\bibitem[N13]{N13} {N.C. Nguyen} : {\it Subsolution Theorem for the Complex Hessian Equation.} Univ. Iagiell. Acta Math. 50, (2013), 69-88.

\bibitem[N14]{N14} N.C. Nguyen : {\it  H\"older continuous solutions to complex Hessian equations.} Potential Anal. 41, no. 3 (2014), 887-902.

\bibitem[N18]{N18a} {N.C. Nguyen} : {\it On the H\"older continuous subsolution problem for the complex Monge-Amp\`ere equation.}  Calc. Var. Partial Differential Equations 57 (2018), no. 1, Art. 8, 15 pp.

\bibitem[N20]{N18b} {N.C. Nguyen} : {\it On the H\"older continuous subsolution problem for the complex Monge-Amp\`ere equation II.}  Analysis and  PDE, Vol. 13, No 2 (2020), 435-453.

\bibitem[Pl14]{Pl14} {S. Plis} : {\it  The smoothing of $m-$subharmonic functions.} arxiv: 1312.1906v2 (2014).

\bibitem[Po16]{Po16} {S. Pons} : {\it Elliptic PDEs, Measures
and Capacities. From the Poisson Equation to Nonlinear Thomas–Fermi Problems}. EMS Tracts in Mathematics, 23 (2016).

%\bibitem[Sic81]{Sic81} J. Siciak  : {\it  Extremal plurisubharmonic functions in $\C^n$.} Ann. Pol. Math. 39 (1981), 175-211.


\bibitem[SA13]{SA13} A. Sadullaev,  B.  Abdullaev : {\it  Capacities and Hessians in the class of m-subharmonic functions.} (Russian) Dokl. Akad. Nauk 448 (2013), no. 5, 515-517; translation in Dokl. Math. 87 (2013), no. 1, 88-90.

\bibitem[SW]{SW} J. Song and B. Weinkove :  On the convergence and singularities of the J-flow with applications to the Mabuchi energy, Comm. Pure. Appl. Math. 61 (2008), 210-229.

%\bibitem[Wal69]{Wal69} J.B. Walsh : {\it Envelopes of plurisubharmonic functions}. J. Math. Mech. 18 (1969), 143-148.

\bibitem[Ze20]{Ze20}  A. Zeriahi  : {\it Remarks on the modulus of continuity of subharmonic functions}. Preprint (2020),  http://www.math.univ-toulouse.fr/~zeriahi/

 \end{thebibliography}



\end{document}




%%%%%%%%%%%%%%%%%%%%%%%%%%%%%%%%%%%%%%%%%%%%%%%%%%%%%%%%%%%%%%%%%%%%%%%%%%%%%%%%%%%%%%%%%%%%%%%%%%%%%%%%%%%%%%%%%%%%%%%%%%%%%%%%%%%%%%%%%%%%%%

  