\section{Introduction}
%Dinatural transformations are the generalisation of one of the very pillars of Category Theory: \emph{natural transformations}. They first appeared in the context of Algebraic Topology in a paper by Yoneda~\cite{yoneda_ext_1960}, where he introduced the notion of \emph{(co)integration} (in more modern terminology, \emph{(co)end}) of a functor of mixed variance, which is a particular kind of dinatural transformation. They were later formally defined by Dubuc and Street~\cite{dubuc_dinatural_1970} as a common setting for natural and \emph{extranatural} transformations of Eilenberg and Kelly~\cite{eilenberg_generalization_1966}: as such, they provide the correct notion of a family of morphisms satisfying a naturality condition between functors of mixed variance.
%
%Numerous applications of dinatural transformations have been found in Mathematics and Computer Science: in Enriched Category Theory with the calculus of ends and coends.

The problem of coherence for a certain theory (like monoidal, monoidal closed\dots) consists in understanding which diagrams necessarily commute as a consequence of the axioms. One of the most famous results is Mac Lane's theorem on coherence for monoidal categories~\cite{mac_lane_natural_1963}: every diagram built up only using associators and unitors, which are the data that come with the definition of monoidal category, commutes. One of the consequences of this fact is that every monoidal category is monoidally equivalent to a \emph{strict} monoidal category, where associators and unitors are, in fact, identities. What this tells us is that those operations that one would like to regard as not important--such as the associators and unitors etc--really are not important. Solving the coherence problem for a theory, therefore, is fundamental to the complete understanding of the theory itself.

In this article we aim to set down the foundations for the answer to an open question left by Kelly in his task to study the coherence problem abstractly, started with~\cite{kelly_abstract_1972,kelly_many-variable_1972}. 
%In there, 
Kelly argued that coherence problems are concerned with categories carrying an extra structure: a collection of functors and natural transformations subject to various equational axioms. For example, in a monoidal category $\A$ we have $\otimes \colon \A^2\! \to \A$, $I \colon \A^0 \to \A$; if $\A$ is also closed then we would have a functor of mixed variance $(-) \implies (-) \colon \Op\A\! \times \A \to \A$. The natural transformations that are part of the data, like associativity in the monoidal case:
\[
\alpha_{A,B,C} \colon (A \otimes B) \otimes C \to A \otimes (B \otimes C),
\]
connect not the basic functors directly, but rather functors obtained from them by \emph{iterated substitution}. By ``substitution'' we mean the process where, given functors
\[
K \colon \A \times \Op\B \times \C \to \D, \quad F \colon \E \times \mathbb G \to \A, \quad G \colon \mathbb H \times \Op\L \to \B, \quad H \colon \Op\M \to \C
\]
we obtain the new functor
\[
K(F,\Op G, H) \colon \E \times \mathbb G \times \Op{\mathbb H} \times \L \times \Op\M \to \D\label{substitution functors example}
\]
sending $(A,B,C,D,E)$ to $K(F(A,B),\Op G (C,D), H(E))$. Hence substitution generalises composition of functors, to which it reduces if we only consider one-variable functors. In the same way, the equational axioms for the structure, like the pentagonal axiom for monoidal categories:
\[
\begin{tikzcd}[column sep={3.5em,between origins},row sep=2em]
& & & (A \otimes B) \otimes (C \otimes D) \ar[drr,"\alpha_{A,B,C\otimes D}"] \\
\bigl( (A \otimes B) \otimes C \bigr) \otimes D \ar[urrr,"\alpha_{A\otimes B, C, D}"] \ar[dr,"\alpha_{A,B,C} \otimes D"'] & & & & & A \otimes \bigl( B \otimes (C \otimes D) \bigr) \\
& \bigl( A \otimes (B \otimes C) \bigr) \otimes D \ar[rrr,"\alpha_{A,B \otimes C, D}"'] & & & A \otimes \bigl( (B \otimes C) \otimes D \bigr) \ar[ur,"A \otimes \alpha_{B,C,D}"']
\end{tikzcd}
\]
involve natural transformations obtained from the basic ones by ``substituting functors into them and them into functors'', like $\alpha_{A \otimes B, C, D}$ and $\alpha_{A,B,C} \otimes D$ above. 

By substitution of functors into transformations and transformations into functors we mean therefore a generalised \emph{whiskering} operation or, more broadly, a generalised \emph{horizontal composition} of transformations. For these reasons Kelly argued in~\cite{kelly_many-variable_1972} that an abstract theory of coherence requires ``a tidy calculus of substitution'' for functors of many variables and appropriately general kinds of natural transformations, generalising the usual Godement calculus~\cite[Appendice]{godement_topologie_1958} for ordinary functors in one variable and ordinary natural transformations. (The ``five rules of the functorial calculus'' set down by Godement are in fact equivalent to saying that sequential composition of functors and vertical and horizontal composition of natural transformations are associative, unitary and satisfy the usual interchange law; see~\cite[Introduction]{santamaria_towards_2019} for more details.)

%One could ask why bother introducing the notion of substitution, given that it is not primitive, as the functor  $K(F,\Op G, H)$ above can be easily seen to be the usual composite $K \circ (F \times \Op G \times H)$. Kelly's argument is that there is \emph{no need} to consider functors whose codomain is a product of categories, like $F \times \Op G \times H$, or the twisting functor $T(A,B) = (B,A)$, or the diagonal functor $\Delta \colon \A \to \A \times \A$ given by $\Delta(A) = (A,A)$, if we consider substitution as an operation on its own. In fact, in some cases these functors are just not enough: take a cartesian closed category $\A$, and consider the diagonal transformation $\delta_A \colon A \to A \times A$, the symmetry $\gamma_{A,B} \colon A \times B \to B \times A$ and the evaluation transformation $\eval A B \colon A \times (A \implies B) \to B$. It is true that we can see $\delta$ and $\gamma$ as transformations $\id \A \to \Delta$ and $\times \to \times \circ T$, but there is no way to involve $\Delta$ into the codomain of $\eval{}{}$, given that the variable $A$ appears covariantly and contravariantly at once. Kelly suggested, then, to use the same idea of \emph{graph} for \emph{extranatural} transformations that he had with Eilenberg in a previous paper~\cite{eilenberg_generalization_1966} to natural transformations as well; that is, he proposed to consider natural transformations $\phi \colon F \to G$ between functors of many variables together with a graph $\Gamma(\phi)$ that tells us which arguments of $F$ and $G$ are to be equated when we write down the general component of $\phi$.  

One could ask why bother introducing the notion of substitution, given that it is not primitive, as the functor  $K(F,\Op G, H)$ above can be easily seen to be the usual composite $K \circ (F \times \Op G \times H)$. Kelly's argument is that there is \emph{no need} to consider functors whose codomain is a product of categories, like $F \times \Op G \times H$, or the twisting functor $T(A,B) = (B,A)$, or the diagonal functor $\Delta \colon \A \to \A \times \A$ given by $\Delta(A) = (A,A)$, if we consider substitution as an operation on its own. However, take a Cartesian closed category $\A$, and consider the diagonal transformation $\delta_A \colon A \to A \times A$, the symmetry $\gamma_{A,B} \colon A \times B \to B \times A$ and the evaluation transformation $\eval A B \colon A \times (A \implies B) \to B$. It is true that we can see $\delta$ and $\gamma$ as transformations $\id \A \to \Delta$ and $\times \to \times \circ T$, but there is no way to involve $\Delta$ into the codomain of $\eval{}{}$, given that the variable $A$ appears covariantly and contravariantly at once. Kelly suggested adapting the notion of \emph{graph} for \emph{extranatural} transformations that he had introduced with Eilenberg~\cite{eilenberg_generalization_1966} to handle the case of natural transformations; that is, he proposed to consider natural transformations $\phi \colon F \to G$ between functors of many variables together with a graph $\Gamma(\phi)$ that tells us which arguments of $F$ and $G$ are to be equated when we write down the general component of $\phi$. The information carried by the graph is what allows us to get by without explicit mention of functors like $T$ and $\Delta$ and, moreover, it paves the way to the substitution calculus he sought.

With the notion of ``graph of a natural transformation'', Kelly constructed a full Godement calculus for covariant functors only. His starting point was the observation that the usual Godement calculus essentially asserts that $\Cat$ is a 2-category, but this is saying less than saying that $\Cat$ is actually \emph{Cartesian closed}, $- \times \B$ having a right adjoint $[\B,-]$ where $[\B,\C]$ is the functor category. Since every Cartesian closed category is enriched over itself, we have that $\Cat$ is a $\Cat$-category, which is just another way to say 2-category. Now, vertical composition of natural transformations is embodied in $[\B,\C]$, but sequential composition of functors and horizontal composition of natural transformations are embodied in the functor
\[
M \colon [\B,\C] \times [\A,\B] \to [\A,\C]
\]
given by the closed structure (using the adjunction and the evaluation map twice). What Kelly does, therefore, is to create a generalised functor category $\FC \B \C $ over a category of graphs $\Per$ and to show that the functor $\FC - -$ is the internal-hom of $\catover\Per$, which is then monoidal closed (in fact, far from being Cartesian or even symmetric), the left adjoint of $\FC \B -$ being denoted as $\ring - \B$. The analogue of the $M$ above, now of the form $\ring {\FC \B \C} {\FC \A \B} \to \FC \A \C,$ is what provides the desired substitution calculus.

When trying to deal with the mixed-variance case, however, Kelly ran into problems. He considered the every-variable-twice extranatural transformations of~\cite{eilenberg_generalization_1966} and, although he got ``tantalizingly close'', to use his words, to a sensible calculus, he could not find a way to define a category of graphs that can handle cycles in a proper way. This is the reason for the ``I'' in the title \emph{Many-Variable Functorial Calculus, I} of~\cite{kelly_many-variable_1972}: he hoped to solve these issues in a future paper, which sadly has never seen the light of day.

What we do in this article is, in fact, consider transformations between mixed-variance functors whose type is even more general than Eilenberg and Kelly's, corresponding to $\text{\uuline{G}}^*$ in~\cite{kelly_many-variable_1972}, recognising that they are a straightforward generalisation of \emph{dinatural transformations}~\cite{dubuc_dinatural_1970} in many variables. This poses an immediate, major obstacle: dinatural transformations notoriously fail to compose, as already observed by Dubuc and Street when they introduced them in 1970. There are certain conditions, known already to their discoverers, under which two dinatural transformations $\phi$ and $\psi$ compose: if either of them is natural, or if a certain square happens to be a pullback or a pushout, then the composite $\psi\circ\phi$ turns out to be dinatural. However, these are far from being satisfactory solutions for the compositionality problem, for either they are too restrictive (as in the first case), or they speak of properties enjoyed not by $\phi$ and $\psi$ themselves, but rather by other structures, namely one of the functors involved.  Many studies have been conducted about them~\cite{bainbridge_functorial_1990,blute_linear_1993,freyd_dinaturality_1992,girard_normal_1992,lataillade_dinatural_2009,mulry_categorical_1990,pare_dinatural_1998,pistone_dinaturality_2017,plotkin_logic_1993,simpson_characterisation_1993,wadler_theorems_1989}, and many attempts have been made to find a proper calculus for dinatural transformations, but until recently only \emph{ad hoc} solutions have been found and, ultimately, they have remained poorly understood.

In 2003, Petri\'c~\cite{petric_g-dinaturality_2003} studied coherence result for bicartesian closed categories, and found himself in need, much like Kelly in his more general case, of understanding the compositionality properties of \emph{g-dinatural transformations}, which are slightly more general dinatural transformations than those of Dubuc and Street~\cite{dubuc_dinatural_1970} in what their domain and codomain functors are allowed to have different variance and, moreover, they always come with a graph (whence the ``g'' in ``g-dinatural'') which reflects their signature. Petri\'c successfully managed to find a sufficient and essentially necessary condition for two consecutive g-dinatural transformations $\phi$ and $\psi$ to compose: if the composite graph, obtained by appropriately ``glueing'' together the graphs of $\phi$ and $\psi$, is acyclic, then $\psi\circ\phi$ is again g-dinatural. This result, which effectively solves the compositionality problem of dinatural transformations, 
%has, incredibly as it may sound, eluded the community of Category Theory: 
surprisingly does not appear to be well known: 
fifteen years after Petri\'c's paper, the authors of the present article, completely oblivious to Petri\'c's contribution, independently re-discovered the same theorem, which was one of the results of~\cite{mccusker_compositionality_2018} and of the second author's PhD thesis~\cite{santamaria_towards_2019}\footnotemark. We, too, associated to each dinatural transformation a graph, inspired by Kelly's work of~\cite{kelly_many-variable_1972}, such graph being slightly different from Petri\'c's; we also proved that acyclicity of the composite graph of $\phi$ and $\psi$ is ``essentially enough'' for $\psi\circ\phi$ to be dinatural. The proof of our and Petri\'c's theorem are, deep down, following the same argument, but the main difference is in the approach we took to formalise it: Petri\'c's went purely syntactic, using re-writing rules to show how the arbitrary morphism of the universal quantification of the dinaturality property for $\psi\circ\phi$ can ``travel through the composite graph'' when the graph is acyclic, whereas we showed this by interpreting the composite graph as a \emph{Petri Net}~\cite{petri_kommunikation_1962} and re-casting the dinaturality property of $\psi\circ\phi$ into a \emph{reachability} problem. We then proceeded to solve it by exploiting the general theory of Petri Nets: in other words, we took a more semantic approach.

\footnotetext{We also presented our result as novel in various occasions, including in a plenary talk at the Category Theory conference in Edinburgh in 2019, yet nobody redirected us to Petri\'c's paper, which we found by chance only in September 2019.}

Because of this appreciable difference of Petri\'c's and our proof of the compositionality result for dinatural transformations, we believe it is worth presenting in this paper our theorem despite the non-novelty of its statement; moreover, we give here a more direct proof for it than the one in~\cite{mccusker_compositionality_2018}: this is done in Section~\ref{section vertical compositionality}. In Section~\ref{chapter horizontal}, we define a working notion of horizontal composition, that we believe will play the role of substitution of dinaturals into dinaturals, precisely as horizontal composition of natural transformation does, as shown by Kelly in~\cite{kelly_many-variable_1972}. Next, we form a generalised functor category $\FC \B \C$ for these transformations (Definition~\ref{def: generalised functor category}). Finally, we prove that $\FC \B -$ has indeed a left adjoint $\ring - \B$, which gives us the definition of a category of formal substitutions $\ring \A \B$ generalising Kelly's one. Although the road paved by Kelly towards a substitution calculus for dinatural transformations still stretches a long way, our work sets the first steps in the right direction for a full understanding of the compositionality properties of dinaturals, which hopefully will be achieved soon.





\paragraph{Notations} 
$\N$ is the set of natural numbers, including 0, and we shall ambiguously write $n$ for both the natural number $n$ and the set $\{1,\dots,n\}$. We denote by $\I$ the category with one object and one morphism. Let $\alpha\in \List{\{+,-\}}$, $\length\alpha=n$, with $\length{-}$ denoting the length function (and also the cardinality of an ordinary finite set). We refer to the $i$-th element of $\alpha$ as $\alpha_i$. Given a category $\C$, if $n\ge 1$, then we define $\C^\alpha=\C^{\alpha_1} \times \dots \times \C^{\alpha_n}$, with $\C^+=\C$ and $\C^-=\Op\C$, otherwise $\C^\alpha=\I$. 

Composition of morphisms $f \colon A \to B$ and $g \colon B \to C$ will be denoted by $g\circ f$, $gf$ or also $f;g$. The identity morphism of an object $A$ will be denoted by $\id A$, $1_A$ (possibly without subscripts, if there is no risk of confusion), or $A$ itself. 
	Given $A$, $B$ and $C$ objects of a category $\C$ with coproducts, and given $f \colon A \to C$ and $g \colon B \to C$, we denote by $[f,g] \colon A + B \to C$ the unique map granted by the universal property of $+$.

We use boldface capital letters $\bfA,\bfB\dots$ for tuples of objects, whose length will be specified in context. Say $\bfA=(A_1,\dots,A_n) \in \C^n$: we can see $\bfA$ as a function from the set $n$ to the objects of $\C$. If $\sigma \colon k \to n$ is a function of sets, the composite $\bfA \sigma$ is the tuple $(A_{\sigma 1}, \dots, A_{\sigma k})$. For $\bfB \in \C^n$ and $i \in \{1,\dots,n\}$, we denote by $\subst B X i$ the tuple obtained from $\bfB$ by replacing its $i$-th entry with $X$, and by $\subst B \cdot i$ the tuple obtained from $\bfB$ by removing its $i$-th entry altogether. In particular, the tuple $\subst A X i \sigma$ is equal to $(Y_1,\dots,Y_k)$ where
\[
Y_j = \begin{cases}
X & \sigma j=i \\
A_{\sigma j} & \sigma j \ne i
\end{cases}.
\]
Let $\alpha \in \List\{+,-\}$, $\bfA = (A_1,\dots,A_n)$, $\sigma \colon \length\alpha \to n$, $i \in \{1,\dots,n\}$. We shall write $\substMV A X Y i \sigma$ for the tuple $(Z_1,\dots,Z_{\length\alpha})$ where
\[
Z_j = \begin{cases}
X & \sigma j = i, \alpha_j = - \\
Y & \sigma j = i, \alpha_j = + \\
A_{\sigma j} & \sigma j \ne i
\end{cases}\label{not:A[X,Y/i]sigma}
\]
We shall also write $\subst B {\bfA} i$ for the tuple obtained from $\bf B$ by substituting $\bfA$ into its $i$-th entry. For example, if $\bfA = (A_1,\dots,A_n)$ and $\bfB = (B_1,\dots,B_m)$, we have
\[
\subst B {\bfA} i = (B_1,\dots, B_{i-1},A_1,\dots A_n, B_{i+1}, \dots B_m).
\]

If $F \colon \B^{\alpha} \to \C$ is a functor, we define $\funminplus F {A_i} {B_i} i {\length\alpha}$ to be the following object (if $A_i$, $B_i$ are objects) or morphism (if they are morphisms) of $\C$:
\[
\funminplus F {A_i} {B_i} i {\length\alpha}= F(X_1,\dots,X_{\length\alpha}) \text{ where } X_i =
\begin{cases}
A_i & \alpha_i = - \\
B_i & \alpha_i = +
\end{cases}
\]
If $A_i = A$ and $B_i = B$ for all $i \in \length\alpha$, then we will simply write $\funminplusconst F A B$ for the above.

We denote by $\Not\alpha$ the list obtained from $\alpha$ by swapping the signs. Also, we call $\Op F \colon \B^{\Not\alpha} \to \Op\C$ the \emph{opposite functor}, which is the obvious functor that acts like $F$ between opposite categories.
