
%%%packages - alphabetically
	\usepackage{amsmath}	
	\usepackage{amssymb}
	\usepackage{amsthm}
	\usepackage[british]{babel}
	\usepackage{comment}
	\usepackage{datetime}
	\usepackage{enumitem}
	\usepackage[inner=30mm, outer=30mm, 
    %top=12mm, 
	bottom=28mm,
				a4paper]{geometry}
	%\usepackage[sc, slantedGreek]{mathpazo}
	\usepackage[framemethod=TikZ, roundcorner=1mm, linecolor=gray]{mdframed}
	%\usepackage{faktor}
	\usepackage{HanFakt} %quotients
	\usepackage{tikz-cd}
	\usepackage{url}
	\usepackage{stmaryrd}
	\usepackage{xcolor}
	\usepackage{mathrsfs}
	\usepackage{mathtools}
	\usepackage{todonotes}
	\usepackage{bm}

%Tikz
\usepackage{tikz-cd, tikz-3dplot, pgfplots}
\usetikzlibrary{patterns, arrows, positioning, decorations.markings}

	\tikzset{graph/.style={grau}}
\tikzcdset{
  cells={font=\everymath\expandafter{\the\everymath\displaystyle}},
}

% Kopfzeilen
\usepackage[automark, autooneside=false]{scrlayer-scrpage} 
\renewcommand*{\subsectionmarkformat}{}


\newpairofpagestyles[scrheadings]{default}{
\automark[subsection]{section}
\clearpairofpagestyles
\chead{\leftmark : \rightmark}
\cfoot*{\pagemark}
}


\newpairofpagestyles[scrheadings]{special}{
\automark[subsection]{section}
\clearpairofpagestyles
\chead{\leftmark}
\cfoot*{\pagemark}
}

% adjust some margins

\RedeclareSectionCommand[
  %runin=false,
  afterindent=false,
  beforeskip=\baselineskip,
  afterskip=\baselineskip]{section}
  
  \RedeclareSectionCommand[
  beforeskip=.5\baselineskip,
  afterskip=-1em]{paragraph}

%hide a section from ToC

\newcommand{\nocontentsline}[3]{}
\newcommand{\tocless}[2]{\bgroup\let\addcontentsline=\nocontentsline#1{#2}\egroup}

%layout
	\linespread{1.05} 

%%Definition grey box
	\renewmdenv[%
   	 backgroundcolor=gray!25,
   	 linecolor=white,
    	outerlinewidth=1pt,
    	roundcorner=3mm,
    	%skipabove=\baselineskip,
    	%skipbelow=\baselineskip,
	]{boxed}	

%% Definition der Sätze und Co.
	\swapnumbers
	\theoremstyle{definition}
	\newtheorem*{thmplain}{Theorem}
	\newtheorem*{propplain}{Proposition}
	\newtheorem{thm}{Theorem}[section]
	\newtheorem{conj}[thm]{Conjecture}
	\newtheorem{prop}[thm]{Proposition}
	\newtheorem{cor}[thm]{Corollary}
	\newtheorem{lem}[thm]{Lemma}
	\newtheorem{ex}[thm]{Exercise}
	\newtheorem{bsp1}[thm]{Example}
	\newtheorem{bspe1}[thm]{Examples}
	\newtheorem{definition}[thm]{Definition}
	\newtheorem{rk}[thm]{Remark}
	\newtheorem{lessons}[thm]{Lessons}
	\newtheorem{hyp}[thm]{Hypotheses}
	\newtheorem{caveat}[thm]{Caveat}
	\newtheorem{theorem}{thm}[section]
	\newtheorem{lemma}[thm]{Lemma}
	\newtheorem{corollary}[thm]{Corollary}
	\newtheorem{proposition}[thm]{Proposition}
	\newtheorem{conjecture}[thm]{Conjecture}
	\newtheorem{question}[thm]{Question}
	\newtheorem{example}[thm]{Example}
	\newtheorem{examples}[thm]{Examples}
	\newtheorem{remark}[thm]{Remark}
	\newtheorem{remarks}[thm]{Remark}
	\newtheorem*{acknowledgments}{Acknowledgements}
	\newtheorem{hypothesis}[thm]{Hypothesis}
	\newtheorem{hypotheses}[thm]{Hypotheses}

%%Definitio of respective environments
	\newenvironment{thmbox}
	 {\begin{boxed} \begin{thm}}
	 {\end{thm}\end{boxed}}
	  
	\newenvironment{defgray}
	 {\begin{boxed}\begin{definition}}
	 {\end{definition}\end{boxed}}
	 
	 \newenvironment{corgray}
	 {\begin{boxed}\begin{cor}}
	 {\end{cor}\end{boxed}}
	  
	\newenvironment{propgray}
	 {\begin{boxed}\begin{prop}}
	 {\end{prop}\end{boxed}}
	  
	\newenvironment{lemgray}
	 {\begin{boxed}\begin{lem}}
	 {\end{lem}\end{boxed}}
	  
	 \newenvironment{proofbox}
	  {%\begin{mdframed} 
	  \begin{proof}}
	 { \end{proof} 
	 %\end{mdframed} 
	 \medskip}

	\newenvironment{bsp}
	{\begin{bsp1}}
	{  \end{bsp1}}
	 
	\newenvironment{bspe}
	{\begin{bspe1}
	\begin{liste}}
	{\end{liste}
	\end{bspe1}}

	\newenvironment{liste}
	{\begin{enumerate}[label=(\alph*)]}
	{\end{enumerate}}

	\newenvironment{cdiagram}
	{\begin{center} \begin{tikzcd}}
	{\end{tikzcd} \end{center}}


%%Crazy Shortcuts - alphabetically
	\renewcommand{\:}{\colon}
	\renewcommand{\a}{\mathfrak{a}}
	%\newcommand{\Ann}{\operatorname{Ann}}
	%\newcommand{\Aut}{\operatorname{Aut}}
	\newcommand{\bigO}{\mathcal{O}}
	\newcommand{\characteristic}{\operatorname{char}}
	\newcommand{\Char}{\operatorname{char}}
	\newcommand{\cl}{\operatorname{cl}}
	\newcommand{\C}{\mathbb{C}}
	%\newcommand{\coker}{\operatorname{coker}}
	\newcommand{\dd}{\mathrm{d}}
	\newcommand{\Eig}{\operatorname{Eig}}
	%\newcommand{\Ext}{\operatorname{Ext}}
	\renewcommand{\emptyset}{\varnothing}
	\renewcommand{\emph}[1]{\textit{\textbf{#1}}}
	\newcommand{\Fitt}{\operatorname{Fitt}}
	\newcommand{\Frob}{\mathrm{Frob}}
	\newcommand{\gal}[2]{\text{Gal}( #1 | #2)}
	%\newcommand{\Hom}{\operatorname{Hom}}
	\renewcommand{\iff}{\quad \Leftrightarrow \quad}
	\newcommand{\id}{\mathrm{id}}
	\newcommand{\legendre}[2]{ \left( \frac{#1}{#2} \right)}
	 \newcommand{\Log}{\text{Log}} 
	 \newcommand{\N}{\mathbb{N}}
	\newcommand{\nZ}[1]{\faktor{\mathbb{Z}}{#1 \mathbb{Z}}}
	\newcommand{\p}{\mathfrak{p}}
	\renewcommand{\P}{\mathfrak{P}}
	\newcommand{\m}{\mathfrak{m}}
	\newcommand{\q}{\mathfrak{q}}
	\newcommand{\Q}{\mathbb{Q}}
	\newcommand{\Ord}{\mathrm{Ord}}
	\newcommand{\ord}{\mathrm{ord}}
	\newcommand{\R}{\mathbb{R}}
	\renewcommand{\Re}{\operatorname{Re}}
	\newcommand{\Res}[2]{\operatorname{Res}\left (#1; \, #2 \right )}	
	\newcommand{\Rec}{\mathrm{Rec}}
	\newcommand{\so}{\quad \Rightarrow \quad}
	\newcommand{\sgn}{\operatorname{sgn}}
	\renewcommand{\ord}{\operatorname{ord}}
	\newcommand{\Stab}{\mathrm{Stab}}	
	%\newcommand{\Spec}{\operatorname{Spec}}
	\newcommand{\svector}[3]{\begin{pmatrix} #1 \\ #2 \\ #3 \end{pmatrix}}
	\newcommand{\Tr}{\operatorname{Tr}}
	\newcommand{\UN}{\mathrm{UN}}
	\newcommand{\Z}{\mathbb{Z}}
	%\newcommand{\Det}{\text{Det}}
	
	\renewcommand{\thethm}{(\arabic{section}.\arabic{thm})}
	
	
	%%%verlinktes toc für Domi
\AfterPreamble{\hypersetup{colorlinks,
pdfpagelabels,
pdfstartview = FitH,
bookmarksopen = true,
bookmarksnumbered = true,
linkcolor = black,
plainpages = false,
hypertexnames = false,
citecolor = black, urlcolor=black}}


%%%Alex's preamble


%bunch of environment definitions

%\font\russ=wncyr10 \font\Russ=wncyr10   scaled\magstep 1
%\def\sha{\hbox{\russ\char88}}

%%More crazy shortcuts
\newcommand{\vir}[1]{{\bf [} #1{\bf ]}}
\newcommand{\eins}{\boldsymbol{1}}
\DeclareMathOperator{\Aut}{Aut}
\DeclareMathOperator{\Cl}{Cl}
\DeclareMathOperator{\Det}{Det}
\DeclareMathOperator{\Ext}{Ext}
\DeclareMathOperator{\Tor}{Tor}
\DeclareMathOperator{\Gal}{Gal}
\DeclareMathOperator{\Ho}{H}
\DeclareMathOperator{\Hom}{Hom}
\DeclareMathOperator{\End}{End}
\DeclareMathOperator{\Pic}{Pic}
\DeclareMathOperator{\Spec}{Spec}
%\DeclareMathOperator{\N}{N}
%\newcommand{\fit}{\mathrm{Fit}_{\mathcal{O}}}
\DeclareMathOperator{\ind}{ind}
\DeclareMathOperator{\coker}{coker}
\DeclareMathOperator{\res}{res}
\DeclareMathOperator{\im}{im}

\newcommand{\CC}{\mathbb{C}}
\newcommand{\bc}{\mathbb{C}}
\newcommand{\FF}{\mathbb{F}}
\newcommand{\GG}{\mathbb{G}}
\newcommand{\NN}{\mathbb{N}}
\newcommand{\QQ}{\mathbb{Q}}
%\newcommand{\Q}{\mathbb{Q}}
\newcommand{\RR}{\mathbb{R}}
\newcommand{\br}{\mathbb{R}}
\newcommand{\ZZ}{\mathbb{Z}}
%\newcommand{\Z}{\mathbb{Z}}
\newcommand{\bfp}{\mathbb{F}}
\newcommand{\ZG}{\mathbb{Z}[G]}

\newcommand{\calD}{\mathcal{D}}
\newcommand{\calE}{\mathcal{E}}
\newcommand{\calF}{\mathcal{F}}
\newcommand{\calL}{\mathcal{L}}
\newcommand{\calM}{\mathcal{M}}
\newcommand{\cQ}{\mathcal{Q}}
%\newcommand{\C}{\mathbb{C}}
%\newcommand{\R}{\mathbb{R}}
\newcommand{\calT}{\mathcal{T}}
\newcommand{\G}{\mathcal{G}}
\newcommand{\cG}{\mathcal{G}}
\newcommand{\cK}{\mathcal{K}}
\newcommand{\M}{\mathcal{M}}
\newcommand{\calN}{\mathcal{N}}
\newcommand{\cO}{\mathcal{O}}
\newcommand{\cF}{\mathcal{F}}
\newcommand{\calV}{\mathcal{V}}

\newcommand{\frakm}{\mathfrak{m}}
\newcommand{\frp}{\mathfrak{p}}
\newcommand{\frakA}{\mathfrak{A}}
\newcommand{\fq}{\mathfrak{q}}
\newcommand{\fp}{\mathfrak{p}}

\newcommand{\arch}{\mathrm{arch}}
\DeclareMathOperator{\aug}{aug}
\DeclareMathOperator{\Br}{Br}
\DeclareMathOperator{\centre}{Z}
\newcommand{\cok}{\text{cok}}
\newcommand{\cone}{\mathrm{cone}}
\newcommand{\et}{\text{\'et}}
\newcommand{\even}{\mathrm{ev}}
\newcommand{\finite}{\mathrm{finite}}
\newcommand{\glob}{\mathrm{glob}}
%\newcommand{\id}{\mathrm{id}}
\newcommand{\Imag}{\mathbf{I}}
\DeclareMathOperator{\inv}{inv}
\newcommand{\Irr}{\mathrm{Irr}}
\newcommand{\loc}{\mathrm{loc}}
\newcommand{\nr}{\mathrm{Nrd}}
\newcommand{\odd}{\mathrm{od}}
\newcommand{\old}{\mathrm{old}}
\newcommand{\perf}{\mathrm{p}}
\newcommand{\quot}{\mathrm{q}}
\newcommand{\real}{\mathrm{Re}}
\newcommand{\Real}{\mathbf{R}}
\newcommand{\Reg}{\mathrm{Reg}}
%\newcommand{\Fitt}{\mathrm{Fitt}}
\newcommand{\tors}{\mathrm{tors}}
\newcommand{\pr}{\mathrm{pr}}
\newcommand{\Ann}{\mathrm{Ann}}
\newcommand{\bz}{\mathbb{Z}}
\newcommand{\La}{\Lambda}
\newcommand{\bq}{\mathbb{Q}}
\def\bigcapp{\raise1ex\hbox{\rotatebox{180}{$\biguplus$}}}
 \def\bigcappd{\raise1ex\hbox{\rotatebox{180}{$\displaystyle\biguplus$}}}


 \newcommand{\ES}{\mathrm{ES}}
 \newcommand{\cyc}{\mathrm{cyc}}
 \newcommand{\cV}{\mathcal{V}}
 \newcommand{\tor}{\mathrm{tor}}
 \newcommand{\tf}{\mathrm{tf}}
 \newcommand{\cE}{\mathcal{E}}
 \newcommand{\cC}{\mathcal{C}}
 \newcommand{\Fr}{\mathrm{Fr}}
 \newcommand{\cR}{\mathcal{R}}
% \newcommand{\UN}{\mathrm{UN}}
 \newcommand{\ram}{\mathrm{ram}}
 \newcommand{\cores}{\mathrm{cores}}
 \newcommand{\bidual}{\bigcap\nolimits}
\newcommand{\exprod}{\bigwedge\nolimits}
\newcommand{\rank}{\mathrm{rk}}
\newcommand{\spc}{\mathrm{split}}
\newcommand{\NS}{\mathrm{NS}}
\newcommand{\Iw}{\mathrm{Iw}}
\newcommand{\Cyc}{\mathrm{Cyc}}
\newcommand{\cchar}{\mathrm{char}}


  \tikzset{
    rotated/.style={rotate=-90, anchor = south},
    rotatedswap/.style={rotate=-90, anchor=north, outer sep=0.75mm}
}


\newcommand{\bLambda}{{\mathpalette\makebLambda\relax}}
\newcommand{\makebLambda}[2]{%
  \raisebox{\depth}{\scalebox{1}[-1]{$\mathsurround=0pt#1\mathbb{V}$}}%
}

\newcommand{\straightLambda}{{\mathpalette\makestraightLambda\relax}}
\newcommand{\makestraightLambda}[2]{%
  \raisebox{\depth}{\scalebox{1}[-1]{$\mathsurround=0pt#1\text{V}$}}%
}
%\renewcommand{\Lambda}{\straightLambda}

\usepackage{dsfont}
\renewcommand{\mathbb}{\mathds}