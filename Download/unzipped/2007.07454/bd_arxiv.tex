\documentclass[a4paper, 
%twoside=semi, %openany, 
headsepline=off, DIV=12, titlepage=false]{scrartcl}

\title{\vspace{-4em}On Universal Norms for $\bm{p}$-adic Representations in Higher Rank Iwasawa Theory}
\date{}
\author{Dominik Bullach \and Alexandre Daoud\vspace{-2em}}
 
 \usepackage[backend=bibtex
 , style=alphabetic
 ]{biblatex}
		\addbibresource{literature}
		\renewcommand*{\bibfont}{\footnotesize}
\renewbibmacro{in:}{%
  \ifentrytype{article}{}{\printtext{\bibstring{in}\intitlepunct}}}
  \DeclareFieldFormat{journaltitle}{#1\isdot}
  \DeclareFieldFormat[article, inproceedings, unpublished]{title}{\textit{#1}}
  \DeclareFieldFormat[inproceedings, incollection, unpublished, book, misc, preprint]{date}{(#1)}
 
\usepackage[utf8]{inputenc}
% \usepackage[english]{babel}
\usepackage{url}

\textwidth=14.5cm
\oddsidemargin=1cm
\evensidemargin=1cm

\usepackage[expansion=false]{microtype}
\usepackage{amssymb,amsmath,amsopn,amsxtra,amsthm,amsfonts}
\usepackage{xr}
\usepackage[mathcal]{euscript}
% \usepackage{mathrsfs}
% \usepackage{dsfont}
% \usepackage{bbm}
% \usepackage{pxfonts}
\usepackage{mathalfa}

% draft
% \usepackage{todonotes}
\usepackage[english, status=final, nomargin, inline]{fixme}
\fxusetheme{color}

\usepackage[pdfusetitle,unicode,hidelinks]{hyperref}

\def\shane#1{{\color{blue}#1}}
\def\elden#1{{\color{purple}#1}}
\def\marc#1{{\color{brown}#1}}
\def\ryomei#1{{\color{red}#1}}

%% lists
\usepackage{enumitem}
% normal lists (outside of theorem environments)
\setlist[enumerate,1]{label={(\arabic*)},itemsep=\parskip} %,leftmargin=0pt
\setlist[itemize,1]{itemsep=\parskip} %,leftmargin=0pt
% lists for theorem environments
\newlist{thmlist}{enumerate}{2}
\setlist[thmlist,1]{label={\em(\roman*)},ref={(\roman*)},%
  itemsep=\parskip,leftmargin=*,align=left}
\setlist[thmlist,2]{label={\em(\alph*)},ref={(\alph*)},%
  itemsep=\parskip,leftmargin=*,align=left,topsep=0.1cm}
% lists for "remark-style" theorem environments
\newlist{remlist}{enumerate}{2}
\setlist[remlist,1]{label={(\roman*)},ref={(\roman*)},itemsep=\parskip,%
  leftmargin=*,align=left}
\setlist[remlist,2]{label={(\alph*)},ref={(\alph*)},itemsep=\parskip,%
  leftmargin=*,align=left,topsep=0.1cm}

%% environments

% \numberwithin{equation}{section}
\makeatletter
\let\c@equation\c@subsubsection
\let\theequation\thesubsubsection
\let\theHequation\theHsubsubsection
\makeatother

\newtheorem{cor}[subsubsection]{Corollary}
\newtheorem{lem}[subsubsection]{Lemma}
\newtheorem{prop}[subsubsection]{Proposition}
\newtheorem{thmconstr}[subsubsection]{Theorem-Construction}
\newtheorem{propconstr}[subsubsection]{Proposition-Construction}
\newtheorem{lemconstr}[subsubsection]{Lemma-Construction}
\newtheorem{ax}[subsubsection]{Axiom}
\newtheorem{conj}[subsubsection]{Conjecture}
\newtheorem{thm}[subsubsection]{Theorem}
\newtheorem{qthm}[subsubsection]{Quasi-Theorem}
\newtheorem{qlem}[subsubsection]{Quasi-Lemma}
\newtheorem{claim}[subsubsection]{Claim}
\newtheorem*{claim*}{Claim}

\newtheorem*{thmA}{Theorem A}
\newtheorem*{thmB}{Theorem B}
\newtheorem*{thmC}{Theorem C}
\newtheorem*{thmD}{Theorem D}


\newcommand{\thenotation}{}  % to make the notation environment unnumbered

\theoremstyle{definition}
\newtheorem{defn}[subsubsection]{Definition}
% \newtheorem{rem}[subsubsection]{Remark}
\newtheorem{quest}[subsubsection]{Question}

\newtheorem{rem}[subsubsection]{Remark}
\newtheorem{constr}[subsubsection]{Construction}
\newtheorem{exam}[subsubsection]{Example}
\newtheorem{notat}[subsubsection]{Notation}

\newcommand{\propconstrref}[1]{Proposition-Construction~\ref{#1}}
\newcommand{\lemconstrref}[1]{Lemma-Construction~\ref{#1}}
\newcommand{\thmref}[1]{Theorem~\ref{#1}}
\newcommand{\qthmref}[1]{Quasi-Theorem~\ref{#1}}
\newcommand{\qlemref}[1]{Quasi-Lemma~\ref{#1}}
\newcommand{\secref}[1]{Section~\ref{#1}}
\newcommand{\ssecref}[1]{\sectsign\ref{#1}}
\newcommand{\sssecref}[1]{\ref{#1}}
\newcommand{\lemref}[1]{Lemma~\ref{#1}}
\newcommand{\propref}[1]{Proposition~\ref{#1}}
\newcommand{\corref}[1]{Corollary~\ref{#1}}
\newcommand{\conjref}[1]{Conjecture~\ref{#1}}
\newcommand{\remref}[1]{Remark~\ref{#1}}
\newcommand{\defref}[1]{Definition~\ref{#1}}
\newcommand{\questref}[1]{Question~\ref{#1}}
\newcommand{\constrref}[1]{Construction~\ref{#1}}
\newcommand{\notatref}[1]{Notation~\ref{#1}}
\newcommand{\examref}[1]{Example~\ref{#1}}
\renewcommand{\eqref}[1]{(\ref{#1})}
\newcommand{\itemref}[1]{\ref{#1}}

%% diagrams
\usepackage{tikz}
\usetikzlibrary{matrix}
\usepackage{tikz-cd}

%% subscripts: http://tex.sheafexchange.com/questions/1013/how-to-typeset-subscript-in-usual-text-mode
% \usepackage{fixltx2e}

% change tocdepth locally within document
\newcommand{\changelocaltocdepth}[1]{%
  \addtocontents{toc}{\protect\setcounter{tocdepth}{#1}}%
  \setcounter{tocdepth}{#1}}

% define commands that appear not to eat spaces
\usepackage{xspace}

% ifthenelse, isempty, etc.
\usepackage{xifthen}

% more flexible \newcommand
\usepackage{xparse}

%%%%%%%%%%%%%%%%%%%%%%%%%%%%%%%%%%%%%%%%%%%%%%%%%%%%%%%%%%%%%%%%%%%%%%%%%%%%%%%
%% MACROS
%%%%%%%%%%%%%%%%%%%%%%%%%%%%%%%%%%%%%%%%%%%%%%%%%%%%%%%%%%%%%%%%%%%%%%%%%%%%%%%

\newcommand{\nc}{\newcommand}
\nc{\renc}{\renewcommand}
\nc{\ssec}{\subsection}
\nc{\sssec}{\subsubsection}
\nc{\on}{\operatorname}
\nc{\term}[1]{#1\xspace}

%% frak
\newcommand{\m}{\mathfrak{m}}
\newcommand{\p}{\mathfrak{p}}
\newcommand{\q}{\mathfrak{q}}


%% make capital letters not italic in math mode
\DeclareMathSymbol{A}{\mathalpha}{operators}{`A}
\DeclareMathSymbol{B}{\mathalpha}{operators}{`B}
\DeclareMathSymbol{C}{\mathalpha}{operators}{`C}
\DeclareMathSymbol{D}{\mathalpha}{operators}{`D}
\DeclareMathSymbol{E}{\mathalpha}{operators}{`E}
\DeclareMathSymbol{F}{\mathalpha}{operators}{`F}
\DeclareMathSymbol{G}{\mathalpha}{operators}{`G}
\DeclareMathSymbol{H}{\mathalpha}{operators}{`H}
\DeclareMathSymbol{I}{\mathalpha}{operators}{`I}
\DeclareMathSymbol{J}{\mathalpha}{operators}{`J}
\DeclareMathSymbol{K}{\mathalpha}{operators}{`K}
\DeclareMathSymbol{L}{\mathalpha}{operators}{`L}
\DeclareMathSymbol{M}{\mathalpha}{operators}{`M}
\DeclareMathSymbol{N}{\mathalpha}{operators}{`N}
\DeclareMathSymbol{O}{\mathalpha}{operators}{`O}
\DeclareMathSymbol{P}{\mathalpha}{operators}{`P}
\DeclareMathSymbol{Q}{\mathalpha}{operators}{`Q}
\DeclareMathSymbol{R}{\mathalpha}{operators}{`R}
\DeclareMathSymbol{S}{\mathalpha}{operators}{`S}
\DeclareMathSymbol{T}{\mathalpha}{operators}{`T}
\DeclareMathSymbol{U}{\mathalpha}{operators}{`U}
\DeclareMathSymbol{V}{\mathalpha}{operators}{`V}
\DeclareMathSymbol{W}{\mathalpha}{operators}{`W}
\DeclareMathSymbol{X}{\mathalpha}{operators}{`X}
\DeclareMathSymbol{Y}{\mathalpha}{operators}{`Y}
\DeclareMathSymbol{Z}{\mathalpha}{operators}{`Z}

%% \mathcal shortcuts
\nc{\sA}{\ensuremath{\mathcal{A}}\xspace}
\nc{\sB}{\ensuremath{\mathcal{B}}\xspace}
\nc{\sC}{\ensuremath{\mathcal{C}}\xspace}
\nc{\sD}{\ensuremath{\mathcal{D}}\xspace}
\nc{\sE}{\ensuremath{\mathcal{E}}\xspace}
\nc{\sF}{\ensuremath{\mathcal{F}}\xspace}
\nc{\sG}{\ensuremath{\mathcal{G}}\xspace}
\nc{\sH}{\ensuremath{\mathcal{H}}\xspace}
\nc{\sI}{\ensuremath{\mathcal{I}}\xspace}
\nc{\sJ}{\ensuremath{\mathcal{J}}\xspace}
\nc{\sK}{\ensuremath{\mathcal{K}}\xspace}
\nc{\sL}{\ensuremath{\mathcal{L}}\xspace}
\nc{\sM}{\ensuremath{\mathcal{M}}\xspace}
\nc{\sN}{\ensuremath{\mathcal{N}}\xspace}
\nc{\sO}{\ensuremath{\mathcal{O}}\xspace}
\nc{\sP}{\ensuremath{\mathcal{P}}\xspace}
\nc{\sQ}{\ensuremath{\mathcal{Q}}\xspace}
\nc{\sR}{\ensuremath{\mathcal{R}}\xspace}
\nc{\sS}{\ensuremath{\mathcal{S}}\xspace}
\nc{\sT}{\ensuremath{\mathcal{T}}\xspace}
\nc{\sU}{\ensuremath{\mathcal{U}}\xspace}
\nc{\sV}{\ensuremath{\mathcal{V}}\xspace}
\nc{\sW}{\ensuremath{\mathcal{W}}\xspace}
\nc{\sX}{\ensuremath{\mathcal{X}}\xspace}
\nc{\sY}{\ensuremath{\mathcal{Y}}\xspace}
\nc{\sZ}{\ensuremath{\mathcal{Z}}\xspace}

%% \mathbf shortcuts
\nc{\bA}{\ensuremath{\mathbf{A}}\xspace}
\nc{\bB}{\ensuremath{\mathbf{B}}\xspace}
\nc{\bC}{\ensuremath{\mathbf{C}}\xspace}
\nc{\bD}{\ensuremath{\mathbf{D}}\xspace}
\nc{\bE}{\ensuremath{\mathbf{E}}\xspace}
\nc{\bF}{\ensuremath{\mathbf{F}}\xspace}
\nc{\bG}{\ensuremath{\mathbf{G}}\xspace}
\nc{\bH}{\ensuremath{\mathbf{H}}\xspace}
\nc{\bI}{\ensuremath{\mathbf{I}}\xspace}
\nc{\bJ}{\ensuremath{\mathbf{J}}\xspace}
\nc{\bK}{\ensuremath{\mathbf{K}}\xspace}
\nc{\bL}{\ensuremath{\mathbf{L}}\xspace}
\nc{\bM}{\ensuremath{\mathbf{M}}\xspace}
\nc{\bN}{\ensuremath{\mathbf{N}}\xspace}
\nc{\bO}{\ensuremath{\mathbf{O}}\xspace}
\nc{\bP}{\ensuremath{\mathbf{P}}\xspace}
\nc{\bQ}{\ensuremath{\mathbf{Q}}\xspace}
\nc{\bR}{\ensuremath{\mathbf{R}}\xspace}
\nc{\bS}{\ensuremath{\mathbf{S}}\xspace}
\nc{\bT}{\ensuremath{\mathbf{T}}\xspace}
\nc{\bU}{\ensuremath{\mathbf{U}}\xspace}
\nc{\bV}{\ensuremath{\mathbf{V}}\xspace}
\nc{\bW}{\ensuremath{\mathbf{W}}\xspace}
\nc{\bX}{\ensuremath{\mathbf{X}}\xspace}
\nc{\bY}{\ensuremath{\mathbf{Y}}\xspace}
\nc{\bZ}{\ensuremath{\mathbf{Z}}\xspace}

%% \mathds shortcuts
\nc{\dA}{\ensuremath{\mathds{A}}\xspace}
\nc{\dB}{\ensuremath{\mathds{B}}\xspace}
\nc{\dC}{\ensuremath{\mathds{C}}\xspace}
\nc{\dD}{\ensuremath{\mathds{D}}\xspace}
\nc{\dE}{\ensuremath{\mathds{E}}\xspace}
\nc{\dF}{\ensuremath{\mathds{F}}\xspace}
\nc{\dG}{\ensuremath{\mathds{G}}\xspace}
\nc{\dH}{\ensuremath{\mathds{H}}\xspace}
\nc{\dI}{\ensuremath{\mathds{I}}\xspace}
\nc{\dJ}{\ensuremath{\mathds{J}}\xspace}
\nc{\dK}{\ensuremath{\mathds{K}}\xspace}
\nc{\dL}{\ensuremath{\mathds{L}}\xspace}
\nc{\dM}{\ensuremath{\mathds{M}}\xspace}
\nc{\dN}{\ensuremath{\mathds{N}}\xspace}
\nc{\dO}{\ensuremath{\mathds{O}}\xspace}
\nc{\dP}{\ensuremath{\mathds{P}}\xspace}
\nc{\dQ}{\ensuremath{\mathds{Q}}\xspace}
\nc{\dR}{\ensuremath{\mathds{R}}\xspace}
\nc{\dS}{\ensuremath{\mathds{S}}\xspace}
\nc{\dT}{\ensuremath{\mathds{T}}\xspace}
\nc{\dU}{\ensuremath{\mathds{U}}\xspace}
\nc{\dV}{\ensuremath{\mathds{V}}\xspace}
\nc{\dW}{\ensuremath{\mathds{W}}\xspace}
\nc{\dX}{\ensuremath{\mathds{X}}\xspace}
\nc{\dY}{\ensuremath{\mathds{Y}}\xspace}
\nc{\dZ}{\ensuremath{\mathds{Z}}\xspace}

%% \mathbb shortcuts
\nc{\bbA}{\ensuremath{\mathbb{A}}\xspace}
\nc{\bbB}{\ensuremath{\mathbb{B}}\xspace}
\nc{\bbC}{\ensuremath{\mathbb{C}}\xspace}
\nc{\bbD}{\ensuremath{\mathbb{D}}\xspace}
\nc{\bbE}{\ensuremath{\mathbb{E}}\xspace}
\nc{\bbF}{\ensuremath{\mathbb{F}}\xspace}
\nc{\bbG}{\ensuremath{\mathbb{G}}\xspace}
\nc{\bbH}{\ensuremath{\mathbb{H}}\xspace}
\nc{\bbI}{\ensuremath{\mathbb{I}}\xspace}
\nc{\bbJ}{\ensuremath{\mathbb{J}}\xspace}
\nc{\bbK}{\ensuremath{\mathbb{K}}\xspace}
\nc{\bbL}{\ensuremath{\mathbb{L}}\xspace}
\nc{\bbM}{\ensuremath{\mathbb{M}}\xspace}
\nc{\bbN}{\ensuremath{\mathbb{N}}\xspace}
\nc{\bbO}{\ensuremath{\mathbb{O}}\xspace}
\nc{\bbP}{\ensuremath{\mathbb{P}}\xspace}
\nc{\bbQ}{\ensuremath{\mathbb{Q}}\xspace}
\nc{\bbR}{\ensuremath{\mathbb{R}}\xspace}
\nc{\bbS}{\ensuremath{\mathbb{S}}\xspace}
\nc{\bbT}{\ensuremath{\mathbb{T}}\xspace}
\nc{\bbU}{\ensuremath{\mathbb{U}}\xspace}
\nc{\bbV}{\ensuremath{\mathbb{V}}\xspace}
\nc{\bbW}{\ensuremath{\mathbb{W}}\xspace}
\nc{\bbX}{\ensuremath{\mathbb{X}}\xspace}
\nc{\bbY}{\ensuremath{\mathbb{Y}}\xspace}
\nc{\bbZ}{\ensuremath{\mathbb{Z}}\xspace}


%% convenience macros
\nc{\mrm}[1]{\ensuremath{\mathrm{#1}}\xspace}
\nc{\mbf}[1]{\ensuremath{\mathbf{#1}}\xspace}
\nc{\mcal}[1]{\ensuremath{\mathcal{#1}}\xspace}
\nc{\msc}[1]{\ensuremath{\mathscr{#1}}\xspace}

%% some general shortcuts
\renc{\bar}[1]{\overline{#1}}

\let\sectsign\S
\let\S\relax

%% inclusions and arrows
\nc{\sub}{\subset}
\nc{\too}{\longrightarrow}
\nc{\hook}{\hookrightarrow}
\nc*{\hooklongrightarrow}{\ensuremath{\lhook\joinrel\relbar\joinrel\rightarrow}}
\nc{\hooklong}{\hooklongrightarrow}
\nc{\twoheadlongrightarrow}{\relbar\joinrel\twoheadrightarrow}
\nc{\shiso}{\approx}
\nc{\isoto}{\xrightarrow{\sim}}
% \nc{\isofrom}{\stackrel{\sim}{\longleftarrow}}
\nc{\isofrom}{\xleftarrow{\sim}}
% \renc{\mapsto}{\rightsquigarrow}
\renc{\ge}{\geqslant}
\renc{\le}{\leqslant}
\renc{\geq}{\geqslant}
\renc{\leq}{\leqslant}

%% general notation
\nc{\id}{\mathrm{id}}
%\DeclareMathOperator{\ker}{\mathrm{ker}}
\DeclareMathOperator{\Ker}{\mathrm{Ker}}
\DeclareMathOperator{\Coker}{\mathrm{Coker}}
\DeclareMathOperator{\Coim}{\mathrm{Coim}}
\let\Im\relax
\DeclareMathOperator{\Im}{\mathrm{Im}}
\DeclareMathOperator{\rk}{\mathrm{rk}}
\DeclareMathOperator{\Hom}{\mathrm{Hom}}
\nc{\uHom}{\underline{\smash{\Hom}}}
\DeclareMathOperator{\Maps}{\mathrm{Maps}}
\DeclareMathOperator{\Aut}{\mathrm{Aut}}
\DeclareMathOperator{\End}{\mathrm{End}}
\DeclareMathOperator{\Sym}{\mathrm{Sym}}
\nc{\Pre}{\mathrm{PSh}{}}
\nc{\Shv}{\mathrm{Shv}{}}
\nc{\uEnd}{\underline{\smash{\End}}}
\DeclareMathOperator{\codim}{\mathrm{codim}}
\renc{\lim}{\operatorname*{lim}}
\nc{\colim}{\operatorname*{colim}}
%\nc{\colim}{\varinjlim}
%\renc{\lim}{\varprojlim}
\nc{\Cofib}{\on{Cofib}}
\nc{\Fib}{\on{Fib}}
\nc{\initial}{\varnothing}
\nc{\op}{\mathrm{op}}

% operators
\DeclareMathOperator*{\fibprod}{\times}
\DeclareMathOperator*{\fibcoprod}{\operatorname{\sqcup}}
\let\bigcoprod=\coprod
\renc{\coprod}{\sqcup}
\DeclareMathOperator*{\biprod}{\,\oplus\,}


\setlength\parindent{0pt}
 
\begin{document}

\maketitle

\begin{abstract}
    We begin a systematic investigation of universal norms for $p$-adic representations in higher rank Iwasawa theory. After establishing the basic properties of the module of higher rank universal norms we construct an Iwasawa-theoretic pairing that is relevant to this setting. This allows us, for example, to refine the classical Iwasawa Main Conjecture for cyclotomic fields, and also to give applications to various well-known conjectures in arithmetic concerning Iwasawa invariants and leading terms of $L$-functions. 
\end{abstract}





%\classification{}
%\keywords{Units, Iwasawa Theory, Class groups}
\let\thefootnote\relax\footnotetext{2020 {\em Mathematics Subject Classification:} Primary: 11F80, 11R23; Secondary: 11R33.}


\tableofcontents

\section{Introduction}
\pagestyle{special}

The investigation of the deep connection between $L$-functions and arithmetic is at the heart of modern number theory. By now we have a number of partial results on this matter, many of which due to celebrated results obtained via the Euler system method that has been developed by Thaine, Kolyvagin, Rubin and Mazur.\\
However, all of these (unconditional) results are restricted to cases where the order of vanishing of the $L$-function is at most one. Although a notion of \textit{higher rank} Euler system was already established by Perrin-Riou more than 20 years ago, technical issues arising from the use of exterior powers 
hindered the theory surrounding higher rank Euler systems from being fully operational. These technical obstructions have only recently been overcome by Burns, Sakamoto and Sano in a series of articles (\cite{EulerSystemsSagaI},\ \cite{EulerSystemsSagaII},\ \cite{EulerSystemsSagaIII} and \cite{EulerSystemsSagaIV})\@. Key to their approach is the consistent use of \textit{exterior biduals} instead of exterior powers, a notion that is based on the lattice introduced by Rubin in \cite[\S 1.2]{Rubin96} and provides better functorial properties in many aspects.\\
Since Euler systems are, by their very definition, universal norms on $\ZZ_p$-extensions, we feel that the study of \textit{higher rank universal norms} 
undertaken in this article naturally fits into the chain of developments described above. As in the aforementioned works, the use of exterior biduals allows us to develop a theory that naturally extends the classical theory of universal norms to both the higher rank and equivariant settings. 

\paragraph{Overview of results} To explain our results in a little more detail, we first introduce some notation. Let $L | K$ be a finite abelian extension of number fields, $p$ an odd prime and 
take $L_\infty = \bigcup_{n \geq 0} L_n$ to be a $\Z_p$-extension of $L$ that is abelian over $K$ and in which no finite place of $K$ splits completely. Denote by 
$\Lambda = \Z_p \llbracket \gal{L_\infty}{L} \rrbracket$ and $\bLambda = \Z_p \llbracket \gal{L_\infty}{K} \rrbracket$ the relevant Iwasawa algebras. 
For a $p$-adic representation $T$ of $K$ we shall below define natural modules $\UN^r_n (T)$ and $\NS^r (T)$ of universal norms and norm-coherent sequences, respectively, of rank $r$ and level $n$ along the $\Z_p$-extension $L_\infty |L$. \\
Our first step in extending the classical theory of universal norms as established by, for example, Kuz'min \cite{Kuzmin} and Greither \cite{Greither} (see Remark \ref{structure-theorem-remark} (a) for more details on the existing literature) is then the following theorem.


\begin{thmplain}[Thm.\@ \ref{UN-structure-theorem}]
    Fix an integer $1 \leq r \leq r_T$ where $r_T$ denotes the \textit{basic rank} of $T$ (see \ref{main-hypothesis} (2)). Then, under certain mild conditions, the natural codescent map induces an isomorphism of $\Z_p [\gal{L_n}{K}]$-modules
    \begin{align*}
         \NS^r \otimes_{\bLambda} \Z_p [\gal{L_n}{K}] \cong \UN^r_n (T).
    \end{align*}
    Moreover, $\NS^r$ (resp.\@ $\UN^r_n$) is a free module of rank $[L:K]\cdot {r_T \choose r}$ over $\Lambda$ (resp.\@ $\Z_p [\gal{L_n}{L}]$).
\end{thmplain}

While this result shows that non-trivial higher rank universal norms exist, its proof is inherently non-constructive. We shall, however, give an elementary construction of a large $\bLambda$-submodule $\NS^b$ of $\NS^{r_T}$ that is of arithmetic significance as the following result shows. 

\begin{thmplain}[Thm.\@ \ref{pairing-theorem}]
    There exists a free rank one $\bLambda$-submodule $\NS^b$ of $\NS^{r_T}$ together with a perfect pairing of $\bLambda$-modules
    \begin{align*}
        \faktor{\NS^{r_T}}{\NS^b} \times \faktor{\bLambda}{\Fitt_{\bLambda}(H^2_{\Sigma,\Iw}(\cO_{L,S}, T))} \to \faktor{Q(\bLambda)}{\bLambda},
    \end{align*}
    where $Q (\bLambda)$ denotes the total ring of quotients of $\bLambda$ and $H^2_{\Sigma,\Iw}(\cO_{L,S}, T))$ is a modified Iwasawa cohomology group. 
\end{thmplain}

This pairing combines with the cyclotomic equivariant Iwasawa Main Conjecture proven by Burns and Greither \cite{BurnsGreither} to give in Theorem \ref{IMC-refinement} an explicit refinement of the classical cyclotomic Iwasawa main conjecture as follows (see Remark \ref{IMC-refinement-remark} for more details of the precise nature of the relation to the Main Conjecture).

\begin{thmplain}[Thm.\@ \ref{IMC-refinement}]
    Let $K = \QQ$ and let $L$ be the maximal totally real subfield of the cyclotomic field $\Q (\xi_{pf})$ for an integer $f$ coprime to $p$. If $p \nmid [L:\QQ]$, then for every character $\chi$ of $\gal{L}{K}$ 
    there is an isomorphism of $\Lambda_\chi := \ZZ_p(\im(\chi))\llbracket \Gamma \rrbracket$-modules
    \begin{align*}
        \faktor{U^{\infty,\chi}}{\Cyc^{\infty,\chi}} \cong \alpha  \left ( \faktor{\Lambda_\chi}{\Fitt^0_{\Lambda_\chi}(A^{\infty,\chi})} \right),
    \end{align*}
    where $(-)^\chi$ is the functor taking $\chi$-isotypic parts, $U^\infty := \varprojlim_n \cO_{L_n}^\times \otimes_\ZZ \ZZ_p$, $\Cyc^\infty$ is the inverse limit of the groups of $p$-completed cyclotomic units of the field $L_n$, $A^\infty$ is the inverse limit of $p$-parts of the class groups of the fields $L_n$, and $\alpha (-) = \Ext^1_{\Lambda_\chi} ( -, \Lambda_\chi)$ denotes the Iwasawa adjoint. 
\end{thmplain}

It is conjectured, in great generality, that $H^2_\Iw(\cO_{L,S}, T)$ should be finitely generated as a $\Z_p$-module (\textit{c.f.}\@ Conjecture \ref{mu-vanishing-conjecture}). The above pairing allows us to give a reformulation of this conjecture in terms of the quotient $\NS^{r_T}/\NS^b$.

\begin{propplain}[Prop.\@ \ref{mu-vanishing-result}]
    $H^2_\Iw(\cO_{L,S}, T)$ is finitely generated as a $\Z_p$-module if and only if the same is true of $\NS^{r_T}/\NS^b$.
\end{propplain}

Since the aforementioned conjecture is known to hold in several cases one can use this equivalence to obtain several unconditional examples of the finite generation of $\NS^{r_T}/\NS^b$ as a $\Z_p$-module. We give one such example in the setting of elliptic curves in Corollary \ref{mu-vanishing-example}.\\

We also explore connections to Greenberg's conjecture and equivariant leading term conjectures. For statements of these results the reader is referred to Proposition \ref{GreenbergCriterion} and Theorem \ref{etnc-thm}, respectively.  

\begin{acknowledgments}
    The authors would like to extend their gratitude to David Burns and Takamichi Sano for several stimulating conversations and for their valuable comments on earlier versions of the present manuscript. They would also like to thank Andrew Graham, Martin Hofer and Daniel Macias Castillo for useful comments and discussions.
\end{acknowledgments}

\markboth{Higher rank universal norms}{The set-up}
\section{Higher rank universal norms}


\subsection{The set-up}\label{set-up-section}\pagestyle{default}
Fix an odd prime $p$ and let $K$ be a number field with $G_K$ its absolute Galois group. We write $S_\infty(K)$ for the set of archimedean places of $K$, and $S_p(K)$ for the set of $p$-adic places of $K$.
Given a Galois extension $F| K$ we write $S_\ram(F|K)$ for the places of $K$ that ramify in $F$ and $S_\mathrm{split}(F|K)$ for the places of $K$ that split completely in $F$. 
If $S$ is a set of places of $K$, we denote by $S_F$ the set of places of $F$ that lie above those contained in $S$. We will however omit the explicit reference to the field in case it is clear from the context. For example, $\bigO_{F, S}$ shall denote the ring of $S_F$-integers of $F$. 
\\

Given any commutative unital ring $R$ we write $Q(R)$ for the total quotient ring of $R$; that is to say, the localisation of $R$ at the multiplicative set of non-zero-divisors. If $M$ is an $R$-module, then we denote $M^\lor := \Hom_{R}(M, Q(R)/R)$.

For an abelian group $A$ we denote by $A_\tor$ its torsion-subgroup and by $A_\tf = \faktor{A}{A_\tor}$ its torsion-free part.
If $A$ is finite, we denote by $\widehat{A} = \Hom_\Z (A, \C^\times)$ its character group, and for any $\chi \in \widehat{A}$ we let
\[
e_\chi = \frac{1}{|A|} \sum_{\sigma \in A} \chi (\sigma) \sigma^{-1} \quad \in \C [A]
\]
be the usual primitive orthogonal idempotent associated to $\chi$.
\\

Let $\mathcal{Q}$ be a finite extension of $\Q_p$ with ring of integers $\mathcal{R}$. We also establish the following objects and notations: 
\begin{itemize}
\item $L | K$ a finite abelian extension of number fields with Galois group $\cG$ in which every archimedean place splits completely,
\item $L_\infty | L$ a $\Z_p$-extension in which no non-archimedean place splits and such that the extension $L_\infty | K$ is Galois and has Galois group $\Gamma \times \cG$, where $\Gamma = \gal{L_\infty}{L} \cong \Z_p$, 
\item $\Gamma^n = \gal{L_\infty}{L_n}$ the unique subgroup of $\Gamma$ of index $p^n$, and $\Gamma_n = \faktor{\Gamma}{\Gamma^n}$, 
\item $\cG_n = \gal{L_n}{K}$,
\item $\Lambda = \mathcal{R} \llbracket \Gamma \rrbracket = \varprojlim_n \cR [\Gamma_n]$ the Iwasawa algebra and $\bLambda = \mathcal{R} \llbracket \gal{L_\infty}{K} \rrbracket = \varprojlim_n \cR [\cG_n]$ its equivariant counterpart. Due to our assumptions, we have a decomposition $\bLambda = \Lambda [\cG]$. 
\end{itemize}

We now fix a $p$-adic representation $T$ with coefficients in $\cR$. That is to say, a free $\cR$-module endowed with a continuous $G_K$-action that we regard as a sheaf on the \'etale site of $\Spec K$. Assume that the set $S_{\ram}(T)$ of places of $K$ at which $T$ has bad reduction is finite. We then fix a finite set $S$ of places of $K$ containing
\begin{align*}
    S_\infty(K) \cup S_p(K) \cup S_\ram(T).
\end{align*}
We write $S_n = S \cup S_\ram(L_n | K)$ and note that these sets stabilise for large enough $n$. Let $T^\ast (1) := \Hom_\cR (T, \cR(1))$, where $\cR (1) = \cR \otimes_{\Z_p} \Z_p (1)$. 

Given a $\bLambda$-module $M$ we write $M^\#$ for the $\bLambda$-module which has the same underlying $\Lambda$-module structure as $M$ but with the $\cG$-action twisted by the involution $g \mapsto g^{-1}$ for $g \in \cG$. Similarly, if $\gamma$ is a topological generator of $\Gamma$, then we write $M^\circ$ for the $\bLambda$-module which has the same underlying $\cR[\cG]$-module structure as $M$ but with the $\Gamma$-action twisted by the involution $\gamma \mapsto \gamma^{-1}$.



\paragraph{$\Sigma$-modified \'etale cohomology}
In this article it is necessary to slightly modify the usual compactly supported \'etale cohomology complex of the representation $T$ in order to ensure that the cohomology in the lowest degree is $\cR$-torsion free. This should, however, be regarded as a convenient technical device rather than an integral feature of the theory since in many interesting cases it can actually be disregarded (see the Examples \ref{RepExamples}). We first briefly recall the definitions of the relevant complexes from \cite[\S 2.3]{EulerSystemsSagaI}.  \\

Let $\Sigma$ be a finite set of places of $K$ that is disjoint from $S_n$ for all $n \in \N_0$. For any place $w \in \Sigma_{L_n}$ denote by $\kappa_w$ the residue field of $\bigO_{L_n, S}$ at $w$. Then we define the \emph{$\bm{\Sigma}$-modified \'etale cohomology complex} of $T$ to be
\[
\text{R} \Gamma_\Sigma (\bigO_{L_n, S}, T) := \cone \Big \{ \text{R} \Gamma_{\et} ( \bigO_{L_n, S}, T) \to \bigoplus_{w \in \Sigma_{L_n}} \text{R} \Gamma_{\et} ( \kappa_w, T) \Big \} [-1]
\]
and set $H^i_\Sigma (\bigO_{L_n, S}, T) := H^i (\text{R} \Gamma_\Sigma (\bigO_{L_n, S}, T))$ for all $i \in \Z$. 
 We shall fix such a choice of $\Sigma$ for the remainder of this article. For any $i \in \Z$, the \emph{$\bm{\Sigma}$-modified Iwasawa cohomology} of $T$ with respect to the $\Z_p$-extension $L_\infty$ is defined as 
\[
H^i_{\Sigma, \text{Iw}} (\bigO_{L, S}, \; T) = \varprojlim_{n \in \N} H^i_\Sigma(\cO_{L_n, S}, T),
\]
and this limit can be naturally endowed with the structure of a $\bLambda$-module. \\

Throughout this article we suppose, unless explcitly stated otherwise, that the tuple $(T,L_\infty, \Sigma)$ satisfies the following mild hypotheses:
\begin{hypotheses}\label{main-hypothesis}\text{}
    \begin{enumerate}[label=(\arabic*)]
        \item{For every $n \in \NN$ one has that the module of invariants $H^0_\Sigma(L_n, T)$ vanishes.}
        \item{The $\cR$-free module $Y_K(T) = \bigoplus_{v \in S_\infty(K)} H^0(K_v,T^*(1))$ has non-zero rank $r_T$ (which we may often refer to as the \emph{basic rank} of the tuple $(T, L_\infty, \Sigma)$).}
        \item{$H^1_\Sigma(\cO_{L_n, S}, T)$ is $\cR$-torsion-free for every $n \in \N_0$}.
        \item{$H^2_{\Sigma,\Iw}(\cO_{L,S},T)$ is a torsion $\Lambda$-module.}
    \end{enumerate}
\end{hypotheses}

\begin{remark} \label{HypothesesRemark}\phantom{David}
    \begin{liste}
        \item{
        It is shown in \cite[Lem. B.6]{daoud}, for example, that Hypothesis \ref{main-hypothesis}(3) is satisfied for any singleton $\Sigma = \{ v \}$ consisting of a place $v \not \in S$ that satisfies $H^0(K_v,T) = 0$.
        }
        \item{Hypothesis \ref{main-hypothesis}(4) is (the $\Sigma$-modified version of) the weak Leopoldt conjecture for $p$-adic representations due to Perrin-Riou \cite[\S 1.3]{PR95}. 
        In fact, in Lemma \ref{mu-vanishing-independent-lemma} below we show that, under additional mild conditions on the tuple $(T,L_\infty, \Sigma)$, this hypothesis is independent of the choice of $\Sigma$ and is thus equivalent to requiring that $H^2 ( \gal{L^S}{L_\infty}, T \otimes_{\cR} \faktor{\cQ}{\cR} (1)) = 0$, where $L^S$ is the maximal Galois extension of $L$ unramified outside $S$ (see, for example, \cite[Prop.\ 1.3.2]{PR95})).
        The conjecture is known in many cases naturally arising in arithmetic (see \cite[Appendix B]{PR95}}) and we shall recall some of these examples below. 
    \end{liste}
\end{remark}

\begin{bspe1} \label{RepExamples} \phantom{phantom}
\begin{liste}
\item     Let $\cR = \ZZ_p$ and $T = \ZZ_p(1)$, then $T$ always satisfies the hypotheses \ref{main-hypothesis} (1) and (2). Moreover, for each $n \geq 1$, Kummer theory gives a canonical identification
    \begin{align*}
        H^1_\Sigma(\cO_{L_n, S}, T) = \Z_p \otimes_\Z \ker \Big \{ 
         \bigO_{L_n, S }^\times \to \bigoplus_{w \in \Sigma_{L_n}} \Big ( \faktor{\bigO_{L_n, S}}{ w}  \Big)^\times
        \Big \}.
    \end{align*}
    The group on the right is the $p$-completion of the $(S , \Sigma)$-unit group $\bigO_{L, S, \Sigma}^\times$ and plays an important role in the context of the Stark conjectures. 
    In particular, if $L$ and $K$ are both totally real, then we may take $\Sigma = \varnothing$.\\
    The validity of the weak Leopoldt conjecture in this setting is known if $L_\infty | L$ is the cyclotomic $\Z_p$-extension (by a result of Iwasawa \cite{Iwasawa73}, see \cite[\S 1.3, Rem.\@ ii)]{PR95}). 
    
    \item If $T = \text{T}_p E = H^1_\et ( E_{\overline{\Q}}, \Z_p)^\ast$ is the Tate module of an elliptic curve $E$ defined over $K$, then \ref{main-hypothesis} (1) holds
    , and \ref{main-hypothesis} (2) holds because $\text{T}_p E$ is an odd representation (due to the Weil pairing).
    If $E ( L_n)$ is $p$-torsion free for all $n \in \N_0$, then we may take $\Sigma = \emptyset$. 
    
    Since every elliptic curve defined over $\Q$ is modular, the validity of the weak Leopoldt conjecture \ref{main-hypothesis}(4) for $K = \Q$ follows from \cite[Thm.\@ 12.4(i)]{kato}. 

    \item Let $f$ be a normalised cuspidal newform of weight $k \geq 2$ and level $N \geq 5$, and take $\cR$ to be a finite extension of $\Z_p$ that contains the Fourier coefficients of $f$ (using some fixed embedding $\overline{\Q_p} \hookrightarrow \C$). Then one can attach a rational $p$-adic representation $V_f$ of $G_\Q$ to $f$, 
    that is to say a finite dimensional $\mathcal{Q}$-vector space with a continuous $G_\Q$-action,
    see for example \cite{Deligne}. Let $T_f \subseteq V_f$ be a Galois-stable lattice. 
    Since the complex absolute values of the eigenvalues of $\Frob_\q$ for $\q \nmid p N$ are $p^{(k - 1) / 2}$, the representation $T_f$ satisfies hypothesis \ref{main-hypothesis} (1). Moreover, the representation $T_f$ is odd, so we have $H^0 (\R, V_f^\ast (1)) \neq 0$ and $T_f$ also satisfies hypothesis \ref{main-hypothesis} (2). Finally, $T_f$ satisifies hypothesis \ref{main-hypothesis} for $K = \Q$ by \cite[Thm.\@ 12.4(i)]{kato}. 
    \end{liste}
\end{bspe1}

Given these definitions, we have the \emph{$\bm{\Sigma}$-modified compactly supported \'etale cohomology complex} 
\[
\text{R}\Gamma_{c, \Sigma} ( \bigO_{L_n, S}, T) = 
\text{R}\Hom_\cR ( \text{R} \Gamma_\Sigma ( \bigO_{L_n, S}, T^\ast (1)), \cR) [-3] \oplus \Big ( \bigoplus_{w \in S_\infty (L_n)} H^0 ( (L_{n})_w, T) \Big ) [-1]
\]
as well as the complex
\begin{align*}
    C_n^\bullet := \text{R}\Hom_\cR(\text{R}\Gamma_{c,\Sigma}(\cO_{L_n, S(L_n)}, T^*(1)), \cR)[-2].
\end{align*}
Below we record the properties of these constructions that are needed in this article. 

\begin{prop}[\cite{EulerSystemsSagaI}, Prop.\@ 2.21] \label{FiniteComplex}\text{}
\begin{liste}
    \item{$C^\bullet_n$ is acyclic outside degrees zero and one, and is perfect as an element of the derived category $D(\cR [\cG_n])$.}
    \item{There is a canonical isomorphism
        \begin{align*}
            H^0(C^\bullet_n) \cong H^1_{\Sigma}(\cO_{L_n, S}, T)
        \end{align*}
        and a split short exact sequence
        \begin{equation} \label{yoneda-extension-sequence-H2-finite}
        \begin{tikzcd}
           0 \arrow{r} & H^2_\Sigma ( \bigO_{L_n, S}, T) \arrow{r} & H^1 (C^\bullet_n) \arrow{r} & Y_K (T)^\ast \otimes_\cR \cR [\cG_n] \arrow{r} & 0
        \end{tikzcd}
        \end{equation}
        in which the first map is canonical and the second depends on the choice of a set of representatives of the orbits of $\gal{L_n}{K}$ on $S_\infty(K)$.
    }
\end{liste}
\end{prop}

Next we introduce the Iwasawa-theoretic variants of the above constructions.\\ 

If we denote by $C^\bullet_\infty$ the complex of $\bLambda$-modules $\varprojlim_n C^\bullet_n$ (where the limit is taken with respect to the natural codescent morphisms), then we have the following analogue of Proposition \ref{FiniteComplex}.

\begin{prop}\phantom{phantom}
\begin{liste}
    \item{$C^\bullet_\infty$ is acyclic outside degrees zero and one, and is perfect as an element of the derived category $D(\bLambda)$.}
    \item{There is a canonical isomorphism
        \begin{align*}
            H^0(C^\bullet_\infty) \cong H^1_{\Sigma, \Iw}(\cO_{L,S}, T)
        \end{align*}
        and a split short exact sequence
    \begin{equation} \label{yoneda-extension-sequence-H2}
        \begin{tikzcd}[column sep=small] 
            0 \arrow{r} & H^2_{\Sigma,\Iw}(\cO_{L, S}, T) \arrow{r} & H^1(C^\bullet_\infty) \arrow{r} & \varprojlim_n (Y_K(T)^* \otimes_\cR \cR[\cG_n]) \arrow{r} & 0,
        \end{tikzcd}
        \end{equation}
        where the injection is canonical and the surjection depends on a choice of a set of representatives of the orbits of $\gal{L_\infty}{K}$ on $S_\infty(K)$.
    }
\end{liste}
\end{prop}

\begin{proof}
    Let $\mathcal{P}$ be the abelian category of complexes of profinite $\bLambda$-modules. Then the inverse limit functor from the category of inverse systems pro-$\mathcal{P}$ to $\mathcal{P}$ is exact since each object of $\mathcal{P}$ is a complex of compact Hausdorff spaces. In particular, this functor commutes with passage to cohomology. Now, the inverse system  given by the $C_n$ with their natural transition morphisms is an object of the category pro-$\mathcal{P}$. As such, the first claim of (a) and both claims of (b) follow immediately from Proposition \ref{FiniteComplex}. The fact that $C_\infty^\bullet$ is a perfect complex is proven in \cite[Lem.\@ 1.6.5(ii)]{fukaya-kato}.
\end{proof}

\begin{lemma}\label{standard-representative-lemma}
    There exists a quadratic standard representative $[\Pi \xrightarrow{\psi} \Pi]$ (in the sense of \cite[Def.\@ A.6]{EulerSystemsSagaI}) of the complex $C_\infty^\bullet$ with respect to the surjection $H^1(C_\infty^\bullet) \xrightarrow{f} Y_K(T)^* \otimes_\cR \bLambda$. 
\end{lemma}

\begin{proof}
    By definition we are required to exhibit a representative $[\Pi \xrightarrow{\psi} \Pi]$ of $C_\infty^\bullet$ in $D(\bLambda)$ with the property that for the free module $\Pi$ there exists a basis $\{b_1,\dots, b_d\}$ of $\Pi$ and an exact sequence
    \begin{align*}
        \langle b_{r_T+1}, \dots, b_d\rangle_{\bLambda} \to H^1(C) \xrightarrow{f} Y_K(T)^* \otimes_\cR \Lambda \to 0
    \end{align*}
    where the first map is induced by the natural map $\Pi \to H^1(C_\infty^\bullet)$. This is proved in
    \cite[Lem.\@ 7.10]{BuSaNC} (where the complex $C_\infty^\bullet$ is denoted $C_{L_\infty, S(L_0)}(T)$ in \textit{loc.\@ cit.\@}).
\end{proof}


Fix a representative $[\Pi \xrightarrow{\psi} \Pi]$ of $C_\infty^\bullet$ where $\Pi$ is a free $\bLambda$-module of rank $d$. Then for any given $n \in \NN_0$, the complex $C_n^\bullet$ is represented by $[\Pi_n \xrightarrow{\psi_n} \Pi_n]$ where we write $\Pi_n := \Pi \otimes_{\bLambda} \cR[\cG_n]$ and similarly for $\psi_n$. In particular, we have short exact sequences
\begin{align}\label{yoneda-extension-sequence}
& \begin{tikzcd}[ampersand replacement=\&]
0 \arrow{r} \& H^1_\Sigma(\cO_{L_n, S}, T) \arrow{r}{\phi_n} \& \Pi_n \arrow{r}{\psi_n}  \& \Pi_n.
\end{tikzcd} \\
\label{yoneda-extension-sequence-Iwasawa}
& \begin{tikzcd}[ampersand replacement=\&]
0 \arrow{r} \& H^1_{\Sigma, \; \Iw}(\cO_{L, S}, T) \arrow{r}{\phi} \& \Pi \arrow{r}{\psi}  \& \Pi.
\end{tikzcd}
\end{align}




\subsection{The structure of universal norms in higher rank}


In this section we prove a number of basic results about higher rank universal norms. In doing so, we will heavily rely on the notion of \textit{exterior biduals}. The theory of these objects is reviewed in Appendix \ref{AppendixBiduals}, and we shall use these results freely throughout this article. 

\paragraph{Definition of universal norms}

Given integers $m \geq n \geq 1$ we have the corestriction maps
\begin{align*}
    \cores_{m,n} \: H^1_\Sigma(\cO_{L_m, S}, T) \to H^1_\Sigma(\cO_{L_n, S}, T).
\end{align*}
If $r \geq 0$, then these maps induce natural maps on the exterior biduals
\begin{align*}
    \cores_{m,n}^r \: \bidual_{\cR[\cG_m]}^r H^1_\Sigma(\cO_{L_m, S}, T) \to \bidual_{\cR[\cG_n]}^r H^1_\Sigma(\cO_{L_n, S}, T).
\end{align*}

\begin{definition}\label{UN-definition}
    Fix integers $r \in \N$ and $n \in \NN_0$. 
    \begin{liste}
    \item We define the module of \emph{universal norms} of rank $r$ and level $n$ for $T$ to be
    \begin{align*}
        \UN^r_n = \UN^r_n(T,L_\infty) := \bigcap_{m \geq n} \im(\cores_{m,n}^r)
    \end{align*}
    We remark that $\UN^r_n(T,L_\infty)$ can be naturally regarded as an $\cR[\cG_n]$-module.
    \item We define the module of \emph{norm-coherent sequences} of rank $r$ for $T$ to be
    \begin{align*}
        \NS^r = \NS^r(T,L_\infty) := \varprojlim_{n \in \NN} \bidual_{\cR[\cG_n]}^r H^1_\Sigma(\cO_{L_n, S}, T)
        = \bidual^r_\bLambda H^1_{\Sigma, \text{Iw}} ( \cO_{L, S}, T)
    \end{align*}
    where the inverse limit is taken with respect to the maps $\cores_{m,n}^r$ and we have used Lemma \ref{BidualsLimits} for the last identification. We remark that $\NS^r(T,L_\infty)$ can be naturally regarded as a $\bLambda$-module.
    \end{liste}
\end{definition}


\paragraph{The descent isomorphism}
The following is one of the main results of this article. 


\begin{thm} \label{UN-structure-theorem}
Fix an integer $1 \leq r \leq r_T$. 
\begin{liste}
\item The module $\bidual^r_\bLambda H^1_{\Sigma, \text{Iw}} ( \cO_{L, S}, T)$ is $\Lambda$-free of rank $[L : K] \cdot \binom{r_T}{r}$.
\item{The natural map
            \begin{align*}
                \NS^r \to \bidual_{\cR[\cG_n]}^r H^1_\Sigma(\cO_{L_n, S}, T)
            \end{align*}
            induces an isomorphism of $\cR[\cG_n]$-modules
            \begin{align*}
                \NS^r \otimes_{\bLambda} \cR[\cG_n] \cong \UN_n^r.
            \end{align*}
            In particular, $\UN_n^{r}$ is a free $\cR[\Gamma_n]$-module of rank $[L:K]\cdot{r_T \choose r}$.
        }
	\item There is a natural identification
	\[
	\UN^r_n = \bidual^r_{\cR [\cG_n]} \UN^1_n.
	\]
\end{liste}
\end{thm}

\begin{rk} \label{structure-theorem-remark}\text{}
\begin{liste}
\item In the case of $r_T = 1$, universal norms have previously been studied by many authors: The first to obtain a result similar to Theorem \ref{UN-structure-theorem} (for $T = \Z_p (1)$) was Kuz'min \cite{Kuzmin},
later Greither \cite{Greither} also gave a proof in the abstract setting of a system of Galois modules satisfying certain natural axioms. The article \cite{kato2006universal} considers the non-commutative case but also gives an overview of the classical theory that is similar in spirit to our treatment. 
In the setting of elliptic curves a similar result is due to Mazur and Rubin \cite[Theorem 4.2]{MazurRubin03}.
\item Suppose $p \nmid [L:K]$. Then it is well-known that any $\bLambda$-module that is $\Lambda$-projective is necessarily $\bLambda$-projective (see, for example, \cite[Lem. 5.4.16]{NSW}). Since $\bLambda$ is a semi-local ring in this case, and $\NS^{r}$ has constant local rank ${r_T \choose r}$ by the calculation of the proof below, it follows that $\NS^{r}$ is necessarily $\bLambda$-free of rank ${r_T \choose r}$. An analogous statement for universal norms now also follows by codescent. 
\end{liste}
\end{rk}

\begin{comment}
\paragraph{A criterion for $\Lambda$-freeness}
In order to prove Theorem \ref{UN-structure-theorem} we shall use the following criterion \cite[Prop. 1.3]{belliard} of Belliard. For the convenience of the reader we reproduce the statement of his criterion in English.

\begin{proposition}[Belliard]\label{belliard-criterion}
    We suppose to be given, for each $n \in \N$, a pair of $\cR[\Gamma_n]$-modules $M_n \subseteq L_n$. We assume that the $L_n$ come equipped, for each $m \geq n$, with a map $N_{m,n} \: L_m \to L_n$ and a map $\nu_{n,m} \: L_n \to L_m$ with the following properties:
    \begin{enumerate}[label=(\arabic*)]
        \item{$N_{m,n}$ restricts to give a map $N_{m,n} \: M_m \to M_n$ and similarly for $\nu_{n,m}$;}
        \item{The composite $\nu_{n,m} \circ N_{m,n}$ coincides with multiplication by $\sum_{\sigma \in H_{m,n}} \sigma \in \cR[\Gamma_m]$;}
        \item{The composite $N_{m,n} \circ \nu_{n,m}$ coincides with multiplication by $p^{m-n}$;}
        \item{The map $\nu_{n,m} \: L_n \to L_m^{H_{m,n}}$ is injective and, for large enough $n$, induces an equality when restricted to $M_n$.}
    \end{enumerate}
    Then $\varprojlim_n M_n$ is $\Lambda$-free as soon as $\varprojlim_n L_n$ is.
\end{proposition}

We note that the original result is stated for modules over $\ZZ_p \llbracket \Gamma \rrbracket$ but the proof of the more general statement for modules over $R$ follows easily by replacing every instance of $\ZZ_p$ in Belliard's argument by $\cR$.\\
\end{comment}


\textit{Proof of Theorem \ref{UN-structure-theorem}:}
Consider the complex $D^\bullet$ represented by 
\begin{cdiagram}
\exprod^r_\bLambda \Pi \arrow{r}{\psi} & \Pi \otimes_\bLambda \exprod^{r - 1}_\bLambda \Pi.
\end{cdiagram}
By virtue of Lemma \ref{LittleLemma} (a) we have that $H^0 ( D^\bullet) = \bidual_{\bLambda}^{r} H^1_{\Sigma, \Iw}(\cO_{L, S}, T)$. In particular, $\left ( \bidual_{\bLambda}^{r} H^1_{ \Sigma, \Iw}(\cO_{L, S}, T) \right)^\Gamma  = 0$ since $\exprod^r_\bLambda \Pi$ is $\Lambda$-free. \\
Moreover, the complex $D^\bullet$ is clearly perfect and the complex $D_n^\bullet = D^\bullet \otimes_\bLambda^\mathbb{L} \cR [\cG_n]$ is represented by 
\begin{cdiagram}
\exprod^r_{\cR [\cG_n]} \Pi_n \arrow{r}{\psi_n} & \Pi_n \otimes_{\cR [\cG_n]} \exprod^{r - 1}_{\cR [\cG_n]} \Pi_n
\end{cdiagram}
which has $H^0 ( D^\bullet_n) = \bidual_{\cR[\cG_n]}^{r} H^1_\Sigma(\cO_{L_n, S}, T)$ as its cohomology in its lowest degree.

Now fix a topological generator $\gamma_n$ of $\Gamma^n$. Then the decomposition $\bLambda = \Lambda[\cG]$ implies that there is an exact sequence
\begin{cdiagram}
    0 \arrow{r} & \bLambda \arrow{r}{\cdot (1-\gamma_n)} \arrow{r} & \bLambda \arrow{r} &  \cR[\cG_n] \to 0 .
\end{cdiagram}
From this it follows that for any $\bLambda$-module $M$ and $i \geq 2$, the module $\Tor_i^\bLambda(\cR[\cG_n], M)$ vanishes.
Since the complex $D^\bullet$ is acyclic outside degrees zero and one, we then deduce that the spectral sequence
\begin{equation} \label{spectral-sequence}
E^{i , j}_2 = \Tor_{ - i}^\bLambda ( \cR [\cG_n], \; H^j (D^\bullet)) \; \Rightarrow \; E^{i + j} = H^{i + j} ( D^\bullet \otimes_{\bLambda}^\mathbb{L} \cR [\cG_n] ) 
\end{equation}
degenerates on its second page into a collection of short exact sequences. In particular, there is an injection
\begin{equation}\label{bidual-injection}
\left ( \bidual_{\bLambda}^{r} H^1_{\Sigma, \Iw}(\cO_{L, S}, T) \right) \otimes_\bLambda \cR [\cG_n] \hookrightarrow \bidual^r_{\cR [\cG_n]} H^1_\Sigma (\bigO_{L_n, S}, T)
\end{equation}
 from which one sees that the coinvariants module $\left ( \bidual_{\bLambda}^{r} H^1_{\Sigma, \Iw}(\cO_{L, S}, T) \right)_\Gamma$ is $\cR$-free. This implies that $\bidual_{\bLambda}^{r} H^1_{\Sigma, \Iw}(\cO_{L, S}, T)$ is $\Lambda$-free (see, for example, \cite[Proposition 5.3.19]{NSW}). 

To prove Part (a) of Theorem \ref{UN-structure-theorem} it now remains to demonstrate that the $\Lambda$-rank of $\NS^{r}$ is $[L:K]\cdot{r_T \choose r}$. 
Since $Q(\bLambda)$ is semi-simple,
an analysis of the exact sequence (\ref{yoneda-extension-sequence-H2}) implies that
\begin{align*}
\rank_{Q(\bLambda)}(Q(\bLambda) \otimes_{\bLambda} H^1(C^\bullet_\infty)) & = 
    \rank_{Q(\bLambda)}(Q(\bLambda) \otimes_{\bLambda} H^1_{ \Sigma, \Iw}(\cO_{L, S}, T))  \\
    & \phantom{=} + \rank_{Q(\bLambda)}(Q(\bLambda) \otimes_{\bLambda} \varprojlim_{n \in \N} (Y_K(T) \otimes_\cR \cR[\cG_n])).
\end{align*}
By the assumed validity of the weak Leopoldt conjecture, one has an equality of ranks
\begin{align*}
    \rank_{Q(\bLambda)}(Q(\bLambda) \otimes_{\bLambda} H^1_{\Iw, \Sigma}(\cO_{L, S(L)}, T)) &= \rank_{Q(\bLambda)}(Q(\bLambda) \otimes_{\bLambda} \varprojlim_n (Y_K(T) \otimes_\cR \cR[\cG_n])) = r_T.
\end{align*}
We may thus calculate
\begin{align*}
    \rank_{Q(\Lambda)}(Q(\Lambda) \otimes_\Lambda \NS^{r}) &= [L:K]\cdot \rank_{Q(\bLambda)}(Q(\bLambda) \otimes_{\bLambda} \NS^{r})\\
    &= [L:K] \cdot \rank_{Q(\bLambda)}\left(Q(\bLambda) \otimes_{\bLambda} \bidual_{\bLambda}^{r} H^1_{\Sigma, \Iw}(\cO_{L, S}, T)\right)\\
    &= [L:K]\cdot \rank_{Q(\bLambda)}\left(\exprod_{Q(\bLambda)}^{r} Q(\bLambda) \otimes_{\bLambda} H^1_{\Sigma, \Iw}(\cO_{L, S}, T)\right)\\
    &= [L:K]\cdot {r_T \choose r}.
\end{align*}

Before we proceed with the proof of Part (2) we first require the following Lemma which is presumably well-known but for which we include the proof for lack of a better reference.

\begin{lemma} \label{compactness-argument}
    Let $M = (M_n, \phi_{m,n})$ be an inverse system in the category of compact Hausdorff spaces with limit $M_\infty$. Fix $n \in \NN_0$ and write
    \begin{align*}
        \UN_n(M) := \bigcap_{m \geq n} \im(\phi_{m,n}).
    \end{align*}
    Then the natural map $M_\infty \to M_n$ has image $\UN_n(M)$.
\end{lemma}


\begin{proof}
    Without loss of generality (and for notational simplicity) we prove the statement for $n = 0$.\\
    Suppose to be given an element $u \in \UN_0(M)$. We shall inductively construct an element $m \in M_\infty$ with the property that $m_0 = u$. Suppose that for some $s \geq 1$ we have constructed a coherent tuple $(m_i)_{0 \leq i \leq s}$ such that $m_0 = u$ and each $m_0 \in \UN_i(M)$. For $j \geq s$ define the sets
    \begin{align*}
        X_s = \phi_{s+1,s}^{-1}(m_s), \quad Y_{j,s} = \im(\phi_{j,s+1})
    \end{align*}
    Then both $X_s$ and $Y_{j,s}$ are closed. Indeed, the former is so by the fact that $\phi_{s+1,s}$ is continuous and the latter from the fact that $Y_{j,s}$ is a compact subspace of a Hausdorff space.\\
    It then follows that the intersection $X_s \cap Y_{j,s}$ is also closed and, by hypothesis, non-empty. In particular, the descending filtration $(X_s \cap Y_{j,s})_{j > s}$ satisfies the finite intersection property. Since $M_{s+1}$ is compact we then have that the intersection $\bigcap_{j > s} X_s \cap Y_{j,s}$
    is non-empty. We can therefore take $m_{s+1}$ to be any element of this intersection. Continuing in this fashion we may inductively construct an element $m = (m_i)$ of $M_\infty$ with the desired property. 
\end{proof}


Returning now to the proof of Theorem \ref{UN-structure-theorem} (b), we note that the augmentation ideal in $\bLambda$ relative to $\cG_n$ applied to $\NS^r$ is contained in the kernel of the natural map \begin{align*}
    \NS^r \to \bidual_{\cR[\cG_n]}^r H^1_\Sigma(\cO_{L_n, S}, T)
\end{align*}
Given this, we may apply the above Lemma \ref{compactness-argument} to conclude that for all $r \in \NN$ and $n \in \N_0$ the induced map
\begin{align*}
    \NS^{r} \otimes_\bLambda \cR[\cG_n] \longrightarrow \bidual_{\cR[\cG_n]}^{r} H^1_\Sigma(\cO_{L_n, S}, T).
\end{align*}
has image $\UN_{n}^{r}$.
On the other hand, the map of (\ref{bidual-injection}) clearly factors through this homomorphism and so it is in fact an isomorphism. This establishes part (b) of the Theorem.

Finally, for part (c) we first note that by applying Lemma \ref{RyotarosLemma1} to the inclusion $\UN_n^1 \subseteq H^1_\Sigma(\cO_{L_n,S},T)$ one obtains a natural injection 
\[
\bidual^r_{\cR [\cG_n]} \UN^1_n \hookrightarrow \bidual^r_{\cR [\cG_n]} H^1_\Sigma (\bigO_{L_n, S}, T).
\]
Moreover, it is clear from the definitions that this map has image inside $\UN^r_n$. By Lemma \ref{BaseChangeLem}, the codescent map
\[
\left ( \bidual^r_\bLambda H^1_{\Sigma, \Iw}(\cO_{L, S}, T) \right ) \otimes_\bLambda \cR [\cG_n] \to \bidual^r_{\cR [\cG_n]} H^1_\Sigma (\bigO_{L_n, S}, T)
\]
has image inside $\bidual^r_{\cR [\cG_n]} \UN^1_n$. This combines with part (b) to imply that $\bidual^r_{\cR [\cG_n]} \UN^1_n = \UN^r_n$. 
\qed



%%%%%
%%%%Pairing
%%%%%%%%%%%%%
\subsection{An Iwasawa-theoretic pairing}

\paragraph{Basic Norms}

At the outset of this section we define the following projection map:
\begin{align*}
    \Theta \: \Det_{\bLambda}(C^\bullet_\infty) &\hookrightarrow Q(\bLambda) \otimes_{\bLambda} \Det_{\bLambda}(C^ \bullet_\infty)\\
    &\xrightarrow{\simeq} \Det_{Q(\bLambda)}(Q(\bLambda) \otimes_{\bLambda} C^\bullet_\infty)\\
    &\xrightarrow{\simeq} Q(\bLambda) \otimes_{\bLambda} \bigg(\exprod_{\bLambda}^{r_T} H^1_{\Sigma, \Iw}(\cO_{L,S}, T) \otimes_{\bLambda} \Big(\exprod_{\bLambda}^{r_T} \varprojlim_n (Y_K(T)^* \otimes_\cR \cR[\cG_n])\Big)^*\bigg)\\
    &\xrightarrow{\simeq} Q(\bLambda) \otimes_{\bLambda} \exprod_{\bLambda}^{r_T} H^1_{\Sigma, \Iw}(\cO_{L,S}, T),
\end{align*}
where the second arrow follows from the base-change property of determinant functors, the third follows by passing to cohomology and noting that the weak Leopoldt conjecture is assumed to hold, and the final arrow follows from collapsing the exterior power with respect to a fixed $\bLambda$-basis of the inverse limit.

We now have the following Lemma which will prove useful in the sequel.

\begin{lemma} 
\label{projection-map-lemma-1}
   The image of $\Theta$ is contained in $\NS^{r_T}$.
\end{lemma}

\begin{proof}
   Recall that in Lemma \ref{standard-representative-lemma} we have chosen a quadratic standard representative $[\Pi \stackrel{\psi}{\longrightarrow} \Pi]$ of the complex $C^\bullet_\infty$ and thereby also fixed a basis $\{ b_1, \dots, b_d \}$ of $\Pi$. 
    By applying \cite[Lem.\@ A.7]{EulerSystemsSagaI} to the complex $Q (\bLambda) \otimes_\bLambda^\mathbb{L} C^\bullet_\infty$  we then see that the projection map $\Theta$ coincides with the rank reduction map
    \[
    \pi_{\psi} = (-1)^{r_T (d - r_T)} \cdot \bigwedge_{r_T < i \leq d} (\psi \circ b_i^\ast) \: \exprod^{d}_\bLambda \Pi \to \exprod^{r_T}_\bLambda \Pi,
    \]
    where each $b_i^\ast \in \Pi^\ast$ denotes the dual of $b_i$.
    By Lemma \ref{LittleLemma}, the image of $\pi_\psi$
    is contained in $\bidual^{r_T}_\bLambda H^1_{\Sigma, \Iw} (\bigO_{L, S}, T)$.
\end{proof}

 
\begin{definition}
We define the $\bLambda$-module of \emph{basic norm-coherent sequences} $\NS^b = \NS^b(T, L_\infty)$ to be the image of the homomorphism
    \begin{align*}
       \Theta \: \Det_{\bLambda}(C^\bullet_\infty) \longrightarrow \bidual_{\bLambda}^{r_T} H^1_{\Sigma, \Iw}(\cO_{L,S}, T) .
    \end{align*} 
\end{definition}


\begin{proposition}\label{basic-theorem}\text{}
 The $\bLambda$-submodule $\NS^b = \NS^b(T,L_\infty)$ of $\NS^{r_T}$ is $\bLambda$-free of rank one. In particular, the quotient $\faktor{\NS^{r_T}}{\NS^b}$ is $\bLambda$-torsion.
\end{proposition}

\begin{proofbox}
It suffices to observe that $\Det_{\bLambda}(C^\bullet_\infty)$, and thus $\NS^b$, is a free $\bLambda$-module of rank one. In particular, both $\NS^{r_T}$ and $\NS^b$ are free $\Lambda$-modules of rank $[L:K]$ and so their quotient is $\bLambda$-torsion.
\end{proofbox}

The following result shows that, at least conjecturally, the elements of $\NS^b(\ZZ_p(1), L_\infty)$ are familiar objects.


\begin{proposition}\label{eimc-proposition}
    Assume that for each $n \in \NN$ the $p$-part of the Rubin-Stark conjecture is valid for the data $(L_n | K, S, \Sigma, S_\infty(K))$ (as is formulated, for example, in \cite[Conj. 2.1]{BKS2}) and let $\varepsilon_n$ be the corresponding Rubin-Stark element. Assume, moreover, that the equivariant Iwasawa Main Conjecture (eIMC) is valid for the data $(L_\infty | K, S, \Sigma, p)$ (as is formulated in \cite[Conj. 3.1]{BKS2}). Then
    \begin{align*}
        \NS^b(\ZZ_p(1), L_\infty) = \langle(\varepsilon_n)_n\rangle_{\bLambda}.
    \end{align*}
\end{proposition}

\begin{proofbox}
    It is well-known that the family $(\varepsilon_n)_n$ constitutes an element of the module\\
    $\NS^{r_T}(\ZZ_p(1), L_\infty)$, see \cite[Prop.\@ 6.1]{Rubin96}. Denote by
    \begin{align*}
        \mathfrak{z}_\infty \in Q(\bLambda) \otimes_{\bLambda} \Det_{\bLambda}(C_\infty)
    \end{align*}
    the inverse image of $(\varepsilon_n)_n$ under the latter three arrows in the definition of $\Theta$. Then, after taking into account the equivalent formulation \cite[Conj. 3.7]{BKS2} of the eIMC, one knows that $\mathfrak{z}_\infty$ is a $\bLambda$-basis of $\Det_{\bLambda}(C^\bullet_\infty)$. The Proposition now follows immediately from the definition of the module $\NS^b$.
\end{proofbox}




The following Theorem provides a direct link between the Galois module structures of the quotient appearing in Theorem \ref{basic-theorem} and $H^2_{\Sigma,\Iw}(\cO_{L, S}, T)$.

\begin{thm}\label{pairing-theorem}\text{}
    \begin{liste}
        \item{There exists a canonical isomorphism of $\bLambda$-modules
            \begin{align*}
                \Ext^1_{\bLambda}  \left ( \faktor{\NS^{r_T}}{\NS^b}, \; \bLambda \right ) \cong \faktor{\bLambda}{\Fitt^0_{\bLambda}(H^2_{\Sigma,\Iw}(\cO_{L,S}, T))}.
            \end{align*}
        }
        \item{There exists a perfect pairing of $\bLambda$-modules
            \begin{align*}
                \faktor{\NS^{r_T}}{\NS^b} \times \faktor{\bLambda}{\Fitt^0_{\bLambda}(H^2_{\Sigma,\Iw}(\cO_{L,S}, T))} \to \faktor{Q(\bLambda)}{\bLambda}
            \end{align*}
            which is explicitly given by the assignment $(u,v) \mapsto \overline{v}\cdot\eta^*(\overline{u})$, where $\eta$ is any choice of $\bLambda$-basis of $\NS^b$ and $\overline{u}$ and $\overline{v}$ are any lifts of $u$ and $v$ to $\NS^{r_T}$ and $\bLambda$, respectively.
        }
        \item{There exists a pseudo-isomorphism of $\Lambda$-modules
            \begin{align*}
                \faktor{\NS^{r_T}}{\NS^b} \approx \left ( \faktor{\bLambda}{\Fitt_{\bLambda}(H^2_{\Sigma,\Iw}(\cO_{L,S}, T))} \right)^\circ.
            \end{align*}
            If $p \nmid [L:K]$ then this can be taken to be a pseudo-isomorphism of $\bLambda$-modules where $\circ$ now also inverts the $\cG$-action.
        }
    \end{liste}
\end{thm}

\begin{rk}
The existence of pairings of the displayed shape in part (2) of Theorem \ref{pairing-theorem} was first observed (at least in the case of representations with coefficients in Gorenstein orders in finite-dimensional $\cQ$-algebras) by Burns, Sano and Tsoi in \cite[]{bst}. The above result can therefore be seen as a natural Iwasawa-theoretic analogue of the pairing constructed for $T$ by the aforementioned authors.
Similar Iwasawa-theoretic results in the setting of $K =\Q$ have previously appeared in \cite[Prop. 2.12]{NQD14}, which in turn is based on \cite{KraftSchoof}, and \cite[Thm. 4]{Solomon2014}.
\end{rk}

\textit{Proof of Theorem \ref{pairing-theorem}:}
    By applying Lemma \ref{LittleLemma} (c) to the representative of $C_\infty^\bullet$ given by Lemma \ref{standard-representative-lemma}, one has
    \begin{align*}
        \mathrm{Fitt}^0_{\bLambda}(H^2_{\Sigma,\mathrm{Iw}}(\mathcal{O}_{L,S}, T)) & = \mathrm{Fitt}^{r_T}_{\bLambda}(H^1 (C^\bullet_\infty))
        = \left\{\phi(\eta) \mid \phi \in \exprod_{\bLambda}^{r_T} (H^1_{\Sigma, \Iw}(\cO_{L,S}, T))^*\right\} \\
        & =: I(\eta),
    \end{align*}
    where $\eta$ is any $\bLambda$-basis of $\NS^b$.
    On the other hand, by the definition of $I(\eta)$ we have an equality
    \begin{align*}
        \left(\exprod_{\bLambda}^{r_T} H^1_{\Sigma, \Iw}(\cO_{L,S}, T)^*\right)(\eta) = I(\eta)
    \end{align*}
    and so the map which sends $\phi$ to $\phi(\eta)$ induces an isomorphism
    \begin{align*}
        \left(\bidual_\bLambda^{r_T} H^1_{\Sigma, \Iw}(\cO_{L,S}, T)\right)^* \cong I(\eta).
    \end{align*}
    There is a $\bLambda$-module isomorphism (see (\ref{Exts}))
    \begin{align*}
        \Ext^1_\bLambda\left(\bidual_\bLambda^r H^1_{ \Sigma, \Iw}(\cO_{L,S}, T), \; \bLambda\right)^\# \cong \Ext_{\Lambda}^1\left(\bidual_\bLambda^r H^1_{ \Sigma, \Iw}(\cO_{L,S}, \; T), \Lambda\right),
    \end{align*}
    and so Theorem \ref{UN-structure-theorem} (a) implies that the above module is trivial. One then deduces the existence of a commutative diagram
    \begin{center}
        \begin{tikzcd}
            0 \arrow[r] &\displaystyle\left(\bidual_\bLambda^{r_T} H^1_{ \Sigma, \Iw}(\cO_{L,S}, T)\right)^* \arrow[r] \arrow[d] &
            ( \bLambda \eta )^* \arrow[r] \arrow[d] &\Ext^1_\bLambda \left  (\faktor{\NS^{r_T}}{\NS^b}, \bLambda \right) \arrow[r] \arrow[d] & 0\\
            0 \arrow[r] &I(\eta) \arrow[r] &\bLambda \arrow[r] &\faktor{\bLambda}{I(\eta)} \arrow[r] &0
        \end{tikzcd}
    \end{center}
    The isomorphism given in (a) now follows via an application of the Five-Lemma. \\
    
    To prove (c), note that $\NS^{r_T}/\NS^b$ has projective dimension 1 as a $\Lambda$-module and thus has no non-zero finite submodules (see, for example, \cite[Prop. 5.3.19]{NSW}). The pseudo-isomorphism of the Theorem is then obtained by virtue of \cite[Prop. 5.5.13]{NSW}.\\
    
   It remains to demonstrate the existence of the pairing in (b). Observe that if $M$ is $\bLambda$-torsion, then by applying the functor $\Hom_\bLambda(M,-)$ to the tautological sequence $0 \to \bLambda \to Q(\bLambda) \to Q(\bLambda)/\bLambda \to 0$ one finds a canonical identification $M^\lor \cong \Ext_\bLambda^1(M, \bLambda)$. In addition, one knows by 
   \cite[Prop. 5.5.8(iv)]{NSW} and the isomorphism (\ref{Exts})
   that
    \begin{align*}
        \left (\faktor{\bLambda}{I(\eta)}\right )^\lor \cong \left ( \faktor{\NS^{r_T}}{\NS^b}\right)^{\lor\lor} \cong \faktor{\NS^{r_T}}{\NS^b}.
    \end{align*}
    These two facts taken together establish both the existence and the perfectness of the desired pairing. Since the quotient $\NS^{r_T}/\NS^b$ is $\bLambda$-torsion, we may regard $\eta^*$ as being an element of $\NS^{r_T,*} \otimes_\bLambda Q(\bLambda)$. A straightforward calculation then shows that one can in fact regard this as an element of $(\NS^{r_T}/\NS^b)^\lor$ from which one deduces the given explicit description of the pairing.
\qed

\subsection{Results on finite level}\label{finite-level-section}

In analogy to the Iwasawa-theoretic definition of basic norm coherent sequences, it is natural to make the following corresponding definition on finite level.

\begin{definition}
For each $n \in \NN$, we define the $\cR[\cG_n]$-module of \emph{basic universal norms} $\UN^b_n = \UN^b_n(T,L_\infty)$ to be the image of $\NS^b(T,L_\infty)$ under the map of Theorem \ref{UN-structure-theorem} (b).
\end{definition}

There is another, equivalent, way of constructing the module of basic universal norms that is closer in spirit to the definition of basic norm-coherent sequences. In order to explain this, we let $e_{L_n, T} \in \mathcal{Q}[\cG_n]$ be the sum of the primitive idempotents that annihilate $H^2_\Sigma (\bigO_{L_n, S}, T)$. In this regard we remark that Jannsen has conjectured in \cite[Conj.\@ 1]{jannsen} that this module should be finite in all but a few exceptional cases. We then have the projection map
\begin{align*}
    \Theta_{L_n} \: \Det_{\cR [\cG_n]} (C^\bullet_n) & \hookrightarrow
    \mathcal{Q} \otimes_{\cR} \Det_{\cR [\cG_n]} (C^\bullet_n) \\
    & \stackrel{\simeq}{\longrightarrow} \Det_{\mathcal{Q}[\cG_n]} ( \mathcal{Q} \otimes_{\cR} C^\bullet_n) \\
    & \stackrel{\simeq}{\longrightarrow} \Det_{\mathcal{Q} [\cG_n]} ( \mathcal{Q} \otimes_\cR H^0 (C^\bullet_n)) \otimes_{\cQ [\cG_n]} ( \Det_{\cQ [\cG_n]} ( \cQ \otimes_{\cR} H^1 (C^\bullet_n)))^{-1} \\
    & \stackrel{\cdot e_{L_n, T}}{\longrightarrow} 
    e_{L_n, T} \Big ( \Det_{\mathcal{Q} [\cG_n]} ( \mathcal{Q} \otimes_\cR H^0 (C^\bullet_n)) \otimes_{\cQ [\cG_n]}  ( \Det_{\cQ [\cG_n]} ( \cQ \otimes_{\cR} H^1 (C^\bullet_n)))^{-1} \Big ) \\
    & \stackrel{\simeq}{\longrightarrow} e_{L_n, T}  \Big ( \exprod^{r_T}_{\cQ [\cG_n]} \cQ \otimes_\cR H^1_\Sigma (\bigO_{L, S}, T) \Big )
    \otimes_{\cQ [\cG_n]}  \exprod^{r_T}_{\cQ [\cG_n]} ( Y_K (T) \otimes_\cR \cQ [\cG_n])^\ast \\
    & \stackrel{\simeq}{\longrightarrow} 
    e_{L_n, T}  \Big ( \exprod^{r_T}_{\cQ [\cG_n]} \cQ \otimes_\cR H^1_\Sigma (\bigO_{L, S}, T) \Big ),
\end{align*}
where the second arrow follows from the base-change property of determinant functors, the third from the natural passage-to-cohomology map, the fourth by multiplication by the idempotent $e_{L_n, T}$, and the final one by applying the (non-canonical) isomorphism $ \exprod^{r_T}_{\cQ [\cG_n]} ( Y_K (T) \otimes_\cR \cQ [\cG_n])^\ast \cong \cQ [\cG_n]$ resulting from the fact that $Y_K (T) \otimes_\cR \cR [\cG_n]$ is a free $\cR [G_n]$-module of rank $r_T$. 

\begin{lem} \label{FiniteLem}
The image of the map $\Theta_{L_n}$ is contained in $\left( \bidual^{r_T}_{\cR [\cG_n]} H^1_\Sigma (\bigO_{L_n, S}, T) \right ) [1 - e_{L_n, T}]$, and coincides with $\UN^b_n$.
\end{lem}

\begin{proofbox}
Since the complex $C^\bullet_{L_n}$ admits a standard quadratic representative with respect to the map $H^1(C^\bullet_{L_n}) \to Y_K(T)^* \otimes_\cR \cR[\cG_n]$, the first claim is exactly \cite[Prop. A.7 (i)]{EulerSystemsSagaI}.
As for the second claim, we observe that the descriptions of $\Theta$ and $\Theta_{L_n}$ in terms of rank reduction maps $\pi_{\psi_\infty}$ and $\pi_{\psi_n}$, respectively, yield 
a commutative diagram
\begin{cdiagram}
\Det_\bLambda ( C^\bullet_\infty) \arrow{r}{\Theta} \arrow[twoheadrightarrow]{d} & 
\bidual^{r_T}_\bLambda H^1_{\Sigma, \Iw} (\bigO_{L, S}, T) \arrow{d} \\
\Det_{\cR [\cG_n]} ( C^\bullet_n) \arrow{r}{\Theta_{L_n}} &  
\bidual^{r_T}_{\cR [\cG_n]} H^1_\Sigma (\bigO_{L_n, S}, T),
\end{cdiagram}
where the vertical arrows are the natural codescent maps.
\end{proofbox}

It turns out that the module of basic universal norms may not be very interesting at the bottom layers of the extension $L_\infty$. We remark however that the behaviour explicated in the next Lemma will always stop if one climbs high enough up the tower due to our assumption that no finite place splits completely in $L_\infty | K$. 

\begin{lem}
Suppose there is a finite place $v \in S$ that splits completely in $L_n | K$ and is such that $H^0 (K_v, T^\ast (1))$ is non-zero. Then $\UN^b_n = 0$.
\end{lem}

\begin{proofbox}
We have a commutative diagram
\begin{cdiagram}
\NS^{r_T} \arrow[hookrightarrow]{r} \arrow{d} & \exprod^{r_T}_\bLambda \Pi \arrow{d} & \\
\UN^{r_T}_n \arrow[hookrightarrow]{r} & \exprod^{r_T}_{\cR [\cG_n]} \Pi_n,
\end{cdiagram}
where the vertical maps are the natural codescent maps. It therefore suffices to demonstrate that any basis $\eta$ of $\NS^b$ is contained in $I_{\Gamma^n} \cdot \exprod^{r_T}_{\cR [\cG_n]} \Pi$, where we write
\[
I_{\Gamma^n} = \ker \{ \bLambda \to \cR [\Gamma_n] \} 
\]
for the augmentation ideal relative to $\Gamma^n$. From \cite[Prop. A.2 (ii)]{EulerSystemsSagaI} we know that for any $f \in \exprod^{r_T}_\bLambda \Pi^\ast$ we have 
\[
f (\eta) \in \Fitt_\bLambda^{r_T} (H^1 (C^\bullet_\infty)) = \Fitt^0_\bLambda ( H^2_{\Sigma, \Iw} ( \bigO_{L, S}, T)). 
\]
If $b_1, \dots, b_d \in \Pi$ constitutes a $\bLambda$-basis, then for any $\sigma \in \mathfrak{S}_{d, r_T}$ (see the definiton following (\ref{ExplicitFormula})) the inclusion
\[
(b_{\sigma (1)}^\ast \wedge \dots \wedge b_{\sigma({r_T})}^\ast ) \in \Fitt^0_\bLambda ( H^2_{\Sigma, \Iw} ( \bigO_{L, S}, T)) 
\]
holds. 
Since $\{ \bigwedge_{1 \leq i \leq r_T} b_{\sigma (i)} \mid \sigma \in \mathfrak{S}_{d, r_T} \}$ constitutes a basis of $\exprod^{r_T}_\bLambda \Pi$, we conclude that
\[
\eta \in \Fitt^0_\bLambda ( H^2_{\Sigma, \Iw} ( \bigO_{L, S}, T)) \cdot \exprod^{r_T}_\bLambda \Pi.
\]
We are therefore reduced to showing that $\Fitt^0_\bLambda ( H^2_{\Sigma, \Iw} ( \bigO_{L, S}, T)) \subseteq I_{\Gamma^n}$. By Proposition \ref{SplitPrimesProp} there is a surjection
\[
H^2_{\Sigma, \Iw} (\bigO_{L, S}, T) \twoheadrightarrow H^2_\Sigma (\bigO_{L_n, S}, T) \twoheadrightarrow H^0 (K_v, T^\ast (1)) \otimes_\cR \cR [\cG_n].
\]
Since we assumed $H^0 (K_v, T^\ast (1)) \neq 0$, the module $H^0 (K_v, T^\ast (1)) \otimes_\cR \cR [\cG_n]$ is $\cR [\cG_n]$-free of non-zero rank $t$, say. The Lemma therefore follows from the inclusion
\[
\Fitt^0_\bLambda ( H^2_{\Sigma, \Iw} ( \bigO_{L, S}, T)) \subseteq
\Fitt_\bLambda^0 ( H^0 (K_v, T^\ast (1)) \otimes_\cR \cR [\cG_n]) = \Fitt^0_\bLambda ( \cR [\cG_n]^{t} ) = I_{\Gamma^n}^t. \tag*{\qedhere} 
\]
\end{proofbox}

We next turn to a description of the quotient $\bidual^{r_T}_{\cR [\cG_n]} H^1_\Sigma (\bigO_{L, S}, T) / \UN^{r_T}_n$. This should be regarded as a complement to the study of the quotient $\bidual^{r_T}_{\cR [\cG_n]} H^1_\Sigma (\bigO_{L, S}, T) / \UN^{b}_n$ undertaken in \cite{bst}. 


\begin{proposition} \label{Dodgy-Proposition}
Assume that $p \nmid |\cG|$. 
\begin{liste}
\item We have an equality
\[
\Big\{ f (a) \mid a \in \UN^{r_T}_n, \; f \in \exprod^{r_T}_{\cR [\cG_n]} H^1_\Sigma ( \bigO_{L_n, S}, T)^\ast  \Big\} = \Fitt^0_{\cR [\cG_n]} \big ( H^2_{\Sigma, \Iw}( \bigO_{L, S}, T)^{\Gamma^n, \vee}_\tor \big ).
\]
\item The following holds:
\[
\left ( \faktor{\bidual^{r_T}_{\cR [\cG_n]} H^1_\Sigma ( \bigO_{L_n, S}, T)}{\UN^{r_T}_n} \right )_{\tor} \cong \left ( \faktor{\cR [\cG_n]}{\Fitt^0_{\cR [\cG_n]} ( H^2_{\Sigma, \Iw}( \bigO_{L, S}, T)^{\Gamma^n, \vee}_\tor )} \right)^\vee.
\]
\item There is an injection
\[
\left ( \faktor{\exprod^{r_T}_{\cR [\cG]} H^1_\Sigma ( \bigO_{L, S}, T)}{\UN^{r_T}_0} \right )_{\tf} \hookrightarrow ( H^2_{\Sigma, \Iw} (\bigO_{L, S}, T)^\Gamma)_\tf \otimes_{\cR [\cG]} \exprod^{r_T - 1}_{\cR [\cG]} H^1_\Sigma ( \bigO_{L, S}, T)
\]
that is induced by the boundary morphism
\[
\delta \: \left ( \faktor{
H^1_\Sigma (\bigO_{L, S}, T) 
}{\UN^1_0  } \right )_\tf
\stackrel{\simeq}{\longrightarrow} H^2_{\Sigma, \Iw} (\bigO_{L, S}, T)^\Gamma_\tf.
\]
\end{liste}
\end{proposition}


\begin{proofbox}
By truncating the exact sequence representing the complex $C^\bullet_\infty$ we obtain an exact sequence
\begin{cdiagram}
   0 \arrow{r} & H^1_{\Sigma, \Iw} ( \bigO_{L, S}, T) \arrow{r} & \Pi \arrow{r}{\psi} & \im \psi \arrow{r} & 0 .
\end{cdiagram}
Since $\im(\psi)$ has trivial $\Gamma^n$-invariants, this sequence descends to give an exact sequence
\begin{equation} \label{DescendedSequence}
\begin{tikzcd}
   0 \arrow{r} & \UN^1_n \arrow{r} & \Pi_n \arrow{r} & (\im \psi)_{\Gamma^n} \arrow{r} & 0.
\end{tikzcd}
\end{equation}
Dualising, and using the identification $\Ext^1_{\cR [\cG_n]} ( (\im \psi)_{\Gamma_n}, \cR [\cG_n]) \cong ( (\im \psi)_{\Gamma_n,\tor})^\vee$, we get the exact sequence
\begin{cdiagram}
\Pi_n^\ast \arrow{r} & (\UN^1_n)^\ast \arrow{r} & ((\im \psi)_{\Gamma_n, \tor} )^\vee \arrow{r} & 0.
\end{cdiagram}
We have observed in Remark \ref{structure-theorem-remark} (b) that the assumption $p \nmid |\cG |$ implies that $\UN^1_n$ is $\cR [\cG_n]$-free of rank $r_T$, so the above exact sequence is in fact a free presentation of $((\im \psi)_{\Gamma^n, \tor})^\vee$. This implies that
\begin{align*}
\im \big \{ \exprod^{r_T}_{\cR [\cG_n]} 
\Pi_n^\ast \to \exprod^{r_T}_{\cR [\cG_n]} (\UN^1_n )^\ast \cong \cR [\cG_n] \big \} 
& = \Fitt^0_{\cR [\cG_n]} \left ( (\im \psi)_{\Gamma^n, _\tor}^\vee \right).
\end{align*}
Combining this with the identification $\UN^{r_T}_n = \bidual^{r_T}_{\cR [\cG_n]} \UN^1_n$ from Theorem \ref{UN-structure-theorem} (c) then gives 
\[
\Big \{ f (a) \mid a \in \UN^{r_T}_n, \; f \in \exprod^{r_T}_{\cR [\cG_n]} \Pi_n^\ast \Big\} = \Fitt^{0}_{\cR [\cG_n]} ( (\im \psi)_{\Gamma^n, \tor}^\vee). 
\]
The exact sequence
\begin{cdiagram}
   0 \arrow{r} & \im \psi \arrow{r} & \Pi \arrow{r} & H^1 (C^\bullet_\infty ) \arrow{r} & 0
\end{cdiagram}
gives the exact sequence
\begin{cdiagram}
0 \arrow{r} & H^1 (C^\bullet_\infty)^{\Gamma^n} \arrow{r} & (\im \psi )_{\Gamma^n} \arrow{r} & \Pi_n.
\end{cdiagram}
Since $\Pi_n$ is torsion-free, we deduce that 
\[
(\im \psi)_{\Gamma^n, \tor} = H^1 (C^\bullet_\infty)^{\Gamma^n}_\tor = H^2_{\Sigma, \Iw}( \bigO_{L, S}, T)^{\Gamma^n}_\tor. 
\]

Finally, since the cokernel of $H^1_\Sigma (\bigO_{L_n, S}, T) \hookrightarrow \Pi_n$ is $\cR$-torsion free, the restriction map
$
\Pi_n^\ast \to H^1_\Sigma (\bigO_{L_n, S}, T)^\ast
$
is surjective. This shows that 
\[
\Big \{ f (a) \mid a \in \UN^{r_T}_n, \; f \in \exprod^{r_T}_{\cR [\cG_n]} \Pi_n^\ast \Big\}
= 
\Big\{ f (a) \mid a \in \UN^{r_T}_n, \; f \in \exprod^{r_T}_{\cR [\cG_n]} H^1_\Sigma (\bigO_{L_n, S}, T)^\ast \Big\}
\]
and concludes the proof of (a). For (b), let $C$ denote the cokernel of the map $\UN^{r_T}_n \to \bidual^{r_T}_{\cR [\cG_n]} H^1_\Sigma (\bigO_{L_n, S}, T)$. Then dualising gives a commutative diagram
\begin{cdiagram}[column sep=tiny]
   & \left ( \bidual^{r_T}_{\cR [\cG_n]} H^1_\Sigma (\bigO_{L_n, S}, T) \right)^\ast \arrow{r} \arrow{d}{\simeq} & (\UN^{r_T}_n)^\ast \arrow{r} \arrow{d}{\simeq} & \Ext^1_{\cR [\cG_n]} ( C, \cR [\cG_n] ) \arrow{r} \arrow{d} & 0 \\
   0 \arrow{r} & \Fitt^0_{\cR [\cG_n]} ( H^1 (C^\bullet_\infty)^{\Gamma^n, \vee}_\tor ) \arrow{r} &
   \cR [\cG_n] \arrow{r} & \faktor{\cR [\cG_n]}{\Fitt^0_{\cR [\cG_n]} ( H^1 (C^\bullet_\infty)^{\Gamma^n, \vee}_\tor )} \arrow{r} & 0,
\end{cdiagram}
where the two vertical isomorphisms are given by evaluating at the generator of the free rank one module $\UN^{r_T}_n$. Applying the snake lemma to this diagram reveals that the rightmost downward map is an isomorphism as well. The claim follows now upon noting that
\[
\Ext^1_{\cR [\cG_n]} ( C, \cR [\cG_n] ) \cong \Ext^1_{\cR} ( C, \cR ) \cong ( C_\tor)^\vee. 
\]
Turning our sights now to (c), we first record that the spectral sequence (\ref{spectral-sequence}) applied to the complex $C^\bullet_\infty$ gives an exact sequence
\begin{cdiagram}
   0 \arrow{r} & \UN^1_0 \arrow{r}{\iota} & H^1_\Sigma (\bigO_{L, S}, T) \arrow{r}{\delta} & H^2_{\Sigma,\Iw} (\bigO_{L, S}, T)^\Gamma \arrow{r} & 0.
\end{cdiagram}
Dualising this sequence, we obtain 
\begin{cdiagram}
  0 \arrow{r} & ( H^2_{\Sigma,\Iw} (\bigO_{L, S}, T)^\Gamma)^\ast \arrow{r} &  H^1_\Sigma (\bigO_{L, S}, T)^\ast \arrow{r} & \im \iota^\ast \arrow{r} & 0, 
\end{cdiagram}
where $\iota^\ast$ denotes the dual map of $\iota$. This induces the exact sequence
\begin{cdiagram}[column sep=tiny]
   ( H^2_{\Sigma, \Iw} (\bigO_{L, S}, T)^\Gamma)^\ast \otimes_{\cR [\cG]} \exprod^{r_T - 1}_{\cR [\cG]} H^1_\Sigma (\bigO_{L, S}, T)^\ast \arrow{r}
   &  
   \exprod^{r_T}_{\cR [\cG]} H^1_\Sigma (\bigO_{L, S}, T)^\ast 
   \arrow{r}
   &\exprod^{r_T}_{\cR [\cG]} \im \iota^\ast \arrow{r} & 0 .
\end{cdiagram}
Dualising again, we find that there is an exact sequence
\begin{cdiagram}[column sep=tiny]
   0 \arrow{r} & 
   \Big ( \exprod^{r_T}_{\cR [\cG]} \im \iota^\ast \Big)^\ast \arrow{r} & \exprod^{r_T}_{\cR [\cG]} H^1_\Sigma (\bigO_{L, S}, T)
   \arrow{r}
   & \left ( ( H^2_{\Sigma, \Iw} (\bigO_{L, S}, T)^\Gamma)^\ast \otimes_{\cR [\cG]} \exprod^{r_T - 1}_{\cR [\cG]} H^1_\Sigma (\bigO_{L, S}, T)^\ast \right)^\ast . 
\end{cdiagram}
Since $p \nmid |\cG|$, the module $H^1_\Sigma (\bigO_{L, S}, T)^*$ is $\cR [\cG]$-projective, so we have 
\begin{align*}
 & \phantom{=} \left ( ( H^2_{\Sigma, \Iw} (\bigO_{L, S}, T)^\Gamma)^\ast \otimes_{\cR [\cG]} \exprod^{r_T - 1}_{\cR [\cG]} H^1_\Sigma (\bigO_{L, S}, T)^\ast \right)^\ast \\
& = 
( H^2_{\Sigma, \Iw} (\bigO_{L, S}, T)^\Gamma)^{\ast \ast} \otimes_{\cR [\cG]} \left ( \exprod^{r_T - 1}_{\cR [\cG]} H^1_\Sigma (\bigO_{L, S}, T)^\ast \right)^\ast \\
& \cong ( H^2_{\Sigma, \Iw} (\bigO_{L, S}, T)^\Gamma)_\tf 
\otimes_{\cR [\cG]} \exprod^{r_T - 1}_{\cR [\cG]} H^1_\Sigma (\bigO_{L, S}, T).
\end{align*}
It therefore remains to show that $\faktor{\exprod^{r_T}_{\cR [\cG]} H^1_\Sigma (\bigO_{L, S}, T)}{\big ( \exprod^{r_T}_{\cR [\cG]} \im \iota^\ast \big)^\ast}$ is exactly the torsion-free part of $\faktor{\exprod^{r_T}_{\cR [\cG]} H^1_\Sigma (\bigO_{L, S}, T)}{\UN^{r_T}_0}$. \\

The above exact sequence shows that the former quotient is torsion-free. From the exact sequence 
\begin{cdiagram}
   0 \arrow{r} & \im \iota^\ast \arrow{r} & (\UN^1_0)^\ast \arrow{r} & 
   \Ext^1_{\cR [\cG]} ( H^2_\Iw (\bigO_{L, S}, T)^\Gamma, \cR [\cG]) \arrow{r} & 0
\end{cdiagram}
we see that $\im \iota^\ast$ has finite index inside $(\UN^1_0)^\ast$. Finally, from the diagram
\begin{cdiagram}
   \mathcal{Q} \otimes_\cR \exprod^{r_T}_{\cR [\cG]} \im \iota^\ast \arrow{r}{\simeq} & \mathcal{Q} \otimes_\cR \exprod^{r_T}_{\cR [\cG]}  (\UN^1_0)^\ast \\
   \exprod^{r_T}_{\cR [\cG]} \im \iota^\ast \arrow{u} \arrow{r} & \exprod^{r_T}_{\cR [\cG]} (\UN^1_0)^\ast \arrow{u}
\end{cdiagram}
we get, via dualising, the commutative diagram
\begin{cdiagram}
   \mathcal{Q} \otimes_\cR \Big ( \exprod^{r_T}_{\cR [\cG]} \im \iota^\ast \Big )^\ast  & \mathcal{Q} \otimes_\cR \exprod^{r_T}_{\cR [\cG]}  \UN^1_0 \arrow{l}{\simeq} \\
   \Big ( \exprod^{r_T}_{\cR [\cG]} \im \iota^\ast \Big)^\ast \arrow[hookrightarrow]{u}  & \exprod^{r_T}_{\cR [\cG]} \UN^1_0 \arrow[hookrightarrow]{u} \arrow{l}
\end{cdiagram}
from which we deduce that $\UN^{r_T}_0 = \exprod^{r_T}_{\cR [\cG]} \UN^1_0$  injects with finite index into $\big ( \exprod^{r_T}_{\cR [\cG]} \im \iota^\ast \big)^\ast$. 
\end{proofbox}
As a byproduct of Proposition \ref{Dodgy-Proposition} we obtain the following curious consequence.

\begin{cor}
Assume $r_T = 1$ and $p \nmid | \cG|$. Then the $\Z_p$-torsion submodule of $H^2_{\Sigma, \Iw} (\bigO_{L, S}, T)^{\Gamma}$ is a cyclic $\cR [\cG]$-module. 
\end{cor}

\begin{proofbox}
We have already observed in the proof of Proposition \ref{Dodgy-Proposition} that the spectral sequence (\ref{spectral-sequence}) gives an exact sequence
\begin{cdiagram}
   0 \arrow{r} & \UN^1_0 \arrow{r} & 
   H^1_\Sigma (\bigO_{L, S}, T) \arrow{r} & H^2_{\Sigma, \Iw} (\bigO_{L, S}, T)^{\Gamma} \arrow{r} & 0.
\end{cdiagram}
Combining this with Proposition \ref{Dodgy-Proposition} (b) we find that 
\begin{align*}
\left ( H^2_{\Sigma, \Iw} (\bigO_{L, S}, T)^{\Gamma} \right)_\tor 
& \cong 
\left ( \faktor{H^1_\Sigma (\bigO_{L, S}, T)}{\UN^1_0} \right )_\tor \\
& \cong \Ext^1_{\cR [\cG]} \Big ( \faktor{\cR [\cG_n]}{\Fitt^0_{\cR [\cG]} ( H^2_{\Sigma, \Iw}( \bigO_{L, S}, T)^{\Gamma, \vee}_\tor )}, \cR [\cG] \Big) .
\end{align*}
Since $p \nmid |\cG|$, the ideal $\Fitt^0_{\cR [\cG]} ( H^2_{\Sigma, \Iw}( \bigO_{L, S}, T)^{\Gamma, \vee}_\tor )$ is principal and generated by a non-zero divisor $x$, say (see \cite[Prop. 2.2.2]{Cornelius}). It is then immediate from the exact sequence
\begin{cdiagram}
0 \arrow{r} & \cR [\cG] \arrow[r, "\cdot x"] & \cR [\cG] \arrow{r} & \faktor{\cR [\cG]}{(x)} \arrow{r} & 0
\end{cdiagram}
that $\Ext^1_{\cR [\cG]} (\faktor{\cR [\cG]}{(x)}, \cR [\cG])$ is $\cR [\cG]$-cyclic. 
\end{proofbox}

%arxiv hack
%arxiv uses TeXLive 2016 and the markboth here just doesn't work on that version for some reason
%it seemingly only works when you insert some content from the previous section into the new page
\pagebreak
\phantom{blank}
\vspace{-3em}
%end hack

\markboth{Applications to arithmetic}{Tate twist $T=\Z_p (1)$}
\section{Applications to arithmetic}


In this section we exemplify how one can use the general framework laid out in the previous section to derive concrete arithmetic consequences. 

\subsection{Tate twist $T = \Z_p (1)$}

We shall first specialise to the representation $T = \Z_p (1)$. This representation
satisfies Hypothesis \ref{main-hypothesis} as observed in Example \ref{RepExamples} (b) and so the general results from the last section are applicable in this situation. 
We introduce the following additional notation. \\

Let $F$ be a number field. For finite sets  of places $V$ and $\Sigma$ of $F$ satisfying $V \cap \Sigma = \emptyset$ and $S_\infty (K) \cap \Sigma = \emptyset$ we denote
\begin{itemize}
    \item $U_{F, V, \Sigma} = \Z_p \otimes_\Z \bigO_{F, V, \Sigma}^\times = H^1_\Sigma ( \bigO_{F, S}, \Z_p (1))$ the $p$-completion of the $(V, \Sigma)$-unit group of $F$, 
    \item $A_{V,\Sigma} (F) = \Z_p \otimes_\Z \cl_{V,\Sigma} (F)$ the $p$-Sylow subgroup of the $V$-ray class group mod $\Sigma$ of $F$,
    \item $Y_{F, V} = \bigoplus_{v \in V} \Z_p$ the free $\Z_p$-module on the set of places contained in $V$,
    \item $X_{F, V}$ the kernel of the natural augmention map $Y_{F, V} \to \Z_p$.
\end{itemize}
When $\Sigma = \emptyset$, we omit the reference to $\Sigma$ from the above notation. If $F_\infty = \bigcup_{n \geq 0} F_n$ defines a $\Z_p$-extension of $F$, then for any of the objects $\square$ above we denote by $\square^\infty$ the projective limit over $n$ of the respective objects $\square$ for $F_n$, where the limits are taken with respect to the natural transition maps in each situation.

\subsubsection{Iwasawa Main Conjecture}

In this section we give a straightforward application of the result of Theorem \ref{pairing-theorem}. In particular, we show that the isomorphism of Theorem \ref{pairing-theorem}(a) refines the (plus-part of the) classical Iwasawa Main Conjecture.\\

For any integer $m \geq 1$, let $\xi_m = e^{2 \pi i / m}$, which we regard as an element of $\overline{\Q}$ via a fixed embedding $\overline{\Q} \hookrightarrow \C$. 
Take $f >0$ to be an integer such that $f \not \equiv 2 \mod 4$ and
$p \nmid f$. Then for every $n \geq 0$ we set $L_n = \Q (\xi_{f p^{n + 1}})$ and note that the collection of maximal totally real subfields $L_n^+$ constitutes a $\Z_p$-extension that satisfies the assumptions of \S\ref{set-up-section}. We will therefore resume the notation introduced there and hope that this does not cause any confusion. \\

For every $n$ we denote by $\Cyc (L_n)$ the group of cyclotomic units of $L_n$. In other words, 
\begin{align*}
    \Cyc_n := \langle -1, 1 - \xi_d : d \mid f p^{n+1} \rangle_{\Z_p [\cG_n]} \cap \bigO_{L_n}^\times. 
\end{align*}
We then set $\Cyc^\infty := \varprojlim_n \Cyc_n$, where the limit is taken with respect to the norm maps.

\begin{thm} \label{IMC-refinement}
    Let $\chi \in \widehat{\cG}$ be an even character and assume that $p \nmid [L : \Q]$, where $L = L_0$. 
    Then there is an isomorphism of $\Lambda_\chi := \ZZ_p(\im(\chi))\llbracket \Gamma \rrbracket$-modules
    \begin{align*}
        \faktor{U^{\infty,\chi}}{\Cyc^{\infty,\chi}} \cong \alpha  \left ( \faktor{\Lambda_\chi}{\cchar_{\Lambda_\chi}(A^{\infty,\chi})} \right),
    \end{align*}
    where $(-)^\chi$ is the functor $- \otimes_{\ZZ_p[\cG]} \ZZ_p(\im(\chi)) = - \otimes_\bLambda \Lambda_\chi$, and $\alpha (-) = \Ext^1_{\Lambda_\chi} ( -, \Lambda_\chi)$ denotes the Iwasawa adjoint. 
\end{thm}

\begin{rk} \label{IMC-refinement-remark}
Theorem \ref{IMC-refinement} can bee seen as a refinement of the classical Iwasawa Main Conjecture for the following reason: \cite[Prop. 5.5.13]{NSW} gives a pseudo-isomorphism $\alpha  (\Lambda_\chi/\text{char}_{\Lambda_\chi}(A^{\infty,\chi}))
\approx (\Lambda_\chi/\cchar_{\Lambda_\chi}(A^{\infty, \chi}))^\circ
$ and so taking characteristic ideals on both sides of the isomorphism stated in Theorem \ref{IMC-refinement} yields
\[
\mathrm{char}_{\Lambda_\chi}\left ( \faktor{U^{\infty, \chi}}{\Cyc^{\infty, \chi}} \right) = \mathrm{char}_{\Lambda_\chi}(A^{\infty,\chi}).
\]
This is one form of the classical Iwasawa Main Conjecture first proved by Mazur and Wiles \cite{MazurWiles} (see also \cite[Thm. 5.1]{LangRubin}). That is, Theorem \ref{IMC-refinement} amounts to the assertion that not only the characteristic ideals of the aforementioned modules agree but that in fact their $\Lambda_\chi$-module structures are intimately related. 
\end{rk}

In order to prove Theorem \ref{IMC-refinement} we require the following simple Lemma. 

\begin{lem} \label{comparison-lemma}
Assume that $p \nmid [L : \Q]$ and let $\chi \in \widehat{\cG}$ be a character. If $M$ is a finitely generated torsion $\Lambda_\chi$-module, then we have
\[
\alpha \left ( \faktor{\Lambda_\chi}{\Fitt^0_{\Lambda_\chi} (M)} \right ) = 
\alpha \left ( \faktor{\Lambda_\chi}{\cchar_{\Lambda_\chi} (M)} \right ).
\]
\end{lem}

\begin{proof}
Recall (for example from \cite[Lem. 3.4.2]{NQDV}) that we have $\Fitt^0_{\Lambda_\chi} (M) = \Fitt_{\Lambda_\chi}^0 ( M_\tor) \cdot \cchar_{\Lambda_\chi} ( M_\tf)$. In particular, $\Fitt^0_{\Lambda_\chi} (M) \subseteq \cchar_{\Lambda_\chi} (M)$ and hence there is a surjection
\begin{equation} \label{Map}
\faktor{\Lambda_\chi}{\Fitt^0_{\Lambda_\chi} (M)} \twoheadrightarrow \faktor{\Lambda_\chi}{\cchar_{\Lambda_\chi} (M)}.
\end{equation}
Since $\Fitt^0_{\Lambda_\chi} (M)_\p = \cchar_{\Lambda_\chi} (M)_\p$ for any height one prime $\p$ of $\Lambda_\chi$, the map (\ref{Map}) has finite kernel. The Lemma follows now from \cite[Prop. 5.5.3 (ii)]{NSW}. 
\end{proof}


\textit{Proof of Theorem \ref{IMC-refinement}:}
    At the outset we remark that the equivariant Iwasawa Main Conjecture is known to be valid (by the work of Burns and Greither \cite{BurnsGreither}) for the data $(L_\infty | \Q, S \cup S_\infty (\Q), \varnothing, p)$, where $S = \{ v \mid f p \}$, and so Proposition \ref{eimc-proposition} implies that 
    \[
    \NS^b(\ZZ_p(1), L^+_\infty) = \langle e^+ \eta_{f} \rangle_{\bLambda},
    \]
    where for each $m \mid f$ we put $\eta_m = ( 1 - \xi_{m p^{n + 1}})_{n \geq 0} \in U^\infty_S$ and use the idempotent $e^+ = \frac12 (1 + c)$ with complex conjugation $c \in \cG : = \gal{L}{\Q}$. \\
    
    Given this, part (a) of Theorem \ref{pairing-theorem} implies that there is an isomorphism
    \begin{align*}
        e^+ \left ( \faktor{U_{S}^\infty}{\langle \eta_{f} \rangle} \right ) \cong
        e^+
        \Ext^1_\bLambda(
        \faktor{\bLambda}{ \Fitt_{\bLambda}^0 
    (H^2_\Iw (\bigO_{L, S}, \Z_p (1))}
    , \ \bLambda).
    \end{align*}
    By assumption the order of $\cG$ is invertible in $\ZZ_p$ and hence $\Lambda_\chi$ is a projective $\bLambda$-module. In particular, the functor $(-)^\chi$ is exact and thus we obtain from Lemma \ref{comparison-lemma} an isomorphism of $\Lambda_\chi$-modules
    \begin{align*}
        \faktor{U_S^{\infty,\chi}}{\langle e_\chi \eta_{f}\rangle_{\bLambda}} \cong
        \Ext^1_{\Lambda_\chi}
        \left ( \faktor{\Lambda_\chi}{\cchar_{\Lambda_\chi} (H^2_\Iw (\bigO_{L, S}, \Z_p (1)))},
        \Lambda_\chi
        \right).
    \end{align*}
    

    The explicit description of the pairing given in Theorem \ref{pairing-theorem} then shows that this isomorphism is induced by the map
    \begin{equation} \label{TheIsom}
    e^+ U^{\infty}_S \to e^+ \Hom_{\bLambda} ( \faktor{\bLambda}{\Fitt^0_\bLambda ( H^2_\Iw (\bigO_{L, S}, \Z_p (1))}, 
    \; \faktor{Q (\bLambda)}{\bLambda} )
    \quad e^+ u \mapsto \{ \lambda \mapsto e^+ (\lambda \cdot \eta_{f}^\ast (u)) \},
    \end{equation}
    where $\eta^\ast_{f} \in Q (\bLambda) \otimes_\bLambda ( U^{\infty}_S)^\ast$ is the dual of $\eta_{f}$. We next note that the Euler system norm relation in this case reads
    \begin{equation} \label{ESNormRelations}
    N_{ \Q ( \xi_{f}) | \Q (\xi_{m})} ( \eta_k)
    = 
    [\Q ( \xi_{f}) : \Q (\xi_{k})] \cdot \Big ( 
    \prod_{\substack{l \mid k \\ l \nmid m}} (1 - \Frob_l^{-1}) \Big ) \cdot \eta_m
    \end{equation}
    for any pair of integers $(k,m)$ satisfiying the divisibility relation $m \mid k \mid f$. Now let $d$ be the conductor of $\chi$, then for $d \mid k \mid f$ we get that
    \[
    e_\chi \eta_k = [\Q ( \xi_{k}) : \Q (\xi_{d})]^{-1} \cdot e_\chi \Big ( 
    \prod_{\substack{l \mid k \\ l \nmid d}} (1 - \Frob_l^{-1}) \Big ) \cdot \eta_d
    \]
    is a $\bLambda$-multiple of $\eta_d$.
    For any $m \nmid d$, on the other hand, we have
    \[
    e_\chi \eta_m = [\Q ( \xi_{f}) : \Q (\xi_{m})]^{-1} \cdot e_\chi N_{\Q ( \xi_{f}) | \Q (\xi_{m})} ( \eta_m) = 0
    \]
    since $\chi$ is non-trivial on $\gal{\Q ( \xi_{f})}{\Q (\xi_m)}$. 
    Let $T = (\gamma - 1)$ for a topological generator $\gamma$ of $\Gamma$, then the above reasoning shows that
    \[
    \Cyc^{\infty, \chi} = 
    \langle e_\chi T^{\delta_\chi} \eta_d \rangle_{\Lambda_\chi} 
   \qquad \text{ for } \delta_\chi = \begin{cases}
   0 & \text{ if } \chi \neq 1, \\
   1 & \text{ if } \chi =1,
   \end{cases}
    \]
    where we have used the fact that, if $d \neq 1$, both $\eta_d, T \eta_1 \in U^\infty$ since $p \nmid d$. 
    Moreover, from (\ref{ESNormRelations}) we deduce that the isomorphism (\ref{TheIsom}) maps $e_\chi \eta_d$ onto the element corresponding to multiplication by
    \begin{equation} \label{Image}
    e_\chi \theta_d^{-1} =e_\chi  \Big ( 
    \prod_{\substack{l \mid f \\ l \nmid d}} (1 - \Frob_l^{-1}) \Big )^{-1}.
       \end{equation}
    From the exact sequence
    \begin{cdiagram}
    0 \arrow{r} & 
    A_S^\infty \arrow{r} & H^2_\Iw (\bigO_{L, S}, \Z_p (1)) \arrow{r} & X_S^{\infty} \arrow{r} & 0 
    \end{cdiagram}
    one obtains, using the fact
    that characteristic ideals are multiplicative, an equality
    \begin{align*}
        \cchar_{\Lambda_\chi} ( H^2_\Iw (\bigO_{L, S}, \Z_p (1))^\chi)  
        & = \cchar_{\Lambda_\chi} ( A_S^{\infty, \chi})
        \cdot \cchar_{\Lambda_\chi} ( X^{\infty, \chi}_S ).
    \end{align*}
    An explicit calculation furthermore shows (\textit{c.f.}\@ \cite[Lem. 5.5]{Flach} but note that in our case $p \in S$) that
    \[
    \cchar_{\Lambda_\chi} ( X^{\infty, \chi}_S) = 
    \Big (e_\chi T^{\epsilon_\chi} \prod_{\substack{l \mid f \\ l \nmid d}} (1 - \Frob_l^{-1} ) \Big ) = (e_\chi T^{\epsilon_\chi }\theta_d),
    \]
    where 
    \[
    \epsilon_\chi = \begin{cases} 1 & \text{ if } \chi (p) = 1 \text{ and } \chi \neq 1, \\
    0 & \text{ otherwise}.
    \end{cases}
    \]
      Hence we have the exact sequence
    \begin{cdiagram}[column sep=small]
    0 \arrow{r} & \faktor{\Lambda_\chi}{T^{\epsilon_\chi} \cchar_{\Lambda_\chi} (A_S^\chi)} \arrow{r}{\cdot \theta_d} & 
    \faktor{\Lambda_\chi}{\cchar_{\Lambda_\chi} H^2_\Iw (\bigO_{L, S}, \Z_p (1))^\chi} 
    \arrow{r} & 
    \faktor{\Lambda_\chi}{T^{-\epsilon_\chi} \cchar_{\Lambda_\chi} (X_S^\infty)} \arrow{r} & 0
    \end{cdiagram}
    that combines with the statement (\ref{Image}) to give a commutative diagram
    \begin{cdiagram}[column sep=tiny]
    0 \arrow{r} & \faktor{\langle e_\chi \eta_d \rangle}{\langle e_\chi \eta_{f} \rangle} 
    \arrow{d}{\simeq} \arrow{r} &
    \faktor{U^{\infty, \chi}_S}{\langle e_\chi \eta_{f} \rangle} 
    \arrow{r} \arrow{d}{\simeq} & 
    \faktor{U^{\infty, \chi}_S}{\langle e_\chi \eta_d \rangle}
\arrow[dashed]{d} \arrow{r} & 
0 \\
0 \arrow{r} & \alpha \left(
\displaystyle\faktor{\Lambda_\chi}{T^{-\epsilon_\chi} \cchar_{\Lambda_\chi} (X_S^\infty)} \right ) 
\arrow{r} &
\alpha \left(
\displaystyle
 \faktor{\Lambda_\chi}{\cchar_{\Lambda_\chi} (H^2_\Iw (\bigO_{L, S}, \Z_p (1))^\chi)} 
 \right )
 \arrow{r}{\cdot \theta_d}
 &
 \alpha \left(
 \displaystyle
 \faktor{\Lambda_\chi}{T^{\epsilon_\chi} \cchar_{\Lambda_\chi} (A_S^\chi)}
 \right)
 \arrow{r} & 
 0
    \end{cdiagram}

    An application of the snake lemma then implies that the right hand map is an isomorphism.
    
     Now, observe that the module 
     $\faktor{U^{\infty, \chi}}{\langle e_\chi T^{\delta_\chi} \eta_d \rangle} =  \faktor{U^{\infty, \chi}}{\Cyc^{\infty, \chi}}$ is $\Lambda_\chi$-cyclic since $U^{\infty, \chi}$ is $\Lambda_\chi$-free of rank one. The aforementioned quotient is furthermore $\Z_p$-torsion free as it injects into
     $\faktor{U_S^{\infty, \chi}}{\langle e_\chi \eta_d \rangle} \cong \alpha (  \faktor{\Lambda_\chi}{T^{\epsilon_\chi} \cchar_{\Lambda_\chi} (A_S^\chi)})$. It follows that 
     \begin{align*}
     \Ann_{\Lambda_\chi} \left ( \faktor{U^{\infty, \chi}}{\Cyc^{\infty, \chi}} \right ) & = 
     \Fitt^0_{\Lambda_\chi} \left ( \faktor{U^{\infty, \chi}}{\Cyc^{\infty, \chi}} \right ) = \cchar_{\Lambda_\chi} \left ( \faktor{U^{\infty, \chi}}{\Cyc^{\infty, \chi}} \right ) \\
     & = 
     \cchar_{\Lambda_\chi} ( A^{\chi} ),
     \end{align*}
    where we have used the classical Iwasawa Main Conjecture \cite[Thm. 5.1]{LangRubin} to establish the final equality. By using an explicit description analogous to (\ref{TheIsom}), we see that the image of $\faktor{U^{\infty, \chi}}{\Cyc^{\infty, \chi}}$
    under the isomorphism 
    \begin{equation} \label{AnotherIsom}
    \faktor{U^{\infty, \chi}_S}{\langle e_\chi \eta_d \rangle}
    \cong 
    \alpha \left(
 \displaystyle
 \faktor{\Lambda_\chi}{T^{\epsilon_\chi} \cchar_{\Lambda_\chi} (A_S^\chi)}
 \right)
    \end{equation}
    coincides with the kernel of multiplication by a generator of $ \cchar_{\Lambda_\chi} ( A^{\chi} )$ on $\alpha \left(
 \faktor{\Lambda_\chi}{\cchar_{\Lambda_\chi} (A_S^\chi)}
 \right)$. An argument entirely similar to the one utilised above show that this kernel is exactly $\alpha (
 \faktor{\Lambda_\chi}{\cchar_{\Lambda_\chi} (A^\chi)})$.
Hence the isomorphism (\ref{AnotherIsom}) restricts to give the isomorphism claimed in the statement of Theorem \ref{IMC-refinement}. 
    \qed
    

\subsubsection{Greenberg's conjecture}

In this subsection we will rely on the results of Appendix \ref{AppendixIwasawa}. We resume the notation and assumptions of \S\ref{set-up-section}.

\begin{prop} \label{GreenbergCriterion}
Assume the $\mu$-invariant of $A_{S, \Sigma}^\infty$ vanishes. The following are equivalent:
\begin{liste}
\item The module $A_{S, \Sigma}^\infty$ is finite,
\item there is an isomorphism
\[
\faktor{\NS^{r_T} ( \Z_p (1), L_\infty)}{\NS^b (\Z_p (1), L_\infty)} \cong \Ext^1_\bLambda \Big ( \faktor{\bLambda}{\Fitt^0_\bLambda ( X_{S \setminus S_\infty}^\infty)},\bLambda \Big).
\]
\end{liste}
\end{prop}

\begin{rk}
\begin{liste}
\item If $L$ is a totally real field, we may take $\Sigma = \emptyset$ since $p$ is odd. Then $A_{S, \Sigma}^\infty$ agrees with the inverse limit of $S$-class groups $A_S^\infty$. Greenberg has conjectured \cite{Greenberg} that both the $\mu$ and $\lambda$-invariant of $A^\infty$ vanish, i.e.\@ that $A^\infty$ is finite. It is well known (see, for example, \cite[Section 4]{Iwasawa}) that if one assumes that the coinvariants $(A_S^\infty)_\Gamma$ are finite (as is conjectured by Gross \cite{Gross} and Kuz'min \cite{Kuzmin}), then
\[
A_S^\infty \text{ is finite} \iff A^\infty \text{ is finite}.
\]
The Gross-Kuz'min conjecture is known in a number of important cases including the case of $L | \Q$ being abelian and $L_\infty | L$ having only one ramified prime. We refer the reader to the recent article \cite{HoferKleine} for a more detailed discussion of known instances of the Gross-Kuz'min conjecture and its relation to other conjectures. 
\item Theoretical evidence for Greenberg's conjecture is still very sparse. The only general class of fields known to satisfy the conjecture are fields $L$ with a unique prime above $p$ and such that $A_{S_p} (L) = 1$ (in this case Greenberg's conjecture follows from Nakayama's Lemma). However, there are many explicit examples giving evidence for the conjecture, starting with Greenberg's original article \cite[\S 8]{Greenberg}. For example, Kraft and Schoof \cite{KraftSchoof} have numerically verified the conjecture for $p = 3$ and all real quadratic fields $\Q ( \sqrt{f})$ such that $f \not \equiv 1 \mod 3$ and $f < 10000$. 
\item The case $r_T = 1$ of Proposition \ref{GreenbergCriterion} is classical and well-known. To the best of the knowledge of the authors, a result of this shape first appeared in \cite[Lem. 1]{Gold}. 
\item{Suppose that $L$ is not ramified at $p$ and that $L_\infty| L$ is the cyclotomic $\ZZ_p$-extension in which only one prime of $K$ ramifies. By considering the construction of \textit{basic Iwasawa-Euler systems} of the second named author in \cite[\S4.3]{daoud} one can show that the equality $\NS^{r_T} = \NS^b$ implies that any element of $\NS^{r_T}$ extends to an Euler system of rank $r_T$ for the pair $(T,\cK K_\infty)$ (in the sense of \cite[Def. 2.2]{daoud}) where $\cK$ is the maximal abelian extension of $K$ unramified at the primes in $S$ and $K_\infty$ is the cyclotomic $\ZZ_p$-extension of $K$.}
\end{liste}
\end{rk}


\textit{Proof of Proposition \ref{GreenbergCriterion}:}
We have an exact sequence
\begin{cdiagram}
   0 \arrow{r} & A_{S, \Sigma}^\infty \arrow{r} & H^2_{\Sigma, \Iw} (\bigO_{L, S}, \Z_p (1)) \arrow{r} & X^\infty_{S \setminus S_\infty} \arrow{r} & 0
\end{cdiagram}
and it follows that at every regular height one prime $\p$ of $\bLambda$ there is an equality
\begin{equation} \label{factorisation}
\Fitt^0_\bLambda ( H^2_{\Sigma, \Iw} (\bigO_{L, S}, \Z_p (1)))_\p = \Fitt^0_\bLambda (A_{S, \Sigma}^\infty)_\p \cdot \Fitt^0_\bLambda ( X^\infty_{S \setminus S_\infty} )_\p
\end{equation}
since $\bLambda_\p$ is a discrete valuation ring in this case. If $\p$ is a singular prime, in turn, then $H^2_{\Sigma, \Iw} (\bigO_{L, S}, \Z_p (1))_\p = \Fitt^0_\bLambda (A_{S, \Sigma}^\infty)_\p$ by Lemma \ref{IwasawaInvariants} (a) because $X^\infty_{S \setminus S_\infty}$ has vanishing $\mu$-invariant. \\

Let us now assume that $A_{S, \Sigma}^\infty$ is finite. The previous discussion combines with Lemma \ref{IwasawaInvariants} (b) to imply that $\Fitt^0_\bLambda ( X^\infty_{S \setminus S_\infty})_\p = \Fitt^0_\bLambda (H^2_{\Sigma, \Iw} (\bigO_{L, S}, \Z_p (1)))_\p$ for all height one primes $\p$ of $\bLambda$. This implies that the surjection
\[
\faktor{\bLambda}{\Fitt^0_\bLambda (H^2_{\Sigma, \Iw} (\bigO_{L, S}, \Z_p (1)))} \to \faktor{\bLambda}{\Fitt^0_\bLambda ( X^\infty_{S \setminus S_\infty})}
\]
is a pseudo-isomorphism and hence has finite kernel. Since the module on the left identifies with an $\Ext^1_\bLambda ( -, \bLambda)$ by Theorem \ref{pairing-theorem}, it cannot have any non-trivial finite submodules. The above map is thus in fact an isomorphism and the isomorphism of Theorem \ref{pairing-theorem} becomes
\[
\faktor{\NS^{r_T}}{\NS^b} \cong \Ext^1_\bLambda \Big ( \faktor{\bLambda}{\Fitt^0_\bLambda ( X^\infty_{S \setminus S_\infty})},\bLambda \Big ). 
\]
Conversely, assume to be given such an isomorphism. Taking $\Ext^1_\bLambda ( - , \bLambda)$ and combining with Theorem \ref{pairing-theorem}, we obtain a pseudo-isomorphism
\begin{align} \label{pseudo-isomorphism}
\faktor{\bLambda}{\Fitt^0_\bLambda (H^2_{\Sigma, \Iw} (\bigO_{L, S}, \Z_p (1)))} \approx \faktor{\bLambda}{\Fitt^0_\bLambda ( X^\infty_{S \setminus S_\infty})}. 
\end{align}
Now let $\p$ be a regular height one prime, then the above pseudo-isomorphism gives 
\[
\Fitt^0_\bLambda (H^2_{\Sigma, \Iw} (\bigO_{L, S}, \Z_p (1)))_\p \cong 
\Fitt^0_\bLambda ( X^\infty_{S \setminus S_\infty})_\p.
\]
Since $\bLambda_\p$ is a discrete valuation ring, this implies
\begin{align*}
    (\p \bLambda_\p)^{\text{length}_{\bLambda_\p} (X^\infty_{S \setminus S_\infty})_\p + \text{length}_{\bLambda_\p} (A_{S, \Sigma}^\infty)_\p } & = \Fitt^0_\bLambda (A^\infty_{S, \Sigma})_\p \cdot \Fitt^0_\bLambda ( X^\infty_{S \setminus S_\infty} )_\p \\
    & = \Fitt^0_\bLambda ( X^\infty_{S \setminus S_\infty} )_\p \\
    & = (\p \bLambda_\p)^{\text{length}_{\bLambda_\p} (X^\infty_{S \setminus S_\infty})_\p}.
\end{align*}
We deduce that $\text{length}_{\bLambda} (A^\infty_{S, \Sigma})_\p = 0$, hence $(A^\infty_{S, \Sigma})_\p = 0$. Thus, the finiteness of $A_{S, \Sigma}^\infty$ follows now from 
the assumed vanishing of its $\mu$-invariant and
Lemma \ref{IwasawaInvariants}. 
\qed







\subsubsection{Leading term conjectures}

We continue using the notations an assumptions of \S \ref{set-up-section}. In this section we describe a connection between Proposition \ref{GreenbergCriterion} and conjectures concerning the leading terms of equivariant $L$-functions that appear in the literature. The central player in these conjectures is the \textit{$S$-truncated and $\Sigma$-modified Dirichlet $L$-function} that is defined as
\[
L_{L_n | K, S, \Sigma} (\chi, s) = \prod_{v \in \Sigma} (1 - \chi ( \Frob_v) \text{N}v^{1 - s} ) \cdot \prod_{v \not \in S} (1 - \chi (\Frob_v) \text{N}v^{-s} )^{-1}
\]
for any complex values $s$ satisfying $\text{Re} (s) > 1$, and any character $\chi \in \widehat{\cG_n}$. It is well known that $L_{L_n | K, S, \Sigma} (\chi, s)$ can be continued to a meromorphic function that is defined on the whole complex plane and holomorphic at $s = 0$. For any $r \geq 0$, we denote the $r^{th}$-th coefficient in the Taylor expansion of $L_{L_n | K, S, \Sigma}  (\chi, s)$ at $s = 0$ by 
\[
L^{(r)}_{L_n | K, S, \Sigma} (\chi, 0) = \lim_{s \to 0} s^{-r} L_{L_n | K, S, \Sigma}  (\chi, s). 
\]
and define the \textit{Stickelberger element} to be
\[
\theta_{L_n | K, S, \Sigma}^{(r)} (0) = \sum_{\chi \in \widehat{\cG_n}} e_{\overline{\chi}} L_{L_n | K, S, \Sigma}^{(r)} (0).
\]
Fix an isomorphism $\C \cong \C_p$ and recall that the Dirichlet regulator defines a $\C_p [\cG_n]$-linear isomorphism
\[
\lambda_{L_n, S, \Sigma} \: \C_p \otimes_\Z \bigO_{L_n, S, \Sigma}^\times 
\stackrel{\simeq}{\longrightarrow}
\C_p \otimes_\Z X_{L_n, S}, 
\quad
a \mapsto - \sum_{w \in S_{L_n}} \log (| a|_w ) \cdot w.
\]
We remind the reader that the integer $r_T \geq 0$ for $T = \Z_p (1)$ is given by $r_T = | S_\infty (K) |$ under the running hypotheses. For every $v \in S$, we also fix a place $w \in S_{L_n}$ such that $w | v$. 
\begin{definition}
Pick $w_0 \in S \setminus S_\infty$.
The $r_T$-th \textit{Rubin-Stark element} $\varepsilon_{L_n | K, S, \Sigma}$ is the preimage of $\theta^{(r_T)}_{L_n | K, S, \Sigma} (0) \cdot \bigwedge_{w \in S_\infty} (w - w_0)$
under the isomorphism
\[
\C_p \otimes_{\Z_p} \exprod^{r_T}_{\Z_p [\cG_n]} U_{L_n, S, \Sigma} \stackrel{\simeq}{\longrightarrow} \C_p \otimes_{\Z_p} \exprod^{r_T}_{\Z_p [\cG_n]} X_{L_n, S}
\]
induced by the Dirichlet regulator $\lambda_{L, S, \Sigma}$.
\end{definition}

We shall investigate the following conjectures.

\begin{conjecture} \label{conjectures}
    \begin{liste}
    \item We have an inclusion
    \[
    \varepsilon_{L_n | K, S, \Sigma} \in \bidual^{r_T}_{\Z_p [\cG_n]} U_{L_n, S, \Sigma}.
    \]
    \item There is a basis $\mathfrak{z}_{L_n} \in \Det_{\Z_p [\cG_n]} ( C^\bullet_n)$ such that $\Theta_{L_n} (\mathfrak{z}_{L_n}) = \varepsilon_{L_n | K, S, \Sigma}$. 
    \item There is a basis $\mathfrak{z}_{\infty} \in \Det_\bLambda (C^\bullet_\infty)$ such that $\Theta (\mathfrak{z}_\infty) = (\varepsilon_{L_n | K, S, \Sigma})_n$. 
    \end{liste}
\end{conjecture}

\begin{rk} \label{conjectures-remark}
The above are special cases of conjectures appearing in the literature. Conjecture \ref{conjectures} (a) is the relevant case of the Rubin-Stark conjecture \cite[Conj. B']{Rubin96} in this setting. Conjecture \ref{conjectures} (b) is a consequence of the equivariant Tamagawa number conjecture as stated, for example, in \cite[Conj. 2.3]{BKS2} after taking \cite[Prop. 2.5]{BKS2} into consideration. It is easy to see that if $L | K$ has a unique place above $p$ and $S = S_\infty (K) \cup S_p$, then Conjecture \ref{conjectures} (b) in fact coincides with \cite[Conj. 2.3]{BKS2}. Finally, Conjecture \ref{conjectures} (c) is the higher rank equivariant Iwasawa main conjecture appearing in \cite[Conj. 3.1]{BKS2}. 
\end{rk}

The following is an analogue of the classical index formula for cyclotomic units (see \cite[Thm. 8.2]{Washington}).

\begin{lem} \label{IndexFormula}
Assume that all infinite places split in $L | K$
and that there is a unique place $\p$ above $p$ in $L$. Put $S = S_\infty (K) \cup S_p$
and assume that $A_{S, \Sigma} (L)$ is $\cG$-cohomologically trivial. Then
\[
(\bidual^r_{\Z_p [\cG]} U_{L, S, \Sigma} : \; \Z_p [\cG] \cdot \varepsilon_{L | K, S, \Sigma} ) = h_{L, S, \Sigma}, 
\]
where $h_{L, S, \Sigma} = | A_{S, \Sigma} ( L) |$ is the $p$-part of the $(S, \Sigma)$-class number of $L$. 
\end{lem}

\begin{proofbox}
Since $\cG$ acts trivially on $\p$, the module $X_{L, S}$ is $\Z_p [\cG]$-free, generated by the $\Z_p [\cG]$-linearly independent set $\{ (w - \p) \}$, where $w$ ranges over our set of fixed places in $S_L$. It follows that $H^2 (C^\bullet)$ is $\cG$-cohomologically trivial and the isomorphism
\[
\widehat{H}^i (\cG, \ U_{L, S, \Sigma})\cong \widehat{H}^{i + 2} ( \cG, \ H^2 (C^\bullet))
\]
for all $i \in \Z$ induced by the complex $C^\bullet$ shows that $U_{L,S, \Sigma }$ is $\cG$-cohomologically trivial, hence $\Z_p [\cG]$-projective. The Dirichlet regulator map $\lambda_{L, S, \Sigma}$ induces a rational isomorphism $\Q_p U_{L, S, \Sigma} \cong \Q_p X_{L, S}$ (see \cite[\S I.4.3]{Tate}), which in this setting (see \cite[Thm.\@ 5.6.10 (ii)]{NSW}) implies that there is an isomorphism $U_{L, S, \Sigma} \cong X_{L, S}$ and so $U_{L,S,\Sigma}$ is $\ZZ_p[\cG]$-free. Let $\{u_1, \dots, u_{r_T}\}$ be any $\ZZ_p[\cG]$-basis of this module. Then 
\[
\bidual^{r_T}_{\Z_p [\cG]} U_{L, S, \Sigma} = \exprod^{r_T}_{\Z_p [\cG]} U_{L, S, \Sigma} = \Z_p [\cG] \cdot (u_1 \wedge \dots \wedge u_{r_T}).
\]
It now suffices to calculate the index  
\[
\Z_p [\cG] \cdot \lambda_{L, S, \Sigma} ( \varepsilon_{L | K, S, \Sigma}) = \Z_p [\cG] \cdot \theta^{(r_T)}_{L, S, \Sigma} (0) 
\quad\text{ inside } \quad
\Z_p [\cG] \cdot \lambda_{L, S, \Sigma} ( u_1 \wedge \dots \wedge u_{r_T}) 
\]
as sublattices of $\C_p \exprod_{\Z_p [\cG]}^r X_{L, S} \cong \C_p [\cG]$. If we can find a $\C_p$-linear isomorphism $f \: \C_p [\cG] \to \C_p [\cG]$ that maps the first of these two aforementioned lattices bijectively onto the latter, then by \cite[Lem. 1.1 (b)]{Sinnott} this index is given by the $\C_p$-determinant of $(\det f)^{-1}$. \\

Define $f$ to be the $\C_p [\cG]$-linear extension of $1 \mapsto \theta^{(r_T)}_{L, S, \Sigma} (0)^{-1} \cdot  \lambda_{L, S, \Sigma} ( u_1 \wedge \dots \wedge u_{r_T})$, then this map has the desired properties. Calculating the determinant of multiplication by $\theta^{(r_T)}_{L, S, \Sigma} (0)^{-1}$ with respect to the basis $\{ e_\chi \}_{\chi\in \widehat{\cG}}$, we find that it equals
\[ \prod_{\chi \in \widehat{G}} L^{(r_T)}_{L | K, S, \Sigma} (\overline{\chi}, 0)^{-1} = \zeta_{L, S, T}^{(r_T)} (0)^{-1} = (h_{L, S, \Sigma}\cdot R_{L, S, \Sigma})^{-1}
\]
by the analytic class number formula, where $R_{L,S,\Sigma}$ is the $(S,\Sigma)$-regulator. Finally, using \cite[Ch. III, \S 9.4, Prop. 6]{Bourbaki}, we conclude that multiplication by $\lambda_{L, S, \Sigma} ( u_1 \wedge \dots \wedge u_{r_T})$ has determinant $R_{L, S, \Sigma}$. This completes the proof of the Lemma.
\end{proofbox}

\begin{thm} \label{etnc-thm}
    Let $L$ be a totally real field. Assume that there is a unique ramified prime $\p$ in $L_\infty | K$ and that $p \nmid |\cG|$. If 
    \begin{enumerate}[label=(\roman*)]
        \item the Rubin-Stark conjecture \ref{conjectures} (a) holds for the data $(L_n | K, S, \Sigma)$ for all $n \in \N_0$,
        \item Greenberg's conjecture holds for $L$,
    \end{enumerate}
    then the equivariant Iwasawa Main Conjecture \cite[Conj. 3.1]{BKS2} for $L_\infty | K$ and the equivariant Tamagawa Number Conjecture \cite[Conj. 2.3]{BKS2} for $L | K$ both hold true for $S = S_\infty \cup S_p$.
\end{thm}

\begin{proofbox}
Since $S = S_\infty (K) \cup S_p$, the index of $\UN^{r_T}_0$ inside $\bidual^{r_T}_{\Z_p [\cG]} U_{L, S, \Sigma}$ is finite (this follows, for example, from Proposition \ref{Dodgy-Proposition} (c)). By Proposition \ref{Dodgy-Proposition} (b) and \cite[Prop.\@ 7]{CornacchiaGreither}, this index is given by
\begin{align*}
    \left( \bidual^{r_T}_{\Z_p [\cG]} U_{L, S, \Sigma} \ : \ \UN^{r_T}_0 \right) &
    = ( \Z_p [\cG] : \Fitt^0_{\Z_p [\cG]} ((A_{S, \Sigma}^\infty)^{\Gamma, \vee} ) = | (A_{S, \Sigma}^\infty)^{\Gamma, \vee} | = | (A_{S, \Sigma}^\infty)^{\Gamma} |,
\end{align*}
where we have used that $A_{S, \Sigma}^\infty$ is finite by assumption. This finiteness also implies that the Herbrand quotient of $A_{S, \Sigma}^\infty$ is trivial, hence
\[
| (A_{S, \Sigma}^\infty)^\Gamma | = | (A_{S, \Sigma}^\infty)_\Gamma | = | A_{S, \Sigma} (L) | = h_{L, S, \Sigma},
\]
where the second equality follows from applying \cite[Prop. 13.22]{Washington}. By \cite[Prop. 6.1]{Rubin96} we have $\varepsilon_{L | K, S, \Sigma} \in \UN_0^r$, so Lemma \ref{IndexFormula} implies that
\[
\UN^{r_T}_0 = \Z_p [\cG] \cdot \varepsilon_{L | K, S, \Sigma}. 
\]
It follows now from Nakayama's Lemma that the sequence $(\varepsilon_{L_n | K, S, \Sigma})_{n \geq 0}$ is a $\bLambda$-basis of $\NS^{r_T} ( L_\infty, \Z_p (1))$. 
Since we are assuming the validity of Greenberg's Conjecture, Proposition \ref{GreenbergCriterion} gives the equality $\NS^{r_T} = \NS^b$, hence the equivariant Iwasawa Main Conjecture \ref{conjectures} (c) holds true. Moreover, we know from Lemma \ref{FiniteLem} that $\im \Theta_L = \UN^b_0$, so we also find that $\UN^{r_T}_0 = \im \Theta_L$. Thus, Conjecture \ref{conjectures} (b) is valid. Since we have already observed in Remark \ref{conjectures-remark} that in this setting Conjecture \ref{conjectures} (b) coincides with \cite[Conj. 2.3]{BKS2}, this concludes the proof. 
\end{proofbox}

\subsection{$\mu$-invariant conjectures and the Tate module of elliptic curves}\label{mu-section}

In this section we use the result of Theorem \ref{pairing-theorem} to give a reformulation of the various $\mu$-vanishing conjectures. To do this, we assume that $L_\infty | L$ is the cyclotomic $\ZZ_p$-extension and consider the following Hypothesis:
\begin{hypothesis}\label{sigma-hypothesis}
    $\Sigma$ is a finite (possibly empty) set of places of $K$, disjoint from $S$, such that for every $v \in \Sigma$ the module of invariants $H^0(K_v, T)$ vanishes.
\end{hypothesis}
We caution the reader that we are not yet assuming that $\Sigma$ is chosen in such a way that Hypothesis \ref{main-hypothesis}(3) and (4) are satisfied.\\

It is then natural to formulate the following conjecture.

\begin{conjecture}\label{mu-vanishing-conjecture}
    Assume that $\Sigma$ satisfies Hypothesis \ref{sigma-hypothesis}. Then $H^2_{\Sigma, \Iw}(\cO_{L,S},T)$ is a torsion $\Lambda$-module and, furthermore, has vanishing $\mu$-invariant.
\end{conjecture}

\begin{lemma}\label{mu-vanishing-independent-lemma}
    Conjecture \ref{mu-vanishing-conjecture} is independent of the choice of $\Sigma$.
\end{lemma}

\begin{proof}
    By the definition of $\Sigma$-modified cohomology one has, for each $n \in \NN_0$, an exact sequence
    \begin{cdiagram}
        \displaystyle \bigoplus_{w \in \Sigma_{L_n}} H^1 (\kappa_w, T) \arrow{r} & H^2_\Sigma(\cO_{L_n, S}, T) \arrow{r} &  H^2(\cO_{L_n, S}, T) \arrow{r} &
        0
    \end{cdiagram}
    Fix a place $v \in \Sigma$ and recall that the complex $\bigoplus_{w \mid v} \text{R}\Gamma_\et (\kappa_w,T)$ is represented in $D(\cR[\cG_n])$ by the complex $\bigoplus_{w \mid v}\; [T \xrightarrow{1-\Frob_w^{-1}} T]$.
    In particular, Hypothesis \ref{sigma-hypothesis} implies that $\bigoplus_{w \mid v} H^1 (\kappa_w, T)$ is finite and its dimension as an $\FF_p$-vector space is bounded above by the $\cR$-rank of $\oplus_{w\mid v} T$.\\
    Since $L_\infty | L$ is the cyclotomic $\ZZ_p$-extension, there are only finitely many primes of $L_\infty$ lying above $v$. We may thus pass to the limit to deduce that $\bigoplus_{w \in \{v\}_{L_\infty}}\varprojlim_n H^1 (\kappa_w,T)$ has vanishing $\mu$-invariant.
    
    This fact now combines with the exact sequence above to imply that $H^2_{\Sigma, \Iw}(\cO_{L, S}, T)$ is $\Lambda$-torsion and, moreover, has vanishing $\mu$-invariant if and only if the same is true of $H^2_{\Iw}(\cO_{L, S}, T)$.
\end{proof}

\begin{examples}\label{mu-vanishing-examples}Lemma \ref{mu-vanishing-conjecture} shows that, under the current hypotheses, the question of whether $H^2_{\Sigma, \Iw}(\cO_{L,S}, T)$ is a torsion $\Lambda$-module is equivalent to the weak Leopoldt conjecture for the pair $(T^*(1),L_\infty)$. As for the $\mu$-invariant component of Conjecture \ref{mu-vanishing-conjecture} we make the following observations:
    \begin{liste}
        \item{When $K$ is totally real and $T = \ZZ_p(1)$, then Hypothesis \ref{sigma-hypothesis} is satisfied for any choice of $\Sigma$. The exact sequence
        \begin{cdiagram}
            0 \arrow{r} & A_{S, \Sigma}^\infty \arrow{r} &
            H^2_{\Sigma, \Iw} (\cO_{L,S},T) \arrow{r} &
            X_{S}^\infty \arrow{r} & 0
        \end{cdiagram}
        combines with Lemma \ref{mu-vanishing-independent-lemma} to imply that Conjecture \ref{mu-vanishing-conjecture} is equivalent to Iwasawa's famous conjecture on the vanishing of the $\mu$-invariant of $A^\infty$. In particular, the theorem of Ferrero-Washington implies that Conjecture \ref{mu-vanishing-conjecture} is valid whenever $L | \QQ$ is an abelian extension.
        }
        \item{Let $T = T_p(E)$ be the $p$-adic Tate module of an elliptic curve over $K$, and take $\Sigma$ to be a set of primes of $K$ disjoint from $S$. Since $E$ has good reduction at every prime in $\Sigma$ it follows that $\Sigma$ satisfies Hypothesis \ref{sigma-hypothesis}.\\
        As such, Conjecture \ref{mu-vanishing-conjecture} is equivalent to the Coates-Sujatha conjecture \cite[Conj. A]{coates-sujatha} after taking into consideration Lemma 3.2 of \textit{loc.\@ cit.}
        In particular, if $K = \QQ$ and $L | \QQ$ is any finite abelian extension such that $E(L)[p^\infty] \neq 0$ then Conjecture \ref{mu-vanishing-conjecture} is valid by \cite[Cor. 3.6]{coates-sujatha}.}
        
        \item{More generally, Lim has conjectured in \cite[Conj.\@ A]{lim} that the $\mu$-invariant of $H^2_\Iw(\cO_{L,S}, T)$ vanishes and Conjecture \ref{mu-vanishing-conjecture} is equivalent to his conjecture after taking into consideration Lemma 3.4 of \textit{loc.\@ cit.}}
    \end{liste}
\end{examples}

We can now formulate the main result of this section.

\begin{proposition}\label{mu-vanishing-result}
    Suppose that $\Sigma$ is chosen to satisfy Hypothesis \ref{sigma-hypothesis} and so that the triple $(T,L_\infty, \Sigma)$ satisfies Hypothesis \ref{main-hypothesis}. Then Conjecture \ref{mu-vanishing-conjecture} is valid if and only if $\NS^{r_T}/\NS^b$ is finitely generated as an $\cR$-module.
\end{proposition}

\begin{proof}
    At the outset we first note that if $M$ is a finitely generated torsion $\bLambda$-module, then its $\mu$-invariant vanishes if and only if $M_\p = 0$ for every singular prime $\p$ of $\bLambda$ by Lemma \ref{IwasawaInvariants}.
    To prove the Proposition we now fix such a prime $\p$ of $\bLambda$. By localising the isomorphism of Theorem \ref{pairing-theorem} at $\p$ we obtain an isomorphism
    \begin{align*}
        \Ext_{\bLambda}^1\left(\faktor{\NS^{r_T}}{\NS^b}, \bLambda\right)_\fp \cong \faktor{\bLambda_\p}{\Fitt_{\bLambda_\p}(H^2_{\Sigma, \Iw}(\cO_{L,S}, T)_\p)}.
    \end{align*}
    
    On the other hand, one knows by \cite[Prop. 5.5.13]{NSW} that $\faktor{\NS^{r_T}}{\NS^b}$ and $\Ext^1_\bLambda\left(\faktor{\NS^{r_T}}{\NS^b}, \bLambda\right)$ have the same $\mu$-invariant. 
    As such, one deduces from the above discussion that the $\mu$-invariant of $H^2_{\Sigma, \Iw}(\cO_{L,S}, T)$ vanishes if and only if the same is true of that of $\faktor{\NS^{r_T}}{\NS^b}$.
\end{proof}

Using this Proposition we can now appeal to the known validity of the Coates-Sujatha conjecture for particular elliptic curves over $\QQ$ to say something about the structure of the quotient module $\NS^{r_T}/\NS^b$.

\begin{corollary}\label{mu-vanishing-example}
    Let $K= \QQ$ and $E/\QQ$ be an elliptic curve. If $(T,L_\infty, \Sigma)$ satisfies Hypothesis \ref{main-hypothesis} and $E(L)[p^\infty] \neq 0$, then $\NS^1/\NS^b$ is a free $\ZZ_p$-module of finite rank.
\end{corollary}

\begin{proof}
    This follows by combining Proposition \ref{mu-vanishing-result} with the observation of Example \ref{mu-vanishing-examples}(ii) and noting that $\NS^1/\NS^b$ has no non-zero finite submodules.
\end{proof}

\begin{remark}
    Let $K = L = \QQ$ and let $E/\QQ$ be an elliptic curve of algebraic rank 0 with finite Tate-Shaferevich group and such that $E(\QQ) \neq 0$. Suppose that $\Sigma$ is chosen so that $(T, \QQ_\infty, \Sigma)$ satisfies Hypothesis \ref{main-hypothesis}. If $p > 7$ does not divide any of the Tamagawa numbers at primes of bad reduction for $E$ nor the order of the Tate-Shaferevich group of $E$, then Wuthrich has shown in \cite[Prop. 9.1]{wuthrich} that the fine Selmer group of $E$ over $\QQ_\infty$ is trivial.
    
    If we assume, in addition, that for every $v \in S\setminus S_\infty(\QQ)$ the group $E(\QQ_v)$ has no points of order $p$, then a straightforward argument using the Weil pairing implies that the dual of the fine Selmer group of $E$ over $\QQ_\infty$ coincides with $H^2_\Iw(\cO_{\Q,S}, T)$.
    
    Given this, the proof of Proposition \ref{mu-vanishing-result} implies that, under all the above assumptions, $\NS^1(T,\QQ_\infty) = \NS^b(T,\QQ_\infty)$.
\end{remark}

%%%%%
%Appendix
%%%%%%%%
\markboth{Appendix}{Finite places splitting in $L_\infty$}
\section*{Appendix}
\addcontentsline{toc}{section}{Appendix}



\renewcommand{\thesubsection}{\Alph{subsection}}
\renewcommand{\thethm}{(\Alph{subsection}.\arabic{thm})}
\setcounter{subsection}{0}

\subsection{Finite places splitting in $L_\infty$}


Key to the approach of the main body of the present article is the assumption that no finite place of $K$ splits completely in $L_\infty$. While this is not an incredibly strict assumption (it is always satisfied for the cyclotomic $\ZZ_p$-extension of $L$, for example), it is natural to ask whether one can weaken this hypothesis to include many other interesting situations that arise in arithmetic. 
For example, if $K = L$ is an imaginary quadratic field and $K_\infty$ is not the anti-cyclotomic $\ZZ_p$-extension of $K$, then only finitely many finite places of $K$ can split in $K_\infty$ (see, for example, \cite{emsalem}). 
In particular, to obtain a module of higher rank universal norms in this situation which incorporates all arithmetic data of interest one must modify the notion of basic rank given above to take into account these new non-archimedean places.\\

In this appendix we briefly outline how one can can do this for general $p$-adic representations. Since this exposition will contain essentially no additional ideas we prefer to prove as little as we feel is necessary and refer to existing arguments to justify the claims.\\

At the outset we adopt all notations in \S\ref{set-up-section} with the exception that $L_\infty | L$ is now a $\ZZ_p$-extension in which finitely many places (infinite or finite) of $K$ are allowed to split.

\tocless\subsubsection{$\Sigma$-modified Selmer complexes}

In this subsection we introduce a $\Sigma$-modified Selmer complex which shall prove useful throughout this appendix.\\

Denote by $S_f$ the subset of $S$ comprising the finite places. Let $V$ be a finite set of finite places of $K$, disjoint from $\Sigma$. For any abelian extension $F$ of $K$ we define the \textit{$\Sigma$-modified $V$-supported} complex $\text{R}\Gamma_{\Sigma,V}(\cO_{F,S}, T)$ to be the mapping fibre in $D(\cR[\gal{F}{K}])$ of the morphism
\begin{cdiagram}
    \text{R}\Gamma(\cO_{F,S},T) \oplus \bigoplus_{w \in (S_f\setminus V)_F} \text{R}\Gamma(F_w, T) \arrow{r}{\phi} & \bigoplus_{w \in S_{f,F}} \text{R}\Gamma(F_w, T) \oplus \bigoplus_{w \in \Sigma_{F}} \text{R}\Gamma(\kappa_w, T),
\end{cdiagram}
where $\phi$ is given by $(\oplus_{w \in S_{f,F}} \res_w, \oplus_{w \in \Sigma_{F}} \res_w)$ on the former summand and by $(\iota, 0)$ on the latter summand where $\iota$ is the natural inclusion map. \\

Given this construction, we then have the following Lemma which results from a straightforward application of the octahedral axiom.

\begin{lemma}
    There is an exact triangle
    \begin{equation}\label{selmer-complex-exact-triangle}
    \begin{tikzcd}
        \text{R}\Gamma_{\Sigma,V}(\cO_{F,S}, T) \arrow{r} &  \text{R}\Gamma_{\Sigma}(\cO_{F,S}, T) \arrow{r} &  \bigoplus_{w \in V_F} \text{R}\Gamma(F_w, T)
        \end{tikzcd}
    \end{equation}
    in $D(\cR[\cG_F])$.
\end{lemma}

We now use this Lemma to study the complex $C_n^\bullet$ defined in \S\ref{set-up-section}.

\begin{proposition} \label{SplitPrimesProp}
    Fix $n \in \NN_0$. Denote by $V_n$ the subset of $S$ consisting of those finite places that split completely in $L_n$ and write
    \begin{align*}
        Y_{n,K}(T) := \qquad \mathclap{\bigoplus_{v \in S_\infty(K) \cup V_n}} \quad H^0(K_v, T^*(1)).
    \end{align*}
    Then there is an exact sequence
    \begin{cdiagram}
        H^2_{\Sigma,V_n}(\cO_{L_n, S}, T) \arrow{r} & H^1(C^\bullet_n) \arrow{r} & \bigoplus_{v \in S_\infty(K) \cup V_n}  Y_{n,K}(T)^* \otimes_{\cR} \cR[\cG_n] \arrow{r} &  0
    \end{cdiagram}
    in which the first map is canonical and the second depends on a choice of a set of representatives of the orbits of $\Gal(L_n | K)$ on $V_n$.
\end{proposition}

\begin{proof}
    From Proposition \ref{FiniteComplex} there is a decomposition
    \begin{align*}
        H^1(C_n^\bullet) \cong H^2_{\Sigma}(\cO_{L_n, S}, T) \oplus \bigoplus_{v \in S_\infty(K)} H^0(K_v, T^\ast(1))^\ast.
    \end{align*}
    On the other hand, Tate local duality combines with the fact that $p$ is odd and the exact triangle (\ref{selmer-complex-exact-triangle}) to imply the existence of an exact sequence
    \begin{cdiagram}
        H^2_{\Sigma, V_n}(\cO_{L_n, S}, T) \arrow{r} &  H^2_{\Sigma}(\cO_{L_n, S}, T) \arrow{r} & \bigoplus_{v \in V_n} H^0(K_v, T^\ast(1))^\ast \otimes_\cR \cR[\cG_n] \arrow{r} &  0
    \end{cdiagram}
    in which the middle arrow depends on a choice of a set of representatives of the orbits of $\Gal(L_n/K)$ on $V_n$. In particular, if we write $K_n$ for the kernel of the left hand map then, since every prime in $V_n$ splits completely in $L_n$, we obtain a decomposition
    \begin{align*}
        H^1(C_n^\bullet) \cong K_n \oplus (Y_{n,K}(T)^* \otimes_{\cR} \cR[\cG_n])
    \end{align*}
    which completes the proof of the claim. 
\end{proof}


\tocless\subsubsection{Universal norms}

We write $S_\spc(L_\infty)$ for the set of places of $K$ that split completely in $L_\infty$ and remark that, by our assumptions and the standard properties of $\ZZ_p$-extensions, $S_\spc(L_\infty)$ contains all the archimedean places of $K$. We then fix a finite set of places $S$ containing
\begin{align*}
    S_\spc(L_\infty) \cup S_p(K) \cup S_\ram(T)
\end{align*}
and we suppose that the tuple $(T,L_\infty, \Sigma)$ satisfies the following mild hypotheses:

\begin{hypotheses}\label{main-hypothesis-appendix}\text{}
    \begin{enumerate}[label=(\arabic*)]
        \item{For every $n \in \NN_0$ one has that the module of invariants $H^0_\Sigma(L_n, T)$ vanishes.}
        \item{The $\cR$-free module $Y_K(T) = \bigoplus_{v \in S_\spc(L_\infty)} H^0(K_v,T^*(1))$ has non-zero rank $r_T$.}
        \item{$H^1_\Sigma(\cO_{L_n, S}, T)$ is $\cR$-torsion-free for every $n \in \N_0$}.
    \end{enumerate}
\end{hypotheses}

For every subextension $F$ of $L_\infty | K$ we write
\begin{align*}
    H^2_{\Sigma,\spc}(\cO_{F, S}, T) := H^2(R\Gamma_{\Sigma, S_\spc(L_\infty)}(\cO_{F,S}, T)).
\end{align*}

Now fix $n \in \NN_0$. By considering the sum of primitive idempotents in $\cQ[\cG_n]$ that annihilate the kernel $K_n := \ker(H^2_{\Sigma,\spc}(\cO_{L_n,S}, T) \to H^1(C_n^\bullet))$ of the map of Proposition \ref{SplitPrimesProp} one can construct, as in \S\ref{finite-level-section}, a projection map
\begin{align*}
    \Theta_{L_n} \: \Det_{\cR[\cG_n]}(C_{L_n}^\bullet) \to \bidual_{\cR[\cG_n]}^{r_T} H^1_\Sigma(\cO_{L_n,S},T).
\end{align*}

In particular, if there exists $n \in \NN$ for which the projection map $\Theta_{L_n}$ is non-zero, then the argument of \cite[Prop. 4.30(i)]{EulerSystemsSagaIII} can be used to show that $\varprojlim_n K_n$ is a torsion $\bLambda$-module.\\

Given this, it is straightforward to check that the arguments used to prove Theorem \ref{UN-structure-theorem} and Theorem \ref{pairing-theorem} give analogous results under the running hypotheses of this appendix.

\subsection{Equivariant Iwasawa algebras}
\label{AppendixIwasawa}

This appendix serves the purpose of collecting useful facts on rings of the form $\bLambda = \cR \llbracket \Gamma \rrbracket [G]$, where 
\begin{itemize}
\item $\cR$ is the ring of integers in a finite extension $\mathcal{Q}$ of $\mathbb{Q}_p$,
\item $\Gamma \cong \Z_p$
\item $G$ is a finite abelian group.
\end{itemize}

We shall also write $\Gamma^n$ for the unique subgroup of $\Gamma$ of order $p^n$ and $\Gamma_n$ for the quotient $\faktor{\Gamma}{\Gamma^n}$. Moreover, we set $\Lambda = \cR \llbracket \Gamma \rrbracket$. 

\tocless\subsubsection{Basic properties}



Let $P$ be the $p$-Sylow subgroup of $G$ and write $H \cong \faktor{G}{P}$. Then we have a character-part decomposition
\[
\bLambda \cong \bigoplus_{\chi \in \widehat{H} / \sim} \Lambda_\chi \quad \text{ with } \Lambda_\chi = \cR [\im \chi] \llbracket \Gamma \rrbracket [P].
\]
Here the relation $\sim$ is defined as $\chi \sim \chi'$ if there is a $\sigma \in G_\mathcal{Q}$ such that $\chi = \sigma \circ \chi'$. 

\begin{lem} \label{LambdaGorenstein}
The ring $\bLambda$ is Gorenstein.
\end{lem}

\begin{proofbox}
Let $\gamma$ be a topological generator of $\Gamma$. For every $\chi \in \widehat{H} / \sim$, the quotient $\faktor{\Lambda_\chi}{(\gamma - 1)} \cong \bigO [\im \chi] [P]$ is a finite group ring and is therefore Gorenstein by \cite[10.29]{CurtisReiner}. Since $\gamma - 1$ is a non-zero divisor in $\Lambda_\chi$, this implies that $\Lambda_\chi$ is Gorenstein as well. \\
If $\p \subseteq \bLambda$ is any prime ideal, then the localisation $\bLambda_\p$ is given by the sum $\bigoplus_{\chi \in \widehat{H} / \sim} (\Lambda_\chi)_{\p_\chi}$, where $\p_\chi$ denotes the projection of $\p$ onto $\Lambda_\chi$. Each summand $(\Lambda_\chi)_{\p_\chi}$ is already a local ring, so we must have $\bLambda_\p = (\Lambda_\chi)_{\p_\chi}$ for a certain character $\chi$. Since $\Lambda_\chi$ is a Gorenstein ring, this shows that $\bLambda$ is Gorenstein as well. 
\end{proofbox}


We also note that for any $\bLambda$-module $M$, we have a natural isomorphism of $\bLambda$-modules
\[
\Hom_\Lambda (M, \Lambda) \xrightarrow{\sim} \Hom_\bLambda (M, \bLambda)^\#, \quad f \mapsto \Big \{ m \mapsto \sum_{\sigma \in G} \sigma^{-1} f(\sigma \cdot m) \Big \}
\]
that extends to give an isomorphism
\begin{equation} \label{Exts}
\Ext^i_\Lambda (M, \Lambda) \cong \Ext^i_\bLambda (M, \bLambda)^\# \quad \text{ for all } i \geq 0. 
\end{equation}


\tocless\subsubsection{Height one primes} \label{HeightOnePrimesSection}

The aim of this section is to describe the relation between the classical Iwasawa $\lambda$ and $\mu$-invariants of a module $M$ and the localisation of $M$ at height one primes of $\bLambda$. In order to do this, it is useful to make the following distinction among the latter. 

\begin{definition}
A height one prime $\p$ of $\bLambda$ is called \emph{singular} if $p \in \p$ and \emph{regular} otherwise. 
\end{definition}

If $\p$ is regular, then $\bLambda_\p$ is identified with the localisation of $\bLambda [\tfrac{1}{p}]$ at $\p \bLambda [\tfrac{1}{p}]$. From the decomposition
\[
\bLambda [\tfrac{1}{p}] = \bigoplus_{\chi \in \widehat{G} / \sim} \Lambda_\chi [\tfrac{1}{p}] \qquad
\text{ with } \Lambda_\chi [\tfrac{1}{p}] = \bigO [\im \chi] \llbracket \Gamma \rrbracket [\tfrac{1}{p}]
\]
we deduce via an argument similar to the one used in the proof of Lemma \ref{LambdaGorenstein} that the localisation $(\bLambda [\tfrac{1}{p}])_{\p \bLambda [\frac{1}{p}]}$ agrees with $(\Lambda_\chi [\tfrac{1}{p}])_{\p\Lambda_\chi [\frac{1}{p}] }$ for a certain character $\chi$. 
Note that $\Z_p [\im \chi] \llbracket \Gamma \rrbracket$ is a conventional Iwasawa algebra. In particular, its localisation $\Z_p [\im \chi] \llbracket \Gamma \rrbracket [\tfrac{1}{p}]$ is also a regular local domain. It follows that $\bLambda_\p$ is a discrete valuation ring. 
\\
If $\p$ is such that $\bLambda_\p = (\Lambda_\chi [\tfrac{1}{p}])_{\p\Lambda_\chi [\frac{1}{p}] }$ for the character $\chi \in \widehat{G} / \sim$, then we say the character $\chi$ is \textit{associated} to $\p$. Similarly, if $\p$ is a singular prime and $\bLambda_\p = (\Lambda_\chi)_{\p \Lambda_\chi}$ for some character $\chi \in \widehat{H} /\sim$, then we call $\chi$ \text{associated} to $\p$ as well. It should always be clear from the context what is meant. 


\begin{lem} \label{IwasawaInvariants}
Let $M$ be a finitely generated $\bLambda$-torsion module.
\begin{liste}
\item The $\mu$-invariant of $M$ vanishes if and only if $M_\p = 0$ for every singular prime $\p$ of $\bLambda$.
\item The module $M$ is finite (equivalently, its $\mu$ and $\lambda$-invariant both vanish) if and only if $M_\p = 0$ for every height one prime $\p$ of $\bLambda$. 
\end{liste}
\end{lem}

\begin{proofbox}
The `only if' part of (a) follows from \cite[Lem. 5.6]{Flach} (see also \cite[Lem. 6.3]{BurnsGreither}). As for the converse implication, first observe that there is an embedding $M \subseteq \bigoplus_{\chi \in \widehat{\cG}} M_\chi$. Given $\chi \in \widehat{\cG}$, let $\p_\chi$ be the singular prime whose associated character is the prime-to-$p$ part of $\chi$. Then by \textit{loc.\@ cit.}\@ one knows that the $\mu$-invariant of $M_\chi$ (as an $\cR(\chi)\llbracket\Gamma\rrbracket$-module) vanishes. In other words, $M_\chi$ is finitely generated as an $\cR(\chi)$-module. But each $\cR(\chi)$ is a finitely generated $\cR$-algebra. From this one deduces that $M$ is itself finitely generated as an $\cR$-module and so has vanishing $\mu$-invariant.\\

For part (b) we first note that if the $\lambda$-invariant of $M$ vanishes then $M$ is necessarily $\ZZ_p$-torsion (\textit{c.f.}\@ \cite[Rem.\@ 3 following Def. 5.3.9]{NSW}). It then follows that $M_\p = 0$ for any regular height one prime $\p$ of $\bLambda$. In combination with (a) this gives one direction of assertion (b). \\
Conversely, assume that $M_\p = 0$ for any height one prime $\p$ of $\bLambda$. Appealing to \cite[Lemma B.11]{Sakamoto20} one sees that
\[
0 = \Ext^1_\bLambda (M, \bLambda ) \cong \Ext^1_\Lambda (M, \Lambda ). 
\]
This combines with \cite[Prop. 5.5.8(iv)]{NSW} to imply that $M$ is finite as desired.
\end{proofbox}

\subsection{Exterior biduals}
\label{AppendixBiduals}

In this appendix we both survey the existing theory of exterior biduals and also establish the new results concerning these objects that are needed in the main body of this article. 

\tocless\subsubsection{Definition and general properties}


Let $R$ be a commutative Noetherian ring, and for any $R$-module put $M^\ast = \Hom_R (M, R)$. 

\begin{definition}
Let $M$ be an $R$-module. 
For any integer $r \geq 0$ we define the  $r$-th \emph{exterior bidual} of $M$ as
\[
\bidual^r_R M = \left ( \exprod^r_R M^\ast \right )^\ast.
\]
In particular, the symbol $\bidual$ does \textit{not} refer to an intersection in this context. 
\end{definition}

This definition is inspired by the notion of \textit{Rubin lattice} introduced in \cite{Rubin96}, and first appeared in the above formalised form in \cite{EulerSystemsSagaI}. See \cite[Remark A.9]{EulerSystemsSagaII} for the relation between these two definitions.

\paragraph{Some maps}
Let us now introduce a collection of morphisms that is ubiquitous throughout the theory of exterior biduals. For this purpose, let $M$ and $N$ be $R$-modules. For any integer $r \geq 1$ and morphism $f \in \Hom_R (M, N)$ we define 
\[ 
f^{(r)} \: \exprod^r_R M \to N \otimes_R \exprod^{r - 1}_R M
\]
by
\[
f^{(r)} (m_1 \wedge \dots \wedge m_r) = \sum_{i = 1}^r ( - 1)^{i + 1} f (m_i) \otimes m_1 \wedge \dots \wedge \widehat{m_{i}} \wedge \dots \wedge m_r, 
\]
where we write $\widehat{m_i}$ to mean omission of this particular coefficient. By abuse of notation, we shall simply denote $f^{(r)}$ by $f$ as well. \\

For any integer $s \leq r$ this construction 
(specialised at $N = R$
) 
induces a natural morphism
\[
\exprod^r_R M^\ast \to \Hom_R \left( \exprod^r_R M, \; \exprod^{r - s}_R M\right), \quad
f_1 \wedge \dots \wedge f_r \mapsto f_r \circ \dots \circ f_1
\]
which is sometimes referred to as the \textit{rank reduction} by $f$. 
If $\mathfrak{S}_r$ is the permutation group on $r$ elements, then a more explicit description of the above map is given by 
\begin{equation} \label{ExplicitFormula}
f_1 \wedge \dots \wedge f_r \mapsto 
\bigg \{ m_1 \wedge \dots \wedge m_r \mapsto 
\sum_{\sigma \in \mathfrak{S}_{r,s}} \text{sgn} (\sigma) \det( f_i (m_{\sigma_{j}})) m_{\sigma (s +1)} \wedge \dots \wedge m_{\sigma (r)}
\bigg \},
\end{equation}
where $\mathfrak{S}_{r,s} = \{ \sigma \in \mathfrak{S}_r \mid \sigma (1) < \dots < \sigma (s) \text{ and } \sigma (s + 1) < \dots < \sigma (s) \}$. By virtue of this map, we shall regard any element of $\exprod^s_R M^\ast$ as an element of $\Hom_R (\exprod^r_R M, \; \exprod^{r - s}_R M^\ast)$. 

\paragraph{General properties}

Let $N$ and $M$ be $R$-modules, and $r \geq 0$ an integer. 
The following properties are immediate from the properties of the functors $\exprod^r_R -$ and $\Hom_R ( -, R)$:
\begin{itemize}
\item $\bidual^0_R M = R$ and $\bidual^1_R M = M^{\ast \ast}$.
\item Every morphism of $R$-modules $N \to M$ induces a morphism $\bidual^r_R N \to \bidual^r_R M$.
\item There is a natural morphism
\[ 
\xi^r_M \: \exprod^r_R M \to \bidual^r_R M, \quad m \mapsto \{ f \mapsto f (a) \}
\]
which is neither injective nor surjective in general. For any $s \leq r$ and $f \in \exprod^s_R M^\ast$ the morphism $\xi^r_M$ fits into a commutative diagram
\begin{cdiagram}
\exprod^r_R M \arrow{r}{f} \arrow{d}{\xi^r_M} & \exprod^{r - s}_R M \arrow{d}{\xi^{r - s}_M} \\
\bidual^r_R M \arrow{r}{f} & \bidual^{r - s}_R M,
\end{cdiagram}
where the bottom map is define as the dual of 
\[
\exprod^{r - s}_R M^\ast \to \exprod^r_R M^\ast, \quad g \mapsto f \wedge g. 
\]
\end{itemize}

\begin{lem}
For any finitely generated projective $R$-module $P$, the map $\xi^r_M$ is an isomorphism.
\end{lem}

\begin{proofbox}
This is proved in \cite[Lem. A.1]{EulerSystemsSagaI}.
\end{proofbox}


\tocless\subsubsection{Functoriality aspects}

We follow \cite[Appendix B3]{Sakamoto20} in considering the following two conditions on $R$:
\begin{itemize}
\item[($G_n$)] The ring $R$ is said to satisfy the condition $(G_n)$ if the localisation $G_\p$ is Gorenstein for any prime $\p \in \Spec R$ of height $\text{ht} (\p) \leq n$. 
\item[($S_n$)] The ring $R$ is said to satisfy Serre's condition $(S_n)$ if $\text{depth } R_\p \geq \text{inf} (n, \text{ht} (\p) )$ holds for any prime ideal $\p \in \Spec R$. 
\end{itemize}

\begin{rk}
For example, $R$ is Cohen-Macaulay if and only if $R$ satsfies $(S_n)$ for all $n \in \N_0$. In particular, Gorenstein rings satisfiy both $(G_n)$ and $(S_n)$ for all $n \in \N_0$. Examples of Gorenstein rings include equivariant Iwasawa algebras (see Lemma \ref{LambdaGorenstein}) and finite group rings (see \cite[p. 779]{CurtisReiner}).
\end{rk}


\begin{lem} \label{RyotarosLemma1}
Suppose that $R$ satisfies both $(G_0)$ and $(S_1)$. If $N \hookrightarrow M$ is an injective morphism of $R$-modules, then for any integer $r \geq 1$ the induced map
\[
\bidual^r_R N \to \bidual^r_R M
\] 
is injective as well. 
\end{lem}

\begin{proof} See \cite[Lemma C.1]{Sakamoto20}.
\end{proof}


\begin{lem} \label{BidualsLimits}
Suppose that $R$ satisfies $(S_2)$ and $(G_1)$ and admits a presentation as an inverse limit $R = \varprojlim_{i \in I} R_i$ $(R_i)_{i \in I}$ of Noetherian rings satisfying $(S_2)$ and $(G_1)$.
Let $C^\bullet$ be a perfect complex of $R$-modules and assume (without loss of generality) that $H^i (C^\bullet) = 0$ for $i < 0$. Then there is an isomorphism
\[
\bidual^r_R H^0 (C^\bullet) \stackrel{\simeq}{\longrightarrow} \varprojlim_{i \in I} \bidual^r_{R_i} H^0 (C^\bullet \otimes^\mathbb{L}_R R_n ).
\]
\end{lem}

\begin{proofbox}
See \cite[Lemma B.15]{Sakamoto20}.
\end{proofbox}

\begin{lem} \label{BaseChangeLem}
Let $R$ be a Noetherian ring, and let $R \to R'$ be a ring morphism that endows $R'$ with the structure of an $R$-module of projective dimension at most one. If $M$ is an $R$-module such that $\Ext^1_R (M, R) = 0$, then there is a natural injection
\[
\left ( \bidual^r_R M \right ) \otimes_R R' \hookrightarrow \bidual^r_{R'} ( M \otimes_R R'). 
\]
\end{lem}

\begin{rk}
Unlike exterior powers, exterior biduals are not in general compatible with base change (see \cite[Rem. B.9]{Sakamoto20} for a short discussion). Lemma \ref{BaseChangeLem} is therefore an exceptional phenomenon.
\end{rk}


\textit{Proof of \ref{BaseChangeLem}}:
Since $R'$ is of projective dimension at most one, we have an exact sequence
\begin{cdiagram}
   0 \arrow{r} & P_2 \arrow{r} & P_1 \arrow{r} & R' \arrow{r} & 0,
\end{cdiagram}
where $P_1$ and $P_2$ are projective $R$-modules. We therefore have a commutative diagram
\begin{cdiagram}
   & \Hom_R (M, R) \otimes_R P_2 \arrow{d}{\simeq} \arrow{r} & \Hom_R (M, R) \otimes_R P_1 \arrow{d}{\simeq} \arrow{r} & \Hom_R (M, R) \otimes_R R' \arrow{d} \arrow[r] &0\\
   0 \arrow{r} & \Hom_R (M, P_2) \arrow{r} & \Hom_R (M, P_1) \arrow{r} & \Hom_R (M, R')  &.
\end{cdiagram}
Since $P_2$ is a direct summand of a free $R$-module, the assumption $\Ext^1(M,R) = 0$ implies that the bottom row of this diagram is exact on the right. Now, the exactness of the bottom row tells us that the top row is also exact on the left, so the snake lemma yields an isomorphism
\begin{cdiagram}
 \Hom_R (M, R) \otimes_R R' \cong \Hom_R (M, R') .
\end{cdiagram}
Taking $R'$-exterior powers and $R'$-duals, we deduce an isomorphism
\begin{align*}
\Hom_{R'} \left( \left(\exprod^r_{R} M^\ast \right) \otimes_R R', \; R'\right) & = \Hom_{R'} \left( \exprod^r_{R'} M^\ast \otimes_R R', \; R'\right) \\
& \cong \Hom_{R'} \left( \exprod^r_{R'} \Hom_R (M, R'), \; R'\right).
\end{align*}
The tensor-hom adjunction implies that
\begin{equation} \label{adjunction}
    \Hom_R (M, R') = \Hom_R (M, \Hom_{R'} (R', R')) = \Hom_{R'} ( M \otimes_R R', R'),
\end{equation}
so we infact have an isomorphism
\[
\Hom_{R'} \left( \left(\exprod^r_{R} M^\ast \right) \otimes_R R', \; R'\right) \cong \bidual^r_{R'} ( M \otimes_R R').
\]
Using (\ref{adjunction}) again, this time for the left hand side, we can restate this as 
\[
\Hom_{R} \left( \exprod^r_{R} M^\ast, \; R'\right)\cong \bidual^r_{R'} ( M \otimes_R R').
\]
By repeating the process that led to the commutative diagram that was used at the beginning of the proof, we also obtain an injection
\[
\left(\bidual^r_R M \right) \otimes_R R' = \Hom_R \left( \exprod^r_R M^\ast, \; R\right) \otimes_R R' \hookrightarrow \Hom_R \left( \exprod^r_R M^\ast, \; R'\right)
\]
which in combination with the previously obtained isomorphism shows the Lemma. 
\qed




\tocless\subsubsection{Further properties}

\begin{lem} \label{LittleLemma}
Suppose that $R$ satisfies $(G_1)$ and $(S_2)$.
Let $s \geq 1$ be an integer and
\begin{cdiagram}
   0 \arrow{r} & X \arrow{r} & Y \arrow{r}{\bigoplus_{i = 1}^s \varphi_i} & R^{\oplus s} \arrow{r} & Z \arrow{r} & 0
\end{cdiagram}
an exact sequence of $R$-modules, where $Y$ is a free $R$-module of rank $d$. Fix an integer $r$ such that $s \leq r \leq d$ and consider the map 
\[
\varphi = \bigwedge_{1 \leq i \leq s} \varphi_i \: \exprod^r_R Y \to \exprod^{r - s}_R Y. 
\]
Then the following hold:
\begin{liste}
\item There exists an exact sequence
\begin{cdiagram}
0 \arrow{r} & \bidual^r_R X \arrow{r} & \exprod^r_R Y \arrow{r}{\oplus_{i = 1}^s \varphi_i} & \displaystyle \bigoplus_{i = 1}^s \exprod^{r - 1}_R Y .
\end{cdiagram}
\item We have an inclusion
\[
\im \varphi \subseteq \bidual^{r - s}_R X,
\]
where we regard $\bidual^{r - s}_R X$ as a submodule of $\bidual^{r-s}_R Y = \exprod^{r - s}_R Y$ via Lemma \ref{RyotarosLemma1}.
\item There is an equality 
\[
\left\{ f (a) \mid a \in \im \varphi, \; f \in \exprod^{r - s}_R X^\ast \right\} = \Fitt^{s}_R (Z). \]
\end{liste}
\end{lem}

\begin{proofbox}
Part (a) is \cite[Lem. B.12]{Sakamoto20}.  For (b) we first note that $\varphi$ is the dual of the map
\[
\exprod^r_R Y^\ast \to \exprod^{r + s}_R Y^\ast, \quad f \mapsto (\varphi_1 \wedge \dots \wedge \varphi_s) \wedge f.
\]
The kernel of this map is the submodule generated by elements of the form $g \wedge h$, where $g \in \langle \varphi_i \mid 1 \leq i \leq s \rangle_R$ and $h \in \exprod^{r - 1}_R Y^\ast$. We therefore have an exact sequence
\begin{equation} \label{AnExactSequence}
\begin{tikzcd}
\langle \varphi_i \mid 1 \leq i \leq s \rangle_R \otimes_R \exprod^{r - s - 1}_R Y^\ast \arrow{r} & \exprod^{r - s}_R Y^\ast \arrow{r} & \exprod^{r}_R Y^\ast.
\end{tikzcd}
\end{equation}
Dualising the exact sequence
\begin{cdiagram}
0 \arrow{r} & \im \bigoplus_{i = 1}^s \varphi_i \arrow{r} & R^{ \oplus s} \arrow{r} & Z \arrow{r} & 0
\end{cdiagram}
yields the sequence
\vspace{-0.4em}
\begin{cdiagram}
R^{\oplus s} \arrow{r} & \Big( \im \bigoplus_{i = 1}^s \varphi_i\Big)^\ast \arrow{r} & \Ext^1_R (Z, R) \arrow{r} & 0.
\end{cdiagram}
\vspace{-0.4em}
The image of the map on the left is exactly the module $\langle \varphi_i \mid 1 \leq 1 \leq s \rangle_R$, so dualising a second time yields
\begin{cdiagram}
0 \arrow{r} & \Ext^1_R (Z, R)^\ast \arrow{r} & \Big( \im \bigoplus_{i = 1}^s \varphi_i\Big)^{\ast \ast} \arrow{r} &\langle \varphi_i \mid 1 \leq 1 \leq s \rangle_R^\ast.
\end{cdiagram}
Now, if $\p \in \Spec(R)$ is a prime of height $\text{ht } \p \leq 1$, then 
\[ 
\Ext^1_R (Z, R)_\p = \Ext^2_R (\im  \textstyle \bigoplus_{i = 1}^s \varphi_i, \; R)_\p = 0 
\] 
since by assumption $R$ satisfies ($G_1$) and so $R_\p$ has injective dimension 1. That is, $\Ext^1_R (Z, R)$ is a pseudo-null module and it follows from \cite[Lem. B.11]{Sakamoto20} that $\Ext^1_R (Z, R)^\ast = 0$. We therefore get a commutative diagram
\begin{cdiagram}
0 \arrow{r} & \bidual^{r - s}_R X \arrow{r} \arrow[dashed]{d} & \exprod^{r - s}_R Y \arrow{r} & \Big( \im \bigoplus_{i = 1}^s \varphi_i \Big) \otimes_R \exprod^{r - s}_R Y \arrow[hookrightarrow]{d} \\
0 \arrow{r} & A^\ast \arrow{r} & \exprod^{r  - s}_R Y  \arrow[equal]{u} \arrow{r} & \langle \varphi_i \mid 1 \leq i \leq s \rangle_R^\ast \otimes_R \exprod^{r - s}_R Y,
\end{cdiagram}
where the top line comes from (a), the bottom line is obtained from (\ref{AnExactSequence}) by setting $A = \im \{ \exprod^{r - s}_R Y^\ast \to \exprod^{r}_R Y^\ast \}$ and dualising, and the injectivity of the map on the right holds because the map $ \im \bigoplus_{i = 1}^s \varphi_i \to ( \im \bigoplus_{i = 1}^s \varphi_i)^{\ast \ast}$ is injective. 
Applying the snake lemma to this diagram then gives $A^\ast \cong \bidual^{r-s}_R X$. From the factorisation
\begin{cdiagram}
A^\ast  \arrow[hookrightarrow]{r} & \exprod^{r  - s}_R Y^\ast \\
\exprod^{r}_R Y^\ast \arrow{u} \arrow{ru} & 
\end{cdiagram}
it follows that indeed $\im \varphi \subseteq \bidual^{r - s}_R X$. As for (c), the proof of \cite[Prop. A.2 (ii)]{EulerSystemsSagaI}
shows that
\[
\left\{ f (a) \mid a \in \im \varphi, \; f \in \exprod^{r - s}_R Y^\ast \right\} = \Fitt^{s}_R (Z).
\]
Since $\im \varphi \subseteq \bidual^{r - s}_R X$, this set is certainly contained in $\{ f (a) \mid a \in \im \varphi, \; f \in \exprod^{r - s}_R X^\ast \}$.\\
We shall therefore show the reverse inclusion.
We first remind the reader that for any $x \in \bidual^{r - s}_R X$ and $f \in \exprod^{r - s}_R X^\ast$, the value $f(x) \in R$ is given by $x (f)$, where $x$ is regarded, true to its definition, as a map $\exprod^{r - s}_R X^\ast \to R$.\\
We have seen above that $\bidual^{r - s}_R X = A^\ast$, so we can regard $x$ as an element of $A^\ast$, and so the value $x (f)$ coincides with $x$ evaluated at the restriction $f_{\mid_A}$ of $f$ to $A$. By construction, the map $\exprod^{r - s}_R Y^\ast \to A$ is surjective, so we can lift $f_{\mid_A}$ to an element $g$ of $\exprod^{r - s}_R Y^\ast$ as wanted. 
\end{proofbox}


\renewcommand{\emph}[1]{\textit{#1}}

\pagestyle{special}
\vspace{-1.5em}
\addcontentsline{toc}{section}{References}
\tiny
\printbibliography


\small

%\vspace{2cm}

\textsc{King's College London,
Department of Mathematics,
London WC2R 2LS,
United Kingdom} \\
\textit{Email address:} \href{mailto:dominik.bullach@kcl.ac.uk}{dominik.bullach@kcl.ac.uk}\\

\textsc{King's College London,
Department of Mathematics,
London WC2R 2LS,
United Kingdom}\\
\textit{Email address:} \href{mailto:alexandre.daoud@kcl.ac.uk}{alexandre.daoud@kcl.ac.uk}

\end{document}