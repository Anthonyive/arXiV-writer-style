\documentclass[reqno]{amsart}
\usepackage{amsmath,amsrefs,amssymb}

\usepackage[colorlinks=true, pdfborder={0 0 0}]{hyperref}

\numberwithin{equation}{section}

\DeclareMathOperator{\divergence}{div}
\DeclareMathOperator{\loc}{loc}
\DeclareMathOperator{\Scal}{S}
\DeclareMathOperator{\Ricci}{Ric}
\DeclareMathOperator{\Riemann}{R}
\DeclareMathOperator{\Weyl}{W}
\DeclareMathOperator{\Schouten}{P}
\DeclareMathOperator{\Bach}{B}
\DeclareMathOperator{\Beta}{B}
\DeclareMathOperator{\Sym}{Sym}
\DeclareMathOperator{\Vol}{Vol}
\DeclareMathOperator{\Tr}{Tr}
\DeclareMathOperator{\bigO}{O}
\DeclareMathOperator{\smallo}{o}
\newcommand{\R}{\mathbb{R}}
\renewcommand{\S}{\mathbb{S}}
\newcommand{\N}{\mathbb{N}}
\newcommand{\Z}{\mathbb{Z}}
\newcommand{\<}{\left<}
\renewcommand{\>}{\right>}
\renewcommand{\[}{\left[}
\renewcommand{\]}{\right]}
\renewcommand{\(}{\left(}
\renewcommand{\)}{\right)}

\newtheorem{theorem}{Theorem}[section]
\newtheorem{proposition}{Proposition}[section]
\newtheorem{corollary}{Corollary}[section]
\newtheorem{step}{Step}[section]

\begin{document}

\title[Higher-order $Q$-curvature equation]{Existence result for the higher-order $Q$-curvature equation}

\author{Saikat Mazumdar}

\address{Saikat Mazumdar, Department of Mathematics, Indian Institute of Technology Bombay, Mumbai 400076, India}
\email{saikat@math.iitb.ac.in, saikat.mazumdar@iitb.ac.in}


\author{J\'er\^ome V\'etois}

\address{J\'er\^ome V\'etois, Department of Mathematics and Statistics, McGill University, 805 Sherbrooke Street West, Montreal, Quebec H3A 0B9, Canada}
\email{jerome.vetois@mcgill.ca}

\thanks{The second author was supported by the Discovery Grant RGPIN-2016-04195 from the Natural
Sciences and Engineering Research Council of Canada. This work was initiated when the first author held a postdoctoral position at McGill University under the co-supervision of Professors Pengfei Guan, Niky Kamran and the second author, that was partially supported by the NSERC Discovery Grants RGPIN-04443-2018, RGPIN-05490-2018 and RGPIN-04195-2016.}

\date{July 16, 2020}

\begin{abstract}
We obtain an existence result for the $Q$-curvature equation of arbitrary order $2k$ on a closed Riemannian manifold of dimension $n\ge 2k+4$, where $k\ge1$ is an integer. We obtain this result under the assumptions that the operator is coercive and its Green's function is positive, which are satisfied for instance when the manifold is Einstein.
\end{abstract}

\maketitle

\section{Introduction and main results}\label{Sec1}

Given an integer $k\ge1$, a smooth, closed Riemannian manifold $\(M,g\)$ of dimension $n>2k$ and a smooth positive function $f$ in $M$, we consider the equation
\begin{equation}\label{Eq1}
P_{2k}u=f\left|u\right|^{2^*_k-2}u\quad\text{in M},
\end{equation}
where $P_{2k}$ is the GJMS operator with leading part $\Delta^k$, $\Delta:=\delta d$ is the Laplace--Beltrami operator with nonnegative eigenvalues and $2^*_k:=2n/\(n-2k\)$ is the critical Sobolev exponent. The so-called GJMS operators were discovered by Graham, Jenne, Mason and Sparling~\cite{GraJenMasSpa} by using a construction based on the Fefferman--Graham ambient metric~\cites{FefGra1,FefGra2}. They provide a natural extension to higher orders of the Yamabe operator~\cite{Yam} ($k=1$) and the Paneitz--Branson operator~\cites{Bra1,Pan} ($k=2$). When $u$ is positive, \eqref{Eq1} arises in the problem of prescribing \nobreak Branson's $Q$-curvature of order $2k$ in a given conformal class (see Branson~\cites{Bra2}). More precisely, the positive solutions $u$ to the equation \eqref{Eq1} correspond to the conformal metrics $u^{4/\(n-2k\)}g$ with $Q$-curvature of order $2k$ equal to $\frac{2}{n-2k}f$. 

\smallskip
Throughout this paper, we assume that the operator $P_{2k}$ is {\it coercive} in the sense that there exists a constant $C>0$ such that 
$$\int_MuP_{2k}u\,dv_g\ge C\int_Mu^2dv_g$$ 
for all functions $u\in C^{2k}\(M\)$, where $dv_g$ is the volume element with respect to $g$.

\smallskip
The existence of at least one positive solution to the equation \eqref{Eq1} with $f\equiv1$ has been completely solved in the case where $k=1$ (see the historic work of Aubin~\cite{Aub}, Schoen~\cite{Sch1}, Trudinger~\cite{Tru} and Yamabe~\cite{Yam}) and in the case of locally conformally flat manifolds for $k\ge1$ (see Schoen~\cite{Sch2} for $k=1$ and Qing and Raske~\cite{QingRas} for $k\ge2$). This question has also been solved to a large extent in the case where $k=2$ (see Djadli, Hebey and Ledoux~\cite{DjaHebLed}, Esposito and Robert~\cite{EspRob}, Gursky, Hang and Lin~\cites{GurHangLin}, Gursky and Malchiodi~\cite{GurMal} Hang and Yang~\cites{HangYang1,HangYang2} and Robert~\cite{Rob1}). For $k=3$, Chen and Hou~\cite{ChenHou} obtained the existence of at least one solution of \eqref{Eq1} in the case of non-locally conformally flat manifolds of dimension $n\ge 10$. The solution obtained by Chen and Hou~\cite{ChenHou} is positive under the assumption that the Green's function of the operator $P_6$ is positive (see Mazumdar~\cite{Maz}*{Theorem~3}). More general existence results have also been obtained in the case where $f\not\equiv1$ (see among others Aubin~\cite{Aub}, Escobar and Schoen~\cite{EscSch}, Hebey~\cite{Heb} and Hebey and Vaugon~\cite{HebVau} for $k=1$, Djadli, Hebey and Ledoux~\cite{DjaHebLed}, Esposito and Robert~\cite{EspRob} and Robert~\cite{Rob1} for $k=2$, Chen and Hou~\cite{ChenHou} for $k=3$ and Robert~\cite{Rob2} for higher orders). 

\smallskip
We let $\Ricci$ and $\Weyl$ be the Ricci and Weyl curvature tensors of $\(M,g\)$, $\left|\Weyl\right|$ be the norm of $\Weyl$ with respect to $g$ and $\(\cdot,\cdot\)$ be the multiple inner product defined as $\(S,T\)=g^{i_1j_1}\dotsm g^{i_lj_l}S_{i_1\dotsc i_l}T_{j_1\dotsc j_l}$ for all covariant tensors $S$ and $T$ of rank $l\ge1$. Our main result is the following, which extends the results of Aubin~\cite{Aub} ($k=1$), Esposito and Robert~\cite{EspRob} (k=2) and Chen and Hou~\cite{ChenHou} (k=3):

\begin{theorem}\label{Th}
Let $k\ge1$ be an integer, $\(M,g\)$ be a smooth, closed Riemannian manifold of dimension $n\ge 2k+4$ and $f$ be a smooth positive function in $M$. Assume that the operator $P_{2k}$ is coercive and there exists a maximal point $\xi$ of $f$ such that
\begin{equation}\label{ThEq1}
\Delta f\(\xi\)=0\text{ and }\left|\Weyl\(\xi\)\right|^2f\(\xi\)+c\(n,k\)\(\Delta^2f\(\xi\)+2\(\Ricci\(\xi\),\nabla^2f\(\xi\)\)\)>0,
\end{equation}
where $c\(n,k\)$ is a constant depending only on $n$ and $k$ such that $c\(n,k\)=0$ if $n=2k+4$ and $c\(n,k\)>0$ if $n>2k+4$ (see \eqref{ThEq4} for the value of $c\(n,k\)$). Then there exists a nontrivial solution $u\in C^{2k}\(M\)$ to the equation \eqref{Eq1}, which minimizes the energy functional \eqref{Eq2}. If moreover the Green's function of the operator $P_{2k}$ is positive, then $u$ is positive, which implies that the $Q$-curvature of order $2k$ of the metric $u^{4/\(n-2k\)}g$ is equal to $\frac{2}{n-2k}f$. 
\end{theorem}

An immediate consequence of Theorem~\ref{Th} is the following:

\begin{corollary}
Let $k\ge1$ be an integer and $\(M,g\)$ be a smooth, closed, non-locally conformally flat Riemannian manifold of dimension $n\ge 2k+4$. Assume that the operator $P_{2k}$ is coercive and its Green's function is positive. Then there exists a conformal metric to $g$ with constant $Q$-curvature of order $2k$. 
\end{corollary}

When $\(M,g\)$ is Einstein, Gover~\cite{Gov} established the formula
$$P_{2k}=\prod_{j=1}^k\(\Delta+\frac{\(n+2j-2\)\(n-2j\)}{4n\(n-1\)}\Scal\),$$
where $\Scal$ is the Scalar curvature of $\(M,g\)$. In this case, it is easy to see that  if $\Scal$ is positive, then $P_{2k}$ is coercive. Furthermore, successive applications of the maximum principles yield that the Green's function of the operator $P_{2k}$ is positive. Therefore, we obtain the following:

\begin{corollary}
Let $k\ge1$ be an integer and $\(M,g\)$ be a smooth, closed Einstein manifold of positive scalar curvature and dimension $n\ge 2k+4$. Let $f$ be a smooth positive function in $M$ such that there exists a maximal point $\xi$ of $f$ satisfying \eqref{ThEq1}. Then there exists a conformal metric to $g$ with $Q$-curvature of order $2k$ equal to~$\frac{2}{n-2k}f$.
\end{corollary}

The positivity of the Green's function of the operator $P_4$ has been shown to be true by  Gursky and Malchiodi~\cites{GurMal} and Hang and Yang~\cites{HangYang1,HangYang2} under some positivity assumptions on the $Q$-curvature of order 4 and the scalar curvature or the Yamabe invariant of the manifold. As of yet, as far as the authors know, these results have not been extended to higher orders.

\smallskip
When $2k<n<2k+4$, the existence of positive solutions to the equation \eqref{Eq1} usually relies on the proof of a suitable positive mass term (see Schoen~\cite{Sch1} and Schoen and Yau~\cite{SchYau} for $k=1$ and Gursky and Malchiodi~\cites{GurMal}, Hang and Yang~\cite{HangYang1}, Humbert and Raulot~\cite{HumRau} and Michel~\cite{Mic} for $k=2$). Here again, as far as the authors know, no such results have yet been obtained for higher orders. When $n=2k$, the problem of prescribing the $Q$-curvature involves a different equation than \eqref{Eq1}, which contains an exponential non-linearity. Possible references in this case are Chang and Yang~\cite{ChangYang}, Djadli and Malchiodi~\cite{DjaMal} and Li, Li and Liu~\cite{LiLiLiu} for $k=2$ and Baird, Fardoun and Regbaoui~\cite{BaiFarReg} for higher orders.  

\smallskip
The proof of Theorem~\ref{Th} is based on the approach introduced by Aubin~\cite{Aub} in the case where $k=1$ and later extended by Esposito and Robert~\cite{EspRob} and Chen and Hou~\cite{ChenHou} to the cases where $k=2$ and $k=3$, respectively. This approach consists in deriving an asymptotic expansion for the energy functional associated with the equation \eqref{Eq1}, which we apply to a suitable family of test functions depending on a real parameter (see \eqref{Eq2} for the energy functional and \eqref{Eq6} for the family of test functions). To simplify the calculations of curvature terms, we use the conformal normal coordinates introduced by Lee and Parker~\cite{LeePar} and later improved by Cao~\cite{Cao} and G\"{u}nther~\cite{Gun}. Our proof also crucially relies on the derivation of an expression for the highest-order terms of the GJMS operators (see \eqref{PrEq1}), which we obtain by using Juhl's formulae~\cite{Juhl}. Similarly as in the lower-order cases, we find that the proof essentially reduces to determine the sign of a constant $C\(n,k\)$ depending only on $n$ and $k$, which appears in the energy expansion (see \eqref{PrEq1}). In particular, we recover the values found in~\cites{ChenHou,EspRob} for $C\(n,k\)$ with $k\in\left\{2,3\right\}$. We then conclude the proof by using a general minimization result in the spirit of Aubin~\cite{Aub} (see Mazumdar~\cite{Maz}*{Theorem~3}). We point out that at one place in the proof, namely in the very last computation to determine the sign of $C\(n,k\)$ (see \eqref{PrEq11}), we have used the computation software {\it Maple} to expand a complicated polynomial with integer coefficients.

\section{Energy expansion and proof of Theorem~\ref{Th}}\label{Sec2}

This section is devoted to the proof of Theorem~\ref{Th}. Given an integer $k\ge1$ and a smooth positive function $f$ in $M$, we let $I_{k,f}$ be the energy functional defined as
\begin{equation}\label{Eq2}
I_{k,f,g}\(u\):=\frac{\displaystyle\int_MuP_{2k}u\,dv_g}{\displaystyle\(\int_Mf\left|u\right|^{2^*_k}dv_g\)^{\frac{n-2k}{n}}}
\end{equation}
for all functions $u\in C^{2k}\(M\)$ such that $u\not\equiv0$. We fix a point $\xi\in M$. By applying a conformal change of metric (see Cao~\cite{Cao}, G\"{u}nther~\cite{Gun} and Lee and Parker~\cite{LeePar}), we may assume that 
\begin{equation}\label{Eq3}
\det g\(x\)=1\quad\forall x\in U
\end{equation}
for some neighborhood $U$ of the point $\xi$, where $\det g$ is the determinant of $g$ in geodesic normal coordinates at $\xi$. In particular (see~\cite{LeePar}), it follows from \eqref{Eq3} that
\begin{align}\label{Eq4}
\Ricci\(\xi\)=0,\quad&\Sym\nabla\Ricci\(\xi\)=0,\nonumber\\
\text{and}\quad&\Sym\(\Ricci_{ab;cd}\(\xi\)+\frac{2}{9}\sum_{e,f=1}^n\Weyl_{eabf}\(\xi\)\Weyl_{ecdf}\(\xi\)\)=0
\end{align}
in normal coordinates at the point $\xi$, where $\Sym$ stands for the symmetric part and $\Ricci_{ab;cd}$ and $\Weyl_{eabf}$ are the coordinates of $\nabla^2\Ricci$ and $\Weyl$, respectively. By taking traces in \eqref{Eq4} and using Bianchi's identities, we obtain
\begin{equation}\label{Eq5}
\Scal\(\xi\)=0,\quad\nabla \Scal\(\xi\)=0\quad\text{and}\quad\Delta \Scal\(\xi\)=-2\sum_{a,b=1}^n\Ricci_{ab;ab}\(\xi\)=\frac{1}{6}\left|\Weyl\(\xi\)\right|^2.
\end{equation}
Let $r_0>0$ be such that the injectivity radius of the metric $g$ at the point $\xi$ is greater than $3r_0$ and $B\(\xi,3r_0\)\subset U$, where $B\(\xi,r_0\)$ is the ball of center $\xi$ and radius $3r_0$ with respect to $g$. We then let $\chi$ be a smooth cutoff function in $\[0,\infty\)$ such that $\chi\equiv1$ in $\[0,r_0\]$, $0\le\chi\le1$ in $\(r_0,2r_0\)$ and $\chi\equiv0$ in $\[2r_0,\infty\)$. For every $\mu>0$, we then define our test functions as
\begin{equation}\label{Eq6}
U_\mu\(x\):=\chi\(d_g\(x,\xi\)\)\mu^{\frac{2k-n}{2}}U\big(\mu^{-1}\exp_\xi^{-1}x\big)\quad\forall x\in M,
\end{equation}
where $d_g$ is the geodesic distance with respect to $g$, $\exp_\xi$ is the exponential map with respect to $g$ at the point $\xi$ and $U$ is the function in $\R^n$ (we identify $T_\xi M$ with $\R^n$) defined as
$$U\(x\)=\big(1+\left|x\right|^2\big)^{-\frac{n-2k}{2}}\quad\forall x\in\R^n.$$
It is easy to verify that $U$ is a solution of the equation 
$$\Delta_0^k\,U=\[\prod_{j=-k}^{k-1}\(n+2j\)\]U^{2^*_k-1}\quad\text{in }\R^n,$$
where $\Delta_0$ is the Euclidean Laplacian.

\begin{proposition}\label{Pr}
Let $k\ge1$ be an integer, $\(M,g\)$ be a smooth, closed Riemannian manifold of dimension $n\ge 2k+4$ and $f$ be a smooth positive function in $M$. Assume that $g$ satisfies \eqref{Eq3} at some point $\xi\in M$. Let $I_{k,f,g}$ be as in \eqref{Eq2} and $U_\mu$ be as in \eqref{Eq6}. Then there exists a positive constant $C\(n,k\)$ depending only on $n$ and $k$ (see \eqref{PrEq10} for the value of $C\(n,k\)$) such that as $\mu\to0$,
\begin{multline}\label{PrEq1}
I_{k,f,g}\(U_\mu\)=\omega_n^{\frac{2k}{n}}f\(\xi\)^{-\frac{n-2k}{n}}\Bigg(\(2k-1\)!\Beta\(\frac{n}{2}-k,2k\)^{-1}\\
+\frac{\(n-2k\)\(2k-1\)!}{2n\(n-2\)}\Beta\(\frac{n}{2}-k,2k\)^{-1}\frac{\Delta f\(\xi\)}{f\(\xi\)}\mu^2\allowdisplaybreaks\\
-\frac{\(n-2k\)\(2k-1\)!}{4n\(n-2\)}\Beta\(\frac{n}{2}-k,2k\)^{-1}\(\frac{\Delta^2f\(\xi\)}{2\(n-4\)f\(\xi\)}-\frac{\(n-k\)\(\Delta f\(\xi\)\)^2}{n\(n-2\)f\(\xi\)^2}\)\mu^4\\
-C\(n,k\)\,\mu^4\left\{\begin{aligned}&\left|\Weyl\(\xi\)\right|^2\ln\(1/\mu\)+\bigO\(1\)&&\text{if }n=2k+4\\&\left|\Weyl\(\xi\)\right|^2+\smallo\(1\)&&\text{if }n>2k+4.\end{aligned}\right\}\Bigg),
\end{multline}
where $\omega_n$ is the volume of the standard $n$-dimensional sphere and $\Beta$ is the beta function defined as 
$$\Beta\(a,b\)=\frac{\Gamma\(a\)\Gamma\(b\)}{\Gamma\(a+b\)}\quad\forall a,b>0.$$
\end{proposition}

\proof[Proof of Proposition~\ref{Pr}]
We let $\Schouten$ be the Schouten tensor defined as
$$\Schouten:=\frac{1}{n-2}\(\Ricci-\frac{\Scal}{2\(n-1\)}\,g\)$$
and $\Bach$ be the Bach tensor whose coordinates are given by
$$\Bach_{ij}:=g^{ab}g^{cd}\Schouten_{ac}\Weyl_{ibjd}+g^{ab}\(\Schouten_{ij;ab}-\Schouten_{ia;jb}\)$$
in Einstein's summation notation, where $g^{ab}$, $\Schouten_{ac}$, $\Schouten_{ij;ab}$ and $\Weyl_{ibjd}$ are the coordinates of $g^{-1}$, $\Schouten$, $\nabla^2\Schouten$ and $\Weyl$, respectively. The first step in the proof of Proposition~\ref{Pr} is as follows:

\begin{step}\label{Step1}
We have 
\begin{multline}\label{Step1Eq1}
P_{2k}=\Delta^k+k\Delta^{k-1}\(f_1\cdot\)+k\(k-1\)\Delta^{k-2}\(f_2\cdot+\(T_1,\nabla\)+\(T_2,\nabla^2\)\)\\
+k\(k-1\)\(k-2\)\Delta^{k-3}\(\(T_3,\nabla^2\)+\(T_4,\nabla^3\)\)\\
+k\(k-1\)\(k-2\)\(k-3\)\Delta^{k-4}\(T_5,\nabla^4\)+Z,
\end{multline}
where $Z$ is a smooth linear operator of order less than $2k-4$ if $k\ge3$, $Z:=0$ if $k\le2$, the functions $f_1$ and $f_2$ are defined as 
\begin{multline*}
f_1:=\frac{n-2}{4\(n-1\)}\Scal\quad\text{and}\quad f_2:=\frac{1}{6}\bigg(\frac{3n^2-12n-4k+8}{16\(n-1\)^2}\Scal^2-\(k+1\)\(n-4\)\left|\Schouten\right|^2\\-\frac{3n+2k-4}{4\(n-1\)}\Delta\Scal\bigg)
\end{multline*}
and the tensors $T_1$, $T_2$, $T_3$, $T_4$ and $T_5$ are defined as
\begin{align*}
&T_1:=\frac{n-2}{4\(n-1\)}\nabla\Scal-\frac{2}{3}\(k+1\)\delta \Schouten,\quad T_2:=\frac{2}{3}\(k+1\)\Schouten,\allowdisplaybreaks\\
&T_3:=\frac{n-2}{6\(n-1\)}\nabla^2\Scal+\frac{\(k+1\)\(n-2\)}{6\(n-1\)}\Scal\Schouten-\frac{k+1}{3}\(\delta\nabla\Schouten+2\nabla\delta\Schouten+2\Riemann\ast\Schouten\)\\
&\qquad-\frac{2}{15}\(k+1\)\(k+2\)\(3\Schouten^{\#}\Schouten+\frac{\Bach}{n-4}\),\allowdisplaybreaks\\
&T_4:=\frac{2}{3}\(k+1\)\nabla \Schouten\quad\text{and}\quad T_5:=\frac{2}{5}\(k+1\)\(\frac{5k+7}{9}\Schouten\otimes \Schouten+\nabla^2\Schouten\),
\end{align*}
where $\#$ stands for the musical isomorphism with respect to $g$ (i.e. $\Schouten^\#:=g^{-1}\Schouten$) and $\Riemann\ast\Schouten$ stands for the covariant tensor whose coordinates are given by 
\begin{equation}\label{Step1Eq2}
\(\Riemann\ast\Schouten\)_{ij}:=g^{ab}g^{cd}\(\Riemann_{iacb}\Schouten_{jd}+\Riemann_{icaj}\Schouten_{bd}\),
\end{equation}
where $g^{ab}$, $\Riemann_{iacb}$ and $\Schouten_{jd}$ are the coordinates of $g^{-1}$ and the Riemann and Schouten curvature tensors, respectively.
\end{step}

\proof[Proof of Step~\ref{Step1}]
Throughout this proof, for every integer $l$, $\smallo^l$ stands for a linear operator of order less than $l$ if $l>0$ and $\smallo^{l}:=0$ if $l\le0$. Juhl's formulae~\cite{Juhl} (see also Fefferman and Graham~\cite{FefGra3}) give
\begin{multline}\label{Step1Eq3}
P_{2k}=M_2^k-\sum_{j=1}^{k-1}j\(k-j\)M_2^{j-1}M_4M_2^{k-j-1}\\
+\frac{1}{4}\sum_{j=1}^{k-2}j\(j+1\)\(k-j\)\(k-j-1\)M_2^{j-1}M_6M_2^{k-j-2}\\
+\sum_{j=2}^{k-2}\(j+1\)\(k-j-1\)\sum_{i=1}^{j-1}i\(k-i\)M_2^{i-1}M_4M_2^{j-i-1}M_4M_2^{k-j-2}+\smallo^{2k-5},
\end{multline}
where the operators $M_2$, $M_4$ and $M_6$ are defined as
$$M_2:=\Delta+\mu_2,\quad M_4:=4\delta\Schouten^\#d+\mu_4\quad \text{and}\quad M_6:=\delta A_6^\#d+\mu_6,$$
where $\mu_6$ is a smooth function in $M$ which we do not need explicitly, $\mu_2$ and $\mu_4$ are the functions defined as
$$\mu_2:=\frac{n-2}{4\(n-1\)}\Scal\quad\text{ and }\quad \mu_4:=\frac{\Delta\Scal}{2\(n-1\)}+\frac{\Scal^2}{4\(n-1\)^2}+\(n-4\)\left|\Schouten\right|^2$$
and $A_6$ is the tensor defined as
$$A_6:=48\Schouten^{\#}\Schouten+\frac{16}{n-4}\Bach.$$
We point out that throughout this paper, we use the same sign convention for the Riemann curvature tensor as in the paper of Lee and Parker~\cite{LeePar}, which is the opposite of the convention used by Fefferman and Graham~\cite{FefGra3} and Juhl~\cite{Juhl}. Straightforward expansions yield
\begin{align}\label{Step1Eq6}
&M_2^k=\Delta^k+\frac{n-2}{4\(n-1\)}\sum_{j=1}^k\Delta^{j-1}\(\Scal\Delta^{k-j}\)\nonumber\\
&\quad+\frac{\(n-2\)^2}{16\(n-1\)^2}\sum_{j=2}^k\sum_{i=1}^{j-1}\Delta^{i-1}\(\Scal\Delta^{j-i-1}\(\Scal\Delta^{k-j}\)\)+\smallo^{2k-5}\allowdisplaybreaks\nonumber\\
&=\Delta^k+\frac{n-2}{4\(n-1\)}\sum_{j=1}^k\Delta^{j-1}\(\Scal\Delta^{k-j}\)+\frac{\(n-2\)^2}{16\(n-1\)^2}\sum_{j=2}^k\(j-1\)\Delta^{k-2}\(\Scal^2\cdot\)+\smallo^{2k-4}\allowdisplaybreaks\nonumber\\
&=\Delta^k+\frac{n-2}{4\(n-1\)}\sum_{j=1}^k\Delta^{j-1}\(\Scal\Delta^{k-j}\)+\frac{k\(k-1\)\(n-2\)^2}{32\(n-1\)^2}\Delta^{k-2}\(\Scal^2\cdot\)+\smallo^{2k-4},
\end{align}
\begin{align}\label{Step1Eq7}
&M_2^{j-1}M_4M_2^{k-j-1}=4\Delta^{j-1}\delta\Schouten^\#d\Delta^{k-j-1}+\Delta^{j-1}\(\mu_4\Delta^{k-j-1}\)\nonumber\\
&\quad+\frac{n-2}{n-1}\sum_{i=1}^{j-1}\Delta^{i-1}\big(\Scal\Delta^{j-i-1}\delta \Schouten^\#d\Delta^{k-j-1}\big)\nonumber\\
&\quad+\frac{n-2}{n-1}\sum_{i=j+1}^{k-1}\Delta^{j-1}\delta \Schouten^\#d\Delta^{i-j-1}\(\Scal\Delta^{k-i-1}\)+\smallo^{2k-5}\allowdisplaybreaks\nonumber\\
&=4\Delta^{j-1}\delta \Schouten^\#d\Delta^{k-j-1}+\mu_4\Delta^{k-2}-\frac{\(k-2\)\(n-2\)}{n-1}\Delta^{k-3}\(\Scal\Schouten,\nabla^2\)+\smallo^{2k-4},
\end{align}
\begin{equation}\label{Step1Eq8}
M_2^{j-1}M_6M_2^{k-j-2}=\Delta^{j-1}\delta A_6^\#d\Delta^{k-j-2}+\smallo^{2k-5}=-\Delta^{k-3}\(A_6,\nabla^2\)+\smallo^{2k-4}.
\end{equation}
and
\begin{align}\label{Step1Eq9}
M_2^{i-1}M_4M_2^{j-i-1}M_4M_2^{k-j-2}&=16\Delta^{i-1}\delta \Schouten^\#d\Delta^{j-i-1}\delta \Schouten^\#d\Delta^{k-j-2}+\smallo^{2k-5}\nonumber\\
&=16\Delta^{k-4}\(\Schouten\otimes\Schouten,\nabla^4\)+\smallo^{2k-4}
\end{align}
Furthermore, by induction, one can check that
\begin{equation}\label{Step1Eq10}
\Scal\Delta^{j}=\Delta^j\(\Scal\cdot\)-j\Delta\Scal\Delta^{j-1}+2j\Delta^{j-1}\(\nabla\Scal,\nabla\)+2j\(j-1\)\Delta^{j-2}\(\nabla^2\Scal,\nabla^2\)+\smallo^{2j-2}
\end{equation}
and
\begin{multline}\label{Step1Eq11}
\delta \Schouten^\#d\Delta^j=\Delta^j\(\(\delta \Schouten,\nabla\)-\(\Schouten,\nabla^2\)\)+j\Delta^{j-1}\big(\(\delta\nabla\Schouten+2\nabla\delta\Schouten+2\Riemann\ast \Schouten,\nabla^2\)\\
-2\(\nabla \Schouten,\nabla^3\)\big)-2j\(j-1\)\Delta^{j-2}\(\nabla^2\Schouten,\nabla^4\)+\smallo^{2j},
\end{multline}
where $\Riemann\ast \Schouten$ is as in \eqref{Step1Eq2}. The proof of \eqref{Step1Eq11} relies on the commutation formula
\begin{align*}
&u_{;abcd}=\(u_{;acb}-R^e_{abc}u_{;e}\)_{;d}=u_{;acbd}-\Riemann^e_{abc}u_{;ed}+\smallo^2u=u_{;cabd}-\Riemann^e_{abc}u_{;de}+\smallo^2u\\
&\quad=u_{;cadb}-\Riemann^e_{abd}u_{;ce}-\Riemann^e_{cbd}u_{;ae}-\Riemann^e_{abc}u_{;de}+\smallo^2u\\
&\quad=\(u_{;cda}-R^e_{cab}u_{;e}\)_{;b}-\Riemann^e_{abd}u_{;ce}-\Riemann^e_{cbd}u_{;ae}-\Riemann^e_{abc}u_{;de}+\smallo^2u\\
&\quad=u_{;cdab}-\Riemann^e_{cad}u_{;eb}-\Riemann^e_{abd}u_{;ce}-\Riemann^e_{cbd}u_{;ae}-\Riemann^e_{abc}u_{;de}+\smallo^2u\\
&\quad=u_{;cdab}+\Riemann^e_{cda}u_{;be}+\Riemann^e_{adb}u_{;ce}+\Riemann^e_{cdb}u_{;ae}+\Riemann^e_{acb}u_{;de}+\smallo^2u,
\end{align*}
where $\Riemann^e_{abc}:=g^{ef}\Riemann_{fabc}$, which gives 
\begin{multline*}
\delta P^{\#}d\Delta u-\Delta\delta P^{\#}du=g^{ab}g^{cc'}g^{dd'}\big(\(P_{c'd'}u_{;abc}\)_{;d}-\(P_{c'd'}u_{;c}\)_{;dab}\big)\\
=g^{ab}g^{cc'}g^{dd'}\(P_{c'd'}\(u_{;abcd}-u_{;cdab}\)-P_{c'd';ab}u_{;cd}-2P_{c'd';da}u_{;cb}-2P_{c'd';a}u_{;cdb}\)\\
+o^2u=g^{ab}g^{cc'}g^{dd'}(2P_{c'd'}\(\Riemann^e_{cda}u_{;be}+\Riemann^e_{adb}u_{;ce}\)-P_{c'd';ab}u_{;cd}-2P_{c'd';da}u_{;cb}\\
-2P_{c'd';a}u_{;cdb})+\smallo^2u=\(\delta\nabla\Schouten+2\nabla\delta\Schouten+2\Riemann\ast \Schouten,\nabla^2u\)-2\(\nabla \Schouten,\nabla^3u\)+\smallo^2u.
\end{multline*}
By using \eqref{Step1Eq7}--\eqref{Step1Eq11} together with Faulhaber's formula, which gives
\begin{align*}
&\sum_{j=1}^kj=\frac{k\(k+1\)}{2},\quad\sum_{j=1}^kj^2=\frac{k\(k+1\)\(2k+1\)}{6},\quad\sum_{j=1}^kj^3=\frac{k^2\(k+1\)^2}{4},\allowdisplaybreaks\\
&\sum_{j=1}^kj^4=\frac{k\(k+1\)\(2k+1\)\(3k^2+3k-1\)}{30},\quad\sum_{j=1}^kj^5=\frac{k^2\(k+1\)^2\(2k^2+2k-1\)}{12},
\end{align*}
we obtain
\begin{align}
&\sum_{j=1}^k\Delta^{j-1}\(\Scal\Delta^{k-j}\)=k\Delta^{k-1}\(\Scal \cdot\)-\frac{k\(k-1\)}{2}\Delta^{k-2}\(\Delta\Scal\cdot\)\nonumber\\
&+k\(k-1\)\Delta^{k-2}\(\nabla\Scal,\nabla\)+\frac{2k\(k-1\)\(k-2\)}{3                                                                                                                                                                                                                                                          }\Delta^{k-3}\(\nabla^2\Scal,\nabla^2\)+\smallo^{2k-4},\label{Step1Eq12}\allowdisplaybreaks\\
&\sum_{j=1}^{k-1}j\(k-j\)M_2^{j-1}M_4M_2^{k-j-1}=k\(k-1\)\(k+1\)\bigg(\frac{2}{3}\Delta^{k-2}\(\(\delta \Schouten,\nabla\)-\(\Schouten,\nabla^2\)\)\nonumber\\
&+\frac{k-2}{3}\Delta^{k-3}\(\(\delta\nabla\Schouten+2\nabla\delta\Schouten+2\Riemann\ast \Schouten,\nabla^2\)-2\(\nabla \Schouten,\nabla^3\)\)+\frac{1}{6}\Delta^{k-2}\(\mu_4\cdot\)\nonumber\\
&-\frac{2\(k-2\)\(k-3\)}{5}\Delta^{k-4}\(\nabla^2\Schouten,\nabla^4\)-\frac{\(k-2\)\(n-2\)}{6\(n-1\)}\Delta^{k-3}\(\Scal\Schouten,\nabla^2\)\bigg)+\smallo^{2k-4},\label{Step1Eq13}\allowdisplaybreaks\\
&\sum_{j=1}^{k-2}j\(j+1\)\(k-j\)\(k-j-1\)M_2^{j-1}M_6M_2^{k-j-2}\nonumber\\
&\qquad=-\frac{k\(k-1\)\(k-2\)\(k+1\)\(k+2\)}{30}\Delta^{k-3}\(A_6,\nabla^2\)+\smallo^{2k-4}\label{Step1Eq14}
\end{align}
and
\begin{multline}\label{Step1Eq15}
\sum_{j=2}^{k-2}\(j+1\)\(k-j-1\)\sum_{i=1}^{j-1}i\(k-i\)M_2^{i-1}M_4M_2^{j-i-1}M_4M_2^{k-j-2}\\
=\frac{2k\(k-1\)\(k-2\)\(k-3\)\(k+1\)\(5k+7\)}{45}\Delta^{k-4}\(\Schouten\otimes \Schouten,\nabla^4\)+\smallo^{2k-4}.
\end{multline}
Finally, \eqref{Step1Eq1} follows by putting together \eqref{Step1Eq3}, \eqref{Step1Eq6} and \eqref{Step1Eq12}--\eqref{Step1Eq15}. This ends the proof of Step~\ref{Step1}.
\endproof

The next step is as follows:

\begin{step}\label{Step2}
Assume that $n\ge 2k+4$ and $k\ge3$. Then for every smooth linear operator $Z$ of order less than $2k-4$, as $\mu\to0$,
\begin{equation}\label{Step2Eq1}
\int_MU_\mu ZU_\mu\,dv_g=\left\{\begin{aligned}&\bigO\(\mu^4\)&&\text{if }n=2k+4\\&\smallo\(\mu^4\)&&\text{if }n>2k+4.\end{aligned}\right.
\end{equation}
\end{step}

\proof[Proof of Step~\ref{Step2}]
By rewriting the integral in geodesic normal coordinates, we obtain
\begin{equation}\label{Step2Eq2}
\int_MU_\mu ZU_\mu\,dv_g=\int_{B\(0,2r_0\)}\widetilde{U}_\mu\widetilde{Z}\widetilde{U}_\mu\,dx=\sum_{\left|\alpha\right|<2k-4}\int_{B\(0,2r_0\)} z_\alpha\widetilde{U}_\mu\partial^{\alpha}\widetilde{U}_\mu\,dx,
\end{equation}
where 
\begin{equation}\label{Step2Eq3}
\widetilde{U}_\mu\(x\):=\mu^{\frac{2k-n}{2}}U\(x/\mu\)\text{ and }\widetilde{Z}\(x\):=\sum_{\left|\alpha\right|<2k-4}z_\alpha\(x\)\partial^{\alpha}\quad\forall x\in B\(0,2r_0\)
\end{equation}
for some smooth functions $z_\alpha$ in $B\(0,2r_0\)$. A straightforward change of variable then gives
\begin{equation}\label{Step2Eq4}
\int_{B\(0,2r_0\)} z_\alpha\widetilde{U}_\mu\partial^{\(\alpha\)}\widetilde{U}_\mu\,dx=\mu^{2k-\left|\alpha\right|}\int_{B\(0,2r_0/\mu\)} z_\alpha\(\mu x\)U\(x\)\partial^{\(\alpha\)}U\(x\)dx.
\end{equation}
An easy induction yields that for every multi-index $\alpha$, there exists a constant $C_\alpha$ such that 
\begin{equation}\label{Step2Eq5}
\big|\partial^{\(\alpha\)}U\(x\)\big|\le C_\alpha\big(1+\left|x\right|^2\big)^{-\frac{n-2k+\left|\alpha\right|}{2}}\quad\forall x\in\R^n
\end{equation}
It follows from \eqref{Step2Eq4} and \eqref{Step2Eq5} that
\begin{align}\label{Step2Eq6}
\int_{B\(0,2r_0\)} z_\alpha\widetilde{U}_\mu\partial^{\(\alpha\)}\widetilde{U}_\mu\,dx&=\bigO\(\mu^{2k-\left|\alpha\right|}\int_{B\(0,2r_0/\mu\)}\big(1+\left|x\right|^2\big)^{-n+2k-\left|\alpha\right|/2}dx\)\nonumber\\
&=\left\{\begin{aligned}&\bigO\big(\mu^{2k-\left|\alpha\right|}\big)&&\text{if }\left|\alpha\right|>4k-n\\&\bigO\(\mu^{n-2k}\ln\(1/\mu\)\)&&\text{if }\left|\alpha\right|=4k-n\\&\bigO\(\mu^{n-2k}\)&&\text{if }\left|\alpha\right|<4k-n.\end{aligned}\right.
\end{align}
Finally, \eqref{Step2Eq1} follows from \eqref{Step2Eq2} and \eqref{Step2Eq6}. 
\endproof

We then prove the following:

\begin{step}\label{Step3}
Assume that $n\ge 2k+4$. Then, as $\mu\to0$,
\begin{equation}\label{Step3Eq1}
\int_MU_\mu\Delta^kU_\mu\,dv_g=2^{2k-n}\(2k-1\)!\,\omega_n\Beta\(\frac{n}{2}-k,2k\)^{-1}+\bigO\(\mu^{n-2k}\).
\end{equation}
If $k\ge2$, then for every smooth function $f$, 
\begin{multline}\label{Step3Eq2}
\int_MfU_\mu\Delta^{k-2}U_\mu \,dv_g=\frac{2^{2k-n-1}\(n-1\)!\(k-2\)!\,\omega_n}{\(n-2\)\(n-4\)\(n-2k-2\)}\,f\(\xi\)\mu^4\\
\times\sum_{l=k-2}^{2k-4}\frac{l!}{\(l-k+2\)!\(2k-l-4\)!\(n+l-2k-1\)!}\Beta\(\frac{n}{2}-k-1,l+1\)^{-1}\allowdisplaybreaks\\
\times\left\{\begin{aligned}&2\ln\(1/\mu\)&&\text{if }n=2k+4\text{ and }l=k-2\\&\Beta\(\frac{n}{2}+l-2k,2k-l-2\)&&\text{otherwise}\end{aligned}\right\}\\
+\left\{\begin{aligned}&\bigO\(\mu^4\)&&\text{if }n=2k+4\\&\smallo\(\mu^4\)&&\text{if }n>2k+4,\end{aligned}\right.
\end{multline}
for every smooth, covariant tensor $T$ of rank $1$,   
\begin{multline}\label{Step3Eq3}
\int_M\(T,\nabla U_\mu\)\Delta^{k-2}U_\mu \,dv_g=-\frac{2^{2k-n-2}\(n-2k\)\(n-1\)!\(k-2\)!\,\omega_n}{\(n-2\)\(n-4\)\(n-2k-2\)}\sum_{a=1}^nT_{a;a}\(\xi\)\\
\times\mu^4\sum_{l=k-2}^{2k-4}\frac{l!}{\(l-k+2\)!\(2k-l-4\)!\(n+l-2k\)!}\Beta\(\frac{n}{2}-k-1,l+1\)^{-1}\allowdisplaybreaks\\
\times\left\{\begin{aligned}&2\ln\(1/\mu\)&&\text{if }n=2k+4\text{ and }l=k-2\\&\Beta\(\frac{n}{2}+l-2k,2k-l-2\)&&\text{otherwise}\end{aligned}\right\}\\
+\left\{\begin{aligned}&\bigO\(\mu^4\)&&\text{if }n=2k+4\\&\smallo\(\mu^4\)&&\text{if }n>2k+4\end{aligned}\right.
\end{multline}
and for every smooth, covariant tensor $T$ of rank $2$,   
\begin{multline}\label{Step3Eq4}
\int_M\(T,\nabla^2U_\mu\)\Delta^{k-2}U_\mu\,dv_g=\frac{2^{2k-n-4}\(n-2k\)\(n-1\)!\(k-2\)!\,\omega_n}{\(n-2\)\(n-4\)\(n-2k-2\)}\\
\times\sum_{l=k-2}^{2k-4}\frac{l!}{\(l-k+2\)!\(2k-l-4\)!\(n+l-2k+1\)!}\Beta\(\frac{n}{2}-k-1,l+1\)^{-1}\allowdisplaybreaks\\
\times\Bigg(-2\(n-4\)\(n+2l-2k\)\Beta\(\frac{n}{2}-2k+l+1,2k-l-2\)\sum_{a=1}^nT_{aa}\(\xi\)\mu^2\allowdisplaybreaks\\
+\(\(n-2k+2\)\sum_{a,b=1}^n\(T_{ab;ab}\(\xi\)+T_{ab;ba}\(\xi\)\)-\(n+2l-2k\)\sum_{a,b=1}^nT_{aa;bb}\(\xi\)\)\mu^4\allowdisplaybreaks\\
\times\left\{\begin{aligned}&2\ln\(1/\mu\)&&\text{if }n=2k+4\text{ and }l=k-2\\&\Beta\(\frac{n}{2}+l-2k,2k-l-2\)&&\text{otherwise}\end{aligned}\right\}\Bigg)\\
+\left\{\begin{aligned}&\bigO\(\mu^4\)&&\text{if }n=2k+4\\&\smallo\(\mu^4\)&&\text{if }n>2k+4.\end{aligned}\right.
\end{multline}
If $k\ge3$, then for every smooth, covariant tensor $T$ of rank $2$,
\begin{multline}\label{Step3Eq5}
\int_M\(T,\nabla^2U_\mu\)\Delta^{k-3}U_\mu\,dv_g=-\frac{2^{2k-n-5}\(n-2k\)\(n-1\)!\(k-3\)!\,\omega_n}{\(n-2\)\(n-4\)\(n-2k-2\)}\sum_{a=1}^nT_{aa}\(\xi\)\\
\times\mu^4\sum_{l=k-3}^{2k-6}\frac{\(n+2l-2k\)l!}{\(l-k+3\)!\(2k-l-6\)!\(n+l-2k+1\)!}\Beta\(\frac{n}{2}-k-1,l+1\)^{-1}\allowdisplaybreaks\\
\times\left\{\begin{aligned}&2\ln\(1/\mu\)&&\text{if }n=2k+4\text{ and }l=k-3\\&\Beta\(\frac{n}{2}+l-2k+1,2k-l-3\)&&\text{otherwise}\end{aligned}\right\}\\
+\left\{\begin{aligned}&\bigO\(\mu^4\)&&\text{if }n=2k+4\\&\smallo\(\mu^4\)&&\text{if }n>2k+4\end{aligned}\right.
\end{multline}
and for every smooth, covariant tensor $T$ of rank $3$,
\begin{multline}\label{Step3Eq6}
\int_M\(T,\nabla^3U_\mu\)\Delta^{k-3}U_\mu\,dv_g=\frac{2^{2k-n-6}\(n-2k\)\(n-2k+2\)\(n-1\)!\(k-3\)!\,\omega_n}{\(n-2\)\(n-4\)\(n-2k-2\)}\\
\times\sum_{a,b=1}^n\(T_{aab;b}\(\xi\)+T_{aba;b}\(\xi\)+T_{abb;a}\(\xi\)\)\mu^4\\
\times\sum_{l=k-3}^{2k-6}\frac{\(n+2l-2k\)l!}{\(l-k+3\)!\(2k-l-6\)!\(n+l-2k+2\)!}\Beta\(\frac{n}{2}-k-1,l+1\)^{-1}\\
\times\left\{\begin{aligned}&2\ln\(1/\mu\)&&\text{if }n=2k+4\text{ and }l=k-3\\&\Beta\(\frac{n}{2}+l-2k+1,2k-l-3\)&&\text{otherwise}\end{aligned}\right\}\\
+\left\{\begin{aligned}&\bigO\(\mu^4\)&&\text{if }n=2k+4\\&\smallo\(\mu^4\)&&\text{if }n>2k+4.\end{aligned}\right.
\end{multline}
If $k\ge4$, then for every smooth, covariant tensor $T$ of rank $4$,
\begin{multline}\label{Step3Eq7}
\int_M\(T,\nabla^4U_\mu\)\Delta^{k-4}U_\mu\,dv_g=\frac{2^{2k-n-8}\(n-2k\)\(n-2k+2\)\(n-1\)!\(k-4\)!\,\omega_n}{3\(n-2\)\(n-4\)\(n-2k-2\)}\\
\times\sum_{a,b=1}^n\(T_{aabb}\(\xi\)+T_{abab}\(\xi\)+T_{abba}\(\xi\)\)\mu^4\\
\times\sum_{l=k-4}^{2k-8}\frac{\(n+2l-2k\)\(n+2l-2k+2\)l!}{\(l-k+4\)!\(2k-l-8\)!\(n+l-2k+3\)!}\Beta\(\frac{n}{2}-k-1,l+1\)^{-1}\allowdisplaybreaks\\
\times\left\{\begin{aligned}&2\ln\(1/\mu\)&&\text{if }n=2k+4\text{ and }l=k-4\\&\Beta\(\frac{n}{2}+l-2k+2,2k-l-4\)&&\text{otherwise}\end{aligned}\right\}\\
+\left\{\begin{aligned}&\bigO\(\mu^4\)&&\text{if }n=2k+4\\&\smallo\(\mu^4\)&&\text{if }n>2k+4.\end{aligned}\right.
\end{multline}
\end{step}

\proof[Proof of Step~\ref{Step3}]
We let $j$ and $l$ be two integers such that 
$$\max\(2\(k-l-2\),0\)\le j\le k-l\quad\text{and}\quad\max\(k-4,0\)\le l\le k$$ 
and $T$ be a smooth, covariant tensor of rank $j$. By using geodesic normal coordinates, we obtain
\begin{align}\label{Step3Eq8}
&\int_M\(T,\nabla^jU_\mu\)\Delta^lU_\mu\,dv_g-\int_{B\(\xi,r_0\)}\(T,\nabla^jU_\mu\)\Delta^lU_\mu\,dv_g\nonumber\\
&\qquad=\int_{B\(0,2r_0\)\backslash B\(0,r_0\)}\widetilde{Z}_1\widetilde{U}_\mu\widetilde{Z}_2\widetilde{U}_\mu\,dx\nonumber\\
&\qquad=\sum_{\left|\alpha_1\right|\le j}\sum_{\left|\alpha_2\right|\le 2l}\int_{B\(0,2r_0\)\backslash B\(0,r_0\)} z_{1,\alpha}z_{2,\alpha}\partial^{\alpha_2}\widetilde{U}_\mu\partial^{\alpha_1}\widetilde{U}_\mu\,dx,
\end{align}
where $\widetilde{U}_\mu$ is as in \eqref{Step2Eq3} and
$$\widetilde{Z}_1\(x\):=\sum_{\left|\alpha\right|\le j}z_{1,\alpha}\(x\)\partial^{\alpha}\quad\text{and}\quad\widetilde{Z}_2\(x\):=\sum_{\left|\alpha\right|\le2l}z_{2,\alpha}\(x\)\partial^{\alpha}\quad\forall x\in B\(0,2r_0\)$$
for some smooth functions $z_{1,\alpha}$ and $z_{2,\alpha}$ in $B\(0,2r_0\)$. By proceeding as in \eqref{Step2Eq4}--\eqref{Step2Eq6}, we obtain
\begin{equation}\label{Step3Eq9}
\int_{B\(0,2r_0\)\backslash B\(0,r_0\)} z_{1,\alpha}z_{2,\alpha}\partial^{\alpha_2}\widetilde{U}_\mu\partial^{\alpha_1}\widetilde{U}_\mu\,dx=\bigO\(\mu^{n-2k}\).
\end{equation}
It follows from \eqref{Step3Eq8} and \eqref{Step3Eq9} that
\begin{equation}\label{Step3Eq10}
\int_M\(T,\nabla^jU_\mu\)\Delta^lU_\mu\,dv_g=\int_{B\(\xi,r_0\)}\(T,\nabla^jU_\mu\)\Delta^lU_\mu\,dv_g+\bigO\(\mu^{n-2k}\).
\end{equation}
By using \eqref{Eq3} and rewriting the integral in the right-hand side of \eqref{Step3Eq10} in geodesic normal coordinates, we obtain
\begin{equation}\label{Step3Eq11}
\int_{B\(\xi,r_0\)}\(T,\nabla^jU_\mu\)\Delta^lU_\mu\,dv_g=\sum_{j'=0}^j\int_{B\(0,r_0\)}\widehat{T}^{i_1\dotsc i_{j'}}\circ\exp_\xi U_{\mu,i_1\dotsc i_{j'}}\Delta_0^l\,U_\mu\,dx,
\end{equation}
where $U_{\mu,i_1\dotsc i_{j'}}:=\partial^{(i_1\dotsc i_{j'})}\(U_\mu\circ\,\exp_\xi\)$ and the tensor $\widehat{T}$ is defined as
$$\widehat{T}^{i_1\dotsc i_{j'}}:=\left\{\begin{aligned}&\delta^{i_1}_{e_1}\dotsm\delta^{i_j}_{e_j}&&\text{if }j'=j\\&-\Gamma^{i_1\dotsc i_{j'}}_{e_1\dotsc e_j}&&\text{if }j'<j\end{aligned}\right\}g^{i'_1e_1}\dotsm g^{i'_je_j}T_{i'_1\dotsc i'_j},$$
where $\delta^{i_1}_{e_1},\dotsc,\delta^{i_j}_{e_j}$ stand for the Kronecker symbols and $\Gamma^{i_1\dotsc i_{j'}}_{e_1\dotsc e_j}$ is the generalized Christoffel symbol such that $\Gamma^{i_1\dotsc i_{j'}}_{e_1\dotsc e_j}$ is symmetric in $i_1,\dotsc,i_{j'}$ and
$$u_{;e_1\dotsc e_j}=u_{,e_1\dotsc e_j}-\sum_{j'=0}^{j-1}\Gamma^{i_1\dotsc i_{j'}}_{e_1\dotsc e_j}u_{,i_1\dotsc i_{j'}}.$$
By using \eqref{Step3Eq11} together with a  straightforward change of variable and a Taylor expansion, we then obtain
\begin{align}\label{Step3Eq12}
&\int_{B\(\xi,r_0\)}\(T,\nabla^jU_\mu\)\Delta^lU_\mu\,dv_g\nonumber\\
&=\sum_{j'=0}^j\mu^{2k-2l-j'}\int_{B\(0,r_0/\mu\)}\widehat{T}^{i_1\dotsc i_{j'}}\(\exp_\xi\(\mu x\)\)U_{,i_1\dotsc i_{j'}}\(x\)\Delta_0^l\,U\(x\)dx\allowdisplaybreaks\nonumber\\
&=\sum_{j'=\max\(2\(k-l-2\),0\)}^j\sum_{j''=0}^{j'+2l-2k+4}\frac{\mu^{2k-2l-j'+j''}}{j''!}\sum_{i_1,\dotsc,i_{j'+j''}=1}^n\widehat{T}^{i_1\dotsc i_{j'}}{}_{,i_{j'+1}\dotsc i_{j'+j''}}\(\xi\)\nonumber\\
&\quad\times\int_{B\(0,r_0/\mu\)}U_{,i_1\dotsc i_{j'}}x_{i_{j'+1}}\dotsm x_{i_{j'+j''}}\Delta_0^l\,U\,dx+\bigO\Bigg(\sum_{j'=0}^j\mu^{\max(5,2k-2l-j')}\nonumber\\
&\quad\times\int_{B\(0,r_0/\mu\)}\hspace{-3pt}\left|x\right|^{\max(j'+2l-2k+5,0)}\big|U_{,i_1\dotsc i_{j'}}\Delta_0^l\,U\big|\,dx\Bigg).
\end{align}
On the other hand, by using \eqref{Step2Eq5}, we obtain
\begin{align}\label{Step3Eq13}
&\mu^{\max(5,2k-2l-j')}\int_{B\(0,r_0/\mu\)}\left|x\right|^{\max(j'+2l-2k+5,0)}\big|U_{,i_1\dotsc i_{j'}}\Delta_0^l\,U\big|\,dx\nonumber\\
&\quad=\bigO\(\mu^{\max(5,2k-2l-j')}\int_{B\(0,r_0/\mu\)}\left|x\right|^{\max(j'+2l-2k+5,0)}\big(1+\left|x\right|^2\big)^{-\frac{2n+j'+2l-4k}{2}}dx\)\nonumber\\
&\quad=\left\{\begin{aligned}&\bigO\(\mu^4\)&&\text{if }n=2k+4\\&\smallo\(\mu^4\)&&\text{otherwise.}\end{aligned}\right.
\end{align}
It follows from \eqref{Step3Eq10}, \eqref{Step3Eq12} and \eqref{Step3Eq13} that
\begin{multline}\label{Step3Eq14}
\int_M\(T,\nabla^jU_\mu\)\Delta^lU_\mu\,dv_g=\sum_{j'=\max\(2\(k-l-2\),0\)}^{j}\sum_{j''=0}^{j'+2l-2k+4}\frac{\mu^{2k-2l-j'+j''}}{j''!}\\
\times\sum_{i_1,\dotsc,i_{j'+j''}=1}^n\widehat{T}^{i_1\dotsc i_{j'}}{}_{,i_{j'+1}\dotsc i_{j'+j''}}\(\xi\)\int_{B\(0,r_0/\mu\)}U_{,i_1\dotsc i_{j'}}x_{i_{j'+1}}\dotsm x_{i_{j'+j''}}\Delta_0^l\,U\,dx\\
+\left\{\begin{aligned}&\bigO\(\mu^4\)&&\text{if }n=2k+4\\&\smallo\(\mu^4\)&&\text{if }n>2k+4.\end{aligned}\right.
\end{multline}
An easy induction gives
\begin{multline}\label{Step3Eq15}
U_{,i_1\dotsc i_j}\(x\)=\sum_{m=0}^{\lfloor j/2\rfloor}\frac{2^{j-2m}}{m!\(j-2m\)!}\,\partial_r^{j-m}U\(r\)\\
\times\sum_{\sigma\in\mathfrak{S}(j)}\delta_{i_{\sigma\(1\)}i_{\sigma\(2\)}}\dotsm\delta_{i_{\sigma\(2m-1\)}i_{\sigma\(2m\)}}x_{i_{\sigma\(2m+1\)}}\dotsm x_{i_{\sigma\(j\)}}\quad\forall x\in\R^n,
\end{multline}
where $r:=\left|x\right|^2$, $U\(r\):=U\(x\)=\(1+r\)^{-\(n-2k\)/2}$ and $\mathfrak{S}\(j\)$ is the set of all permutations of $\(1,\dotsc,j\)$. Furthermore, it is easy to see that
\begin{align}\label{Step3Eq16}
\partial_r^{j}U\(r\)&=\(-1\)^j2^{-j}\(n-2k\)\(n-2k+2\)\dotsm\(n-2k+2j-2\)\(1+r\)^{-\frac{n-2k+2j}{2}}\nonumber\allowdisplaybreaks\\
&=\frac{2\(-1\)^jj!}{\(n-2k-2\)}\Beta\(\frac{n}{2}-k-1,j+1\)^{-1}\(1+r\)^{-\frac{n-2k+2j}{2}}.
\end{align}
Another induction yields
\begin{equation}\label{Step3Eq17}
\Delta_0^l\,U\(x\)=\left\{\begin{aligned}&\frac{2^{2l+1}\,l!}{\(n-2k-2\)\(k-l-1\)!}\sum_{l'=l}^{2l}\frac{l'!\(k+l-l'-1\)!}{\(l'-l\)!\(2l-l'\)!}\\
&\qquad\times\Beta\(\frac{n}{2}-k-1,l'+1\)^{-1}\(1+r\)^{-\frac{n+2l'-2k}{2}}&&\text{if }l<k\\&2^{2k}\(2k-1\)!\Beta\(\frac{n}{2}-k,2k\)^{-1}\(1+r\)^{-\frac{n+2k}{2}}&&\text{if }l=k\end{aligned}\right.
\end{equation}
for all $x\in\R^n$. In the case where $j=0$, $l=k$ and $T\equiv1$, it follows from \eqref{Step3Eq17} that
\begin{equation}\label{Step3Eq18}
\int_{B\(0,r_0/\mu\)}U\Delta^k_0\,U\,dx=2^{2k-1}\(2k-1\)!\,\omega_{n-1}\Beta\(\frac{n}{2}-k,2k\)^{-1}\int_0^{\(r_0/\mu\)^2}\frac{r^{\frac{n-2}{2}}}{\(1+r\)^n}\,dr,
\end{equation}
where $\omega_{n-1}=\Vol\(\S^{n-1},g_0\)$ is the volume of the standard $\(n-1\)$-dimensional sphere. On the other hand, in the case where $l<k$, by putting together \eqref{Step3Eq15}--\eqref{Step3Eq17} , we obtain 
\begin{multline}\label{Step3Eq19}
\int_{B\(0,r_0/\mu\)}U_{,i_1\dotsc i_{j'}}x_{i_{j'+1}}\dotsm x_{i_{j'+j''}}\Delta^l_0\,U\,dx=\frac{2^{2l+1}\,l!}{\(n-2k-2\)^2\(k-l-1\)!}\sum_{l'=l}^{2l}\sum_{m=0}^{\lfloor j'/2\rfloor}\\
\times\frac{\(-1\)^{j'-m}2^{j'-2m}\,l'!\(k+l-l'-1\)!\(j'-m\)!}{\(l'-l\)!\(2l-l'\)!\,m!\(j'-2m\)!}\Beta\(\frac{n}{2}-k-1,l'+1\)^{-1}\allowdisplaybreaks\\
\times\Beta\(\frac{n}{2}-k-1,j'-m+1\)^{-1}\int_0^{\(r_0/\mu\)^2}\frac{r^{\frac{n+j'+j''-2m-2}{2}}}{\(1+r\)^{n+j'-m+l'-2k}}\,dr\sum_{\sigma\in\mathfrak{S}(j')}\\
\delta_{i_{\sigma(1)}i_{\sigma(2)}}\dotsm\delta_{i_{\sigma(2m-1)}i_{\sigma(2m)}}\int_{\S^{n-1}}y_{i_{\sigma(2m+1)}}\dotsm y_{i_{\sigma(j')}}y_{i_{j'+1}}\dotsm y_{i_{j'+j''}}dv_{g_0}\(y\).
\end{multline}
A standard computation gives
\begin{equation}\label{Step3Eq20}
\int_0^{\(r_0/\mu\)^2}\frac{r^{a-1}dr}{\(1+r\)^b}=\left\{\begin{aligned}&2\ln\(1/\mu\)+\bigO\(1\)&&\text{if }b=a\\&\Beta\(a,b-a\)+\bigO\big(\mu^{2\(b-a\)}\big)&&\text{if }b>a.\end{aligned}\right.
\end{equation}
On the other hand, by using the fact (see for instance Brendle~\cite{Bre}*{Proposition~28}) that for every homogeneous polynomial $\varPhi$ of degree $j\ge2$, 
$$\int_{\S^{n-1}}\varPhi\(y\)\,dv_{g_0}\(y\)=\frac{-1}{j\(n+j-2\)}\int_{\S^{n-1}}\Delta_{0}\varPhi\(y\)\,dv_{g_0}\(y\),$$ 
another induction yields that when $j$ is even,
\begin{multline}\label{Step3Eq21}
\int_{\S^{n-1}}\hspace{-2pt}y_{i_1}\dotsm y_{i_j}\,dv_{g_0}\(y\)=\frac{\(n-2\)\omega_{n-1}}{2^{j+1}\(j/2\)!^2}\Beta\(\frac{n-2}{2},\frac{j+2}{2}\)\\
\times\sum_{\sigma\in\mathfrak{S}(j)}\delta_{i_{\sigma\(1\)}i_{\sigma\(2\)}}\dotsm\delta_{i_{\sigma\(j-1\)}i_{\sigma\(j\)}}.
\end{multline}
The integral in \eqref{Step3Eq21} vanishes when $j$ is odd. By remarking that 
\begin{equation}\label{Step3Eq22}
\omega_n=2^{n-1}\Beta\(\frac{n}{2},\frac{n}{2}\)\omega_{n-1},
\end{equation}
we obtain that for even $j$,
\begin{equation}\label{Step3Eq23}
\Beta\(\frac{n-2}{2},\frac{j+2}{2}\)=\frac{2^{2-n}\(n-1\)!\(j/2\)!\,\omega_n}{\(n-2\)\(n+j/2-1\)!\,\omega_{n-1}}\Beta\(\frac{n}{2},\frac{n+j}{2}\)^{-1}.
\end{equation}
By using \eqref{Step3Eq20}--\eqref{Step3Eq23} together with the identity
\begin{multline*}
\Beta\(\frac{n}{2},\frac{n+j'+j''-2m}{2}\)^{-1}\Beta\(\frac{n+j'+j''-2m}{2},\frac{n+j'-j''+2l'-4k}{2}\)\\
=\frac{\(\frac{j'+j''}{2}+n-m-1\)!}{\(n+j'-m+l'-2k-1\)!\(\frac{j''-j'}{2}+2k-l'-1\)!}\\
\times\Beta\(\frac{n+j'-j''+2l'-4k}{2},\frac{j''-j'+4k-2l'}{2}\),
\end{multline*}
we obtain that if $j'+j''$ is even, then
\begin{multline}\label{Step3Eq24}
\int_0^{\(r_0/\mu\)^2}\frac{r^{\frac{n+j'+j''-2m-2}{2}}}{\(1+r\)^{n+j'-m+l'-2k}}\,dr\int_{\S^{n-1}}y_{i_{\sigma(2m+1)}}\dotsm y_{i_{\sigma(j')}}y_{i_{j'+1}}\dotsm y_{i_{j'+j''}}dv_{g_0}\(y\)\\
=\frac{2^{1-n-j'-j''+2m}\(n-1\)!\,\omega_n}{\(n+j'-m+l'-2k-1\)!\(\frac{j''-j'}{2}+2k-l'-1\)!\(\frac{j'+j''}{2}-m\)!}\\\times\left\{\begin{aligned}&2\ln\(1/\mu\)+\bigO\(1\)\hspace{117pt}\text{if }n+j'-j''+2l'-4k=0\\&\Beta\(\frac{n+j'-j''+2l'-4k}{2},\frac{j''-j'+4k-2l'}{2}\)\\
&\hspace{80pt}+\bigO\big(\mu^{n+j'-j''+2l'-4k}\big)\quad\text{if }0<n+j'-j''+2l'-4k<n\end{aligned}\right\}\\
\times\sum_{\sigma'\in\mathfrak{S}(S_{j',j'',m,\sigma})}\delta_{i_{\sigma'(\sigma(2m+1))}i_{\sigma'(\sigma(2m+2))}}\dotsm\delta_{i_{\sigma'(\sigma(j'-1))}i_{\sigma'(\sigma(j'))}}\\
\times\delta_{i_{\sigma'(j'+1)}i_{\sigma'(j'+2)}}\dotsm\delta_{i_{\sigma'(j'+j''-1)}i_{\sigma'(j'+j'')}},
\end{multline} 
where 
$$S_{j',j'',m,\sigma}:=\big(\sigma\(2m+1\),\dotsc,\sigma\(j'\),j'+1,\dotsc,j'+j''\big)$$
and $\mathfrak{S}\(S_{j',j'',m,\sigma}\)$ stands for the set of all permutations of $S_{j',j'',m,\sigma}$. In the case where $j=0$, $l=k$ and $T\equiv1$, \eqref{Step3Eq1} follows from \eqref{Step3Eq10}, \eqref{Step3Eq11}, \eqref{Step3Eq18}, \eqref{Step3Eq20} and \eqref{Step3Eq22}. On the other hand, in the case where $l<k$, by combining \eqref{Step3Eq14}, \eqref{Step3Eq19} and \eqref{Step3Eq24} (and replacing $j''$ by $j'-2m'+2l-2k+4$ for $m'\in\left\{0,\dotsc,\lfloor j'/2\rfloor+l-k+2\right\}$ so that $j'+j''$ is even and $0\le j''\le j'+2l-2k+4$), we obtain
\begin{multline}\label{Step3Eq25}
\int_M\(T,\nabla^jU_\mu\)\Delta^lU_\mu\,dv_g=\frac{2^{2k-n-2}\(n-1\)!\,l!\,\omega_n}{\(n-2k-2\)^2\(k-l-1\)!}\sum_{l'=l}^{2l}\sum_{j'=\max\(2\(k-l-2\),0\)}^{j}\sum_{m=0}^{\lfloor j'/2\rfloor}\\
\sum_{m'=0}^{\lfloor j'/2\rfloor+l-k+2}\frac{2^{2m'-j'}l'!\(k+l-l'-1\)!\,c\(n,k,j',l,l',m,m'\)\mu^{4-2m'}}{\(l'-l\)!\(2l-l'\)!\(k+l-l'-m'+1\)!\(j'-2m'+2l-2k+4\)!}\allowdisplaybreaks\\
\times\Beta\(\frac{n}{2}-k-1,l'+1\)^{-1}\sum_{i_1,\dotsc,i_{2(j'-m'+l-k+2)}=1}^n\widehat{T}^{i_1\dotsc i_{j'}}{}_{,i_{j'+1}\dotsc i_{2(j'-m'+l-k+2)}}\(\xi\)\allowdisplaybreaks\\
\times\sum_{\sigma\in\mathfrak{S}(j')}\sum_{\sigma'\in\mathfrak{S}(S_{j',j'-2m'+2l-2k+4,m,\sigma})}\delta_{i_{\sigma(1)}i_{\sigma(2)}}\dotsm\delta_{i_{\sigma(2m-1)}i_{\sigma(2m)}}\allowdisplaybreaks\\
\times\delta_{i_{\sigma'(\sigma(2m+1))}i_{\sigma'(\sigma(2m+2))}}\dotsm\delta_{i_{\sigma'(\sigma(j'-1))}i_{\sigma'(\sigma(j'))}}\allowdisplaybreaks\\
\times\delta_{i_{\sigma'(j'+1)}i_{\sigma'(j'+2)}}\dotsm\delta_{i_{\sigma'(2(j'-m'+l-k+2)-1)}i_{\sigma'(2(j'-m'+l-k+2))}}\allowdisplaybreaks\\
\times\left\{\begin{aligned}&2\ln\(1/\mu\)+\bigO\(1\)\hspace{119pt}\text{if }n=2k+4,\ l'=l\text{ and }m'=0\\&\Beta\(\frac{n}{2}+m'+l'-l-k-2,k+l-l'-m'+2\)+\smallo\big(\mu^{2m'}\big)\quad\text{otherwise}\end{aligned}\right\}\\
+\left\{\begin{aligned}&\bigO\(\mu^4\)&&\text{if }n=2k+4\\&\smallo\(\mu^4\)&&\text{if }n>2k+4,\end{aligned}\right.
\end{multline}
where
\begin{multline*}
c\(n,k,j',l,l',m,m'\):=\(-1\)^{j'-m}\Beta\(\frac{n}{2}-k-1,j'-m+1\)^{-1}\\
\times\frac{\(j'-m\)!}{m!\(j'-2m\)!\(n+j'-m+l'-2k-1\)!\(j'-m-m'+l-k+2\)!}\,.
\end{multline*}
Straightforward computations yield
\begin{multline}\label{Step3Eq26}
\sum_{\sigma\in\mathfrak{S}(j')}\sum_{\sigma'\in\mathfrak{S}(S_{j',j'',m,\sigma})}\delta_{i_{\sigma(1)}i_{\sigma(2)}}\dotsm\delta_{i_{\sigma(2m-1)}i_{\sigma(2m)}}\delta_{i_{\sigma'(\sigma(2m+1))}i_{\sigma'(\sigma(2m+2))}}\dotsm\allowdisplaybreaks\\
\dotsm\delta_{i_{\sigma'(\sigma(j'-1))}i_{\sigma'(\sigma(j'))}}\delta_{i_{\sigma'(j'+1)}i_{\sigma'(j'+2)}}\dotsm\delta_{i_{\sigma'(j'+j''-1)}i_{\sigma'(j'+j'')}}\allowdisplaybreaks\\
=\left\{\begin{aligned}
&1&&\text{if }j'=j''=m=0\\
&2\,\delta_{i_1i_2}&&\text{if }j'=j''=1\text{ and }m=0\\
&2\(2-m\)\delta_{i_1i_2}&&\text{if }j'=2,\ j''=0\text{ and }m\le1\\
&16\(\delta_{i_1i_2}\delta_{i_3i_4}+\delta_{i_1i_3}\delta_{i_2i_4}+\delta_{i_1i_4}\delta_{i_2i_3}\)&&\text{if }j'=j''=2\text{ and }m=0\\
&4\,\delta_{i_1i_2}\delta_{i_3i_4}&&\text{if }j'=j''=2\text{ and }m=1\\
&2\(4-2m\)!\(\delta_{i_1i_2}\delta_{i_3i_4}+\delta_{i_1i_3}\delta_{i_2i_4}+\delta_{i_1i_4}\delta_{i_2i_3}\)&&\text{if }j'=3,\ j''=1\text{ and }m\le1\\
&8\(4-2m\)!\(\delta_{i_1i_2}\delta_{i_3i_4}+\delta_{i_1i_3}\delta_{i_2i_4}+\delta_{i_1i_4}\delta_{i_2i_3}\)&&\text{if }j'=4,\ j''=0\text{ and }m\le2.\\
\end{aligned}\right.
\end{multline}
On the other hand, by using \eqref{Eq4} and the fact that for all $a,b,c,d,e\in\left\{1,\dotsc,n\right\}$,
\begin{align*}
&g^{ab}\(\xi\)=\delta^{ab},\quad g^{ab}{}_{,c}\(\xi\)=0,\quad g^{ab}{}_{,cd}\(\xi\)=-\frac{1}{3}\(\Riemann_{acdb}\(\xi\)+\Riemann_{adcb}\(\xi\)\),\\
&\Gamma^a_{bc}\(\xi\)=0,\quad\Gamma^a_{bc,d}\(\xi\)=\frac{1}{3}\(\Riemann_{abdc}\(\xi\)+\Riemann_{acdb}\(\xi\)\)\quad\text{and}\quad\Gamma^{ab}_{cde}\(\xi\)=0,
\end{align*}
we obtain
\begin{align}
&\widehat{T}\(\xi\)=T\(\xi\)\quad\text{if }j=0,\quad\sum_{a=1}^n\widehat{T}^{a}{}_{,a}\(\xi\)=\sum_{a=1}^nT_{a;a}\(\xi\)\quad\text{if }j=1,\label{Step3Eq27}\allowdisplaybreaks\\
&\left\{\begin{aligned}
&\sum_{a=1}^n\widehat{T}^{a}{}_{,a}\(\xi\)=0,\quad\sum_{a=1}^n\widehat{T}^{aa}\(\xi\)=\sum_{a=1}^nT_{aa}\(\xi\),\\
&\sum_{a,b=1}^n\widehat{T}^{aa}{}_{,bb}\(\xi\)=\sum_{a,b=1}^nT_{aa;bb}\(\xi\)\quad\text{and}\\
&\sum_{a,b=1}^n\widehat{T}^{ab}{}_{,ab}\(\xi\)=\sum_{a,b=1}^n\widehat{T}^{ab}{}_{,ba}\(\xi\)=\sum_{a,b=1}^nT_{ab;ab}\(\xi\)
\end{aligned}\right\}\quad\text{if }j=2,\label{Step3Eq28}\allowdisplaybreaks\\
&\left\{\begin{aligned}
&\sum_{a=1}^n\widehat{T}^{aa}\(\xi\)=0\quad\text{and}\quad\sum_{a,b=1}^n\big(\widehat{T}^{aab}{}_{,b}\(\xi\)+\widehat{T}^{aba}{}_{,b}\(\xi\)\\
&+\widehat{T}^{abb}{}_{,a}\(\xi\)\big)=\sum_{a,b=1}^n\big(T_{aab;b}\(\xi\)+T_{aba;b}\(\xi\)+T_{abb;a}\(\xi\)\big)
\end{aligned}\right\}\quad\text{if }j=3,\label{Step3Eq29}
\end{align}
and
\begin{align}\label{Step3Eq30}
&\sum_{a,b=1}^n\big(\widehat{T}^{aabb}\(\xi\)+\widehat{T}^{abab}\(\xi\)+\widehat{T}^{abba}\(\xi\)\big)\nonumber\\
&\qquad=\sum_{a,b=1}^n\(T_{aabb}\(\xi\)+T_{abab}\(\xi\)+T_{abba}\(\xi\)\)\quad\text{if }j=4.
\end{align}
We then obtain \eqref{Step3Eq2} by putting together \eqref{Step3Eq25}, \eqref{Step3Eq26} and \eqref{Step3Eq27} and using the identities
$$c\(n,k,0,k-2,l',0,0\)=\frac{n-2k-2}{2\(n+l'-2k-1\)!}$$
and
\begin{multline*}
\Beta\(\frac{n}{2}+l'-2k,2k-l'+4\)\\
=\frac{4\(2k-l'-1\)\(2k-l'-2\)}{\(n-2\)\(n-4\)}\Beta\(\frac{n}{2}+l'-2k,2k-l'-2\).
\end{multline*}
The estimates \eqref{Step3Eq3}--\eqref{Step3Eq7} follow in the same way from \eqref{Step3Eq25}, \eqref{Step3Eq26} and \eqref{Step3Eq27}--\eqref{Step3Eq30} by using the identities
\begin{align*}
&c\(n,k,1,k-2,l',0,0\)=-\frac{\(n-2k-2\)\(n-2k\)}{4\(n+l'-2k\)!}\,,\allowdisplaybreaks\\
&c\(n,k,2,k-2,l',0,0\)=\frac{\(n-2k-2\)\(n-2k\)\(n-2k+2\)}{32\(n+l'-2k+1\)!}\,,\allowdisplaybreaks\\
&2c\(n,k,2,k-2,l',0,1\)+c\(n,k,2,k-2,l',1,1\)\\
&\qquad=4c\(n,k,2,k-2,l',0,0\)+c\(n,k,2,k-2,l',1,0\)\allowdisplaybreaks\\
&\qquad=2c\(n,k,2,k-3,l',0,0\)+c\(n,k,2,k-3,l',1,0\)\\
&\qquad=-\frac{\(n-2k-2\)\(n-2k\)\(n+2l'-2k\)}{8\(n+l'-2k+1\)!}\,,\allowdisplaybreaks\\
&24c\(n,k,3,k-3,l',0,0\)+2c\(n,k,3,k-3,l',1,0\)\allowdisplaybreaks\\
&\qquad=\frac{\(n-2k-2\)\(n-2k\)\(n-2k+2\)\(n+2l'-2k\)}{8\(n+l'-2k+2\)!}\allowdisplaybreaks\\
&24c\(n,k,4,k-4,l',0,0\)+2c\(n,k,4,k-4,l',1,0\)+c\(n,k,4,k-4,l',2,0\)\\
&\qquad=\frac{\(n-2k-2\)\(n-2k\)\(n-2k+2\)\(n+2l'-2k\)\(n+2l'-2k+2\)}{64\(n+l'-2k+3\)!}\,.
\end{align*}
and
\begin{align*}
&\Beta\(\frac{n}{2}+l'-l-k-2,k+l-l'+2\)\nonumber\\
&\qquad=\frac{2\(k+l-l'+1\)}{n-2}\Beta\(\frac{n}{2}+l'-l-k-2,k+l-l'+1\)\nonumber\allowdisplaybreaks\\
&\qquad=\frac{4\(k+l-l'+1\)\(k+l-l'\)}{\(n-2\)\(n-4\)}\Beta\(\frac{n}{2}+l'-l-k-2,k+l-l'\).
\end{align*}
This ends the proof of Step~\ref{Step3}.
\endproof

As regards the integral in the denominator of $I_{k,f,g}\(u\)$, we obtain the following:

\begin{step}\label{Step4}
Assume that $n\ge 2k+4$. Then for every smooth function $f$, as $\mu\to0$,
\begin{equation}\label{Step4Eq}
\int_MfU_\mu^{2^*_k}\,dv_g=\frac{\omega_n}{2^n}f\(\xi\)-\frac{\omega_n\Delta f\(\xi\)\mu^2}{2^{n+1}\(n-2\)}\\+\frac{\omega_n\Delta^2f\(\xi\)\mu^4}{2^{n+3}\(n-2\)\(n-4\)}+\smallo\(\mu^4\).
\end{equation}
\end{step}

\proof[Proof of Step~\ref{Step4}]
By remarking that $U_\mu^{2^*_k}$ does not depend on $k$ in $B\(0,r_0\)$, we obtain that \eqref{Step4Eq} is in fact identical to an estimate obtained by Esposito and Robert~\cite{EspRob} in the case where $k=2$ (note that in our case, $\Ricci\(\xi\)=0$ and $\nabla\Scal\(\xi\)=0$ since we are working with conformal normal coordinates, see \eqref{Eq4} and \eqref{Eq5}). 
\endproof

We can now end the proof of Proposition~\ref{Pr} by putting together the results of Steps~\ref{Step1}--~\ref{Step4}:

\proof[End of proof of Proposition~\ref{Pr}]
We assume that $k\ge2$ and refer to Aubin~\cite{Aub} for the case where $k=1$. By using \eqref{Step4Eq}, we obtain
\begin{multline}\label{PrEq2}
\(\int_MfU_\mu^{2^*_k}\,dv_g\)^{-\frac{n-2k}{n}}=\(\frac{\omega_n}{2^n}f\(\xi\)\)^{-\frac{n-2k}{n}}
\Bigg[1+\frac{\(n-2k\)\Delta f\(\xi\)\mu^2}{2n\(n-2\)f\(\xi\)}\\
-\frac{n-2k}{4n\(n-2\)}\(\frac{\Delta^2f\(\xi\)}{2\(n-4\)f\(\xi\)}-\frac{\(n-k\)\(\Delta f\(\xi\)\)^2}{n\(n-2\)f\(\xi\)^2}\)\mu^4+\smallo\(\mu^4\)\Bigg].
\end{multline}
We let $f_1$, $f_2$,  $T_1$, $T_2$, $T_3$, $T_4$, $T_5$ and $Z$ be as in Step~\ref{Step1}. 
Since $k\ge1$, by integrating by parts, we obtain
\begin{multline}\label{PrEq3}
\int_MU_\mu\Delta^{k-1}\(f_1U_\mu\)dv_g=\int_M\Delta\(f_1U_\mu\)\Delta^{k-2}U_\mu\,dv_g\\
=\int_M\(U_\mu\Delta f_1-2\(\nabla f_1,\nabla U_\mu\)+f_1\Delta U_\mu\)\Delta^{k-2}U_\mu\,dv_g.
\end{multline}
By integrating by parts again, it follows from \eqref{Step1Eq1} and \eqref{PrEq3} that
\begin{multline}\label{PrEq4}
\int_MU_\mu P_{2k}U_\mu\,dv_g=\int_MU_\mu\Delta^kU_\mu\,dv_g+k\int_M\big(\(\(k-1\)f_2+\Delta f_1\)U_\mu\\
+\(\(k-1\)T_1-2\nabla f_1,\nabla U_\mu\)+\(\(k-1\)T_2-f_1g,\nabla^2U_\mu\)\big)\Delta^{k-2}U_\mu\,dv_g\allowdisplaybreaks\\
+k\(k-1\)\(k-2\)\int_M\(\(T_3,\nabla^2U_\mu\)+\(T_4,\nabla^3U_\mu\)\)\Delta^{k-3}U_\mu\,dv_g\\
+k\(k-1\)\(k-2\)\(k-3\)\int_M\(T_5,\nabla^4U_\mu\)\Delta^{k-4}U_\mu dv_g+\int_MU_\mu ZU_\mu\,dv_g.
\end{multline}
By using \eqref{Eq5}, we obtain 
\begin{equation}\label{PrEq5}
\sum_{a,b=1}^n\Schouten_{aa;bb}\(\xi\)=\sum_{a,b=1}^n\Schouten_{ab;ab}\(\xi\)=\sum_{a,b=1}^n\Schouten_{ab;ba}\(\xi\)=-\frac{\left|\Weyl\(\xi\)\right|^2}{12\(n-1\)}\,.
\end{equation}
By using \eqref{Eq4}, \eqref{Eq5} and \eqref{PrEq5} together with straightforward computations, we obtain
\begin{align*}
&\sum_{a=1}^n\(T_3\)_{aa}\(\xi\)=-\frac{n+3k+1}{36\(n-1\)}\left|\Weyl\(\xi\)\right|^2=-\frac{n+3k+1}{36n\(n-1\)}\left|\Weyl\(\xi\)\right|^2\sum_{a=1}^ng_{aa}\(\xi\),\allowdisplaybreaks\\
&\sum_{a,b=1}^n\big(\(T_4\)_{aab;b}\(\xi\)+\(T_4\)_{aba;b}\(\xi\)+\(T_4\)_{abb;a}\(\xi\)\big)=-\frac{k+1}{6\(n-1\)}\left|\Weyl\(\xi\)\right|^2\\
&=\frac{k+1}{\(n-1\)\(n+2\)}\sum_{a,b=1}^n\big(\(\nabla S\otimes g\)_{aab;b}\(\xi\)+\(\nabla S\otimes g\)_{aba;b}\(\xi\)+\(\nabla S\otimes g\)_{abb;a}\(\xi\)\big)
\end{align*}
and
\begin{multline*}
\sum_{a,b=1}^n\big(\(T_5\)_{aabb}\(\xi\)+\(T_5\)_{abab}\(\xi\)+\(T_5\)_{abba}\(\xi\)\big)=-\frac{k+1}{10\(n-1\)}\left|\Weyl\(\xi\)\right|^2\\
=-\frac{\(k+1\)\left|\Weyl\(\xi\)\right|^2}{10n\(n-1\)\(n+2\)}\sum_{a,b=1}^n\big(\(g\otimes g\)_{aabb}\(\xi\)+\(g\otimes g\)_{abab}\(\xi\)+\(g\otimes g\)_{abba}\(\xi\)\big).
\end{multline*}
By using these identities together with \eqref{Step3Eq5}--\eqref{Step3Eq7} and remarking that
$$\(\nabla S\otimes g,\nabla^3U_\mu\)=-\Delta\(\nabla S,\nabla U_\mu\)-2\(\nabla^2S,\nabla^2U_\mu\)-\(\nabla^3S,\nabla U_\mu\otimes g\),$$
and
$$\sum_{a=1}^n\(\nabla^2 S\)_{aa}\(\xi\)=-\frac{1}{6}\left|\Weyl\(\xi\)\right|^2=-\frac{1}{6n}\left|\Weyl\(\xi\)\right|^2\sum_{a=1}^ng_{aa}\(\xi\),$$
we obtain that for $k\ge3$,
\begin{align}
&\int_M\(T_3,\nabla^2U_\mu\)\Delta^{k-3}U_\mu\,dv_g\nonumber\\
&\quad=\frac{n+3k+1}{36n\(n-1\)}\left|\Weyl\(\xi\)\right|^2\int_M\Delta U_\mu\Delta^{k-3}U_\mu\,dv_g+\left\{\begin{aligned}&\bigO\(\mu^4\)&&\text{if }n=2k+4\\&\smallo\(\mu^4\)&&\text{if }n>2k+4\end{aligned}\right.\nonumber\\
&\quad=\frac{n+3k+1}{36n\(n-1\)}\left|\Weyl\(\xi\)\right|^2\int_MU_\mu\Delta^{k-2}U_\mu\,dv_g+\left\{\begin{aligned}&\bigO\(\mu^4\)&&\text{if }n=2k+4\\&\smallo\(\mu^4\)&&\text{if }n>2k+4,\end{aligned}\right.\label{PrEq6}\allowdisplaybreaks\\
&\int_M\(T_4,\nabla^3U_\mu\)\Delta^{k-3}U_\mu\,dv_g=-\frac{k+1}{\(n-1\)\(n+2\)}\int_M\bigg(\Delta\(\nabla S,\nabla U_\mu\)\nonumber\\
&\quad+\frac{1}{3n}\left|\Weyl\(\xi\)\right|^2\Delta U_\mu+\(\nabla^3S,\nabla U_\mu\otimes g\)\bigg)\Delta^{k-3}U_\mu\,dv_g+\left\{\begin{aligned}&\bigO\(\mu^4\)&&\text{if }n=2k+4\\&\smallo\(\mu^4\)&&\text{if }n>2k+4\end{aligned}\right.\nonumber\allowdisplaybreaks\\
&\quad=-\frac{k+1}{3n\(n-1\)\(n+2\)}\int_M\(\left|\Weyl\(\xi\)\right|^2U_\mu+3n\(\nabla S,\nabla U_\mu\)\)\Delta^{k-2}U_\mu\,dv_g\nonumber\\
&\quad-\frac{k+1}{\(n-1\)\(n+2\)}\int_MU_\mu\Delta^{k-3}\(\nabla^3S,\nabla U_\mu\otimes g\)dv_g+\left\{\begin{aligned}&\bigO\(\mu^4\)&&\text{if }n=2k+4\\&\smallo\(\mu^4\)&&\text{if }n>2k+4\end{aligned}\right.\label{PrEq7}
\end{align}
and for $k\ge4$,
\begin{align}
&\int_M\(T_5,\nabla^4U_\mu\)\Delta^{k-4}U_\mu\,dv_g\nonumber\\
&\quad=-\frac{\(k+1\)\left|\Weyl\(\xi\)\right|^2}{10n\(n-1\)\(n+2\)}\int_M\(\Delta^2U_\mu\)\Delta^{k-4}U_\mu\,dv_g+\left\{\begin{aligned}&\bigO\(\mu^4\)&&\text{if }n=2k+4\\&\smallo\(\mu^4\)&&\text{if }n>2k+4\end{aligned}\right.\nonumber\\
&\quad=-\frac{\(k+1\)\left|\Weyl\(\xi\)\right|^2}{10n\(n-1\)\(n+2\)}\int_MU_\mu\Delta^{k-2}U_\mu\,dv_g+\left\{\begin{aligned}&\bigO\(\mu^4\)&&\text{if }n=2k+4\\&\smallo\(\mu^4\)&&\text{if }n>2k+4.\end{aligned}\right.\label{PrEq8}
\end{align}
It follows from \eqref{PrEq4} and \eqref{PrEq6}--\eqref{PrEq8} that
\begin{multline}\label{PrEq9}
\int_MU_\mu P_{2k}U_\mu\,dv_g=\int_MU_\mu\Delta^kU_\mu\,dv_g+k\int_M\bigg(\bigg(\(k-1\)f_2+\Delta f_1\\
+\(\frac{\(k-1\)\(k-2\)\(n+3k+1\)}{36n\(n-1\)}-\frac{\(k-1\)\(k-2\)\(k+1\)\(3k+1\)}{30n\(n-1\)\(n+2\)}\)\left|\Weyl\(\xi\)\right|^2\bigg)U_\mu\allowdisplaybreaks\\
+\(\(k-1\)T_1-2\nabla f_1-\frac{\(k+1\)\(k-1\)\(k-2\)}{\(n-1\)\(n+2\)}\nabla S,\nabla U_\mu\)\\
+\(\(k-1\)T_2-f_1g,\nabla^2U_\mu\)\bigg)\Delta^{k-2}U_\mu\,dv_g\allowdisplaybreaks\\
+\int_MU_\mu\(ZU_\mu-\frac{k\(k+1\)\(k-1\)\(k-2\)}{\(n-1\)\(n+2\)}\Delta^{k-3}\(\nabla^3S,\nabla U_\mu\otimes g\)\)\,dv_g\\
+\left\{\begin{aligned}&\bigO\(\mu^4\)&&\text{if }n=2k+4\\&\smallo\(\mu^4\)&&\text{if }n>2k+4.\end{aligned}\right.
\end{multline}
By using \eqref{Eq4}, \eqref{Eq5} and \eqref{PrEq5} together with straightforward computations, we obtain
\begin{align*}
&f_1\(\xi\)=0,\quad \Delta f_1\(\xi\)=\frac{n-2}{24\(n-1\)}\left|\Weyl\(\xi\)\right|^2,\quad f_2\(\xi\)=-\frac{3n+2k-4}{144\(n-1\)}\left|\Weyl\(\xi\)\right|^2,\allowdisplaybreaks\\
&\sum_{a=1}^n\(T_1\)_{a;a}\(\xi\)=-\frac{3n+4k-2}{72\(n-1\)}\left|\Weyl\(\xi\)\right|^2,\quad\sum_{a=1}^n\(T_2\)_{aa}\(\xi\)=0
\end{align*}
and
$$\sum_{a,b=1}^n\(T_2\)_{aa;bb}\(\xi\)=\sum_{a,b=1}^n\(T_2\)_{ab;ab}\(\xi\)=\sum_{a,b=1}^n\(T_2\)_{ab;ba}\(\xi\)=-\frac{k+1}{18\(n-1\)}\left|\Weyl\(\xi\)\right|^2.$$
By using these identities together with \eqref{Step2Eq1}, \eqref{Step3Eq2}--\eqref{Step3Eq4} and \eqref{PrEq9}, we obtain that \eqref{PrEq1} holds true with $C\(n,k\)$ defined as
\begin{multline}\label{PrEq10}
C\(n,k\):=\frac{\(n-3\)\(n-5\)!\,k!}{16\(k-1\)\(n-2k-2\)}\\
\times\sum_{l=k-2}^{2k-4}\frac{l!}{\(l-k+2\)!\(2k-l-4\)!\(n+l-2k+1\)!}\bigg(8\(n+l-2k\)\(n+l-2k+1\)\\
\bigg(\frac{\(k-1\)\(3n+2k-4\)}{144}-\frac{n-2}{24}-\frac{\(k-1\)\(k-2\)\(n+3k+1\)}{36n}\\
+\frac{\(k-1\)\(k-2\)\(k+1\)\(3k+1\)}{30n\(n+2\)}\bigg)+4\(n-2k\)\(n+l-2k+1\)\bigg(\frac{n-2}{12}\\
-\frac{\(k-1\)\(3n+4k-2\)}{72}+\frac{\(k+1\)\(k-1\)\(k-2\)}{6\(n+2\)}\bigg)\\
+\(n-2k\)\bigg(\frac{\(k+1\)\(k-1\)\(n-2k-2l+4\)}{18}\\
-\frac{\(n-2\)\(2\(n-2k+2\)-n\(n+2l-2k\)\)}{24}\bigg)\bigg)\\
\times\Beta\(\frac{n}{2}-k-1,l+1\)^{-1}\left\{\begin{aligned}&2\chi_{\left\{l=k-2\right\}}&&\text{if }n=2k+4\\&\Beta\(\frac{n}{2}+l-2k,2k-l-2\)&&\text{otherwise}\end{aligned}\right.\\
=\frac{\(n-3\)\(n-5\)!\,k!}{5760n\(n+2\)\(k-1\)\(n-2k-2\)}\\
\times\sum_{l=k-2}^{2k-4}\frac{l!\,c\(n,k,l\)}{\(l-k+2\)!\(2k-l-4\)!\(n+l-2k+1\)!}\allowdisplaybreaks\\
\times\Beta\(\frac{n}{2}-k-1,l+1\)^{-1}\left\{\begin{aligned}&2\chi_{\left\{l=k-2\right\}}&&\text{if }n=2k+4\\&\Beta\(\frac{n}{2}+l-2k,2k-l-2\)&&\text{otherwise,}\end{aligned}\right.
\end{multline}
where
\begin{multline*}
c\(n,k,l\):=4\(n+l-2k\)\(n+l-2k+1\)(5n\(n+2\)\(k-1\)\(3n+2k-4\)\\
-30n\(n+2\)\(n-2\)-20\(n+2\)\(k-1\)\(k-2\)\(n+3k+1\)\allowdisplaybreaks\\
+24\(k-1\)\(k-2\)\(k+1\)\(3k+1\))+20n\(n-2k\)\(n+l-2k+1\)\allowdisplaybreaks\\
\times\(6\(n+2\)\(n-2\)-\(n+2\)\(k-1\)\(3n+4k-2\)+12\(k+1\)\(k-1\)\(k-2\)\)\allowdisplaybreaks\\
+5n\(n+2\)\(n-2k\)(4\(k+1\)\(k-1\)\(n-2k-2l+4\)\\
-3\(n-2\)\(2\(n-2k+2\)-n\(n+2l-2k\)\)).
\end{multline*}
By letting $k:=3+a$, $n:=2k+4+b$ and $l:=k-2+c$ and using the software {\it Maple} to expand the expression of $c\(n,k,l\)$, we then obtain
\begin{multline}\label{PrEq11}
c\(n,k,l\)=4(15ab^3+1200a^2b+3880ab+1920+10656a+480b+4528a^2+624a^3\\
+40b^2+450ab^2+80a^3b+80a^2b^2+32a^4)c^2+2(71552a^2b+414912a+500a^2b^3+247984ab\allowdisplaybreaks\\
+31840a^3+53660ab^2+3200a^4+640a^3b^2+11020b^3+150ab^4+128a^5+9056a^3b+660b^4\allowdisplaybreaks\\
+161440a^2+448a^4b+15b^5+311040b+10520a^2b^2+4830ab^3+426240+85840b^2)c\allowdisplaybreaks\\
+128a^6+576a^5b+1088a^4b^2+1020a^3b^3+560a^2b^4+150ab^5+15b^6+3904a^5+14720a^4b\allowdisplaybreaks\\
+21896a^3b^2+15940a^2b^3+5640ab^4+720b^5+49408a^4+149280a^3b+167032a^2b^2\allowdisplaybreaks\\
+81120ab^3+13780b^4+332096a^3+754720a^2b+563824ab^2+134240b^3+1250304a^2\\
+1900224ab+704640b^2+2499840a+1900800b+2073600.
\end{multline}
Since all the coefficients in this expression are positive, it follows that $C\(n,k\)$ is positive whenever $k\ge3$, $n\ge2k+4$ and $l\ge k-2$. Furthermore, in the case where $k=2$ and $l=0$, we find
$$c\(n,2,0\)=5n(n+2)(n-4)^2(n^2-4n-4)>0\quad\forall n\ge8.$$
Therefore, in all cases, we find that $C\(n,k\)$ is positive. This ends the proof of Proposition~\ref{Pr}.
\endproof

We can now prove Theorem~\ref{Th} by using Proposition~\ref{Pr}.

\proof[Proof of Theorem~\ref{Th}]
Let $\xi\in M$ be a maximal point of $f$ and $\widetilde{g}=\varphi^{4/\(n-2\)}g$ be a conformal metric to $g$ such that $\det\widetilde{g}\(x\)=1$ for all $x$ in a neighborhood of the point $\xi$. As easily follows from the analysis of Lee and Parker~\cite{LeePar}, we may choose $\varphi$ such that
\begin{equation}\label{ThEq2}
\varphi\(\xi\)=1,\quad\nabla\varphi\(\xi\)=0\quad\text{and}\quad\nabla^2\varphi\(\xi\)=\frac{n-2}{2}\Schouten\(\xi\),
\end{equation}
where the covariant derivatives and the Schouten tensor are with respect to the metric $g$. By using \eqref{ThEq2} together with the fact that $\nabla f\(\xi\)=0$, we then obtain that if $\Delta_gf\(\xi\)=0$, then
\begin{equation}\label{ThEq3}
\Delta_{\widetilde{g}}f\(\xi\)=0\quad\text{and}\quad\Delta_{\widetilde{g}}^2f\(\xi\)=\Delta_g^2f\(\xi\)+2\(\Ricci\(\xi\),\nabla^2f\(\xi\)\),
\end{equation}
where $\Delta_g$ and $\Delta_{\widetilde{g}}$ are the Laplace--Beltrami operators with respect to the metrics $g$ and $\widetilde{g}$, respectively, and the covariant derivatives, the Ricci tensor and the multiple inner product in the right-hand side of the second identity are with respect to the metric $g$. Let $c\(n,k\)$ be the constant defined as 
\begin{equation}\label{ThEq4}
c\(n,k\):=\left\{\begin{aligned}&0&&\text{if }n=2k+4\\&\frac{\(n-2k\)\(2k-1\)!}{8n\(n-2\)\(n-4\)C\(n,k\)}\Beta\(\frac{n}{2}-k,2k\)^{-1}&&\text{if }n>2k+4,\end{aligned}\right.
\end{equation}
where $C\(n,k\)$ is as in \eqref{PrEq1} (see also \eqref{PrEq10}). By applying Proposition~\ref{Pr} together with \eqref{ThEq3} and the fact that $\left|W\right|$ is conformally invariant, we then obtain that if \eqref{ThEq1} holds true, then
\begin{equation}\label{ThEq5}
\inf_{u\in C^{2k}\(M\)\backslash\left\{0\right\}}\hspace{-2pt}I_{k,f,\widetilde{g}}\(u\)<\omega_n^{\frac{2k}{n}}\(2k-1\)!\Beta\(\frac{n}{2}-k,2k\)^{-1}\Big(\max_{x\in M} f\(x\)\Big)^{-\frac{n-2k}{n}}.
\end{equation}
On the other hand, by conformal invariance of the operator $P_{2k}$, we obtain
\begin{equation}\label{ThEq6}
\inf_{u\in C^{2k}\(M\)\backslash\left\{0\right\}}\hspace{-2pt}I_{k,f,\widetilde{g}}\(u\)=\inf_{u\in C^{2k}\(M\)\backslash\left\{0\right\}}\hspace{-2pt}I_{k,f,g}\(u\).
\end{equation}
By putting together \eqref{ThEq5} and \eqref{ThEq6} and applying Theorem~3 of Mazumdar~\cite{Maz}, we then obtain that the conclusions of Theorem~\ref{Th} hold true. 
\endproof

\begin{thebibliography}{99}

\bib{Aub}{article}{
   author={Aubin, Th.},
   title={\'{E}quations diff\'{e}rentielles non lin\'{e}aires et probl\`eme de Yamabe concernant la courbure scalaire},
   journal={J. Math. Pures Appl. (9)},
   volume={55},
   date={1976},
   number={3},
   pages={269--296},
   language={French},
}

\bib{BaiFarReg}{article}{
   author={Baird, P.},
   author={Fardoun, A.},
   author={Regbaoui, R.},
   title={Prescribed $Q$-curvature on manifolds of even dimension},
   journal={J. Geom. Phys.},
   volume={59},
   date={2009},
   number={2},
   pages={221--233},
}

\bib{Bra1}{article}{
   author={Branson, T. P.},
   title={Differential operators canonically associated to a conformal structure},
   journal={Math. Scand.},
   volume={57},
   date={1985},
   number={2},
   pages={293--345},
}

\bib{Bra2}{article}{
   author={Branson, T. P.},
   title={Sharp inequalities, the functional determinant, and the complementary series},
   journal={Trans. Amer. Math. Soc.},
   volume={347},
   date={1995},
   number={10},
   pages={3671--3742},
}

\bib{Bre}{article}{
   author={Brendle, S.},
   title={Blow-up phenomena for the Yamabe equation},
   journal={J. Amer. Math. Soc.},
   volume={21},
   date={2008},
   number={4},
   pages={951--979},
}

\bib{Cao}{article}{
   author={Cao, J. G.},
   title={The existence of generalized isothermal coordinates for higher-dimensional Riemannian manifolds},
   journal={Trans. Amer. Math. Soc.},
   volume={324},
   date={1991},
   number={2},
   pages={901--920},
}

\bib{ChangYang}{article}{
   author={Chang, S.-Y. A.},
   author={Yang, Paul C.},
   title={Extremal metrics of zeta function determinants on $4$-manifolds},
   journal={Ann. of Math. (2)},
   volume={142},
   date={1995},
   number={1},
   pages={171--212},
}

\bib{ChenHou}{article}{
   author={Chen, X.},
   author={Hou, F.},
   title={Remarks on GJMS operator of order six},
   journal={Pacific J. Math.},
   volume={289},
   date={2017},
   number={1},
   pages={35--70},
}

\bib{DjaHebLed}{article}{
   author={Djadli, Z.},
   author={Hebey, E.},
   author={Ledoux, M.},
   title={Paneitz-type operators and applications},
   journal={Duke Math. J.},
   volume={104},
   date={2000},
   number={1},
   pages={129--169},
}

\bib{DjaMal}{article}{
   author={Djadli, Z.},
   author={Malchiodi, A.},
   title={Existence of conformal metrics with constant $Q$-curvature},
   journal={Ann. of Math. (2)},
   volume={168},
   date={2008},
   number={3},
   pages={813--858},
}

\bib{EscSch}{article}{
   author={Escobar, J. F.},
   author={Schoen, R. M.},
   title={Conformal metrics with prescribed scalar curvature},
   journal={Invent. Math.},
   volume={86},
   date={1986},
   number={2},
   pages={243--254},
}

\bib{EspRob}{article}{
   author={Esposito, P.},
   author={Robert, F.},
   title={Mountain pass critical points for Paneitz-Branson operators},
   journal={Calc. Var. Partial Differential Equations},
   volume={15},
   date={2002},
   number={4},
   pages={493--517},
}

\bib{FefGra1}{article}{
   author={Fefferman, C.},
   author={Graham, C. R.},
   title={Conformal invariants},
   journal={\'Elie Cartan et les math\'ematiques d'aujourd'hui - Lyon, 25-29 juin 1984, Ast\'{e}risque},
   date={1985},
   number={S131},
   pages={95--116},
}

\bib{FefGra2}{book}{
   author={Fefferman, C.},
   author={Graham, C. R.},
   title={The ambient metric},
   series={Annals of Mathematics Studies},
   volume={178},
   publisher={Princeton University Press, Princeton, NJ},
   date={2012},
}

\bib{FefGra3}{article}{
   author={Fefferman, C.},
   author={Graham, C. R.},
   title={Juhl's formulae for GJMS operators and $Q$-curvatures},
   journal={J. Amer. Math. Soc.},
   volume={26},
   date={2013},
   number={4},
   pages={1191--1207},
}

\bib{Gov}{article}{
   author={Gover, A. R.},
   title={Laplacian operators and $Q$-curvature on conformally Einstein manifolds},
   journal={Math. Ann.},
   volume={336},
   date={2006},
   number={2},
   pages={311--334},
}

\bib{GraJenMasSpa}{article}{
   author={Graham, C. R.},
   author={Jenne, R.},
   author={Mason, L. J.},
   author={Sparling, G. A. J.},
   title={Conformally invariant powers of the Laplacian. I. Existence},
   journal={J. London Math. Soc. (2)},
   volume={46},
   date={1992},
   number={3},
   pages={557--565},
}

\bib{Gun}{article}{
   author={G\"{u}nther, M.},
   title={Conformal normal coordinates},
   journal={Ann. Global Anal. Geom.},
   volume={11},
   date={1993},
   number={2},
   pages={173--184},
}

\bib{GurHangLin}{article}{
   author={Gursky, M. J.},
   author={Hang, F.},
   author={Lin, Y.-J.},
   title={Riemannian manifolds with positive Yamabe invariant and Paneitz operator},
   journal={Int. Math. Res. Not. IMRN},
   date={2016},
   number={5},
   pages={1348--1367},
}

\bib{GurMal}{article}{
   author={Gursky, M. J.},
   author={Malchiodi, A.},
   title={A strong maximum principle for the Paneitz operator and a non-local flow for the $Q$-curvature},
   journal={J. Eur. Math. Soc. (JEMS)},
   volume={17},
   date={2015},
   number={9},
   pages={2137--2173},
}

\bib{HangYang1}{article}{
   author={Hang, F.},
   author={Yang, P. C.},
   title={Sign of Green's function of Paneitz operators and the $Q$ curvature},
   journal={Int. Math. Res. Not. IMRN},
   date={2015},
   number={19},
   pages={9775--9791},
}

\bib{HangYang2}{article}{
   author={Hang, F.},
   author={Yang, P. C.},
   title={$Q$-curvature on a class of manifolds with dimension at least 5},
   journal={Comm. Pure Appl. Math.},
   volume={69},
   date={2016},
   number={8},
   pages={1452--1491},
}

\bib{Heb}{article}{
   author={Hebey, E.},
   title={Changements de m\'{e}triques conformes sur la sph\`ere. Le probl\`eme de
   Nirenberg},
   language={French, with English summary},
   journal={Bull. Sci. Math.},
   volume={114},
   date={1990},
   number={2},
   pages={215--242},
}

\bib{HebVau}{article}{
   author={Hebey, E.},
   author={Vaugon, M.},
   title={Courbure scalaire prescrite pour des vari\'{e}t\'{e}s non conform\'{e}ment
   diff\'{e}omorphes \`a la sph\`ere},
   language={French},
   journal={C. R. Acad. Sci. Paris S\'{e}r. I Math.},
   volume={316},
   date={1993},
   number={3},
   pages={281--282},
}

\bib{HumRau}{article}{
   author={Humbert, E.},
   author={Raulot, S.},
   title={Positive mass theorem for the Paneitz-Branson operator},
   journal={Calc. Var. Partial Differential Equations},
   volume={36},
   date={2009},
   number={4},
   pages={525--531},
}

\bib{Juhl}{article}{
   author={Juhl, A.},
   title={Explicit formulas for GJMS-operators and $Q$-curvatures},
   journal={Geom. Funct. Anal.},
   volume={23},
   date={2013},
   number={4},
   pages={1278--1370},
}

\bib{LeePar}{article}{
   author={Lee, J. M.},
   author={Parker, T. H.},
   title={The Yamabe problem},
   journal={Bull. Amer. Math. Soc. (N.S.)},
   volume={17},
   date={1987},
   number={1},
   pages={37--91},
}

\bib{LiLiLiu}{article}{
   author={Li, J.},
   author={Li, Y.},
   author={Liu, P.},
   title={The $Q$-curvature on a 4-dimensional Riemannian manifold $(M,g)$ with $\int_MQdV_g=8\pi^2$},
   journal={Adv. Math.},
   volume={231},
   date={2012},
   number={3-4},
   pages={2194--2223},
}

\bib{Maz}{article}{
   author={Mazumdar, S.},
   title={GJMS-type operators on a compact Riemannian manifold: best constants and Coron-type solutions},
   journal={J. Differential Equations},
   volume={261},
   date={2016},
   number={9},
   pages={4997--5034},
}

\bib{Mic}{article}{
   author={Michel, B.},
   title={Masse des op\'erateurs GJMS},
   journal={arXiv:1012.4414v1},
   date={2010}, 
   language={French},
}

\bib{Pan}{article}{
   author={Paneitz, S. M.},
   title={A quartic conformally covariant differential operator for arbitrary pseudo-Riemannian manifolds (summary)},
   journal={SIGMA Symmetry Integrability Geom. Methods Appl.},
   volume={4},
   date={2008},
   number={36},
}

\bib{QingRas}{article}{
   author={Qing, J.},
   author={Raske, D.},
   title={On positive solutions to semilinear conformally invariant equations on locally conformally flat manifolds},
   journal={Int. Math. Res. Not.},
   volume={2006},
   date={2006},
   number={94172},
}

\bib{Rob1}{article}{
   author={Robert, F.},
   title={Positive solutions for a fourth order equation invariant under isometries},
   journal={Proc. Amer. Math. Soc.},
   volume={131},
   date={2003},
   number={5},
   pages={1423--1431},
}

\bib{Rob2}{article}{
   author={Robert, F.},
   title={Admissible $Q$-curvatures under isometries for the conformal GJMS operators},
   conference={
      title={Nonlinear elliptic partial differential equations},
   },
   book={
      series={Contemp. Math.},
      volume={540},
      publisher={Amer. Math. Soc., Providence, RI},
   },
   date={2011},
   pages={241--259},
}

\bib{Sch1}{article}{
   author={Schoen, R. M.},
   title={Conformal deformation of a Riemannian metric to constant scalar curvature},
   journal={J. Differential Geom.},
   volume={20},
   date={1984},
   number={2},
   pages={479--495},
}

\bib{Sch2}{article}{
   author={Schoen, R. M.},
   title={On the number of constant scalar curvature metrics in a conformal class},
   book={title={Differential geometry}, series={Pitman Monogr. Surveys Pure Appl. Math.},volume={52},publisher={ Longman Sci. Tech.},place={Harlow},},
   date={1991},
   pages={311--320},
}

\bib{SchYau}{article}{
   author={Schoen, R.},
   author={Yau, S.-T.},
   title={On the proof of the positive mass conjecture in general
   relativity},
   journal={Comm. Math. Phys.},
   volume={65},
   date={1979},
   number={1},
}

\bib{Tru}{article}{
   author={Trudinger, N. S.},
   title={Remarks concerning the conformal deformation of Riemannian structures on compact manifolds},
   journal={Ann. Scuola Norm. Sup. Pisa (3)},
   volume={22},
   date={1968},
   pages={265--274},
}

\bib{Yam}{article}{
   author={Yamabe, H.},
   title={On a deformation of Riemannian structures on compact manifolds},
   journal={Osaka Math. J.},
   volume={12},
   date={1960},
   pages={21--37},
}

\end{thebibliography}

\end{document}
