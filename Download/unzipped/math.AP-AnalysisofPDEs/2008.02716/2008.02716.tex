\documentclass[12pt]{amsart}
%\usepackage[active]{srcltx}

%\usepackage[english,french]{babel}

\usepackage[latin1]{inputenc}


\usepackage[margin=2.5cm,bottom=3cm,top=3cm]{geometry}
% \usepackage{showkeys} %\usepackage{tikz}
\usepackage{color}


\usepackage{hyperref}
\hypersetup{colorlinks=true}


\usepackage{amsmath,amsfonts,amsthm,amssymb} \pagestyle{plain}
%\oddsidemargin .5in \evensidemargin .5pt \marginparsep 0pt \topmargin
%.0pt \marginparwidth 0pt \textwidth 5.5in \textheight 8.5in



 
\newcommand\N{{\mathbb N}} \newcommand\R{{\mathbb R}}
\newcommand\Z{{\mathbb Z}} \newtheorem{thm}{Theorem}
\newtheorem{conjecture}{Conjecture}
\newtheorem{rmq}{Remark}[section] \newtheorem{lemma}{Lemma} %
\newtheorem{cor}{Corollaire} \newtheorem{prop}{Proposition}
%\newtheorem{definition}[remark]{D�finition}
\newcommand\e{{\epsilon}}
\newcommand{\xx}{\mathrm{x}}
\newcommand{\yy}{\mathrm{y}}
\newcommand{\DomainXYT}{\mathcal{R}}


\setlength{\parindent}{0em}
\setlength{\parskip}{0.5em}
\begin{document}


 \title{New counterexamples to Strichartz estimates for the wave equation on a 2D model convex domain}


  
  
\author{Oana Ivanovici}
\address{Sorbonne Universit�, CNRS, Laboratoire Jacques-Louis Lions, LJLL, F-75005 Paris, France} \email{oana.ivanovici@sorbonne-universite.fr}

  \author{Gilles Lebeau}
  \address{Universit\'e C�te d'Azur, CNRS, Laboratoire JAD, France} \email{gilles.lebeau@univ-cotedazur.fr}

  \author{Fabrice Planchon}
  \address{ Sorbonne Universit�, CNRS, Institut Math�matique de Jussieu-Paris Rive Gauche, IMJ-PRG F-75005 Paris, France}
  \email{fabrice.planchon@sorbonne-universite.fr} 
  
      \thanks{{\it Key words}  Dispersive estimates, wave equation, Dirichlet boundary condition.\\
    \\
 O.Ivanovici and F. Planchon were supported by ERC grant ANADEL 757 996.
    } 
%\date{\today}



\begin{abstract} We prove that the range of Strichartz estimates on a model 2D convex domain may be further restricted compared to the known counterexamples from \cite{doi,doi2}. Our new family of counterexamples is built on the parametrix construction from \cite{Annals} and revisited in \cite{ILP3}. Interestingly enough, it is sharp in at least some regions of phase space.
\end{abstract}
\maketitle

\section{Introduction and main results}

Let us consider the wave equation on a domain $\Omega$ with boundary
$\partial \Omega$ ,
\begin{equation} \label{WE} \left\{ \begin{array}{l} (\partial^2_t-
\Delta) u(t, x)=0, \;\; x\in \Omega \\ u|_{t=0} = u_0 \; \partial_t
u|_{t=0}=u_1,\\ Bu=0,\quad x\in \partial\Omega.
 \end{array} \right.
 \end{equation}
Here,  $\Delta$ stands for the Laplace-Beltrami operator on
$\Omega$. If $\partial\Omega\neq \emptyset$, the boundary condition
could be either Dirichlet ($B$ is the identity map:
$u|_{\partial\Omega}=0$) or Neuman ($B=\partial_{\nu}$ where $\nu$ is
the unit normal to the boundary.)

The so called Strichartz estimates aim at quantifying dispersive
properties of the solutions to this linear wave equation: for given
data in the natural energy space, the solution will have better decay
for suitable time averages.  This is of value for several applications, of which we quote only two:
\begin{itemize}
\item nonlinear problems, where Strichartz may be used as a tool to improve on Sobolev embeddings and allow for better nonlinear mapping properties of solutions;
\item  localization properties of (clusters of) eigenfunctions of the Laplacian (through square function estimates for the wave equation which are closely related to Strichartz estimates).
\end{itemize}


On any Riemannian manifold with empty boundary, the solution to \eqref{WE} is such that, at least for a suitable $T<+\infty$, for all $h<1$,
\begin{equation}\label{stricrd} h^{\beta}\|\chi(hD_t)u\|_{L^q([0,T],
L^r)}\leq C\left(\|u(0,x)\|_{L^2}+\|hD_t u\|_{L^2}\right),
\end{equation} 
where $\chi\in C^{\infty}_0$ is a smooth truncation in a neighborhood of $1$. Let $d$ be the spatial dimension of $\Omega$, then rescaling dictates that $\beta=d\left(\frac 12-\frac 1r\right)-\frac 1q$, where $(q,r)$ is a so-called admissible pair, e.g.
\begin{equation}\label{adm} 
\frac 1q\leq \frac{(d-1)} 2 \left(\frac{1}{2}-\frac{1}{r}\right),\quad  q>2.
\end{equation}
When \eqref{stricrd} holds for $T=\infty$, it is said to be a global in time Strichartz estimate. For $\Omega=\mathbb{R}^d$ with flat metric, the solution  $u_{\mathbb{R}^d}(t,x)$ to
\eqref{WE} with initial data $(u_0=\delta_{x_{0}}, u_1=0)$ has an explicit representation formula 
 \[ u_{\mathbb{R}^d}(t,x)=\frac{1}{(2\pi)^d}\int
\cos(t|\xi|)e^{i(x-x_{0})\xi}d\xi
 \]
and by usual stationary phase methods one gets dispersion:
\begin{equation}\label{disprd}
\|\chi(hD_t)u_{\mathbb{R}^d}(t,.)\|_{L^{\infty}(\mathbb{R}^d)}\leq
C(d) h^{-d}\min\{1, (h/t)^{\frac{d-1}{2}}\}.
\end{equation}
Interpolation between  \eqref{disprd} and energy estimates, together with a duality argument, routinely provides
\eqref{stricrd}. On any boundary less Riemannian manifold $(\Omega,g)$ one may follow the same path, replacing the exact formula by a parametrix (which may be constructed locally within a small ball, thanks to finite speed of propagation.)

On a manifold with boundary, the geometry of light rays becomes much more complicated, and one may no longer think that one is slightly bending flat trajectories. There may be gliding rays (along a convex boundary) or grazing rays (tangential to a convex obstacle) or combinations of both. Strichartz estimates outside a strictly convex obstacles were obtained in \cite{smso95} and turned out to be similar to the free case (see \cite{ildispext} for the more complicated case of the dispersion). Strichartz estimates with losses were obtained later on general domains,\cite{blsmso08}, using short time parametrices constructions from \cite{smso06}, which in turn were inspired by works on low regularity metrics \cite{tat02}.

In our work \cite{Annals}, a parametrix for the wave equation inside a model of strictly convex domain was constructed that provided optimal decay estimates, uniformly with respect to the distance of the source to the boundary, over a time length of constant size. This involves dealing with an arbitrarily large number of caustics and retain control of their order. Our dispersion estimate is optimal and immediately yields by the usual argument Strichartz estimates with a range of pairs $(q,r)$ such that
\begin{equation}\label{adm-1/4} 
\frac 1q\leq \left(\frac{(d-1)} 2 -\frac 1 4\right)\left(\frac{1}{2}-\frac{1}{r}\right),\quad  q>2\,
\end{equation}
where, informally, the new $1/4$ factor, when compared to \eqref{adm}, is related to the $1/4$ loss in the dispersion estimate from \cite{Annals}, when compared to \eqref{disprd}. On the other hand, earlier works \cite{doi,doi2} proved that Strichartz estimates on strictly convex domains can hold only if, when $r>4$, $(1/q,1/r)$ are below a line connecting the pair $(1/q_{4},1/4)$ (from free space) and $(1/q_{\infty},0)$  such that
\begin{equation}\label{adm-1/4bis} 
\frac 1 {q_{4}}=\frac{(d-1)} 2\left(\frac 1 2 -\frac 1 4\right) \,\,\text{ and }\,\, \frac 1 {q_{\infty}}= \left(\frac{(d-1)} 2 -\frac {1}{12}\right)\frac{1}{2}\,.
\end{equation}
We will restate the exact result later on as we provide a simplified proof for it. Our main purpose in the present work is to improve upon the negative results in dimension $d=2$; improvements on the positive side were obtained in \cite{ILP3}. In particular, for suitable microlocalized solutions we close the gap between known estimates and known counterexamples, providing a
near complete picture when $x+ |\xi/\eta|^{2}\sim h^{1/3}$, $|\eta|\sim h^{-1}$. Before stating our main results, we start by describing our convex model domain. Our Friedlander model is the half-space, for $d\geq 2$,
 $\Omega_d=\{(x,y)| x>0, y\in\mathbb{R}^{d-1}\}$ with the metric $g_F$ inherited from the following Laplace operator,
 $\Delta_F=\partial^2_x+(1+x)\Delta_{\mathbb{R}^{d-1}_y}$. The domain $(\Omega_d,g_F)$ is easily seen to b a strictly convex set, as a first order approximation of the unit disk $D(0,1)$ in polar coordinates $(r,\theta)$: set $r=1-x/2$, $\theta=y$.

 
We start by stating our results for $d=2$ and later provide the general statement in higher dimensions, using the same reduction as \cite{doi2} to take advantage of the 2D setting.
\begin{thm}
  \label{thm-1}
 Strichartz estimates \eqref{stricrd} may hold true on the domain $(\Omega_{2},g_{F})$ only if possible pairs $(q,r)$ are 
such that
\begin{equation}\label{adm-1/12} 
\frac 1q\leq \left(\frac{1} 2 -\frac 1 {10}\right)\left(\frac{1}{2}-\frac{1}{r}\right)\,.
\end{equation}
In particular, for $r=+\infty$, we have $q\geq 5$.

\end{thm}
\begin{rmq} Theorem \ref{thm-1} improves on the results from
  \cite{doi}: the range of admissible pairs is further restricted as $1/12$ is replaced by $1/10$ in the admissibility condition. Moreover, we no longer have a restricted range of $r$, unlike \cite{doi}.
\end{rmq}
\begin{rmq}
  In \cite{ILP3}, we obtained positive results at the endpoint $r=+\infty$ for $q\geq 36/7$. Moreover, whenever the data is moreover restricted to $x+|\xi/\eta|^{2}\sim h^{1/3}$, then Strichartz estimates hold for $q>5$. Hence in this region of phase space Theorem \ref{thm-1} is optimal except for the endpoint $q=5$.
\end{rmq}
Counterexamples in \cite{doi} were constructed by carefully
propagating a cusp starting in a suitable position around $a\sim
h^{1/2}$. Here we start with a suitably smoothed out cusp, which may be seen as a wave packet around $a\sim h^{1/3}$
and let it propagate, estimating the resulting solution with the
parametrix and proving it saturates the bound with a set of exponents satisfying \eqref{adm-1/12}. Our special solution may be seen as a sum of consecutive wave reflections, and at any given point in space-time we only see essentially one of these waves. Each wave has its peak around a specific location related to the number of reflections, and we can estimate the area (in $(x,y)$) where the amplitude of the wave remains close to its peak value, allowing to lower bound any of its physical Lebesgue norms. The time norm is then estimated taking advantage of the separation between any two different wave reflections.

From the 2D construction, we can easily follow the strategy from \cite{doi2}, and construct a good approximate $d-$dimensional wave by tensor product: retain our 2D wave in a given spatial tangential direction and multiply by a Gaussian of width $h^{1/2}$ in all other tangential directions. Such a wave packet will then provide a special solution that saturate some $d-$dimensional estimates. However, it turns out that we do not recover better counterexamples than the ones from \cite{doi2}: in fact, we recover the exact same set of exponents, albeit for a slightly different class of examples. As such we state the result and its proof for the sake of completeness as well as providing a much simpler argument than both \cite{doi,doi2}.
\begin{thm}
  \label{thm-2}
For $d=3,4,5$, Strichartz estimates \eqref{stricrd} may hold true on the domain $(\Omega_{d},g_{F})$ only if possible pairs $(q,r)$ are 
such that 
\begin{equation}\label{adm-1/10d} 
\frac 1q\leq \left(\frac{d-1} 2 -\frac {1-4/r}{12-24/r}\right)\left(\frac{1}{2}-\frac{1}{r}\right)\,.
\end{equation}
\end{thm}
Note that we get the same dimension restriction out of necessity: we have an additional condition $r\geq 4$ that restricts meaningful ranges to lower dimensions.

\section{The half-wave propagator: spectral analysis and parametrix construction}


\subsection{Digression on Airy functions}
Before dealing with the Friedlander model, we recall a few notations, where $Ai$ denotes the standard Airy function (see e.g. \cite{AFbook} for well-known properties of the Airy function): define
\begin{equation}
  \label{eq:Apm}
  A_\pm(z)=e^{\mp i\pi/3} Ai(e^{\mp i\pi/3} z)\,,\,\,\text{ for } \,
  z\in \mathbb{C}\,,
\end{equation}
then one checks that $Ai(-z)=A_+(z)+A_-(z)$. The following lemma is proved in the Appendix:
\begin{lemma}
  \label{lemL}
Define 
\begin{equation}
  \label{eq:Lom}
  L(\omega)=\pi+i\log \frac{A_-(\omega)}{A_+(\omega)} \,,\,\,\text{
    for }\, \omega \in \R\,,
\end{equation}
then $L$ is real analytic and strictly increasing. We also
have
\begin{equation}
  \label{eq:propL}
  L(0)=\pi/3\,,\,\,\lim_{\omega\rightarrow -\infty} L(\omega)=0\,,\,\,
  L(\omega)=\frac 4 3 \omega^{\frac 3 2}+\frac{\pi}{2}-B(\omega^{\frac 3
    2})\,,\,\,\text{ for } \,\omega\geq 1\,,
\end{equation}
with 
\begin{equation}
  \label{eq:B}
  B(u)\simeq \sum_{k=1}^\infty b_k u^{-k}\,,\,\, b_k\in\R\,,\,\,
  b_1> 0\,.
\end{equation}
Finally, one may check that
\begin{equation}
  \label{eq:propL2}
  Ai(-\omega_k)=0 \iff L(\omega_k)=2\pi k \text{ and }
  L'(\omega_k)=2\pi \int_0^\infty Ai^2(x-\omega_k) \,dx\,,
\end{equation}
where here and thereafter, $\{-\omega_k\}_{k\geq 1}$ denote the zeros of the Airy function in decreasing order.
\end{lemma}


\subsection{Spectral analysis of the Friedlander model}

Recall $\Omega_2=\{(x,y)\in\mathbb{R}^2|, x>0,y\in\mathbb{R}\}$ and
$\Delta_F=\partial^{2}_{x}+(1+x)\partial^{2}_{y}$ with Dirichlet boundary condition. After a Fourier transform in the $y$ variable, the operator $-\Delta_{F}$ is now
$-\partial^2_x+(1+x)\theta^2$. For  $\theta\neq 0$, this operator is a positive self-adjoint operator 
on $L^2(\mathbb{R}_+)$, with compact resolvent and we have explicit eigenfunctions and eigenvalues:

\begin{lemma}\label{lemorthog}
There exist orthonormal eigenfunctions $\{e_k(x,\theta)\}_{k\geq 0}$ with their corresponding eigenvalues $\lambda_k(\theta)=|\theta|^2+\omega_k|\theta|^{4/3}$, which form an Hilbert basis of $L^{2}(\mathbb{R}_{+})$.  These eigenfunctions have an explicit form
\begin{equation}\label{eig_k}
 e_k(x,\theta)=\frac{\sqrt{2\pi}|\theta|^{1/3}}{\sqrt{L'(\omega_k)}}
Ai\Big(|\theta|^{2/3}x-\omega_k\Big),
\end{equation}
where $L'(\omega_k)$ is given by \eqref{eq:propL2}, which yields
$\|e_k(.,\theta)\|_{L^2(\mathbb{R}_+)}=1$.
\end{lemma}
The proof of Lemma \ref{lemorthog} is postponed to the Appendix.


In a classical way, for $a>0$, the Dirac distribution $\delta_{x=a}$ on $\mathbb{R}_+$ may be decomposed as
\[
 \delta_{x=a}=\sum_{k\geq 1} e_k(x,\theta)e_k(a,\theta).
\]
Then if we consider a data at time $t=s$ such that
$u_0(x,y)=\psi(hD_y)\delta_{x=a,y=b}$, where $h\in (0,1]$ is a small parameter and
 $\psi\in C^{\infty}_0([\frac 12,2])$, we can write the (localized in $\theta$) Green function associated to the half-wave operator on $\Omega_{2}$ as
 \begin{equation}
\label{greenfct} G^{\pm}_{h}((x,y,t),(a,b,s))=\sum_{k\geq 1}
\int_{\mathbb{R}}e^{\pm i(t-s)\sqrt{\lambda_k(\theta)}}
e^{i(y-b)\theta} 
 \psi(h\theta)e_k(x,\theta)e_k(a,\theta)d\theta\,.
 \end{equation}
\subsection{Airy-Poisson formula}
We briefly recall a variant of the Poisson summation formula, introduced to deal with a parametrix construction for the general case of a generic strictly convex domain in \cite{ILLP} and used in \cite{ILP3} to improve Strichartz estimates in the model case. It will turn out to be crucial to analyze the spectral sum defining $G^{\pm}_{h}$ and map it to a sum over reflections of waves.
\begin{lemma}
  \label{AiryPoisson}
  In $\mathcal{D}'(\R_\omega)$, one has
  \begin{equation}
    \label{eq:AiryPoisson}
        \sum_{N\in \Z} e^{-i NL(\omega)}= 2\pi \sum_{k\in \N^*} \frac 1
    {L'(\omega_k)} \delta(\omega-\omega_k)\,.
  \end{equation}
In other words, for $\phi(\omega)\in C_{0}^{\infty}$, 
  \begin{equation}
    \label{eq:AiryPoissonBis}
        \sum_{N\in \Z} \int e^{-i NL(\omega)} \phi(\omega)\,d\omega = 2\pi \sum_{k\in \N^*} \frac 1
    {L'(\omega_k)} \phi(\omega_k)\,.
  \end{equation}

\end{lemma}
The Lemma is easily proved using the usual Poisson summation formula
followed by the change of variable $x=L(\omega)$ and we provide details in the Appendix.

\section{Counterexamples}
In this section, $a$ is a parameter to be optimized later on. Recall that a (2D) Strichartz estimate is
\begin{equation}
  \label{eq:31}
  \| u\|_{L^{q}(0,T;L^{r}(\Omega))}\lesssim h^{-\beta} \|u_{0}\|_{L^{2}(\Omega)}\,,
\end{equation}
where $\beta=d(1/2-1/r)-1/q$ with $d=2$ (scaling condition). We also define $\alpha$ to be such that $1/q=\alpha(1/2-1/r)$ and recall that in free space, $\alpha=(d-1)/2=1/2$.
\subsection{Rescaled variables}
Let $a$ be small enough, such that $ h^{2/3}\ll a \ll 1$. From our knowledge from the parametrix construction in \cite{Annals} (see also \cite{ILP3}), where the source point is $(x=a,y=0)$, we rescale as follows:  set $\lambda=a^{3/2}/h$ and let $M_a=a^{-1/2}$,
\begin{equation}
  \label{eq:38}
  t=a^{1/2} T\,,\,\, x=aX\,,\,\,y=-t\sqrt{1+a}+a^{3/2}Y\,,\,\, U(T,X,Y)=u(t,x,y)\,.
\end{equation}
If $F(X,Y)=f(aX,a^{3/2}Y-T\sqrt a \sqrt{1+a})$, then
\begin{equation}
  \label{eq:39}
  \| F(X,Y)\|_{L^{r}_{X>0,Y}}=a^{-5/(2r)} \|f\|_{L^{r}_{x>0,y}}
\end{equation}
and 
\begin{equation}
  \label{eq:40}
   \| U(T, X,Y)\|_{L^{q}(0,M_a;L^{r})}=a^{-1/(2q)-5/(2r)} \|u\|_{L^{q}(0,1;L^{r})}\,.
\end{equation}
Since $h=M_a^{-3} \lambda^{-1}$, in rescaled variables the estimate \eqref{eq:31} becomes
\begin{equation}
  \label{eq:41}
  M_a^{-1/q-5/r} \| U\|_{L^{q}(0,M_a;L^{r})}\lesssim (\lambda M_a^{3})^{1-1/q-2/r} a^{5/4} \| U_{0}\|_{L^{2}}
\end{equation}
hence we are reduced to the following estimates 
\begin{equation}
  \label{eq:41bis}
 \| U\|_{L^{q}(0,M_a;L^{r})}\lesssim \lambda^{1-1/q-2/r} M_a^{1/2-1/r-2/q} \| U_{0}\|_{L^{2}}\,.
\end{equation}

\subsection{Choice of an initial data}
Let us consider our model equation,
\begin{equation}
  \label{eq:42}
(  \partial^{2}_{t} -(\partial^{2}_{x}+(1+x)\partial^{2}_{y})) u(t,x,y)=0 \text{ on } x\geq 0, y\in\mathbb{R}
\end{equation}
with Dirichlet boundary condition $u_{|x=0}=0$, data $u_{|t=0}=u_{0}(x,y)$ and $WF(u)\subset \{\tau>0\}$, where $\tau$ is the Fourier variable associated to time. We will seek solutions $u$ under the following form
\begin{equation}
  \label{eq:43}
  u(t,x,y)=\frac 1h \int e^{i\frac \eta h y} v(t,x,\eta/h) \psi(\eta)\,d\eta\,,
\end{equation}
where $\psi\in C^{\infty}_{0}$, $\psi=1$ for $1\leq \eta\leq 2$ and $\psi=0$ outside $[\frac 12,2]$. As such, if we set $\hbar=h/\eta$ and  $v_{\hbar}(t,x)=v(t,x,1/\hbar)$, it follows that $v_{\hbar}$ is a solution to
 \begin{equation}
   \label{eq:44}
(   \hbar^{2}\partial^{2}_{t}+(-\hbar^{2}\partial^{2}_{x}+(1+x)) v_{\hbar}(t,x)=0 \text{ for } x\geq 0,
 \end{equation}
$v_{\hbar|x=0}=0$, $WF(v_{\hbar})\subset \{\tau>0\}$ and $v_{\hbar|t=0}=v_{0}(x,a,1/\hbar)$ for some $v_0$ to be suitably chosen. 
Recalling from Lemma \ref{lemorthog} that the eigenmodes are $e_{k}(x,\hbar^{-1})$ and using \eqref{eig_k}, we decompose $v_{0}$ over the eigenmodes and write
\begin{multline}
  \label{eq:46}
  v_{\hbar}(t,x)=\sum_{k\geq 1} e^{i\frac t \hbar (1+\omega_{k} \hbar^{2/3})^{\frac 1 2}} e_{k} (x,\hbar^{-1}) \int_{{z}>0} e_{k}({z},\hbar^{-1}) v_{0}({z},a,1/\hbar) \,d{z}\,\\
  = \sum_{k\geq 1} e^{i\frac t \hbar (1+\omega_{k} \hbar^{2/3})^{\frac 1 2}}\frac{\hbar^{-2/3} }{L'(\omega_{k})} Ai(\hbar^{-2/3} x-\omega_{k}) \int_{{z}>0} Ai(\hbar^{-2/3} {z}-\omega_{k}) v_{0}({z},a,1/\hbar) \,d{z}\,.
\end{multline}
Alternatively we may use the Green function formula \eqref{greenfct} and apply it to our datum $v_{0}$. Using the Airy-Poisson formula \eqref{eq:AiryPoissonBis}, we transform the sum of eigenmodes (over $k$) into a sum over $N\in \Z$; its summands will be later seen to be waves corresponding to the number of reflections on the boundary, indexed by $N$ :
\begin{multline}
  v(t,x,\hbar^{-1})= \frac 1 {2\pi} \sum_{N\in \Z} \int_{\R}\int_{{z}>0} e^{-i NL(\omega)} \hbar^{-2/3} e^{i\frac t \hbar (1+\omega \hbar^{2/3})^{\frac 1 2}} \chi_{1}(\hbar^{2/3}\omega)\\
 Ai (\hbar^{-2/3} x-\omega) Ai(h^{-2/3}{z}-\omega) v_{0}({z},a,1/\hbar) \,d{z} d\omega\,.
\end{multline}
Here $\chi_{1}(\zeta)=1$ for $\zeta>0$ and $\chi_{1}(\zeta)=0$ for $\zeta<-1$. We may introduce such a cut-off function as the sequence $(-\omega_{k})_{k}$ is positive. Recall that
\begin{equation}
  \label{eq:47}
  Ai(\hbar^{-2/3} x-\omega)=\frac 1 {2\pi \hbar^{1/3} } \int e^{\frac i \hbar (\frac{\sigma^{3}}{3}+\sigma(x-\hbar^{2/3}\omega))} \,d\sigma\,.
\end{equation}
If we rescale with $\zeta=\hbar^{2/3} \omega$, we get
\begin{equation}
  \label{eq:48}
    v(t,x,\hbar^{-1})= \frac 1 {(2\pi)^{3}\hbar^{2}} \sum_{N\in \Z} \int_{\R}\int_{{z}>0}\int_{\R^{2}} e^{\frac i \hbar  \tilde \Phi_{N}}   \chi_{1}(\zeta) v_{0}({z},a,1/\hbar)  \, ds d\sigma d{z} d\omega\,.
\end{equation}
where
\begin{equation}
  \label{eq:49}
    \tilde \Phi_{N}=\frac{\sigma^{3}} 3+\sigma(x-\zeta)+\frac {s^{3}} 3+s({z}-\zeta)-N\hbar L(\hbar^{-2/3} \zeta)+t \sqrt{1+\zeta}
\end{equation}
and therefore, with $\Phi_{N}=\tilde\Phi_{N}+y$, we find
\begin{equation}
  \label{eq:48bis}
  u(t,x,y)= \frac 1 {(2\pi)^{3}h} \sum_{N\in \Z} \int_{\R^{2}}\int_{{z}>0}\int_{\R^{2}} e^{i \frac \eta \hbar  \Phi_{N}}  \chi_{1}(\zeta) %\\{}  \times
  (\eta/h)^{2}{\psi(\eta)} v_{0}({z},a,\eta/h) \, ds d\sigma d{z} d\zeta d\eta\,.
\end{equation}
Let us rescale now like we did in \eqref{eq:38}, with moreover
\begin{equation}
  \label{eq:50}
  \zeta=a E\,,\,\,s=a^{1/2}S\,,\,\, \sigma =a^{1/2} \Sigma\,,\,\, {z}=a {Z},
\end{equation}
then $u(t,x,y)$ becomes $U(T,X,Y)$ where, for $\lambda=a^{3/2}/h$ as before, we have
\begin{equation}
  \label{eq:51}
  U(T,X,Y)=\frac 1 {(2\pi)^{3}h} \lambda^{2} \sum_{N\in \Z} \int e^{i\lambda \eta (Y+\Psi_{N})}V_{0}({Z},\lambda \eta) \chi_{1}(a E)%\\ {}\times
  \psi(\eta) \eta^{2} \, dS d\Sigma dE d{Z}d\eta.
\end{equation}
Here the phase function is given by
\begin{equation}
  \label{eq:52}
  \Psi_{N}=\frac{\Sigma^{3}}3 +\Sigma(X-E)+\frac {S^{3}} 3+S({Z}-E)-\frac N{\lambda \eta} L((\lambda \eta)^{\frac 2 3} E)+\frac { T(E-1)}{\sqrt {1+a E}+\sqrt{1+a}}\,.
\end{equation}
The last term comes from the time propagator and takes into account the change of variable in $y$ that includes a time translation. Our datum $v_0(z,a,\eta/h)$ is now $V_0({Z},\eta\lambda)$, which we set to be
\begin{equation}
  \label{eq:53}
  V_{0}({Z},\lambda \eta)=\chi_{2}({Z}) \int e^{i\lambda \eta(({Z}-1) s+\frac {s^{3}} 3 +\frac i 2 \frac {s^{2}} M)} \,ds
\end{equation}
where $\chi_{2}=1$ near ${Z}=1$ and zero away from it, which implies readily that $V_{0}\Big|_{{Z}=0}=0$. Here $M$ is large and will be chosen in the end of this section, depending on $a,h$. Defined in this way, $V_{0}$ is (microlocally) concentrated around $\{{Z}=1\}$ and the Fourier direction $\{ \Xi=0\}$ and we may ``forget'' about the cut-off $\chi_{2}$: indeed, we may explicitly compute the integral defining $V_0$ as follows, with $\tilde \lambda=\lambda \eta$ and taking $\tau=\tilde\lambda^{1/3}s$ :
  \begin{align*}
\int e^{i\lambda \eta(({Z}-1) s+\frac {s^{3}} 3 +\frac i 2 \frac {s^{2}} M)} \,ds& =  \frac 1 {\tilde \lambda^{1/3}} \int e^{i(\frac {\tau^{3}} 3+ \tilde \lambda^{2/3}\tau({Z} -1)+ \frac i {2M} \tilde \lambda^{1/3} \tau^{2}) }\,d\tau\\
  &  =  \frac 1 {\tilde \lambda^{1/3}} \int e^{i\Big(\frac 13  (\tau+ \frac i {2M} \tilde \lambda^{1/3})^{ 3}+ \tilde \lambda^{2/3}\tau({Z} -1+\frac 1 {4M^{2}}) +\frac i 3 \tilde \lambda \frac 1 {(2M)^{3}}\Big)} \,d\tau\\
 & =  \frac 1 {\tilde \lambda^{1/3}}  e^{ \frac{\tilde \lambda}{2M} ({Z}-1+\frac 2 3 \frac 1 {4M^{2}})} 2\pi Ai\Big(\tilde \lambda^{2/3}({Z}-1+\frac 1 {4M^{2}})\Big)\,.
  \end{align*}
For ${Z}=0$, we get
\begin{equation}
  \label{eq:54}
\int e^{i\lambda \eta(({Z}-1) s+\frac {s^{3}} 3 +\frac i 2 \frac {s^{2}} M)} \,ds\Big|_{{Z}=0}=\frac 1 {\tilde \lambda^{1/3}}  e^{ -\frac{\tilde \lambda}{2M} (1-\frac 2 3 \frac 1 {4M^{2}})} 2\pi Ai\Big(\tilde \lambda^{2/3}(-1+\frac 1 {4M^{2}})\Big)\,,
\end{equation}
which is $O(\tilde \lambda^{-\infty})$ provided that $1-1/(6M^{2})>0$ and $\tilde \lambda /(2M) \geq \tilde \lambda^{\varepsilon}$ for a suitable (small) $\varepsilon>0$. This is always true if we chose $1 \ll M\ll \lambda$. This will be our first condition on $M$. With such a choice of $M$, we can therefore forget the cut-off $\chi_2$ in the definition of $V_0$ since now
$$
V_0|_{{Z}\leq 0}=\int e^{i\lambda \eta(({Z}-1) s+\frac {s^{3}} 3 +\frac i 2 \frac {s^{2}} M)} \,ds\Big|_{{Z}\leq 0}=O(\tilde \lambda^{-\infty})\,.
$$
From these considerations, it follows that we can compute the Fourier transform of $V_0$ defined \eqref{eq:53} for some $1\ll M\ll \lambda $, up to $O(\tilde \lambda^{-\infty})$ terms, as follows 
\begin{multline}
  \label{eq:55}
  \hat V_{0} (\tilde\lambda \xi,\tilde\lambda)= \int_{\mathbb{R}}e^{-i\tilde\lambda \xi {Z} }V_0({Z},\tilde\lambda)d{Z}\\
  =\int_{\mathbb{R}}e^{-i\tilde\lambda \xi {Z} }\chi_2({Z}) \int e^{i\tilde\lambda (({Z}-1) s+\frac {s^{3}} 3 +\frac i 2 \frac {s^{2}} M)} \,dsd{Z}\\
  =\int \Big(\int_{\mathbb{R}}e^{i\tilde\lambda (s-\xi){Z}}d{Z}\Big) e^{i\tilde\lambda (\frac {s^{3}} 3 -s +\frac i 2 \frac {s^{2}} M)} \,ds\\
  =\int \frac{1}{\tilde\lambda}\delta(s-\xi)e^{i\tilde\lambda (\frac {s^{3}} 3 -s +\frac i 2 \frac {s^{2}} M)} \,ds  =\frac 1 {\tilde\lambda} e^{i\tilde\lambda (\xi^{3}/3-\xi+\frac i 2 \xi^{2}/M)}.
\end{multline}
\subsection{$L^2$ norm of the initial data}
Since
\[
u(t,x,y)|_{t=0}=\frac 1h\int e^{i\frac{\eta}{h}y}v_0(x,a,\eta/h)\psi(\eta)d\eta,
\]
where, for $x=aX$, we set $v_0(x,a,\eta/h)=V_0(X,\lambda\eta)$ with $V_{0}$ defined by \eqref{eq:53}, it follows that 
\[
U_0(X,Y):=u(0,x,y)=\frac 1h \int e^{i\lambda \eta Y}V_0(X,\lambda \eta)\psi(\eta)d\eta.
\]
We now compute the Fourier transform of $U_0$:
\begin{align}\label{hatU0}
\hat{U_0}(\zeta,\lambda\eta) & =\int e^{-i\zeta X} e^{-i\lambda\eta Y}\frac 1h \int e^{i\lambda \tilde \eta Y}V_0(X,\lambda \tilde  \eta)\psi(\tilde\eta)d\tilde\eta dX dY\\
                             & =\frac 1h\int \int e^{i\lambda (\tilde\eta-\eta)Y}dY \psi(\tilde\eta) \hat{V_0}(\zeta,\lambda\tilde\eta)d\tilde\eta\nonumber\\
                 &  =\frac{1}{h\lambda}\int \delta_{\tilde\eta=\eta}\psi(\tilde\eta)\hat{V_0}(\zeta,\lambda\tilde\eta)d\tilde\eta\nonumber\\
 & =\frac{1}{h\lambda}\psi(\eta)\hat{V_0}(\zeta,\lambda\eta).
\end{align}
We are now in a position to estimate the $L^2$ norm of $U_0$, using the explicit form of the $\hat V_{0}$ we already obtained:
\begin{align}
  \label{eq:56}
  \|U_{0}\|^2_{L^{2}_{X\geq 0,Y}} & \simeq  \|U_{0}\|^2_{L^{2}_{X\in\mathbb{R},Y}}= \|\hat{U_{0}}\|^2_{L^{2}_{\zeta,\theta}}=\int |\hat{U_0}|^2(\zeta,\theta)d\zeta d\theta\\
% & =\lambda\int |\hat{U_0}|^2(\zeta,\lambda\eta)d\zeta d\eta\nonumber\\
 &  =\lambda \int \frac{1}{h^2\lambda^2}\psi^2(\eta)|\hat{V_0}|^2(\zeta,\lambda\eta)d\zeta d\eta\nonumber\\
 &  =\frac{1}{h^2}\int \eta\psi^2(\eta)|\hat{V_0}|^2(\lambda\eta\xi,\lambda\eta)d\xi d\eta\\
                                  &   =\frac{1}{h^2\lambda^2}\int \eta^{-1}\psi^2(\eta)\Big|e^{i\eta\lambda(\xi^{3}/3-\xi+\frac i 2 \xi^{2}/M)}\Big|^2d\xi d\eta\nonumber\\
&                      =  \frac{1}{h^2\lambda^2}\int \eta^{-1}\psi^2(\eta) e^{-\lambda \eta\xi^2/M} d\xi d\eta\nonumber\\
&  \simeq h^{-2}\lambda^{-5/2} M^{1/2}\,,
\end{align}
where we have used \eqref{hatU0} to get the second line in \eqref{eq:56} and \eqref{eq:55} for the fourth line. This yields $ \|U_{0}\|_{L^{2}_{X\geq 0,Y}}\simeq h^{-1}\lambda^{-5/4}M^{1/4}$.

\subsection{Computing the parametrix $U$} 
In the remaining of this section we show that, for some suitable $M$, the Strichartz estimates \eqref{eq:41bis} hold but with a loss in the parameter $\alpha$: $\alpha\leq 1/2-1/10$. For that, we start by computing the $L^{\infty}$ norm of $U$, followed by its $L^q(0,1;L^{\infty})$ norm ; next we show that if \eqref{eq:41bis} holds for $r=\infty$, this forces $q\geq 5$, which is equivalent to the aforementioned loss on $\alpha$. This provide our counterexample for the endpoint Strichartz estimate $(q,+\infty)$. We then compute the $L^{r}$ norm of $U$ to recover other exponents, and this is ultimately useful in higher dimensions as well.


Since the phases $\Psi_N$ in the sum defining $U$ (in \eqref{eq:51}) are all linear in ${Z}$, replacing $V_0$ given by \eqref{eq:53} in the expression in \eqref{eq:51} of $U$ and integrating over ${Z}\in \R$ (using that its contribution for ${Z}\leq 0$ is asymptotically small), yields
$$
\int e^{i\lambda \eta {Z}(s+S)}\,d{Z}= \frac{2\pi}{\lambda \eta} \delta(s+S)\,,
$$
therefore we get
\begin{equation}
  \label{eq:57}
  U(T,X,Y)=\frac{\lambda}{(2\pi)^{2}h}  \sum_{N\in \Z} \int e^{i\lambda \eta(Y+\varphi_{N})} \chi_{1}(aE) \psi(\eta) \eta \,d\eta ds d\Sigma dE
\end{equation}
where $\varphi_{N}$ is the (complex) phase
\begin{equation}
  \label{eq:58}
 \varphi_{N}= T \frac{(E-1)}{ \sqrt{1+aE}+\sqrt{1+a}} - \frac N {\lambda \eta} L((\lambda \eta)^{2/3} E)+s (E-1) + i \frac {s^{2}}{2M} +\frac{\Sigma^{3}} 3+ \Sigma (X-E).
\end{equation}
Notice that
\begin{gather}
  \mathrm{Im} (\varphi_{N})= \frac{s^{2}}{2M}\\
\partial_{s} \varphi_{N} = (E-1)+i \frac s M\\
\partial_{\Sigma} \varphi_{N}=\Sigma^{2}+X-E\\
\partial_{E} \varphi_{N}= T\partial_{E}\left(   \frac{(E-1)}{ \sqrt{1+aE}+\sqrt{1+a}}\right) - \frac N {(\lambda \eta)^{1/3}} L'((\lambda \eta)^{2/3} E)+s-\Sigma\,.
\end{gather}
Therefore, the set $\{\mathrm{Im}(\varphi_{N})=0,\,\, \nabla_{(s,\Sigma,E)}\varphi_{N}=0\}$ coincides with\begin{equation}
  \label{eq:59}
  \{ s=0 \,,\,\, E=1\,,\,\, \Sigma=\frac{T}{2\sqrt {1+a}} -  \frac N {(\lambda \eta)^{1/3}} L'((\lambda \eta)^{2/3} E)\,,\,\,X=1-\Sigma^{2}\}.
\end{equation}
In the $(T,X)$ plane, this is the trajectory moving to the right from $X=1$, $\Sigma=0$. We introduce the following notations : let $\varepsilon_m>0$, $m\in\{0,1,2\}$ be small and set
\[
I_{J}=4J \sqrt{1+a}+(-2\varepsilon_{0},2\varepsilon_{0}),
\]
\[
\DomainXYT_{J}=\{T\in I_{J}\,,\,\, |X-1|\leq \varepsilon_{1}\,,\,\, |Y|\leq \varepsilon_{2}\,\}.
\]
From now on we will focus on $U$ restricted to a set $\DomainXYT_J$ on which we obtain a lower bound of its $L^{\infty}$ norm. We first need the following result, which states that, if for a given $J$ we consider only points $(T,X,Y)\in \DomainXYT_J$, then in the sum \eqref{eq:57} defining $U(T,X,Y)$ indexed over the number of reflections $N$ there is only one single integral that provides a non-trivial contribution, corresponding to $N=J$.
\begin{prop}
\label{propCE}
For all $n \in \N^{*}$, there exists $C_{n}$ such that for all $0\leq J\lesssim M_a$, for all $1\ll M\ll \lambda$ and for all $(T,X,Y)\in \DomainXYT_{J}$, the following holds
\begin{equation}
  \label{eq:60}
  |U(T,X,Y)-\frac 1 {(2\pi)^{2}h} \lambda \int e^{i\lambda \eta(Y+\varphi_{J})} \chi_{1}(aE) \psi(\eta)\, d\eta d\Sigma d s d E | \leq C_{n} \lambda^{-n}\,.
\end{equation}
\end{prop}
\begin{proof}
Let $U$ be given by \eqref{eq:57} : we start by eliminating the $s$ variable in the integrals with complex phase functions $\varphi_N$ defined in \eqref{eq:58}. We have
\begin{equation}
  \label{eq:61}
  \int e^{i\lambda \eta  (s(E-1)+i\frac {s^{2}}{2M})} 
\,ds= \sqrt{\frac{2\pi}{\lambda\eta}} \sqrt M e^{-\frac{\lambda\eta M(E-1)^{2}}{2}} 
\end{equation}
and therefore \eqref{eq:57} becomes
\begin{equation}
  \label{eq:62}
  U(T,X,Y)=\frac{\lambda^{1/2}M^{1/2}}{ (2\pi)^{3/2}h}\sum_{N\in\mathbb{Z}}\int e^{i\lambda \eta(Y+\tilde\varphi_{N})} \chi_{1}(aE)\psi(\eta)\eta^{1/2} \,d\Sigma dE d\eta
\end{equation}
with phase given by
\begin{equation}
 \label{eq:63}
\tilde\varphi_{N}   = T \frac{(E-1)}{ \sqrt{1+aE}+\sqrt{1+a}} - \frac N {\lambda \eta} L((\lambda \eta)^{2/3} E)\\{}+ i \frac {M(E-1)^{2}}{2} +\frac{\Sigma^{3}} 3+ \Sigma (X-E)\,.
  \end{equation} 
Recall that $L(\omega)=\frac 43 \omega^{3/2}-B(\omega^{3/2})$ where $B(u)=\sum_{n\geq 1} b_{n}u^{-n}$, hence 
\[
\frac N {\lambda \eta} L((\lambda \eta)^{2/3} E)=\frac N {\lambda \eta} \Big(\frac 43 \lambda\eta E^{3/2}-B(\lambda\eta E^{3/2})\Big)=\frac 43 NE^{3/2}-\frac N {\lambda \eta}B(\lambda\eta E^{3/2}).
\]

Let $0\leq J\lesssim M_a$ and let $(T,X,Y)\in\DomainXYT_J$: we can write 
\[
T=(4J+2\tilde T)\sqrt{1+a}, \quad X=1+\tilde X,\text{ where } |\tilde T|\leq \varepsilon_0, \quad |\tilde X|\leq \varepsilon_1.
\] 
We also make the change of variables $E=1+(1+a){\tilde E}$, as,from the Gaussian nature of $\tilde\varphi_N$, $E$ has to stay close to $1$.
\begin{rmq}\label{rmqBNB}
Notice that for values $h^{1/2}\lesssim a$, the factor $\exp{(iNB)}$ in our phase does not oscillate anymore: indeed, the phase $\varphi_N$ given in \eqref{eq:58} is stationary in $E$ only when $N\simeq T\simeq \frac{t}{\sqrt{a}}$ and for $E$ near $1$,
\[
NB(\lambda\eta E^{3/2})\simeq \frac{N}{\lambda}\simeq \frac{t}{\sqrt{a}}\times \frac{h}{a^{3/2}}\lesssim \frac{h}{a^2}.
\]
Therefore, when $h^{1/2} \ll a$ we can actually bring the $NB(\cdot)$ term in the symbol rather than leave it in the phase (in order to do explicit computations).
\end{rmq}
In the new variables $\tilde X$ and ${\tilde E}$, the phase functions $\tilde\varphi_N$ read as follows
\begin{multline}
  \label{eq:65}
\tilde\varphi_{N} = \frac{T}{\sqrt{1+a}} \frac{(1+a) {\tilde E}}{ 1+\sqrt{1+a{\tilde E}}} - \frac 4 3 N (1+(1+a){\tilde E})^{3/2} \\{}+\frac{\Sigma^{3}} 3+ \Sigma (\tilde X-(1+a){\tilde E})+\frac i 2 M (1+a)^2 {\tilde E}^{2}\,.
  \end{multline}
The derivatives with respect to ${\tilde E},\Sigma$ are
\[
  \partial_{\Sigma}\tilde\varphi_N=\Sigma^2+\tilde X-(1+a){\tilde E}\,
\]
and
\begin{multline*}
  \partial_{{\tilde E}}\tilde\varphi_N=T\sqrt{1+a}\Big(\frac{1}{1+\sqrt{1+a{\tilde E}}}-\frac{a{\tilde E}}{2\sqrt{1+a{\tilde E}}(1+\sqrt{1+a{\tilde E}})^2}\Big)\\{}-2N(1+a)(1+(1+a){\tilde E})^{1/2}-(1+a)\Sigma+iM(1+a)^2{\tilde E}\,.
\end{multline*}
Obviously the set $\{ \mathrm{Im}(\tilde\varphi_{N})=0\,,\,\, \nabla_{({\tilde E},\Sigma)} \tilde\varphi_{N}=0\}$ is given by
\begin{equation}
  \label{eq:66}
\Big\{\,  {\tilde E}=0\,,\,\, \tilde X+\Sigma^{2}=(1+a){\tilde E}=0\,,\,\, \Sigma=\Big(\frac T {2(\sqrt{1+a})} -2N\Big)\,\Big\},
\end{equation}
and therefore, imposing $|\tilde X|\leq \varepsilon_{1}$ implies $|\Sigma|\leq \varepsilon_{1}^{1/2}$ which yields
\[
|\frac T{2\sqrt{1+a}}-2N|=|2J+\tilde T-2N|\leq \varepsilon_{1}^{1/2}.
\] 
Since $|\tilde T|\leq \varepsilon_{0}$, it follows that for $\varepsilon_{0,1}<\frac 14$, the last inequality forces $N=J$. This proves Proposition \ref{propCE} as for $N\neq J$, we can perform non stationary phase, gaining powers of $\lambda$ as well as powers of $N$ through $\partial_{\tilde E}\tilde \varphi_{N}$ (to insure summability in $N$.)
\end{proof}

Using Proposition \ref{propCE} we may rewrite, for $(T,X,Y)\in \DomainXYT_{J}$,
\begin{equation}
  \label{eq:67}
  U((4J+2\tilde T)\sqrt{1+a},1+\tilde X,Y) = \frac{(1+a)\sqrt {\lambda M}}{(2\pi)^{\frac 32}h}
\int e^{i\lambda\eta (Y+\tilde\psi_{M}+JF)} \, d\Sigma d {\tilde E} d\eta +O(\lambda^{-\infty}),
  \end{equation}
where $\tilde\varphi_J$ was replaced by $\tilde\psi_{M}(\cdot)+JF({\tilde E})$ and where, in the new variables, $\tilde\psi_{M}$ and $F$ are respectively
\[
\tilde\psi_{M} (\tilde T,{\tilde E},\Sigma)=\frac{2\tilde T {\tilde E}(1+a)}{1+\sqrt{1+a{\tilde E}}}+i\frac M2 (1+a)^2 {\tilde E}^2+\frac{\Sigma^3}{3}+\Sigma(\tilde X-(1+a){\tilde E})
\]
and 
\begin{equation}
  \label{eq:68}
F({\tilde E})=\frac{4{\tilde E}(1+a)}{1+\sqrt{1+a{\tilde E}}}-\frac 43 (1+(1+a){\tilde E})^{3/2}\,.
\end{equation}
Since $\mathrm{Im}(\tilde\psi_{M})=0$ only at ${\tilde E}=0$, we expand $F$ near ${\tilde E}=0$,
\[
  F(0)=-\frac 43, \quad F'(0)=0, \quad F''(0)=-(1+a)(1+2a)\,,
\]
hence,
$$
F({\tilde E})= -\frac 43-\frac{{\tilde E}^2}{2}(1+a)(1+2a)+O({\tilde E}^3)\,.  
$$
Our new phase function $\tilde\psi_{M}+JF$ depends on two large parameters: $M$, to be chosen such that $1\ll M\ll\lambda $ and $J$, taking all values from $1$ to $M_a=a^{-1/2}$, depending on the region $\DomainXYT_J$ containing $(T,X,Y)$.

Let us take $J\leq  M_a\lesssim M$ : in the phase $\lambda\eta(\tilde\psi_{M}+JF)$, we consider the large parameter to be $M\lambda$ and, for $\Lambda= M \lambda (1+a)$, we get
\begin{multline}\label{eq:phaUj}
\lambda\eta(\tilde\psi_{M}+JF)=\lambda \eta \Big(\frac{\Sigma^3}{3}+\Sigma \tilde X-\frac 43 J\Big)+
\Lambda \eta\Big[\Big(\frac{2\tilde T}{1+\sqrt{1+a{\tilde E}}}-\Sigma\Big)\frac {\tilde E}M\\
+\frac i2 (1+a){\tilde E}^2-\frac{J}{2M}{\tilde E}^2(1+2a)+O(\frac{J}{M}{\tilde E}^3)\Big].
\end{multline}
\begin{rmq}
One should notice that in the integral defining $U$, we may localize on $|{\tilde E}|\lesssim\Lambda^{-1/2}$ using the imaginary part of the phase; indeed, for larger values of ${\tilde E}$ the phase is exponentially decreasing ; we can then localize near the critical points in $\Sigma$, and $\Sigma^{2}=(1+a){\tilde E}-X$ hence $\Sigma$ becomes uniformly bounded and $\frac 1M\Big|\frac{2\tilde T}{1+\sqrt{1+a{\tilde E}}}-\Sigma\Big | \in O(1/M)$. 

Notice moreover that, for $J\leq M_a\lesssim M$, the imaginary phase factor $e^{i\Lambda\eta(1+a)O(\frac JM {\tilde E}^3)}$ doesn't oscillate for values $|{\tilde E}|\lesssim \Lambda^{-1/2}$ (i.e. for ${\tilde E}$ such that the contribution of the integral is not exponentially small).
\end{rmq}
\begin{rmq}
Writing, for small ${\tilde E}$, $\frac{2\tilde T}{1+\sqrt{1+a{\tilde E}}}\simeq \tilde T(1-\frac a4 {\tilde E}+O(a^2{\tilde E}^2))$, we obtain the first approximation of the phase with large parameter $\Lambda \eta$ as follows
\[
(\tilde T-\Sigma)\frac {\tilde E}M+\frac i2 \nu_a {\tilde E}^2+O(\frac JM {\tilde E}^3),\quad \nu_a=1+a+i \Big(\frac JM(1+2a)+\frac{a \tilde T}{2M}\Big).
\] 
\end{rmq}
\begin{rmq}
  Notice that we are still carrying a symbol $\chi_{1}(aE)$: we may safely discard it as $E$ is now localized near $E=1$, and therefore the contribution coming from $(1-\chi_{1}(aE))$ are harmless by non stationary phase.
\end{rmq}
We rewrite the integral in ${\tilde E},\Sigma$ in \eqref{eq:67} under the following form
$$
  \int e^{i \lambda\eta (\tilde\psi_{M}+JF)} \,  d {\tilde E} d\Sigma =  \int e^{i \lambda\eta (\frac{\Sigma^3}{3}+\Sigma \tilde X-\frac 43 J)}  e^{i \Lambda \eta  \left((\tilde T-\Sigma) \frac {\tilde E} M + \frac i2 \nu_a {\tilde E}^{2}+O(\frac J M {\tilde E}^{3})\right)} \,  d {\tilde E} d\Sigma  \,
$$
and apply the stationary phase in ${\tilde E}$ with complex function $(\tilde T-\Sigma) \frac {\tilde E} M + \frac i2 \nu_a {\tilde E}^{2}+O(\frac J M {\tilde E}^{3})$ and large parameter $\Lambda\eta$. Since the absolute value of the second derivative equals $|1+i\frac JM+O(a)|+O(\frac JM {\tilde E})\simeq \sqrt{1+\frac{J^2}{M^2}}+O(\frac JM\times \Lambda^{-1/2})\simeq 1$ for $J\leq M_a\lesssim M$, the stationary phase yields
\begin{equation}
  \label{eq:70}
  \int e^{i \lambda\eta (\tilde\psi_{M}+JF)} \,  d {\tilde E} d\Sigma =\int e^{i \lambda\eta (\frac{\Sigma^3}{3}+\Sigma \tilde X-\frac 43 J)}\sqrt{\frac{2\pi}{\nu_a \Lambda\eta}} e^{ \Lambda\eta  \frac{\nu_a}{2}( {\tilde E}^{2}_{c}+O({\tilde E}_{c}^{3}))}\,d\Sigma 
\end{equation}
where the critical point is of the form ${\tilde E}_{c}=i(\tilde T-\Sigma)/(M\nu_a) (1+O(|\tilde T-\Sigma|/M))$.
Since $\Lambda=\lambda M(1+a)$, we have, at $T=(4J+2\tilde T)\sqrt{1+a}$, $X=1+\tilde X$,
\begin{equation}
  \label{eq:71}
|U(T,X,Y)| \simeq \frac 1h \left| \int e^{i\lambda\eta (Y+\frac i 2 M (1+a) \nu_a P^{2}(1+O(P)) +\frac{\Sigma^{3}}{3}+\Sigma\tilde X)}\psi(\eta) \, d\Sigma d \eta \right|\,,
\end{equation}
with $P=(\tilde T-\Sigma)/(M\nu_a)$. We are now left with the $\Sigma$ integration:
\begin{equation}
  \label{eq:72}
  I(\tilde T,\tilde X)=\int e^{i\lambda \eta G(\Sigma,\tilde T,\tilde X)}\,d\Sigma,
\end{equation}
with phase function
\begin{equation}
  \label{eq:73}
  G(\Sigma, \tilde T, \tilde X)=\frac i 2 \frac{(1+a)}{M\nu_a} (\tilde T-\Sigma)^{2}\Big(1+O(\frac{\tilde T-\Sigma}{M\nu_a})\Big) +\frac{\Sigma^{3}}{3}+\Sigma\tilde X.
\end{equation}
We neglect the $O(P)$ term for the moment, since it can be seen as a perturbation and will not change the main value of the integral in $\Sigma$; consider
\begin{equation}
  \label{eq:72bis}
  I_{0}(\tilde T,\tilde X)=\int e^{i\lambda \eta G_{0}(\Sigma,\tilde T,\tilde X)}\,d\Sigma,
\end{equation}
with phase
\begin{equation}
  \label{eq:73bis}
  G_{0}(\Sigma,\tilde T, \tilde X)=\frac{i(1+a)}{2 M \nu_a} (\tilde T-\Sigma)^{2}+\frac{\Sigma^{3}}{3}+\Sigma\tilde X=\gamma (\tilde T-\Sigma)^2+\frac{\Sigma^3}{3}+\Sigma \tilde X,
\end{equation}
where we have set $\gamma := \frac i2\frac{(1+a)}{M\nu_a}$, e.g. $\gamma=\frac{i}{2(M+i J+ i\frac{a}{(1+a)}(J+\tilde T/2))}$.
Since we are looking for the best possible bounds from below for the $L^{\infty}$ norm of $U$, we want to find values of $\tilde X$ where $I_0$ reaches its maximum value. Write
\[
  G_{0}(\Sigma,\tilde T, \tilde X)=\gamma(\gamma^2+2\gamma \tilde T-\tilde X)+\gamma \tilde T^2-\gamma^3/3+ (\Sigma+\gamma)^3/3-(\Sigma+\gamma)(\gamma^2+2\gamma \tilde T-\tilde X).
\]
The last two terms (the only ones depending on $\Sigma$) may be seen as an Airy phase function, and therefore we have
\begin{multline}\label{Airybound}
I_0(\tilde T,\tilde X)=e^{i\tilde \lambda (\gamma(\gamma^2+2\gamma \tilde T-\tilde X)+\gamma \tilde T^2-\gamma^3/3)}\int e^{i\tilde \lambda ( (\Sigma+\gamma)^3/3-(\Sigma+\gamma)(\gamma^2+2\gamma \tilde T-\tilde X))} d\Sigma\\
=\tilde\lambda^{-1/3}e^{i\tilde \lambda (\gamma(\gamma^2+2\gamma \tilde T-\tilde X)+\gamma \tilde T^2-\gamma^3/3)}Ai(-\tilde\lambda^{2/3}(\gamma^2+2\gamma \tilde T-\tilde X)).
\end{multline}
Recall $Ai(0)=\frac{1}{3^{2/3}\Gamma(2/3)}\simeq 0.355$; moreover there exists a small constant $1\geq c>0$ such that $|Ai(z)|>1/10$ for all $z\in\mathbb{C}$ with $|z|\leq c$ (so such that $z$ belongs to a fixed, complex neighborhood of $0$). We can therefore bound from below the absolute value of the Airy function in \eqref{Airybound} as follows
\begin{equation}\label{boundAiryTX}
|Ai(-\tilde\lambda^{2/3}(\gamma^2+2\gamma \tilde T-\tilde X))|>\frac 1 {10}
\end{equation}
for all $|\tilde\lambda^{2/3}(\gamma^2+2\gamma \tilde T-\tilde X)|\leq c$. Here $\tilde T,\tilde X$ are real, while $\gamma$ takes complex values and satisfies $|\gamma|=|\frac{i}{2(M+iJ+O(a))}|\simeq \frac{1}{M}$ for $J\leq M_a\lesssim M$. Taking $M\geq \frac{4\lambda^{1/3}}{c}$ it follows that \eqref{boundAiryTX} holds true for all $\tilde T\leq  \frac{1}{\lambda^{1/3}}$ and all $|\tilde X|\leq \frac{c}{2\lambda^{2/3}}$.
We now study to the behavior of the exponential factor in \eqref{Airybound}. For $\tilde T,\tilde X$ such that \eqref{boundAiryTX} holds we have
\[
 \lambda |\gamma(\gamma^2+2\gamma \tilde T-\tilde X)|\leq c\lambda^{1/3}|\gamma|\leq 1/4.
\]
For $|\gamma|\simeq \frac{1}{M}\leq \frac{c}{4\lambda^{1/3}}$ and $\tilde T\leq \frac{1}{\lambda^{1/3}}$, the remaining term in the exponential factor of the Airy integral in \eqref{Airybound} can be bounded as follows
\begin{equation}\label{condT}
\lambda |\gamma \tilde T^2-\gamma^3/3|\leq \lambda^{1/3}|\gamma|+\lambda |\gamma|^3/3\leq c/3\leq 1/3.
\end{equation}
\begin{rmq}
Notice that the condition $\lambda^{1/3}\lesssim M$ must hold anyway in order \eqref{boundAiryTX} to hold and the term $|\tilde\lambda\gamma^3|$ in the exponential factor to stay bounded. For such $M$, in order to have $|2\gamma \tilde T-\tilde X|\lesssim \lambda^{-2/3}$ we require $|\tilde T|\lesssim \frac{M}{\lambda^{2/3}}$ and $|\tilde X |\lesssim \lambda^{-2/3}$. On the other hand, the condition $\lambda |\gamma|\tilde T^2\lesssim 1$ gives $|\tilde T|\lesssim \sqrt{M/\lambda}$ and since $M/\lambda^{2/3}\geq \sqrt{M/\lambda}$ for all $M>\lambda^{1/3}$ it follows that, in order to have $|I_0(\tilde T,\tilde X)|\simeq \lambda^{-1/3}$ we must ask \begin{equation}\label{condTX}
\lambda^{1/3}\lesssim M,\quad |\tilde T|\lesssim  \sqrt{M/\lambda}, \quad |\tilde X|\lesssim \lambda^{-2/3}.
\end{equation}
In particular, taking $M\simeq \lambda^{1/3}$ gives $|\tilde T|\lesssim \lambda^{-1/3}$.
\end{rmq}
Let's suppose for the moment that the part of phase function depending on $\Sigma$ in \eqref{eq:71} is $G_0$ (instead of $G$) i.e. such that
\[
|U((4J+2\tilde T)\sqrt{1+a},1+\tilde X,Y)| \simeq \frac 1h \left| \int e^{i\lambda\eta (Y+G_0(\Sigma,\tilde T,\tilde X))}\psi(\eta) \, d\Sigma d \eta \right|\,.
\]
Then, using \eqref{Airybound}, we would immediately get
\begin{multline}\label{U}
|U((4J+2\tilde T)\sqrt{1+a},1+\tilde X,Y)|\simeq \frac{\lambda^{-1/3}}{h}\Big|\int e^{i\lambda\eta (Y+\gamma \tilde T^2-\frac{\gamma^3}{3})}\\\times e^{i\eta \lambda \gamma(\gamma^2+2\gamma \tilde T-\tilde X)}Ai(-(\eta\lambda)^{2/3}(\gamma^2+2\gamma \tilde T-\tilde X))\eta^{-1/3}\psi(\eta)d\eta\Big|,
\end{multline}
and from the discussion above we can see that for $\tilde T,\tilde X$ and $M$ like in \eqref{condTX} the factor
$e^{i\eta \lambda \gamma(\gamma^2+2\gamma \tilde T-\tilde X)}Ai(-(\eta\lambda)^{2/3}(\gamma^2+2\gamma \tilde T-\tilde X))$ can be seen as being part of the symbol since it doesn't oscillate. Since \eqref{condT} also holds it means that the factor $\eta \lambda(\gamma \tilde T^2-\gamma^3/3)$ can also be seen as a symbol.
The (remaining) phase in \eqref{U} is $\eta \lambda Y$ and therefore $Y$ takes values in a ball of center $0$ and radius $\lambda^{-1}$.
\begin{rmq}
We have assumed that $G$ could be replaced by $G_0$ and we have obtained an additional condition on $M$, that is $\lambda^{1/3}\lesssim M$. Recall the form \eqref{eq:73} of $G$ whose (degenerate) critical point satisfies
\[
\Sigma^2+\tilde X-2\gamma (\tilde T-\Sigma)\Big(1+O(\gamma(\tilde T-\Sigma))\Big)=0, \quad \Sigma+\gamma \Big(1+O(\gamma (\tilde T-\Sigma))\Big)=0,
\]
where there is no $\tilde X$ variable in the second equation (since the phase is linear in $\tilde X$). Since $M$ is large and $|\tilde T|\leq \varepsilon_0$, the last equation has an unique solution $\Sigma_c=\Sigma_c(\gamma, \tilde T)\simeq -\gamma \Big(1+O(\gamma\tilde T)\Big)$. It follows that, for $\Sigma$ near $\Sigma_c$, the difference
\[
\lambda \eta(G(\Sigma, \tilde T,\tilde X)-G_0(\Sigma,\tilde T,\tilde X)) =\lambda \eta O(\gamma^2(\tilde T-\Sigma)^3)
\]
can be ignored since it is very small and $e^{i\lambda \eta(G(\Sigma, \tilde T,\tilde X)-G_0(\Sigma,\tilde T,\tilde X)) }$ doesn't oscillate. Since it is more convenient to work with an explicit phase function we do take advantage of this remark and work with $G_0$ instead of $G$, since it became now clear that it represents the only contribution of the phase that matters.
\end{rmq}


\subsection{Choice of $M$. Dispersive and Strichartz norms for $U$} 

Let $M\geq 4\lambda^{1/3}/c$ and $|\tilde T|\leq \sqrt{M/\lambda}$, as well as $|\tilde X|\leq \lambda^{-2/3}$, $|Y|\lesssim \lambda^{-1}$; then we get from \eqref{U}
\begin{equation}
  \label{eq:75}
 h |U((4j+2\tilde T)\sqrt{1+a},1+\tilde X,Y)|\geq c \lambda^{-1/3} \,.
\end{equation}
Recall that in the sum over $N$ defining $U$ there are at most $M_a$ terms : summing over $M_a$ intervals $I_{k}$ of size $\sqrt{M/\lambda}$ gives
\begin{equation}
  \label{eq:76}
  \|h U\|_{L^{q}(0,M_a; L^{\infty}_{X,Y})} \geq c_{0} (M_a\sqrt{M/\lambda})^{1/q}\lambda^{-1/3}\,.
\end{equation}
Asking moreover $M_a\sqrt{M/\lambda}\geq 1$ gives $M_a^2\geq \lambda/M$, $M\geq M_a$.
Recalling \eqref{eq:41bis} and \eqref{eq:56}, we get a condition on $\lambda$:
\begin{equation}
  \label{eq:77}
 (M_a\sqrt{M/\lambda})^{1/q}  \lambda^{-1/3}\lesssim \lambda^{1-1/q} M_{a}^{(1/2-2/q)}\lambda^{-5/4} M^{\frac 14}
\end{equation}
and it turns out that the best choice of parameters in order to maximize $q$ is $a\sim h^{1/3}$,  $M\simeq M_a\simeq \lambda^{1/3}$ which yields, for large $\lambda$, $q\geq 5$, e.g. $\alpha\leq 2/5$ and a loss $\beta\geq 1/10$ at the endpoint $(5,\infty)$.

We now compute the $L^{r}_{X,Y}$ norms for $r<+\infty$ while retaining the chosen values of $a$ and $M$: for $|\tilde T|\lesssim \lambda^{-1/3}$, we get that
\begin{align*}
  \int_{X,Y} | h U(T,X,Y)|^{r}\, dXdY \geq   \int_{|\tilde X|\leq \lambda^{-2/3},|Y|\lesssim \lambda^{-1}} |h U(T,X,Y)|^{r}\, dXdY\\
  \gtrsim  \int_{|\tilde X|\leq \lambda^{-2/3},|Y|\lesssim \lambda^{-1}} \lambda^{-r/3}\, dXdY\\
      \gtrsim  \lambda^{-5/3-r/3}\,.
\end{align*}
Then, we have
\begin{align*}
  \int_{I_{J}} \left(\int_{X,Y} | h U(T,X,Y)|^{r}\, dXdY\right)^{\frac q r} dT  & \geq \int_{|\tilde T|\lesssim \lambda^{-1/3}} \left(\int_{X,Y} |h U(T,X,Y)|^{r}\, dXdY\right)^{\frac q r} dT\,,\\
                                                    &    \gtrsim \lambda^{-1/3} \lambda^{-5q/(3r)-q/3}\,,\\
\sum_{J}\int_{I_{J}} \left(\int_{X,Y} |h U(T,X,Y)|^{r}\, dXdY\right)^{\frac q r} dT            & \gtrsim  M_{a}\lambda^{-1/3} \lambda^{-5q/(3r)-q/3}\,,\\
\int_0^{M_{a}} \left(\int_{X,Y} |h U(T,X,Y)|^{r}\, dXdY \right)^{\frac q r}dT            & \gtrsim \lambda^{-5q/(3r)-q/3}\,,
\end{align*}
and recalling \eqref{eq:41bis}, one has
$$
\lambda^{-\frac 5{3r}-\frac 1 3}\lesssim \lambda^{1-\frac 1 q-\frac 2 r} \lambda^{\frac 1 3 (\frac 1 2 -\frac 1 r -\frac 2 q)} \lambda^{-\frac 5 4} \lambda^{\frac 1 {12}}
$$
which translates into
$$
\frac 5 q +\frac 2 r-1\leq 0\,,
$$
which is our statement \eqref{adm-1/12}. This proves Theorem \ref{thm-1}.\qed

We now consider a different choice of parameters: assume $a\sim h^{1/2-\varepsilon}$ and retain $M\sim M_{a}=h^{-1/4-\varepsilon/2}$. Then $M\sim h^{2\varepsilon}\lambda$, we still have $|\tilde X|\lesssim \lambda^{-2/3}$ and $|Y|\lesssim \lambda^{-1}$, but now $|\tilde T|\lesssim h^{\varepsilon}$. Therefore we now get a condition on $\lambda$ that reads
$$
\lambda^{\frac 1 q}h^{\varepsilon/q} \times \lambda^{-\frac 5{3r}-\frac 1 3}\lesssim \lambda^{1-\frac 1 q-\frac 2 r} \lambda^{\frac 1 2 -\frac 1 r -\frac 2 q}  \lambda^{-\frac 5 4} \lambda^{\frac 1 {4}}h^{2\varepsilon(\frac 3 4 -\frac 1 r -\frac 3q)}\,;
$$
this turns out to match exactly the requirements from \cite{doi} (see Remark 1.7): for $(q,r)$ such that $r\geq 4$, we necessarily have, with a positive $C(\cdot)$ in the meaningful range,
$$
   \frac 3 q +\frac 1 r \leq \frac {15} {24}-\varepsilon C(\varepsilon,q,r)\,.
$$
 One may take $\varepsilon$ to zero and rewrite this condition on $(q,r)$ to highlight its distance to the free space requirement:
 \begin{equation}
   \label{eq:doi2d}
   \frac 1 q \leq  \left( \frac 1 2 -\frac {1-4/r}{12-24/r}\right) \left(\frac 1 2 -\frac 1 r\right)
 \end{equation}
 making clear the restriction $r\geq 4$ to be relevant as well as the loss ($1/12$ for the $(q,\infty)$ pair, e.g. $q\geq 24/5>4$.)
 \subsection{Higher dimensions}

In this section we prove Theorem \ref{thm-2} by taking advantage of the 2D example we just constructed: consider for simplicity, for $d\geq 3$, the isotropic model convex domain   $\Omega_d=\{(x,y)\in\mathbb{R}^{1+d}|, x>0,y\in\mathbb{R}^{d}\}$ and
$\Delta_F=\partial^{2}_{x}+(1+x)\Delta_{y}$ with Dirichlet boundary condition (one may without loss of generality replace $x\Delta_{y}$ by $xQ(y)$ where $Q$ is a constant coefficient second order elliptic operator). Denote by $u(t,x,y_{1})$ the solution to the 2D equation we previously constructed (in unscaled variables), and let $\phi$ be a smooth function from $\mathbb{R}^{d-2}$ to $\mathbb{R}$  such that $\hat \phi$ is positive, has compact support in a ball of size one and $\hat \phi=1$ near the origin. We may moreover select such a bump function so that, for $|y'|\leq 1$ $\phi(y')\geq 1/10$. Set $\phi_{h}(y')= h^{-(d-2)/4}\phi(y'/\sqrt h)$, which is $L^{2}-$normalized.

We seek a solution to the $d-$dimensional wave equation of the form
$$
v(t,x,y_{1},y')=u(t,x,y_{1}) \phi_{h}(y')+w(t,x,y)\,,
$$
with $w(0,x,y)=0$. Plugging our ansatz into the wave equation, we get
$$
(\partial_{t}^{2} -\Delta_{F}) w+\phi_{h}(y')(\partial_{t}^{2} -(\partial_{x}^{2}+(1+x)\partial^{2}_{y_{1}}) u -u(t,x,y_{1}) (1+x)\Delta_{y'} \phi_{h}=0\,.
$$
The middle term vanishes since $u$ is a solution to the 2D wave equation, and $\Delta_{y'} \phi_{h}(y')= \tilde \phi_{h}(y') /h$, where $\tilde \phi_{h}$ is again $L^{2}-$normalized. Therefore,
$$
(\partial_{t}^{2} -\Delta_{F}) w= \frac 1 h  u(t,x,y_{1}) (1+x)\tilde \phi_{h}(y')\,.
$$
If we denote by $F$ the source term, $w$ is the solution given by the Duhamel formula: if the wave equation satisfies an homogeneous Strichartz estimate with exponents $(q,r)$, then
\begin{equation}\label{stricrdinhomo} h^{\beta}\|\chi(hD_t)w\|_{L^q([0,T],
L^r)}\lesssim  h \int_{0}^{T}\|F\|_{L^{2}_{x,y}}\,,
\end{equation}
and therefore we have
\begin{align*}
h^{\beta}\|\chi(hD_t)w\|_{L^q([0,T],
  L^r)} & \lesssim  T \sup_{t}\| u(t,x,y_{1}) (1+x)\tilde \phi_{h}(y')\|_{L^{2}_{x,y}}\\
  & \lesssim  T \| u(0,x,y_{1}) \|_{L^{2}_{x,y_{1}}}\,.
\end{align*}
We are left with computing the $L^{r}_{y'}$ norm of $ \phi_{h}$: from its construction, we have $\|\phi\|_{L^{r}}\sim 1$ and by rescaling,
$$
\|\phi_h\|_{L^{r}}\sim h^{-\frac{d-2} 4} h^{\frac{d-2} {2r}}=h^{\frac{d-2} 2 (\frac{1} {r}-\frac 1 2)}\,.
$$
  From
  $$
  \|u\|_{L^{q}_{t}L^{r}_{x,y_{1}}} \|\phi_{h}\|_{L^{r}_{y'}} -\|w\|_{L^{q}_{t}L^{r}_{x,y}}\leq \|v\|_{L^{q}_{t}L^{r}_{x,y}}
  $$
  and our computation from the 2D case in the case where $a\sim h^{1/2-\varepsilon}$, together with $\lambda \sim h^{-1/4-3\varepsilon/2}$, we eventually get the limiting condition
  $$
2(d-2)(\frac 1 2-\frac 1 r)- \frac 4{q}-\frac{4}{3r}+\frac 5 6\leq 0
$$
in other words
$$
\frac 1 q \leq \left( \frac {d-1} 2 -\frac {1-4/r}{12-24/r}\right) \left(\frac 1 2 -\frac 1 r\right)\,,
$$
which is the desired condition.

\begin{rmq}
    The general philosophy is that of the usual Knapp counterexample: the main propagation is in the direction $y_{1}$, and our wave packet has no time to decorrelate in transverse directions. A similar argument was used in \cite{doi2} to extend the previous counterexamples from the 2D model to the general case of any strictly convex domain in higher dimensions.
  \end{rmq}
  \begin{rmq}
    If one plugs the other case, $a\sim h^{1/3}$ and $M\sim \lambda^{1/3}$ in the higher dimensional setting, it does not provide any interesting condition. The main difference appears to be that in that later case, the 2D counterexample is reaching its peak on very small subintervals (size $\lambda^{-1/3}$) whereas in the limiting sense, for $a\sim h^{1/2}$ and $M\sim \lambda$, the constructed example maintains its peak on the whole time interval, like the usual Knapp counterexample.
  \end{rmq}

  \appendix
\section{Complements on Airy and related functions}
\subsection{Proof of Lemma \ref{lemL}}
 We have the following well-known asymptotic expansion 
 \begin{equation}\label{eq:ApmAE}
 A_{\pm}(z)=\Psi(e^{\mp i\pi/3} z)e^{\mp\frac 23 i z^{3/2} },
 \end{equation}
 where $\Psi(z)\simeq z^{-1/4}\sum_{j=0}^{\infty} a_j z^{-3j/2}$, with $a_0=\frac{1}{4\pi^{3/2}}$. The first part of the statement of Lemma \ref{lemL} then follows easily using the properties of $A_{\pm}$ and their asymptotic behavior; the statement \eqref{eq:propL2} follows by explicit computations. 
Indeed, at $\omega=0$ we have $\frac{A_-(0)}{A_+(0)}=e^{2\pi i/3}$, hence $L(0)=\frac{\pi}{3}$.
Since $A_+$ is analytic with values in $\mathbb{C}$ and it does never vanish on the real line, there exist unique analytic functions $\rho(\omega)>0$ and $\theta(\omega)\in\mathbb{R}$ with $\theta(0)=-\pi/3$, such that $A_+(\omega)=\rho(\omega)e^{i\theta(\omega)}$. Then one has $A_-(\omega)=\rho(\omega)e^{-i\theta(\omega)}$ and by definition $L(\omega)=\pi+2\theta(\omega)$ is real on the real axis. Since the Wronskian $A'_+(\omega)A_-(\omega)-A'_-(\omega)A_+(\omega)$ is constant, the following identity holds
\begin{align}\label{LL}
L'(\omega)&=i\frac{A_+}{A_-}(\omega)\Big(\frac{A'_-}{A_+}(\omega)-\frac{A_-A'_+}{A_+^2}(\omega)\Big)\\
&=-\frac{i}{A_+(\omega)A_-(\omega)}(A'_+(\omega)A_-(\omega)-A'_-(\omega)A_+(\omega)) \nonumber \\
&=\frac{c_0}{ \rho^2(\omega)}, \quad c_0=-\sqrt{3}Ai'(0)Ai(0)>0.\nonumber
\end{align}
Indeed, the Wronskian has value
\begin{align*}
W(\omega) & :=A'_+(\omega)A_-(\omega)-A'_-(\omega)A_+(\omega)\\
& =e^{-i\pi/3}Ai'(e^{-i\pi/3}\omega)Ai(e^{+i\pi/3}\omega)-e^{i\pi/3}Ai'(e^{i\pi/3}\omega)Ai(e^{-i\pi/3}\omega)\\
 & =  W(0) =(e^{-i\pi/3}-e^{i\pi/3})Ai'(0)Ai(0)\\
 &  =-2i \sin(\pi/3)Ai'(0)Ai(0)=-i\sqrt{3}Ai'(0)Ai(0)\,.
\end{align*}
This means that $L$ is strictly increasing. The asymptotic expansion of $L(\omega)$ at $\pm\infty$ follows from the asymptotic expansion of the Airy function, which is valid in any strict sub-sector of $|\arg(z)|<\pi$, where
\[
Ai(z)\simeq \frac{e^{-\frac 23 z^{3/2}}}{2\sqrt{\pi}|z|^{1/4}}\Big(\sum_{n\geq 0}(-1)^n(3/4)^n\frac{\Gamma(n+5/6)\Gamma(n+1/6)}{2\pi n! z^{3n/2}}\Big).
\]
Set $A(\omega)=Ai(-\omega)$, then $A(\omega)=2\rho(\omega)\cos(\theta(\omega))$. Therefore, $A(\omega)=0$ is equivalent to $\theta(\omega)=\pi/2+l\pi$, $l\in \mathbb{Z}$, which is equivalent to $L(\omega)=2\pi(1+l)$. From $L$ being a diffeomorphism from $\mathbb{R}$ onto $(0,\infty)$, one has that for all integer $k\geq 1$, $Ai(-\omega_k)=0$ if and only if $L(\omega_k)=2\pi k$.

Finally, using the Airy equation $A''(z)+z A(z)=0$ and integrating by parts, we get
\begin{align*}
\int_{\infty}^{\omega}A^2(z)dz & =\omega A^2(\omega)-\int_{\infty}^{\omega}2zA(z)A'(z)dz\\
 & =\omega A^2(\omega)+\int_{\infty}^{\omega}2A''(z)A'(z)dz\\
 & =\omega A^2(\omega)+A'^2(\omega).
\end{align*}
Since $A'(\omega_k)=2\rho'(\omega_k)\cos(\theta(\omega_k))+2\rho(\omega_k)\theta'(\omega_k)\sin(\theta(\omega_k))$, and $A(\omega_k)=2\rho(\omega_k)\cos(\theta(\omega_k))=0$, hence $\sin(\theta(\omega_k))\in \{\pm1\}$, we get using \eqref{LL}
\[
\int_0^{\infty}Ai^2(x-\omega_k)dx=A'^2(\omega_k)=Ai'^2(-\omega_k)=4\rho^2(\omega_k)\theta'^2(\omega_k)=\rho^2(\omega_k)L'^2(\omega_k)=c_0L'(\omega_k).
\]
From $2\pi Ai(0)=3^{-1/6}\Gamma(1/3)$, $2\pi Ai'(0)=-3^{1/6}\Gamma(2/3)$ and the Euler formula for the $\Gamma$ function, $\Gamma(x)\Gamma(1-x)=\frac{\pi}{\sin(\pi x)}$, we get $2\pi c_0=1$, thus the last assertion in \eqref{eq:propL2} holds true. The proof of the Lemma is complete.\qed
\subsection{Proof of Lemma \ref{lemorthog}}
Using the equation satisfied by the Airy function, it is easy to see that $e_k$ are the eigenfunctions of $-\partial^2_x+(1+x)\theta^2$, associated with the eigenvalues $\lambda_k(\theta)$. It will be enough to prove that they form an orthogonal family on $L^2(\mathbb{R}_+)$. In order to do so, we use well-known formulas for primitives of Airy functions (see \cite[(3.53)]{AFbook}):
\begin{multline}
\int Ai(\alpha(x+\beta_1))Ai(\alpha(x+\beta_2))dx=\\
\frac{1}{\alpha^2(\beta_1-\beta_2)}\Big[Ai'(\alpha(x+\beta_1))Ai(\alpha(x+\beta_2))-Ai(\alpha(x+\beta_1))Ai'(\alpha(x+\beta_2)))\Big].
\end{multline}
Taking the integral on $[0,\infty)$, $\alpha=1$, $\beta_1=-\omega_k$, $\beta_2=-\omega_j$ with $k\neq j$ gives 
\[
<e_k,e_j>_{L^2(\mathbb{R}_+)}=0\,,
\]
and this holds for all $k(\neq j)$.\qed

\subsection{Proof of Lemma \ref{AiryPoisson}}
Consider $\phi(\omega)$, a smooth, compactly supported, function defined for $\omega \in \R$. The function $L(\omega)$ defines a one to one map from $\R$ to $\R_{+}$. Now define $\varphi(x)$ for $x\in \R_{+}$ with
$$
\varphi(L(\omega))=\frac 1 {L'(\omega)} \phi (\omega)\,.
$$
We may extend $\varphi$ to be zero for $x\in \R_{-}$ and still retain a smooth, compactly supported function:  there exists $\omega^{\flat}$ such that $\phi(\omega)=0$ if $\omega<\omega^{\flat}$, and we have $\varphi(x)=0$ for $x<L(\omega^{\flat})$, e.g. $\varphi$ is always supported on $\R_{+}^{*}$. We apply the usual Poisson sumation formula to $\varphi$, which reads
$$
\sum_{k\in \Z} \varphi(2\pi k)=\frac 1 {2\pi} \sum_{N\in \Z} \int_{\R} e^{-i N x}\varphi(x)\,dx\,.
$$
Since $\varphi$ vanishes on $\R_{-}$, this becomes
$$
\sum_{k\in \N^{*}} \varphi(2\pi k)=\frac 1 {2\pi} \sum_{N\in \Z} \int_{\R_{+}} e^{-i N x}\varphi(x)\,dx\,,
$$
and we can now change variables with $x=L(\omega)$:
$$
\sum_{k\in \N^{*}} \varphi(2\pi k)=\frac 1 {2\pi} \sum_{N\in \Z} \int_{\R} e^{-i N L(\omega)}\varphi(L(\omega))L'(\omega)\,d\omega\,,
$$
and recalling that $L(\omega_{k})=2\pi k$ this reads
$$
\sum_{k\in \N^{*}} \varphi(L(\omega_{k}))=\frac 1 {2\pi} \sum_{N\in \Z} \int_{\R} e^{-i N L(\omega)}\varphi(L(\omega))L'(\omega)\,d\omega\,.
$$
Finally, with $\varphi\circ L=\phi/L'$, 
$$
\sum_{k\in \N^{*}} \frac {\phi(\omega_{k})}{L'(\omega_{k})}=\frac 1 {2\pi} \sum_{N\in \Z} \int_{\R} e^{-i N L(\omega)}\phi(\omega)\,d\omega\,,
$$
which is the desired formula. \qed

\def\cprime{$'$} \def\cprime{$'$}
\begin{thebibliography}{10}

\bibitem{blsmso08}
Matthew~D. Blair, Hart~F. Smith, and Christopher~D. Sogge.
\newblock Strichartz estimates for the wave equation on manifolds with
  boundary.
\newblock {\em Ann. Inst. H. Poincar\'e Anal. Non Lin\'eaire},
  26(5):1817--1829, 2009.

\bibitem{doi}
Oana Ivanovici.
\newblock Counterexamples to {S}trichartz estimates for the wave equation in
  domains.
\newblock {\em Math. Ann.}, 347(3):627--673, 2010.

\bibitem{doi2}
Oana Ivanovici.
\newblock Counterexamples to the {S}trichartz inequalities for the wave
  equation in general domains with boundary.
\newblock {\em J. Eur. Math. Soc. (JEMS)}, 14(5):1357--1388, 2012.

\bibitem{ILLP}
Oana Ivanovici, Richard Lascar, Gilles Lebeau, and Fabrice Planchon.
\newblock Dispersion for the wave equation inside strictly convex domains {II}:
  the general case, 2016.
\newblock {\tt arXiv:math/1605.08800}.

\bibitem{ildispext}
Oana Ivanovici and Gilles Lebeau.
\newblock Dispersion for the wave and the {S}chr\"{o}dinger equations outside
  strictly convex obstacles and counterexamples.
\newblock {\em C. R. Math. Acad. Sci. Paris}, 355(7):774--779, 2017.

\bibitem{ILP3}
Oana Ivanovici, Gilles Lebeau, and Fabrice Planchon.
\newblock {Strichartz estimates for the wave equation inside strictly convex
  2D} model domain.
\newblock preprint.

\bibitem{Annals}
Oana Ivanovici, Gilles Lebeau, and Fabrice Planchon.
\newblock Dispersion for the wave equation inside strictly convex domains {I}:
  the {F}riedlander model case.
\newblock {\em Ann. of Math. (2)}, 180(1):323--380, 2014.

\bibitem{smso95}
Hart~F. Smith and Christopher~D. Sogge.
\newblock On the critical semilinear wave equation outside convex obstacles.
\newblock {\em J. Amer. Math. Soc.}, 8(4):879--916, 1995.

\bibitem{smso06}
Hart~F. Smith and Christopher~D. Sogge.
\newblock On the {$L\sp p$} norm of spectral clusters for compact manifolds
  with boundary.
\newblock {\em Acta Math.}, 198(1):107--153, 2007.

\bibitem{tat02}
Daniel Tataru.
\newblock Strichartz estimates for second order hyperbolic operators with
  nonsmooth coefficients. {III}.
\newblock {\em J. Amer. Math. Soc.}, 15(2):419--442 (electronic), 2002.

\bibitem{AFbook}
Olivier Vall\'{e}e and Manuel Soares.
\newblock {\em Airy functions and applications to physics}.
\newblock Imperial College Press, London; Distributed by World Scientific
  Publishing Co. Pte. Ltd., Hackensack, NJ, 2004.

\end{thebibliography}

\end{document}

