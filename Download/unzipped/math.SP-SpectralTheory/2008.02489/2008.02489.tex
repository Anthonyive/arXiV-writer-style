\documentclass[11pt,a4paper]{amsart}
\usepackage{amsfonts,amsmath,amssymb,amsthm,mathtools}
\usepackage[noadjust]{cite}
\usepackage[textsize=tiny]{todonotes}

\numberwithin{equation}{section}

\DeclareMathOperator{\Ker}{Ker}
\DeclareMathOperator{\Ran}{Ran}
\DeclareMathOperator{\Dom}{Dom}
\DeclareMathOperator{\spec}{spec}
\DeclareMathOperator{\sign}{sign}
\DeclareMathOperator{\Tan}{Tan}
\DeclareMathOperator{\divgc}{div}
\DeclareMathOperator{\grad}{grad}

\DeclareSymbolFont{SY}{U}{psy}{m}{n}
\DeclareMathSymbol{\emptyset}{\mathord}{SY}{'306}

\DeclarePairedDelimiter{\abs}{|}{|}
\DeclarePairedDelimiter{\norm}{\lVert}{\rVert}
\DeclarePairedDelimiter{\scprod}{\langle}{\rangle}

\newcommand{\dd}{\mathrm d}
\newcommand{\eps}{\varepsilon}

\newcommand{\ess}{\mathrm{ess}}

\newcommand{\N}{\mathbb{N}}
\newcommand{\R}{\mathbb{R}}
\newcommand{\NN}{\mathbb{N}}
\newcommand{\RR}{\mathbb{R}}
\newcommand{\CC}{\mathbb{C}}
\newcommand{\EE}{\mathsf{E}}

\newcommand{\cC}{{\mathcal C}}
\newcommand{\cD}{{\mathcal D}}
\newcommand{\cH}{{\mathcal H}}
\newcommand{\cK}{{\mathcal K}}
%\newcommand{\cM}{{\mathcal M}}
\newcommand{\cQ}{{\mathcal Q}}

\newcommand{\fa}{{\mathfrak a}}
\newcommand{\fb}{{\mathfrak b}}
\newcommand{\fv}{{\mathfrak v}}
\newcommand{\fD}{{\mathfrak D}}
\newcommand{\fM}{{\mathfrak M}}

\theoremstyle{plain}
\newtheorem{theorem}{Theorem}[section]
\newtheorem{proposition}[theorem]{Proposition}
\newtheorem{lemma}[theorem]{Lemma}
\newtheorem{corollary}[theorem]{Corollary}

\theoremstyle{definition}
\newtheorem{hypothesis}[theorem]{Hypothesis}

\theoremstyle{remark}
\newtheorem{remark}[theorem]{Remark}

%%%%%%%%%%%%%%%%%%%%%%%%%%%%%%%%%%%%%%%%%%%%%%%%%%%%%%%%%%%%%%%%%%%%%%%%%%%%%%%%%%%%%%%%%%%%%%%%%%%%%%%%%%%%%%%%%%%%%%%%%%%%%%%%%
%%% Title and other information on the work
%%%%%%%%%%%%%%%%%%%%%%%%%%%%%%%%%%%%%%%%%%%%%%%%%%%%%%%%%%%%%%%%%%%%%%%%%%%%%%%%%%%%%%%%%%%%%%%%%%%%%%%%%%%%%%%%%%%%%%%%%%%%%%%%%
\title[On a minimax principle in spectral gaps]{On a minimax principle in spectral gaps}
\subjclass[2010]{Primary 49Rxx; Secondary 47A10, 47A75}
\keywords{Minimax values, eigenvalues in gap of the essential spectrum, block diagonalization, Stokes operator}
\date{}

%%%%%%%%%%%%%%%%%%%%%%%%%%%%%%%%%%%%%%%%%%%%%%%%%%%%%%%%%%%%%%%%%%%%%%%%%%%%%%%%%%%%%%%%%%%%%%%%%%%%%%%%%%%%%%%%%%%%%%%%%%%%%%%%%
%%% Author
%%%%%%%%%%%%%%%%%%%%%%%%%%%%%%%%%%%%%%%%%%%%%%%%%%%%%%%%%%%%%%%%%%%%%%%%%%%%%%%%%%%%%%%%%%%%%%%%%%%%%%%%%%%%%%%%%%%%%%%%%%%%%%%%%
\author[A.\ Seelmann]{Albrecht Seelmann}
\address{A.~Seelmann,
 Technische Univer\-si\-t\"at Dortmund, Fakult\"at f\"ur Mathematik, D-44221 Dortmund, Germany}
\email{albrecht.seelmann@math.tu-dortmund.de}


%%%%%%%%%%%%%%%%%%%%%%%%%%%%%%%%%%%%%%%%%%%%%%%%%%%%%%%%%%%%%%%%%%%%%%%%%%%%%%%%%%%%%%%%%%%%%%%%%%%%%%%%%%%%%%%%%%%%%%%%%%%%%%%%%
%%%%%%%%%%%%%%%%%%%%%%%%%%%%%%%%%%%%%%%%%%%%%%%%%%%%%%%%%%%%%%%%%%%%%%%%%%%%%%%%%%%%%%%%%%%%%%%%%%%%%%%%%%%%%%%%%%%%%%%%%%%%%%%%%
%%%%%%%%%%%%%%%%%%%%%%%%%%%%%%%%%%%%%%%%%%%%%%%%%%%%%%%%%%%%%%%%%%%%%%%%%%%%%%%%%%%%%%%%%%%%%%%%%%%%%%%%%%%%%%%%%%%%%%%%%%%%%%%%%
%%% Begin of the work
%%%%%%%%%%%%%%%%%%%%%%%%%%%%%%%%%%%%%%%%%%%%%%%%%%%%%%%%%%%%%%%%%%%%%%%%%%%%%%%%%%%%%%%%%%%%%%%%%%%%%%%%%%%%%%%%%%%%%%%%%%%%%%%%%
%%%%%%%%%%%%%%%%%%%%%%%%%%%%%%%%%%%%%%%%%%%%%%%%%%%%%%%%%%%%%%%%%%%%%%%%%%%%%%%%%%%%%%%%%%%%%%%%%%%%%%%%%%%%%%%%%%%%%%%%%%%%%%%%%
%%%%%%%%%%%%%%%%%%%%%%%%%%%%%%%%%%%%%%%%%%%%%%%%%%%%%%%%%%%%%%%%%%%%%%%%%%%%%%%%%%%%%%%%%%%%%%%%%%%%%%%%%%%%%%%%%%%%%%%%%%%%%%%%%
\begin{document}

%%%%%%%%%%%%%%%%%%%%%%%%%%%%%%%%%%%%%%%%%%%%%%%%%%%%%%%%%%%%%%%%%%%%%%%%%%%%%%%%%%%%%%%%%%%%%%%%%%%%%%%%%%%%%%%%%%%%%%%%%%%%%%%%%
%%% Abstract
%%%%%%%%%%%%%%%%%%%%%%%%%%%%%%%%%%%%%%%%%%%%%%%%%%%%%%%%%%%%%%%%%%%%%%%%%%%%%%%%%%%%%%%%%%%%%%%%%%%%%%%%%%%%%%%%%%%%%%%%%%%%%%%%%
\begin{abstract}
  The minimax principle for eigenvalues in gaps of the essential spectrum by Griesemer, Lewis, and Siedentop
  in~[Doc.~Math.~\textbf{4} (1999), 275--283] is adapted to cover certain abstract perturbative settings with bounded or
  unbounded perturbations, in particular ones that are off-diagonal with respect to the spectral gap under consideration. This in
  part builds upon and extends the considerations in the author's appendix to~[arXiv:1804.07816]. Several monotonicity and
  continuity properties of eigenvalues in gaps of the essential spectrum are deduced, and the Stokes operator is revisited as an
  example.
\end{abstract}

\maketitle

%%%%%%%%%%%%%%%%%%%%%%%%%%%%%%%%%%%%%%%%%%%%%%%%%%%%%%%%%%%%%%%%%%%%%%%%%%%%%%%%%%%%%%%%%%%%%%%%%%%%%%%%%%%%%%%%%%%%%%%%%%%%%%%%%
%%%%%%%%%%%%%%%%%%%%%%%%%%%%%%%%%%%%%%%%%%%%%%%%%%%%%%%%%%%%%%%%%%%%%%%%%%%%%%%%%%%%%%%%%%%%%%%%%%%%%%%%%%%%%%%%%%%%%%%%%%%%%%%%%
%%% Introduction
%%%%%%%%%%%%%%%%%%%%%%%%%%%%%%%%%%%%%%%%%%%%%%%%%%%%%%%%%%%%%%%%%%%%%%%%%%%%%%%%%%%%%%%%%%%%%%%%%%%%%%%%%%%%%%%%%%%%%%%%%%%%%%%%%
%%%%%%%%%%%%%%%%%%%%%%%%%%%%%%%%%%%%%%%%%%%%%%%%%%%%%%%%%%%%%%%%%%%%%%%%%%%%%%%%%%%%%%%%%%%%%%%%%%%%%%%%%%%%%%%%%%%%%%%%%%%%%%%%%
\section{Introduction and main result}\label{sec:intro}

The standard Courant minimax values $\lambda_k(A)$ of a lower semibounded operator $A$ on a Hilbert space $\cH$ are given by
\begin{equation*}
  \lambda_k(A)
  =
  \inf_{\substack{\fM\subset\Dom(A)\\ \dim\fM=k}} \sup_{\substack{x\in\fM\\ \norm{x}=1}} \scprod{ x , Ax }
  =
  \inf_{\substack{\fM\subset\Dom(|A|^{1/2})\\ \dim\fM=k}} \sup_{\substack{x\in\fM\\ \norm{x}=1}} \fa[x,x]
\end{equation*}
for $k\in\N$ with $k\le\dim\cH$, see, e.g.,~\cite[Theorem~12.1]{LL01} and also~\cite[Section~12.1 and Exercise~12.4.2]{Schm12}.
Here, $\langle\cdot,\cdot\rangle$ denotes the inner product of $\cH$, and $\fa$ with
$\fa[x,x] = \scprod{ \abs{A}^{1/2}x , \sign(A)\abs{A}^{1/2}x }$ for $x\in\Dom(\abs{A}^{1/2})$ is the form associated with $A$.

The above minimax values have proved to be a powerful description of the eigenvalues below the essential spectrum of $A$; in
fact, they agree with these eigenvalues in nondecreasing order counting multiplicities. A standard application in this context is
that these eigenvalues exhibit a monotonicity with respect to the operator: for two self-adjoint operators $A$ and $B$ with
$A\le B$ in the sense of quadratic forms one has $\lambda_k(A) \le \lambda_k(B)$ for all $k$, see,
e.g.,~\cite[Corollary~12.3]{Schm12}.

Matters get, however, much more complicated when eigenvalues in a gap of the essential spectrum are considered. If $A_+$ is the
(lower semibounded) part of $A$ with spectrum in an interval of the form $(\gamma,\infty)$, $\gamma\in\R$, then the minimax
values for $A_+$ still describe the eigenvalues of $A_+$ below its essential spectrum and thus the eigenvalues of $A$ in
$(\gamma, \infty)$ below the essential spectrum of $A$ above $\gamma$. However, the subspaces over which the corresponding
infimum is taken are chosen within the spectral subspace for $A$ associated with the interval $(\gamma,\infty)$ and therefore
usually depend on the operator itself rather than just its domain. This makes it difficult to compare minimax values in spectral
gaps of two different operators $A$ and $B$, even if their domains agree.

In~\cite{GLS99}, Griesemer, Lewis, and Siedentop devised an abstract minimax principle for eigenvalues in spectral gaps that
allows to overcome these problems. However, the corresponding hypotheses seem to be hard to verify on an abstract level,
cf.~Remark~\ref{rem:neg}\,(2) below. In the particular situation of bounded additive perturbations, the present author has
adapted this abstract minimax principle in the appendix to~\cite{NSTTV18} with hypotheses that can in some cases be verified
explicitly by means of the Davis-Kahan $\sin2\Theta$ theorem from~\cite{DK70} and variants thereof. This has been successfully
applied in~\cite{NSTTV18} to study lower bounds on the movement of eigenvalues in gaps of the essential spectrum and of edges of
the essential spectrum. In the present note, the considerations from~\cite[Appendix~A]{NSTTV18} are supplemented and extended to
cover also certain unbounded perturbations, in particular ones that are off-diagonal with respect to the spectral gap under
consideration. It should be mentioned that some of the results discussed here might also be obtained with the alternative
approaches from~\cite{DES00, DES06, MM15, LS16}. However, the present work focuses on~\cite{GLS99} as a starting point since the
techniques employed to apply that abstract minimax principle promise to be of a broader interest.

%%%%%%%%%%%%%%%%%%%%%%%%%%%%%%%%%%%%%%%%%%%%%%%%%%%%%%%%%%%%%%%%%%%%%%%%%%%%%%%%%%%%%%%%%%%%%%%%%%%%%%%%%%%%%%%%%%%%%%%%%%%%%%%%%
%%% Main results
%%%%%%%%%%%%%%%%%%%%%%%%%%%%%%%%%%%%%%%%%%%%%%%%%%%%%%%%%%%%%%%%%%%%%%%%%%%%%%%%%%%%%%%%%%%%%%%%%%%%%%%%%%%%%%%%%%%%%%%%%%%%%%%%%
\subsection*{Main results}
In order to formulate our main results, it is convenient to fix the following notational setup in the case where $\gamma = 0$;
the case of general $\gamma \in \RR$ can of course always be reduced to this situation by spectral shift,
cf.~Remark~\ref{rem:spectralShift} and also the proofs of Proposition~\ref{prop:boundedPert} below.

\begin{hypothesis}\label{hyp:minimax}
  Let $A$ be a self-adjoint operator on a Hilbert space. Denote the spectral projectors for $A$ associated with the intervals
  $(0,\infty)$ and $(-\infty,0]$ by $P_+$ and $P_-$, respectively, that is,
  \begin{equation*}
    P_+ := \EE_A\bigl((0,\infty)\bigr)
    ,\quad
    P_- := I-P_+
    ,
  \end{equation*}
  and let
  \begin{equation*}
    \cD_\pm := \Ran P_\pm \cap \Dom(A)
    ,\quad
    \fD_\pm := \Ran P_\pm \cap \Dom(\abs{A}^{1/2})
    .
  \end{equation*}
  Moreover, let $B$ be another self-adjoint operator on the same Hilbert space with analogously defined spectral projections
  \begin{equation*}
    Q_+ := \EE_B\bigl((0,\infty)\bigr)
    ,\quad
    Q_- := I - Q_+
    ,
  \end{equation*}
  and denote by $\fb$ the form associated with $B$, that is,
  \begin{equation*}
    \fb[x,y] = \scprod{ \abs{B}^{1/2}x , \sign(B)\abs{B}^{1/2}y }
  \end{equation*}
  for $x,y\in\Dom(\fb) = \Dom(\abs{B}^{1/2})$.
\end{hypothesis}

Here, $\EE_A$ and $\EE_B$ stand for the projection-valued spectral measures for the operators $A$ and $B$, respectively, and
$\Ran P_\pm$ denotes the range of $P_\pm$. We have also used the notation $I$ for the identity operator.

Denoting the form associated with $A$ by $\fa$, the minimax values of the positive part $A|_{\Ran P_+}$ of $A$ can clearly be
written as
\begin{equation*}
  \lambda_k(A|_{\Ran P_+})
  =
  \inf_{\substack{\fM_+\subset\cD_+\\ \dim\fM_+=k}} \sup_{\substack{x\in\fM_+ \oplus \cD_-\\ \norm{x}=1}} \scprod{ x , Ax }
  =
  \inf_{\substack{\fM_+\subset\fD_+\\ \dim\fM_+=k}} \sup_{\substack{x\in\fM_+ \oplus \fD_-\\ \norm{x}=1}} \fa[x,x]
\end{equation*}
for $k\in\N$ with $k \le \dim \Ran P_+$. In our main results below we give conditions on $B$ under which the minimax values for
the positive part $B|_{\Ran Q_+}$ of $B$ admit the same representations with $\scprod{x , Ax }$ and $\fa[x , x]$ replaced by
$\scprod{x , Bx}$ and $\fb[x , x]$, respectively, but with the infima taken over the same respective families of subspaces as for
$A$. It is natural to consider this in an perturbative framework where $B$ is obtained by an operator or form perturbation of $A$
and, thus, one has $\Dom(A) = \Dom(B)$ and/or $\Dom(\abs{A}^{1/2}) = \Dom(\abs{B}^{1/2})$.

Four results in this direction are presented here, each addressing different situations. We first treat the case of operator
perturbations and start with the direct extension of~\cite[Theorem~A.2]{NSTTV18} to infinitesimal perturbations. Recall that an
operator $V$ is called $A$-bounded with $A$-bound $b_* \ge 0$ if $\Dom(V) \supset \Dom(A)$ and for all $b > b_*$ there is some
$a \ge 0$ with
\begin{equation*}
  \norm{ Vx }
  \le
  a\norm{x} + b\norm{Ax}
  \quad\text{ for all }\
  x \in \Dom(A)
  .
\end{equation*}
If $b_* = 0$, then $V$ is called infinitesimal with respect to $A$.

\begin{theorem}\label{thm:genOpInfinitesimal}
  Assume Hypothesis~\ref{hyp:minimax}. Suppose, in addition, that $B$ is of the form $B = A + V$ with some symmetric operator $V$
  that is infinitesimal with respect to $A$. Furthermore, suppose that $\norm{P_+Q_-} < 1$ and that
  \begin{equation*}
    \scprod{ x , Bx } \le 0 \quad\text{ for all }\ x \in \cD_-.
  \end{equation*}
  Then,
  \begin{equation*}
    \lambda_k(B|_{\Ran Q_+})
    =
    \inf_{\substack{\fM_+\subset\cD_+\\ \dim\fM_+=k}} \sup_{\substack{x\in\fM_+ \oplus \cD_-\\ \norm{x}=1}} \scprod{ x , Bx }
    =
    \inf_{\substack{\fM_+\subset\fD_+\\ \dim\fM_+=k}} \sup_{\substack{x\in\fM_+ \oplus \fD_-\\ \norm{x}=1}} \fb[x,x]
  \end{equation*}
  for all $k \in \NN$ with $k \le \dim\Ran P_+$.
\end{theorem}

Two remarks regarding Theorem~\ref{thm:genOpInfinitesimal} are in order:
(1)
also certain perturbations $V$ that are not infinitesimal with respect to $A$ can be considered here, but at the cost of a
stronger assumption on $\norm{P_+Q_-}$, see Remark~\ref{rem:relBoundK} below;
(2)
the condition $\norm{P_+Q_-} < 1$ is satisfied if the stronger inequality $\norm{P_+-Q_+} < 1$ holds. In the latter case, the
subspaces $\Ran P_+$ and $\Ran Q_+$ have the same dimension, that is, $\dim \Ran P_+ = \dim \Ran Q_+$, see
Remark~\ref{rem:PQbij}\,(a) below.

The stronger condition~$\norm{P_+-Q_+} < 1$ just mentioned in fact also opens the way to employ a different approach than the one
used to prove Theorem~\ref{thm:genOpInfinitesimal}. This alternative approach has previously been used in the context of block
diagonalization of operators and forms, see Section~\ref{sec:blockDiag} below, and is particularly attractive if the unperturbed
operator $A$ is semibounded.

\begin{theorem}\label{thm:genSemibounded}
  Assume Hypothesis~\ref{hyp:minimax}. Suppose, in addition, that $A$ is semibounded and that $\norm{P_+ - Q_+} < 1$.

  If $\Dom(\abs{A}^{1/2}) = \Dom(\abs{B}^{1/2})$ and $\fb[ x , x ] \le 0$ for all $x \in \fD_-$, then
  \begin{equation}\label{eq:genSemibounded:form}
    \lambda_k(B|_{\Ran Q_+})
    =
    \inf_{\substack{\fM_+\subset\fD_+\\ \dim\fM_+=k}} \sup_{\substack{x\in\fM_+ \oplus \fD_-\\ \norm{x}=1}} \fb[x,x]
  \end{equation}
  for all $k \le \dim\Ran P_+ = \dim\Ran Q_+$. If even $\Dom(A) = \Dom(B)$ and $\scprod{ x , Bx } \le 0$ for all $x \in \cD_-$,
  then also
  \begin{equation}\label{eq:genSemibounded:op}
    \lambda_k(B|_{\Ran Q_+})
    =
    \inf_{\substack{\fM_+\subset\cD_+\\ \dim\fM_+=k}} \sup_{\substack{x\in\fM_+ \oplus \cD_-\\ \norm{x}=1}} \scprod{ x , Bx }
  \end{equation}
  for all $k \le \dim\Ran P_+ = \dim\Ran Q_+$.
\end{theorem}

It should be emphasized that the conditions $\Dom(A) = \Dom(B)$ and $\scprod{ x , Bx } \le 0$ for all $x \in \cD_-$ in
Theorem~\ref{thm:genSemibounded} indeed imply that one has also $\Dom(\abs{A}^{1/2}) = \Dom(\abs{B}^{1/2})$ and
$\fb[ x , x ] \le 0$ for all $x \in \fD_-$, see Lemma~\ref{lem:GLS} below. Note also that in contrast to
Theorem~\ref{thm:genOpInfinitesimal}, Theorem~\ref{thm:genSemibounded} makes no assumptions on how the operator $B$ is obtained
from $A$. The latter will, however, be relevant when the hypotheses of Theorem~\ref{thm:genSemibounded} are to be verified in
concrete situations.

The condition $\scprod{ x , Bx } \le 0$ for all $x \in \cD_-$ plays an important role in both
Theorems~\ref{thm:genOpInfinitesimal} and~\ref{thm:genSemibounded}. In the case where $B = A + V$ with some $A$-bounded symmetric
operator $V$ as in Theorem~\ref{thm:genOpInfinitesimal}, this condition is automatically satisfied if $\scprod{ x , Vx } \le 0$
for all $x \in \cD_-$ since $\scprod{ x , Ax } \le 0$ holds for all $x \in \cD_-$ by definition. A particular instance of such
perturbations $V$ are so-called~\emph{off-diagonal} perturbations with respect to the decomposition $\Ran P_+ \oplus \Ran P_-$,
in which case also the condition $\norm{P_+-Q_+} < 1$ can be verified efficiently. In comparison with
Theorem~\ref{thm:genOpInfinitesimal}, we may even relax the assumption on the $A$-bound of $V$ here.

\begin{theorem}\label{thm:offdiagOp}
  Assume Hypothesis~\ref{hyp:minimax}. Suppose, in addition, that $B$ has the form $B = A + V$ with some symmetric $A$-bounded
  operator $V$ with $A$-bound smaller than $1$ and which is off-diagonal on $\Dom(A)$ with respect to the decomposition
  $\Ran P_+ \oplus \Ran P_-$, that is,
  \begin{equation*}
    P_-VP_- x = 0 = P_+VP_+ x \quad\text{ for all }\ x \in \Dom(A).
  \end{equation*}
  Then, one has $\dim\Ran P_+ = \dim\Ran Q_+$ and
  \begin{equation*}
    \lambda_k(B|_{\Ran Q_+})
    =
    \inf_{\substack{\fM_+\subset\cD_+\\ \dim\fM_+=k}} \sup_{\substack{x\in\fM_+ \oplus \cD_-\\ \norm{x}=1}} \scprod{ x , Bx }
    =
    \inf_{\substack{\fM_+\subset\fD_+\\ \dim\fM_+=k}} \sup_{\substack{x\in\fM_+ \oplus \fD_-\\ \norm{x}=1}} \fb[x,x]
  \end{equation*}
  for all $k \in \NN$ with $k \le \dim\Ran Q_+$.
\end{theorem}

The last theorem can to some extend be formulated also for off-diagonal form perturbations, at least in the semibounded setting.
The latter restriction is commented on in Section~\ref{sec:blockDiag} below.

\begin{theorem}\label{thm:offdiagForm}
  Assume Hypothesis~\ref{hyp:minimax}. Suppose, in addition, that $B$ is semibounded and that its form $\fb$ is given by
  $\fb = \fa + \fv$, where $\fa$ is the form associated with $A$ and $\fv$ is a symmetric sesquilinear form satisfying
  \begin{equation*}
    \fv[ P_+x , P_+y ]
    =
    0
    =
    \fv[ P_-x , P_-y ]
    \quad\text{ for all }\
    x,y \in \Dom[\fa] \subset \Dom[\fv]
  \end{equation*}
  and
  \begin{equation}\label{eq:formRelBound}
    \abs{ \fv[ x , x ] }
    \le
    a\norm{x}^2 + b\abs{\fa[ x , x ]}
    \quad\text{ for all }\
    x \in \Dom(\abs{A}^{1/2}) = \Dom[\fa]
  \end{equation}
  with some constants $a,b \ge 0$.

  Then, one has $\dim\Ran P_+ = \dim\Ran Q_+$ and
  \begin{equation*}
    \lambda_k(B|_{\Ran Q_+})
    =
    \inf_{\substack{\fM_+\subset\fD_+\\ \dim\fM_+=k}} \sup_{\substack{x\in\fM_+ \oplus \fD_-\\ \norm{x}=1}} \fb[x,x]
  \end{equation*}
  for all $k \in \NN$ with $k \le \dim\Ran Q_+$.
\end{theorem}

The semiboundedness of $B$ in Theorem~\ref{thm:offdiagForm} forces $A$ to be semibounded as well, see the proof of
Theorem~\ref{thm:offdiagForm} below. In turn, it is natural to suppose that $A$ is semibounded and then to guarantee the
semiboundedness of $B$ via the well-known KLMN theorem by ensuring~\eqref{eq:formRelBound} with $b < 1$. In this regard,
Theorem~\ref{thm:offdiagForm} can be interpreted as a particular case of Theorem~\ref{thm:genSemibounded} with
$\Dom(\abs{A}^{1/2}) = \Dom(\abs{B}^{1/2})$, in which the remaining hypotheses are automatically satisfied due to the structure
of the perturbation. 

The rest of this note is organized as follows. In Section~\ref{sec:applications} we discuss applications of the main theorems and
revisit the Stokes operator as an example in the framework of Theorem~\ref{thm:offdiagForm}.
%
Section~\ref{sec:abstrMinimax} is devoted to an abstract minimax principle based on~\cite{GLS99}.
%
Two approaches are then used to verify the hypotheses of this abstract minimax principle, the~\emph{graph norm approach} and
the~\emph{block diagonalization approach}, respectively, which are discussed separately in Sections~\ref{sec:graphNorm}
and~\ref{sec:blockDiag} below.
%
Theorem~\ref{thm:genOpInfinitesimal} is proved in Section~\ref{sec:graphNorm}, which is based on the author's appendix
to~\cite{NSTTV18} and extends the corresponding considerations to certain unbounded perturbations $V$.
%
Theorems~\ref{thm:genSemibounded}--\ref{thm:offdiagForm} are proved in
Section~\ref{sec:blockDiag}, which builds upon recent developments on block diagonalization of operators and forms
from~\cite{MSS16} and~\cite{GKMSV17}, respectively.
%
Finally, Appendix~\ref{sec:heinz} provides some consequences of the well-known Heinz inequality that are used at various spots in
this note and are probably folklore.

%%%%%%%%%%%%%%%%%%%%%%%%%%%%%%%%%%%%%%%%%%%%%%%%%%%%%%%%%%%%%%%%%%%%%%%%%%%%%%%%%%%%%%%%%%%%%%%%%%%%%%%%%%%%%%%%%%%%%%%%%%%%%%%%%
%%%%%%%%%%%%%%%%%%%%%%%%%%%%%%%%%%%%%%%%%%%%%%%%%%%%%%%%%%%%%%%%%%%%%%%%%%%%%%%%%%%%%%%%%%%%%%%%%%%%%%%%%%%%%%%%%%%%%%%%%%%%%%%%%
%%% Applications
%%%%%%%%%%%%%%%%%%%%%%%%%%%%%%%%%%%%%%%%%%%%%%%%%%%%%%%%%%%%%%%%%%%%%%%%%%%%%%%%%%%%%%%%%%%%%%%%%%%%%%%%%%%%%%%%%%%%%%%%%%%%%%%%%
%%%%%%%%%%%%%%%%%%%%%%%%%%%%%%%%%%%%%%%%%%%%%%%%%%%%%%%%%%%%%%%%%%%%%%%%%%%%%%%%%%%%%%%%%%%%%%%%%%%%%%%%%%%%%%%%%%%%%%%%%%%%%%%%%
\section{Applications and examples}\label{sec:applications}

In this section, we use the main results from Section~\ref{sec:intro} to prove monotonicity and continuity properties of minimax
values in gaps of the essential spectrum in various situations and also revisit the well-known Stokes operator in the framework
of Theorem~\ref{thm:offdiagForm} as an example. We first consider the situation of indefinite or semidefinite bounded
perturbations, which has essentially been discussed in a slightly different form in~\cite{NSTTV18}.

For a bounded self-adjoint operator $V$ we define bounded nonnegative operators $V^{(p)}$ and $V^{(n)}$ with
$V = V^{(p)} - V^{(n)}$ via functional calculus by
\begin{equation}\label{eq:defVpVn}
  V^{(p)} := (1 + \sign(V))V / 2
  ,\quad
  V^{(n)} := (\sign(V) - 1)V / 2
  .
\end{equation}
We clearly have $\norm{ V^{(p)} } \le \norm{V}$ and $\norm{ V^{(n)} } \le \norm{V}$.

\begin{proposition}\label{prop:boundedPert}
  Let the finite interval $(c,d)$ belong to the resolvent set of the self-adjoint operator $A$, and let $V$ be a bounded
  self-adjoint operator on the same Hilbert space. Define $\cD_+ := \Ran \EE_A([d,\infty)) \cap \Dom(A)$ and
  $\cD_- := \Ran \EE_A((-\infty,c]) \cap \Dom(A)$.

  If $\norm{V^{(p)}} + \norm{V^{(n)}} < d - c$ with $V^{(p)}$ and $V^{(n)}$ as in~\eqref{eq:defVpVn}, then the interval
  $(c+\norm{V^{(p)}} , d-\norm{V^{(n)}})$ belongs to the resolvent set of the operator $A + V$, and one has
  $\dim\Ran \EE_A([d,\infty)) = \dim \Ran\EE_{A+V}([d-\norm{V^{(n)}},\infty))$ and
  \begin{equation*}
    \lambda_k\bigl ((A+V)|_{\Ran\EE_{A+V}([d-\norm{V^{(n)}},\infty))} \bigr)
    =
    \inf_{\substack{\fM_+ \subset \cD_+\\ \dim\fM_+ = k}} \sup_{\substack{x \in \fM_+ \oplus \cD_-\\ \norm{x}=1}}
      \scprod{ x , (A+V)x }
  \end{equation*}
  for all $k \in \NN$ with $k \le \dim \Ran\EE_{A+V}([d-\norm{V^{(n)}},\infty))$.
\end{proposition}

\begin{remark}
  A corresponding representation of the minimax values in terms of the form associated with $A+V$ as in
  Theorems~\ref{thm:genOpInfinitesimal}--\ref{thm:offdiagOp} holds here as well. However, for the sake of simplicity and since
  this is not needed in Corollaries~\ref{cor:monotonicity} and~\ref{cor:continuity} below, this has not been formulated in
  Proposition~\ref{prop:boundedPert}.
\end{remark}

The above proposition includes the particular cases where $V$ satisfies $\norm{V} < (d - c)/2$ and where $V$ is semidefinite with
$\norm{V} < d - c$, which essentially have been discussed in the proofs of Theorems~3.14 and~3.15 in~\cite{NSTTV18}; cf.~also the
discussion after Corollaries~\ref{cor:monotonicity} and~\ref{cor:continuity} below. However, Proposition~\ref{prop:boundedPert}
allows also certain indefinite perturbations $V$ with $(d - c)/2 \le \norm{V} < d - c$ that were not covered before.

We discuss here two proofs of Proposition~\ref{prop:boundedPert}, one based on Theorem~\ref{thm:genOpInfinitesimal} that is close
to the proofs of Theorems~3.14 and~3.15 in~\cite{NSTTV18} and the other one based on Theorem~\ref{thm:offdiagOp}. Both emphasize
different aspects on how to deal with the perturbation $V$.

\begin{proof}[Proof of Proposition~\ref{prop:boundedPert} based on Theorem~\ref{thm:genOpInfinitesimal}]
  By Proposition~2.1 in~\cite{Seel20}, the interval $(c + \norm{V^{(p)}} , d - \norm{V^{(n)}})$ belongs to the resolvent set of
  the operator $A + V$.

  Pick $\gamma \in (c + \norm{V^{(p)}} , d - \norm{V^{(n)}})$. We then have $\EE_{A-\gamma}((0,\infty)) = \EE_A([d,\infty))$ as
  well as $\EE_{A-\gamma}((-\infty,0]) = \EE_A((-\infty,c])$.

  For $x \in \cD_-$ we clearly have
  \begin{align*}
    \scprod{ x , (A+V-\gamma)x }
    &=
    \scprod{ x , (A-\gamma)x } + \scprod{ x , V^{(p)}x } - \scprod{ x , V^{(n)}x }\\
    &\le
    (c - \gamma + \norm{V^{(p)}}) \norm{x}^2
    <
    0
    .    
  \end{align*}
  Moreover, with $P_+ := \EE_A([d,\infty))$ and $Q_+ := \EE_{A+V}([d-\norm{V^{(n)}},\infty))$, the variant of the Davis-Kahan
  $\sin2\Theta$ theorem in~\cite[Theorem~1.1]{Seel20} implies that
  \begin{equation*}
  		\norm{ P_+ - Q_+ }
  		\le
  		\sin\Bigl( \frac{1}{2} \arcsin \frac{\norm{V^{(p)}} + \norm{V^{(n)}}}{d-c} \Bigr)
  		<
  		\frac{\sqrt{2}}{2}
  		<
  		1
    .
  \end{equation*}
	Taking into account that $\EE_{A+V-\gamma}((0,\infty)) = Q_+$ and
  \begin{equation*}
    \lambda_k((A+V-\gamma)|_{\Ran Q_+}) = \lambda_k((A+V)|_{\Ran Q_+}) - \gamma
  \end{equation*}
  for all $k \le \dim\Ran Q_+$, the claim now follows by applying Theorem~\ref{thm:genOpInfinitesimal}; cf.~also
  Remark~\ref{rem:PQbij}\,(1) below.
\end{proof}%

\begin{proof}[Proof of Proposition~\ref{prop:boundedPert} based on Theorem~\ref{thm:offdiagOp}]
  As in the first proof, the interval $(c+\norm{V^{(p)}} , d-\norm{V^{(n)}})$ belongs to the resolvent set of the operator $A+V$.
  Pick again $\gamma \in (c+\norm{V^{(p)}} , d-\norm{V^{(n)}})$.

  Let $A_+ := A|_{\Ran\EE_A([d,\infty))}$ and $A_- := A|_{\Ran\EE_A((-\infty,c])}$ denote the parts of $A$ associated with
  $\Ran \EE_{A}([d,\infty))$ and $\Ran \EE_A((-\infty,c])$, respectively. Moreover, for $\bullet \in \{p,n\}$, decompose
  $V^{(\bullet)}$ as
  \begin{equation*}
    V^{(\bullet)}
    =
    V_{\text{diag}}^{(\bullet)} + V_{\text{off}}^{(\bullet)}
    ,
  \end{equation*}
  where $V_{\text{diag}}^{(\bullet)} = V_+^{(\bullet)} \oplus V_-^{(\bullet)}$ is the diagonal part of $V^{(\bullet)}$ and
  $V_{\text{off}}^{(\bullet)}$ is the off-diagonal part of $V^{(\bullet)}$ with respect to
  $\Ran \EE_A([d,\infty)) \oplus \Ran \EE_A((-\infty,c])$. We clearly have $V_\pm^{(\bullet)} \ge 0$ and
  $\norm{V_\pm^{(\bullet)}} \le \norm{V^{(\bullet)}}$. Thus,
  \begin{equation*}
    A_- + V_-^{(p)} - V_-^{(n)}
    \le
    c + \norm{V^{(p)}}
    <
    \gamma
    <
    d - \norm{V^{(n)}}
    \le
    A_+ + V_+^{(p)} - V_+^{(n)}
    ,
  \end{equation*}
  so that
  \begin{equation*}
    \EE_A([d,\infty))
    =
    \EE_{A + V_{\text{diag}}^{(p)} - V_{\text{diag}}^{(n)}}((\gamma,\infty))
  \end{equation*}
  and
  \begin{equation*}
    \EE_A((-\infty,c])
    =
    \EE_{A + V_{\text{diag}}^{(p)} - V_{\text{diag}}^{(n)}}((-\infty,\gamma])
    ,
  \end{equation*}
  cf.~the proof of~\cite[Proposition~2.1]{Seel19}. Taking into account that $V_\text{off}^{(p)} - V_\text{off}^{(n)}$ is
  off-diagonal with respect to $\Ran \EE_A([d,\infty)) \oplus \Ran \EE_A((-\infty,c])$ and that
  $A+V = (A + V_{\text{diag}}^{(p)} - V_{\text{diag}}^{(n)}) + V_\text{off}^{(p)} - V_\text{off}^{(n)}$, the claim now follows
  from Theorem~\ref{thm:offdiagOp} via a spectral shift by $\gamma$ as in the first proof.
\end{proof}%

As corollaries to Proposition~\ref{prop:boundedPert}, we obtain the following monotonicity and continuity statements for the
minimax values in gaps of the essential spectrum.

\begin{corollary}[{cf.~\cite[Theorem~3.15\,(2) and Theorem~3.14]{NSTTV18}}]\label{cor:monotonicity}
  Let $A$ be as in Proposition~\ref{prop:boundedPert}, and let $V_0$ and $V_1$ be bounded self-adjoint operators on the same
  Hilbert space satisfying $\max\{ \norm{V_0^{(p)}} + \norm{V_0^{(n)}} , \norm{V_1^{(p)}} + \norm{V_1^{(n)}} \} < d - c$.

  If, in addition, $V_0 \le V_1$, then
  \begin{equation*}
    \lambda_k\bigl( (A+V_0)|_{\Ran\EE_{A+V_0}([d-\norm{V_0^{(n)}},\infty))} \bigr)
    \le
    \lambda_k\bigl( (A+V_1)|_{\Ran\EE_{A+V_1}([d-\norm{V_1}^{(n)},\infty))} \bigr)
  \end{equation*}
  for $k \le \dim\Ran\EE_A([d,\infty)) = \dim\Ran\EE_{A+V_j}([d-\norm{V_j^{(n)}},\infty))$, $j\in\{0,1\}$.
\end{corollary}

\begin{corollary}\label{cor:continuity}
  Let $A$ and $V$ be as in Proposition~\ref{prop:boundedPert}. Then, the interval $(c+\norm{V^{(p)}} , d-\norm{V^{(n)}})$ belongs
  to the resolvent set of every $A+tV$, $t \in [0,1]$, and for each
  $k \le \dim \Ran \EE_A([d,\infty)) = \dim \Ran \EE_{A+tV}([d-t\norm{V^{(n)}},\infty))$, $t \in [0,1]$, the mapping
  \begin{equation*}
    [0,1]
    \ni
    t
    \mapsto
    \lambda_k((A+tV)|_{\Ran\EE_{A+tV}([d-t\norm{V^{(n)}},\infty))})
  \end{equation*}
  is Lipschitz continuous with Lipschitz constant $\norm{V}$.
\end{corollary}

\begin{proof}
  Taking into account that
  \begin{equation*}
    \scprod{ x , (A+sV)x } - \abs{t-s}\norm{V}
    \le
    \scprod{ x , (A+tV)x }
    \le
    \scprod{ x , (A+sV)x } + \abs{t-s}\norm{V}
  \end{equation*}
  for all $x \in \Dom(A)$, the claim follows immediately from Proposition~\ref{prop:boundedPert}.
\end{proof}%

It should again be mentioned that the above statements include the particular cases where the norm of the perturbations is less
than $(d - c)/2$ or where the perturbations are semidefinite with a norm less than $d - c$. These cases have essentially been
discussed in~\cite{NSTTV18}. There, especially lower bounds on the movement of eigenvalues in gaps of the essential spectrum
under certain conditions and the behaviour of edges of the essential spectrum have been studied. However, since this is not the
main focus of the present note, this is not pursued further here.

The second proof of Proposition~\ref{prop:boundedPert} discussed above is flexible enough to handle also unbounded
perturbations that are small enough in a certain sense, at least in the semibounded setting. This is demonstrated in the
following result for the case where $A$ is lower semibounded.

\begin{proposition}\label{prop:unboundedPert}
  Let $A$, $(c,d)$, and $\cD_\pm$ be as in Proposition~\ref{prop:boundedPert}, and suppose, in addition, that $A$ is lower
  semibounded. Let $V$ be a symmetric operator that is $A$-bounded with $A$-bound smaller than $1$. Moreover, suppose that there
  are constants $a, b \ge 0$, $b < 1$, with
  \begin{equation*}
    \abs{ \scprod{ x , Vx } }
    \le
    a \norm{x}^2 + b \scprod{ x , Ax }
    \quad\text{ for all }\
    x \in \Dom(A)
  \end{equation*}
  and
  \begin{equation}\label{eq:unboundedPert}
    2a + b(c+d) < d - c.
  \end{equation}
  Then, the interval $(a+(1+b)c , (1-b)d-a)$ belongs to the resolvent set of $A+V$, and one has
  $\dim \Ran \EE_A( [d,\infty) ) = \dim \EE_{A+V}( [(1-b)d-a,\infty) )$ and
  \begin{equation*}
    \lambda_k((A+V)|_{\Ran\EE_{A+V}([(1-b)d-a,\infty))})
    =
    \inf_{\substack{\fM_+ \subset \cD_+\\ \dim\fM_+ = k}} \sup_{\substack{x\in\fM_+\oplus\cD_-\\ \norm{x}=1}}
      \scprod{ x , (A+V)x }
  \end{equation*}
  for all $k \in \NN$ with $k \le \dim\Ran \EE_{A+V}([(1-b)d-a,\infty))$.
\end{proposition}

\begin{proof}
  For $x \in \Dom(A) = \Dom(A+V)$, we clearly have
  \begin{equation*}
    (1-b)\scprod{ x , Ax } - a\norm{x}^2
    \le
    \scprod{ x , (A+V)x }
    \le
    (1+b)\scprod{ x , Ax } + a\norm{x}^2
    .
  \end{equation*}
  According to~\eqref{eq:unboundedPert}, we may pick $\gamma \in \RR$ satisfying the two-sided inequality
  $a + (1+b)c < \gamma < (1-b)d - a$. Let $A_\pm$ be as in the second proof of Proposition~\ref{prop:boundedPert}, and again
  decompose the perturbation $V$ as $V = V_\text{diag} + V_\text{off}$ with diagonal part $V_\text{diag} = V_+ \oplus V_-$
  and off-diagonal part $V_\text{off}$. The above then gives
  \begin{equation*}
    A_- + V_-
    \le
    a + (1+b)c
    <
    \gamma
    <
    (1-b)d - a
    \le
    A_+ + V_+
    ,
  \end{equation*}
  so that $\EE_A([d,\infty)) = \EE_{A+V_\text{diag}}((\gamma,\infty)) = \EE_{A+V_\text{diag}}([(1-b)d-a,\infty))$ and
  $\EE_A((-\infty,c]) = \EE_{A+V_\text{diag}}((-\infty,\gamma])$, and the interval $(a + (1+b)c , (1-b)d - a)$ belongs to the
  resolvent set of $A+V_\text{diag}$. By~\cite[Theorem~1]{MS06} (cf.~also~\cite[Theorem~2.1]{AL95}), the interval
  $(a + (1+b)c , (1-b)d - a)$ then belongs also to the resolvent set of $A + V = (A+V_\text{diag}) + V_\text{off}$. The rest of
  the claim is now proved as in the second proof of Proposition~\ref{prop:boundedPert} via Theorem~\ref{thm:offdiagOp} and a
  spectral shift by $\gamma$.
\end{proof}%

\begin{remark}\label{rem:unboundedPert}
  (1)
  If $A$ is lower semibounded and the symmetric operator $V$ is $A$-bounded with $A$-bound smaller than $1$, then constants
  $a, b \ge 0$, $b < 1$, with
  \begin{equation*}
    \abs{ \scprod{ x , Vx } }
    \le
    a\norm{x}^2 + b \scprod{ x , Ax }
    \quad\text{ for all }\
    x \in \Dom(A)
  \end{equation*}
  exist by~\cite[Theorem~VI.1.38]{Kato95}. Condition~\eqref{eq:unboundedPert} can then be guaranteed for $tV$ instead of $V$ for
  $t \in \RR$ with sufficiently small modulus.

  (2)
  A similar result as in Proposition~\ref{prop:unboundedPert} holds also if instead $A$ is upper semibounded. In this case, one
  requires constants $a, b \ge 0$, $b < 1$, satisfying $\abs{ \scprod{ x , Vx } } \le a\norm{x}^2 - b \scprod{ x , Ax }$ for all
  $x \in \Dom(A)$ and $2a - b(c+d) < d-c$. We then get in a completely analogous way a representation for the minimax values of
  $(A+V)|_{\Ran\EE_{A+V}([(1+b)d-a))}$.
\end{remark}

As another consequence of Theorem~\ref{thm:offdiagOp}, we obtain the following lower bound for the minimax values in the setting
of off-diagonal operator perturbations.
\begin{corollary}\label{cor:lowerboundOffOp}
  In the situation of Theorem~\ref{thm:offdiagOp}, we have
  \begin{equation*}
    \lambda_k(A|_{\Ran P_+})
    \le
    \lambda_k(B|_{\Ran Q_+})
  \end{equation*}
  for all $k \le \dim\Ran P_+ = \dim \Ran Q_+$.
\end{corollary}

\begin{proof}
  Let $\fM_+ \subset \cD_+$ with $\dim\fM_+ = k$. Since $\scprod{ x , Vx } = 0$ for all $x \in \cD_+$ by hypothesis, we have
  \begin{equation*}
    \sup_{\substack{x \in \fM_+\\ \norm{x}=1}} \scprod{ x , Ax }
    =
    \sup_{\substack{x \in \fM_+\\ \norm{x}=1}} \scprod{ x , (A+V)x }
    \le
    \sup_{\substack{x \in \fM_+ \oplus \cD_-\\ \norm{x}=1}} \scprod{ x , (A+V)x }
    .
  \end{equation*}
  Taking the infimum over all such subspaces $\fM_+$ proves the claim by Theorem~\ref{thm:offdiagOp}.
\end{proof}%

As in Corollary~\ref{cor:continuity}, we also obtain a continuity statement in the situation of Theorem~\ref{thm:offdiagOp} with
bounded off-diagonal perturbations. Here, however, we do not have to impose any condition on the norm of the perturbation.

\begin{corollary}\label{cor:continuityOffOp}
  Let $A$ and $V$ be as in Theorem~\ref{thm:offdiagOp}, and suppose that $V$ is bounded. Then, for each
  $k \le \dim \Ran\EE_A((0,\infty)) = \dim \Ran\EE_{A+tV}((0,\infty))$, $t \in \RR$, the mapping
  \begin{equation*}
    \RR
    \ni
    t
    \mapsto
    \lambda_k\bigl( (A+tV)|_{\Ran\EE_{A+tV}((0,\infty))} \bigr)
  \end{equation*}
  is Lipschitz continuous with Lipschitz constant $\norm{V}$.
\end{corollary}

In the particular case where $B$ is semibounded, Theorem~\ref{thm:offdiagForm} allows us to extend
Corollaries~\ref{cor:lowerboundOffOp} and~\ref{cor:continuityOffOp} to some degree to off-diagonal form perturbations. Recall
here, that semiboundedness of $B$ implies that also $A$ is semibounded, see the proof of Theorem~\ref{thm:offdiagForm} below.

\begin{corollary}\label{cor:offForm}
  Assume the hypotheses of Theorem~\ref{thm:offdiagForm}.
  \begin{enumerate}
    \renewcommand{\theenumi}{\alph{enumi}}

    \item For each $k \in \NN$ with $k \le \dim\Ran P_+ = \dim\Ran Q_+$ one has
          $\lambda_k(A|_{\Ran P_+}) \le \lambda_k(B|_{\Ran Q_+})$.

    \item Denote for $t \in (-1/b , 1/b)$ by $B_t$ the self-adjoint operator associated with the form $\fb_t := \fa + t\fv$ with
          form domain $\Dom[\fb_t] := \Dom[\fa]$. Then, for each
          $k \le \dim\Ran\EE_A((0,\infty)) = \dim\Ran\EE_{B_t}((0,\infty))$, the mapping
          \begin{equation*}
            ( -1/b , 1/b )
            \ni
            t
            \mapsto
            \lambda_k(B_t|_{\Ran\EE_{B_t}((0,\infty))})
          \end{equation*}
          is locally Lipschitz continuous.

  \end{enumerate}
\end{corollary}

\begin{proof}
  (a).
  Taking into account that $\fv[ x , x ] = 0$ for all $x \in \fD_+$ by hypothesis, the inequality
  $\lambda_k(A|_{\Ran P_+}) \le \lambda_k(B|_{\Ran Q_+})$ is proved by means of Theorem~\ref{thm:offdiagForm} in a way analogous
  to Corollary~\ref{cor:lowerboundOffOp}.

  (b).
  Upon a rescaling, we may assume without loss of generality that $b < 1$. Also recall that each $B_t$ is indeed a semibounded
  self-adjoint operator with $\Dom[\fb_t] = \Dom(\abs{B_t}^{1/2})$ by the well-known KLMN theorem, and note that each $t\fv$
  satisfies the hypotheses of Theorem~\ref{thm:offdiagForm}.
  %
  Pick $t,s \in (-1/b , 1/b)$ with $b\abs{t-s} \le 1-b\abs{s}$.
  
  Consider first the case where $A$ (and hence $\fa$) is lower semibounded with lower bound $m \in \RR$. We then have
  $\abs{\fa[x,x]} \le \fa[x,x] + (\abs{m}-m)\norm{x}^2$ for all $x \in \Dom[\fa]$. With $\tilde{a} := a + \abs{m} - m$, this
  gives
  \begin{equation*}
    \abs{\fv[x,x]}
    \le
    \tilde{a}\norm{x}^2 + b\fa[x,x]
    \le
    \tilde{a}\norm{x}^2 + b\fb_s[x,x] + b\abs{s}\abs{\fv[x,x]}
  \end{equation*}
  and, hence,
  \begin{equation*}
    \abs{\fv[x,x]}
    \le
    \frac{\tilde{a}}{1-b\abs{s}}\norm{x}^2 + \frac{b}{1-b\abs{s}}\fb_s[x,x]
  \end{equation*}
  for all $x \in \Dom[\fa] = \Dom[\fb_s]$. Since $\fb_t = \fb_s + (t-s)\fv$, we thus obtain
  \begin{equation*}
    -\frac{\tilde{a}\abs{t-s}}{1-b\abs{s}} + \Bigl( 1 - \frac{b\abs{t-s}}{1-b\abs{s}} \Bigr) \fb_s
    \le
    \fb_t
    \le
    \frac{\tilde{a}\abs{t-s}}{1-b\abs{s}} + \Bigl( 1 + \frac{b\abs{t-s}}{1-b\abs{s}} \Bigr) \fb_s
    .
  \end{equation*}
  Abbreviating $\lambda_k(t):=\lambda_k(B_t|_{\Ran\EE_{B_t}((0,\infty))})$, Theorem~\ref{thm:offdiagForm} then implies that
  \begin{equation*}
    -\frac{\tilde{a}\abs{t-s}}{1-b\abs{s}} + \Bigl( 1 - \frac{b\abs{t-s}}{1-b\abs{s}} \Bigr) \lambda_k(s)
    \le
    \lambda_k(t)
    \le
    \frac{\tilde{a}\abs{t-s}}{1-b\abs{s}} + \Bigl( 1 + \frac{b\abs{t-s}}{1-b\abs{s}} \Bigr) \lambda_k(s)
  \end{equation*}
  and, therefore,
  \begin{equation}\label{eq:Lipschitz}
    \abs{\lambda_k(t) - \lambda_k(s)}
    \le
    \frac{\tilde{a}\abs{t-s}}{1-b\abs{s}} + \frac{b\abs{t-s}}{1-b\abs{s}} \abs{\lambda_k(s)}
    .
  \end{equation}
  This proves that $t \mapsto \lambda_k(t)$ is continuous on $(-1/b,1/b)$ and, in particular, bounded on every compact
  subinterval of $(-1/b , 1/b)$. In turn, it then easily follows from~\eqref{eq:Lipschitz} that this mapping is even locally
  Lipschitz continuous, which concludes the case where $A$ is lower semibounded.

  If $A$ is upper semibounded with upper bound $m \in \RR$, we proceed similarly. We then have
  $\abs{\fa[x,x]} \le -\fa[x,x] + (m+\abs{m})\norm{x}^2$ for all $x \in \Dom[\fa]$. With $\tilde{a} := a + m + \abs{m}$, this
  leads to
  \begin{equation*}
    \abs{\fv[x,x]}
    \le
    \frac{\tilde{a}}{1-b\abs{s}}\norm{x}^2 - \frac{b}{1-b\abs{s}}\fb_s[x,x]
  \end{equation*}
  for all $x \in \Dom[\fa] = \Dom[\fb_s]$. Analogously as above, we then eventually obtain again~\eqref{eq:Lipschitz}, which
  proves the claim in the case where $A$ is upper semibounded. This completes the proof.
\end{proof}%

\begin{remark}\label{rem:offForm}
  (1)
  In part~(a) of Corollary~\ref{cor:offForm}, one can also give an upper bound for $\lambda_k(B|_{\Ran Q_+})$ in terms of the
  form bounds of $\fv$: If $A$ is lower semibounded with lower bound $m \in \RR$, then
  \begin{align*}
    \abs{ \fv[ x , x ] }
    &\le
    (a + b\abs{m})\norm{x}^2 + b(\fa-m)[ x , x ]\\
    &=
    (a + b\abs{m} - bm)\norm{x}^2 + b\fa[ x , x ]
  \end{align*}
  for all $x \in \Dom[\fa]$, leading to
  \begin{equation*}
    \lambda_k(B|_{\Ran Q_+})
    \le
    (1+b)\lambda_k(A|_{\Ran P_+}) + (a + b\abs{m} - bm)
  \end{equation*}
  for all $k \le \dim\Ran P_+ = \dim\Ran Q_+$. Similarly, if $A$ is upper semibounded with upper bound $m \in \RR$, we have
  \begin{align*}
    \abs{ \fv[ x , x ] }
    &\le
    (a + b\abs{m})\norm{x}^2 + b(m-\fa)[ x , x ]\\
    &=
    (a + b\abs{m} + bm)\norm{x}^2 - b\fa[ x , x ]
  \end{align*}
  for all $x \in \Dom[\fa]$. If, in addition, $b \le 1$, this then leads to
  \begin{equation*}
    \lambda_k(B|_{\Ran Q_+})
    \le
    (1-b)\lambda_k(A|_{\Ran P_+}) + (a + b\abs{m} + bm)
  \end{equation*}
  for all $k \le \dim\Ran P_+ = \dim\Ran Q_+$.

  (2)
  A similar continuity result as in part~(b) of Corollary~\ref{cor:offForm} can be formulated also in the framework of
  Proposition~\ref{prop:unboundedPert}: in addition to the hypotheses of Proposition~\ref{prop:unboundedPert}, let
  $I \subset (-1/b,1/b)$ be an interval such that for all $t \in I$ we have $2a\abs{t} + b\abs{t}(c+d) < d-c$. Then, for all
  $k \in \NN$ satisfying $k \le \dim \Ran \EE_A([d,\infty)) = \dim \EE_{A+tV}([(1-b\abs{t})d-a\abs{t},\infty))$, the mapping
  \begin{equation*}
    I \ni t
    \mapsto
    \lambda_k((A+tV)|_{\Ran\EE_{A+tV}([(1-b\abs{t})d-a\abs{t},\infty))})
  \end{equation*}
  is locally Lipschitz continuous. The proof is analogous to the one of part~(b) of Corollary~\ref{cor:offForm}.

  A corresponding result can be formulated also in the framework of Theorem~\ref{thm:genSemibounded}, provided that the interval
  $I \subset (-1/b,1/b)$ is then chosen such that for all $t \in I$ we have $\norm{\EE_A((0,\infty))-\EE_{A+tV}((0,\infty))} < 1$
  and $\scprod{ x , (A+tV)x } \le 0$ for all $x \in \cD_-$.
\end{remark}

%%%%%%%%%%%%%%%%%%%%%%%%%%%%%%%%%%%%%%%%%%%%%%%%%%%%%%%%%%%%%%%%%%%%%%%%%%%%%%%%%%%%%%%%%%%%%%%%%%%%%%%%%%%%%%%%%%%%%%%%%%%%%%%%%
%%% Example: Stokes operator
%%%%%%%%%%%%%%%%%%%%%%%%%%%%%%%%%%%%%%%%%%%%%%%%%%%%%%%%%%%%%%%%%%%%%%%%%%%%%%%%%%%%%%%%%%%%%%%%%%%%%%%%%%%%%%%%%%%%%%%%%%%%%%%%%
\subsection*{An example. The Stokes operator}
We now briefly revisit the Stokes operator in the framework of Theorem~\ref{thm:offdiagForm}. Here, we mainly rely
on~\cite{GKMSV19}, but the reader is referred also to~\cite[Section~7]{GKMSV17},~\cite[Chapter~5]{SchmDiss},
~\cite{FFMM00}, and the references cited therein.

Let $\Omega \subset \RR^n$, $n \ge 2$, be a bounded domain with $C^2$-boundary, and let $\nu > 0$ and $v_* \ge 0$. On the Hilbert
space $\cH = \cH_+ \oplus \cH_-$ with $\cH_+ = L^2(\Omega)^n$ and $\cH_- = L^2(\Omega)$, we consider the closed, densely defined,
and nonnegative form $\fa$ with $\Dom[\fa] := H_0^1(\Omega)^n \oplus L^2(\Omega)$ and
\begin{equation*}
  \fa[ v \oplus q , u \oplus p ]
  :=
  \nu \sum_{j=1}^n \int_\Omega \scprod{ \partial_j v(x) , \partial_j u(x) }_{\CC^n} \,\dd x
\end{equation*}
for $u \oplus p, v \oplus q \in \Dom[\fa]$. Clearly, $\fa$ is the form associated to the nonnegative self-adjoint operator
$A := -\nu\mathbf{\Delta} \oplus 0$ on the Hilbert space $\cH = \cH_+ \oplus \cH_-$ with
$\Dom(A) := (H^2(\Omega) \cap H_0^1(\Omega))^n \oplus L^2(\Omega)$ and $\Dom(\abs{A}^{1/2}) = \Dom[\fa]$, where
$\mathbf{\Delta} = \Delta\cdot I_{\CC^n}$ is the vector-valued Dirichlet Laplacian on $\Omega$. Moreover,
$P_+ := \EE_A((0,\infty))$ and $P_- := \EE_A((-\infty,0]) = \EE_A(\{0\})$ are the orthogonal projections onto $\cH_+$ and
$\cH_-$, respectively. In particular, we have
\begin{equation*}
  \fD_+
  :=
  \Ran P_+ \cap \Dom(\abs{A}^{1/2})
  =
  H_0^1(\Omega)^n \oplus 0
\end{equation*}
and
\begin{equation*}
  \fD_-
  :=
  \Ran P_- \cap \Dom(\abs{A}^{1/2})
  =
  0 \oplus L^2(\Omega)
  .
\end{equation*}

Define the symmetric sesquilinear form $\fv$ on $\cH = \cH_+ \oplus \cH_-$ with domain $\Dom[\fv] := \Dom[\fa]$ by
\begin{equation*}
  \fv[ v \oplus q , u \oplus p ]
  :=
  -v_* \scprod{ \divgc v , p }_{L^2(\Omega)} - v_* \scprod{ q , \divgc u }_{L^2(\Omega)}
\end{equation*}
for $u \oplus p, v \oplus q \in \Dom[\fa]$. One can show that
$\nu\norm{\divgc u}_{L^2(\Omega)}^2 \le \fa[u \oplus 0 , u \oplus 0]$ for all $u \in \fD_+ = H_0^1(\Omega)^n$, see,
e.g.,~\cite[Proof of Theorem~5.12]{SchmDiss}. Using Young's inequality, this then implies that $\fv$ is infinitesimally form
bounded with respect to $\fa$, see~\cite[Remark~5.1.3]{SchmDiss}; cf.~also~\cite[Section~2]{GKMSV19}. Indeed, for $\eps > 0$ and
$f = u \oplus p \in \Dom[\fa]$ we obtain
\begin{equation}\label{eq:StokesRelBound}
 \begin{aligned}
  \abs{\fv[f , f]}
  &\le
  2 v_* \abs{ \scprod{ p , \divgc u }_{L^2(\Omega)} }
    \le
    2 v_* \norm{p}_{L^2(\Omega)} \norm{\divgc u}_{L^2(\Omega)}\\
  &\le
  \eps \nu \norm{\divgc u}_{L^2(\Omega)}^2 + \eps^{-1} \nu^{-1} v_*^2 \norm{p}_{L^2(\Omega)}^2\\
  &\le
  \eps \fa[ u \oplus 0 , u \oplus 0 ] + \eps^{-1} \nu^{-1} v_*^2 \norm{f}_\cH^2\\
  &=
  \eps \fa[ f , f ] + \eps^{-1} \nu^{-1} v_*^2 \norm{f}_\cH^2
  .
 \end{aligned}
\end{equation}
Thus, by the well-known KLMN theorem, the form $\fb_S := \fa + \fv$ with $\Dom[\fb_S] = \Dom[\fa] = \Dom(\abs{A}^{1/2})$ is
associated to a unique lower semibounded self-adjoint operator $B_S$ on $\cH$ with $\Dom(\abs{B_S}^{1/2}) = \Dom(\abs{A}^{1/2})$,
the so-called~\emph{Stokes operator}. It is a self-adjoint extension of the (non-closed) upper dominant block operator matrix
\begin{equation*}
  \begin{pmatrix}
    -\nu\mathbf{\Delta} & v_*\grad\\
    -v_*\divgc & 0
  \end{pmatrix}
\end{equation*}
defined on $(H^2(\Omega) \cap H_0^1(\Omega))^n \oplus H^1(\Omega)$. In fact, the closure of the latter is a self-adjoint
operator, see~\cite[Theorems~3.7 and~3.9]{FFMM00}, which yields another characterization of the Stokes operator $B_S$.

By rescaling, one obtains from~\cite[Theorem~3.15]{FFMM00} that the essential spectrum of $B_S$ is given by
\begin{equation*}
  \spec_\ess(B_S)
  =
  \Bigl\{ -\frac{v_*^2}{\nu} , -\frac{v_*^2}{2\nu}  \Bigr\}
  ,
\end{equation*}
see~\cite[Remark~2.2]{GKMSV19}. In particular, the essential spectrum of $B_S$ is purely negative. In turn, the positive spectrum
of $B_S$, that is, $\spec(B_S) \cap (0,\infty)$, is discrete~\cite[Theorem~2.1\,(i)]{GKMSV19}.

The above shows that the hypotheses of Theorem~\ref{thm:offdiagForm} are satisfied in this situation, so that we obtain from
Theorem~\ref{thm:offdiagForm} and Corollary~\ref{cor:offForm} the following result.

\begin{proposition}\label{prop:Stokes}
  Let $B_S$ be the Stokes operator as above. Then, the positive spectrum of $B_S$, $\spec(B_S) \cap (0,\infty)$, is discrete, and
  the positive eigenvalues $\lambda_k(B_S|_{\Ran\EE_{B_S}((0,\infty))})$, $k \in \NN$, of $B_S$, enumerated in nondecreasing
  order and counting multiplicities, admit the representation
  \begin{equation*}
    \lambda_k(B_S|_{\Ran\EE_{B_S}((0,\infty))})
    =
    \inf_{\substack{\fM_+ \subset H_0^1(\Omega)^n\\ \dim\fM_+ = k}}
      \sup_{\substack{u \oplus p \in \fM_+ \oplus L^2(\Omega)\\ \norm{u}_{L^2(\Omega)^n}^2 + \norm{p}_{L^2(\Omega)}^2 = 1}}
      \fb_S[ u \oplus p , u \oplus p ]
    .
  \end{equation*}
  The latter depend locally Lipschitz continuously on $\nu$ and $v_*$ and satisfy the two-sided estimate
  \begin{equation*}
    \nu\lambda_k(-\mathbf{\Delta})
    \le
    \lambda_k(B_S|_{\Ran\EE_{B_S}((0,\infty))})
    \le
    \nu\lambda_k(-\mathbf{\Delta}) + \frac{v_*^2}{\nu}
    .
  \end{equation*}
\end{proposition}

\begin{proof}
  In view the above considerations, the representation of the eigenvalues, the continuity statement, and the lower bound on the
  eigenvalues follow from Theorem~\ref{thm:offdiagForm} and Corollary~\ref{cor:offForm}. It remains to show the upper bound on
  the eigenvalues. To this end, let $\fM_+ \subset H_0^1(\Omega)^n$ with $\dim \fM_+ = k \in \NN$, and let
  $f = u \oplus p \in H_0^1(\Omega)^n \oplus L^2(\Omega)$ be a normalized vector with $u \neq 0$. Then,
  $\mu := \fa[ u \oplus 0 , u \oplus 0 ] / \norm{u}_{L^2(\Omega)^n}^2 = \fa[ f , f ] / \norm{u}_{L^2(\Omega)^n}^2$ is positive
  and satisfies
  \begin{equation}\label{eq:muBound}
    \mu
    \le
    \sup_{\substack{v \in \fM_+\\ \norm{v}_{L^2(\Omega)^n}^2 = 1}} \fa[ v \oplus 0 , v \oplus 0 ]
  \end{equation}
  and
  \begin{equation*}
    \frac{\nu\norm{\divgc u}_{L^2(\Omega)}^2}{\mu}
    =
    \frac{\norm{u}_{L^2(\Omega)^n}^2 \nu \norm{\divgc u}_{L^2(\Omega)}^2}{\fa[ u \oplus 0 , u \oplus 0 ]}
    \le
    \norm{u}_{L^2(\Omega)^n}^2
    \le
    1
    .
  \end{equation*}
  Similarly as in~\eqref{eq:StokesRelBound}, we now obtain by means of Young's inequality that
  \begin{align*}
    \abs{ \fv[ f , f ] }
    &\le
    2v_*\norm{p}_{L^2(\Omega)} \norm{\divgc u}_{L^2(\Omega)}
      \le
      \mu \norm{p}_{L^2(\Omega)}^2 + \frac{v_*^2 \norm{\divgc u}_{L^2(\Omega)}^2}{\mu}\\
    &\le
    \mu \norm{p}_{L^2(\Omega)}^2 + \frac{v_*^2}{\nu}
    .
  \end{align*}
  Since $\fa[ f , f ] = \mu\norm{u}_{L^2(\Omega)^n}^2$, this gives
  \begin{equation}\label{eq:StokesFormUpperBound}
    \fb_S[ f , f ]
    \le
    \fa[ f , f ] + \mu \norm{p}_{L^2(\Omega)}^2 + \frac{v_*^2}{\nu}
    =
    \mu + \frac{v_*^2}{\nu}
    .
  \end{equation}
  In light of $\fb_S[ 0 \oplus p , 0 \oplus p ] = \fa[ 0 \oplus p , 0 \oplus p ] = 0$, we conclude from~\eqref{eq:muBound}
  and~\eqref{eq:StokesFormUpperBound} that
  \begin{equation*}
    \sup_{\substack{u \oplus p \in \fM_+ \oplus L^2(\Omega)\\ \norm{u}_{L^2(\Omega)^2}^2 + \norm{p}_{L^2(\Omega)}^2 = 1}}
      \fb_S[ u \oplus p , u \oplus p ]
    \le
    \sup_{\substack{v \in \fM_+\\ \norm{v}_{L^2(\Omega)^n}^2 = 1}} \fa[ v \oplus 0 , v \oplus 0 ] + \frac{v_*^2}{\nu}
    ,
  \end{equation*}
  and taking the infimum over subspaces $\fM_+ \subset H_0^1(\Omega)^n$ with $\dim\fM_+ = k$ proves the upper bound. This
  completes the proof.
\end{proof}%

\begin{remark}
  (1)
  Choosing $\eps = 1$ in~\eqref{eq:StokesRelBound}, the upper bound from Remark~\ref{rem:offForm}\,(1) reads
  \begin{equation*}
    \lambda_k(B_S|_{\Ran\EE_{B_S}((0,\infty))})
    \le
    2\nu\lambda_k(-\mathbf{\Delta}) + \frac{v_*^2}{\nu}
  \end{equation*}
  for all $k \in \NN$, while the choice $\eps = v_*$ in~\eqref{eq:StokesRelBound} leads to
  \begin{equation*}
    \lambda_k(B_S|_{\Ran\EE_{B_S}((0,\infty))})
    \le
    (1+v_*)\nu\lambda_k(-\mathbf{\Delta}) + \frac{v_*}{\nu}
  \end{equation*}
  for all $k \in \NN$.

  (2)
  For the particular case of $k = 1$, a similar upper bound has been established in the proof
  of~\cite[Theorem~2.1\,(i)]{GKMSV19}:
  \begin{equation*}
    \nu\lambda_1(-\Delta)
    \le
    \lambda_1(B_S|_{\Ran\EE_{B_S}((0,\infty))})
    \le
    \nu\lambda_1(-\Delta) + v_*\norm{\divgc u_0}_{L^2(\Omega)}
    ,
  \end{equation*}
  where $u_0 \in (H^2(\Omega) \cap H_0^1(\Omega))^n$ is a normalized eigenfunction for $-\mathbf{\Delta}$ corresponding to the
  first positive eigenvalue $\lambda_1(-\mathbf{\Delta}) = \lambda_1(-\Delta)$.
\end{remark}

%%%%%%%%%%%%%%%%%%%%%%%%%%%%%%%%%%%%%%%%%%%%%%%%%%%%%%%%%%%%%%%%%%%%%%%%%%%%%%%%%%%%%%%%%%%%%%%%%%%%%%%%%%%%%%%%%%%%%%%%%%%%%%%%%
%%%%%%%%%%%%%%%%%%%%%%%%%%%%%%%%%%%%%%%%%%%%%%%%%%%%%%%%%%%%%%%%%%%%%%%%%%%%%%%%%%%%%%%%%%%%%%%%%%%%%%%%%%%%%%%%%%%%%%%%%%%%%%%%%
%%% Abstract minimax principle
%%%%%%%%%%%%%%%%%%%%%%%%%%%%%%%%%%%%%%%%%%%%%%%%%%%%%%%%%%%%%%%%%%%%%%%%%%%%%%%%%%%%%%%%%%%%%%%%%%%%%%%%%%%%%%%%%%%%%%%%%%%%%%%%%
%%%%%%%%%%%%%%%%%%%%%%%%%%%%%%%%%%%%%%%%%%%%%%%%%%%%%%%%%%%%%%%%%%%%%%%%%%%%%%%%%%%%%%%%%%%%%%%%%%%%%%%%%%%%%%%%%%%%%%%%%%%%%%%%%
\section{An abstract minimax principle in spectral gaps}\label{sec:abstrMinimax}

We rely on the following abstract minimax principle in spectral gaps, part~(a) of which is extracted from~\cite{GLS99} and
part~(b) of which is its natural adaptation to the operator framework; cf.~also~\cite[Proposition~A.3]{NSTTV18}.

\begin{proposition}[{cf.~\cite[Theorem 1]{GLS99},~\cite[Proposition~A.3]{NSTTV18}}]\label{prop:GLS}
  Assume Hypothesis \ref{hyp:minimax}.
  \begin{enumerate}
    \renewcommand{\theenumi}{\alph{enumi}}

    \item
    If we have $\Dom(\abs{B}^{1/2}) = \Dom(\abs{A}^{1/2})$, $\fb[x,x]\le0$ for all $x\in\fD_-$, and
    $\Ran(P_+Q_+|_{\fD_+}) \supset \fD_+$, then
    \begin{equation*}
      \lambda_k(B|_{\Ran Q_+})
      =
      \inf_{\substack{\fM_+\subset\fD_+\\ \dim(\fM_+)=k}} \sup_{\substack{x\in\fM_+\oplus\fD_-\\ \norm{x}=1}} \fb[x,x]
    \end{equation*}
    for all $k \in \NN$ with $k \le \dim\Ran P_+$.

    \item
    If we have $\Dom(B) = \Dom(A)$, $\scprod{ x , Bx } \le 0$ for all $x \in \cD_-$, and $\Ran(P_+Q_+|_{\cD_+}) \supset \cD_+$,
    then
    \begin{equation*}
      \lambda_k(B|_{\Ran Q_+})
      =
      \inf_{\substack{\fM_+\subset\cD_+\\ \dim(\fM_+)=k}} \sup_{\substack{x\in\fM_+\oplus\cD_-\\ \norm{x}=1}}
        \scprod{ x , Bx }
    \end{equation*}
    for all $k \in \NN$ with $k \le \dim\Ran P_+$.

  \end{enumerate}
\end{proposition}

\begin{proof}
  For part~(a), we first recall that the spectral projections $P_+$ and $Q_+$ map
  $\fD := \Dom(\abs{A}^{1/2}) = \Dom(\abs{B}^{1/2})$ into itself, so that $P_+$ maps $\Ran Q_+ \cap \fD$ into $\fD_+$. Next, we
  observe that under the hypotheses of part~(a) the restriction
  \begin{equation}\label{eq:restriction}
    P_+|_{\Ran Q_+\cap\fD} \colon \Ran Q_+\cap\fD\to\fD_+
  \end{equation}
  is bijective. Indeed, its surjectivity follows directly from the hypothesis $\Ran(P_+Q_+|_{\fD_+}) \supset \fD_+$. For the
  injectivity, we follow Step~2 of the proof of~\cite[Theorem~1]{GLS99}: assume to the contrary that $P_+x = 0$ for some non-zero
  $x \in \Ran Q_+ \cap \fD$. Then, on the one hand we have $\fb[x , x] > 0$, and on the other hand $x \in \Ran P_-$, that is,
  $x \in \fD_-$. The latter gives $\fb[x , x] \le 0$ by hypothesis, a contradiction. The claim of part~(a) now follows by Step~1
  of the proof of~\cite[Theorem~1]{GLS99}.

  Replacing form domains with operator domains in the above reasoning and the cited Step~1 of the proof in~\cite{GLS99}, that is,
  $\fD$ by $\Dom(A)=\Dom(B)$ and $\fD_\pm$ by $\cD_\pm$, part~(b) can be proved in the same manner.
\end{proof}%

\begin{remark}\label{rem:spectralShift}
  The above proposition is tailored towards spectral gaps around zero, but by a spectral shift we can of course handle also
  spectral gaps around any point $\gamma \in \RR$. Indeed, we have $\EE_{A-\gamma}((0,\infty)) = \EE_A((\gamma,\infty))$ for
  $\gamma \in \RR$ and analogously for $B$. Moreover, the form associated to the operator $B - \gamma$ is known to agree with the
  form $\fb - \gamma$. The latter can be seen for instance with an analogous reasoning as
  in~\cite[Proposition~10.5\,(a)]{Schm12}; cf.~also Lemma~\ref{cor:formpert} in Appendix~\ref{sec:heinz} below.
\end{remark}

\begin{remark}\label{rem:neg}
  (1)
  The hypothesis $\fb[x,x]\le0$ for all $x\in\fD_-$ in part~(a) of Proposition~\ref{prop:GLS} is used not only to verify the
  injectivity of the restriction~\eqref{eq:restriction} but is also a crucial ingredient in the cited Step~1 of the proof
  of~\cite[Theorem~1]{GLS99}. The same applies for the hypothesis $\langle x,Bx \rangle\le 0$ for all $x\in\cD_-$ in part~(b).

  (2)
  Since $P_+$ and $Q_+$ are spectral projections for the respective operators, we always have
  $\Ran(P_+Q_+|_{\fD_+}) \subset \fD_+$ and $\Ran(P_+Q_+|_{\cD_+}) \subset \cD_+$. In this respect, the
  condition $\Ran(P_+Q_+|_{\fD_+}) \supset \fD_+$ in part~(a) of Proposition~\ref{prop:GLS} actually means that the restriction
  $P_+Q_+|_{\fD_+} \colon \fD_+ \to \fD_+$ is surjective. This has not been formulated explicitly in the statement
  of~\cite[Theorem~1]{GLS99} but has instead been guaranteed by the stronger condition
  \begin{equation}\label{eq:GLS}
    \norm{(\abs{A}+I)^{1/2}P_+Q_-(\abs{A}+I)^{-1/2}} < 1.
  \end{equation}
  Since $\fD_+ = \Ran ((\abs{A}+I)^{-1/2}|_{\Ran P_+})$, a standard Neumann series argument then even gives bijectivity of the
  restriction $P_+Q_+|_{\fD_+}$, see Step~2 of the proof of~\cite[Theorem~1]{GLS99}. In this reasoning, the operators
  $(\abs{A}+I)^{\pm 1/2}$ can be replaced by $(\abs{A}+\alpha I)^{\pm 1/2}$ for any $\alpha>0$; if $\abs{A}$ has a bounded
  inverse, also $\alpha=0$ can be considered here.
  
  Of course, the above reasoning also applies in the situation of part~(b) of Proposition~\ref{prop:GLS}, but with
  $(\abs{A}+\alpha I)^{\pm 1/2}$ replaced by $(\abs{A}+\alpha I)^{\pm 1}$.

  Condition~\eqref{eq:GLS} has been considered in~\cite{GLS99} in the setting of Dirac operators, but it seems to be hard to
  verify it on an abstract level. This approach is therefore not pursued further here.
\end{remark}

In the context of our main theorems, the restriction $P_+Q_+|_{\Ran P_+}$, understood as an endomorphism of $\Ran P_+$, will
always be bijective, cf.~Remark~\ref{rem:PQbij}\,(1) below. It turns out that then the hypotheses of part~(b) in
Proposition~\ref{prop:GLS} imply those of part~(a), in which case both representations for the minimax values in
Proposition~\ref{prop:GLS} are valid. More precisely, we have the following lemma, essentially based on the well-known Heinz
inequality, cf.~Appendix~\ref{sec:heinz} below.

\begin{lemma}\label{lem:GLS}
  Assume Hypothesis~\ref{hyp:minimax} with $\Dom(A) = \Dom(B)$.
  \begin{enumerate}
    \renewcommand{\theenumi}{\alph{enumi}}
    
    \item One has $\Dom(\abs{A}^{1/2}) = \Dom(\abs{B}^{1/2})$.

    \item If $\scprod{x , Bx} \le 0$ for all $x \in \cD_-$, then $\fb[x , x] \le 0$ for all $x \in \fD_-$.

    \item If the restriction $P_+Q_+|_{\Ran P_+} \colon \Ran P_+ \to \Ran P_+$ is bijective and
          $\Ran(P_+Q_+|_{\cD_+}) \supset \cD_+$, then also $\Ran(P_+Q_+|_{\fD_+}) \supset \fD_+$.

  \end{enumerate}
\end{lemma}

\begin{proof}
  (a).
  This is a consequence of the well-known Heinz inequality, see, e.g., Corollary~\ref{cor:SchmDiss} below. Alternatively, this
  follows by classical considerations regarding operator and form boundedness, see Remark~\ref{rem:SchmDiss} below.
  
  (b).
  It follows from part~(a) that the operator $\abs{B}^{1/2} ( \abs{A}^{1/2} + I )^{-1}$ is closed and everywhere defined, hence
  bounded by the closed graph theorem. Thus,
  \begin{equation*}
    \norm{\abs{B}^{1/2}x} \le \norm{\abs{B}^{1/2}(\abs{A}^{1/2}+I)^{-1}}\cdot \norm{(\abs{A}^{1/2}+I)x}
  \end{equation*}
  for all $x\in \Dom(\abs{A}^{1/2})=\Dom(\abs{B}^{1/2})$. Since $\cD_-$ is a core for the operator
  $\abs{A|_{\Ran P_-}}^{1/2}=\abs{A}^{1/2}|_{\Ran P_-}$ with $\Dom(\abs{A}^{1/2}|_{\Ran P_-})=\fD_-$, the inequality
  $\fb[x,x]\le 0$ for $x\in\fD_-$ now follows from the hypothesis $\scprod{ x,Bx } \le 0$ for all $x\in\cD_-$ by approximation.

  (c).
  We clearly have $\Ran(P_+Q_+|_{\cD_+})=\cD_+$, $\cD_+=\Dom(A|_{\Ran P_+})$, and $\fD_+=\Dom(\abs{A|_{\Ran P_+}}^{1/2})$.
  Applying Corollary~\ref{cor:fracBij} below with the choices $\Lambda_1=\Lambda_2=A|_{\Ran P_+}$ and $S = P_+Q_+|_{\Ran P_+}$
  therefore implies that $\Ran(P_+Q_+|_{\fD_+})=\fD_+$, which proves the claim.
\end{proof}%

\begin{remark}\label{rem:PQbij}
  (1)
  In light of the identity $P_+Q_+ = P_+ - P_+Q_-$, the bijectivity of $P_+Q_+|_{\Ran P_+} \colon \Ran P_+ \to \Ran P_+$ can be
  guaranteed, for instance, by the condition $\norm{P_+Q_-} < 1$ via a standard Neumann series argument. Since
  $P_+-Q_+=P_+Q_- - P_-Q_+$ and, in particular, $\norm{P_+Q_-}\le\norm{P_+-Q_+}$, this condition holds if the stronger inequality
  $\norm{P_+-Q_+}<1$ is satisfied. In the latter case, there is a unitary operator $U$ with $Q_+U=UP_+$, see,
  e.g.,~\cite[Theorem~I.6.32]{Kato95}, which implies that $\dim \Ran P_+=\dim\Ran Q_+$. It is this situation we encounter in
  Theorems~\ref{thm:genSemibounded}--\ref{thm:offdiagForm}.
 
  (2)
  In the case where $B$ is an infinitesimal operator perturbation of $A$, the inequality $\norm{P_+Q_-}<1$ already implies that
  $\Ran(P_+Q_+|_{\cD_+}) \supset \cD_+$, see the following section; the particular case where $B$ is a bounded perturbation of
  $A$ has previously been considered in~\cite[Lemma~A.6]{NSTTV18}. For more general, not necessarily infinitesimal,
  perturbations, this remains so far an open problem.
\end{remark}

%%%%%%%%%%%%%%%%%%%%%%%%%%%%%%%%%%%%%%%%%%%%%%%%%%%%%%%%%%%%%%%%%%%%%%%%%%%%%%%%%%%%%%%%%%%%%%%%%%%%%%%%%%%%%%%%%%%%%%%%%%%%%%%%%
%%%%%%%%%%%%%%%%%%%%%%%%%%%%%%%%%%%%%%%%%%%%%%%%%%%%%%%%%%%%%%%%%%%%%%%%%%%%%%%%%%%%%%%%%%%%%%%%%%%%%%%%%%%%%%%%%%%%%%%%%%%%%%%%%
%%% Graph norm approach
%%%%%%%%%%%%%%%%%%%%%%%%%%%%%%%%%%%%%%%%%%%%%%%%%%%%%%%%%%%%%%%%%%%%%%%%%%%%%%%%%%%%%%%%%%%%%%%%%%%%%%%%%%%%%%%%%%%%%%%%%%%%%%%%%
%%%%%%%%%%%%%%%%%%%%%%%%%%%%%%%%%%%%%%%%%%%%%%%%%%%%%%%%%%%%%%%%%%%%%%%%%%%%%%%%%%%%%%%%%%%%%%%%%%%%%%%%%%%%%%%%%%%%%%%%%%%%%%%%%
\section{Proof of Theorem~\ref{thm:genOpInfinitesimal}: The graph norm approach}\label{sec:graphNorm}

In this section we show that the inequality $\norm{P_+Q_-} < 1$ in the context of Theorem~\ref{thm:genOpInfinitesimal} implies
that $\Ran(P_+Q_+|_{\cD_+}) \supset \cD_+$, which is essentially what is needed to deduce Theorem~\ref{thm:genOpInfinitesimal}
from Proposition~\ref{prop:GLS} and Lemma~\ref{lem:GLS}. The main technique used to accomplish this can in fact be formulated in
a much more general framework:

Recall that for a closed operator $\Lambda$ on a Banach space with norm $\norm{\,\cdot\,}$, its domain $\Dom(\Lambda)$ can be
equipped with the~\emph{graph norm}
\begin{equation*}
 \norm{x}_\Lambda := \norm{x} + \norm{\Lambda x},\quad x\in\Dom(\Lambda),
\end{equation*}
which makes $(\Dom(\Lambda),\norm{\,\cdot\,}_\Lambda)$ a Banach space. Also recall that a linear operator $K$ with
$\Dom(K) \supset \Dom(\Lambda)$ is called~\emph{$\Lambda$-bounded with $\Lambda$-bound $\beta_* \ge 0$} if for all
$\beta > \beta_*$ there is an $\alpha \ge 0$ with
\begin{equation}\label{eq:relBound}
  \norm{Kx}
  \le
  \alpha \norm{x} + \beta \norm{\Lambda x}
  \quad\text{ for all }\
  x \in \Dom(\Lambda)
  .
\end{equation}

The following lemma extends part~(a) of~\cite[Proposition~A.5]{NSTTV18}, taken from Lemma~3.9 in the author's
Ph.D.~thesis~\cite{SeelDiss}, to relatively bounded commutators.

\begin{lemma}\label{lem:specRad}
  Let $\Lambda$ be a closed operator on a Banach space, $K$ be $\Lambda$-bounded with $\Lambda$-bound $\beta_* \ge 0$, and let
  $S$ be bounded with $\Ran(S|_{\Dom(\Lambda)})\subset\Dom(\Lambda)$ and
  \begin{equation*}
    \Lambda Sx - S\Lambda x
    =
    Kx
    \quad\text{ for all }\
    x\in\Dom(\Lambda)
    .
  \end{equation*}
  Then, $S|_{\Dom(\Lambda)}$ is bounded on $\Dom(\Lambda)$ with respect to the graph norm for $\Lambda$, and the corresponding
  spectral radius satisfies
  \begin{equation*}
    r_\Lambda(S)
    :=
    \lim_{k\to\infty} \norm{(S|_{\Dom(\Lambda)})^k}_\Lambda^{1/k}\le\norm{S} + \beta_*
    .
  \end{equation*}
\end{lemma}

\begin{proof}
  Only small modifications to the reasoning from~\cite[Lemma~3.9]{SeelDiss},~\cite[Proposition~A.5]{NSTTV18} are necessary. For
  the sake of completeness, we reproduce the full argument here:
  
  Let $\beta > \beta_*$ and $\alpha \ge 0$ such that~\eqref{eq:relBound} holds. Then, for $x\in\Dom(\Lambda)$ one has
  \begin{equation*}%\label{eq:Sgraph}
   \norm{\Lambda Sx} \le \norm{S}\norm{\Lambda x} + \norm{Kx} \le (\norm{S}+\beta)\norm{\Lambda x} + \alpha\norm{x},
  \end{equation*}
  so that
  \begin{equation*}
    \norm{Sx}_\Lambda = \norm{Sx} + \norm{\Lambda Sx} \le \bigl(\norm{S}+\beta\bigr)\norm{x}_\Lambda + \alpha\norm{x}.
  \end{equation*}
  In particular, $S|_{\Dom(\Lambda)}$ is bounded with respect to the graph norm $\norm{\,\cdot\,}_\Lambda$ with
  $\norm{S}_\Lambda\le \norm{S} + \beta + \alpha$.

  Now, a straightforward induction yields
  \begin{equation*}
    \norm{S^kx}_\Lambda
    \le
    \bigl(\norm{S} + \beta\bigr)^k\norm{x}_\Lambda + k\alpha\bigl(\norm{S}+\beta\bigr)^{k-1}\norm{x},
    \quad x\in\Dom(\Lambda),
  \end{equation*}
  for $k\in\N$. Hence, $\norm{(S|_{\Dom(\Lambda)})^k}_\Lambda \le (\norm{S}+\beta)^k + k\alpha(\norm{S}+\beta)^{k-1}$, so that
  \begin{align*}
    r_\Lambda(S)
    &=
    \lim_{k\to\infty} \norm{(S|_{\Dom(\Lambda)})^k}_\Lambda^{1/k}
      \le
      \lim_{k\to\infty} \bigl( (\norm{S}+\beta)^k+k\alpha(\norm{S}+\beta)^{k-1} \bigr)^{1/k}\\
    &=
    \norm{S}+\beta
    .
  \end{align*}
  Since $\beta > \beta_*$ was chosen arbitrarily, this proves the claim.
\end{proof}%

We are now in position to prove Theorem~\ref{thm:genOpInfinitesimal}.

\begin{proof}[Proof of Theorem~\ref{thm:genOpInfinitesimal}]
  We mainly follow the line of reasoning in the proof of~\cite[Lemma~A.6]{NSTTV18}. Only a few additional considerations are
  necessary in order to accommodate unbounded perturbations $V$ by means of Lemma~\ref{lem:specRad}. For convenience of the
  reader, we nevertheless reproduce the whole argument here.
 
  Define $S,T \colon \Ran P_+ \to \Ran P_+$ by
  \begin{equation*}
    S := P_+Q_-|_{\Ran P_+},\quad T := P_+Q_+|_{\Ran P_+} = I_{\Ran P_+} - S.
  \end{equation*}
  By hypothesis, we have $\norm{S} \le \norm{P_+Q_-} < 1$, so that $T$ is bijective. In light of Proposition~\ref{prop:GLS} and
  Lemma~\ref{lem:GLS}, it now remains to show the inclusion $\Ran(P_+Q_+|_{\cD_+}) \supset \cD_+$, that is,
  $\Ran(T^{-1}|_{\cD_+})\subset\cD_+$. To this end, we rewrite $T^{-1}$ as a Neumann series,
  \begin{equation*}
    T^{-1}
    =
    (I_{\Ran P_+} - S)^{-1}
    =
    \sum_{k=0}^\infty S^k
    .
  \end{equation*}
  Clearly, $S$ maps the domain $\cD_+=\Dom(A|_{\Ran P_+})$ into itself, so that the inclusion $\Ran(T^{-1}|_{\cD_+})\subset\cD_+$
  holds if the above series converges also with respect to the graph norm for the closed operator $\Lambda:=A|_{\Ran P_+}$. This,
  in turn, is the case if the corresponding spectral radius $r_\Lambda(S)$ of $S$ is smaller than $1$.

  For $x\in\cD_+\subset\Ran P_+$ we compute
  \begin{align*}
    \Lambda Sx
    &=
    AP_+Q_-x
      = P_+(A+V)Q_-x - P_+VQ_-x\\
    &=
    P_+Q_-(A+V)x - P_+VQ_-x\\
    &=
    S\Lambda x + Kx
  \end{align*}
  with
  \begin{equation*}
    K := (P_+Q_-V - P_+VQ_-)|_{\Ran P_+}.
  \end{equation*}
  We show that the operator $K$ is $\Lambda$-bounded with $\Lambda$-bound $0$. Indeed, let $b > 0$, and choose $a\ge0$ with
  $\norm{Vx}\le a\norm{x}+b\norm{Ax}$ for all $x\in\Dom(A)$; recall that $V$ is infinitesimal with respect to $A$ by hypothesis.
  Then,
  \begin{equation*}
    \norm{VQ_-x}
    \le
    a\norm{Q_-x} + b\norm{AQ_-x}
    \le
    a\norm{x} + b\norm{(A+V)x} + b\norm{VQ_-x},
  \end{equation*}
  so that
  \begin{align*}
    \norm{VQ_-x}
    &\le
    \frac{a}{1-b}\norm{x} + \frac{b}{1-b}\bigl(\norm{Ax}+\norm{Vx}\bigr)\\
    &\le
    \frac{a(1+b)}{1-b}\norm{x} + \frac{b(1+b)}{1-b}\norm{Ax}.
  \end{align*}
  Thus,
  \begin{equation}\label{eq:relBoundK}
   \begin{aligned}
    \norm{Kx}
    &\le
    \norm{P_+Q_-}\norm{Vx} + \norm{VQ_-x}\\
    &\le
    a\Bigl(\norm{P_+Q_-} + \frac{1+b}{1-b}\Bigr)\norm{x} + b\Bigl(\norm{P_+Q_-}+\frac{1+b}{1-b}\Bigr)\norm{\Lambda x}
   \end{aligned}
  \end{equation}
  for $x\in\Dom(\Lambda)=\cD_+$. Since $b>0$ was chosen arbitrarily, this implies that $K$ is $\Lambda$-bounded with
  $\Lambda$-bound $0$. It therefore follows from Lemma~\ref{lem:specRad} that $r_\Lambda(S)\le\norm{S}<1$, which completes the
  proof.
\end{proof}%

\begin{remark}\label{rem:relBoundK}
  (1)
  Estimate~\eqref{eq:relBoundK} suggests that also relatively bounded perturbations $V$ that are not necessarily infinitesimal
  with respect to $A$ can be considered here. In fact, if $b_*\in[0,1)$ is the $A$-bound of $V$, then by~\eqref{eq:relBoundK} and
  Lemma~\ref{lem:specRad} we have
  \begin{equation*}
    r_\Lambda(S)
    \le
    \norm{P_+Q_-} + b_*\Bigl(\norm{P_+Q_-}+\frac{1+b_*}{1-b_*}\Bigr)
    ,
  \end{equation*}
  and the right-hand side of the latter is smaller than $1$ if and only if
  \begin{equation*}
    \norm{P_+Q_-} < \frac{1-2b_*-b_*^2}{1-b_*^2}.
  \end{equation*}
  This is a reasonable condition on the norm $\norm{P_+Q_-}$ only for $b_*<\sqrt{2}-1$.

  (2)
  A similar result as in (1) can be obtained in terms of the $(A+V)$-bound of $V$: If for some $\tilde{b}\in[0,1)$ and
  $\tilde{a}\ge 0$ one has $\norm{Vx} \le \tilde{a}\norm{x} + \tilde{b}\norm{(A+V)x}$ for all $x\in\Dom(A)=\Dom(A+V)$, then
  standard arguments as in the above proof of Theorem~\ref{thm:genOpInfinitesimal} show that
  \begin{equation*}
    \norm{Vx}
    \le
    \frac{\tilde{a}}{1-\tilde{b}}\norm{x} + \frac{\tilde{b}}{1-\tilde{b}}\norm{Ax}
  \end{equation*}
  and, in turn,
  \begin{align*}
    \norm{VQ_-x}
    &\le
    \tilde{a}\norm{x} + \tilde{b}\norm{(A+V)x}
      \le
      \tilde{a}\norm{x} + \tilde{b}\norm{Ax} + \tilde{b}\norm{Vx}\\
    &\le
    \tilde{a}\Bigl( 1 + \frac{\tilde{b}}{1-\tilde{b}} \Bigr)\norm{x}
      + \tilde{b}\Bigl( 1 + \frac{\tilde{b}}{1-\tilde{b}} \Bigr)\norm{Ax}\\
    &=
    \frac{\tilde{a}}{1-\tilde{b}}\norm{x} + \frac{\tilde{b}}{1-\tilde{b}}\norm{Ax}
  \end{align*}
  for all $x\in\Dom(A)$. Plugging these into~\eqref{eq:relBoundK} gives
  \begin{align*}
    \norm{Kx}
    &\le
    \norm{P_+Q_-}\norm{Vx} + \norm{VQ_-x}\\
    &\le
    (1+\norm{P_+Q_-})\Bigl( \frac{\tilde{a}}{1-\tilde{b}}\norm{x} + \frac{\tilde{b}}{1-\tilde{b}}\norm{\Lambda x} \Bigr)
  \end{align*}
  for all $x \in \Dom(\Lambda) = \cD_+$, which eventually leads to
  \begin{equation*}
    r_\Lambda(S)
    \le
    \norm{P_+Q_-} + \frac{\tilde{b}(1+\norm{P_+Q_-})}{1-\tilde{b}}
    =
    \frac{\norm{P_+Q_-}+\tilde{b}}{1-\tilde{b}}
    .
  \end{equation*}
  The right-hand side of the latter is smaller than $1$ if and only if
  \begin{equation*}
    \norm{P_+Q_-} < 1-2\tilde{b}.
  \end{equation*}
  This is a reasonable condition on $\norm{P_+Q_-}$ only for $\tilde{b} < 1/2$.
\end{remark}

%%%%%%%%%%%%%%%%%%%%%%%%%%%%%%%%%%%%%%%%%%%%%%%%%%%%%%%%%%%%%%%%%%%%%%%%%%%%%%%%%%%%%%%%%%%%%%%%%%%%%%%%%%%%%%%%%%%%%%%%%%%%%%%%%
%%%%%%%%%%%%%%%%%%%%%%%%%%%%%%%%%%%%%%%%%%%%%%%%%%%%%%%%%%%%%%%%%%%%%%%%%%%%%%%%%%%%%%%%%%%%%%%%%%%%%%%%%%%%%%%%%%%%%%%%%%%%%%%%%
%%% Block diagonalization approach
%%%%%%%%%%%%%%%%%%%%%%%%%%%%%%%%%%%%%%%%%%%%%%%%%%%%%%%%%%%%%%%%%%%%%%%%%%%%%%%%%%%%%%%%%%%%%%%%%%%%%%%%%%%%%%%%%%%%%%%%%%%%%%%%%
%%%%%%%%%%%%%%%%%%%%%%%%%%%%%%%%%%%%%%%%%%%%%%%%%%%%%%%%%%%%%%%%%%%%%%%%%%%%%%%%%%%%%%%%%%%%%%%%%%%%%%%%%%%%%%%%%%%%%%%%%%%%%%%%%
\section{The block diagonalization approach}\label{sec:blockDiag}

In this section, we discuss an approach to verify the hypotheses of Proposition~\ref{prop:GLS} and Lemma~\ref{lem:GLS} which
relies on techniques previously discussed in the context of block diagonalizations of operators and forms, for instance
in~\cite{MSS16} and~\cite{GKMSV17}, respectively; see also Remark~\ref{rem:MSS16} below.

Recall that for the two orthogonal projections $P_+$ and $Q_+$ from Hypothesis~\ref{hyp:minimax} the inequality
$\norm{P_+-Q_+}<1$ holds if and only if $\Ran Q_+$ can be represented as
\begin{equation}\label{eq:graph}
  \Ran Q_+ = \{ f \oplus Xf \mid f\in\Ran P_+ \}
\end{equation}
with some bounded linear operator $X\colon\Ran P_+\to\Ran P_-$; in this case, one has
\begin{equation}\label{eq:normPQX}
  \norm{P_+-Q_+}
  =
  \frac{\norm{X}}{\sqrt{1+\norm{X}^2}}
  ,
\end{equation}
see, e.g.,~\cite[Corollary~3.4\,(i)]{KMM03:181}. The orthogonal projection $Q_+$ can then be represented as the $2\times2$ block
operator matrices
\begin{equation}\label{eq:reprQ}
 \begin{aligned}
  Q_+
  &=
  \begin{pmatrix}
    (I_{\Ran P_+}+X^*X)^{-1} & (I_{\Ran P_+}+X^*X)^{-1}X^*\\
    X(I_{\Ran P_+}+X^*X)^{-1} & X(I_{\Ran P_+}+X^*X)^{-1}X^*
  \end{pmatrix}\\
  &=
  \begin{pmatrix}
    (I_{\Ran P_+}+X^*X)^{-1} & X^*(I_{\Ran P_-}+XX^*)^{-1}\\
    (I_{\Ran P_-}+XX^*)^{-1}X & XX^*(I_{\Ran P_-}+XX^*)^{-1}
  \end{pmatrix}
 \end{aligned}
\end{equation}
with respect to $\Ran P_+\oplus\Ran P_-$, see, e.g.,~\cite[Remark~3.6]{KMM03:181}. In particular, we have
\begin{equation}\label{eq:PQY}
  P_+Q_+|_{\Ran P_+}
  =
  (I_{\Ran P_+} + X^*X)^{-1}
  ,
\end{equation}
which is in fact the starting point for the current approach. With regard to the desired relations
$\Ran (P_+Q_+|_{\cD_+}) \supset \cD_+$ and $\Ran (P_+Q_+|_{\fD_+}) \supset \fD_+$, we need to establish that the operator
$I_{\Ran P_+} + X^*X$ maps $\cD_+$ and $\fD_+$ into $\cD_+$ and $\fD_+$, respectively.

Define the skew-symmetric operator $Y$ via the $2\times2$ block operator matrix
\begin{equation}\label{eq:defY}
  Y
  =
  \begin{pmatrix} 0 & -X^*\\ X & 0 \end{pmatrix}
\end{equation}
with respect to $\Ran P_+\oplus\Ran P_-$. Then, the operators $I\pm Y$ are bijective with
\begin{equation}\label{eq:Ypm}
  (I-Y)(I+Y)
  =
  \begin{pmatrix}
    I_{\Ran P_+}+X^*X & 0\\
    0 & I_{\Ran P_-}+XX^*
  \end{pmatrix}
  .
\end{equation}

The following lemma is extracted from various sources. We comment on this afterwards in Remark~\ref{rem:PQXY} below.
\begin{lemma}\label{lem:PQXY}
  Suppose that the projections $P_+$ and $Q_+$ from Hypothesis~\ref{hyp:minimax} satisfy $\norm{P_+ - Q_+} < 1$, and let the
  operators $X$ and $Y$ be as in~\eqref{eq:graph} and~\eqref{eq:defY}, respectively. Moreover, let $\cC$ be an invariant subspace
  for $P_+$ and $Q_+$ such that $\cC = (\cC \cap \Ran P_+) \oplus (\cC \cap \Ran P_-) =: \cC_+ \oplus \cC_-$.
  
  Then, the following are equivalent:
  \begin{enumerate}
    \renewcommand{\theenumi}{\roman{enumi}}

    \item $I_{\Ran P_+} + X^*X$ maps $\cC_+$ into itself;

    \item $I_{\Ran P_-} + XX^*$ maps $\cC_-$ into itself;

    \item $Y$ maps $\cC$ into itself;

    \item $(I+Y)$ maps $\cC$ into itself;

    \item $(I-Y)$ maps $\cC$ into itself.

  \end{enumerate}
\end{lemma}

\begin{proof}
  Clearly, the hypotheses imply that $P_+Q_+$ maps $\cC$ into $\cC_+$ and $P_-Q_+$ maps $\cC$ into $\cC_-$.
  
  (i)$\Rightarrow$(ii).
  Let $g \in \cC_-$. Using the first representation in~\eqref{eq:reprQ}, we then have
  $(I_{\Ran P_+} + X^*X)^{-1}X^*g = (P_+Q_+|_{\Ran P_-})g \in \cC_+$. Hence, $X^*g \in \cC_+$ by~(i) and, in turn,
  $h := (I_{\Ran P_+} + X^*X)X^*g \in \cC_+$. Using again~\eqref{eq:reprQ}, this yields
  \begin{align*}
    (I_{\Ran P_-} + XX^*)g
    &=
    g + XX^*g
      =
      g + X(I_{\Ran P_+}+X^*X)^{-1}h\\
    &=
    g + (P_-Q_+|_{\Ran P_+})h \in \cC_-
    .
  \end{align*}
  As a byproduct, we have also shown that $X^*$ maps $\cC_-$ into $\cC_+$.

  (ii)$\Rightarrow$(i).
  Using the identities $(I_{\Ran P_-} + XX^*)^{-1}X = P_-Q_+|_{\Ran P_+}$ and $X^*(I_{\Ran P_-}+XX^*)^{-1} = P_+Q_+|_{\Ran P_-}$
  taken from the second representation in~\eqref{eq:reprQ}, the proof is completely analogous to the implication
  (i)$\Rightarrow$(ii). In particular, we likewise obtain as a byproduct that $X$ maps $\cC_+$ into $\cC_-$.

  (i),(ii)$\Rightarrow$(iii).
  We have already seen that $X$ maps $\cC_+$ into $\cC_-$ and that $X^*$ maps $\cC_-$ into $\cC_+$. Taking into account that
  $\cC = \cC_+ \oplus \cC_-$, this means that $Y$ maps $\cC$ into itself.

  (iii)$\Leftrightarrow$(iv),(v).
  This is clear.

  (iv),(v)$\Rightarrow$(i),(ii).
  This follows immediately from identity~\eqref{eq:Ypm}.
\end{proof}%

\begin{remark}\label{rem:PQXY}
  The proof of the equivalence (i)$\Leftrightarrow$(ii) and the one of the implication (i),(ii)$\Rightarrow$(iii) in
  Lemma~\ref{lem:PQXY} are extracted from the proof of~\cite[Theorem~5.1]{GKMSV17}; see also~\cite[Theorem~6.3.1 and
  Lemma~6.3.3]{SchmDiss}.

  The equivalence (iv)$\Leftrightarrow$(v) can alternatively be directly obtained from the identity
  \begin{equation*}
    \begin{pmatrix}
      I_{\Ran P_+} & 0\\
      0 & -I_{\Ran P_-}
    \end{pmatrix}
    (I + Y)
    \begin{pmatrix}
      I_{\Ran P_+} & 0\\
      0 & -I_{\Ran P_-}
    \end{pmatrix}
    =
    I - Y
    .
  \end{equation*}
  Such an argument has been used in the proof of~\cite[Proposition~3.3]{MSS16}.

  The implication (iv),(v)$\Rightarrow$(i) can essentially be found in the proof of~\cite[Theorem~5.1]{GKMSV17}
  and~\cite[Remark~6.3.2]{SchmDiss}.
\end{remark}

Below, we apply Lemma~\ref{lem:PQXY} with $\cC = \Dom(A) = \Dom(B) = \cD_+ \oplus \cD_-$ or
$\cC = \Dom(\abs{A}^{1/2}) = \Dom(\abs{B}^{1/2}) = \fD_+ \oplus \fD_-$, depending on the situation. The easiest case is
encountered in Theorem~\ref{thm:genSemibounded}:

\begin{proof}[Proof of Theorem~\ref{thm:genSemibounded}]
  Let $\Dom(\abs{A}^{1/2}) = \Dom(\abs{B}^{1/2})$ and $\fb[x , x] \le 0$ for all $x \in \fD_-$. We then have $\fD_- = \Ran P_-$
  if $A$ is bounded from below and $\fD_+ = \Ran P_+$ if $A$ is bounded from above. Hence, item~(ii) or (i) in
  Lemma~\ref{lem:PQXY}, respectively, with $\cC = \fD_+ \oplus \fD_-$ is automatically satisfied. In any case, we have by
  Lemma~\ref{lem:PQXY} that $I_{\Ran P_+} + X^*X$ maps $\fD_+$ into $\fD_+$, which by identity~\eqref{eq:PQY} means that
  $\Ran (P_+Q_+|_{\fD_+}) \supset \fD_+$. The representation~\eqref{eq:genSemibounded:form} now follows from
  Proposition~\ref{prop:GLS}\,(a) and Remark~\ref{rem:PQbij}\,(1). If even $\Dom(A) = \Dom(B)$ and $\scprod{ x , Bx } \le 0$ for
  all $x \in \cD_-$, we use the same reasoning as above with $\fD_+$ and $\fD_-$ replaced by $\cD_+$ and $\cD_-$, respectively,
  and obtain representation~\eqref{eq:genSemibounded:op} from Proposition~\ref{prop:GLS}\,(b) and Remark~\ref{rem:PQbij}\,(1).
  The representation~\eqref{eq:genSemibounded:form} is then still valid by Lemma~\ref{lem:GLS} and the first part of the proof.
\end{proof}%

While certain conditions for Proposition~\ref{prop:GLS} and Lemma~\ref{lem:GLS} are part of the hypotheses of
Theorems~\ref{thm:genOpInfinitesimal} and~\ref{thm:genSemibounded}, in the situations of Theorems~\ref{thm:offdiagOp}
and~\ref{thm:offdiagForm} these need to be verified explicitly from the specific hypotheses at hand. Here, we rely on previous
considerations on block diagonalizations for block operator matrices and forms. In case of Theorem~\ref{thm:offdiagOp}, the
crucial ingredient is presented in the following result, extracted from~\cite{MSS16}. An earlier result in this direction is
commented on in Remark~\ref{rem:MSS16}\,(2) below.

\begin{proposition}[see~{\cite[Theorem~6.1]{MSS16}}]\label{prop:MSS16}
  In the situation of Theorem~\ref{thm:offdiagOp} one has $\norm{P_+-Q_+} \le \sqrt{2}/2 <1$, and the operator identity
  \begin{equation}\label{eq:blockDiag}
    (I-Y)(A+V)(I-Y)^{-1} = A-YV
  \end{equation}
  holds with $Y$ as in~\eqref{eq:defY}.
\end{proposition}

\begin{proof}
  Set $V_{\text{off}} := V|_{\Dom(A)}$, so that we have $B = A+V = A+V_\text{off}$ as well as $A-YV = A-YV_\text{off}$. Clearly,
  the hypotheses on $V$ ensure that $V_\text{off}$ is $A$-bounded with $A$-bound $b_*<1$ and off-diagonal with respect to the
  decomposition $\Ran P_+\oplus\Ran P_-$. By~\cite[Lemma~6.3]{MSS16} we now have 
  \begin{equation*}
    \Ker(A+V_\text{off})
    \subset
    \Ker A
    \subset
    \Ran P_-
    .
  \end{equation*}
  In light of~\eqref{eq:normPQX}, the claim therefore is just an instance of~\cite[Theorem~6.1]{MSS16}.
\end{proof}%

\begin{remark}\label{rem:MSS16}
  (1)
  Let $A_\pm:=A|_{\Ran P_\pm}$ be the parts of $A$ associated with the subspaces $\Ran P_\pm$, and write
  \begin{equation*}
    V|_{\Dom(A)}
    =
    \begin{pmatrix} 0 & W\\ W^* & 0 \end{pmatrix}
    ,
  \end{equation*}
  where $W\colon \Ran P_-\supset\cD_-\to\Ran P_+$ is given by $Wx:=P_+Vx$, $x\in\cD_-$. Then,
  \begin{equation*}
    A - YV
    =
    \begin{pmatrix}
      A_+ - X^*W^* & 0\\
      0 & A_- + XW
    \end{pmatrix}
    .
  \end{equation*} 
  In this sense, identity~\eqref{eq:blockDiag} can be viewed as a block diagonalization of the operator $A+V$. For a more
  detailed discussion of block diagonalizations and operator Riccati equations in the operator setting, the reader is referred
  to~\cite{MSS16} and the references cited therein.

  (2)
  In the particular case where $0$ belongs to the resolvent set of $A$, the conclusion of Proposition~\ref{prop:MSS16} can be
  inferred also from~\cite[Theorems~2.7.21 and~2.8.5]{Tre08}.
\end{remark}

\begin{proof}[Proof of Theorem~\ref{thm:offdiagOp}]
  For $x \in \cD_-$, we have
  \begin{equation*}
    \scprod{ x , Vx }
    =
    \scprod{ P_-x , VP_-x }
    =
    \scprod{ x , P_-VP_-x }
    =
    0
  \end{equation*}
  and, thus,
  \begin{equation*}
    \scprod{ x , (A+V)x }
    =
    \scprod{ x , Ax }
    \le
    0
    .
  \end{equation*}
  Moreover, by Proposition~\ref{prop:MSS16} the inequality $\norm{P_+-Q_+}<1$ is satisfied. Let $Y$ be as in~\eqref{eq:defY}.
  Since $\Dom(A+V)=\Dom(A)=\Dom(A-YV)$, it then follows from identity~\eqref{eq:blockDiag} that $I-Y$ maps
  $\cC := \Dom(A) = \cD_+ \oplus \cD_-$ into itself. In turn, Lemma~\ref{lem:PQXY} implies that $I_{\Ran P_+} + X^*X$ maps
  $\cD_+$ into itself, which by identity~\eqref{eq:PQY} means that $\Ran(P_+Q_+|_{\cD_+}) \supset \cD_+$. The claim now follows
  from Proposition~\ref{prop:GLS}, Lemma~\ref{lem:GLS}, and Remark~\ref{rem:PQbij}\,(1).
\end{proof}

To the best of the author's knowledge, no direct analogue of Proposition~\ref{prop:MSS16} is known so far in the setting of form
rather than operator perturbations. Although the inequality $\norm{P_+-Q_+} \le \sqrt{2}/2$ can be established here as well under
fairly reasonable assumptions, see~\cite[Theorem~3.3]{GKMSV17}, the mapping properties of the operators $I \pm Y$ connected with
a corresponding diagonalization related to~\eqref{eq:blockDiag} are much harder to verify. The situation is even more subtle
there since also the domain equality $\Dom(\abs{A}^{1/2}) = \Dom(\abs{B}^{1/2})$ needs careful treatment. The latter is
conjectured to hold in a general off-diagonal form perturbation framework~\cite[Remark~2.7]{GKMV13}. Some characterizations have
been discussed in~\cite[Theorem~3.8]{Schm15}, but they all are hard to verify in a general abstract setting. A compromise in this
direction is to require that the form $\fb$ is semibounded, see~\cite[Lemma~3.9]{Schm15} and~\cite[Lemma~2.7]{GKMSV17}, which
forces the diagonal form $\fa$ to be semibounded as well, see below. As in the proof of Theorem~\ref{thm:genSemibounded} above,
this simplifies the situation immensely:

\begin{proof}[Proof of Theorem~\ref{thm:offdiagForm}]
  Set $\fv_{\text{off}} := \fv|_{\Dom[\fa]}$, so that $\fb = \fa + \fv = \fa + \fv_{\text{off}}$. For
  $x \in \fD_- = \Ran P_- \cap \Dom[\fa]$ we have
  \begin{equation*}
    \fv_{\text{off}}[ x , x ]
    =
    \fv[ P_-x , P_-x ]
    =
    0
  \end{equation*}
  and, thus,
  \begin{equation*}
    \fb[ x , x ]
    =
    \fa[ x , x ]
    \le
    0
    .
  \end{equation*}
  In the same way, we see that $\fb[ x , x ] = \fa[ x , x ]$ for $x \in \fD_+$, which by the identity
  $\fa[ x , x ] = \fa[ P_+x , P_+x ] + \fa[ P_-x , P_-x ]$ for all $x \in \Dom[\fa]$ implies that along with $\fb$ the form $\fa$
  is semibounded as well; cf.~also the proof of~\cite[Lemma~2.7]{GKMSV17}. In particular, we have $\fD_- = \Ran P_-$ if $\fa$ is
  bounded from below and $\fD_+ = \Ran P_+$ if $\fa$ is bounded from above.

  Let $m \in \RR$ be the lower (resp.~upper) bound of $\fa$. We then have
  \begin{equation*}
    \abs{(\fa-m)[ x , x ]}
    =
    \norm{ \abs{A-m}^{1/2}x }^2
    \le
    \norm{ \abs{A-m}^{1/2}(\abs{A}^{1/2}+I)^{-1} } \norm{ (\abs{A}^{1/2}+I)x }^2
  \end{equation*}
  for all $x \in \Dom[\fa]$, where $\abs{A-m}^{1/2}(\abs{A}^{1/2}+I)^{-1}$ is closed and everywhere defined, hence bounded by the
  closed graph theorem. From this and the hypothesis on $\fv$ we see that
  \begin{equation*}
    \abs{ \fv[ x , x ] }
    \le
    \beta\bigl( \norm{\abs{A}^{1/2}x}^2 + \norm{x}^2 \bigr)
  \end{equation*}
  for some $\beta \ge 0$ and all $x \in \Dom[\fa]$, which means that $\fb = \fa + \fv$ is a semibounded saddle-point form in the
  sense of~\cite[Section~2]{GKMSV17}.

  Since $\Dom(\abs{B}^{1/2}) = \Dom[\fb] = \Dom[\fa] = \Dom(\abs{A}^{1/2})$ by hypothesis, $\cC = \Dom[\fa]$ is invariant for
  both $P_+$ and $Q_+$. Moreover, by~\cite[Theorem~3.3]{GKMSV17} (cf.~also~\cite[Theorem~2.13]{Schm15}) we have
  \begin{equation*}
    \Ker B
    \subset
    \Ker A
    \subset
    \Ran P_-
  \end{equation*}
  and $\norm{ P_+ - Q_+ } \le \sqrt{2}/2 < 1$. Taking into account that $\fa$ is semibounded as observed above,
  Lemma~\ref{lem:PQXY} with $\cC = \Dom[\fa] = \fD_+ \oplus \fD_-$ and identity~\eqref{eq:PQY} then imply as in the proof of
  Theorem~\ref{thm:genSemibounded} that $\Ran(P_+Q_+|_{\fD_+}) \supset \fD_+$. The claim now follows from
  Proposition~\ref{prop:GLS}\,(a) and Remark~\ref{rem:PQbij}\,(1).
\end{proof}%

%%%%%%%%%%%%%%%%%%%%%%%%%%%%%%%%%%%%%%%%%%%%%%%%%%%%%%%%%%%%%%%%%%%%%%%%%%%%%%%%%%%%%%%%%%%%%%%%%%%%%%%%%%%%%%%%%%%%%%%%%%%%%%%%%
%%%%%%%%%%%%%%%%%%%%%%%%%%%%%%%%%%%%%%%%%%%%%%%%%%%%%%%%%%%%%%%%%%%%%%%%%%%%%%%%%%%%%%%%%%%%%%%%%%%%%%%%%%%%%%%%%%%%%%%%%%%%%%%%%
%%%%%%%%%%%%%%%%%%%%%%%%%%%%%%%%%%%%%%%%%%%%%%%%%%%%%%%%%%%%%%%%%%%%%%%%%%%%%%%%%%%%%%%%%%%%%%%%%%%%%%%%%%%%%%%%%%%%%%%%%%%%%%%%%
\appendix

%%%%%%%%%%%%%%%%%%%%%%%%%%%%%%%%%%%%%%%%%%%%%%%%%%%%%%%%%%%%%%%%%%%%%%%%%%%%%%%%%%%%%%%%%%%%%%%%%%%%%%%%%%%%%%%%%%%%%%%%%%%%%%%%%
%%%%%%%%%%%%%%%%%%%%%%%%%%%%%%%%%%%%%%%%%%%%%%%%%%%%%%%%%%%%%%%%%%%%%%%%%%%%%%%%%%%%%%%%%%%%%%%%%%%%%%%%%%%%%%%%%%%%%%%%%%%%%%%%%
%%% Appendix: Heinz inequality
%%%%%%%%%%%%%%%%%%%%%%%%%%%%%%%%%%%%%%%%%%%%%%%%%%%%%%%%%%%%%%%%%%%%%%%%%%%%%%%%%%%%%%%%%%%%%%%%%%%%%%%%%%%%%%%%%%%%%%%%%%%%%%%%%
%%%%%%%%%%%%%%%%%%%%%%%%%%%%%%%%%%%%%%%%%%%%%%%%%%%%%%%%%%%%%%%%%%%%%%%%%%%%%%%%%%%%%%%%%%%%%%%%%%%%%%%%%%%%%%%%%%%%%%%%%%%%%%%%%
\section{Heinz inequality}\label{sec:heinz}

In this appendix we discuss some consequences of the well-known Heinz inequality. These consequences or particular cases thereof
are used at various spots of the main part of the paper, but they may also be of independent interest. Although probably
folklore, in lack of a suitable reference they are nevertheless presented here in full detail.

Throughout this appendix, we denote the norm associated with the inner product of a Hilbert space $\cH$ by
$\norm{\,\cdot\,}_\cH$.

The following variant of the Heinz inequality is taken from~\cite{Kre71}.

\begin{proposition}[{\cite[Theorem~I.7.1]{Kre71}}]\label{propHeinz}
  Let $\Lambda_1$ and $\Lambda_2$ be strictly positive self-adjoint operators on Hilbert spaces $\cH_1$ and $\cH_2$,
  respectively. Moreover, let $S\colon\cH_1\to\cH_2$ be a bounded operator mapping $\Dom(\Lambda_1)$ into $\Dom(\Lambda_2)$, and
  suppose that there is a constant $C\ge0$ such that
  \begin{equation*}
    \norm{\Lambda_2Sx}_{\cH_2}
    \le
    C\cdot\norm{\Lambda_1x}_{\cH_1}
    \quad\text{ for all }\quad
    x\in\Dom(\Lambda_1)
    .
  \end{equation*}
  Then, for all $\nu\in[0,1]$, the operator $S$ maps $\Dom(\Lambda_1^\nu)$ into $\Dom(\Lambda_2^\nu)$, and for all
  $x\in\Dom(\Lambda_1^\nu)$ one has
  \begin{equation*}
    \norm{\Lambda_2^\nu Sx}_{\cH_2}
    \le
    C^\nu\norm{S}_{\cH_1\to\cH_2}^{1-\nu}\norm{\Lambda_1^\nu x}_{\cH_1}
    .
  \end{equation*}
\end{proposition}

The above result admits the following extension to closed densely defined operators between Hilbert spaces. For a generalization
of Proposition~\ref{propHeinz} to maximal accretive operators, see~\cite{Kato61}.

\begin{proposition}\label{prop:heinz}
  Let $\cH_1$, $\cH_2$, $\cK_1$, and $\cK_2$ be Hilbert spaces, and let $\Lambda_1\colon\cH_1\supset\Dom(\Lambda_1)\to\cK_1$ and
  $\Lambda_2\colon\cH_2\supset\Dom(\Lambda_2)\to\cK_2$ be closed densely defined operators. Moreover, let $S\colon\cH_1\to\cH_2$
  be a bounded operator mapping $\Dom(\Lambda_1)$ into $\Dom(\Lambda_2)$, and suppose that there is a constant $C\ge0$ such that
  \begin{equation*}
    \norm{\Lambda_2Sx}_{\cK_2}
    \le
    C\cdot\norm{\Lambda_1x}_{\cK_1}
    \quad\text{ for all }\quad
    x\in\Dom(\Lambda_1)
    .
  \end{equation*}
  Then, for all $\nu\in[0,1]$, the operator $S$ maps $\Dom(\abs{\Lambda_1}^\nu)$ into $\Dom(\abs{\Lambda_2}^\nu)$.
\end{proposition}

\begin{proof}
  Recall that $\norm{\Lambda_jy}_{\cK_j}=\norm{\abs{\Lambda_j}y}_{\cH_j}$ for all $y\in\Dom(\Lambda_j)=\Dom(\abs{\Lambda_j})$,
  $j=1,2$. Moreover, the operator $S$ maps $\Dom(\abs{\Lambda_1}+I_{\cH_1})=\Dom(\Lambda_1)$ into
  $\Dom(\abs{\Lambda_2}+I_{\cH_2})=\Dom(\Lambda_2)$ by hypothesis. We estimate
  \begin{align*}
    \norm{(\abs{\Lambda_2}+I_{\cH_2})Sx}_{\cH_2}
    &\le
    \norm{\Lambda_2Sx}_{\cK_2} + \norm{Sx}_{\cH_2} \le C \norm{\Lambda_1x}_{\cK_1} + \norm{Sx}_{\cH_2}\\
    &\le
    \widetilde{C} \norm{(\abs{\Lambda_1}+I_{\cH_1})x}_{\cH_1}
  \end{align*}
  for all $x\in\Dom(\Lambda_1)$ with
  \begin{equation*}
    \widetilde{C}
    :=
    C\norm{\Lambda_1(\abs{\Lambda_1}+I_{\cH_1})^{-1}}_{\cH_1\to\cK_1} + \norm{S(\abs{\Lambda_1}+I_{\cH_1})^{-1}}_{\cH_1\to\cH_2}
    .
  \end{equation*}
  Here, we have taken into account that $\Lambda_1(\abs{\Lambda_1}+I_{\cH_1})^{-1}$ is a closed and everywhere defined operator
  from $\cH_1$ to $\cK_1$, hence bounded by the closed graph theorem.

  Applying Proposition~\ref{propHeinz} now yields that $S$ maps $\Dom((\abs{\Lambda_1}+I_{\cH_1})^\nu)$ into
  $\Dom((\abs{\Lambda_2}+I_{\cH_2})^\nu)$ for all $\nu\in[0,1]$. It remains to observe that by functional calculus one has
  $\Dom((\abs{\Lambda_j}+I_{\cH_j})^\nu)=\Dom(\abs{\Lambda_j}^\nu)$ for $j\in\{1,2\}$, which completes the proof.
\end{proof}%

We now obtain several easy corollaries.

\begin{corollary}[{cf.~\cite[Corollary 2.1.3]{SchmDiss}}]\label{cor:SchmDiss}
  Let $\cH$, $\cK_1$, and $\cK_2$ be Hilbert spaces, and let $\Lambda_1\colon\cH\supset\Dom(\Lambda_1)\to\cK_1$ and
  $\Lambda_2\colon\cH\supset\Dom(\Lambda_2)\to\cK_2$ be closed densely defined operators.

  If $\Dom(\Lambda_1) \subset \Dom(\Lambda_2)$, then $\Dom(\abs{\Lambda_1}^\nu) \subset \Dom(\abs{\Lambda_2}^\nu)$ for all
  $\nu \in [0,1]$. If even $\Dom(\Lambda_1) = \Dom(\Lambda_2)$, then also
  $\Dom(\abs{\Lambda_1}^\nu) = \Dom(\abs{\Lambda_2}^\nu)$ for all $\nu \in [0,1]$.
\end{corollary}

\begin{proof}
  Suppose that $\Dom(\Lambda_1) \subset \Dom(\Lambda_2)$. Since $\Dom(\abs{\Lambda_1}) = \Dom(\Lambda_1)$, we have as in the
  proof of the preceding proposition that $\Lambda_2(\abs{\Lambda_1}+I_\cH)^{-1}$ is a closed everywhere defined, hence bounded,
  operator from $\cH$ to $\cK_2$. Thus,
  \begin{equation*}
    \norm{\Lambda_2x}_{\cK_2}
    \le
    \norm{\Lambda_2(\abs{\Lambda_1}+I_\cH)^{-1}}_{\cH\to\cK_2} \cdot \norm{(\abs{\Lambda_1}+I_\cH)x}_{\cH}
  \end{equation*}
  for all $x\in\Dom(\Lambda_1)$, and applying Proposition~\ref{prop:heinz} with $S=I_\cH$ yields that
  \begin{equation*}
    \Dom(\abs{\Lambda_1}^\nu)
    =
    \Dom((\abs{\Lambda_1}+I_\cH)^\nu)\subset\Dom(\abs{\Lambda_2}^\nu)
    .
  \end{equation*}
  If also $\Dom(\Lambda_1) \supset \Dom(\Lambda_2)$, the above with switched roles of $\Lambda_1$ and $\Lambda_2$ yields that
  also $\Dom(\abs{\Lambda_2}^\nu) \subset \Dom(\abs{\Lambda_1}^\nu)$, which completes the proof.
\end{proof}%

\begin{remark}\label{rem:SchmDiss}
  For the particular case of $\nu = 1/2$, Corollary~\ref{cor:SchmDiss} can alternatively also be proved with classical
  considerations regarding operator and form boundedness:

  If $\Dom(\Lambda_1) \subset \Dom(\Lambda_2)$, then also $\Dom(\abs{\Lambda_1}) \subset \Dom(\abs{\Lambda_2})$, so that
  $\abs{\Lambda_2}$ is relatively operator bounded with respect to $\abs{\Lambda_1}$, see, e.g.,~\cite[Remark~IV.1.5]{Kato95}. In
  turn, by~\cite[Theorem~VI.1.38]{Kato95}, $\abs{\Lambda_2}$ is also form bounded with respect to $\abs{\Lambda_1}$, which
  extends to the closure of the forms. The latter includes that
  $\Dom(\abs{\Lambda_1}^{1/2}) \subset \Dom(\abs{\Lambda_2}^{1/2})$.
\end{remark}

\begin{corollary}\label{cor:fracBij}
  Let $\Lambda_1$ and $\Lambda_2$ be as in Proposition~\ref{prop:heinz}, and suppose that $S\colon\cH_1\to\cH_2$ is bounded and
  bijective with $\Ran(S|_{\Dom(\Lambda_1)})=\Dom(\Lambda_2)$. Then, one has
  $\Ran(S|_{\Dom(\abs{\Lambda_1}^\nu)})=\Dom(\abs{\Lambda_2}^\nu)$ for all $\nu\in[0,1]$.
\end{corollary}

\begin{proof}
  Consider the closed densely defined operator $\Lambda_3 := \Lambda_1S^{-1}$ with domain
  $\Dom(\Lambda_3) = \Ran(S|_{\Dom(\Lambda_1)}) = \Dom(\Lambda_2)$. By definition, $S$ maps $\Dom(\Lambda_1)$ onto
  $\Dom(\Lambda_3)$. Moreover, we have $\Lambda_3 Sx = \Lambda_1 x$ and, in particular,
  \begin{equation*}
    \norm{ \Lambda_3 Sx }_{\cK_1} = \norm{\Lambda_1x}_{\cK_1}
  \end{equation*}
  for all $x\in\Dom(\Lambda_1)$. Proposition~\ref{prop:heinz} now implies that $S$ maps $\Dom(\abs{\Lambda_1}^\nu)$ into
  $\Dom(\abs{\Lambda_3}^\nu)$. Since $\Dom(\abs{\Lambda_3}^\nu)=\Dom(\abs{\Lambda_2}^\nu)$ for all $\nu \in [0,1]$ in light of
  Corollary~\ref{cor:SchmDiss}, this proves the inclusion $\Ran(S|_{\Dom(\abs{\Lambda_1}^\nu)})\subset\Dom(\abs{\Lambda_2}^\nu)$.

  Since $S$ is bijective and $S^{-1}$ maps $\Dom(\Lambda_2)$ onto $\Dom(\Lambda_1)$ by hypothesis, one verifies in an analogous
  way that $S^{-1}$ maps $\Dom(\abs{\Lambda_2}^\nu)$ into $\Dom(\abs{\Lambda_1}^\nu)$. This shows the converse inclusion and,
  hence, completes the proof.
\end{proof}%

The last corollary discussed here is related to the question whether an operator sum agrees with the operator associated to the
sum of the corresponding forms, at least in the semibounded setting. Part~(b) of this corollary can in some sense also be
regarded as an extension of~\cite[Lemma~2.2.7]{SchmDiss} to not necessarily off-diagonal perturbations.

\begin{corollary}[{cf.~\cite[Lemma~2.2.7]{SchmDiss}}]\label{cor:formpert}
  Let $\Lambda$ be a self-adjoint operator on a Hilbert space with inner product $\scprod{\cdot , \cdot}$, and let $K$ be an
  operator on the same Hilbert space.

  \begin{enumerate}
    \renewcommand{\theenumi}{\alph{enumi}}

    \item
    If $K$ is symmetric with $\Dom(K) \supset \Dom(\abs{\Lambda}^{1/2})$, then the operator sum $\Lambda + K$ defines a
    self-adjoint operator with
    \begin{equation*}
      \scprod{ \abs{ \Lambda + K }^{1/2}x , \sign( \Lambda + K )\abs{ \Lambda + K}^{1/2}y }
      =
      \scprod{ \abs{\Lambda}^{1/2}x , \sign(\Lambda)\abs{\Lambda}^{1/2}y } + \scprod{ x, Ky }
    \end{equation*}
    for all $x,y \in \Dom(\abs{\Lambda}^{1/2}) = \Dom(\abs{\Lambda+K}^{1/2})$.

    \item
    If $K$ is self-adjoint and $\Lambda$-bounded with $\Lambda$-bound smaller than $1$, then $\Lambda + K$ is self-adjoint with
    \begin{multline*}
      \scprod{ \abs{ \Lambda + K }^{1/2}x , \sign( \Lambda + K )\abs{ \Lambda + K }^{1/2}y }\\
      =
      \scprod{ \abs{\Lambda}^{1/2}x , \sign(\Lambda)\abs{\Lambda}^{1/2}y } +
        \scprod{ \abs{K}^{1/2}x, \sign(K)\abs{K}^{1/2}y }
    \end{multline*}
    for all $x,y \in \Dom(\abs{\Lambda}^{1/2}) = \Dom(\abs{\Lambda+K}^{1/2})$.

  \end{enumerate}
\end{corollary}

\begin{proof}
  (a).
  Since $\Dom(\abs{\Lambda}^{1/2}) \subset \Dom(K)$ by hypothesis, Corollary~2.1.20 in~\cite{Tre08} yields that $K$ is operator
  infinitesimal with respect to $\Lambda$. In particular, the operator sum $\Lambda+K$ is self-adjoint on
  $\Dom(\Lambda+K) = \Dom(\Lambda)$ by the well-known Kato-Rellich theorem. In turn, Corollary~\ref{cor:SchmDiss} implies that
  $\Dom(\abs{\Lambda+K}^{1/2}) = \Dom(\abs{\Lambda}^{1/2})$. In particular, $\abs{\Lambda+K}^{1/2}$ is relatively operator
  bounded with respect to $\abs{\Lambda}^{1/2}$, see, e.g.,~\cite[Remark~IV.1.5 and Section~V.3.3]{Kato95}. Since the symmetric
  operator $K$ satisfies the inclusion $\Dom(\abs{\Lambda}^{1/2}) \subset \Dom(K)$ by hypothesis, $K$ is likewise relatively
  bounded with respect to $\abs{\Lambda}^{1/2}$.

  Now, for $x,y \in \Dom(\Lambda) = \Dom(\Lambda+K)$, both sides of the claimed identity clearly agree. For
  $x,y \in \Dom(\abs{\Lambda}^{1/2}) = \Dom(\abs{\Lambda+K}^{1/2})$, this identity then follows by approximation, taking into
  account that $\Dom(\Lambda)$ is an operator core for $\abs{\Lambda}^{1/2}$.

  (b).
  The operator sum $\Lambda+K$ with $\Dom(\Lambda+K) = \Dom(\Lambda)$ is self-adjoint by the well-known Kato-Rellich theorem, and
  Corollary~\ref{cor:SchmDiss} implies that $\Dom(\abs{\Lambda+K}^{1/2}) = \Dom(\abs{\Lambda}^{1/2})$ and
  $\Dom(\abs{\Lambda}^{1/2}) \subset \Dom(\abs{K}^{1/2})$. In turn, as in part~(a), both $\abs{\Lambda+K}^{1/2}$ and
  $\abs{K}^{1/2}$ are relatively bounded with respect to $\abs{\Lambda}^{1/2}$. The claimed identity now follows just as in
  part~(a) by approximation upon observing that it certainly holds for $x,y \in \Dom(\Lambda)$.
\end{proof}%

%%%%%%%%%%%%%%%%%%%%%%%%%%%%%%%%%%%%%%%%%%%%%%%%%%%%%%%%%%%%%%%%%%%%%%%%%%%%%%%%%%%%%%%%%%%%%%%%%%%%%%%%%%%%%%%%%%%%%%%%%%%%%%%%%
%%%%%%%%%%%%%%%%%%%%%%%%%%%%%%%%%%%%%%%%%%%%%%%%%%%%%%%%%%%%%%%%%%%%%%%%%%%%%%%%%%%%%%%%%%%%%%%%%%%%%%%%%%%%%%%%%%%%%%%%%%%%%%%%%
%%% Acknowledgements
%%%%%%%%%%%%%%%%%%%%%%%%%%%%%%%%%%%%%%%%%%%%%%%%%%%%%%%%%%%%%%%%%%%%%%%%%%%%%%%%%%%%%%%%%%%%%%%%%%%%%%%%%%%%%%%%%%%%%%%%%%%%%%%%%
%%%%%%%%%%%%%%%%%%%%%%%%%%%%%%%%%%%%%%%%%%%%%%%%%%%%%%%%%%%%%%%%%%%%%%%%%%%%%%%%%%%%%%%%%%%%%%%%%%%%%%%%%%%%%%%%%%%%%%%%%%%%%%%%%
\section*{Acknowledgements}
The author is grateful to Ivan Veseli\'c, Matthias T\"aufer and Stephan Schmitz for fruitful and inspiring discussions. He is
especially indebted to Stephan Schmitz for also commenting on an earlier version of this manuscript.

%%%%%%%%%%%%%%%%%%%%%%%%%%%%%%%%%%%%%%%%%%%%%%%%%%%%%%%%%%%%%%%%%%%%%%%%%%%%%%%%%%%%%%%%%%%%%%%%%%%%%%%%%%%%%%%%%%%%%%%%%%%%%%%%%
%%%%%%%%%%%%%%%%%%%%%%%%%%%%%%%%%%%%%%%%%%%%%%%%%%%%%%%%%%%%%%%%%%%%%%%%%%%%%%%%%%%%%%%%%%%%%%%%%%%%%%%%%%%%%%%%%%%%%%%%%%%%%%%%%
%%% Bibliography
%%%%%%%%%%%%%%%%%%%%%%%%%%%%%%%%%%%%%%%%%%%%%%%%%%%%%%%%%%%%%%%%%%%%%%%%%%%%%%%%%%%%%%%%%%%%%%%%%%%%%%%%%%%%%%%%%%%%%%%%%%%%%%%%%
%%%%%%%%%%%%%%%%%%%%%%%%%%%%%%%%%%%%%%%%%%%%%%%%%%%%%%%%%%%%%%%%%%%%%%%%%%%%%%%%%%%%%%%%%%%%%%%%%%%%%%%%%%%%%%%%%%%%%%%%%%%%%%%%%
\begin{thebibliography}{[10]}

  \bibitem{AL95} V.~M.~Adamjan, H.~Langer,
    \emph{Spectral properties of a class of rational operator valued functions},
    J.~Operator Theory~\textbf{33} (1995), 259--277.

  \bibitem{DK70} C.~Davis, W.~M.~Kahan,
    \emph{The rotation of eigenvectors by a perturbation.~III},
    SIAM J.~Numer.~Anal.~\textbf{7} (1970), 1--46.  

  \bibitem{DES00} J.~Dolbeault, M.~J.~Esteban, E.~S\'er\'e,
    \emph{On the eigenvalues of operators with gaps. Application to Dirac Operators},
    J.~Funct.~Anal.~\textbf{174} (2000), 208--226.

  \bibitem{DES06} J.~Dolbeault, M.~J.~Esteban, E.~S\'er\'e,
    \emph{General results on the eigenvalues of operators with gaps, arising from both ends of the gaps. Application to Dirac
      operators},
    J.~Eur.~Math.~Soc.~(JEMS) \textbf{8} (2006), 243--251.

  \bibitem{FFMM00} M.~Faierman, R.~J.~Fries, R.~Mennicken, M.~M\"oller,
    \emph{On the essential spectrum of the linearized Navier-Stokes operator},
    Integral Equations Operator Theory~\textbf{38} (2000), 9--27.

  \bibitem{GKMV13} L.~Grubi\v{s}i\'c, V.~Kostrykin, K.~A.~Makarov, K.~Veseli\'c,
    \emph{The $\Tan 2\Theta$ Theorem for indefinite quadratic forms},
    J.~Spectr.~Theory~\textbf{3} (2013), 83--100.

  \bibitem{GKMSV17} L.~Grubi\v{s}i\'c, V.~Kostrykin, K.~A.~Makarov, S.~Schmitz, K.~Veseli\'c,
    \emph{Diagonalization of indefinite saddle point forms},
    In: Analysis as a Tool in Mathematical Physics: in Memory of Boris Pavlov,
    Oper.~Theory Adv.~Appl., vol.~276, Birkh\"auser, Basel, 2020, pp.~373--400.

  \bibitem{GKMSV19} L.~Grubi\v{s}i\'c, V.~Kostrykin, K.~A.~Makarov, S.~Schmitz, K.~Veseli\'c,
    \emph{The $\Tan2\Theta$ Theorem in fluid dynamics},
    J.~Spectr.~Theory~\textbf{9} (2019), 1431--1457.

  \bibitem{GLS99} M.~Griesemer, R.~T.~Lewis, H.~Siedentop,
    \emph{A minimax principle for eigenvalues in spectral gaps: Dirac operators with Coulomb potentials},
    Doc.~Math.~\textbf{4} (1999), 275--283.

  \bibitem{Kato61} T.~Kato,
    \emph{A generalization of the Heinz inequality},
    Proc.~Japan Acad.~\textbf{37} (1961), 305--308.

  \bibitem{Kato95} T.~Kato,
    \emph{Perturbation Theory for Linear Operators},
    Classics Math., Springer, Berlin, 1995.

  \bibitem{KMM03:181} V.~Kostrykin, K.~A.~Makarov, A.~K.~Motovilov,
    \emph{Existence and uniqueness of solutions to the operator Riccati equation. A geometric approach},
    In: Advances in Differential Equations and Mathematical Physics (Birmingham, AL, 2002),
    Contemp.~Math., vol.~327, Amer.~Math.~Soc., Providence, RI, 2003, pp.~181--198.

  \bibitem{Kre71} S.~G.~Kre{\u\i}n,
    \emph{Linear Differential Equations in Banach Space},
    Transl.~Math.~Monogr., vol.~29, Amer.~Math.~Soc., Providence, RI, 1969.

  \bibitem{LL01} E.~H.~Lieb, M.~Loss,
    \emph{Analysis}, second edition,
    Grad.~Stud.~Math., vol.~14, Amer.~Math.~Soc., Providence, RI, 2001.

  \bibitem{LS16} M.~Langer, M.~Strauss,
    \emph{Triple variational principles for self-adjoint operator functions},
    J.~Funct.~Anal.~\textbf{270} (2016), 2019--2047.

  \bibitem{MSS16} K.~A.~Makarov, S.~Schmitz, A.~Seelmann,
    \emph{On invariant graph subspaces},
    Integral Equations Operator Theory~\textbf{85} (2016), 399--425.

  \bibitem{MM15} S.~Morozov, D.~M\"uller,
    \emph{On the minimax principle for Coulomb-Dirac operators},
    Math.~Z.~\textbf{280} (2015), 733--747.

  \bibitem{MS06} A.~K.~Motovilov, A.~V.~Selin,
    \emph{Some sharp norm estimates in the subspace perturbation problem},
    Integral Equations Operator Theory~\textbf{56} (2006), 511--542.

  \bibitem{NSTTV18} I.~Naki\'c, M.~T\"aufer, M.~Tautenhahn, I.~Veseli\'c,
    \emph{Unique continuation and lifting of spectral band edges of Schr\"odinger operators on unbounded domains},
    with an appendix by Albrecht Seelmann,
    to appear in J.~Spectr.~Theory. E-print arXiv:1804.07816 [math.SP] (2018).

  \bibitem{Schm12} K.~Schm\"udgen,
    \emph{Unbounded Self-Adjoint Operators on Hilbert Space},
    Grad.~Texts in Math., vol.~265, Springer, Dordrecht, 2012.  

  \bibitem{SchmDiss} S.~Schmitz,
    \emph{Representation theorems for indefinite quadratic forms and applications},
    Dissertation, Johannes Gutenberg-Universit\"at Mainz, 2014.

  \bibitem{Schm15} S.~Schmitz,
    \emph{Representation theorems for indefinite quadratic forms without spectral gap},
    Integral Equations Operator Theory~\textbf{83} (2015), 73--94.

  \bibitem{SeelDiss} A.~Seelmann,
    \emph{Perturbation theory for spectral subspaces},
    Dissertation, Johannes Gutenberg-Universit\"at Mainz, 2014.

  \bibitem{Seel19} A.~Seelmann,
    \emph{Semidefinite perturbations in the subspace perturbation problem},
     J.~Operator Theory~\textbf{81} (2019), 321--333.

	\bibitem{Seel20} A.~Seelmann,
		\emph{Unifying the treatment of indefinite and semidefinite perturbations in the subspace perturbation problem},
		e-print arXiv:2006.16102 [math.SP] (2020). (submitted)

  \bibitem{Tre08} C.~Tretter,
    \emph{Spectral Theory of Block Operator Matrices and Applications},
    Imperial College Press, London, 2008.

\end{thebibliography}

\end{document}

%%%%%%%%%%%%%%%%%%%%%%%%%%%%%%%%%%%%%%%%%%%%%%%%%%%%%%%%%%%%%%%%%%%%%%%%%%%%%%%%%%%%%%%%%%%%%%%%%%%%%%%%%%%%%%%%%%%%%%%%%%%%%%%%%
%%%%%%%%%%%%%%%%%%%%%%%%%%%%%%%%%%%%%%%%%%%%%%%%%%%%%%%%%%%%%%%%%%%%%%%%%%%%%%%%%%%%%%%%%%%%%%%%%%%%%%%%%%%%%%%%%%%%%%%%%%%%%%%%%
%%%%%%%%%%%%%%%%%%%%%%%%%%%%%%%%%%%%%%%%%%%%%%%%%%%%%%%%%%%%%%%%%%%%%%%%%%%%%%%%%%%%%%%%%%%%%%%%%%%%%%%%%%%%%%%%%%%%%%%%%%%%%%%%%
%%% End of file
%%%%%%%%%%%%%%%%%%%%%%%%%%%%%%%%%%%%%%%%%%%%%%%%%%%%%%%%%%%%%%%%%%%%%%%%%%%%%%%%%%%%%%%%%%%%%%%%%%%%%%%%%%%%%%%%%%%%%%%%%%%%%%%%%
%%%%%%%%%%%%%%%%%%%%%%%%%%%%%%%%%%%%%%%%%%%%%%%%%%%%%%%%%%%%%%%%%%%%%%%%%%%%%%%%%%%%%%%%%%%%%%%%%%%%%%%%%%%%%%%%%%%%%%%%%%%%%%%%%
%%%%%%%%%%%%%%%%%%%%%%%%%%%%%%%%%%%%%%%%%%%%%%%%%%%%%%%%%%%%%%%%%%%%%%%%%%%%%%%%%%%%%%%%%%%%%%%%%%%%%%%%%%%%%%%%%%%%%%%%%%%%%%%%%