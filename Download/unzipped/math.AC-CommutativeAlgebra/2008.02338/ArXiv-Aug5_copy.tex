\documentclass[12pt]{amsart}
 \usepackage{amssymb,amsmath,amstext}
\usepackage{amsthm,amsmath,amsfonts,amssymb,amscd,latexsym,stmaryrd}
\usepackage[colorlinks=false, urlcolor=blue, linkcolor=blue,
citecolor=blue]{hyperref} 
\usepackage[margin=1in]{geometry}
%\usepackage{hyperref}
\usepackage{color}
\usepackage[usenames,dvipsnames]{xcolor}
\usepackage{colortbl}
\usepackage[all,cmtip]{xy}
\usepackage{graphicx}
\usepackage{ytableau}
\usepackage{arydshln}
\usepackage{blkarray}
\usepackage{arydshln}
\usepackage {tikz}
\usepackage{graphicx}
\usepackage {cite}
\usepackage{comment}
\usetikzlibrary{arrows,decorations.pathmorphing,backgrounds,positioning,fit,matrix}
\definecolor{processblue}{cmyk}{0.96,0,0,0} \usepackage{enumerate}
\input{xy} \xyoption{all} \usepackage{xypic}
\hypersetup{
  colorlinks   = true, %Colours links instead of ugly boxes
  urlcolor     = blue, %Colour for external hyperlinks
  linkcolor    = blue, %Colour of internal links
  citecolor   = blue %Colour of citations
}
\setlength{\oddsidemargin}{0.0in} \setlength{\evensidemargin}{0.0in}
\setlength{\textwidth}{6.5in} \setlength{\parskip}{0.15cm}
\setlength{\parindent}{0.5cm}

\numberwithin{equation}{section}

% \swapnumbers Theorem environments with italic font
\theoremstyle{plain} \newtheorem{theorem}{Theorem}[section]
\newtheorem{proposition}[theorem]{Proposition} \newtheorem{lemma}[theorem]{Lemma}
\newtheorem{corollary}[theorem]{Corollary}

% Theorem environments with roman or slanted font
\theoremstyle{definition} \newtheorem{definition}[theorem]{Definition}
\newtheorem{example}[theorem]{Example} \newtheorem{remark}[theorem]{Remark}

% New math operators New math operators

\DeclareMathOperator{\Hess}{Hess}\DeclareMathOperator{\rk}{rk}
\DeclareMathOperator{\ann}{Ann}\DeclareMathOperator{\Char}{char}
\DeclareMathOperator{\Cat}{Cat}\DeclareMathOperator{\B}{\mathcal{B}}
\DeclareMathOperator{\K}{\mathsf{k}}
\DeclareMathOperator{\diag}{\mathrm{diag}}

\setcounter{MaxMatrixCols}{20}

% -------------------------------------------------------------------------------------------------------------------------%
\begin{document}

\author{Nasrin Altafi} \address{Department of Mathematics, KTH Royal
  Institute of Technology, S-100 44 Stockholm, Sweden}
\email{nasrinar@kth.se}


\title[]{Jordan types with small parts for Artinian Gorenstein algebras of codimension three}\keywords{Artinian Gorenstein algebra, Hilbert function, catalecticant matrix,  Hessians,  Macaulay dual  generators, Jordan type, partition.}
\subjclass[2010]{Primary: 13E10, 13D40; Secondary: 13H10, 05A17, 05E40.}
\maketitle

\begin{abstract}
We study Jordan types of linear forms for  graded Artinian  Gorenstein algebras with arbitrary codimension. We introduce rank matrices of linear forms for such algebras which represent the ranks of multiplication maps in various degrees. Rank matrices correspond to Jordan degree types. For Artinian Gorenstein algebras with codimension three we classify all rank matrices which occur for linear forms with vanishing third power. As a consequence for such algebras we show that Jordan types with parts of length at most four are uniquely determined by at most three parameters.
\end{abstract}


\section{Introduction}
The Jordan type of a graded Artinian algebra $A$ and linear form $\ell$ is a partition determining the Jordan block decomposition for the (nilpotent) multiplication map by $\ell$ on $A$ which is denoted by $P_{\ell,A}=P_\ell$. Jordan type determines the weak and strong Lefschetz properties of Artinian  algebras. A graded Artinian algebra $A$ is said to satisfy the weak Lefschetz property (WLP) if multiplication map by a linear form on $A$ has maximal rank in every degree. It is said to satisfy the strong Lefschetz property (SLP) if multiplication maps by all powers of a linear form on $A$ has maximal rank in every degree. It is known that an Artinian  algebra $A$ has the weak Lefschetz property if there is a linear form $\ell$ where the number of parts in $P_\ell$ is equal to the Sperner number of $A$, the maximum value of the Hilbert function $h_A$. Also $A$ has the strong Lefschetz property if there is a linear form $\ell$ such that $P_\ell = h^\vee_A$ the conjugate partition of $h_A$ see \cite{IMM}.
Jordan type of a linear form for Artinian  algebras captures more information than the weak and Strong Lefschetz properties. Recently, there has been studies about Jordan types of Artinian  algebras also in more general settings, see \cite{IMM, IMM2, IKVZ} and their references.  
Studying Artinian    Gorenstein algebras is of great interest among the researchers in the area.   Gorenstein algebras are Poincar\'e duality algebras \cite{MW} and thus natural algebraic objects to cohomology rings of smooth complex projective varieties. 
There has been many studies in the Lefschetz properties and Jordan types of Artinian    Gorenstein algebras \cite{CG, HW, GZ, Gondim, MH}.
  Gorenstein algebras of codimension two are  complete intersections and they all satisfy the SLP.  The list of all possible Jordan types of linear forms, not necessarily generic linear forms, for complete intersection algebras of codimension two is provided in \cite{CIJT}. 
   
In this article we  study the ranks of multiplication maps by linear forms on graded Artinian  Gorenstein algebras that are quotients of polynomial ring $S = \K[x_1,\dots ,x_n]$ where $\K$ is a field of characteristic zero. In Section \ref{section-rksec},  we study such algebras with arbitrary codimension in terms of their Jordan types. We present an approach to determine the Jordan types of Artinian    Gorenstein algebras using Macaulay duality. We assign a natural invariant to an Artinian    Gorenstein algebra $A$ providing the ranks of multiplication maps by a linear form $\ell$ in different degrees, called \emph{rank matrix}, $M_{\ell,A}$,  Definition \ref{rkmatrix-def}. There is a one-to-one correspondence between rank matrices and so called \emph{Jordan degree types} in Proposition \ref{rkmatrix-1-1-JDT-prop}. We provide necessary conditions for a rank matrix in Lemmas \ref{diffO-seq} and \ref{additiveRank}.
We use this approach in Section  \ref{codim3section} for Artinian Gorenstein algebras in polynomial rings with three variables. We give a complete list of rank matrices which occur for some Artinian Gorenstein algebra $A$ and linear form $\ell$ where $\ell^3=0$ and $\ell^2\neq 0$, see Theorems \ref{3linesHFtheorem-even} and \ref{3linesHFtheorem-odd} for algebras with even and odd socle degrees respectively. As an immediate consequence in Corollary \ref{2linescorollary} we list rank matrices for linear forms where $\ell^2=0$. 
In Theorem \ref{JT-theorem} we prove that the Jordan types of Artinian Gorenstein algebras with codimension three and linear forms $\ell$ where $\ell^4=0$ is uniquely determined by the ranks of at most three multiplication maps or equivalently three mixed Hessians.
\section{Preliminaries}
Let $S = \K[x_1,\dots ,x_n]$ be a polynomial ring with standard grading over a field $\K$ of characteristic zero. Let $A=S/I$ be a graded Artinian  ( its Krull dimension is zero) algebra where $I$ is an homogeneous ideal. 
The \emph{Hilbert function} of a graded Artinian  algebra $A=S/I$ is  a vector of non-negative integers and we denote it by $h_A=(1,h_1,\dots ,h_d)$ where  $h_A(i)=h_i=\dim_{\K}(A_i)$. Integer $d$ is called the \emph{socle degree} of $A$, that is the largest integer $i$ such that $h_A(i)>0$. Artinian  graded algebra $A$ is \emph{  Gorenstein} if $h_d=1$ and its Hilbert function is symmetric, i.e. $h_A(i)=h_A(d-i)$ for $0\leq i\leq d$.\par 
A famous result of F.H.S. Macaulay \cite{Macaulay}  provides a bound on the growth of Hilbert functions of Artinian  graded algebras. F.H.S. Macaulay characterizes all vectors of non-negative integers which occur as Hilbert functions of standard graded algebras. Such a sequence is called an \emph{O-sequence}. \par 
Let $R=\K[X_1,\dots , X_n ]$ be the Macualay dual ring of $S$. Given a homogeneous ideal $I\subset S$ the \emph{inverse system} of $I$ is defined to be a graded $S$-module $M\subset R$ such that $S$ acts on  $R$ by differentiation. For more details of Macaulay's inverse system see \cite{Geramita} and \cite{IK}.
For graded Artinian    Gorenstein algebras the inverse system is generated by only one form.
\begin{theorem}\cite{MW}\label{dualGen}
 Let $A=S/I$ be a graded Artinian  algebra. Then $A$ is   Gorenstein if and only if there exists a polynomial $F\in R =\K[X_1,\dots ,X_n]$ such that $I=\ann_S(F)$.
 \end{theorem}
From  a  result by F.H.S. Macaulay \cite{F.H.S} it is known that an Artinian standard graded $\mathsf{k}$-algebra $A=S/I$ is Gorenstein if and only if there exists $F\in R_d$, such that $I=\ann_S(F)$.  T. Maeno and J. Watanabe \cite{MW} described higher Hessians of dual generator $F$ and provided a criterion for Artinian Gorenstein algebras having the SLP or WLP
\begin{definition}\cite[Definition 3.1]{MW}
Let $F$ be a polynomial in $R$ and $A= S/\ann_S(F)$ be its associated Artinian Gorenstein algebra. Let $\mathcal{B}_{j} = \lbrace \alpha^{(j)}_i+\ann_S(F) \rbrace_i$ be a  $\mathsf{k}$-basis of $A_j$. The entries of the  $j$-th Hessian matrix of $F$ with respect to $\mathcal{B}_j$ are given  by
$$
(\Hess^j(F))_{u,v}=(\alpha^{(j)}_u\alpha^{(j)}_v \circ F).
$$
We note that when $j=1$ the form $\Hess^1(F)$ coincides with the usual Hessian. Up to  a non-zero constant multiple  $\det \Hess^j(F)$ is independent of the basis $\mathcal{B}_j$.  By abusing notation we will write   $\mathcal{B}_{j} = \lbrace \alpha^{(j)}_i \rbrace_i$ for a basis of $A_j$.
\end{definition}
R. Gondim and G. Zappal\`a \cite{GZ} introduced a generalization of Hessians which provides the rank of multiplication maps by powers a linear form which are not necessarily symmetric.
\begin{definition}\cite[Definition 2.1]{GZ}
Let $F$ be a polynomial in $R$ and $A= S/\ann_S(F)$ be its associated Gorenstein algebra. Let $\mathcal{B}_{j} = \lbrace \alpha^{(j)}_i\rbrace_i$ and $\mathcal{B}_{k} = \lbrace \beta^{(k)}_i\rbrace_i$ be $\mathsf{k}$-bases of $A_j$ and $A_k$ respectively. The \emph{Hessian matrix of order $(j,k)$} of $F$ with respect to $\mathcal{B}_j$  and $\mathcal{B}_k$ is 
$$
(\Hess^{(j,k)}(F))_{u,v}=(\alpha^{(j)}_u\beta^{(k)}_v \circ F).
$$
When $j=k$, $\Hess^{(j,j)}(F)=\Hess^{j}(F)$. 
\end{definition}
\begin{definition}
 Let $A= S/\ann(F)$ where $F\in R_d$. Pick bases $\B_j = \lbrace \alpha^{(j)}_u\rbrace_u$ and $\B_{d-j} = \lbrace \beta^{(d-j)}_u\rbrace_u$ be $\K$-bases of $A_j$ and $A_{d-j}$ respectively. The \emph{catalecticant matrix of $F$} with respect to $\B_j$ and $\B_{d-j}$ is 
$$
\Cat^j_F=(\alpha^{(j)}_u\beta^{(d-j)}_v F)_{u,v=1}.
$$
\end{definition}
The rank of the $j$-th catalecticant matrix of $F$ is equal to the Hilbert function of $A$ in degree $j$, see \cite[Definition 1.11]{IK}.

\section{Rank matrices for Artinian    Gorenstein algebras of linear forms}\label{section-rksec}
Throughout this section let $S=\K[x_1,\dots, x_n]$ be a polynomial ring with $n\geq 2$ variables equipped with standard grading over a filed  $\K$ of characteristic zero. We let $A=S/\ann(F)$ be a graded Artinian Gorenstein algebra with dual generator $F\in R=\K[X_1,\dots , X_n ]$ that is a homogeneous polynomial of degree $d\geq 2$.
\begin{definition}\label{rkmatrix-def}
Let $A=S/\ann(F)$ be an Artinian    Gorenstein algebra with socle degree $d$. For linear form $\ell\in A$ define the \textit{rank matrix} of $A$ and $\ell$ to be  an upper triangular square matrix of size  $d+1$ and denote it by $M_{\ell,A}$ such that its $i,j$-th entry is 
$$(M_{\ell,A})_{i,j} = \rk\left(\times \ell^{j-i} : A_i\longrightarrow A_j\right),$$
for every $i\leq j$. For $i>j$ we set $ (M_{\ell,A})_{i,j}=0$.
\end{definition}

\begin{definition}
Let $A=S/\ann(F)$  be an Artinian    Gorenstein algebra  with socle degree $d$ and linear form $\ell$. For each $0\leq i\leq d$ define Artinian Gorenstein algebra with dual generator $\ell^i\circ F$ and denote it by
$$
A^{(i)} := S/\ann(\ell^i\circ F).
$$
\end{definition}
\noindent We note that when $i=0$ the algebra $A^{(0)}$ coincides with $A$. \par
\noindent 
\begin{remark}\label{r_ij-remark}
By the definition of higher and mixed Hessians for every $0\leq i<j$ we have that 
\begin{equation}
\rk \Hess^{(i,d-j)}_\ell (F) = (M_{\ell,A})_{i,j}.
\end{equation}
\end{remark} 
For each $0\leq i\leq d$ denote the $i$-the diagonal vector of $M_{\ell,A}$ by $\diag(i,M_{\ell,A})$,
$$\diag(i,M_{\ell,A}):=((M_{\ell,A})_{0,i},(M_{\ell,A})_{1,i+1},\dots ,(M_{\ell,A})_{d-i,d}).$$
We show that for every $0\leq i\leq d$ the vector $\diag(i,M_{\ell,A})$ is the Hilbert function of some Artinian    Gorenstein algebra.
We denote the Macaulay inverse system module of $A=S/\ann(F)$ by $\langle F\rangle$.
\begin{proposition}\label{diagprop}
Let $A=S/\ann(F)$ be an Artinian    Gorenstein algebra with socle degree $d\geq 2$ and $\ell$ be a linear form. For each $0\leq i\leq d$
$$
\diag(i,M_{\ell,A}) = h_{A^{(i)}}.
$$
\end{proposition}
\begin{proof}
By the definition of rank matrix $M_{\ell,A}$ we have that the entries on the $i$-th diagonal of $M_{\ell,A}$ are exactly the ranks of multiplication map by $\ell^i$ on $A$ in various degrees.
Using Macaulay duality for every $0\leq j\leq \lfloor\frac{d-i}{2}\rfloor$ we get the following 
\begin{align*}
rk\left(\times \ell^{i} : A_j\longrightarrow A_{i+j}\right)&=rk\left(\circ \ell^{i} : \langle F\rangle_{i+j}\longrightarrow \langle F\rangle_{j}\right)\\&=\dim_{\K} \langle \ell^i\circ F \rangle_j\\
&=\dim_{\K}(S/\ann(\ell^i\circ F))_{j}.
\end{align*}%\footnote{check it}
Note that the socle degree of $A^{(i)}$ is equal to $d-i$. The proof is complete since $h_{A^{(i)}}$ is symmetric about $\lfloor\frac{d-i}{2}\rfloor$.
\end{proof}
\begin{example}\label{firstEx}
Let $A=\K[x_1,x_2,x_3]/\ann(F)$ where $F=X_1^2X_2^2X_3^2$ and $\ell=x_1$. Then we have that 
$h_A=\left(1,3,6,7,6,3,1\right).$ We also observe that
$$h_{A^{(1)}}=h_{S/\ann(x_1\circ F)}=\left(1,3,5,5,3,1\right),\quad 
h_{A^{(2)}}=h_{S/\ann(x_1^2\circ F)}=\left(1,2,3,2,1\right),
$$
and $x^i\circ F=0$ for  $i\geq 3$.  Then the rank matrix is as follows 
$$M_{\ell,A}= \begin{pmatrix}
1&1&1&0&0&0&0\\
0&3&3&2&0&0&0\\
0&0&6&5&3&0&0\\
0&0&0&7&5&2&0\\
0&0&0&0&6&3&1\\
0&0&0&0&0&3&1\\
0&0&0&0&0&0&1\\
\end{pmatrix}.
$$
By Remark \ref{r_ij-remark} we have that
\begin{align*}
&\rk \Hess_{x_1}^{(0,5)}=\rk \Hess_{x_1}^{(0,4)}=1, \rk \Hess_{x_1}^{(1,4)}=3,\rk \Hess_{x_1}^{(1,3)}=2,\\
 & \rk \Hess_{x_1}^{(2,3)}=5,\hspace*{2mm}\text{and}\hspace*{2mm} \rk \Hess_{x_1}^{(2,2)}=3.
\end{align*}
\end{example}

In the following two lemmas we provide necessary conditions for an upper triangular square matrix of size $d+1$ with non-negative integers to occur for an Artinian Gorenstein algebra $A$ and linear form $\ell\in A_1$. First we need a notation. For a vector $\mathbf{v}$ of positive integers of length $l$ denote by $\mathbf{v}_+$ the vector of length $l+1$ obtained by adding zero to vector $\mathbf{v}$, that is $\mathbf{v}_+ = (0,\mathbf{v})$.

\begin{lemma}\label{diffO-seq}
For every $0\leq i\leq d-1$ the difference vector $ h_{A^{(i)}}-(h_{A^{(i+1)}})_+$ is an O-sequence.
\end{lemma}
\begin{proof}% \footnote{more detailed proof}
Using Macaulay duality, for every $j\geq 1$ we have 
\begin{align*}
& h_{A^{(i)}}(j)-h_{A^{(i+1)}}(j-1) = \dim_{\K}\langle \ell^i\circ F\rangle_j -\dim_{\K}\langle \ell^{i+1}\circ F\rangle_{j-1}=\dim_{\K}\left(\langle \ell^i\circ F\rangle/\langle \ell^{i+1}\circ F\rangle\right)_{j}.
\end{align*}
For $j=0$ we have that $\dim_{\K}\left(\langle \ell^i\circ F\rangle/\langle \ell^{i+1}\circ F\rangle\right)_{0}=1$ if ${A^{(i)}}\neq 0$. If ${A^{(i)}}=0$ then clearly ${A^{(i+1)}}= 0$ and    $ h_{A^{(i)}}-(h_{A^{(i+1)}})_+$ is the zero vector.\par 
\noindent  We conclude that $h_{A^{(i)}}-(h_{A^{(i+1)}})_+$ is the Hilbert function of $\left(\langle \ell^i\circ F\rangle/\langle \ell^{i+1}\circ F\rangle\right)$ and hence is an O-sequence.
\end{proof}
\begin{lemma}\label{additiveRank}
For every $i,j\geq 1$ we have the following
$$
h_{A^{(i-1)}}(j)+h_{A^{(i+1)}}(j-1)\geq h_{A^{(i)}}(j)+h_{A^{(i)}}(j-1).
$$
\end{lemma}
\begin{proof}
The inclusion map $\langle \ell^{i+1}\circ F\rangle\hookrightarrow \langle \ell^{i}\circ F\rangle$ for every $i\geq 0$ induces the following commutative diagram
$$
\xymatrix{
0\ar[r]& \langle \ell^{i+1}\circ F\rangle \ar[d]\ar[r]&\langle \ell^{i}\circ F\rangle  \ar[d]\ar[r]& \langle \ell^{i}\circ F\rangle/ \langle \ell^{i+1}\circ F\rangle \ar[d]^{\varphi}\ar[r]&0\\
0\ar[r]& \langle \ell^i\circ F\rangle\ar[r]&\langle \ell^{i-1}\circ F\rangle\ar[r]& \langle \ell^{i-1}\circ F\rangle/ \langle \ell^i\circ F\rangle\ar[r]&0\\
}
$$
which shows that $\varphi$ is also injective.
Using Lemma \ref{diffO-seq} we get that $h_{A^{(i)}}(j)-h_{A^{(i+1)}}(j-1) = \dim_{\K}\left(\langle \ell^i\circ F\rangle/\langle \ell^{i+1}\circ F\rangle\right)_{j},$ for every $i,j\geq 1$ which implies the desired inequality.
\end{proof}
\begin{remark}
The above lemma shows that for every $i,j\geq 1$ the following inequality holds
$$
\rk \Hess_\ell^{(j,d-i-j+1)}+\rk \Hess_\ell^{(j-1,d-i-j)}\geq \rk \Hess_\ell^{(j,d-i-j)}+\rk \Hess_\ell^{(j-1,d-i-j+1)}
$$
\end{remark}
\begin{example}
Using Lemma \ref{diffO-seq} we get that the following matrix does not occur as the rank matrix of some Artinian Gorenstein algebra and linear form $\ell$.
$$ \begin{pmatrix}
1&1&1&0&0&0\\
0&3&2&2&0&0\\
0&0&3&3&2&0\\
0&0&0&3&2&1\\
0&0&0&0&3&1\\
0&0&0&0&0&1\\
\end{pmatrix}.
$$
Since $(1,3,3,3,3,1)-(0,1,2,3,2,1)=(1,2,1,0,1,0)$ is not an O-sequence.\par 
Lemma \ref{additiveRank} also implies that the following matrix is not a possible rank matrix for some $A$ and $\ell$.
$$ \begin{pmatrix}
1&1&1&0&0&0\\
0&3&3&1&0&0\\
0&0&5&4&1&0\\
0&0&0&5&3&1\\
0&0&0&0&3&1\\
0&0&0&0&0&1\\
\end{pmatrix}.
$$
In fact we have 
\begin{align*}
(1,3,5,5,3,1)-(0,1,3,4,3,1)=(1,2,2,1)\ngeqq (1,2,3,2)= (1,3,4,3,1)-(0,1,1,1,1).
\end{align*}
\end{example}
\begin{definition}[Jordan degree type matrix] Let $A=S/\ann(F)$ be an Artinian    Gorenstein algebra and $\ell\in A$ a linear form. Assume that $M_{\ell,A}$ is the rank matrix of $A$ and $\ell$. We define a \emph{Jordan degree type matrix} of $A$ and $\ell$ to be a matrix with non-negative entries denoted by $J_{\ell,A}$ and defined as follows
\begin{align}\label{J(A,l)definition}
(J_{\ell,A})_{i,j} :
=& (M_{\ell,A})_{i,j}+(M_{\ell,A})_{i-1,j+1}-(M_{\ell,A})_{i-1,j}-(M_{\ell,A})_{i,j+1},
\end{align}
where we set $(M_{\ell,A})_{i,j}$ if either  $i< 0$ or $j< 0$.
Since $M_{\ell,A}$ is upper triangular, $J_{\ell,A}$ is also upper triangular. For each $0\leq i\leq j$ we set $k=j-i$, then 
\begin{equation}\label{JDT_ij}
(J_{\ell,A})_{i,j} = h_{A^{(k)}}(i)+h_{A^{(k+2)}}(i-1)-h_{A^{(k+1)}}(i-1)-h_{A^{(k+1)}}(i),
\end{equation}
such that for every $k$, $h_{A^{(k)}}(-1):=0$.
\end{definition}
Recall from Lemma \ref{additiveRank} that for each $1\leq i\leq j$, $(J_{\ell,A})_{ij}$ is non-negative.
Equation (\ref{JDT_ij}) may be expressed in terms of the mixed Hessians.
\begin{small}
\begin{equation}
(J_{\ell,A})_{i,j} = \rk\Hess_\ell^{(i,d-i-k)}(F)+\rk\Hess_\ell^{(i-1,d-i-k-1)}(F)-\rk\Hess_\ell^{(i-1,d-i-k)}(F)-\rk\Hess_\ell^{(i,d-i-k-1)}(F).
\end{equation}
\end{small}
This recovers a result by R. Gondim and B. Costa \cite[Theorem 4.7]{CG} determining Jordan types of  Artinian Gorenstein algebras and linear forms using the ranks of mixed Hessians. 
\begin{example}
Consider the Artinian    Gorenstein algebra given in Example \ref{firstEx} and linear form $\ell=x_1$. The Jordan degree type matrix of $A$ and $\ell$ is equal to the following matrix 
$$J_{\ell,A} = \begin{pmatrix}
0&0&1&0&0&0&0\\
0&0&0&2&0&0&0\\
0&0&0&0&3&0&0\\
0&0&0&0&0&2&0\\
0&0&0&0&0&0&1\\
0&0&0&0&0&0&0\\
0&0&0&0&0&0&0\\
\end{pmatrix}.
$$
So $P_{\ell,A} = (3^9)$ an the degree of each part in $P_{\ell,A}$ is equal the row of the corresponding entry in $J_{\ell,A}$ minus one. So the Jordan degree type of $\ell$ for $A$ is equal to  $(3_0,3_1^2,3_2^3,3_3^2,3_4)$.
\end{example}
\begin{proposition}\label{rkmatrix-1-1-JDT-prop}
There is a one-to-one correspondence between two matrices $M_{\ell,A}$ and $J_{\ell,A}$ associated to Artinian  Gorenstein algebra $A$ and linear form $\ell\in A$.
\end{proposition}
\begin{proof}

We use Equation (\ref{J(A,l)definition}) to provide an algorithm to obtain $J_{\ell,A}$ from $M_{\ell,A}$. For each $1\leq i\leq j$ define matrix $J^\prime_{\ell,A}$ as the following 
\begin{equation}\label{J'matrixdef}
(J^\prime_{\ell,A})_{i,j} := (M_{\ell,A})_{i,j}-(M_{\ell,A})_{i,j+1},
\end{equation}
where we set $(M_{\ell,A})_{i,j}=0$ if either $i\leq 0$ or $j\leq 0$.
Then define the upper triangular matrix $J_{\ell,A}$ where its entry $i,j$ for every $1\leq i\leq j$ is equal to 
\begin{equation}\label{JfromJ'def}
(J_{\ell,A})_{i,j} = (J^\prime_{\ell,A})_{i,j}-(J^\prime_{\ell,A})_{i-1,j},
\end{equation}
where we set $(J^\prime_{\ell,A})_{i,j}=0$  if either $i\leq 0$ or $j\leq 0$.

We obtain  $M_{\ell,A}$ from $J^\prime_{\ell,A}$ in two steps.
First we get the matrix $J^\prime_{\ell,A}$ from $J_{\ell,A}$. For each $j\geq i\geq 1$,  we have the following
\begin{equation}
(J^\prime_{\ell,A})_{i,j} = (J_{\ell,A})_{i,j}+(J_{\ell,A})_{i-1,j},
\end{equation}
where we set $(J_{\ell,A})_{i,j}=0$ if either $i\leq 0$ or $j\leq 0$.
Then for each $j\geq i\geq 1$, 
\begin{equation}
(M_{\ell,A})_{i,j}=(J^\prime_{\ell,A})_{i,j}+(J^\prime_{\ell,A})_{i,j+1},
\end{equation}
where we set $(J^\prime_{\ell,A})_{i,j}=0$ if either $i\leq 0$ or $j\leq 0$.
\end{proof}
\begin{example}
We illustrate the procedure provided in Proposition \ref{rkmatrix-1-1-JDT-prop} for the Artinian  Gorenstein algebra given in Example \ref{firstEx} with the rank matrix $M_{\ell,A}$. Using Equations (\ref{J'matrixdef}) and  (\ref{JfromJ'def}) we get the following matrices.
$$M_{\ell,A}= \begin{pmatrix}
1&1&1&0&0&0&0\\
0&3&3&2&0&0&0\\
0&0&6&5&3&0&0\\
0&0&0&7&5&2&0\\
0&0&0&0&6&3&1\\
0&0&0&0&0&3&1\\
0&0&0&0&0&0&1\\
\end{pmatrix},
\hspace*{0mm} J^\prime_{\ell,A} = \begin{pmatrix}
0&0&1&0&0&0&0\\
0&0&1&2&0&0&0\\
0&0&1&2&3&0&0\\
0&0&0&2&3&2&0\\
0&0&0&0&3&3&1\\
0&0&0&0&0&2&1\\
0&0&0&0&0&0&1\\
\end{pmatrix}, \hspace*{0mm} J_{\ell,A} = \begin{pmatrix}
0&0&1&0&0&0&0\\
0&0&0&2&0&0&0\\
0&0&0&0&3&0&0\\
0&0&0&0&0&2&0\\
0&0&0&0&0&0&1\\
0&0&0&0&0&0&0\\
0&0&0&0&0&0&0\\
\end{pmatrix}. $$
\end{example}
Define decreasing sequence  $\mathbf{d}:=(\dim_{\mathsf{k}}A^{(0)}, \dim_{\mathsf{k}}A^{(1)}, \dots , \dim_{\mathsf{k}}A^{(d)})$ and recall that the second difference sequence of $\mathbf{d}$ is denoted by $\Delta^2 \mathbf{d}$ and its  $i$-th entry is given by $$\Delta^2 \mathbf{d}(i)=\dim_{\mathsf{k}}A^{(i)}+\dim_{\mathsf{k}}A^{(i+2)}-2\dim_{\mathsf{k}}A^{(i+1)},$$
where we set $\dim_{\mathsf{k}}A^{(i)}=0$ for $i>d$.
\begin{proposition}\label{JT}
Let $A=S/\ann(F)$ be an Artinian  Gorenstein algebra with socle degree $d\geq 2$ and let $\ell\in A$ be a linear form. Then the Jordan type partition of $\ell$ for $A$ is given by
$$
P_{\ell,A} = \big(\underbrace{d+1,\dots ,d+1}_{n_{d}} ,\underbrace{d,\dots ,d}_{n_{d-1}} ,\dots ,\underbrace{2,\dots ,2}_{n_1},\underbrace{1,\dots ,1}_{n_0}\big),
$$
such that $\mathbf{n}=(n_0,n_{1}, \dots , n_d)=\Delta^2 \mathbf{d}.$
\end{proposition}
\begin{proof}
The Jordan type partition of $\ell$ for $A$ is equal to the dual partition of the following partition 
\begin{equation}\label{jordantype}
\Big(\rk (\times \ell^0)-\rk (\times \ell^1),\rk (\times \ell^1)-\rk (\times \ell^2),\dots , \rk (\times \ell^{d-1})-\rk (\times \ell^d),\rk (\times \ell^d)\Big).
\end{equation}
Since for each $0\leq i\leq d$ the rank of the multiplication map $ \times \ell^{i}:A_j\longrightarrow A_{j+i}$ is equal to the rank of differentiation map  $\circ \ell^i :\langle F \rangle_{i+j}\longrightarrow\langle F\rangle_{j}$, where $\langle F\rangle$  is the dual algebra to $A$. Thus the rank of $ \times \ell^i:A_j\longrightarrow A_{j+i}$ is equal to $\dim_{\K}\left( S/\ann(\ell^i\circ F)\right)_j$ and therefore we have 
$$\rk \left( \times \ell^i: A\longrightarrow A\right) = \sum^{d-i}_{j=0} \dim_{\K}\left( S/\ann(\ell^i\circ F)\right)_j=\dim_{\mathrm{k}}A^{(i)}.$$
So (\ref{jordantype}) is equal to the following partition
$$
\big(\dim_{\mathrm{k}}A^{(0)}-\dim_{\mathrm{k}}A^{(1)},\dim_{\mathrm{k}}A^{(1)}-\dim_{\mathrm{k}}A^{(2)},\dots ,\dim_{\mathrm{k}}A^{({d-1})}-\dim_{\mathrm{k}}A^{(d)},\dim_{\mathrm{k}}A^{(d)}\big).
$$
The dual partition to the above partition is the Jordan type partition of $A$ and $\ell$ as we claimed.
\end{proof}

\section{Jordan types of Artinian  Gorenstein algebras of codimension three}\label{codim3section}
In this section we consider graded Artinian Gorenstein quotientd of $S=\K[x,y,z]$ where $\mathrm{char}(\K)=0$. For an  Artinian  Gorenstein algebra  $A=S/\ann(F)$ with dual generator $F\in R=\mathsf{k}[X,Y,Z]$ of degree $d\geq 2$ and a linear form $\ell$ we explain how we find the the rank matrix $M_{\ell,A}$ and as a consequence the Jordan type $P_{\ell,A}$.  

Let $L_1, L_2,L_3$ be linear forms in the dual ring $R=\mathsf{k}[X,Y,Z]$  such that $\ell\circ L_1\neq 0$ and $\ell\circ L_2=\ell\circ L_3 = 0$.
By linear change of coordinates we may assume that $L_1=X$, $L_2=Y$ and $L_3=Z$. Then $F$ can be written in the following form 
$$
F = \sum_{i=0}^dX^iG_{d-i},
$$
where for each $0\leq i\leq d$, $G_{d-i}$ is a homogeneous polynomial of degree $d-i$ in the variables $Y$ and $Z$. In general $G_{d-i}$ could be a zero polynomial for some $i$. 


\subsection{Jordan types with parts of length at most four}\label{length3}

We will provide a list of all possible rank matrices $M_{\ell,A}$ where $A$ is an Artinian Gorenstein algebra and $\ell$ is a linear form in $A$ where $\ell^3=0$. 
Assuming  $\ell^3=0$ implies that $M_{\ell,A}$ has at most three  non-zero diagonals. Consequently, we provide a formula to compute the  Jordan type partitions for Artinian Gorenstein algebras and linear forms $\ell$ such that $\ell^4=0$ which are Jordan types with parts of length at most four.

Consider Artinian  Gorenstein algebra $A=S/\ann(F)$ with socle degree $d\geq 2$ and linear form $\ell$ such that $\ell^3=0$. Without loss of generality we assume that $\ell=x$ and that $F$ is in the following form
\begin{equation}\label{F}
F = X^2G_{d-2}+XG_{d-1}+G_d,
\end{equation}
where
$$G_{d}=\sum_{j=0}^{d}\frac{a_{j}}{j!(d-j)!}{Y^{d-j}Z^{j}}, \quad  G_{d-1}=\sum_{j=0}^{d-1}\frac{b_{j}}{j!(d-j-1)!}{Y^{d-j-1}Z^{j}}$$
and 
$$ G_{d-2}=\frac{1}{2}\sum_{j=0}^{d-2}\frac{c_{j}}{j!(d-j-2)!}{Y^{d-j-2}Z^{j}}.$$
The coefficients of the terms in $F$ are chosen in a way that the entries of the catalecticant matrices of $F$ are either zero or one to make the computations simpler.

We first consider the case when $\ell^3=0$ but $\ell^2\neq 0$. Therefore, we assume that  $G_{d-2}\neq 0$ since otherwise we get $\ell^2=0$. Recall that $A^{(0)}=A$, $A^{(1)}=S/\ann(\ell\circ F)$, $A^{(2)}=S/\ann(\ell^2\circ F)$ and $A^{(i)}=S/\ann(\ell^i\circ F)=0$, for every $i\geq 3$.

We determine all rank matrices which occur for such algebras and linear forms $\ell$ where $\ell^3=0$. Equivalently, we determine all possible Hilbert functions for $A, A^{(1)}$ and $A^{(2)}$. The rank tables are slightly different for even  and odd socle degrees, as excepted, thus we treat these cases separately. We first prove our result for Artinian Gorenstein algebras with even socle degree $d\geq 2$. Later in  similar cases for odd socle degrees we refer to the proof for even socle degrees. 

In the following lemma we  provide all possible combinations for maximum values of $h_A$, $h_{A^{(1)}}$ and $h_{A^{(2)}}$.
\begin{lemma}[Even socle degree]\label{maxvaluesevenLemma}
There exists an Artinian Gorenstein algebra $A$ with even socle degree $d\geq 2$ and linear form $\ell\in A_1$ where $\ell^2\neq 0$ but $\ell^3=0$, such that $$(r,s,t)=(h_{A^{(2)}}(\frac{d}{2}-1) ,h_{A^{(1)}}(\frac{d}{2}-1),h_A(\frac{d}{2}))$$ if and only if 
\begin{itemize}
\item [$(1)$]  $r\in [1,\frac{d}{2}-1]$, $s\in[2r,\frac{d}{2}+r]$ and $t\in[2s-r,\frac{d}{2}+s+1]$, for $d\geq 4$;  or
\item [$(2)$] $r=\frac{d}{2}$, $s=d-1$ and $t\in[\frac{3d}{2}-2,\frac{3d}{2}]$,  for $d\geq 2$.  
\end{itemize}
\end{lemma}
\begin{proof}
We prove the statement by analysing the catalecticant matrices in the desired degrees. In each case we first determine all possible ranks for each catalecticant matrix and then for each possible value we provide polynomials $G_{d-2}, G_{d-1}$ and $G_d$ as in (\ref{F}) which gives the certain rank. \par 
The maximum value of the  Hilbert function of $A$ occurs in degree $\frac{d}{2}$ and it is equal to $\rk\Cat_F(\frac{d}{2})$. Pick the following monomial basis for $A_{\frac{d}{2}}$
$$
\B_{\frac{d}{2}} = \{x^{\frac{d}{2}},x^{\frac{d}{2}-1}y,x^{\frac{d}{2}-1}z,x^{\frac{d}{2}-2}y^2,x^{\frac{d}{2}-2}yz,x^{\frac{d}{2}-2}z^2,\dots , y^{\frac{d}{2}},y^{\frac{d}{2}-1}z,\dots ,z^{\frac{d}{2}}\}.
$$
Then the catalecticant matrix of $F$ with respect to $\B_{\frac{d}{2}}$ is equal to 
\begin{equation}
\Cat_F(\frac{d}{2})=\left[\begin{array}{@{}c|c|c@{}}
\mathbf{0}& \mathbf{0}&{\Cat_{G_{d-2}}{(\frac{d}{2}-2)}}
\\\hline
\mathbf{0}&\Cat_{G_{d-2}}{(\frac{d}{2}-1)}&\Cat_{G_{d-1}}{(\frac{d}{2}-1)}\\\hline
\Cat_{G_{d-2}}{(\frac{d}{2})}&\Cat_{G_{d-1}}{(\frac{d}{2})}&\Cat_{G_{d}}{(\frac{d}{2})}
\end{array}
\right].\\
\end{equation}
Which is equal to 
\begin{equation}\label{catmatrix3lines}
 \Cat_F(\frac{d}{2}) = \left[\begin{array}{@{}cccc|cccc|cccc@{}}
    0 & 0 & \cdots & 0 &  0 & 0 & \cdots & 0 & c_{0} & c_{1}&\cdots & c_{\frac{d}{2}} \\
     0 & 0 & \cdots & 0 & 0 & 0 &\cdots & 0 & c_{1} & c_{2} &\cdots & c_{\frac{d}{2}+1} \\
    \vdots & \vdots & \reflectbox{$\ddots$}  &\vdots & \vdots & \vdots & \reflectbox{$\ddots$}  &\vdots  & \vdots  &  \vdots & \reflectbox{$\ddots$} &\vdots\\
 0 & 0 & \cdots & 0 &  0 & 0 &\cdots & 0 &  c_{\frac{d}{2}-2} & c_{\frac{d}{2}-1} &\cdots & c_{d-2} \\\hline
  0 & 0 & \cdots & 0 &  c_{0} & c_{1}&\cdots & c_{\frac{d}{2}-1} & b_{0} & b_{1}&\cdots & b_{\frac{d}{2}} \\
 0 & 0 & \cdots & 0 &   c_{1} & c_{2} &\cdots & c_{\frac{d}{2}}  & b_{1} & b_{2} &\cdots & b_{\frac{d}{2}+1} \\
    \vdots  &\vdots & \reflectbox{$\ddots$}  &  \vdots&   \vdots  &\vdots & \reflectbox{$\ddots$}  &  \vdots& \vdots  &  \vdots & \reflectbox{$\ddots$} &\vdots\\
  0 & 0 & \cdots & 0 &  c_{\frac{d}{2}-1} & c_{\frac{d}{2}} &\cdots & c_{d-2} & b_{\frac{d}{2}-1} & b_{\frac{d}{2}} &\cdots & b_{d-1} \\\hline
     c_{0} & c_{1}&\cdots & c_{\frac{d}{2}-2} & b_{0} & b_{1}&\cdots & b_{\frac{d}{2}-1} &a_{0} & a_{1}&\cdots & a_{\frac{d}{2}} \\
 c_{1} & c_{2} &\cdots & c_{\frac{d}{2}-1}  & b_{1} & b_{2} &\cdots & b_{\frac{d}{2}}& a_{1} & a_{2} &\cdots & a_{\frac{d}{2}+1} \\
     \vdots  &\vdots & \reflectbox{$\ddots$}  &  \vdots& \vdots  &  \vdots & \reflectbox{$\ddots$} &\vdots& \vdots  &  \vdots & \reflectbox{$\ddots$} &\vdots\\
    c_{\frac{d}{2}} & c_{\frac{d}{2}+1} &\cdots & c_{d-2} & b_{\frac{d}{2}} & b_{\frac{d}{2}+1} &\cdots & b_{d-1}& a_{\frac{d}{2}} & a_{\frac{d}{2}+1} &\cdots & a_{d} \\
     \end{array}\right].
\end{equation}
Since any Artinian  algebra of codimension two has the SLP  the rank of the $j$-th Hessian matrices of  polynomials $G_{d-2},G_{d-1}$ and $G_d$ are equal to the rank of their $j$-th catalecticant matrices. By linear change of coordinates we may assume that $z$ is the strong Lefschetz element for Artinian  Gorenstein algebra $\mathsf{k}[y,z]/\ann(G_{d-2})$ this implies that the lower right square submatrices of the catalecticant matrices of $G_{d-2}$ in various degrees all have maximal rank. Likewise, we may assume that $y$ is the strong Lefschetz element for Artinian Gorenstein algebra $\mathsf{k}[y,z]/\ann(G_{d-1})$ which means that the upper left square submatrices of the catalecticant matrices of $G_{d-1}$ in different degrees are all full rank. \par 
Observe that $r=h_{A^{(2)}}(\frac{d}{2}-1)\in [1,\frac{d}{2}]$. To show $(1)$ we assume that $r=h_{A^{(2)}}(\frac{d}{2}-1)\in [1,\frac{d}{2}-1]$ which implies that $
 h_{A^{(2)}}(\frac{d}{2}-2)= h_{A^{(2)}}(\frac{d}{2}-1)= h_{A^{(2)}}(\frac{d}{2})=r.
$ We assume that the ranks of the lower right submatrices of $\Cat_{G_{d-2}}(\frac{d}{2}-2), \Cat_{G_{d-2}}(\frac{d}{2}-1)$ and $\Cat_{G_{d-2}}(\frac{d}{2})$ are equal to $r$ and choosing $c_{d-r-1}=1$ and $c_{i}=0$ for every $i\neq d-r-1 $ will provide the desired property. So we choose 
\begin{equation}\label{G_(d-2)even(1)}
G_{d-2}= \frac{Y^{r-1}Z^{d-r-1}}{(r-1)!(d-r-1)!}, \quad \text{for all}\quad r\in[1,\frac{d}{2}-1].
\end{equation}
Now in order to obtain possible values for $s=h_{A^{(1)}}(\frac{d}{2}-1)$, we notice that $s\in [2r, 2r+\rk \mathbf{B}]$ where $\mathbf{B}$ is the following matrix 
$$
\mathbf{B}=\left(\begin{array}{@{}ccccccc@{}}
 b_{0} &\cdots & b_{\frac{d}{2}-r} \\
    \vdots  & \reflectbox{$\ddots$} &\vdots\\
b_{\frac{d}{2}-1-r}&\cdots & b_{d-1-2r}      \end{array}\right).
$$
Since the socle degree of $A^{(1)}$ is an odd integer and is equal to $d-1$ we get that $ h_{A^{(1)}}(\frac{d}{2}-1)= h_{A^{(1)}}(\frac{d}{2}) = s$. For every $s\in [2r, 2r+\rk\mathbf{B}]$ we have $\rk\mathbf{B}=s-2r$. We may assume that the upper left submatrix of $\mathbf{B}$ has rank $s-2r$. For $\rk\mathbf{B}=s-2r=0$ setting $G_{d-1}=0$ and otherwise setting $b_{s-2r-1}=1$ and $b_i=0$ for every $i\neq s-2r-1$ implies that $\rk\mathbf{B}=s-2r$.  Equivalently, we set 
\begin{equation}\label{G_(d-1)even(1)}
G_{d-1}=\left\{
                \begin{array}{ll}
                  0 & \text{if $s-2r=0$},\\
                  \frac{Y^{d-s+2r}Z^{s-2r-1}}{(d-s+2r)!(s-2r-1)!} & \text{if $1\leq s-2r\leq \frac{d}{2}-r.$}\\
                \end{array}
              \right.
\end{equation}
This implies that there exists $A$ such that $h_{A^{(1)}}(\frac{d}{2}-1)=s $ if and only if  $s\in [2r,\frac{d}{2}+r].$\par 
To obtain possible values for $t=h_A(\frac{d}{2})$, first notice that $t\in [2s-r, 2s-r+\rk\mathbf{A}]$ where 
$$
\mathbf{A}=\left(\begin{array}{@{}ccccccc@{}}
 a_{2s-4r} &\cdots & a_{\frac{d}{2}-3r+s} \\
    \vdots  & \reflectbox{$\ddots$} &\vdots\\
a_{\frac{d}{2}-3r+s}&\cdots & a_{d-2r}      \end{array}\right).
$$
For every $t\in [2s-r, 2s-r+\rk\mathbf{A}]$ we have that $\rk\mathbf{A}=t-2s+r$. We may assume that the upper left submatrix of $\mathbf{A}$ has rank equal to $t-2s+r$. 
%Now consider the following submatrix of the matrix in (\ref{catmatrix3lines}), for every $j\in [0,\frac{d}{2}-r]$
%$$
%\mathbf{A}=\left(\begin{array}{@{}ccccccc@{}}
 %a_{2j} &\cdots & a_{\frac{d}{2}-r+j} \\
  %  \vdots  & \reflectbox{$\ddots$} &\vdots\\
%a_{\frac{d}{2}-r+j}&\cdots & a_{d-2r}      \end{array}\right).
%$$
For $\rk\mathbf{A}=t-2s+r=0$ setting $G_d=0$ and  otherwise setting  $a_{t-3r-1}=1$ and $a_i=0$ for every $i\neq t-3r-1$ provides the desired ranks. In other words we choose $G_d$ as the following
\begin{equation}\label{G_(d)even(1)}
G_{d}=\left\{
                \begin{array}{ll}
                  0 & \text{if $t-2s+r=0$},\\
                  \frac{Y^{d-t+3r+1}Z^{t-3r-1}}{(d-t+3r+1)!(t-3r-1)!} & \text{if $1\leq t-2s+r\leq \frac{d}{2}+r-s+1.$}\\
                \end{array}
              \right.
\end{equation}
This implies that there exists $A$ such that $t=h_A(\frac{d}{2})$ if and only if 
$t\in [2s-r,\frac{d}{2}+s+1]$.
%$$\quad h_{A}(\frac{d}{2}-1) = 3r+2j+k,\quad \text{for all}\hspace*{2mm} j\in [0,\frac{d}{2}-r],\hspace*{2mm}\text{and}\hspace*{2mm}k\in [0,\frac{d}{2}-r-j+1],$$
%or equivalently,
%$$\quad h_{A}(\frac{d}{2}-1) = t,\quad \text{for all}\hspace*{2mm} t\in [2s-r,\frac{d}{2}+s+1].$$

To prove $(2)$ assume that $h_{A^{(2)}}(\frac{d}{2}-1)=\frac{d}{2}$ which  implies that the Hilbert function of $h_{A^{(2)}}$ has the maximum possible value up to degree $\frac{d}{2}-1$ and since the socle degree of $A^{(2)}$ is even and is equal to $d-2$ we have  $$h_{A^{(2)}}(\frac{d}{2}-2)=h_{A^{(2)}}(\frac{d}{2})=\frac{d}{2}-1.$$ So setting $c_{\frac{d}{2}-1}=1$ and $c_i=0$ for every $i\neq \frac{d}{2}-1$, or equivalently
\begin{equation}\label{Gd-2maxeven}
G_{d-2} = \frac{Y^{\frac{d}{2}-1}Z^{\frac{d}{2}-1}}{(\frac{d}{2}-1)!(\frac{d}{2}-1)!}
\end{equation}
provides  the desired ranks for the catalecticant matrices $\Cat_{G_{d-2}}(\frac{d}{2}-2)$, $\Cat_{G_{d-2}}(\frac{d}{2}-1)$ and $\Cat_{G_{d-2}}(\frac{d}{2})$. \\ 
We have that $h_{{A^{(1)}}}(\frac{d}{2}-1)=\rk\Cat_{x\circ F}(\frac{d}{2}-1)$, and 
\begin{equation}
\Cat_{x\circ F}(\frac{d}{2}-1) = \left[\begin{array}{@{}c|c@{}}
 \mathbf{0}&{\Cat_{G_{d-2}}{(\frac{d}{2}-2)}}
\\\hline
\Cat_{G_{d-2}}{(\frac{d}{2}-1)}&\Cat_{G_{d-1}}{(\frac{d}{2}-1)}\\
\end{array}
\right].\\
\end{equation}
Since $\rk {\Cat_{G_{d-2}}{(\frac{d}{2}-2)}}=\frac{d}{2}-1$ and $\rk\Cat_{G_{d-2}}{(\frac{d}{2}-1)}=\frac{d}{2}$ the rank of the above matrix is maximum possible and is equal to $d-1$. This means that for every choice of polynomial $G_{d-1}$ in this case we have 
$$
h_{A^{(1)}}(\frac{d}{2}-1)=d-1.
$$
In order to find possbile values for $h_A(\frac{d}{2})$ note that the rank of $\Cat_F(\frac{d}{2})$ is at most $\frac{3d}{2}$ and also
$$\frac{3d}{2}-2=\rk\Cat_{G_{d-2}}(\frac{d}{2}-2)+\rk\Cat_{G_{d-2}}(\frac{d}{2}-1)+\rk\Cat_{G_{d-2}}(\frac{d}{2})\leq \rk\Cat_F(\frac{d}{2})\leq \frac{3d}{2}.$$ 
Note that setting $G_{d-2}$ as  (\ref{Gd-2maxeven}), $G_{d-1}=0$ and  $G_d$ equal to the following 
\begin{equation}\label{Gdmaxeven}
G_{d}=\left\{
                \begin{array}{ll}
                  0 & \text{for $t=\frac{3d}{2}-2$},\\
                  \frac{Y^{d}}{(d)!} & \text{for $t=\frac{3d}{2}-1,$}\\
                  \frac{Y^{d}}{(d)!}+ \frac{Z^{d}}{(d)!}& \text{for $t=\frac{3d}{2}$}.\\
                \end{array}
              \right.
\end{equation}
provides the desired ranks for the catalecticant matrix $\Cat_F(\frac{d}{2})$  in (\ref{catmatrix3lines}). 
\end{proof}
We now prove that a rank matrix of $A$ or equivalently  Hilbert functions of $A$, $A^{(1)}$ and $A^{(2)}$  are  completely determined by the maximum values of  $h_{A^{(2)}}$, $h_{A^{(1)}}$ and $h_{A}$. We also provide all rank matrices for each possible combination of integers  $(r,s,t)$ listed in Lemma \ref{maxvaluesevenLemma}. 
\begin{theorem}[Even socle degree]\label{3linesHFtheorem-even}
Let $A$ be an Artinian Gorenstein algebra with even socle degree $d\geq 2$ and $\ell\in A_1$ such that $\ell^2\neq 0$ and  $\ell^3=0$. Then Hilbert functions of $A$, $A^{(1)}$ and $A^{(2)}$  are  completely determined by $(r,s,t)=(h_{A^{(2)}}(\frac{d}{2}-1) ,h_{A^{(1)}}(\frac{d}{2}-1),h_A(\frac{d}{2}))$. More precisely, 
\begin{itemize}
\item[$(1)$] if $d\geq 4$, $r\in [1,\frac{d}{2}-1]$, $s\in[2r,\frac{d}{2}+r]$ and $t\in [2s-r,\frac{d}{2}+s+1]$, then  
\begin{equation}\label{HFeven(1)}
h_{A^{(2)}}(i)=\left\{
                \begin{array}{ll}
                  i+1 &  0\leq i\leq r-1,\\
                  r & r\leq i\leq \frac{d}{2}-1,\\
                \end{array}
              \right.\quad
h_{A^{(1)}}(i)=\left\{
                \begin{array}{ll}
                  2i+1 &  0\leq i\leq r-1,\\
                  i+r+1 & r\leq i\leq s-r-1,\\
                  s & s-r\leq i\leq \frac{d}{2}-1.
                \end{array}
              \right.
 \end{equation}
  \begin{itemize}
\item If $t=3r$ then there are two possible Hilbert functions for $A$
\begin{equation}\label{HFevenA(1,1)}
h_{A}(i)=\left\{
                \begin{array}{ll}
                 1 & i=0,\\
                  3i &  1\leq i\leq r-1,\\
                  3r & r\leq i\leq\frac{d}{2},\\
                \end{array}
              \right.
\text{and} \quad 
h_{A}(i)=\left\{
                \begin{array}{ll}
                   1 & i=0,\\
                  3i &  1\leq i\leq r-1,\\
                  3r-1& i=r,\\
                  3r & r+1\leq i\leq \frac{d}{2},\\
                \end{array}
              \right.
 \end{equation}

\item otherwise, i.e., $t>3r$ we have
\begin{equation}\label{HFevenA(1)o.w.}
h_{A}(i)=\left\{
                \begin{array}{ll}
                   1 & i=0,\\
                  3i &  1\leq i\leq r,\\
                 2i+r+1 & r+1\leq i\leq s-r-1,\\
                 2i+r+1& i=s-r, \hspace*{2mm}\text{if}\hspace*{2mm}t>2s-r\hspace*{2mm}\text{and}\hspace*{2mm}s>2r,\\
                  2i+r& i=s-r, \hspace*{2mm}\text{if}\hspace*{2mm}t>2s-r \hspace*{2mm}\text{and}\hspace*{2mm}s=2r,\\
                 i+s+1 &s-r+1\leq i\leq t-s-1,\\
                 t & t-s \leq i\leq \frac{d}{2}.
                \end{array}
              \right.
 \end{equation}
 \end{itemize}
 \item[$(2)$]
If $d\geq 2$, $r=\frac{d}{2}$, $s=d-1$ and $t\in[\frac{3d}{2}-2,\frac{3d}{2}]$, 
then for every  $0\leq i\leq \frac{d}{2}-1$
 \begin{align}\label{HFeven(2)}
  h_{A^{(2)}}(i)=i+1,\quad  h_{A^{(1)}}(i)=2i+1, \hspace*{2mm}\text{and}
 \end{align}
  \begin{align}\label{HFevenA(2.1)}
 h_{A}(i)=\left\{
                \begin{array}{ll}
                 1 & i=0,\\
                  3i &  1\leq i\leq \frac{d}{2}-1,\\
                   t &  i=\frac{d}{2}.\\
                \end{array}
              \right.
  \end{align}
    \end{itemize}
\end{theorem}

\begin{proof}
We first show $(1)$. Since the Hilbert function of Artinian    Gorenstein algebras are symmetric it is enough to determine it up to the middle degree.
We have that 
$$
A^{(2)} = S/\ann(\ell^2\circ F) = S/\ann(G_{d-2}).
$$
So $A^{(2)}$ is an Artinian    Gorenstein algebra with codimension at most two and the maximum value of $h_{A^{(2)}}$ is equal to $r$. The Hilbert function of $A^{(2)}$ increases by exactly one until it reaches $r$ and it stays $r$ up to the middle degree, $\frac{d}{2}-1$. So we get $h_{A^{(2)}}$ as we claimed.\par 
The assumption on $r$ implies that $h_{A^{(2)}}(\frac{d}{2}-2)=h_{A^{(2)}}(\frac{d}{2}-1)=r$. So 
$$
(h_{A^{(1)}}-(h_{A^{(2)}})_+)(\frac{d}{2}-1) = h_{A^{(1)}}(\frac{d}{2}-1)-h_{A^{(2)}}(\frac{d}{2}-2) = s-r.
$$
Since $(h_{A^{(1)}}-(h_{A^{(2)}})_+)(1)\leq 2$, Lemma \ref{diffO-seq} implie that for every $0\leq i\leq s-r-1$
$$(h_{A^{(1)}}-(h_{A^{(2)}})_+)(i)=i+1.$$
So since $0\leq r-1\leq s-r-1$, for every $0\leq i\leq r-1$ we have that 
$$
h_{A^{(1)}}(i) = i+1+h_{A^{(2)}}(i-1)= i+1+i = 2i+1.
$$
If $r-1< s-r-1$ then for $r\leq i\leq s-r-1$ we have 
$$
h_{A^{(1)}}(i) = i+1+h_{A^{(2)}}(i-1)= i+1+r.
$$
For $s-r\leq i\leq \frac{d}{2}-1$ we get $h_{A^{(1)}}(i) = s$. This is because we assumed $r\leq s-r$ so we have  $h_{A^{(2)}}(i)=r$ for $s-r-1\leq i\leq \frac{d}{2}-1$. 

We now determine the Hilbert function of $A$. By assumption we have $(h_A-(h_{A^{(1)}})_+)({\frac{d}{2}})= h_{A}(\frac{d}{2})-h_{A^{(1)}}(\frac{d}{2}-1)=t-s$. On the other hand, we have that $(h_{A}-(h_{A^{(1)}})_+)(1)\leq 2$ and by Lemma \ref{diffO-seq} we conclude that $h_{A}-(h_{A^{(1)}})_+$ is the Hilbert function of some algebra with codimension at most two. So for every $0\leq i\leq t-s-1$ we have 
$$
(h_A-(h_{A^{(1)}})_+)({i}) = i+1.
$$
By assumption we have $0\leq r-1\leq s-r-1\leq t-s-1$, so for every $1\leq i\leq r-1$
$$
h_A(i) =i+1+h_{A^{(1)}}(i-1)= i+1+2(i-1)+1=3i.
$$
\begin{itemize}
\item Suppose that $r=t-s$ then we get $s=2r$ and $t=3r$. Since we have $r\leq \frac{d}{2}-1$ and the Hilbert function of an  algebra with codimension two is unimodal, we get
$$
\left(h_A-(h_{A^{(1)}})_+\right)({r})\geq \left(h_A-(h_{A^{(1)}})_+\right)({r-1})
$$ and thus 
$$h_A(r)\geq r+h_{A^{(1)}}(r-1) = r+2(r-1)+1=3r-1.
$$
Thus we have two possible values for $h_A(r)$, that is either equal to $3r-1$ or $3r$. Clearly, for $r+1\leq i\leq \frac{d}{2}$  we have $h_A(i)=3r$. 
\item Now suppose that $r<t-s$. Then $$h_A(r) = r+1+h_{A^{(1)}}(r-1)=r+1+2(r-1)+1=3r.$$
If $r<s-r-1$ then for every $r+1\leq i\leq s-r-1$ we get 
$$
h_A(i)=i+1+h_{A^{(1)}}(i-1) = i+1+r+i=2i+r+1.
$$
If $s-r-1<t-s-1$ then 
\begin{equation*}
h_A(s-r) =s-r+1+h_{A^{(1)}}(s-r-1) = \left\{
                \begin{array}{ll}
                  s-r+1+s &  \text{if}\hspace*{2mm} s>2r,\\
                  s-r+1+s-1 & \text{if}\hspace*{2mm} s=2r.\\
                \end{array}
              \right.
\end{equation*}
If $s-r<t-s-1$ then for every $s-r+1\leq i\leq t-s-1$ we get 
$$
h_A(i)=i+1+h_{A^{(1)}}(i-1)=i+1+s.
$$
Since the Hilbert function of an algebra with codimension two is unimodal and $t-s\leq \frac{d}{2}+1$ we have that
$$
\left(h_A-(h_{A^{(1)}})_+\right)({t-s})\geq \left(h_A-(h_{A^{(1)}})_+\right)({t-s-1})=t-s.
$$
Therefore, 
\begin{equation}\label{h(t-s)}
h_A(t-s)\geq t-s+h_{A^{(1)}}(t-s-1).
\end{equation}
If $s-r-1<t-s-1$ then $h_A(t-s) \geq t-s+s = t$ therefore for every $t-s\leq i\leq \frac{d}{2}$ we have that $h_A(i)=t$.

\noindent If $s-r=t-s$ then if $r= s-r$ we have that $s=2r$ and $t=3r$ which contradicts the assumption that $r<t-s$. So we have $r\leq s-r-1$. Using (\ref{h(t-s)}) we get that
$$
h_A(t-s)\geq t-s+h_{A^{(1)}}(t-s-1)= t-s+s=t .
$$ Therefore, for every $t-s\leq i\leq \frac{d}{2}$ we have that $h_A(i)=t$. 
\end{itemize}
We now prove $(2)$. Notice that 
$$\frac{d}{2}-1=\frac{3d}{2}-2 -(d-1)\leq \left(h_A-(h_{A^{(1)}})_+\right)({\frac{d}{2}})\leq \frac{3d}{2} -(d-1)=\frac{d}{2}+1.$$
If $ \frac{d}{2}\leq \left(h_A-(h_{A^{(1)}})_+\right)({\frac{d}{2}})\leq \frac{d}{2}+1$ then for every $1\leq i\leq \frac{d}{2}-1$ we have that $\left(h_A-(h_{A^{(1)}})_+\right)(i)=i+1$ which implies that
$$
h_A(i)=i+1+2(i-1)+1=3i.
$$
If $\left(h_A-(h_{A^{(1)}})_+\right)({\frac{d}{2}})= \frac{d}{2}-1$ then for $1\leq i\leq \frac{d}{2}-2$ we have $h_A(i)=i+1+2(i-1)+1=3i$. On the other hand, for every $d\geq 6$ we have that 
\begin{equation*}
\Cat_F(\frac{d}{2}-1)=\left[\begin{array}{@{}c|c|c@{}}
\mathbf{0}& \mathbf{0}&{\Cat_{G_{d-2}}{(\frac{d}{2}-3)}}
\\\hline
\mathbf{0}&\Cat_{G_{d-2}}{(\frac{d}{2}-2)}&\Cat_{G_{d-1}}{(\frac{d}{2}-2)}\\\hline
\Cat_{G_{d-2}}{(\frac{d}{2}-1)}&\Cat_{G_{d-1}}{(\frac{d}{2}-1)}&\Cat_{G_{d}}{(\frac{d}{2}-1)}
\end{array}
\right].\\
\end{equation*}
Which implies that 
 $$
 \frac{3d}{2}-3= h_{A^{(2)}}(\frac{d}{2}-3)+h_{A^{(2)}}(\frac{d}{2}-2)+h_{A^{(2)}}(\frac{d}{2}-1)\leq h_A(\frac{d}{2}-1), 
 $$
 and since  $\Cat_F(\frac{d}{2}-1)$ is a square matrix of size $\frac{3d}{2}-3$ we get  $h_A(\frac{d}{2}-1)=\frac{3d}{2}-3$. For $d=4$ similar argument implies that 
  $$
3= h_{A^{(2)}}(0)+h_{A^{(2)}}(1)\leq h_A(1).
 $$
For $d=2$ there is noting to show.
\end{proof}
Now we state and prove the analogues statements to Lemma \ref{maxvaluesevenLemma} and Theorem \ref{3linesHFtheorem-even} for Artinian Gorenstein  algebras with odd socle degrees.

\begin{lemma}[Odd socle degree]\label{maxvaluesoddLemma}
There exists an Artinian Gorenstein algebra $A$ with odd socle degree $d\geq 3$ and linear form $\ell\in A_1$ where $\ell^2\neq 0$ and $\ell^3=0$, such that 
$$
(r,s,t)=\left( h_{A^{(2)}}(\frac{d-1}{2}),  h_{A^{(1)}}(\frac{d-1}{2}),  h_{A}(\frac{d-1}{2})\right)
$$
if and only if 
\begin{itemize}
\item[$(1)$] $r\in [1,\frac{d-1}{2}-1], s\in[2r,\frac{d-1}{2}+r]$ and $t\in[2s-r,\frac{d-1}{2}+s+1]$, for $d\geq 5$; or 
\item[$(2)$] $r\in [1,\frac{d-1}{2}-1], s=\frac{d-1}{2}+r+1$ and $t=d+r$, for $d\geq 5$; or 
\item[$(3)$] $r=\frac{d-1}{2}, s\in [d-1,d]$ and $t\in [\frac{d-1}{2}+s-1,3\frac{d-1}{2}]$, for $d\geq 3$.
\end{itemize}
\end{lemma}
\begin{proof}
The maximum value of the Hilbert function of $A$ occurs in degree $\frac{d-1}{2}$ and it is equal to the rank of the following catalecticant matrix
 \begin{equation}
\Cat_F(\frac{d-1}{2})=\left[\begin{array}{@{}c|c|c@{}}
\mathbf{0}& \mathbf{0}&{\Cat_{G_{d-2}}{(\frac{d-1}{2}-2)}}
\\\hline
\mathbf{0}&\Cat_{G_{d-2}}{(\frac{d-1}{2}-1)}&\Cat_{G_{d-1}}{(\frac{d-1}{2}-1)}\\\hline
\Cat_{G_{d-2}}{(\frac{d-1}{2})}&\Cat_{G_{d-1}}{(\frac{d-1}{2})}&\Cat_{G_{d}}{(\frac{d-1}{2})}
\end{array}
\right]\\
\end{equation}
which is equal to
\begin{Small}
\begin{equation}\label{catmatrix3linesodd}
 \Cat_F(\frac{d-1}{2}) = \left[\begin{array}{@{}cccc|cccc|cccc@{}}
    0 & 0 & \cdots & 0 &  0 & 0 & \cdots & 0 & c_{0} & c_{1}&\cdots & c_{\frac{d+1}{2}} \\
     0 & 0 & \cdots & 0 & 0 & 0 &\cdots & 0 & c_{1} & c_{2} &\cdots & c_{\frac{d+1}{2}+1} \\
    \vdots & \vdots & \reflectbox{$\ddots$}  &\vdots & \vdots & \vdots & \reflectbox{$\ddots$}  &\vdots  & \vdots  &  \vdots & \reflectbox{$\ddots$} &\vdots\\
 0 & 0 & \cdots & 0 &  0 & 0 &\cdots & 0 &  c_{\frac{d-1}{2}-2} & c_{\frac{d-1}{2}-1} &\cdots & c_{d-2} \\\hline
  0 & 0 & \cdots & 0 &  c_{0} & c_{1}&\cdots & c_{\frac{d-1}{2}} & b_{0} & b_{1}&\cdots & b_{\frac{d+1}{2}} \\
 0 & 0 & \cdots & 0 &   c_{1} & c_{2} &\cdots & c_{\frac{d-1}{2}+1}  & b_{1} & b_{2} &\cdots & b_{\frac{d+1}{2}+1} \\
    \vdots  &\vdots & \reflectbox{$\ddots$}  &  \vdots&   \vdots  &\vdots & \reflectbox{$\ddots$}  &  \vdots& \vdots  &  \vdots & \reflectbox{$\ddots$} &\vdots\\
  0 & 0 & \cdots & 0 &  c_{\frac{d-1}{2}-1} & c_{\frac{d-1}{2}} &\cdots & c_{d-2} & b_{\frac{d-1}{2}-1} & b_{\frac{d-1}{2}} &\cdots & b_{d-1} \\\hline
     c_{0} & c_{1}&\cdots & c_{\frac{d-1}{2}-1} & b_{0} & b_{1}&\cdots & b_{\frac{d-1}{2}} &a_{0} & a_{1}&\cdots & a_{\frac{d+1}{2}} \\
 c_{1} & c_{2} &\cdots & c_{\frac{d-1}{2}}  & b_{1} & b_{2} &\cdots & b_{\frac{d-1}{2}+1}& a_{1} & a_{2} &\cdots & a_{\frac{d+1}{2}+1} \\
     \vdots  &\vdots & \reflectbox{$\ddots$}  &  \vdots& \vdots  &  \vdots & \reflectbox{$\ddots$} &\vdots& \vdots  &  \vdots & \reflectbox{$\ddots$} &\vdots\\
    c_{\frac{d-1}{2}} & c_{\frac{d-1}{2}+1} &\cdots & c_{d-2} & b_{\frac{d-1}{2}} & b_{\frac{d-1}{2}+1} &\cdots & b_{d-1}& a_{\frac{d-1}{2}} & a_{\frac{d-1}{2}+1} &\cdots & a_{d} \\
     \end{array}\right].
\end{equation}
\end{Small}
We note that $r=h_{A^{(2)}}(\frac{d-1}{2})\in [1,\frac{d-1}{2}]$. First assume that $r\in [1,\frac{d-1}{2}-1]$ and note that  the socle degree of $A^{(2)}$ is odd, then we have that $
 h_{A^{(2)}}(\frac{d-1}{2}-2)= h_{A^{(2)}}(\frac{d-1}{2}-1)= \linebreak h_{A^{(2)}}(\frac{d-1}{2})=r.
$
 We may assume that the ranks of the lower right submatrices of \linebreak$\Cat_{G_{d-2}}(\frac{d-1}{2}-2), \Cat_{G_{d-2}}(\frac{d-1}{2}-1)$ and $\Cat_{G_{d-2}}(\frac{d-1}{2})$ are equal to $r$. Setting  $c_{d-r-1}=1$ and $c_i=0$ for every $i\neq d-r-1$, or equivalently setting $G_{d-2}$ as the following  provides the desired property
\begin{equation}\label{Gd-2}
G_{d-2}= \frac{Y^{r-1}Z^{d-r-1}}{(r-1)!(d-r-1)!}, \quad \text{for each}\quad r\in[1,\frac{d-1}{2}-1].
\end{equation}
The Hilbert function of $A^{(1)}$ in degree $\frac{d-1}{2}$ is equal to $2r+\rk\mathbf{B}$ where 
\begin{equation}\label{matrixBodd}
\mathbf{B}=\left(\begin{array}{@{}ccccccc@{}}
 b_{0} &\cdots & b_{\frac{d-1}{2}-r} \\
    \vdots  & \reflectbox{$\ddots$} &\vdots\\
b_{\frac{d-1}{2}-r}&\cdots & b_{d-1-2r}      \end{array}\right).
\end{equation}
So $s=h_{A^{(1)}}(\frac{d-1}{2})\in [2r, \frac{d-1}{2}+r+1]$. Suppose that  $s\in [2r, \frac{d-1}{2}+r]$. This implies that $ h_{A^{(1)}}(\frac{d-1}{2}-1)= h_{A^{(1)}}(\frac{d-1}{2}) = s$. We prove $(1)$ we use the same argument that we used to prove Lemma \ref{maxvaluesevenLemma} part $(1)$. Therefore, for the following choice of $G_{d-1}$ and $G_d$ we get the Hilbert functions as in $(1)$.
\begin{equation*}
G_{d-1}=\left\{
                \begin{array}{ll}
                  0 & \text{if $s-2r=0$},\\
                  \frac{Y^{d-s+2r}Z^{s-2r-1}}{(d-s+2r)!(s-2r-1)!} & \text{if $1\leq s-2r\leq \frac{d-1}{2}-r,$}\\
                \end{array}
              \right.
\end{equation*}
and 
\begin{equation*}
G_{d}=\left\{
                \begin{array}{ll}
                  0 & \text{if $t-2s+r=0$},\\
                  \frac{Y^{d-t+3r+1}Z^{t-3r-1}}{(d-t+3r+1)!(t-3r-1)!} & \text{if $1\leq t-2s+r\leq \frac{d-1}{2}+r-s+1.$}\\
                \end{array}
              \right.
\end{equation*}
Now assume that $s=h_{A^{(1)}}(\frac{d-1}{2})=\frac{d-1}{2}+r+1$ which is the maximum possible for $r\in [1,\frac{d-1}{2}-1]$.  The following submatrices  of $\Cat_{G_{d-1}}(\frac{d-1}{2})$ having maximal rank that is equal to $\frac{d-1}{2}-r+1$ implies that $\rk\Cat_{x\circ F}(\frac{d-1}{2})=\frac{d-1}{2}+r+1$.
\begin{equation*}
\mathbf{B}=\left[\begin{array}{@{}ccccccc@{}}
 b_{0} &\cdots & b_{\frac{d-1}{2}-r} \\
    \vdots  & \reflectbox{$\ddots$} &\vdots\\
b_{\frac{d-1}{2}-r}&\cdots & b_{d-1-2r}      \end{array}\right].
\end{equation*}
This forces the following submatrix of $\Cat_{G_{d-1}}(\frac{d-1}{2}-1)$ to have maximal rank that is equal to  $\frac{d-1}{2}-r$. 
\begin{equation*}
\mathbf{B^\prime}=\left[\begin{array}{@{}ccccccc@{}}
 b_{0} &\cdots & b_{\frac{d+1}{2}-r} \\
    \vdots  & \reflectbox{$\ddots$} &\vdots\\
b_{\frac{d-1}{2}-1-r}&\cdots & b_{d-1-2r}      \end{array}\right].
\end{equation*}
Setting $b_{\frac{d-1}{2}+r}=1$ and $b_{i}=0$ for every $i\neq \frac{d-1}{2}+r$, or equivalently 
$$
G_{d-1} = \frac{Y^{\frac{d-1}{2}-r}Z^{\frac{d-1}{2}+r}}{(\frac{d-1}{2}-r)!(\frac{d-1}{2}+r)!}
$$
provides that 
$$h_{A^{(1)}}(\frac{d-1}{2})=\frac{d-1}{2}+r+1, \hspace*{2mm}\text{and}h_{A^{(1)}}(\frac{d-1}{2}-1)=\frac{d-1}{2}+r.$$
Since $\mathbf{B}$ and $\mathbf{B^\prime}$ both have maximal ranks for every choice of $G_d$ we get 
$$
h_A(\frac{d-1}{2})=\rk\Cat_{F}(\frac{d-1}{2}) = 3r+\rk\mathbf{B}+\rk\mathbf{B^\prime} = d+r.
$$
Now we assume that $r=\frac{d-1}{2}$ as in $(3)$. Since $d$ is an odd integer $h_{A^{(2)}}(\frac{d-1}{2}-1)=h_{A^{(2)}}(\frac{d-1}{2})=\frac{d-1}{2}$ and  $h_{A^{(2)}}(\frac{d-1}{2}-2) = \frac{d-1}{2}-1$. Setting $c_{\frac{d-1}{2}-1}=1$ and $c_i=0$ for every $i\neq \frac{d-1}{2}-1$, that is
\begin{equation}\label{random}
G_{d-2} = \frac{Y^{\frac{d-1}{2}}Z^{\frac{d-1}{2}-1}}{(\frac{d-1}{2})!(\frac{d-1}{2}-1)!},
\end{equation}
implies that $h_{A^{(2)}}(\frac{d-1}{2})=\frac{d-1}{2}$.
In order to find possible values for $h_{A^{(1)}}(\frac{d-1}{2})$, note that
\begin{equation*}
h_{A^{(1)}}(\frac{d-1}{2}) = \rk\left[\begin{array}{@{}c|c@{}}
 \mathbf{0}&{\Cat_{G_{d-2}}{(\frac{d-1}{2}-1)}}
\\\hline
\Cat_{G_{d-2}}{(\frac{d-1}{2})}&\Cat_{G_{d-1}}{(\frac{d-1}{2})}\\
\end{array}
\right],\\
\end{equation*}
is a square matrix of size $d$. On the other hand $$d-1=\rk\Cat_{G_{d-2}}(\frac{d-1}{2}-1)+\rk\Cat_{G_{d-2}}(\frac{d-1}{2}-1)\leq h_{A^{(1)}}(\frac{d-1}{2}).$$ 
For the polynomial $G_{d-2}$ as in (\ref{random}) we get that the last column of the above matrix is zero. So setting $G_{d-1}=0$ gives $h_{A^{(1)}}(\frac{d-1}{2})=d-1$ and setting $G_{d-1}=\frac{Z^{d-1}}{(d-1)!}$ gives that $h_{A^{(1)}}(\frac{d-1}{2})=d$.
To find possible values for $h_A(\frac{d-1}{2})$ note that the number of rows in the catalecticant matrix (\ref{catmatrix3linesodd}) is equal to  $3\frac{d-1}{2}$. 
If $h_{A^{(1)}}(\frac{d-1}{2})=d$ then  $h_A(\frac{d-1}{2})$ is the maximum possible independent of  the choice of $G_d$ since 
$$\rk\Cat_F(\frac{d-1}{2})\geq  h_{A^{(1)}}(\frac{d-1}{2})+h_{A^{(2)}}(\frac{d-1}{2}-2)=d+\frac{d-1}{2}-1=3\frac{d-1}{2}.$$
If $h_{A^{(1)}}(\frac{d-1}{2})=d-1$, then 
$$
\rk\Cat_F(\frac{d-1}{2})\geq  h_{A^{(1)}}(\frac{d-1}{2})+h_{A^{(2)}}(\frac{d-1}{2}-2)=d-1+\frac{d-1}{2}-1=3\frac{d-1}{2}-1.
$$ Setting $G_{d-1}=0$ and $G_{d}=0$ provides that 
$ h_{A}(\frac{d-1}{2})=3\frac{d-1}{2}-1$ and setting $G_{d-1}=0$ and $G_{d}=\frac{Z^d}{d!}$ provides that $ h_{A}(\frac{d-1}{2})=3\frac{d-1}{2}$ by making the last column of $\Cat_F(\frac{d-1}{2})$ linearly independent. 
\end{proof}
We now determine the rank matrices for Artinian Gorenstein algebras having odd socle degree $d\geq 3$ for every possible triples $(r,s,t)$ provided in the above lemma. 

\begin{theorem}[Odd socle degree]\label{3linesHFtheorem-odd}
Let $A$ be an Artinian Gorenstein algebra with odd socle degree $d\geq 3$ and $\ell\in A_1$ such that $\ell^2\neq 0$ and $\ell^3=0$. Then Hilbert functions of $A$, $A^{(1)}$ and $A^{(2)}$ are completely determined by $(r,s,t)=( h_{A^{(2)}}(\frac{d-1}{2}),  h_{A^{(1)}}(\frac{d-1}{2}),  h_{A}(\frac{d-1}{2}))$. More precisely,   
\begin{itemize}
\item[$(1)$] if $d\geq 5$, $r\in [1,\frac{d-1}{2}-1], s\in[2r,\frac{d-1}{2}+r]$ and $t\in [2s-r,\frac{d-1}{2}+s+1]$, then 
\begin{equation}\label{HFodd(1)}
h_{A^{(2)}}(i)=\left\{
                \begin{array}{ll}
                  i+1 & 0\leq i\leq r-1,\\
                  r & r\leq i\leq\frac{d-1}{2},\\
                \end{array}
              \right.\quad
h_{A^{(1)}}(i)=\left\{
                \begin{array}{ll}
                  2i+1 &  0\leq i\leq r-1,\\
                   i+r+1 &  r\leq i\leq s-r-1,\\
                  s & s-r\leq i\leq \frac{d-1}{2}.\\
                \end{array}
              \right.
 \end{equation}
 \begin{itemize}
\item If $t=3r$ then there are two possible Hilbert functions for $A$
\begin{equation}\label{HFoddA(1,1)}
h_{A}(i)=\left\{
                \begin{array}{ll}
                 1 & i=0,\\
                  3i &  1\leq i\leq r-1,\\
                  3r & r\leq i\leq\frac{d-1}{2}.\\
                \end{array}
              \right.
\text{and} \quad 
h_{A}(i)=\left\{
                \begin{array}{ll}
                   1 & i=0,\\
                  3i &  1\leq i\leq r-1,\\
                  3r-1& i=r,\\
                  3r & r+1\leq i\leq \frac{d-1}{2},\\
                \end{array}
              \right.
 \end{equation}
\item otherwise
\begin{equation}\label{HFoddA(1,2)}
h_A(i)= \left\{
                \begin{array}{ll}
                   1 & i=0,\\
                  3i &  1\leq i\leq r,\\
                  2i+r+1& r+1\leq i\leq s-r-1,\\
                  2i+r+1& i=s-r, \hspace*{2mm}\text{if}\hspace*{2mm}t>2s-r\hspace*{2mm}\text{and}\hspace*{2mm}s>2r,\\
                  2i+r& i=s-r, \hspace*{2mm}\text{if}\hspace*{2mm}t>2s-r \hspace*{2mm}\text{and}\hspace*{2mm}s=2r,\\
                  i+s+1 & s-r+1\leq i\leq t-s-1,\\
                  t & t-s\leq i\leq \frac{d-1}{2}.\\
                \end{array}
              \right.
\end{equation}
 \end{itemize}
 \item[$(2)$] If $d\geq 3$, $r\in[1,\frac{d-1}{2}-1],s=\frac{d-1}{2}+r+1 $ and $t=d+r$, then 
\begin{equation}\label{HFodd(2)}
h_{A^{(1)}}(i)= \left\{
                \begin{array}{ll}
                  2i+1 &  0\leq i\leq r,\\
                  i+1+r & r+1\leq i\leq\frac{d-1}{2},\\
                \end{array}
              \right. \text{and}\quad h_{A}(i)= \left\{
                \begin{array}{ll}
                1& i=0,\\
                  3i &  1\leq i\leq r+1,\\
                  2i+1+r & r+2\leq i\leq\frac{d-1}{2}.\\
                \end{array}
              \right. 
\end{equation}
\item[$(3)$] For $d\geq 3$ if  $r=\frac{d-1}{2}, s\in [d-1,d]$ and $t\in [\frac{d-1}{2}+s-1,3\frac{d-1}{2}]$, then for every $i\in [0,\frac{d-1}{2}-1]$
  \begin{equation}\label{HFodd(3)}
 h_{A^{(2)}}(i) = i+1, \quad h_{A^{(1)}}(i) = 2i+1\hspace*{2mm}\text{and} \hspace*{2mm} h_A(0)=1,  h_A(i) =3i.
  \end{equation}
 \end{itemize}
\end{theorem}
\begin{proof}
To prove $(1)$ we observe that the assumption on $r$ and $s$ implies that 
$$ h_{A^{(1)}}(\frac{d-1}{2}-1)=h_{A^{(1)}}(\frac{d-1}{2})=h_{A^{(1)}}(\frac{d-1}{2}+1)=s,
$$
and 
$$ h_{A^{(2)}}(\frac{d-1}{2}-1)=h_{A^{(2)}}(\frac{d-1}{2})=r.
$$
Therefore, applying Theorem \ref{3linesHFtheorem-even} part $(1)$ for $d-1$ implies the result.\par 
\noindent Now we show $(2)$ first note that $h_{A^{(2)}}$ is the same as the previous case. By the assumption we have that 
\begin{align*}
(h_{A^{(1)}}-(h_{A^{(2)}})_+)(\frac{d-1}{2}) &= h_{A^{(1)}}(\frac{d-1}{2})-h_{A^{(2)}}(\frac{d-1}{2}-1)\\
&= h_{A^{(1)}}(\frac{d-1}{2})-h_{A^{(2)}}(\frac{d-1}{2})\\
&=\frac{d-1}{2}+r+1-r= \frac{d-1}{2}+1.
\end{align*}
Using Lemma \ref{diffO-seq} we get that
$(h_{A^{(1)}}-(h_{A^{(2)}})_+)(i) = i+1
$  for every $i\in[0,\frac{d-1}{2}]$.
Therefore, we get $h_{A^{(1)}}$ as we claimed.
To obtain $h_A$ we note that 
\begin{align*}
(h_{A}-(h_{A^{(1)}})_+)(\frac{d-1}{2})  = h_{A}(\frac{d-1}{2})-h_{A^{(1)}}(\frac{d-1}{2}-1)= d+r- h_{A^{(1)}}(\frac{d-1}{2}-1).
\end{align*}
If $r<\frac{d-1}{2}-1$ then we have $h_{A^{(1)}}(\frac{d-1}{2}-1)=\frac{d-1}{2}+r$ and if $r=\frac{d-1}{2}-1$ then $h_{A^{(1)}}(\frac{d-1}{2}-1)=2(\frac{d-1}{2}-1)+1=d-2.$ In both cases we get that 
\begin{align*}
(h_{A}-(h_{A^{(1)}})_+)(\frac{d-1}{2})  = d+r-(d-2) = r+2 = \frac{d-1}{2}+1.
\end{align*}
So for every $i\in[0,\frac{d-1}{2}]$ we have $(h_{A}-(h_{A^{(1)}})_+)(i) = i+1,
$ and therefore
$
h_A(i)=i+1+h_{A^{(1)}}(i-1)
$
which implies the desired Hilbert function for $A$.

To prove $(3)$ we get $h_{A^{(2)}}$ by replacing $r$ by $\frac{d-1}{2}$ in the previous case. By the assumption we have that 
$$
\frac{d-1}{2}\leq (h_{A^{(1)}}-(h_{A^{(2)}})_+)(\frac{d-1}{2})\leq \frac{d-1}{2}+1.
$$
So for every $i\in [0,\frac{d-1}{2}-1]$ we get that $$h_{A^{(1)}}(i)=i+1+h_{A^{(2)}}(i-1)=2i+1.$$
To obtain $h_A$ we observe that
\begin{align*}
(h_{A}-(h_{A^{(1)}})_+)(\frac{d-1}{2}) \geq \frac{d-1}{2}+s-1-(d-2)
=s-\frac{d-1}{2}\geq \frac{d-1}{2}.
\end{align*}
Therefore, for every $i\in [0,\frac{d-1}{2}-1]$ we have $(h_{A}-(h_{A^{(1)}})_+)(i)=i+1$ and equivalently we have $h_A(0)=1$ and $h_A(i)=i+1+2(i-1)+1=3i$ for every $i\in [1,\frac{d-1}{2}-1]$.
\end{proof}
We prove that the lists of rank tables given in Theorems \ref{3linesHFtheorem-even} and \ref{3linesHFtheorem-odd} are exhaustive lists. 
\begin{theorem}\label{allHF´sPossible-even}
A vector of non-negative integers $h$ is the Hilbert function of some Artinian    Gorenstein algebra  $A=S/\ann(F)$ such that there exists a linear form $\ell$ satisfying $ \ell^2\neq 0$ and $\ell^3=0$ if and only if $h$ is equal to one of the Hilbert functions provided in Theorems \ref{3linesHFtheorem-even} and \ref{3linesHFtheorem-odd}.
\end{theorem}
\begin{proof}
In Lemmas \ref{maxvaluesevenLemma} and \ref{maxvaluesoddLemma} we provide the complete list of possible values for the maximum of the Hilbert function of any Artinian Gorenstein algebras $A$ where $\ell^3=0$. In fact, for each maximum value we produce a dual generator $F$ for $A$. On the other hand, in Theorems \ref{3linesHFtheorem-even} and \ref{maxvaluesoddLemma}  we prove that for each possible maximum value the Hilbert function of $A$ is uniquely determined by the maximum value in all the cases except when $r\in [1,\lfloor\frac{d}{2}\rfloor-1]$ and $t=3r$ for every $d\geq 4$, in which we have two possibilities for $h_A$. We show that both Hilbert functions provided for $h_A$ occur for some Artinian Gorenstein algebra $A$.\par 
First assume that $d\geq 6$, $r\in[2,\lfloor\frac{d}{2}\rfloor-1]$ and $t=3r$ which implies that $s=2r$.  Fixing $h_A(r)$ to be either $3r-1$ or $3r$ we provide a degree $d$ polynomial satisfying (\ref{F}) as the dual generator for Artinian Gorenstein algebra $A$ such that $h_A(\lfloor\frac{d}{2}\rfloor)=3r$.  
%We provide two polynomials of degree $d$ as dual generators for Artinian Gorenstein algebras satisfying $h_A(\lfloor\frac{d}{2}\rfloor)=3r$  and the Hilbert function of each of which in degree $r$ is equal to $3r-1$ and $3r$ are both possible values for the Hilbert function in degree $r$ of some  Artinian Gorenstein algebra. Fixing $h_A(r)$ to be either $3r-1$ or $3r$ we provide a degree $d$ polynomial as the dual generator for Artinian Gorenstein algebra $A$.  
Pick the following monomial basis for $A_r$
$$\B_r=\{x^r,x^{r-1}y,x^{r-1}z,x^{r-2}y^2,\dots ,y^r,y^{r-1},z,\dots , z^r\}$$
so
\begin{equation}
\Cat_F(r)=\left[\begin{array}{@{}c|c|c@{}}
\mathbf{0}& \mathbf{0}&{\Cat_{G_{d-2}}{(r-2)}}
\\\hline
\mathbf{0}&\Cat_{G_{d-2}}{(r-1)}&\Cat_{G_{d-1}}{(r-1)}\\\hline
\Cat_{G_{d-2}}{(r)}&\Cat_{G_{d-1}}{(r)}&\Cat_{G_{d}}{(r)}
\end{array}
\right].\\
\end{equation}
Which is equal to 
\begin{equation}\label{catmatrix3lines-r}
 \Cat_F(r) = \left[\begin{array}{@{}cccc|cccc|cccc@{}}
    0 & 0 & \cdots & 0 &  0 & 0 & \cdots & 0 & c_{0} & c_{1}&\cdots & c_{d-r} \\
     0 & 0 & \cdots & 0 & 0 & 0 &\cdots & 0 & c_{1} & c_{2} &\cdots & c_{d-r+1} \\
    \vdots & \vdots & \reflectbox{$\ddots$}  &\vdots & \vdots & \vdots & \reflectbox{$\ddots$}  &\vdots  & \vdots  &  \vdots & \reflectbox{$\ddots$} &\vdots\\
 0 & 0 & \cdots & 0 &  0 & 0 &\cdots & 0 &  c_{r-2} & c_{r-1} &\cdots & c_{d-2} \\\hline
  0 & 0 & \cdots & 0 &  c_{0} & c_{1}&\cdots & c_{d-r-1} & b_{0} & b_{1}&\cdots & b_{d-r} \\
 0 & 0 & \cdots & 0 &   c_{1} & c_{2} &\cdots & c_{d-r}  & b_{1} & b_{2} &\cdots & b_{d-+1} \\
    \vdots  &\vdots & \reflectbox{$\ddots$}  &  \vdots&   \vdots  &\vdots & \reflectbox{$\ddots$}  &  \vdots& \vdots  &  \vdots & \reflectbox{$\ddots$} &\vdots\\
  0 & 0 & \cdots & 0 &  c_{r-1} & c_{r} &\cdots & c_{d-2} & b_{r-1} & b_{r} &\cdots & b_{d-1} \\\hline
     c_{0} & c_{1}&\cdots & c_{d-r-2} & b_{0} & b_{1}&\cdots & b_{d-r-1} &a_{0} & a_{1}&\cdots & a_{d-r} \\
 c_{1} & c_{2} &\cdots & c_{d-r-1}  & b_{1} & b_{2} &\cdots & b_{d-r}& a_{1} & a_{2} &\cdots & a_{d-r+1} \\
     \vdots  &\vdots & \reflectbox{$\ddots$}  &  \vdots& \vdots  &  \vdots & \reflectbox{$\ddots$} &\vdots& \vdots  &  \vdots & \reflectbox{$\ddots$} &\vdots\\
    c_{r} & c_{r+1} &\cdots & c_{d-2} & b_{r} & b_{r+1} &\cdots & b_{d-1}& a_{r} & a_{r+1} &\cdots & a_{d} \\
     \end{array}\right].
\end{equation}
By what we have shown in Lemmas \ref{maxvaluesevenLemma} and \ref{maxvaluesoddLemma} part $(1)$ setting $c_{d-r-1}=1$ and all other coefficients in the polynomial $F$ to be zero or equivalently 
\begin{equation}\label{3rFeven}
F = \frac{X^2 Y^{r-1}Z^{d-r-1}}{2(r-1)!(d-r-1)!}.
\end{equation}
provides that $h_A(\lfloor\frac{d}{2}\rfloor)=3r$.
Therefore, since $G_{d-1}=G_d=0$ we get that 
$$
h_A(r)=\rk\Cat_F(r)=\rk\Cat_{F}(r-2)+ \rk\Cat_{F}(r-1)+\rk\Cat_{F}(r)=3r-1,
$$
where the ranks of $\Cat_{F}(r-2), \Cat_{F}(r-1)$ and $\Cat_{F}(r)$ or equally $h_{A^{(2)}}(r-2),h_{A^{(2)}}(r-1)$ and $h_{A^{(2)}}(r)$ are given in Theorems \ref{3linesHFtheorem-even} and \ref{3linesHFtheorem-odd}.

In order to provide a polynomial $F$ as the dual generator of $A$ where $h_A(\lfloor\frac{d}{2}\rfloor)=3r=h_A(r)=3r$ we set $c_{d-r-1}=a_{d-r}=1$ and all other coefficients to zero, so
\begin{equation}\label{3rFeven}
F = \frac{X^2 Y^{r-1}Z^{d-r-1}}{2(r-1)!(d-r-1)!} + \frac{Y^rZ^{d-r}}{r!(d-r)!}.
\end{equation}
We observe that setting $a_{d-r}=1$ in the matrices (\ref{catmatrix3lines}) and  (\ref{catmatrix3linesodd}) the number of linearly independent columns does not increase and is equal to $3r$. On the other hand, setting $a_{d-r}=1$ increases the number of linearly independent columns of $\Cat_F(r)$ in (\ref{catmatrix3lines-r}) by one. In fact, by setting $a_{d-r}=1$ the last columns of (\ref{catmatrix3lines-r}) is not in the span of $3r-1$ previous linearly independent columns. Thus the number of linearly independent columns in (\ref{catmatrix3lines-r}) is equal to $3r$,  so $h_A(r)=3r$.

\noindent Now assuming that $d\geq 4$ and  $r=1$ we use the same argument as the previous case for the following matrix. 
\begin{equation}\label{catmatrix3lines(r=1)}
\Cat_F(r)=\left[\begin{array}{@{}c|c|c@{}}
 \mathbf{0}&{\Cat_{G_{d-2}}{(r-1)}}&{\Cat_{G_{d-1}}{(r-1)}}
\\\hline
\Cat_{G_{d-2}}{(r)}&\Cat_{G_{d-1}}{(r)}&\Cat_{G_{d}}{(r)}\\
\end{array}
\right]\\
\end{equation}
so similarly setting $F = \frac{X^2 Z^{d-2}}{2(d-2)!}$  we get that $\rk\Cat_F(1)=2$ and  setting $F = \frac{X^2 Z^{d-2}}{2(d-2)!} + \frac{Y Z^{d-1}}{(d-1)!}$ we get $\rk\Cat_F(1)=3$. We notice that in both cases we have that $h_A(\lfloor\frac{d}{2}\rfloor)=3$.

\end{proof}
As an immediate consequence of above results we get a complete list of possible rank matrices for Artinian Gorenstein algebras and linear forms where $\ell^2=0$.
\begin{corollary}\label{2linescorollary}
There exists an Artinian Gorenstein algebra $A$ with socle degree $d\geq 2$ and $\ell\in A_1$ where $\ell\neq 0$ and $\ell^2=0$,  such that 
$$
(r,s)=\left(h_{A^{(1)}}(\lfloor\frac{d}{2}\rfloor), h_{A}(\lfloor\frac{d}{2}\rfloor)\right)
$$
 if and only if 
 \begin{itemize}
 \item $r\in[1,\lceil\frac{d}{2}\rceil-1]$ and $s\in [2r,\lceil\frac{d}{2}\rceil+r ]$, for $d\geq 3$; or
 \item $r= \lceil\frac{d}{2}\rceil$ and $s=d$ if $d\geq 3$ is odd; $s=d,d+1$ if  $d\geq 2$ is even.
 \end{itemize}
 Moreover, the Hilbert functions of $A$ and $A^{(1)}$ are completely determined by $(r,s)$ as the following 
\begin{equation}\label{HF2lines}
h_{A^{(1)}}(i)=\left\{
                \begin{array}{ll}
                  i+1 & 0\leq i\leq r-1,\\
                  r & r\leq i\leq\lfloor\frac{d}{2}\rfloor.\\
                \end{array}
              \right.\quad
h_{A}(i)=\left\{
                \begin{array}{ll}
                  2i+1 &  0\leq i\leq r-1,\\
                   i+r+1 &  r\leq i\leq s-r-1,\\
                  s & s-r\leq i\leq \lfloor\frac{d}{2}\rfloor.\\
                \end{array}
              \right.
 \end{equation}
\end{corollary}
\begin{proof}
We get the desired result by considering rank matrices with two non-zero diagonals given by $h_{A^{(1)}}$ and $h_{A^{(2)}}$ provided in Theorems \ref{3linesHFtheorem-even} and  \ref{3linesHFtheorem-odd}.
\end{proof}
\begin{remark}
The above threorems provides a complete list of rank matrices for Artinian Gorenstein algebras of codimension two and three such that $\ell^3=0$. In fact, there might exists a linear form $\ell^\prime\neq \ell$ such that $\ell^\prime= 0$. 
\end{remark}
We are now able to formulate our last result which provides a formula to compute Jordan types of Artinian Gorenstein algebras with parts of length at most four in terms of at most three parameters corresponding to $(r,s,t)$ in the above theorems.
\begin{theorem}\label{JT-theorem}
Let $A=S/\ann(F)$ be an Artinian  Gorenstein algebra with socle degree $d\geq 2$ and  $\ell\neq 0$ be a linear form such that $\ell^4=0$. The Jordan type $P_{\ell,A}$ is one of the followings.
\begin{itemize}
\item If $\ell^3\neq 0$ then the Jordan type partition of $A$ for $\ell$ is given by 
\begin{equation}\label{JT4}
P_{\ell,A}=(\underbrace{4,\dots ,4}_{\Delta^2\mathbf{d}(3)},\underbrace{3,\dots ,3}_{\Delta^2\mathbf{d}(2)}, \underbrace{2,\dots ,2}_{\Delta^2\mathbf{d}(1)},\underbrace{1,\dots ,1}_{\Delta^2\mathbf{d}(0)}),
\end{equation}
where $\mathbf{d}=(\dim_\mathsf{k} A,\dim_\mathsf{k} A^{(1)},\dim_\mathsf{k} A^{(2)}, \dim_\mathsf{k} A^{(3)})$ and the Hilbert functions of $A^{(1)}$, $A^{(2)}$ and $A^{(3)}$ are given in Theorems  \ref{3linesHFtheorem-even} and  \ref{3linesHFtheorem-odd} for parameters $$r=\rk\Hess_\ell^{(\lfloor\frac{d}{2}\rfloor-1, \lceil\frac{d}{2}\rceil-2)}(F), s=\rk\Hess_\ell^{(\lfloor\frac{d}{2}\rfloor-1, \lceil\frac{d}{2}\rceil-1)}(F), \hspace*{2mm} \text{and} \hspace*{2mm} t=\rk\Hess_\ell^{(\lfloor\frac{d-1}{2}\rfloor, \lfloor\frac{d}{2}\rfloor)}(F).$$ 
\noindent Moreover, if $t\neq 3r$ then $P_{\ell,A}$ is uniquely determined by non-zero integers $(r,s,t)$. Otherwise, if $t=3r$ then $P_{\ell,A}$ is uniquely determined by non-zero integers \linebreak$(r,\rk\Hess_\ell^{(r,d-r-1)}(F)).$

\item If $\ell^3=0$ and $\ell^2\neq 0$, then 
\begin{equation}\label{JT3}
P_{\ell,A}=(\underbrace{3,\dots ,3}_{\Delta^2\mathbf{d}(2)},\underbrace{2,\dots ,2}_{\Delta^2\mathbf{d}(1)}, \underbrace{1,\dots ,1}_{\Delta^2\mathbf{d}(0)}),
\end{equation}
where $\mathbf{d}=(\dim_\mathsf{k} A,\dim_\mathsf{k} A^{(1)},\dim_\mathsf{k} A^{(2)})$ and the Hilbert functions of $A^{(1)}$ and $A^{(2)}$ are given in Corollary \ref{2linescorollary} for parameters  $$r=\rk\Hess_\ell^{(\lfloor\frac{d-1}{2}\rfloor, \lfloor\frac{d}{2}\rfloor)}(F),\hspace*{2mm} \text{and} \hspace*{2mm}  s=\rk\Hess_\ell^{(\lfloor\frac{d-1}{2}\rfloor, \lfloor\frac{d}{2}\rfloor-1)}(F).$$
 Moreover, $P_{\ell,A}$ is uniquely determined by non-zero integers $(r,s).$

\item If $\ell^2=0$ and $\ell\neq 0$, then 
\begin{equation}\label{JT2}
P_{\ell,A}=(\underbrace{2,\dots ,2}_{\Delta^2\mathbf{d}(1)},\underbrace{1,\dots ,1}_{\Delta^2\mathbf{d}(0)}),
\end{equation}
where $\mathbf{d}=(\dim_\mathsf{k} A,\dim_\mathsf{k} A^{(1)})$ 
where the Hilbert function of $A^{(1)}$ is given in Corollary \ref{2linescorollary} for parameter $$r=\rk\Hess_\ell^{(\lfloor\frac{d-1}{2}\rfloor, \lfloor\frac{d}{2}\rfloor)}(F).$$
Moreover, $P_{\ell,A}$ is uniquely determined by the non-zero integer $r.$
\end{itemize}
\end{theorem}
\begin{proof}
First assume that $\ell^3\neq 0$ and notice that the socle degree of  $A^{(1)}=S/\ann(\ell\circ F)$ equals to $d-1$. Recall from Remark \ref{r_ij-remark} that
$$r=\rk\Hess_\ell^{(\lfloor\frac{d}{2}\rfloor-1,\lceil\frac{d}{2}\rceil-2 )}(F)= h_{A^{(3)}}(\lfloor\frac{d}{2}\rfloor-1), $$
$$s=\rk\Hess_\ell^{(\lfloor\frac{d}{2}\rfloor-1,\lceil\frac{d}{2}\rceil-1 )}(F)= h_{A^{(2)}}(\lfloor\frac{d}{2}\rfloor-1), $$
and 
$$t=\rk\Hess_\ell^{(\lfloor\frac{d-1}{2}\rfloor,\lfloor\frac{d}{2}\rfloor )}(F)= h_{A^{(1)}}(\lfloor\frac{d-1}{2}\rfloor).$$
Then using Theorems  \ref{3linesHFtheorem-even} and \ref{3linesHFtheorem-odd} we get the rank matrix of $A^{(1)}$ in terms of $r,s,t$ or equivalently the ranks of multiplication maps by $\ell$,  $\ell^2$ and $\ell^3$ on $A$ in various degrees. Then using Proposition \ref{JT}, we get that $P_{\ell,A}$ as we claimed in  (\ref{JT4}). Moreover, we proved in Theorems  \ref{3linesHFtheorem-even} and \ref{3linesHFtheorem-odd} that the rank table of $A^{(1)}$ is uniquely determined in terms of $r,s$ and $t$ except when $t=3r$ where there are two possible rank tables for $A^{(1)}$ in which it is determined uniquely in terms of $r$ and $\Hess_\ell^{(r,d-r-1)}(F)$.

Now suppose that $\ell^3=0$ and $\ell^2\neq 0$. Ranks of multiplication maps on $A$ by $\ell$ and $\ell^2$ are uniquely determined by $r$ and $s$ in Corollary \ref{2linescorollary}, where 
$$r=\rk\Hess_\ell^{(\lfloor\frac{d-1}{2}\rfloor, \lfloor\frac{d}{2}\rfloor)}(F)=h_{A^{(2)}}(\lfloor\frac{d-1}{2}\rfloor),\hspace*{2mm} \text{and} \hspace*{2mm}  s=\rk\Hess_\ell^{(\lfloor\frac{d-1}{2}\rfloor, \lfloor\frac{d}{2}\rfloor-1)}(F)=h_{A^{(1)}}(\lfloor\frac{d-1}{2}\rfloor).$$ Therefore, Proposition \ref{JT} implies that $P_{\ell,A}$ is equal to (\ref{JT3}).

Assume that $\ell^2=0$ and $\ell\neq 0$ and that $r=\rk\Hess_\ell^{(\lfloor\frac{d-1}{2}\rfloor, \lfloor\frac{d}{2}\rfloor)}(F)=h_{A^{(1)}}(\lfloor\frac{d-1}{2}\rfloor)$.  Then using Corollary \ref{2linescorollary} we get $h_{A^{(1)}}$ and equivalently the rank of the multiplication by $\ell$ on $A$ and therefore Proposition \ref{JT} provides the desired Jordan type $P_{\ell,A}$.
\end{proof}
\begin{remark}
One may use  \ref{3linesHFtheorem-even} and \ref{3linesHFtheorem-odd} and get the Jordan degree types with maximum part of length four, similar to the above theorem. Thus such Jordan degree type is also determined uniquely by at most the ranks of three mixed Hessians.\par
More precise formulas for $P_{\ell,A}$ could be obtained directly from rank matrices provided in Theorems \ref{3linesHFtheorem-even} and \ref{3linesHFtheorem-odd}. 
\end{remark}
\begin{example}
Let $A=S/\ann(F)$ be an Artinian Gorenstein algebra where $$F=X^3Y^4+X^3Z^4+X^2YZ^4+Y^3Z^3.$$ We have that 
$$\rk\Hess^{(2,2)}_x(F)=2, \quad \rk\Hess^{(2,3)}_x(F)=4, \hspace*{2mm}\text{and}\hspace*{2mm}\rk\Hess^{(3,3)}_x(F)=7.$$
So using Theorem \ref{3linesHFtheorem-even} for $(r,s,t)=(2,4,7)$ we get the rank matrix for $A^{(1)}=S/\ann(x\circ F)$ and $\ell=x$ that is equal to
$$M_{x,B}= \begin{pmatrix}
1&1&1&0&0&0&0\\
0&3&3&2&0&0&0\\
0&0&6&4&2&0&0\\
0&0&0&7&4&2&0\\
0&0&0&0&6&3&1\\
0&0&0&0&0&3&1\\
0&0&0&0&0&0&1\\
\end{pmatrix}.
$$
Therefore, using  Equation \ref{JT4} we get the Jordan type of $A$ and $x$, which is equal to
$$
P_{x,A}=(4^8,2^3, 1^{\dim_{\mathsf{k}}A-38})= (4^8,2^3)
$$
sicne we have $h_A=(1,3,6,9,9,6,3,1)$.
Moreover, using the correspondence to the Jordan degree type matrix in Proposition \ref{rkmatrix-1-1-JDT-prop} we get that the Jordan degree type partition of $A$ for $x$ is equal to $(4_0,4_1^2,4_2^2,4_3^2,4_4,2_2, 2_3,2_4).$
\end{example}
\begin{remark}
Based on computations in Macaulay2, we have no example of Artinian Gorenstein algebras over polynomial rings with three variables that the necessary conditions given in Lemmas \ref{diffO-seq} and \ref{additiveRank} are not sufficient.
\end{remark}


\section{Acknowledgment}
The author would like to thank Mats Boij for many helpful discussions. Experiments using the algebra software Macaulay2 \cite{Mac2} were essential to get the ideas behind some of the proofs. This work was supported by the grant VR2013-4545. 
\bibliography{my.bib}{}
\bibliographystyle{plain}
\end{document}