\documentclass[12pt, reqno]{amsart}
\usepackage{graphicx}
\usepackage{hyperref}
\usepackage[caption = false]{subfig}
\usepackage{amsthm}
\usepackage{amssymb}
\usepackage{mathrsfs}
\usepackage{mathtools}
\usepackage{enumerate}
\usepackage{amssymb}
%\usepackage[document]{ragged2e}
\usepackage{wrapfig}
\usepackage{float}
\allowdisplaybreaks

\usepackage[twoside,paperwidth=190mm, paperheight=297mm, top=35mm, bottom=20mm, left=20mm, right=20mm, marginparsep=3mm, marginparwidth=40mm]{geometry}

\numberwithin{equation}{section}
\DeclareMathOperator{\RE}{Re}
\DeclareMathOperator{\IM}{Im}
\theoremstyle{plain}

\newcommand{\set}[1]{\left\{#1\right\}}
\newtheorem{theorem}{Theorem}[section]
\newtheorem{corollary}[theorem]{Corollary}
\newtheorem{example}[theorem]{Example}
\newtheorem{lemma}{Lemma}[section]
\newtheorem{prop}[theorem]{Proposition}
\theoremstyle{definition}
\newtheorem{definition}[theorem]{Definition}
\theoremstyle{remark}
\newtheorem{remark}{Remark}[section]
\newtheorem{proofpart}{}[theorem]
\makeatother
\renewcommand{\proof}{\noindent\textbf{Proof.}}
\renewcommand\theproofpart{\textbf{Part \arabic{proofpart}}}
\renewcommand{\thefootnote}{}
%\setlength{\parindent}{0pt}
\setlength{\parskip}{3pt}

\begin{document}

\title{subordination and radius problems for certain  starlike functions}
	\thanks{The work of the second author is supported by University Grant Commission, New-Delhi, India  under UGC-Ref. No.:1051/(CSIR-UGC NET JUNE 2017).}	
	
	\author[S. Sivaprasad Kumar]{S. Sivaprasad Kumar}
	\address{Department of Applied Mathematics, Delhi Technological University,
		Delhi--110042, India}
	\email{spkumar@dce.ac.in}
	
	
	\author[Kamaljeet]{Kamaljeet Gangania}
	\address{Department of Applied Mathematics, Delhi Technological University,
		Delhi--110042, India}
	\email{gangania.m1991@gmail.com}
	

\maketitle	
	
\begin{abstract} 
	We study the following class of starlike functions $$\mathcal{S}^*_{\wp}:=\left\{f\in\mathcal{A}: {zf'(z)}/{f(z)}\prec\ 1+ze^z=:\wp(z) \right\},$$ that are associated with the cardioid domain $\wp(\mathbb{D})$, by deriving certain convolution results, radius problems, majorization result, radius problems in terms of coefficients and differential subordination implications. Consequently, we establish some interesting generalizations of our results for the Ma-Minda class of starlike functions $\mathcal{S}^{*}(\psi)$. We also provide, the set of extremal functions maximizing 
$\Re\Phi\left(\log{(f(z)/z)}\right)$ or $\left|\Phi\left(\log{(f(z)/z)}\right)\right|$
for functions in  $\mathcal{S}^{*}(\psi)$, where $\Phi$ is a non-constant entire function. Further T. H. MacGregor's result for the class $\mathcal{S}^{*}(\alpha)$ and $\mathcal{S}^*_{\wp}$ are obtained as special case to our result.
	
\end{abstract}
\vspace{0.5cm}
	\noindent \textit{2010 AMS Subject Classification}. Primary 30C45, Secondary 30C50, 30C80.\\
	\noindent \textit{Keywords and Phrases}.  Starlike functions; Convolution; Radius problem; Extremal problem; Subordination; Majorization; Special functions.

\maketitle
	
	\section{Introduction}
Let $\mathcal{A}_{0}$ be the collection of analytic functions of the form $p(z)=1+p_1z+p_2z+\cdots$ and $\mathcal{A}$ consists of analytic functions, $f$ normalized by the conditions $f(0)=0$ and $f'(0)=1$ defined in the unit disk $\mathbb{D}:=\mathbb{D}_1$, where $\mathbb{D}_r:=\{z: |z|<r\}$. The Carath\'{e}odory class, $\mathcal{P}$ consists of functions $p\in{\mathcal{A}}_0$ with $\Re{p(z)}>0$. Let us denote the class of normalized univalent functions by $\mathcal{S}$. For two analytic functions $f$  and $g$, we say $f$ subordinate to $g$, written as $f\prec g$, if there exists a Schwarz function $w$ such that $f(z)=g(w(z))$. If $g$ is univalent, then $f\prec g$ if and only if $f(\mathbb{D})\subseteq g(\mathbb{D})$ and $f(0)=g(0)$. Many subclasses of $\mathcal{S}$ were introduced and studied by several authors, the chief amongst them is $\mathcal{S}^{*}$ (or $\mathcal{C}$), the class of normalized starlike (or convex) functions which is characterized by the quantity $zf'(z)/f(z)$ (or $1+zf''(z)/f'(z)$), assuming values in the right half plane. In 1992, using subordination, Ma and Minda~\cite{minda94} defined the classes \eqref{mindasclass} and \eqref{mindakclass}, which unify many subclasses of $\mathcal{S}$ and they also studied some of it's common properties.
\begin{equation}\label{mindasclass}
\mathcal{S}^{*}(\psi):=\left\{f\in\mathcal{A}: \frac{zf'(z)}{f(z)}\prec\psi(z)\right\}
\end{equation}
and 
\begin{equation}\label{mindakclass}
\mathcal{C}(\psi):=\left\{f\in\mathcal{A}: 1+\frac{zf''(z)}{f'(z)}\prec\psi(z)\right\},
\end{equation}
where $\psi\in\mathcal{P}$ is univalent and $\psi(\mathbb{D})$ starlike with respect to $\psi(0)=1$. Note that $\mathcal{S}^{*}(\tfrac{1+z}{1-z})$ reduces to the class $\mathcal{S}^{*}$. Now a good amount of literature exists for different choices of $\psi$ in \eqref{mindasclass}. For example, the class $\mathcal{S}(\alpha,\beta):=\mathcal{S}^{*}(p_{\alpha,\beta}(z))$, where
\begin{equation}\label{s(alpha,beta)}
p_{\alpha,\beta}(z):=1+\frac{\beta-\alpha}{\pi}i\log\frac{1-e^{2\pi i\frac{1-\alpha}{\beta-\alpha}}z}{1-z},
\end{equation}
$\alpha<1$, $\beta>1$ and $p_{\alpha,\beta}$ maps $\mathbb{D}$ onto the convex domain $\{w\in\mathbb{C}: \alpha< \Re{w}<\beta\}$, was introduced by Kuroki and Owa~\cite{kurokiOwa2011}. Sok\'{o}\l\; and Stankiewicz~\cite{sokol1996} introduced the class associated with lemniscate of Bernoulli $\mathcal{SL}^{*}:=\mathcal{S}^{*}(\sqrt{1+z})$. The class $\mathcal{S}^{*}_{e}:=\mathcal{S}^{*}(e^z)$ of exponential starlike functions was introduced by Mendiratta et al.~\cite{mendi2exp} and the class $\mathcal{S}^{*}_{b}:=\mathcal{S}^{*}(e^{e^z-1})$ of starlike functions related with Bell numbers was introduced by Kumar et at.~\cite{virendraBell}. For some recent classes see \cite{raina2015,kumarravi2016,naveen14,sinefun,goel2020}. Evidently the functions $\psi$ under discussion above are all convex and hence satisfy the following result:
\begin{theorem}\cite{minda94}\label{c1}
	Let $\psi(\mathbb{D})$ be convex, $g\in \mathcal{C}$ and $f\in\mathcal{S}^*(\psi)$. Then	$f*g\in \mathcal{S}^*(\psi).$
\end{theorem}
But the above result is not applicable when $\psi(z)=1+ze^z$, $z+\sqrt{1+z^2}$, $e^{e^z-1}$ and $1+4z/3+2z^2/3$ etc. due to the fact that these are not convex. We deal with Theorem~\ref{c1} for the case when $\psi$ starlike and derive some radius related problems.

In Geometric function theory, we usually find the sufficient conditions in terms of it's coefficients for the normalized functions $f$ in $\mathcal{A}$ to be in a desired class.  In recent times, finding the sufficient conditions on functions to be in a desired class using the technique of differential subordination, admissibility conditions and subordination chain has attracted many researcher's attention. See \cite{Banga-2020,goel2020,sub-pg132}. 

In the present work, we consider the following first-order differential subordination implications:
\begin{equation}\label{sub-work}
1+\beta\frac{zp'(z)}{p^{n}(z)}\prec h(z) \Rightarrow p(z)\prec q(z),
\end{equation}
where $n\in\{1,2,3\}$. We find the condition on $\beta$ so that \eqref{sub-work} holds whenever the functions $h$ and $q$ are already known. Note that if $p$ and $q$ are two fixed analytic and univalent functions in $\mathbb{D}$ and $q$ does not have the nice properties like, explicit inverse representation and convexity, then proving $p\prec q$ is not so easy. See that if $p\prec q $, then geometrically the boundary curve of $p(\mathbb{D})$ is contained in $q(\mathbb{D})$, which can be ensured by the technique of radial distances. Further, if the images $p(\mathbb{D})$ and $q(\mathbb{D})$ are symmetric about the real axis  and both the functions $p$ and $q$ have the same orientation, then we can use the maximum arguments of $p(z)$ and $q(z)$ for $|z|=1$ to prove that $p\prec q$.

Recall that for the subfamilies $\mathcal{G}_1$ and $\mathcal{G}_2$ of $\mathcal{A}$, we say that $r_0$ is the $\mathcal{G}_1$-radius of the class $\mathcal{G}_2$, if $r_0\in(0,1)$ is largest number such that $r^{-1}f(rz)\in \mathcal{G}_1$, $0<r\leq r_0$ for all $f\in \mathcal{G}_2$. Recently, the radii of starlikeness and convexity of some normalized special functions were studied as they can be represented as Hadamard factorization under certain conditions, a few such special functions are Bessel functions~\cite{abo-2018}, Struve functions~\cite{bdoy-2016,abo-2018}, Wright functions~\cite{btk-2-18}, Lommel functions~\cite{bdoy-2016,abo-2018} and  Legendre polynomials of odd degree~\cite{bulut-engel-2019}. If $f$ and $g$ be analytic in $\mathbb{D}$, then $g(z)$ is said to be majorized by $f(z)$, if there exists an analytic function $\Phi(z)$ in $\mathbb{D}$ satisfying $|\Phi(z)|\leq1$ and $g(z)=\Phi(z) f(z)$ for all $z\in\mathbb{D}.$ MacGregor~\cite{mc} found the largest radius $r_0$ so that $|g'(z)|\leq|f'(z)|$ holds in $|z|\leq r_0$ and he also proved that $r_0=2-\sqrt{3}$, if $f$ is univalent. Recently, Tang and Deng~\cite{tang} obtained this majorization-radius when $f$ belongs to some class $\mathcal{S}^{*}(\psi)$.


Consider the class of cardioid starlike functions:
\begin{equation}\label{class}
\mathcal{S}^*_{\wp}:=\left\{f\in\mathcal{A}: \frac{zf'(z)}{f(z)}\prec\wp(z):=1+ze^z\right\}.
\end{equation}
Let $\mathbb{E}_{\alpha, \beta}$ be the normalized form of the Mittag-Leffler function:
\begin{equation*}
\mathbb{E}_{\alpha, \beta}(z)=z+\sum_{n\geq2}\frac{\Gamma(\beta)}{\Gamma(\alpha(n-1)+\beta)}z^n, \quad (z, \alpha, \beta\in\mathbb{C}; \Re(\alpha) > 0, \beta\neq 0,-1,\cdots).
\end{equation*}
Then clearly $\wp(z)=1+\mathbb{E}_{1,1}(z)$.
In \cite{Kumar-cardioid}, $\mathcal{S}^*_{\wp}$ was extensively studied with inclusion properties, radius problems, sharp coefficient's estimations including hankel determinants. Where in, it is proved that $r_c:=(3-\sqrt{5})/2$ is the radius of convexity for $\wp$ however in \cite{BP-mitiig-2016}, $\mathbb{E}_{\alpha, \beta}$ is shown to be convex for $\alpha\geq 1$ and $\beta\geq (3 +\sqrt{17})/2$. So, $\mathbb{E}_{1,1}$ has the radius of convexity $r_c$.

In this paper, we continue to study further in finding the largest radius for some normalized special functions to be in $\mathcal{S}^*_{\wp}$. We also obtain largest radius $r$ such that $f\in \mathcal{S}^*_{\wp}$ in $|z|<r$, whenever $f$ belongs to the class $\mathcal{S}(\alpha,\beta)$ and the class $\mathcal{S}_{\lambda}$ defined in \eqref{s(alpha,beta)} and \eqref{s-lamda}, respectively. We find that special function's radii of starlikness results obtained here naturally generalize to $\mathcal{S}^*(\psi)$-radii. Further, radius problem related to the majorization and some sufficient conditions in terms of coefficients for the normalized functions to be in the class $\mathcal{S}^*_{\wp}$ are established. We discuss the modified form of Theorem~\ref{c1} and as an application, we find the largest radius so that Theorem~\ref{c1} holds for $\mathcal{S}^*_{\wp}$ and the convolution of two starlike functions to be in $\mathcal{S}^*_{\wp}$. In the last section, we obtain the condition on $\beta$ so that the differential subordination implication \eqref{sub-work} holds, whenever $h(z)=\wp(z)$ or $q(z)=\wp(z)$.

\section{Radius problems}
We need the following lemma to prove the subsequent results:
\begin{lemma} \cite{Kumar-cardioid}
	\label{disk_lem}
	Let $\wp(z)=1+ze^z$. Then  we have $\{w : |w-a|<R_a\} \subset \wp(\mathbb{D}),$
	%	\begin{equation*}
	%	\{w : |w-a|<R_a\} \subset \wp(\mathbb{D}),
	%	\end{equation*}
	where
	\[  R_a=
	\left\{
	\begin{array}
	{lr}
	(a-1)+{1}/{e}, &  1-{1}/{e}<a\leq1+(e-e^{-1})/{2}; \\
	e-(a-1),   & 1+(e-e^{-1})/{2}\leq a<1+e.
	\end{array}
	\right.
	\]
\end{lemma}

\begin{theorem}\label{1}
	Let $f\in \mathcal{S}(\alpha,\beta)$. Then $f\in \mathcal{S}^*_{\wp}$ in $\mathbb{D}_{r_0}$, where $r_0$ is the least positive root of the equation
	\begin{equation}\label{radius1}
	\frac{\beta-\alpha}{\pi}\left(\log\frac{1+\sqrt{2(1+\cos(2\pi\frac{1-\alpha}{\beta-\alpha}))}r+r^2}{1-r^2}+2\arctan\frac{r}{1-r}\right)-\frac{1}{e}=0.
	\end{equation}
\end{theorem}
\begin{proof}
	Consider the analytic function $$p_{\alpha,\beta}(z):=1+\frac{\beta-\alpha}{\pi}i \log{q(z)},$$
	where $$q(z)=\frac{1-{c}z}{1-z} \quad\text{and}\quad c=\exp\left(2\pi i\frac{1-\alpha}{\beta-\alpha}\right).$$
	Note that $q(z)$ is a bilinear transformation, maps $\mathbb{D}_{r}$ onto the disk:
	$$\left|q(z)-\frac{1+cr^2}{1-r^2}\right|\leq\frac{|1+c|r}{1-r^2},$$
	which implies  
	\begin{equation*}
	|q(z)| 
	\leq\frac{1+|1+c|r+r^2}{1-r^2}
	\end{equation*}
	and therefore
	\begin{equation}\label{maxlog}
	\log|q(z)|\leq \log\left(\frac{1+|1+c|r+r^2}{1-r^2}\right).
	\end{equation}
	For any $\delta\in \mathbb{C}$ with $|\delta|=1$, we have $1+\delta z\prec 1+z$. So to maximize $|\arg(1+\delta z)|$, it suffices to consider $|\arg(1+z)|$. Now for $|z|=r$, we have
	\begin{equation}\label{maxarg}
	|\arg(1+z)|\leq\arctan\frac{r}{1-r}.
	\end{equation}
	Hence to apply Lemma \ref{disk_lem}, we need to maximize $|p_{\alpha,\beta}(z)-1|$, that is,
	\begin{equation}\label{pab}
	|p_{\alpha,\beta}-1|= \frac{\beta-\alpha}{\pi}\left|\log|q(z)|+i\arg\frac{1-cz}{1-z}\right|.
	\end{equation}
	Using \eqref{maxlog} and \eqref{maxarg} in \eqref{pab}, we see that
	\begin{equation*}
	|p_{\alpha,\beta}-1|\leq \frac{\beta-\alpha}{\pi}\left(\log\frac{1+|1+c|r+r^2}{1-r^2}+2\arctan\frac{r}{1-r}\right)\leq\frac{1}{e}
	\end{equation*}
	holds in $|z|<r_0$ whenever $r_0$ is the smallest positive root of the equation \eqref{radius1}. This ends the proof. \qed
\end{proof}	
Note that if we choose $\alpha=1+\frac{\delta-\pi}{2\sin{\delta}}$ and $\beta=1+\frac{\delta}{2\sin{\delta}}$, where $\pi/2\leq\delta<\pi$, then the class $\mathcal{S}(\alpha,\beta)$ reduces to the class $\mathcal{V}(\delta)$ introduced by Kargar et al. \cite{kargar-ebadian}.
\begin{corollary}
	Let $f\in \mathcal{V}(\delta)$. Then $f\in \mathcal{S}^*_{\wp}$ in $\mathbb{D}_{r_{\delta}}$, where $r_{\delta}$ is the least positive root of the equation
	\begin{equation*}
	\frac{1}{2\sin{\delta}}\left(\log\frac{1+\sqrt{2(1+\cos(2(\pi-\delta)))}r+r^2}{1-r^2}+2\arctan\frac{r}{1-r}\right)-\frac{1}{e}=0.
	\end{equation*}
\end{corollary} 
Now we consider the following class introduced in \cite{cho2019}:
\begin{equation}\label{s-lamda}
\mathcal{S}_{\lambda}:=\left\{f\in\mathcal{A} : \frac{f(z)}{z} \in P_{\lambda}\right\},
\end{equation}
where 
$$P_{\lambda}:=\{p\in\mathcal{A}_{0}: \Re(e^{i\lambda}p(z))>0,\quad -{\pi}/{2} \leq\lambda\leq {\pi}/{2} \}$$
denotes the class of tilted Carath\'{e}odory functions \cite{tilt}. Note that $P_0$ reduces to $\mathcal{P}$, the class of Carath\'{e}odory functions. For the function $p\in P_{\lambda}$, upper bound on the quantity $zp'(z)/p(z)$ is given by the following lemma that will be used for our next result:
\begin{lemma}\label{tilt-lem}\cite{tilt}
	If $p\in P_{\lambda}$, then
	$\left|{zp'(z)}/{p(z)}\right|\leq M(\lambda,r),$
	where
	\begin{equation*}
	M(\lambda,r)= 
	\left\{
	\begin{array}{lll}
	\frac{2r\cos{\lambda}}{r^2-2r|\sin{\lambda}|+1} & $for$ & r<|\tan\frac{\lambda}{2}|;\\
	\frac{2r}{1-r^2} & $for$ & r\geq |\tan\frac{\lambda}{2}|.	
	\end{array}	
	\right.
	\end{equation*}
	The equality holds for some point $z=re^{i\theta}$, $r\in(0,1)$ if and only if $p(z)=p_{\lambda}(yz)$, where $p_{\lambda}(z)=\frac{1+e^{-2i\lambda}z}{1-z}$ and  $y=e^{i(\theta_{0}-\theta)}$ with 
	\begin{equation*}
	\theta_{0}= 
	\left\{
	\begin{array}{lll}
	\frac{\pi}{2}+\lambda & $for$ & r<-\tan\frac{\lambda}{2};\\
	-\frac{\pi}{2}+\lambda  & $for$ & r< \tan\frac{\lambda}{2};\\
	\arcsin\left(\frac{1+r^2}{r^2-1}\right) +\lambda & $for$ &r\geq|\tan\frac{\lambda}{2}|.	
	\end{array}	
	\right.
	\end{equation*}
\end{lemma}

We determine the largest radius $r$ such that the function $F(z):=f(z)g(z)/z\in \mathcal{S}^{*}_{\wp}$ in $|z|<r$, whenever $f,g\in \mathcal{S}_{\lambda}.$
\begin{theorem}\label{2}
	Let $c_{\lambda}=\cos{\lambda}$, $s_{\lambda}=\sin{\lambda}$ and $t_{\lambda}=|\tan({\lambda}/{2})|$. If $f,g\in \mathcal{S}_{\lambda}$, then $F\in\mathcal{S}^{*}_{\wp}$ in $\mathbb{D}_{r_0}$, where
	\begin{equation*}
	r_0:= 
	\left\{
	\begin{array}{lll}
	2ec_{\lambda}+|s_{\lambda}|+\sqrt{((4e^2-1)c_{\lambda}+4e|s_{\lambda}|)c_{\lambda}}, & $if$ & r<t_{\lambda};\\
	\sqrt{4e^2+1}-2e, & $if$ & r\geq t_{\lambda}.	
	\end{array}	
	\right.
	\end{equation*}
\end{theorem}
\begin{proof}
	Since $f,g\in \mathcal{S}_{\lambda}$, it follows that the functions $p(z)=f(z)/z$ and $q(z)=g(z)/z$ belong to the class $P_{\lambda}$ such that $F(z)=zp(z)q(z).$
	%	\begin{equation*}
	%	F(z)=zp(z)q(z).
	%	\end{equation*}
	Thus
	\begin{equation*}
	\frac{zF'(z)}{F(z)}-1=\frac{zp'(z)}{p(z)}+\frac{zq'(z)}{q(z)}.
	\end{equation*}
	Now from Lemma \ref{tilt-lem}, we obtain
	\begin{equation*}
	\left|\frac{zF'(z)}{F(z)}-1\right|\leq 2M(\lambda,r).
	\end{equation*}
	Therefore, using Lemma \ref{disk_lem}, we conclude that if $2M(\lambda,r)\leq 1/e$, then $F\in\mathcal{S}^{*}_{\wp}$. Since $2M(\lambda,r)\leq 1/e$ holds whenever
	\begin{equation*}
	\frac{2rc_{\lambda}}{r^2-2|s_{\lambda}|r+1} \leq \frac{1}{2e},\quad\text{if}\quad r<t_{\lambda}
	\end{equation*}
	and
	\begin{equation*}
	\frac{2r}{1-r^2}\leq \frac{1}{2e},\quad \text{if}\quad r\geq t_{\lambda};
	\end{equation*}
	or equivalently
	\begin{equation*}
	r^2-2(|s_{\lambda}|+2ec_{\lambda})r+1\geq0,\quad \text{if}\quad r<t_{\lambda}
	\end{equation*}
	and
	\begin{equation*}
	r^2+4er-1\leq0,\quad \text{if}\quad r\geq t_{\lambda},
	\end{equation*}
	respectively. Hence the result follows with $r_0$ as given in the hypothesis. Further, for the functions
	$$f(z)=g(z)=\frac{z(1+e^{-2i\lambda}yz)}{1-yz},$$
	sharpness hold in view of Lemma \ref{tilt-lem}. \qed
\end{proof}

\subsection{Cardioid starlikeness radius for some special functions and generalizations}
We begin with the following:
The Bessel function $\mathcal{J}_{\beta}$ of first kind of order $\beta\in\mathbb{C}$ is a particular solution of the homogeneous Bessel differential equation $$z^2w''(z)+zw'(z)+(z^2-{\beta}^2)w(z)=0$$ and have the following series expansion:
\begin{equation*}
\mathcal{J}_{\beta}(z):= \sum_{n\geq1}\frac{(-1)^n}{n!\Gamma(n+\beta+1)}\left(\frac{z}{2}\right)^{2n+\beta},
\end{equation*}
where $z\in\mathbb{C}$ and $\beta\not\in \mathbb{Z}^{-}$. The following three normalized functions expressed in terms of $\mathcal{J}_{\beta}(z)$ have been studied extensively by many authors:
\begin{align}\label{fb}
f_{\beta}(z)&=(2^{\beta}\Gamma(\beta+1)\mathcal{J}_{\beta}(z))^{1/\beta}=z-\frac{1}{4\beta(\beta+1)}z^3+\cdots, \quad \beta\neq0,
\nonumber\\
g_{\beta}(z)&=2^{\beta}\Gamma(\beta+1)z^{1-\beta}\mathcal{J}_{\beta}(z)=z-\frac{1}{4(\beta+1)}z^3+\cdots,\nonumber\\
h_{\beta}(z)&=2^{\beta}\Gamma(\beta+1)z^{1-\beta/2}\mathcal{J}_{\beta}(\sqrt{z})=z-\frac{1}{4(\beta+1)}z^2+\cdots.
\end{align}
Since the zeros of $\mathcal{J}_{\beta}$ are real if $\beta>0$, therefore using Weierstrass decomposition, we have for $\beta>0$:
\begin{equation*}
\mathcal{J}_{\beta}(z):=\frac{z^{\beta}}{2^{\beta}\Gamma(\beta+1)}\prod_{n\geq1}\left(1-\frac{z^2}{j^2_{\beta,n}}\right),
\end{equation*}
where $j_{\beta,n}$ is the $n$-th positive zero of $\mathcal{J}_{\beta}$ and satisfies $j_{\beta,n}<j_{\beta,n+1}$ for $n\in\mathbb{N}$. Thus we have
\begin{equation}\label{besel-rep}
\frac{z\mathcal{J}^{'}_{\beta}(z)}{\mathcal{J}_{\beta}(z)}=\beta-\sum_{n\geq1}\frac{2z^2}{j^2_{\beta,n}-z^2}.
\end{equation}
Now using the above representation, we obtain the radii of cardioid-starlikeness of the functions $f_{\beta}, g_{\beta}$ and $h_{\beta}$ for the case $\beta>0$.

\begin{theorem}\label{3}
	Let $\beta>0$. Then the radii of cardioid-starlikeness $r_{\wp}(f_{\beta})$, $r_{\wp}(g_{\beta})$ and $r_{\wp}(h_{\beta})$ of the functions $f_{\beta}$, $g_{\beta}$ and $h_{\beta}$ are the smallest positive roots of the following equations, respectively:
	\begin{enumerate}[(i)]
		\item $er\mathcal{J}'_{\beta}(r)+\beta(e-1)\mathcal{J}_{\beta}(r)=0;$
		\item $er\mathcal{J}'_{\beta}(r)-(e\beta-1)\mathcal{J}_{\beta}(r)=0;$
		\item $e\sqrt{r}\mathcal{J}'_{\beta}(\sqrt{r})-(e\beta-2)\mathcal{J}_{\beta}(\sqrt{r})=0.$
	\end{enumerate}	
\end{theorem}

\begin{proof}We now prove the first part. Using the representation \eqref{besel-rep} and equation \eqref{fb}, we get
	\begin{equation}\label{fj-rel}
	\frac{zf'_{\beta}(z)}{f_{\beta}(z)}=\frac{z\mathcal{J}^{'}_{\beta}(z)}{\beta\mathcal{J}_{\beta}(z)}=1-\frac{1}{\beta}\sum_{n\geq1}\frac{2z^2}{j^2_{\beta,n}-z^2}.
	\end{equation} 
	Further using a result \cite[Lemma~3.2, p.~10]{ganga1997} and from \eqref{fj-rel}, $|z|=r<j_{\beta,1}$ we obtain 
	\begin{equation}\label{fdisk}
	\left|\frac{zf'_{\beta}(z)}{f_{\beta}(z)}-a\right| \leq \frac{2}{\beta}\sum_{n\geq1}\frac{j^2_{\beta,n}r^2}{j^4_{\beta,n}-r^4},
	\end{equation}
	where $a:=1-\frac{2}{\beta}\sum_{n\geq1}\frac{r^4}{j^4_{\beta,n}-r^4}$ and $j_{\beta,n}$ denotes the $n$-th positive zero of the Bessel function $\mathcal{J}_{\beta}$. Also a simple calculation shows that $a\leq1$. Thus for the disk~\eqref{fdisk} to lie inside $\wp(\mathbb{D})$, we need only to consider that $1-\tfrac{1}{e}<a<1+\tfrac{e-e^{-1}}{2},$ and so by Lemma~\ref{disk_lem}, we have
	$$\frac{2}{\beta}\sum_{n\geq1}\frac{j^2_{\beta,n}r^2}{j^4_{\beta,n}-r^4}\leq a-1+\frac{1}{e}=\frac{1}{e}-\frac{2}{\beta}\sum_{n\geq1}\frac{r^4}{j^4_{\beta,n}-r^4},$$
	or equivalently,
	\begin{equation}\label{bes-eq}
	\frac{2}{\beta}\sum_{n\geq1}\frac{r^2}{j^2_{\beta,n}-r^2}-\frac{1}{e}\leq0.
	\end{equation}
	Also using \eqref{fj-rel}, \eqref{bes-eq} can be written as $\frac{er\mathcal{J}'_{\beta}(r)}{\beta\mathcal{J}_{\beta}(r)}+1-e\geq0$. Note that in view of Lemma~\ref{disk_lem}, we can also obtain \eqref{bes-eq} directly from \eqref{fj-rel}. Now let us consider the strictly decreasing continuous function
	$$\Psi(r):=\frac{1}{e}-\frac{2}{\beta}\sum_{n\geq1}\frac{r^2}{j^2_{\beta,n}-r^2}, \quad  r\in (0,j_{\beta,1}).$$
	Then $\lim_{r\rightarrow0}\Psi(r)=1/e>0$ and $\lim_{r\rightarrow j_{\beta,1}}\Psi(r)=-\infty$. Also $\Psi'(r)<0$, since $r<j_{\beta,1}$. So we may assume $r_{\wp}(f_{\beta})$ be the unique root of $\Psi(r)=0$ in $(0,j_{\beta,1})$ such that $f_{\beta}$ in $\mathcal{S}^*_{\wp}$ in $|z|<r_{\wp}(f_{\beta})$. Proof of other parts follows similarly. \qed
	
\end{proof}


The Struve function $\mathcal{\bf{H}}_{\beta}$ of first kind is a particular solution of the second-order inhomogeneous Bessel differential equation $$z^2w''(z)+zw'(z)+(z^2-{\beta}^2)w(z)=\frac{4(\frac{z}{2})^{\beta+1}}{\sqrt{\pi}\Gamma(\beta+\frac{1}{2})}$$ and have the following form:
\begin{equation*}
\mathcal{\bf{H}}_{\beta}(z):=\frac{(\frac{z}{2})^{\beta+1}}{\sqrt{\frac{\pi}{4}}\Gamma(\beta+\frac{1}{2})} {}_1 F_{2}\left(1;\frac{3}{2},\beta+\frac{3}{2};-\frac{z^2}{4}\right) ,
\end{equation*}
where $-\beta-\frac{3}{2}\notin\mathbb{N}$ and ${}_1 F_{2}$ is a hypergeometric function. Since it is not normalized, so we consider the following three normalized functions involving $\mathcal{\bf{H}}_{\beta}$ :
\begin{align}\label{nor-strv-uvw}
U_{\beta}(z)&=\left(\sqrt{\pi}2^{\beta}(\beta+\frac{3}{2}){\bf{H}}_{\beta}(z)\right)^{\frac{1}{\beta+1}},\nonumber\\
V_{\beta}(z)&=\sqrt{\pi}2^{\beta}z^{-\beta}\Gamma(\beta+\frac{3}{2}){\bf{H}}_{\beta}(z),\nonumber\\
W_{\beta}(z)&=\sqrt{\pi}2^{\beta}z^{\frac{1-\beta}{2}}\Gamma(\beta+\frac{3}{2}){\bf{H}}_{\beta}(\sqrt{z}).
\end{align}
Moreover, for $|\beta|\leq\frac{1}{2}$, it has the Hadamard factorization given by
\begin{equation}\label{strv-facto}
{\bf{H}}_{\beta}(z)=\frac{z^{\beta+1}}{\sqrt{\pi}2^{\beta}\Gamma(\beta+\frac{3}{2})}\prod_{n\geq1}\left(1-\frac{z^2}{z^2_{\beta,n}}\right),
\end{equation}
where $z_{\beta,n}$ is the $n$-th positive root of ${\bf{H}}_{\beta}$ such that $z_{\beta,n+1}>z_{\beta,n}$ and $z_{\beta,1}>1$ and also from \eqref{strv-facto}, we obtain
\begin{equation}\label{strv-strlike}
\frac{z{\bf{H}}'_{\beta}(z)}{{\bf{H}}_{\beta}(z)}=(\beta+1)-\sum_{n\geq1}\frac{2z^2}{z^2_{\beta,n}-z^2}.
\end{equation}
Now using the above representation, we can obtain the radii of cardioid-starlikeness of the functions $U_{\beta}, V_{\beta}$ and $W_{\beta}$ for the case $|\beta|\leq {1}/{2}$.
\begin{theorem}\label{Struv-thm}
	Let $|\beta|\leq {1}/{2}$. Then the radii of cardioid-starlikeness $r_{\wp}(U_{\beta})$, $r_{\wp}(V_{\beta})$ and $r_{\wp}(W_{\beta})$ of the functions $U_{\beta}$, $V_{\beta}$ and $W_{\beta}$ are the smallest positive roots of the following equations, respectively:
	\begin{enumerate}[(i)]
		\item $r{\bf{H}}'_{\beta}(r)-(1-\frac{1}{e})(\beta+1){\bf{H}}_{\beta}(r)=0;$
		\item $r{\bf{H}}'_{\beta}(r)-((1+\beta)-\frac{1}{e}){\bf{H}}'_{\beta}(r)=0;$
		\item $\sqrt{r}{\bf{H}}'_{\beta}(\sqrt{r})-(1+\beta-\frac{2}{e}){\bf{H}}_{\beta}(\sqrt{r})=0.$
	\end{enumerate}	
\end{theorem}
\begin{proof}
	From \eqref{nor-strv-uvw} and \eqref{strv-strlike}, by logarithmic differentiation, we get 
	\begin{align}\label{strv-main-eq}
	\frac{zU'_{\beta}(z)}{U_{\beta}(z)}&=\frac{1}{\beta+1}\frac{z{\bf{H}}'_{\beta}(z)}{{\bf{H}}_{\beta}(z)}=1-\frac{1}{\beta+1}\sum_{n\geq1}\frac{2z^2}{z^2_{\beta,n}-z^2},\nonumber\\
	\frac{zV'_{\beta}(z)}{V_{\beta}(z)}&=-\beta+\frac{z{\bf{H}}'_{\beta}(z)}{{\bf{H}}_{\beta}(z)}=1-\sum_{n\geq1}\frac{2z^2}{z^2_{\beta,n}-z^2},\nonumber\\
	\frac{zW'_{\beta}(z)}{W_{\beta}(z)}&= \frac{1-\beta}{2}+\frac{\sqrt{z}{\bf{H}}'_{\beta}(\sqrt{z})}{{2\bf{H}}_{\beta}(\sqrt{z})}=1-\sum_{n\geq1}\frac{z}{z^2_{\beta,n}-z}.
	\end{align}	
	Now applying the triangle inequality $||x|-|y||\leq |x-y|$ and Lemma~\ref{disk_lem} in \eqref{strv-main-eq}, we see that  $U_{\beta}, V_{\beta}$ and $W_{\beta}$ belongs to $\mathcal{S}^*_{\wp}$, respectively whenever
	\begin{align}\label{strv-disk-eq}
	\left|\frac{zU'_{\beta}(z)}{U_{\beta}(z)}-1\right| &\leq \frac{1}{\beta+1}\sum_{n\geq1}\frac{2r^2}{z^2_{\beta,n}-r^2}\leq\frac{1}{e},\nonumber\\
	\left|\frac{zV'_{\beta}(z)}{V_{\beta}(z)}-1\right| &\leq \sum_{n\geq1}\frac{2r^2}{z^2_{\beta,n}-r^2}\leq\frac{1}{e},\nonumber\\
	\left|\frac{zW'_{\beta}(z)}{W_{\beta}(z)}-1\right| &\leq \sum_{n\geq1}\frac{r}{z^2_{\beta,n}-r}\leq\frac{1}{e}
	\end{align} 
	holds, where $|z|=r<z_{\beta,1}$. Now to find the largest positive radius for which \eqref{strv-disk-eq} holds. Let us consider the strictly increasing continuous functions
	\begin{align*}
	\Psi_1(r)&:=\frac{1}{\beta+1}\sum_{n\geq1}\frac{2r^2}{z^2_{\beta,n}-r^2}-\frac{1}{e},\\
	\Psi_2(r)&:= \sum_{n\geq1}\frac{2r^2}{z^2_{\beta,n}-r^2}-\frac{1}{e},\\
	\Psi_3(r)&:= \sum_{n\geq1}\frac{r}{z^2_{\beta,n}-r}-\frac{1}{e} .
	\end{align*}
	Since $\lim_{r\rightarrow0}\Psi_{i}(r)<0$, $\Psi'_{i}(r)>0$ for $i=1$ to $3$, $\lim_{r\rightarrow z_{\beta,1}}\Psi_{i}(r)>0$ for $i=1,2$ and  $\lim_{r\rightarrow {z^2_{\beta,1}}}\Psi_{3}(r)>0$, there exist the unique positive roots, $r_{\wp}(U_{\beta}), r_{\wp}(V_{\beta}) \in (0,z_{\beta,1})$ and $r_{\wp}(W_{\beta})\in(0,{z^2_{\beta,1}})$ for $\Psi_{i}$, respectively so that the inequalities in \eqref{strv-disk-eq} holds in $|z|<r_{\wp}(U_{\beta})$, $|z|<r_{\wp}(V_{\beta})$ and $|z|<r_{\wp}(W_{\beta})$, respectively. Further using \eqref{strv-main-eq} in $\Psi_i(r)=0$, respectively, we obtain the desired equations. This completes the proof. \qed
	
\end{proof}	

The Lommel function $\mathcal{L}_{u,v}$ of first kind is a particular solution of the second-order inhomogeneous Bessel differential equation $$z^2w''(z)+zw'(z)+(z^2-{v}^2)w(z)=z^{u+1},$$
where $u\pm v\notin \mathbb{Z}^{-}$ and is given by
$$\mathcal{L}_{u,v}=\frac{z^{u+1}}{(u-v+1)(u+v+1)}{}_1F_2\left(1;\frac{u-v+3}{2},\frac{u+v+3}{2};-\frac{z^2}{4}\right),$$
where $\frac{1}{2}(-u\pm v-3)\notin \mathbb{N}$ and ${}_1 F_{2}$ is a hypergeometric function. Since it is not normalized, so we consider the following three normalized functions involving $\mathcal{L}_{u,v}$ :
\begin{align}\label{fL}
f_{u,v}(z)&=((u-v+1)(u+v+1)\mathcal{L}_{u,v}(z))^{\tfrac{1}{u+1}},\nonumber\\
g_{u,v}(z)&=(u-v+1)(u+v+1)z^{-u}\mathcal{L}_{u,v}(z),\nonumber\\
h_{u,v}(z)&=(u-v+1)(u+v+1)z^{\frac{1-u}{2}}\mathcal{L}_{u,v}(\sqrt{z}).
\end{align}
Authors in \cite{bdoy-2016,abo-2018} proved the radius of starlikeness for the normalized functions expressed in terms of $\mathcal{L}_{u,v}$:
\begin{equation}\label{lomel-normalized}
f_{u-\tfrac{1}{2},\tfrac{1}{2}}(z),\quad g_{u-\tfrac{1}{2},\tfrac{1}{2}}(z) \quad\text{and}\quad h_{u-\tfrac{1}{2},\tfrac{1}{2}}(z),
\end{equation} 
where $0\neq u\in (-1,1)$.
Now we find the radii of cardioid-starlikeness of the functions defined in \eqref{lomel-normalized}. For simplicity, we write these as $f_{u}, g_{u}$ and $h_{u}$, respectively.
\begin{theorem}\label{4}
	Let $0\neq u\in(-1,1)$ and write $\mathcal{L}_{u-\tfrac{1}{2}, \tfrac{1}{2}}(z)=:\mathcal{L}_{u}(z)$. Then the radii of cardioid-starlikeness $r_{\wp}(f_{u})$, $r_{\wp}(g_{u})$ and $r_{\wp}(h_{u})$ of the functions $f_{u}$, $g_{u}$ and $h_{u}$ are the smallest positive roots of the following equations, respectively:
	\begin{enumerate}[(i)]
		\item $\left\{
		\begin{array}
		{ll}
		2er\mathcal{L}'_{u}(r)-(2u+1)(e-1)\mathcal{L}_{u}(r)=0,     & for\quad u\in(-\frac{1}{2},1) \\
		2er\mathcal{L}'_{u}(r)-(2u+1)(e+1)\mathcal{L}_{u}(r)=0, & for\quad u\in(-1,-\frac{1}{2});
		\end{array}
		\right.$
		
		\item $2er\mathcal{L}'_{u}(r)-(2eu+e-2)\mathcal{L}_{u}(r)=0;$
		\item $2e\sqrt{r}\mathcal{L}'_{u}(\sqrt{r})-(2eu+e-4)\mathcal{L}_{u}(\sqrt{r})=0.$
	\end{enumerate}	
\end{theorem}
\begin{proof}
	We prove the first part. Let $0\neq u\in(0,1)$. Then using a result from \cite{lommel-hadmrd} (also see \cite[Lemma 1, p.~3358]{bdoy-2016}), we can write the Lommel function $\mathcal{L}_{u-\tfrac{1}{2},\tfrac{1}{2}}$ as follows:
	\begin{equation}
	\mathcal{L}_{u-\tfrac{1}{2},\tfrac{1}{2}}(z)=\frac{z^{u+\tfrac{1}{2}}}{u(u+1)}{}_1F_2\left(1;\frac{u+2}{2},\frac{u+3}{2};-\frac{z^2}{4}\right)=\frac{z^{u+\tfrac{1}{2}}}{u(u+1)}\phi_{0}(z),
	\end{equation}
	where $$\phi_{0}(z)=\prod_{n\geq1}\left(1-\frac{z^2}{z^2_{u,0,n}}\right),$$
	and $z_{u,0,n}$ is the simple and real $n$-th positive root of $\phi_{0}$. Also $z_{u,0,n}\in (n\pi, (n+1)\pi)$ which ensures $z_{u,0,n}>z_{u,0,1}>\pi>1$. Now with this representation, after logarithmic differentiation, from \eqref{fL} we get
	\begin{equation*}
	\frac{zf'_{u}(z)}{f_{u}(z)}=\frac{z\mathcal{L}'_{u-\tfrac{1}{2},\tfrac{1}{2}}(z)}{(u+\tfrac{1}{2})\mathcal{L}_{u-\tfrac{1}{2},\tfrac{1}{2}}(z)}=1-\frac{1}{u+\tfrac{1}{2}}\sum_{n\geq1}\frac{2z^2}{z^2_{u,0,n}-z^2}.
	\end{equation*}
	Using the triangle inequality and Lemma~\ref{disk_lem}, we have $f_{u}\in \mathcal{S}^*_{\wp}$ provided 
	$$T(r):=\frac{1}{u+\tfrac{1}{2}}\sum_{n\geq1}\frac{2r^2}{z^2_{u,0,n}-r^2}-\frac{1}{e}\leq0$$
	holds for $|z|=r<z_{u,0,1}$, where $T(r)$ is a strictly increasing continuous function in $(0,z_{u,0,1})$. Since $\lim_{r\rightarrow0}T(r)<0$, $\lim_{r\rightarrow z_{u,0,1}}T(r)>0$ and $T'(r)>0$,  there exists a root $r_{\wp}(f_{u})\in(0,z_{u,0,1})$ so that $f_{u}\in \mathcal{S}^*_{\wp}$ in $|z|<r_{\wp}(f_{u})$. Now for the case $u\in (-1,0)$, we proceed as in the case when $u\in(0,1)$, just replacing $u$ by $u+1$ and $\phi_{0}$ by $\phi_{1}$, where
	$$\phi_{1}(z)={}_1F_2\left(1;\frac{u+1}{2},\frac{u+2}{2};-\frac{z^2}{4}\right)=\prod_{n\geq1}\left(1-\frac{z^2}{z^2_{u,1,n}}\right)$$
	and $z_{u,1,n}$ be the $n$-th positive root of $\phi_1$.\\
	Proof for the part (ii) and (iii)  follows in a similar fashion as in the Theorem~\ref{Struv-thm} by applying Lemma~\ref{disk_lem} on the following two equations, respectively together with the triangle inequality $||x|-|y||\leq|x-y|$:
	\begin{equation*}
	\frac{zg'_{u}(z)}{g_{u}(z)}=-u+\frac{1}{2}+\frac{z\mathcal{L}'_{u-\tfrac{1}{2},\tfrac{1}{2}}(z)}{(u+\tfrac{1}{2})\mathcal{L}_{u-\tfrac{1}{2},\tfrac{1}{2}}(z)}=1-\sum_{n\geq1}\frac{2z^2}{z^2_{u,0,n}-z^2}
	\end{equation*}
	and 
	\begin{equation*}
	\frac{zh'_{u}(z)}{h_{u}(z)}=\frac{3-2u}{4}+\frac{\sqrt{z}\mathcal{L}'_{u-\tfrac{1}{2},\tfrac{1}{2}}(\sqrt{z})}{2\mathcal{L}_{u-\tfrac{1}{2},\tfrac{1}{2}}(\sqrt{z})}=1-\sum_{n\geq1}\frac{z}{z^2_{u,0,n}-z},
	\end{equation*}
	where $z_{u,0,n}$ is the $n$-th positive root of the function $\phi_0$. \qed
\end{proof}	

The Legendre polynomials $P_{n}$ are the solutions of the Legendre differential equation:
$$((1-z^2)P'_{n}(z))'+n(n+1)P_{n}(z)=0,$$
where $n\in \mathbb{Z}^{+}$ and using Rodrigues$'$ formula, $P_{n}$ can be represented in the form:
$$P_{n}(z)=\frac{1}{2^n n!}\frac{d^n(z^2-1)^n}{dz^n}$$
and it also satisfies the geometric condition $P_n(-z)=(-1)^n P_{n}(z)$. Moreover, the odd degree Legendre polynomials $P_{2n-1}(z)$  have only real roots which satisfy 
\begin{equation}\label{legdroot}
0=z_0<z_1<\cdots<z_{n-1}\quad\text{and}\quad -z_1<\cdots<-z_{n-1}.
\end{equation}
Thus the normalized form is as follows:
\begin{equation}\label{legd1}
\mathcal{P}_{2n-1}(z):=\frac{P_{2n-1}(z)}{P'_{2n-1}(0)}=z+\sum_{k=2}^{2n-1}a_{k}z^{k}=a_{2n-1}z\prod_{k=1}^{n-1}(z^2-z^2_{k}).
\end{equation} 

\begin{theorem}\label{5}
	The radii of cardioid-starlikeness $r_{\wp}(\mathcal{P}_{2n-1})\in (0,z_1)$ of the normalized odd degree Legendre polynomial is the smallest positive root of the following equation:
	\begin{equation*}
	er\mathcal{P}'_{2n-1}(r)-(e-1)\mathcal{P}_{2n-1}(r)=0.
	\end{equation*}	
\end{theorem}
\begin{proof}
	From \eqref{legd1}, after logarithmic differentiation, we obtain
	\begin{equation}\label{legd2}
	\frac{z\mathcal{P}'_{2n-1}(z)}{\mathcal{P}_{2n-1}(z)}=1-\sum_{k=1}^{n-1}\frac{2z^2}{z^2_k-z^2}.
	\end{equation}
	Now applying Lemma~\ref{disk_lem} on \eqref{legd2}, we have $\mathcal{P}_{2n-1}\in \mathcal{S}^{*}_{\wp}$ whenever
	\begin{equation}\label{legd3}
	\left|
	\frac{z\mathcal{P}'_{2n-1}(z)}{\mathcal{P}_{2n-1}(z)}-1\right|\leq \sum_{k=1}^{n-1}\frac{2r^2}{z^2_k-r^2}\leq\frac{1}{e},
	\end{equation}
	where $|z|=r<z_1$ and $z_k$ satisfies the condition given in \eqref{legdroot}. Now let us consider the strictly increasing continuous function
	$$T(r):=\sum_{k=1}^{n-1}\frac{2r^2}{z^2_k-r^2}-\frac{1}{e}, \quad r\in(0,z_1).$$
	We have to show that $T(r)\leq0$ in $|z|\leq r<z_1$ so that \eqref{legd3} holds. Since $\lim_{r\rightarrow0}T(r)<0$, $\lim_{r\rightarrow z_1}T(r)>0$ and $T'(r)>0$, there exists a unique positive root $r_{\wp}(\mathcal{P}_{2n-1})\in (0,z_1)$ of $T(r)$ such that $\mathcal{P}_{2n-1}\in \mathcal{S}^*_{\wp}$ in $|z|<r_{\wp}(\mathcal{P}_{2n-1})$. \qed
\end{proof}	
\begin{remark}[\bf{Generalization for $\mathcal{S}^*(\psi)$}]\label{S-General}
	Consider the Ma-Minda function $\psi$ as defined in \eqref{mindasclass}. Let $a \in \psi(\mathbb{D}) \cap \mathbb{R}$, $r_a$ is the radius depending on $a$ and assume the maximal disk $ |w-a|<r_a$ such that 
	$$\{w : |w-a|< r_a\} \subset \psi(\mathbb{D}).$$
	Then we can easily extend Theorem~\ref{1} ($\psi(z)\neq (1+z)/(1-z)$), Theorem~\ref{2}, Theorem~\ref{3}, Theorem~\ref{Struv-thm}, Theorem~\ref{4} and  Theorem~\ref{5} for $\mathcal{S}^{*}(\psi)$ using the corresponding radius $r_1$.
\end{remark}
Here we mention the $\mathcal{S}^*(\psi)$-radius for the special functions without proof using Remark~\eqref{S-General}. In particular, we show that there exists an $\alpha\in (0,1]$ such that
$R[\mathcal{S}^{*}(\psi)] \geq R[\mathcal{S}^{*}(1+\alpha z)]$, where we denote $\mathcal{S}^{*}(\psi)$-radius by $R[\mathcal{S}^{*}(\psi)]$.

In the following theorems we mean $R[\mathcal{S}^{*}(\psi)] = R[\mathcal{S}^{*}(1+r_1 z)]$.	
\begin{theorem}[Bessel function $\mathcal{J}_{\beta}$]\label{gbessel}
	Let $\beta>0$. Then the $\mathcal{S}^*(\psi)$-radii $r_{\wp}(f_{\beta})$, $r_{\wp}(g_{\beta})$ and $r_{\wp}(h_{\beta})$ of the functions $f_{\beta}$, $g_{\beta}$ and $h_{\beta}$ are the smallest positive root of the following equations, respectively:
	\begin{enumerate}[(i)]
		\item $r\mathcal{J}'_{\beta}(r)+\beta(1-{r_1})\mathcal{J}_{\beta}(r)=0;$
		\item $r\mathcal{J}'_{\beta}(r)-(\beta-{r_1})\mathcal{J}_{\beta}(r)=0;$
		\item $\sqrt{r}\mathcal{J}'_{\beta}(\sqrt{r})-(\beta-2{r_1})\mathcal{J}_{\beta}(\sqrt{r})=0.$
	\end{enumerate}	
\end{theorem}
\begin{theorem}[Struve function $\mathcal{\bf{H}}_{\beta}$]\label{gstruve}
	Let $|\beta|\leq {1}/{2}$. Then the $\mathcal{S}^*(\psi)$-radii $r_{\wp}(U_{\beta})$, $r_{\wp}(V_{\beta})$ and $r_{\wp}(W_{\beta})$ of the functions $U_{\beta}$, $V_{\beta}$ and $W_{\beta}$ are the smallest positive root of the following equations, respectively:
	\begin{enumerate}[(i)]
		\item $r{\bf{H}}'_{\beta}(r)-(1-{r_1})(\beta+1){\bf{H}}_{\beta}(r)=0;$
		\item $r{\bf{H}}'_{\beta}(r)-((1+\beta)-{r_1}){\bf{H}}'_{\beta}(r)=0;$
		\item $\sqrt{r}{\bf{H}}'_{\beta}(\sqrt{r})-(1+\beta-2{r_1}){\bf{H}}_{\beta}(\sqrt{r})=0.$
	\end{enumerate}
\end{theorem}
\begin{theorem}[Lommel function $\mathcal{L}_{u,v}$]\label{glommel}
	Let $0\neq u\in(-1,1)$ and write $\mathcal{L}_{u-\tfrac{1}{2}, \tfrac{1}{2}}(z)=:\mathcal{L}_{u}(z)$. Then the $\mathcal{S}^*(\psi)$-radii $r_{\wp}(f_{u})$, $r_{\wp}(g_{u})$ and $r_{\wp}(h_{u})$ of the functions $f_{u}$, $g_{u}$ and $h_{u}$ are the smallest positive root of the following equations, respectively:
	\begin{enumerate}[(i)]
		\item $\left\{
		\begin{array}
		{ll}
		2r\mathcal{L}'_{u}(r)-(2u+1)(1-{r_1})\mathcal{L}_{u}(r)=0,     & for\quad u\in(-\frac{1}{2},1) \\
		2r\mathcal{L}'_{u}(r)-(2u+1)(1+{r_1})\mathcal{L}_{u}(r)=0, & for\quad u\in(-1,-\frac{1}{2});
		\end{array}
		\right.$
		
		\item $2r\mathcal{L}'_{u}(r)-(2u+1-2{r_1})\mathcal{L}_{u}(r)=0;$
		\item $2\sqrt{r}\mathcal{L}'_{u}(\sqrt{r})-(2u+1-4{r_1})\mathcal{L}_{u}(\sqrt{r})=0.$
	\end{enumerate}	
\end{theorem}
\begin{theorem}[Legendre polynomials $P_{n}$]\label{glegendre}
	The $\mathcal{S}^*(\psi)$-radius $r_{\wp}(\mathcal{P}_{2n-1})\in (0,z_1)$ of the normalized odd degree Legendre polynomial is the smallest positive root of the following equation:
	\begin{equation*}
	r\mathcal{P}'_{2n-1}(r)-(1-{r_1})\mathcal{P}(r)=0.
	\end{equation*}		
\end{theorem}

\begin{corollary}
	If $\alpha=r_1$ be the radius of the lagest disk $\{w: |w-1|< \alpha\}$ inside $\psi(\mathbb{D})$, where
	\begin{enumerate}[$(i)$]
		\item   	$\alpha=\min\left\{\left|1-\frac{1+A}{1+B}\right|, \left|1-\frac{1-A}{1-B}\right|\right\}=\frac{A-B}{1+|B|}$ when $\psi(z)= \frac{1+Az}{1+Bz}$, where $-1\leq B<A\leq1$;
		
		\item  	$\alpha=\sqrt{2-2\sqrt{2}+\sqrt{-2+2\sqrt{2}}}$ when  $\psi(z)=\sqrt{2}-(\sqrt{2}-1)\sqrt{\frac{1-z}{1+2(\sqrt{2}-1)z}}$;
		
		\item  	$\alpha=\sqrt{2}-1$ when $\psi(z)=\sqrt{1+z}$;
		
		\item  $\alpha=e-1$ when $\psi(z)=e^z$;
		
		\item  $\alpha=2-\sqrt{2}$ when $\psi(z)=z+\sqrt{1+z^2}$;
		
		\item  $\alpha=\frac{e-1}{e+1}$ when $\psi(z)=\frac{2}{1+e^{-z}}$;
		
		\item   $\alpha=\sin{1}$ when $\psi(z)=1+\sin{z}$.
		
	\end{enumerate}
	Then Theorem~\ref{gbessel}, Theorem~\ref{gstruve}, Theorem~\ref{glommel} and Theorem~\ref{glegendre} hold true for the class 
	$\mathcal{S}^{*}(\psi)$.
\end{corollary}

%All equations, theorems, definitions, lemmas, propositions,
%corollaries, examples, remarks etc. would be better to be numbered
%consecutively and unrepeatedly within each section. For example,
%Definition 2.1, Lemma 2.2, Theorem 2.3 \ldots.
%
%Use \verb|\label| and \verb|\ref| or \verb|\eqref| to
%automatically cross-reference sections, equations, theorems and
%theorem-like environments, tables, figures, etc.

%\begin{theorem}[\cite{ref2}]\label{th:2.1} %Theorem 1.1 could be recalled by using Theorem \ref{th:2.1}
%The statements of theorems, lemmas, definitions, propositions,
%corollaries, conjectures, etc. are set in romantype, by using
%\begin{verbatim}
%\begin{theorem/lemma/definition/proposition/corollary/conjecture}
%\end{theorem/lemma/definition/proposition/corollary/conjecture}.
%\end{verbatim}
%\end{theorem}
%
%\begin{prof}  %you can also use the environment \begin{proof}\end{proof}
%Observe that
%\begin{align}\label{E:1.1}
%AAAAAAAAAA &= BBBBBBBBBBB\nonumber \\
%           &\quad + CCCCCCCCCC\nonumber \\
%           &= DDDDDDDDDDDDD.
%\end{align}
%Now, apply induction on $n$ to \eqref{E:1.1}\ldots
%\end{prof}
%
%\begin{remark}\label{re:1.2}
%Remarks, examples, problems, etc. are set in roman type.
%\end{remark}
%
%\subsection{Figure}
%The text should include references to all illustrations. Refer to illustrations in the
%text as Table~1, Table~2, Figure~1, Figure~2, etc., not with the section or chapter number
%included, for example, Table 3.2, Figure 4.3, etc. Do not use the words ``below'' or ``above''
%referring to the tables, figures, etc.
%
%Do not collect illustrations at the back of your article, but incorporate them in the
%text. Position tables and figures at the top or bottom of a page, with at least 2 lines
%extra space between tables or figures and the running text.
%
%Illustrations should be centered on the page, except for small figures that can fit
%side by side inside the type area. Tables and figures should not have text wrapped
%alongside.
%
%Place figure captions \textit{below} the figure, table captions \textit{above} the table.
%Use bold for table/figure labels and numbers, for example: \textbf{Table 1}, \textbf{Figure 2},
%and roman for the text of the caption. Keep table and figure captions justified. Center
%short figure captions only.
%
%The minimum \textit{font size} for characters in tables is 8 points, and for lettering in other
%illustrations, 6 points.
%
%%\centerline{\includegraphics[scale=1.2]{acta.pdf}}
%%\centerline{\small\bf Figure 1\quad Journal mark}


%\subsection{Table}
%\begin{table}
%\textbf{\caption{Aaa bbb ccc\label{tab}}}
%\begin{tabular}{|c|c|c|l|c|}
%\hline $P(x)$ & $i$& $(e(1),e(2),e(4))$ & $(e(3),e(6),e(12),e(24))$ & $T(E)$ \\
%\hline $P_1$  &    & & &$\emptyset$ \\
%\hline $P_2$  & 4  & & $(1,1,1,0)\rightarrow(0,0,0,1)$ &2\\
%\hline $P_3$  & 2  & &$(1,1,1,0)\rightarrow(0,0,2,0)$ &1\\
%\hline $P_4$  & 2  & $(0,1,1)\rightarrow(1,2,0)$ & &1\\
%\hline $P_5$  & 2  & $(0,1,1)\rightarrow(1,2,0)$ &$(1,1,1,0)\rightarrow(0,0,0,1)$ &$1,2$\\
%\hline $P_6$  & 6  & $(0,1,1)\rightarrow(1,2,0)$ &$(1,1,1,0)\rightarrow(2,2,0,0)$ &1\\
%\hline $P_7$  & 3  & $(0,1,1)\rightarrow(1,0,1)$ &$(1,1,1,0)\rightarrow(2,0,1,0)$ &0\\
%\hline $P_8$  & 3  & $(0,1,1)\rightarrow(2,1,0)$ &$(1,1,1,0)\rightarrow(2,0,1,0) \rightarrow(3,1,0,0)$ &$0,1$\\
%\hline
%\end{tabular}
%\end{table}


\subsection{Majorization result for $\mathcal{S}^*_{\wp}$}
\begin{theorem}
	Let $f\in \mathcal{A}$ and $g\in \mathcal{S}^*_{\wp}$. If $f$ is majorized by $g$ in $\mathbb{D}$. Then
	\begin{equation}\label{mj0}
	|f'(z)|\leq|g'(z)| \quad \text{in}\quad |z|\leq r_0,
	\end{equation}
	where $r_0\approx 0.380056$ is the smallest positive root of the  equation $$(1-r^2)(1-re^r)-r=0.$$
\end{theorem}
\begin{proof}
	Let $f\in\mathcal{A}$ and $g\in\mathcal{S}^*_{\wp}$, then we have
	$$\frac{zg'(z)}{g(z)} = 1+\omega(z)e^{\omega(z)}$$
	or
	\begin{equation}\label{mj1}
	\frac{g(z)}{g'(z)}=\frac{z}{ 1+\omega(z)e^{\omega(z)} },
	\end{equation}
	where $\omega$ is a Schwarz function.	Let $\omega(z)=R e^{i\theta}$, where $R\leq r=|z|<1$, then we have
	\begin{equation}
	\left|\frac{g(z)}{g'(z)}\right| = \left|\frac{z}{ 1+\omega(z)e^{\omega(z)}} \right| \leq \frac{r}{1-re^r}.
	\end{equation}
	Now by definition of majorization, we have $f(z)=\Phi(z) g(z)$, where $\Phi(z)$ is analytic function satisfying $|\Phi(z)|\leq 1$ in $\mathbb{D}$ such that
	$$f'(z)=g'(z)\left(\Phi'(z)\frac{g(z)}{g'(z)}+ \Phi(z) \right).$$
	Thus we have
	\begin{equation}\label{mj2}
	|f'(z)|\leq |g'(z)| \left(\frac{1-|\Phi(z)|^2}{1-r^2} \frac{r}{1-re^r}+ |\Phi(z)| \right)=|g'(z)| h(r,\beta),
	\end{equation}
	where $$h(r,\beta)=\frac{(1-\beta^2)r}{(1-r^2)(1-re^r)}+\beta.$$
	Now to arrive at equation \eqref{mj0}, it suffices to find $r_0$ such that $h(r,\beta)\leq 1$ holds.  Or equivalently,
	$$0\leq (1-r^2)(1-re^r)-(1+\beta)r=: k(r,\beta).$$
	Since $\frac{\partial}{\partial\beta}k(r,\beta)=-r$, therefore $\min_\beta k(r,\beta)=k(r,1)$. Write $k(r,1)$ as $k(r)$ and note that $k(0)=1$ and $k(1)=-1$. So by continuity of $k(r)$, there exists a point $r_0$ such that $k(r)\geq0$ for $0\leq r\leq r_0$, where $r_0$ is the smallest positive root of the equation
	$$(1-r^2)(1-re^r)-r=0.$$ 
	This ends the proof. \qed
\end{proof}

\subsection{Convolution radius}
First, we need to recall the following result due to Ruscheweyh and Sheil-Small:
\begin{lemma}[\cite{rush-sheil-1973}, p.~126]\label{l1}
	Suppose that either	$g\in \mathcal{C}$, $h\in \mathcal{S}^*$ or else $g,h\in\mathcal{S}^*_{1/2}$. Then for any analytic function $G$ in $\mathbb{D}$, we have
	$$\frac{g*hG(z)}{g*h(z)}\in\overline{co}G(\mathbb{D}),$$
	where $\overline{co}G(\mathbb{D})$ is the closed convex hull of $G(\mathbb{D})$.
\end{lemma}
Observe that if we consider the function $\phi\in\mathcal{P}$, which is starlike but not convex then we can not directly apply Theorem \ref{c1} as such function do exist, for instance $\phi(z)=1+ze^z$. But keenly observing the proof of Lemma \ref{l1}, we see that the unit disk $\mathbb{D}$ can be replaced by the sub-disk $\mathbb{D}_{r} :=\{z: |z|<r \}$, where $0<r\leq1$ and consequently, we obtain the following modified result. Since the proof is similar, so it is omitted here.
\begin{lemma}\label{l11}
	Suppose either	$g\in \mathcal{C}$, $h\in \mathcal{S}^*$ or else $g,h\in\mathcal{S}^*_{1/2}$. Then for any analytic function $G$ in $\mathbb{D}_{r}$, we have
	$({g*hG(z)})/({g*h(z)})\in\overline{co}G(\mathbb{D}_{r})$, where $r\in[0,1]$.
\end{lemma}
This immediately gives 
\begin{theorem}\label{c11}
	Let $r_0$ be the radius of convexity of $\psi$. If $g\in \mathcal{C}$ and $f\in\mathcal{S}^*(\psi)$. Then
	$f*g\in \mathcal{S}^*(\psi)$
	in $|z|<r$, where $r=\min\{r_0,1\}$.
\end{theorem}
\begin{corollary}
	Let $f\in\mathcal{S}^{*}_{\wp}$ and $g\in \mathcal{C}$. Then $f*g\in\mathcal{S}^{*}_{\wp}$ in $\mathbb{D}_{r_0}$, where $r_0=(3-\sqrt{5})/2$ is the radius of convexity of $\wp$.
\end{corollary}	
Now consider the operators $\mathcal{F}_i :\mathcal{A}\rightarrow \mathcal{A}$ defined by
$$\mathcal{F}_1(f)(z)=f*g_1(z)=zf'(z)$$
$$\mathcal{F}_2(f)(z)=f*g_2(z)=\frac{1}{2}(f(z)+zf'(z))$$
\begin{equation*}\label{operators}
\mathcal{F}_3(f)(z)=f*g_3(z)=\frac{k+1}{z^k}\int_{0}^{z}t^{k-1}f(t)dt,\quad \Re{k}>0,
\end{equation*}
where 
$g_3(z)=\sum_{n=1}^{\infty}{(k+1)}/{(k+n)}z^n$, $g_2(z)=(z-z^2/2)/(1-z^2)^2$ and $g_1(z)=z/(1-z)^2$. Note that the function $g_1$ is convex in $|z|<2-\sqrt{3}$, $g_2$ is convex in $|z|<1/2$ while $g_3\in \mathcal{C}$.
The above defined operators were introduced by Alexander, Livingston and Bernardi, respectively. Now we obtain the following result, where $\mathcal{S}^*_{SG}:=\mathcal{S}^*(\frac{2}{e^{-z}+1})$, $\mathcal{S}^*_{C}:=\mathcal{S}^*(1+4z/3+2z^2/3)$ and $\mathcal{S}^*_{s}:=\mathcal{S}^*(1+\sin{z})$ :

\begin{corollary} 
	Let $\mathcal{F}_i$, $i=1$ to $3$ be the operators as defined above.
	\begin{itemize}
		\item [$(i)$] Let $f\in \mathcal{S}^*_{\wp}$. Then $\mathcal{F}_i(f) \in \mathcal{S}^*_{\wp}$ in $\mathbb{D}_{r_i}$, where
		$r_1=2-\sqrt{3}$, $r_2=(3-\sqrt{5})/2$ and $r_3=(3-\sqrt{5})/2$.
		\item [$(ii)$] Let $f\in  \mathcal{S}^*_{C}$. Then $\mathcal{F}_i(f) \in \mathcal{S}^*_{C}$ in $\mathbb{D}_{r_i}$, where
		$r_1=2-\sqrt{3}$, $r_2=1/2$ and $r_3=1/2$. 
		\item [$(iii)$] Let $f\in \mathcal{S}^*_{s}$. Then $\mathcal{F}_i(f) \in \mathcal{S}^*_{s}$ in $\mathbb{D}_{r_i}$, where
		$r_1=2-\sqrt{3}$, $r_2=0.345$ and $r_3=0.345$.
		\item [$(iv)$] Let $f \in \mathcal{S}^*_{SG}$. Then $\mathcal{F}_i(f) \in \mathcal{S}^*_{SG}$ in $\mathbb{D}_{r_i}$, where
		$r_1=2-\sqrt{3}$, $r_2=1/2$ and $r_3=1$. 
	\end{itemize}
\end{corollary}

In 2010, Ali et al.~\cite{con2010} dealt with the problem of finding $\mathcal{S}^*(\psi)$-radii of the convolution $f*g$, between two starlike functions. In particular, they showed that if $f,g\in \mathcal{S}^*$ and $h_{\rho}(z)=f*g(\rho z)/\rho$, then $h_{\rho}\in\mathcal{SL}^*$ for $0\leq\rho\leq(\sqrt{5}-2)/(\sqrt{2}-1)\approx0.09778$. They used the property of the function $\psi$ being convex. Now using Theorem \ref{c11}, we can obtain the result even for the case when $\psi(\mathbb{D})$ is starlike. Here we have shown the usability of radius of convexity of $\psi$. 
\begin{theorem}\label{con-star}
	Let $f,g\in \mathcal{S}^*$ and $h_{\rho}(z):=f*g(\rho z)/\rho$. Then 
	\begin{itemize}
		\item [$(i)$]	$h_{\rho}\in \mathcal{S}^*_{\wp}$ for $0\leq\rho\leq (2e-\sqrt{4e^2-2e+1})/(2e-1)\approx0.0957$,
		
		\item [$(ii)$] $h_{\rho}\in \mathcal{S}^*_{C}$ for  $0\leq \rho \leq(3-\sqrt{7})/2\approx0.177124$,
		
		\item [$(iii)$] $h_{\rho}\in \mathcal{S}^*_{s}$ for $0\leq\rho\leq(\sqrt{{\sin1}^2+2\sin1+4}-2)/(2+\sin1)\approx0.185835$,
		
		\item [$(iv)$] $h_{\rho}\in\mathcal{S}^*_{b}$ for $0\leq\rho\leq(2e-\sqrt{3e^2+e^{2/e}})/(e+e^{1/e})\approx0.122919$,
		
		\item [$(v)$] $h_{\rho}\in\mathcal{S}^*_{SG}$ for $0\leq\rho\leq(\sqrt{7e^2+6e+3}-2(1+e))/(3e+1)\approx0.108309$.
		
		%	\item [$(vi)$] $h_{\rho}\in\mathcal{M}(\beta)$ for $0\leq\rho\leq(\sqrt{3+{\beta}^2}-2)/(1+\beta)$.
	\end{itemize}
	The constants are best possible.
\end{theorem} 
\begin{proof}
	\begin{itemize}\label{disk}
		\item [$(i)$] Let $H(z)=z+\sum_{n=2}^{\infty}n^2z^n=(z(1+z))/(1-z)^3$. It is easy to see that 
		\begin{equation}\label{thm-disk}
		\biggl|\frac{zH'(z)}{H(z)} -\frac{1+r^2}{1-r^2}\biggl| \leq \frac{4r}{1-r^2}, \quad |z|=r<1. 
		\end{equation}
		Now by Lemma \ref{disk_lem}, the disk \eqref{thm-disk}  lies inside the cardioid $\wp(\mathbb{D})$, provided 
		$$\frac{4r}{1-r^2} \leq \frac{1+r^2}{1-r^2}-1+\frac{1}{e} $$  
		which in turn gives $ r\leq r_0:= (2e-\sqrt{4e^2-2e+1})/(2e-1).$  Define the function $h:\mathbb{D} \rightarrow \mathbb{C}$  by $h(z):=f(z)*g(z)$. Then $h(z)=F(z)*G(z)*H(z)$, where $F$ and $G$ are, respectively defined as $zF'(z)=f(z)$  and $zG'(z)=g(z)$. Since $f, g\in \mathcal{S}^*$, it follows that $F*G\in \mathcal{C}$. Also, $H(r_0 z)/r_0 \in \mathcal{S}^*_{\wp}$. Hence, using Theorem \ref{c11}, we have
		$${F(z)*G(z)*H(\rho_0)}/{\rho_0} \in \mathcal{S}^*_{\wp},$$
		where  $\rho_0=\min\{r_0, r_c\}=r_0$ and $r_c=(3-\sqrt{5})/2$  is the radius of convexity of $\wp$. For $z=-\rho_0$, $zH'(z)/H(z)=(1+4z+z^2)/(1-z^2)=1-1/e$, which implies that $\rho_0$ is sharp. 
		
		%	\item [$(ii)$] Let $\rho_0=(3-\sqrt{7})/2$. Now by Lemma(--) the disk \eqref{thm-disk} is contained inside $\phi(\mathbb{D})$, where $\phi(z)=1+4z/3+2z^2/3$, provided
		%	$$\frac{4r}{1-r^2} \leq \frac{1+r^2}{1-r^2}-\frac{1}{3} $$
		%	which is equivalent to $2r^2-6r+1\geq0$ and gives $r\leq \rho_0$. Since the radius of convexity, $r_c$, for $1+4z/3+2z^2/3$ is greater than $\rho_0$. Therefore, following the same argument as in part(i),the function
		%	$$h(\rho_0 z)/\rho_0=F(z)*G(z)*H(\rho_0 z)/\rho_0\in\mathcal{S}^*_{C}.$$
		%	For $z=-\rho_0$, $zH'(z)/H(z)=(1+4z+z^2)/(1-z^2)=1/3$ which implies that the number $\rho_0$ is sharp.
		%	
		%	\item [$(iii)$] Using Lemma(--), the disk \eqref{thm-disk} lies inside $\phi(\mathbb{D})$, where $\phi(z)=1+\sin(z)$ provided
		%	$$\frac{4r}{1-r^2}\leq 1+\sin1+\frac{1+r^2}{1-r^2}$$
		%	which is equivalent to $(2+\sin1)r^2+4r-\sin1\leq0$ and gives $r\leq r_0=(-2+\sqrt{ {\sin1}^2 + 2\sin1+4}) /(2+\sin1)$, where $r_0$ is the smallest positive root of the following equation $$(2+\sin1)r^2+4r-\sin1 =0.$$ Since radius of convexity of the $1+\sin(z)$ is greater than $r_0$. Therefore, following the same argument as in part(i), it follows that 
		%	$$h(r_0 z)/r_0=F(z)*G(z)*H(r_0 z)/r_0\in\mathcal{S}^*_{s}.$$
		%	For $z=r_0$, $zH'(z)/H(z)=(1+4z+z^2)/(1-z^2)=1+\sin1$ which implies that $r_0$ is sharp.
		%	
		%	\item [$(iv)$] Using Lemma(--), the disk \eqref{thm-disk} is contained inside $\phi(\mathbb{D})$, where $\phi(z)=e^{e^z-1}$ whenever
		%	$$\frac{4er}{1-r^2}\leq\frac{e(1+r^2)-(1-r^2)e^{1/e}}{1-r^2}$$ 
		%	which is equivalent to $(e+e^{1/e})r^2-4er+(e-e^{1/e})\geq0$ and gives $r\leq r_0=(2e-\sqrt{3e^2+e^{2/e}})/(e+e^{1/e})$, where $r_0$ is the 
		%	least positive root of the following equation $$(e+e^{1/e})r^2-4er+(e-e^{1/e}) =0.$$ Since the radius of convexity $r_c$, for the function $e^{e^z-1}$ is greater than $r_0$. Therefore, following the same argument as in part(i), it follows that 
		%	$$h(r_0 z)/r_0=F(z)*G(z)*H(r_0 z)/r_0\in\mathcal{S}^*_{b}.$$
		%	At $z=-r_0$, we have $zH'(z)/H(z)=(1+4z+z^2)/(1-z^2)=e^{-1+1/e}$. Thus the number $r_0$ is sharp.
		%	
		%	\item [$(v)$] Using Lemma(--), the disk \eqref{thm-disk} ls contained inside $\phi(z)=2/(1+e^{-z})$, whenever
		%	$$\frac{4r}{1-r^2}\leq \frac{2e}{1+e}-\frac{1+r^2}{1-r^2}$$
		%	which is equivalent to $(3e+1)r^2+4(1+e)r-(e-1)\leq0$ and it follows that $r\leq r_0=(\sqrt{7e^2+6e+3}-2(1+e))/(3e+1)$, where $r_0$ is the smallest positive root of the equation $$(3e+1)r^2+4(1+e)r-(e-1)=0.$$ Since $\phi$ is convex function. Therefore, for $r_c=1$ together with Lemma \ref{L1} and Theorem \ref{c1} and following the same argument as in part(i), we see that
		%	$h_\rho \in \mathcal{S}^*_{SG}$ for $0\leq \rho\leq \rho_0,$
		%	where $\rho_0=\min\{r_0,r_c\}=r_0$. Also at $z=\rho_0$, $zH'(z)/H(z)=(1+4z+z^2)/(1-z^2)=2e/(1+e)$, which implies that $\rho_0$ is sharp.
		%	
		%	\item [$(vi)$] The disk \eqref{thm-disk} lies inside the region $\{w\in \mathbb{C}: 0< \Re w<\beta,\beta>1\}$, whenever $$(r^2+4r+1)/(1-r^2)\leq \beta$$
		%    which in turn gives $r\leq r_0=(\sqrt{3+{\beta}^2}-2)/(1+\beta)$, where $r_0$ is the least positive root of the following equation $$(1+\beta)r^2+4r-(\beta-1)=0.$$ since the class $\mathcal{M}(\beta)$ is closed under convolution with convex functions. Therefore, following the same argument as in part(i), it follows that $h_\rho\in \mathcal{M}(\beta)$ for $0\leq\rho\leq r_0$. Since at $z=r_0$, $\Re{zH'(z)/H(z)}=\Re{(1+4r_0+{r_0}^2)/(1-{r_0}^2)}=\beta$. Therefore, the number $r_0$ is sharp. 
	\end{itemize}
	Rest part's proof also follow in a similar fashion. \qed
\end{proof}


\begin{theorem}
	Let $f,g\in \mathcal{S}^*$ and $h_{\rho}(z):=f*g(\rho z)/\rho$. Then $h_{\rho}(z)\in \mathcal{S}^{*} \left(\frac{1+Az}{1+Bz} \right)$ for
	$$0\leq\rho\leq \frac{2(B^2-1)+\sqrt{4(1-B^2)^2+(A-B)^2}}{A-B}=:\rho_0,$$
	where $-1<B<A\leq1$.
\end{theorem}
\begin{proof}
	Since for the function $p(z)\prec {(1+Az)}/{(1+Bz)}$, we have 
	\begin{equation}\label{j-disk}
	\left|p(z)-\frac{1-AB}{1-B} \right|\leq \frac{A-B}{1-B^2}.
	\end{equation}
	Therefore, for the disk \eqref{thm-disk} to lie inside the disk \eqref{j-disk}, we must have
	$$\frac{1-AB}{1-B^2}- \frac{A-B}{1-B^2} \leq \frac{1+r^2}{1-r^2} \leq \frac{1-AB}{1-B^2}+\frac{A-B}{1-B^2}$$
	and
	$$\frac{4r}{1-r^2}\leq \frac{A-B}{1-B^2}$$
	which upon simplification hold for $r\leq r_0= \sqrt{(A-B)/(2+A+B)}$ and $r\leq\rho_0$ respectively, where $\rho_0$ is the smallest positive root of the following equation
	$$(A-B)r^2+4(1-B^2)r-(A-B)=0.$$
	Since $\min\{r_0,\rho_0\}=\rho_0$ and the class $\mathcal{S}^{*}\left(\frac{1+Az}{1+Bz}\right)$ is closed under convolution with convex functions, now the result follows in a similar way as in the part $(i)$ of Theorem~\ref{con-star}. \qed
\end{proof}





%The text output area is automatically set within an area 145mm
%horizontally and 218mm vertically. Please do not use any
%\LaTeX{} or \TeX{} commands that affect the layout or formatting of
%your document (that is, commands like \verb|\textheight|,
%\verb|\textwidth|, etc.).

\section{Extremal problem for the class $\mathcal{S}^*(\psi)$}
In 1961, Goluzin \cite{golu} obtained the set of extremal functions $f(z)=z/(1-xz)^2$, $|x|=1$ for the problem of maximization of the quantity 
$\Re\Phi\left(\log({f(z)}/{z})\right)$ or $\left|\Phi\left(\log({f(z)}/{z})\right)\right|$
over the class $\mathcal{S}^*$, where $\phi$ is a non-constant entire function. In 1973, MacGregor \cite{T.H.Mac1973} proved the result for the class $\mathcal{S}^*(\alpha):=\{f\in\mathcal{A}: \Re(zf'(z)/f(z))>\alpha, \alpha\in[0,1)\}$. We observed that the proof given by MacGregor can be generalized to the Ma-Minda class $\mathcal{S}^*(\psi)$. Thus we have the following result:
\begin{theorem}\label{mc1973}
	Suppose $\Phi$ is a non-constant entire function and $0<|z_0|<1$ and assume that the class $\mathcal{S}^{*}(\psi)$ is closed. Then maximum of either
	\begin{equation}\label{functinal1-2}
	\Re\Phi\left(\log\frac{f(z_0)}{z_0}\right) \quad\text{or}\quad 	\left|\Phi\left(\log\frac{f(z_0)}{z_0}\right)\right|
	\end{equation}
	for functions in the class $\mathcal{S}^{*}(\psi)$ is attained only when the function is of the form 
	\begin{equation}
	f(z)=z\exp\int_{0}^{\zeta z}\frac{\psi(t)-1}{t}dt,
	\end{equation}
	where $|\zeta|=1.$
\end{theorem}
\begin{proof}
	Since the class $\mathcal{S}^{*}(\psi)$ is compact, therefore the problem under consideration has a solution. Moreover, in view of a result of Goluzin \cite{golu}, in \eqref{functinal1-2} it suffices to consider the continuous functional 
	\begin{equation*}
	\Re\Phi\left(\log\frac{f(z_0)}{z_0}\right).
	\end{equation*} 
	Let $f\in \mathcal{S}^{*}(\psi)$. Then using a result from \cite{minda94}, ${f(z)}/{z}\prec {f_0(z)}/{z}=:F(z),$
	%\begin{equation*}
	%\frac{f(z)}{z}\prec \frac{f_0(z)}{z}=F(z),
	%\end{equation*}
	where $	f_0(z)=z\exp\int_{0}^{z}\frac{\psi(t)-1}{t}dt$ or equivalenlty $\log(f(z)/z) \prec \log F(z)$. Thus, 
	\begin{equation*}
	g(z)=\Phi\left(\log\frac{f(z)}{z}\right) \prec \Phi(\log F(z))=G(z).
	\end{equation*}
	Note that $G$ is also non-constant as is $\Phi$. Thus for each $r\in (0,1)$ by subordination principle, we obtain $g({\mathbb{\overline{D}}_r}) \subset 	G({\mathbb{\overline{D}}_r})=\Omega.$
	%\begin{equation*}
	%g({\mathbb{\overline{D}}_r}) \subset 	G({\mathbb{\overline{D}}_r})=\Omega.
	%\end{equation*}
	Since $G(xz)\prec G(z)$ for $|x|\leq1$ is obvious, therefore for $|z_0|=r$, we have
	$\{g(z_0): g\prec G \;\text{in}\; \mathbb{D} \} = \Omega.$
	%	Here we have collected all points $w=g(z_0)$, which comprises $\Omega$.
	Now by considering a support line to the compact set $\Omega$, we conclude that
	\begin{equation*}
	\max_{f\in \mathcal{S}^{*}(\psi)} 
	\Re\Phi\left(\log\frac{f(z_0)}{z_0}\right)=\Re{w_1},\quad w_1\in\partial{\Omega}.
	\end{equation*}
	Since $G$ is also an open map, therefore there exists a point $z_1$ where $|z_1|=r$ and $G(z_1)=w_1$ such that among finitely many $w_1$, for one suitable $w_1$, we have
	\begin{equation*}
	\Phi\left(\log\frac{f(z_0)}{z_0}\right)=w_1,
	\end{equation*} 
	where $f$ is the solution for the extremal problem. Now by the well known Lindel$\ddot{o}$f Principle, we have 
	\begin{equation}\label{e3}
	\Phi\left(\log\frac{f(z)}{z}\right)= \Phi(\log{F(xz)}),
	\end{equation}
	that is, if $f$ is the desired solution, then \eqref{e3} holds for some $x$, $|x|=1$. Since $\Phi$ is non-constant analytic function, so we may write
	$$\Phi(w)=c_0+c_nw^n+c_{n+1}w^{n+1}+\cdots;\; c_n\neq0.$$
	If we set $\log(f(z)/z)=\alpha_1z+\alpha_2z^2+\cdots$ and $\log(F(z))=\beta_1z+\beta_2z^2+\cdots,$ then from \eqref{e3}, comparing the coefficients, we get $c_n\alpha^n_1=c_n\beta^n_1.$ Or equivalently, $\alpha^n_1=\beta^n_1$, which in particular implies that $|\alpha_1|=|\beta_1|$. Since $\log(f(z)/z)\prec \log{F(xz)}$, $|\alpha_1|=|\beta_1|$ is possible only if $\log(f(z)/z)= \log{F(xyz)}$ for some $|y|=1$. Therefore, we conclude that 
	$$f(z)=z\exp\int_{0}^{uz}\frac{\psi(t)-1}{t}dt,$$
	where $|u|=1$ if $f$ is a solution to the extremal problem. \qed
\end{proof}
Now as an application of the Theorem \ref{mc1973}, we obtain a result due to MacGregor \cite{T.H.Mac1973}:
\begin{corollary}\cite{T.H.Mac1973}
	Suppose $\Phi$ is a non-constant entire function and $0<|z_0|<1$. Then the maximum of the expression \eqref{functinal1-2} for functions in the class $\mathcal{S}^{*}(\alpha)$ is attained only when the function is of the form 
	\begin{equation*}
	f(z)=\frac{z}{(1-\zeta z)^{2-2\alpha}},\; |\zeta|=1.
	\end{equation*}
\end{corollary}
\begin{proof}
	If $f\in \mathcal{S}^*(\alpha)$, then $f(z)/z \prec 1/(1-z)^{2-2\alpha}$ and hence the result. 
\end{proof}

\begin{corollary}
	Suppose $\Phi$ is a non-constant entire function and $0<|z_0|<1$. Then the maximum of the expression \eqref{functinal1-2} for functions in the class $\mathcal{S}^{*}_{\wp}$ is attained only when the function is of the form 
	\begin{equation*}
	f(z)=z\exp(e^{\zeta z}-1),\; |\zeta|=1.
	\end{equation*}
\end{corollary}
\begin{proof}
	If $f\in \mathcal{S}^*_{\wp}$, then $f(z)/z \prec \exp{(e^z-1)}$ and hence the result. \qed
\end{proof}
%\begin{figure}[h]
%	\begin{tabular}{c}
%		\includegraphics[scale=0.5]{variabilty-region.pdf}
%	\end{tabular}
%	\caption{Shapes of variability region of $\log\frac{f(z)}{z}$ for the classes $\mathcal{S}^*_{\wp}$(blue) and $\mathcal{S}^*$(orange).}
%\end{figure}




%The font type for running text (body text) is 10~point Times New Roman.
%There is no need to code normal type (roman text). For literal text, please use
%\texttt{type\-writer} (\verb|\texttt{}|)
%or \textsf{sans serif} (\verb|\textsf{}|). \emph{Italic} (\verb|\emph{}|)
%or \textbf{boldface} (\verb|\textbf{}|) should be used for emphasis.

\section{Some sufficient conditions for $\mathcal{S}^*_{\wp}$}
In this section, we determine the sufficient conditions for the functions $z/(1+\sum_{k=1}^{\infty}a_kz^k)$, $z/(1+z^n)^k$ and certain other types of functions to be in $\mathcal{S}^*_{\wp}$. 
\begin{theorem}
	Let $f(z)={z}/({1+\sum_{k=1}^{\infty}a_kz^k})$. If the coefficients of $f$ satisfy
	$$|1-a|+\sum_{k=1}^{\infty}(R_a+|1-a-k|)|a_k|\leq R_a,$$
	where $a$ and $R_a$ is as defined in Lemma \ref{disk_lem}. Then $f\in \mathcal{S}^*_{\wp}$.
\end{theorem}
\begin{proof}
	For $f(z)={z}/({1+\sum_{k=1}^{\infty}a_kz^k})$, we have
	$$\left|\frac{zf'(z)}{f(z)}-a\right|=\left|1-a-\frac{\sum_{k=1}^{\infty}ka_kz^k}{1+\sum_{k=1}^{\infty}a_kz^k}\right|.$$
	Thus by Lemma \ref{disk_lem}, $f\in \mathcal{S}^*_{\wp}$, if
	$$\left|1-a-\frac{\sum_{k=1}^{\infty}ka_kz^k}{1+\sum_{k=1}^{\infty}a_kz^k}\right|\leq R_a.$$
	The above inequality holds whenever
	$$|1-a|+\sum_{k=1}^{\infty}|1-a-k||a_k|r^k\leq R_a(1-\sum_{k=1}^{\infty}|a_k|r^k)$$
	or equivalently,
	$$|1-a|+\sum_{k=1}^{\infty}(|1-a-k|+R_a)|a_k|r^k\leq R_a.$$
	Letting $r$ tends to $1^{-}$, completes the proof. \qed
\end{proof}

\begin{theorem}
	Let $f(z)=z/(1+z^k)^n$, where $n,k\in \mathbb{Z}^{+}$ are fixed. Then $f\in \mathcal{S}^*_{\wp}$ for
	$$|z|<\left(\frac{R_a-|1-a|}{R_a+|1-a-kn|}\right)^{1/k},$$
	where $a$ and $R_a$ is as defined in Lemma \ref{disk_lem}.
\end{theorem}
\begin{proof}
	For $f(z)=z/(1+z^k)^n$, we have
	$$\left|\frac{zf'(z)}{f(z)}-a\right|=\left|1-a\frac{knz^k}{1+z^k}\right|.$$
	Thus by Lemma \ref{disk_lem}, $f\in \mathcal{S}^*_{\wp}$, if
	$$\left|1-a-\frac{knz^k}{1+z^k}\right|<R_a.$$ 
	The above inequality holds whenever
	$$|1-a|+|1-a-kn||z|^k<R_a(1-|z|^k)$$
	which simplifies to
	$$|z|^k<\frac{R_a-|1-a|}{R_a+|1-a-kn|}.$$
	Hence the result. \qed
\end{proof}

\begin{theorem}
	Let $p(z)$ be a polynomial such that $p(0)=1$ and $\deg p(z)=m$. Let $R=\min\{|z| : p(z)=0, z\neq0\}$. Then the function $f(z)=z(p(z))^{\beta/m} \in \mathcal{S}^*_{\wp}$ for
	$$|z|<\frac{R(R_a-|1-a|)}{|\beta|+R_a-|1-a|},$$ 
	where $a$ and $R_a$ is as defined in Lemma \ref{disk_lem}.
\end{theorem}
\begin{proof}
	Assume that $z_k$, $(k=1,2,..., m)$ are zeros of the polynomial $p(z)$. For the function $f(z)=z(p(z))^{\beta/m}$, we have
	$$\frac{zf'(z)}{f(z)}=1+\frac{\beta}{m}\sum_{k=1}^{\infty}\frac{z}{z-z_k}$$
	or equivalently,
	$$\frac{zf'(z)}{f(z)}-a=1-a+\frac{\beta}{m}\sum_{k=1}^{\infty}\left(\frac{z}{z-z_k}+\frac{r^2}{R^2-r^2}-\frac{r^2}{R^2-r^2}\right).$$
	Thus by Lemma \ref{disk_lem}, $f\in \mathcal{S}^*_{\wp}$, if
	$$(|1-a|-R_a)(R^2-r^2)+|\beta|(Rr+r^2)<0.$$
	The above inequality is satisfied if $|z|=r<{R(R_a-|1-a|)}/{(|\beta|+R_a-|1-a|)}$. This completes the proof. \qed
\end{proof}



%Use the standard \LaTeXe{} commands for headings: {\small \verb|\section|, \verb|\subsection|, \verb|\subsubsection|, \verb|\paragraph|}.
%Headings will be automatically numbered.
%
%Use initial capitals for the heading in  \small \verb|\section|, except for articles (a, an, the), coordinate
%conjunctions (and, or, nor), and prepositions, unless they appear at the beginning
%of the heading.

\section{Subordination results for $\mathcal{S}^*_{\wp}$}
To prove our differential subordination results, we will need the following lemma due to Miller and Mocanu:
\begin{lemma}\cite{sub-pg132} \label{L1}
	Let $q$ be the univalent in $\mathbb{D}$. Let $\theta$ and $\phi$ be analytic 
	in a domain $D$ containing $q(\mathbb{D})$, with $\phi(w)\neq0$, when $w\in q(\mathbb{D})$. Set $Q(z)=zq'(z)\phi(q(z))$, $h(z)=\theta(q(z))+Q(z)$ and suppose that $Q$ is starlike. In addition, assume that
	\begin{equation*}
	\Re\frac{zh'(z)}{Q(z)}=\Re\biggl(\frac{\theta'(q(z))}{\phi(q(z))}+\frac{zQ'(z)}{Q(z)}\biggl)>0.
	\end{equation*}
	If $p$ is analytic in $\mathbb{D}$, with $p(0)=q(0)$, $p(\mathbb{D})\subset D$ and 
	\begin{equation}
	\theta(p(z))+zp'(z)\phi(p(z)) \prec h(z),
	\end{equation}
	then $p\prec q$ and $q$ is the best dominant.
\end{lemma}

\begin{theorem}
	Let $0<\alpha<1$, $0<B<A<1$ and $k=1+\sqrt{2}$. If $p\in \mathcal{A}_{0}$ and satisfies the differential subordination
	\begin{equation*}
	1+\beta zp'(z) \prec \wp(z),  
	\end{equation*}  
	then
	\begin{itemize}
		\item [$(i)$] $p(z) \prec \phi_0(z):=1+\frac{z}{k}\left(\frac{k+z}{k-z}\right)$ for $\beta\geq \frac{k(e-1)(k+1)}{e(k-1)}\approx 3.68427$.
		\item [$(ii)$] $p(z)\prec \sqrt{1+z}$   for $\beta\geq \frac{e-1}{\sqrt{2}-1}$.
		\item [$(iii)$] $p(z)\prec \frac{1+Az}{1+Bz}$   for $\beta\geq \max\left\{\frac{(1-1/e)(1-B)}{A-B}, \frac{(e-1)(1+B)}{A-B}\right\}$.
		\item [$(iv)$] $p(z)\prec 1+\sin{z}$   for $\beta\geq \frac{e-1}{\sin{1}}$.
		\item [$(v)$] $p(z)\prec z+\sqrt{1+z^2}$   for $\beta\geq\frac{e-1}{\sqrt{2}}$.
		\item [$(vi)$] $p(z)\prec e^z$   for $\beta\geq1$.
	\end{itemize}
	The bounds are sharp.
\end{theorem}
\begin{proof}
	We set $\theta(w)=1$ and $\phi(w)=\beta\neq0$. Let $\Psi_\beta(z, p(z)):= 1+\beta zp'(z)$.
	\begin{itemize}
		\item [$(i)$] The differential equation $\Psi_\beta(z, p(z))=\wp(z)$ has an analytic solution given by
		\begin{equation*}
		q_{\beta}(z)=1+\frac{1}{\beta}(e^z-1).
		\end{equation*}
		Since the function 
		$Q(z)=zq'_{\beta}(z)\phi(q_{\beta}(z))=\wp(z)-1=ze^z\in\mathcal{S}^{*}$ and the function	$h(z)=\theta(q_{\beta}(z))+Q(z)$ satisfies  
		$$\Re\biggl(\frac{zh'(z)}{Q(z)}\biggl)=\Re\biggl(\frac{zQ'(z)}{Q(z)}\biggl)>0.$$ 
		Therefore, by Lemma \ref{L1}, the following differential subordination implication holds:
		$$\Psi_\beta(z, p(z))\prec 1+\beta zq'_{\beta}(z)\Rightarrow p\prec q_{\beta}.$$
		Now to show that $p\prec \phi_0$, we need to show that $q_{\beta}\prec\phi_0$ and for $q_{\beta}\prec\phi_0$ it is necessary that $\phi_0(-1)\leq q_{\beta}(-1)$ and $q_{\beta}(1)\leq\phi_0(1)$ which gives $\beta\geq\beta_1$ and $\beta\geq\beta_2$ respectively, where
		$$\beta_1=\frac{k(e-1)(k+1)}{e(k-1)} \quad\text{and}\quad \beta_2=\frac{k(e-1)(k-1)}{k+1}.$$
		Let $\beta\geq\max\{\beta_1,\beta_2\}=\beta_1$. Since the difference of the square of the distances from the point $(1,0)$ to the points on the boundary curves  $\phi_0( e^{i\theta})$ and $q_{\beta}( e^{i\theta})$, respectively is given by
		$$T(\theta):= \frac{4(1+\sin^2{\theta})}{(k^2-2k\cos{\theta}+1)^2}-\frac{e^{2\cos{\theta}}+1}{{\beta}^2}+\frac{2e^{\cos{\theta}}\cos(\sin\theta)}{\beta},\quad 0\leq\theta\leq\pi.$$
		A calculation shows that $T'(\theta)\leq0$. Therefore, $T(\theta)$ decreases on $[0,\pi]$, which further implies that the condition $\phi_0(-1)\leq q_{\beta}(-1) \leq q_{\beta}(1)\leq\phi_0(1)$, that is $\beta\geq\beta_1$  is also sufficient for $q_{\beta}\prec\phi_0$. Hence the result. Rest of the results follow in a similar fashion.\qed
	\end{itemize}
\end{proof}

\begin{theorem}
	Let $0<\alpha<1$, $0<B<A<1$ and $k=1+\sqrt{2}$. If $p\in \mathcal{A}_{0}$ and satisfies the differential subordination
	\begin{equation*}
	1+\beta \frac{zp'(z)}{p(z)} \prec \wp(z),  
	\end{equation*}  
	then
	\begin{itemize}
		\item [$(i)$] $p(z)\prec e^z$   for $\beta\geq e-1$.
		\item [$(ii)$] $p(z) \prec \phi_0(z)$   for $\beta\geq (1/e-1)/\log\left({1-\frac{k-1}{k(k+1)}}\right)\approx 3.3583 $.
		\item [$(iii)$] $p(z)\prec \sqrt{1+z}$   for $\beta\geq\frac{e-1}{\log{\sqrt{2}}}$.
		\item [$(iv)$] $p(z)\prec \frac{1+Az}{1+Bz}$   for $\beta\geq \max\left\{\frac{1-1/e}{\log((1-B)/(1-A))}, \frac{e-1}{\log((1+A)/(1+B))}\right\}$.
		\item [$(v)$] $p(z)\prec 1+\sin{z}$   for $\beta\geq \frac{e-1}{\log({1+\sin{1}})}$.
		\item [$(vi)$] $p(z)\prec z+\sqrt{1+z^2}$   for $\beta\geq \frac{e-1}{\log(k)}$.
		
	\end{itemize}
	The bounds are sharp.
\end{theorem}
\begin{proof}
	Let $\theta(w)=1$ and $\phi(w)={\beta}/{\omega}\neq0$. Let 
	$	\Phi_\beta(z, p(z)):=1+\beta \frac{zp'(z)}{p(z)}. $
	\begin{itemize}
		\item [$(i)$] The differential equation $\Phi_\beta(z, p(z))=\wp(z)$ has an analytic solution given by
		\begin{equation*}
		q_{\beta}(z)=\exp\left(\frac{e^z-1}{\beta}\right).
		\end{equation*}
		Since the function 
		$Q(z)=zq'_{\beta}(z)\phi(q_{\beta}(z))=\wp(z)-1=ze^z\in\mathcal{S}^{*}$
		and	$h(z)=\theta(q_{\beta}(z))+Q(z)$ satisfies  
		$\Re(zh'(z)/Q(z))=\Re(zQ'(z)/Q(z))>0.$ 
		Therefore, by Lemma \ref{L1}, the following differential subordination implication holds:
		$$\Phi_\beta(z, p(z))\prec 1+\beta \frac{zq'_{\beta}(z)}{q_{\beta}(z)}\Rightarrow p\prec q_{\beta}.$$
		Now to show that $p\prec e^z$, we need to show that $q_{\beta}\prec e^z$ and for $q_{\beta} \prec e^z$, it is necessary that  $e^{-1}\leq q_{\beta}(-1)$ and $q_{\beta}(1)\leq e$ which gives $\beta\geq\beta_1$ and $\beta\geq\beta_2$ respectively, where
		$$\beta_1= (1-1/e) \quad\text{and}\quad \beta_2=e-1.$$
		Let $\beta\geq\beta_2$. To prove that $q_{\beta}(z) \prec e^z$, now it only suffices to show that $(e^z-1)/{\beta} \prec z$, which is equivalent to prove that for $z\in \mathbb{\overline{D}}$
		\begin{equation}\label{exp-eqn}
		|(e^z-1)/{\beta}|\leq1,
		\end{equation}
		and thus, for $z=e^{i\theta}$, we have
		\begin{equation*}
		|(e^{e^{i\theta}}-1)/{\beta}|= \sqrt{(e^{2\cos\theta}+1-2e^{\cos\theta}\cos(\sin\theta))/{\beta}^2}=:\sqrt{T(\theta)}.
		\end{equation*}
		Note that $T(\theta)=T(-\theta)$. So we may consider the interval $0\leq\theta\leq\pi$. Now since 
		$$T'(\theta)=2e^{\cos\theta}(\sin(\theta-\sin\theta)-\sin\theta)/{\beta}^2=0$$
		if and only if 
		$$\cos\left(\theta-\frac{\sin\theta}{2}\right)\sin{\left(\frac{\sin\theta}{2}\right)}=0$$
		if and only if $\theta\in \{0, \theta_0, \pi\}$, where $\theta_0\approx2.02098$ is the only root of the equation $$2\theta-\sin\theta-\pi=0.$$ Thus, we see that $T'(\theta)<0$ over $(0,\theta_0)$ and $T'(\theta)>0$ over $(\theta_0, \pi)$, which implies that $T(\theta)$ decreases over $[0,\theta_0]$ and increases over $[\theta_0,\pi]$ and therefore, 
		$$\max_{0\leq\theta\leq\pi} T(\theta)= \max\{T(0), T(\pi)\}=T(0)\leq1$$ and \eqref{exp-eqn} holds.
		Hence, $\beta\geq\beta_2$ is also sufficient for $q_{\beta}(z)\prec e^z$. Rest of the results follow in a similar fashion.\qed
	\end{itemize}	
\end{proof}

\begin{theorem}\label{n-gretr-2}
	Let $0<\alpha<1$, $0<B<A<1$ and $k=1+\sqrt{2}$. If $p\in \mathcal{A}_{0}$ and satisfies the differential subordination
	\begin{equation*}
	1+\beta \frac{zp'(z)}{p^2(z)} \prec \wp(z),  
	\end{equation*}  
	then
	\begin{itemize}
		\item [$(i)$] $p(z)\prec e^z$   for $\beta\geq \frac{e-1}{1-1/e}$.
		\item [$(ii)$] $p(z) \prec \phi_0(z)$   for $\beta\geq (e-1)/\left(1-\frac{1}{1+\frac{1}{k}\left(\frac{k+1}{k-1}\right)}\right)$.
		\item [$(iii)$] $p(z)\prec \sqrt{1+z}$   for $\beta\geq \frac{(e-1)(k-1)}{k-2}$.
		\item [$(iv)$] $p(z)\prec \frac{1+Az}{1+Bz}$   for $\beta\geq \max\left\{\frac{(1-1/e)(1-A)}{A-B}, \frac{(e-1)(1+A)}{A-B}\right\}$.
		\item [$(v)$] $p(z)\prec 1+\sin{z}$   for $\beta\geq \frac{(e-1)(1+\sin{1})}{\sin{1}}$.
		\item [$(vi)$] $p(z)\prec z+\sqrt{1+z^2}$   for $\beta\geq \frac{(e-1)k}{k-1}$.
	\end{itemize}
	The bounds are sharp.
\end{theorem}
\begin{proof}
	We set $\theta(w)=1$ and $\phi(w)=\beta/{w}^2\neq0$. Let
	$\mho_\beta(z, p(z)):=1+\beta \frac{zp'(z)}{p^2(z)}$.
	\begin{itemize}
		\item [$(i)$] The differential equation $\mho_\beta(z, p(z))=\wp(z)$ has an analytic solution given by
		\begin{equation*}
		q_{\beta}(z)=\frac{1}{1-\frac{1}{\beta}(e^z-1)}.
		\end{equation*}
		Since the function 
		$Q(z)=zq'_{\beta}(z)\phi(q_{\beta}(z))=\wp(z)-1=ze^z\in\mathcal{S}^{*}$ and the function	$h(z)=\theta(q_{\beta}(z))+Q(z)$ satisfies  
		$$\Re\biggl(\frac{zh'(z)}{Q(z)}\biggl)=\Re\biggl(\frac{zQ'(z)}{Q(z)}\biggl)>0.$$ 
		Therefore, by Lemma \ref{L1}, the following differential subordination implication holds:
		$$\Psi_\beta(z, p(z))\prec 1+\beta zq'_{\beta}(z)\Rightarrow p\prec q_{\beta}.$$
		Now to show that $p\prec e^z$, we need to show that $q_{\beta}\prec e^z$ and for $q_{\beta}\prec e^z$, it is necessary that $e^{-1}\leq q_{\beta}(-1)$ and $q_{\beta}(1)\leq e$ which gives $\beta\geq\beta_1$ and $\beta\geq\beta_2$ respectively, where
		$$\beta_1=\frac{1-1/e}{e-1} \quad\text{and}\quad \beta_2=\frac{e-1}{1-1/e}.$$
		Let $\beta\geq\max\{\beta_1,\beta_2\}=\beta_2$. Note that $e^z$ and $e^{-z}$ both maps $\mathbb{D}$ onto $\Omega:=\{w\in \mathbb{C}: w=e^z \;\text{and}\; z\in\mathbb{D} \}$. Therefore, to prove that $q_{\beta} \prec e^z$, or equivalently $q_{\beta}(z) \in \Omega$, it suffices to show that $q_{\beta} \prec e^{-z}$, which is equivalent to show that 
		\begin{equation}\label{w2}
		1-\frac{1}{\beta}(e^z-1) \prec e^z.
		\end{equation}  
		Now since the difference of the square of the distances from the point $(1,0)$ to the points on the boundary curves  $e^{e^{i\theta}}$ and $1/q_{\beta}( e^{i\theta})$, respectively is given by
		$$T(\theta):=\left(1-\frac{1}{{\beta}^2}\right)S(\theta), $$
		where $S(\theta)=1+e^{2\cos\theta}-2e^{\cos\theta}\cos(\sin\theta).$  
		Here, we may consider the interval $0\leq\theta\leq\pi$, since $T(\theta)=T(-\theta)$.
		A calculation shows that $S'(\theta)\leq0$ for $\theta\in[0,\pi]$. Therefore, $T(\theta)$ decreases on $[0,\pi]$ and also $$\min_{0\leq\theta\leq\pi}T(\theta)=\left(1-\frac{1}{{\beta}^2}\right)S(\pi)=\left(1-\frac{1}{{\beta}^2}\right)(1+e^{-2}-2e^{-1})>0,$$
		which means \eqref{w2} is true. Hence $\beta\geq\beta_2$  is also sufficient for $q_{\beta} \prec e^z$. Proof of the rest of the results are much akin and so are omitted here.\qed
	\end{itemize}
\end{proof}

\begin{remark}
	If $p\in \mathcal{A}_{0}$, $\beta>0$ and satisfies the differential subordination
	$$1+\beta\frac{zp'(z)}{p^{n}(z)}\prec\wp(z),\quad n\geq2$$
	then 
	$$p(z)\prec q_{\beta}(z):=\left(\frac{1}{1-\frac{n-1}{\beta}(e^z-1)}\right)^{\frac{1}{n-1}},$$
	where $q_{\beta}$ is the best dominant.
\end{remark}


\begin{theorem}\label{sb1}
	Let $p$ be an analytic function in $\mathbb{D}$ with $p(0)=1$. Let $0<\alpha<1$, $0<B<A<1$, $k=1+\sqrt{2}$ and $\wp(z)=1+ze^z$. Set
	\begin{equation*}
	\Psi_\beta(z, p(z))=1+\beta zp'(z). 
	\end{equation*}  
	If $p\in \mathcal{A}_{0}$ and  satisfies any of the following differential subordinations:
	\begin{itemize}
		\item [$(i)$] $\Psi_\beta(z, p(z)) \prec \phi_0(z)$ for $\beta\geq e(-1+2k\log(1+1/k))/k \approx0.75822$.
		\item [$(ii)$] $\Psi_\beta(z, p(z)) \prec \sqrt{1+z}$ for $\beta \geq 2e(1-\log{2})\approx 1.66822$.
		\item [$(iii)$] $\Psi_\beta(z, p(z)) \prec \varphi_\alpha(z):=1+\tfrac{z}{1-\alpha z^2}$ for $\beta\geq e\log(\frac{1+\sqrt{\alpha}}{1-\sqrt{\alpha}})/(2\sqrt{\alpha})$.
		\item [$(iv)$] $\Psi_\beta(z, p(z)) \prec \tfrac{1+Az}{1+Bz}$ for $\beta \geq \max\{\frac{e(A-B)}{B}\log(1-B), \frac{A-B}{eB}\log(1+B)\}$.
		\item [$(v)$] $\Psi_\beta(z, p(z)) \prec 1+\sin{z}$ for $\beta \geq e\sum_{n=0}^{\infty}\frac{(-1)^n}{(2n+1)!(2n+1)}\approx2.57172$.
		\item [$(vi)$] $\Psi_\beta(z, p(z)) \prec z+\sqrt{1+z^2}$ for $\beta\geq e(3-k+\log{(k/2)})\approx2.10399$.
		\item [$(vii)$] $\Psi_\beta(z, p(z)) \prec e^z$ for $\beta\geq e\sum_{n=1}^{\infty}\frac{(-1)^{n+1}}{(n!)(n)}\approx2.16538$.
	\end{itemize}
	Then $p \prec \wp$ and the bounds are sharp.
\end{theorem}
\begin{proof}
	We set $\theta(w)=1$ and $\phi(w)=\beta\neq0$. To prove our results, we make use of the inequality: 
	\begin{equation}\label{cond-0}
	\wp(-1)\leq q_{\beta}(-1)\leq q_{\beta}(1)\leq \wp(1),
	\end{equation} 
	if $q_{\beta}\prec \wp$ which gives necessary as well as sufficient condition for the required subordination.
	\begin{itemize}
		\item [$(i)$] The differential equation $\Psi_\beta(z, p(z))=\phi_0(z)$ has an analytic solution given by
		\begin{equation*}
		q_{\beta}(z)=1-\frac{1}{\beta k}\biggl(z+2k\log\biggl(1-\frac{z}{k}\biggl)\biggl).
		\end{equation*}
		Since the function 
		$$Q(z)=zq'_{\beta}(z)\phi(q_{\beta}(z))=\phi_0(z)-1=\frac{z}{k}\left(\frac{k+z}{k-z}\right)\in\mathcal{S}^{*}$$ and the function	$h(z)=\theta(q_{\beta}(z))+Q(z)$ satisfies  
		$$\Re\biggl(\frac{zh'(z)}{Q(z)}\biggl)=\Re\biggl(\frac{zQ'(z)}{Q(z)}\biggl)>0.$$ 
		Therefore, by Lemma \ref{L1}, the following differential subordination implication holds:
		$$\Psi_\beta(z, p(z))\prec 1+\beta zq'_{\beta}(z)\Rightarrow p\prec q_{\beta}.$$
		Now to show that $p\prec \wp$, we need to show that $q_{\beta}\prec\wp$. Now if $q_{\beta}\prec\wp$, then \eqref{cond-0} must hold, that is, the inequalities $\wp(-1)\leq q_{\beta}(-1)$ and $q_{\beta}(1)\leq\wp(1)$ respectively gives $\beta\geq\beta_1$ and $\beta\geq\beta_2$, where
		$$\beta_1=\frac{e}{k}\biggl(2k\log\biggl(\frac{k+1}{k}\biggl)-1\biggl) \quad\text{and}\quad \beta_2=\frac{1}{ek}\biggl(2k\log\biggl(\frac{k}{k-1}\biggl)+1\biggl).$$
		Let $\beta\geq\max\{\beta_1,\beta_2\}=\beta_1$. Note that $q_{\beta}(\mathbb{D})$ is convex, symmetric about real axis and  $q_{\beta}(\mathbb{D})\subset q_{\beta_{1}}(\mathbb{D})$ for each $\beta\geq{\beta_1}$. Also, $|\arg(1+z)|\leq\arctan(r/(1-r))$ for $|z|=r<1$. Now to prove $q_{\beta}\prec\wp$, in view of Lemma~\ref{disk_lem}, it only suffices to show that 
		$\max_{0\leq\theta\leq\pi}|\arg{q_{\beta_1}(e^{i\theta})}|=\max_{0\leq\theta\leq\pi}\left|\arg{\wp(e^{i\theta})}\right|$. Since    
		\begin{align*}
		&\quad \max_{0\leq\theta\leq\pi}|\arg{q_{\beta_1}(e^{i\theta})}|\\
		&=\max_{0\leq\theta\leq\pi}\left|\arctan({\Im{q_{\beta_1}(e^{i\theta})}}/{\Re{q_{\beta_1}(e^{i\theta})}})\right|\\
		&=\max_{0\leq\theta\leq\pi}\left|\arctan\left(\frac{(-2k\arg\left(1-\frac{e^{i\theta}}{k}\right)-\sin\theta)}{M-\cos\theta-{k\log((1+\frac{1}{k^2}-\frac{2\cos\theta}{k})}}\right)\right|\\
		&\leq \arctan\left(\frac{2k\arctan(\frac{1}{k-1})+1}{M(1-1/e)}\right) \approx 0.7719\\
		&<\max_{0\leq\theta\leq\pi}|\arg(1+ze^z)| \approx 1.41022,
		\end{align*}
		where $M=e(-1+2k\log(1+1/k))$.
		Therefore, $\beta\geq\beta_{1}$ is also sufficient for $q_{\beta}\prec\wp$ to hold. Hence the result. Other results follow similarly.\qed
		%		\item [$(ii)$] The differential equation $\Psi_\beta(z, p(z))=\sqrt{1+z}$ has an analytic solution in $\mathbb{D}$ given by
		%		$$q_{\beta}(z)=1+\frac{2(\sqrt{1+z}-\log(1+\sqrt{1+z})+\log2-1)}{\beta}.$$
		%		Since the function $Q(z)=zq'_{\beta}(z)\phi(q_{\beta}(z))=\sqrt{1+z}-1\in \mathcal{S}^{*}$ and $h(z)$ satisfies $\Re(zh'(z)/Q(z))=\Re(zQ'(z)/Q(z))>0$. Therefore, by Lemma \ref{L1}, the following differential subordination holds:
		%		$$1+\beta zp'(z)\prec 1+\beta zq'_{\beta}(z)\Rightarrow p\prec q_{\beta}.$$
		%		The required subordination $p\prec\wp$ holds if $q_{\beta}\prec\wp$, which is true if the inequalities in \ref{cond-1} holds which consequently gives  the necessary condition for the subordination $p\prec\wp$. The inequalities $\wp(-1)\leq q_{\beta}(-1)$ and $q_{\beta}(1)\leq\wp(1)$ gives $\beta\geq\beta_1$ and $\beta\geq\beta_2$ respectively, where
		%		$$\beta_1=2e(1-\log2)\quad\text{and}\quad \beta_2=(2/e)\left(\log({2}/{k})+k-2\right),\quad(k=\sqrt{2}+1).$$
		%		Hence, $q_{\beta}\prec\wp$ holds if $\beta\geq\max\{\beta_1,\beta_2\}$.
		%		  
		%		\item [$(iii)$] The differential equation $\Psi_\beta(z, p(z))=\varphi_{\alpha}(z)$ has an analytic solution in $\mathbb{D}$ given by
		%		$$q_{\beta}(z)=\frac{2\sqrt{\alpha}\beta+\log\biggl(\frac{1+\sqrt{\alpha}z}{1-\sqrt{\alpha}z}\biggl)}{2\sqrt{\alpha}\beta}.$$
		%		Since the function $Q(z)=zq'_{\beta}(z)\phi(q_{\beta}(z))=z/(1-\alpha z^2)\in \mathcal{S}^{*}$ and $h(z)=\theta(q_{\beta}(z))+Q(z)$ satisfies $\Re(zh'(z)/Q(z))=\Re(zQ'(z)/Q(z))>0$. Therefore, from Lemma \ref{L1}, the following differential subordination implication holds:
		%	    $$1+\beta zp'(z)\prec 1+\beta zq'_{\beta}(z)\Rightarrow p\prec q_{\beta}.$$
		%		Thus, using the condition given in equation \ref{cond-1}, we obtain the necessary conditions on $\beta$ for the subordination $p\prec q_{\beta}\prec\wp$ to hold. The inequalities $\wp(-1)\leq q_{\beta}(-1)$ and $q_{\beta}(1)\leq\wp(1)$ gives $\beta\geq\beta_1$ and $\beta\geq\beta_2$ respectively, where
		%		$$\beta_1=\frac{e}{2\sqrt{\alpha}}\log\biggl(\frac{1+\sqrt{\alpha}}{1-\sqrt{\alpha}}\biggl)\quad\text{and}\quad \beta_2=\frac{1}{2\sqrt{\alpha}e}\log\biggl(\frac{1+\sqrt{\alpha}}{1-\sqrt{\alpha}}\biggl).$$
		%		Hence, $q_{\beta}\prec\wp$ holds if $\beta\geq\max\{\beta_1,\beta_2\}$.
		%		
		%		\item [$(iv)$] The differential equation $\Psi_\beta(z, p(z))=1+(A-B)z/(1+Bz)$ has an analytic solution in $\mathbb{D}$ given by
		%		$$q_{\beta}(z)=\frac{\beta B+(A-B)\log(1+Bz)}{\beta B}.$$
		%		Since $(A-B)z/(1+Bz)\in \mathcal{S}^{*}$, it follows that $Q(z)=zq'_{\beta}(z)\phi(q_{\beta}(z))\in \mathcal{S}^{*}$. Also $h(z)=\theta(q_{\beta}(z))+Q(z)$ satisfies $\Re(zh'(z)/Q(z))=\Re(zQ'(z)/Q(z))>0$. Therefore, from Lemma \ref{L1}, the following differential subordination implication holds:
		%		$$1+\beta zp'(z)\prec 1+\beta zq'_{\beta}(z)\Rightarrow p\prec q_{\beta}.$$
		%	    Now for the required subordination $p\prec \wp$ to hold, we just need to show that $q_{\beta}\prec \wp$ which holds if 
		%		$$\beta\geq\max\{\beta_1,\beta_2\}=\beta_1,$$ where 
		%		$$\beta_1=\frac{e(A-B)}{B}\log(1-B)\quad\text{and}\quad \beta_2=\frac{A-B}{eB}\log(1+B)$$ 
		%	    are	obtained from the inequalities $\wp(-1)\leq q_{\beta}(-1)$ and $q_{\beta}\leq \wp(1)$, respectively.
		%	    
		%	    \item [$(v)$] The differential equation $\Psi_\beta(z, p(z))=1+\sin{ z}$ has an analytic solution in $\mathbb{D}$ given by
		%	    $$q_{\beta}(z)=1+\frac{1}{\beta}\sum_{n=0}^{\infty}\frac{(-1)^nz^{2n+1}}{(2n+1)!(2n+1)}.$$ 
		%	    Since $Q(z)=zq'_{\beta}(z)\phi(q_{\beta}(z))=\sin z\in \mathcal{S}^{*}$ and the function $h(z)=\theta(q_{\beta}(z))+Q(z)$ satisfies $\Re(zh'(z)/Q(z))=\Re(zQ'(z)/Q(z))>0$. Therefore, from Lemma \ref{L1}, the following differential subordination implication holds:
		%	   	$$1+\beta zp'(z)\prec 1+\beta zq'_{\beta}(z)\Rightarrow p\prec q_{\beta}.$$
		%	   The required subordination $p\prec\wp$ holds if $q_{\beta}\prec\wp$, which is true if the inequalities in \ref{cond-1} holds which consequently gives  the necessary condition for the subordination $p\prec\wp$. The inequalities $\wp(-1)\leq q_{\beta}(-1)$ and $q_{\beta}(1)\leq\wp(1)$ gives $\beta\geq\beta_1$ and $\beta\geq\beta_2$ respectively, where
		%	    $$\beta_1=e\sum_{n=0}^{\infty}\frac{(-1)^{3n+2}}{(2n+1)!(2n+1)}\quad\text{and}\quad \beta_2=\frac{1}{e}\sum_{n=0}^{\infty}\frac{(-1)^n}{(2n+1)!(2n+1)}.$$
		%	   	Hence, $q_{\beta}\prec\wp$ holds if $\beta\geq\max\{\beta_1,\beta_2\}=\beta_1$.
		%	   	
		%	   	\item [$(vi)$] The differential equation $\Psi_\beta(z, p(z))=z+\sqrt{1+z^2}$ has an analytic solution in $\mathbb{D}$ given by
		%	   	$$ q_{\beta}(z)=1+\frac{z+\sqrt{1+z^2}-\log(1+\sqrt{1+z^2})+\log2-1}{\beta}.$$
		%	   	Since the function $Q(z)=zq'_{\beta}(z)\phi(q_{\beta}(z))=z+\sqrt{1+z^2}-1\in \mathcal{S}^{*}$ and $h(z)=\theta(q_{\beta}(z))+Q(z)$ satisfies $\Re(zh'(z)/Q(z))>0$. Therefore, from Lemma \ref{L1},the following differential subordination implication holds:
		%	   	$$1+\beta zp'(z)\prec 1+\beta zq'_{\beta}(z)\Rightarrow p\prec q_{\beta}.$$
		%	   Now the inequalities $\wp(-1)\leq q_{\beta}(-1)$ and $q_{\beta}(1)\leq\wp(1)$ gives $\beta\geq\beta_1$ and $\beta\geq\beta_2$ respectively, where
		%	   $$\beta_1=e(3-k+\log(k/2))\quad\text{and}\quad \beta_2=\frac{k-1+\log(2/k)}{e}, \quad(k=\sqrt{2}+1).$$
		%	   Hence, $q_{\beta}\prec\wp$ holds if $\beta\geq\max\{\beta_1,\beta_2\}$.
		%	   
		%	   \item [$(vii)$] The differential equation $\Psi_\beta(z, p(z))=e^z$ has an analytic solution in $\mathbb{D}$ given by
		%	   $$ q_{\beta}(z)=1+\frac{1}{\beta}\sum_{n=1}^{\infty}\frac{z^n}{n!n}.$$
		%	   Since the function $Q(z)=zq'_{\beta}(z)\phi(q_{\beta}(z))=e^z\in \mathcal{S}^{*}$ and the function $h(z)$ satisfies $\Re(zh'(z)/Q(z))>0$. Therefore, by Lemma \ref{L1}, the following differential subordination holds:
		%	   	$$1+\beta zp'(z)\prec 1+\beta zq'_{\beta}(z)\Rightarrow p\prec q_{\beta}.$$
		%	   Now for the required subordination $p\prec\wp$ to hold, we need to show $q_{\beta}\prec\wp$ using the conditions in \ref{cond-1}. The inequalities $\wp(-1)\leq q_{\beta}(-1)$ and $q_{\beta}(1)\leq\wp(1)$ gives $\beta\geq\beta_1$ and $\beta\geq\beta_2$ respectively, where
		%	   $$\beta_1=e\sum_{n=1}^{\infty}\frac{(-1)^{n+1}}{n!n}\quad\text{and}\quad\beta_2=\frac{1}{e}\sum_{n=1}^{\infty}\frac{1}{n!n}.$$  
		%	   Hence, $q_{\beta}\prec\wp$ holds if $\beta\geq\max\{\beta_1,\beta_2\}$.
	\end{itemize}
\end{proof}

Let $f\in\mathcal{A}$ and setting $p(z)=zf'(z)/f(z)$ and $p(z)=f(z)/z$. Then using Theorem \ref{sb1}, we obtain sufficient conditions for $f$ to be in $\mathcal{S}^*_{\wp}$.

\begin{corollary}
	Let $f\in \mathcal{A}$ and set
	\begin{equation*}
	\Psi_\beta\biggl(z,\frac{zf'(z)}{f(z)}\biggl)=1+\beta\frac{zf'(z)}{f(z)}\biggl(1-\frac{zf'(z)}{f(z)}+\frac{zf''(z)}{f'(z)}\biggl).
	\end{equation*}
	If $ \Psi_\beta\left(z,\frac{zf'(z)}{f(z)}\right)$ satisfies any of the conditions (i)-(vii) of the Theorem \ref{sb1}, then $ f\in \mathcal{S}^*_{\wp} $.
\end{corollary}
\begin{corollary}
	Let $f\in \mathcal{A}$ and set
	\begin{equation*}
	\Psi_\beta\biggl(z,\frac{f(z)}{z}\biggl)=1+\beta\left(f'(z)-\frac{f(z)}{z}\right).
	\end{equation*}
	If $ \Psi_\beta\left(z,\frac{f(z)}{z}\right)$ satisfies any of the conditions (i)-(vii) of the Theorem \ref{sb1}, then $ f\in \mathcal{S}^*_{\wp} $.
\end{corollary}

%The figure in Table1 illustrate that parameters defined in Theorem \ref{sb1} are sharp.
% \begin{table}{h}
%	\begin{tabular}{c}\label{fig:incl_rel}
%		\includegraphics[scale=0.8]{diff-fig1.pdf}
%	\end{tabular}
%    \caption{Best subordinants for $\wp=1+ze^z$, Theorem \ref{sb1}.}
% \end{table}

\begin{theorem}\label{sb2}
	Let $0<\alpha<1$, $0<B<A<1$, $\gamma=\log(e/(e-1))$, $k=1+\sqrt{2}$ and $\wp(z)=1+ze^z$ and $p\in \mathcal{A}_{0}$. Set
	\begin{equation*}
	\Phi_\beta(z, p(z))=1+\beta\frac{zp'(z)}{p(z)}.
	\end{equation*}  
	If $p$ satisfies any of the following differential subordinations: 	
	\begin{itemize}
		\item [$(i)$] $\Phi_\beta(z, p(z)) \prec \phi_0(z)$ for $\beta\geq (-1+2k\log(1+1/k))/\gamma k \approx0.60812$.
		\item [$(ii)$] $\Phi_\beta(z, p(z)) \prec \sqrt{1+z}$ for $\beta \geq 2(1-\log{2})/\gamma\approx 1.33799$.
		\item [$(iii)$] $\Phi_\beta(z, p(z)) \prec \varphi_\alpha(z)$ for $\beta\geq  \log(\frac{1+\sqrt{\alpha}}{1-\sqrt{\alpha}})/(2\gamma\sqrt{\alpha})$.
		\item [$(iv)$] $\Phi_\beta(z, p(z)) \prec (1+Az)/(1+Bz)$ for $\beta \geq \max\left\{\frac{(A-B)\log(1/(1-B))}{\gamma B}, \frac{(A-B)\log(1+B)}{B\log(1+e)}\right\}.$
		\item [$(v)$] $\Phi_\beta(z, p(z)) \prec 1+\sin{z}$ for $\beta \geq (1/\gamma)\sum_{n=0}^{\infty}\frac{(-1)^(3n+2)}{(2n+1)!(2n+1)}\approx2.06264$.
		\item [$(vi)$] $\Phi_\beta(z, p(z)) \prec z+\sqrt{1+z^2}$ for $\beta\geq (3-k+\log{(k/2)})/\gamma\approx1.6875$.
		\item [$(vii)$] $\Phi_\beta(z, p(z)) \prec e^z$ for $\beta\geq (1/\gamma)\sum_{n=1}^{\infty}\frac{(-1)^{n+1}}{(n!)(n)}\approx1.736740$.
	\end{itemize}
	Then $p \prec \wp$ and all the above bounds are sharp. 
\end{theorem}
\begin{proof}
	Let $\theta(w)=1$ and $\phi(w)={\beta}/{\omega}\neq0$. To prove our results, we make use of the following inequality: 
	\begin{equation}\label{cond-1}
	\wp(-1)\leq q_{\beta}(-1)\leq q_{\beta}(1)\leq \wp(1)
	\end{equation} 
	so that $q_{\beta}\prec \wp$. Observe that \eqref{cond-1} gives necessary as well as sufficient condition for the required subordination.
	\begin{itemize}
		\item [$(i)$] The differential equation $\Phi_\beta(z, p(z))=\phi_0(z)$ has an analytic solution given by
		\begin{equation}\label{e-u-v}
		q_{\beta}(z)=\exp\left(-(z+2k\log(1-z/k))/\beta k\right).
		\end{equation}
		Since the function 
		$$Q(z)=zq'_{\beta}(z)\phi(q_{\beta}(z))=\phi_0(z)-1=\frac{z}{k}\left(\frac{k+z}{k-z}\right)\in\mathcal{S}^{*}$$ 
		and	$h(z)=\theta(q_{\beta}(z))+Q(z)$ satisfies  
		$\Re(zh'(z)/Q(z))=\Re(zQ'(z)/Q(z))>0.$ 
		Therefore, by Lemma \ref{L1}, the following differential subordination implication holds:
		$$\Phi_\beta(z, p(z))\prec 1+\beta \frac{zq'_{\beta}(z)}{q_{\beta}(z)}\Rightarrow p\prec q_{\beta}.$$
		Now to show that $p\prec \wp$, we need to show that $q_{\beta}\prec\wp$. Thus, \eqref{cond-1} gives necessary condtion for the subordination $p\prec \wp$ to hold. That is,
		$\beta\geq\beta_1$ and $\beta\geq\beta_2$, respectively, where
		$$\beta_1= \frac{-1+2k\log(1+1/k)}{k\log(e/(e-1))}\quad\text{and}\quad \beta_2=\frac{-1+2k\log(k/(k-1))}{k\log(1+e)}.$$
		Let $\beta\geq \beta_1$. Also for simplicity, we write the function given in \eqref{e-u-v} as follows for $z=e^{i\theta}$:
		\begin{equation*}
		q_{\beta}(z):=\exp(u+iv),
		\end{equation*}
		where $u$ and $v$ are function of $\theta\in[-\pi,\pi]$ given as
		\begin{equation*}
		u(\theta):=\left(\frac{-1}{\beta k}\right)\left(\cos\theta+ 2k\log{\sqrt{1+\frac{1}{k^2}-\frac{2\cos\theta}{k}}}\right)
		\end{equation*}
		and
		\begin{equation*}
		v(\theta):=\left(\frac{-1}{\beta k}\right)\left(\sin\theta+ 2k\arctan\left(\frac{\sin\theta}{\cos\theta-k}\right)\right).
		\end{equation*}
		Further, without loss of generality, we may assume that $\beta=\beta_1$. Now the difference of the square of the distances from the point $(1,0)$ to the points on the boundary curves  $\wp(e^{i\theta})$ and $q_{\beta}( e^{i\theta})$, respectively is given by
		$$d(\theta):=e^{2\cos{\theta}}-1-e^{2u(\theta)}+2e^{u(\theta)}\cos(v(\theta)).$$
		Since $d(\theta)=d(-\theta)$. So we may consider the interval $0\leq\theta\leq\pi$. Now we need to prove that $d(\theta)\geq0$ and so we consider 
		\begin{equation*}
		d'(\theta)=-e^{2\cos\theta}\sin\theta -2e^{u(\theta)}(e^{u(\theta)}u'(\theta) +u'(\theta)\cos(v(\theta))  -\sin(v(\theta)) v'(\theta)).
		\end{equation*}
		It is easy to see that $d'(\theta)=0$ for $\theta=0$ and $\pi$. Further a numerical computation also shows that $d'(\theta)=0$ for a unique $\theta_0\in(0,\pi)$ such that $d(\theta)$ increases on $[0,\theta_0]$ and decreases on $[\theta_0, \pi]$, where $\theta_0\approx0.351807$. Note that $d(0)>0$ and $d(\pi)=0$. Therefore,
		$$\min_{0\leq\theta\leq\pi}d(\theta)=0.$$ 
		Hence $\beta\geq\beta_1$  is also sufficient for $q_{\beta} \prec \wp$ and this completes the proof. Similarly we can prove the other results. \qed
		
		
		%			\item [$(ii)$] The differential equation $\Phi_\beta(z, p(z))=\sqrt{1+z}$ has an analytic solution given by
		%		\begin{equation*}
		%		q_{\beta}(z)=\exp\left(2(\sqrt{1+z}-\log(1+\sqrt{1+z})+\log2-1)/\beta\right).
		%		\end{equation*}
		%		Since the function 
		%		$$Q(z)=zq'_{\beta}(z)\phi(q_{\beta}(z))=\sqrt{1+z}-1\in \mathcal{S}^{*}$$ and	$h(z)=\theta(q_{\beta}(z))+Q(z)$ satisfies  
		%		$\Re({zh'(z)}/{Q(z)})>0.$ 
		%		Therefore, by Lemma \ref{L1}, the following differential subordination implication holds:
		%		$$1+\beta zp'(z)\prec 1+\beta zq'_{\beta}(z)\Rightarrow p\prec q_{\beta}.$$
		%		Now to show that $p\prec \wp$, we need to show that $q_{\beta}\prec\wp$. Thus, the equation \ref{cond-1} gives necessary condtion for the subordination $p\prec \wp$ to hold.
		%		The inequalities $\wp(-1)\leq q_{\beta}(-1)$ and $q_{\beta}(1)\leq\wp(1)$ gives $\beta\geq\beta_1$ and $\beta\geq\beta_2$ respectively, where
		%		$$\beta_1=\frac{2(1-\log2)}{\log(e/(e-1))}    \quad\text{and}\quad \beta_2=\frac{2(\sqrt{2}-1+\log(2/(1+\sqrt{2})))}{\log{1+e}} .$$
		%		Hence, $q_{\beta}\prec\wp$ holds if $\beta\geq\max\{\beta_1,\beta_2\}$.
		%		
		%			\item [$(iii)$] The differential equation $\Phi_\beta(z, p(z))=1+1/(1-\alpha z^2)$ has an analytic solution given by
		%		\begin{equation*}
		%		q_{\beta}(z)=\exp\left(\frac{1}{2\sqrt{\alpha}\beta} \log\frac{1+\sqrt{\alpha}z}{1-\sqrt{\alpha}z}\right).
		%		\end{equation*}
		%		Since the function 
		%		$$Q(z)=zq'_{\beta}(z)\phi(q_{\beta}(z))=\frac{1}{1-\alpha z^2}\in \mathcal{S}^{*}$$ and $h(z)=\theta(q_{\beta}(z))+Q(z)$ satisfies  
		%		$\Re({zh'(z)}/{Q(z)})>0.$ 
		%		Therefore, by Lemma \ref{L1}, the following differential subordination implication holds:
		%		$$1+\beta zp'(z)\prec 1+\beta zq'_{\beta}(z)\Rightarrow p\prec q_{\beta}.$$
		%		Now to show that $p\prec \wp$, we need to show that $q_{\beta}\prec\wp$. Thus, the equation \ref{cond-1} gives necessary condtion for the subordination $p\prec \wp$ to hold.
		%		The inequalities $\wp(-1)\leq q_{\beta}(-1)$ and $q_{\beta}(1)\leq\wp(1)$ gives $\beta\geq\beta_1$ and $\beta\geq\beta_2$ respectively, where
		%		$$\beta_1=\frac{\log\frac{1+\sqrt{\alpha}}{1-\sqrt{\alpha}}}{2\sqrt{\alpha}\log(e/(e-1))}    \quad\text{and}\quad \beta_2=\frac{\log\frac{1+\sqrt{\alpha}}{1-\sqrt{\alpha}}}{2\sqrt{\alpha}\log(1+e)} .$$
		%		Hence, $q_{\beta}\prec\wp$ holds if $\beta\geq\max\{\beta_1,\beta_2\}$.
		%		
		%			\item [$(iv)$] The differential equation $\Phi_\beta(z, p(z))=(1+A z)/(1+B z)$ has an analytic solution given by
		%		\begin{equation*}
		%		q_{\beta}(z)=\exp\left(((A-B)\log(1+Bz))/B\beta\right).
		%		\end{equation*}
		%		Since the function 
		%		$$Q(z)=zq'_{\beta}(z)\phi(q_{\beta}(z))=\frac{(A-B)z}{1+B z}\in \mathcal{S}^{*}$$ and $h(z)=\theta(q_{\beta}(z))+Q(z)$ satisfies  
		%		$\Re({zh'(z)}/{Q(z)})>0.$ 
		%		Therefore, by Lemma \ref{L1}, the following differential subordination implication holds:
		%		$$1+\beta zp'(z)\prec 1+\beta zq'_{\beta}(z)\Rightarrow p\prec q_{\beta}.$$
		%		Now to show that $p\prec \wp$, we need to show that $q_{\beta}\prec\wp$. Thus, the equation \ref{cond-1} gives necessary condtion for the subordination $p\prec \wp$ to hold.
		%		The inequalities $\wp(-1)\leq q_{\beta}(-1)$ and $q_{\beta}(1)\leq\wp(1)$ gives $\beta\geq\beta_1$ and $\beta\geq\beta_2$ respectively, where
		%		$$\beta_1=\frac{(A-B)\log(1/(1-B))}{B\log(e/(e-1))}    \quad\text{and}\quad \beta_2=\frac{(A-B)\log(1+B)}{B\log(1+e)} .$$
		%		Hence, $q_{\beta}\prec\wp$ holds if $\beta\geq\max\{\beta_1,\beta_2\}$.
		%		
		%			\item [$(v)$] The differential equation $\Phi_\beta(z, p(z))=1+\sin z$ has an analytic solution given by
		%		\begin{equation*}
		%		q_{\beta}(z)=\exp\left(\frac{1}{\beta}\sum_{n=0}^{\infty}\frac{(-1)^nz^{2n+1}}{(2n+1)!(2n+1)}\right).
		%		\end{equation*}
		%		Since the function 
		%		$$Q(z)=zq'_{\beta}(z)\phi(q_{\beta}(z))=\sin z\in \mathcal{S}^{*}$$ and the function	$h(z)=\theta(q_{\beta}(z))+Q(z)$ satisfies  
		%		$\Re({zh'(z)}/{Q(z)})>0.$ 
		%		Therefore, by Lemma \ref{L1}, the following differential subordination implication holds:
		%		$$1+\beta zp'(z)\prec 1+\beta zq'_{\beta}(z)\Rightarrow p\prec q_{\beta}.$$
		%		Now to show that $p\prec \wp$, we need to show that $q_{\beta}\prec\wp$. Thus, the equation \ref{cond-1} gives necessary condtion for the subordination $p\prec \wp$ to hold.
		%		The inequalities $\wp(-1)\leq q_{\beta}(-1)$ and $q_{\beta}(1)\leq\wp(1)$ gives $\beta\geq\beta_1$ and $\beta\geq\beta_2$ respectively, where
		%		$$\beta_1= \frac{1}{\log(e/(e-1))}\sum_{n=0}^{\infty}\frac{(-1)^{3n+2}}{(2n+1)!(2n+1)}$$   
		%		 and
		%		$$ \beta_2= \frac{1}{\log(1+e)}\sum_{n=0}^{\infty}\frac{(-1)^n}{(2n+1)!(2n+1)} .$$
		%		Hence, $q_{\beta}\prec\wp$ holds if $\beta\geq\max\{\beta_1,\beta_2\}$.
		%		
		%			\item [$(vi)$] The differential equation $\Phi_\beta(z, p(z))=z+\sqrt{1+z^2}$ has an analytic solution given by
		%		\begin{equation*}
		%		q_{\beta}(z)=\exp\left((z+\sqrt{1+z^2}-\log(1+\sqrt{1+z^2})-1+\log2)/\beta\right).
		%		\end{equation*}
		%		Since the function 
		%		$$Q(z)=zq'_{\beta}(z)\phi(q_{\beta}(z))=z+\sqrt{1+z^2}-1\in \mathcal{S}^{*}$$ and the function	$h(z)=\theta(q_{\beta}(z))+Q(z)$ satisfies  
		%		$\Re({zh'(z)}/{Q(z)})>0.$ 
		%		Therefore, by Lemma \ref{L1}, the following differential subordination implication holds:
		%		$$1+\beta zp'(z)\prec 1+\beta zq'_{\beta}(z)\Rightarrow p\prec q_{\beta}.$$
		%		Now to show that $p\prec \wp$, we need to show that $q_{\beta}\prec\wp$. Thus, the equation \ref{cond-1} gives necessary condtion for the subordination $p\prec \wp$ to hold.
		%		The inequalities $\wp(-1)\leq q_{\beta}(-1)$ and $q_{\beta}(1)\leq\wp(1)$ gives $\beta\geq\beta_1$ and $\beta\geq\beta_2$ respectively, where
		%		$$\beta_1= \frac{2-\sqrt{2}+\log((1+\sqrt{2})/2)}{\log(e/(e-1))}$$   
		%		and
		%		$$ \beta_2=\frac{\log(2/(1+\sqrt{2}))+\sqrt{2}}{\log(1+e)} .$$
		%		Hence, $q_{\beta}\prec\wp$ holds if $\beta\geq\max\{\beta_1,\beta_2\}$.
		%		
		%		\item [$(vii)$] The differential equation $\Phi_\beta(z, p(z))=e^z$ has an analytic solution given by
		%		\begin{equation*}
		%		q_{\beta}(z)=\exp\left(\frac{1}{\beta}\sum_{n=1}^{\infty} \frac{z^n}{n!n}\right).
		%		\end{equation*}
		%		Since the function 
		%		$$Q(z)=zq'_{\beta}(z)\phi(q_{\beta}(z))=e^z-1\in \mathcal{S}^{*}$$ and the function	$h(z)=\theta(q_{\beta}(z))+Q(z)$ satisfies  
		%		$\Re({zh'(z)}/{Q(z)})>0.$ 
		%		Therefore, by Lemma \ref{L1}, the following differential subordination implication holds:
		%		$$1+\beta zp'(z)\prec 1+\beta zq'_{\beta}(z)\Rightarrow p\prec q_{\beta}.$$
		%		Now to show that $p\prec \wp$, we need to show that $q_{\beta}\prec\wp$. Thus, the equation \ref{cond-1} gives necessary condtion for the subordination $p\prec \wp$ to hold.
		%		The inequalities $\wp(-1)\leq q_{\beta}(-1)$ and $q_{\beta}(1)\leq\wp(1)$ gives $\beta\geq\beta_1$ and $\beta\geq\beta_2$ respectively, where
		%		$$\beta_1= \frac{1}{\log(e/(e-1))}\sum_{n=1}^{\infty}\frac{(-1)^{n+1}}{n!n}$$   
		%		and
		%		$$ \beta_2=\frac{1}{\log(1+e)}\sum_{n=1}^{\infty} \frac{1}{n!n} .$$
		%		Hence, $q_{\beta}\prec\wp$ holds if $\beta\geq\max\{\beta_1,\beta_2\}$.
	\end{itemize}	
\end{proof}

Let $f\in\mathcal{A}$ and set $p(z)=zf'(z)/f(z)$ and $p(z)=f(z)/z$. Then using Theorem \ref{sb2}, we obtain sufficient conditions for $f$ to be in $\mathcal{S}^*_{\wp}$.

\begin{corollary}
	Let $f\in\mathcal{A}$ and set
	\begin{equation*}
	\Phi_\beta\biggl(z,\frac{zf'(z)}{f(z)}\biggl)=1+\beta\biggl(1-\frac{zf'(z)}{f(z)}+\frac{zf''(z)}{f(z)}\biggl).    	
	\end{equation*}
	If $ \Phi_\beta\left(z,\frac{zf'(z)}{f(z)}\right)$ satisfies any of the conditions (i)-(vii) of Theorem \ref{sb2}, then $ f\in \mathcal{S}^*_{\wp} $.
\end{corollary}
\begin{corollary}
	Let $f\in\mathcal{A}$ and set
	\begin{equation*}
	\Phi_\beta\biggl(z,\frac{f(z)}{z}\biggl)=\beta \frac{zf'(z)}{f(z)}+(1-\beta).    	
	\end{equation*}
	If $ \Phi_\beta\left(z,\frac{f(z)}{z}\right)$ satisfies any of the conditions (i)-(vii) of Theorem \ref{sb2}, then $ f\in \mathcal{S}^*_{\wp} $.
\end{corollary}

%The figure in Table2 illustrate that parameters defined in Theorem \ref{sb2} are sharp.
%\begin{table}{h}
%	\begin{tabular}{c}\label{fig:incl_rel}
%		\includegraphics[scale=0.8]{diff-fig2.pdf}
%	\end{tabular}
%	\caption{Best subordinants for $\wp=1+ze^z$, Theorem \ref{sb2}.}
%\end{table}
The proof of the following result is omitted here as it is much akin to the proof of Theorem \ref{sb1} and \ref{sb2}. 
\begin{theorem}\label{sb3}
	Assume that $0<\alpha<1$, $0<B<A<1$, $k=1+\sqrt{2}$ and $\wp(z)=1+ze^z$. Let $p\in \mathcal{A}_{0}$. Set
	\begin{equation*}
	\mho_\beta(z, p(z))=1+\beta\frac{zp'(z)}{p^2(z)}.
	\end{equation*}  
	If $p$ satisfies any of the following differential subordinations: 	
	\begin{itemize}
		\item [$(i)$] $\mho_\beta(z, p(z)) \prec \phi_0(z)$ for $\beta\geq-((e+1)/ke)(1+2k\log(1-1/k))\approx0.896489$.
		\item [$(ii)$] $\mho_\beta(z, p(z)) \prec \sqrt{1+z}$ for $\beta \geq2(e-1)(1-\log{2})\approx1.05451 $.
		\item [$(iii)$] $\mho_\beta(z, p(z)) \prec \varphi_\alpha(z)$ for $\beta\geq ((e-1)/2\sqrt{\alpha})\log(\frac{1+\sqrt{\alpha}}{1-\sqrt{\alpha}})$.
		%\item [$(iv)$] $\mho_\beta(z, p(z)) \prec (1+Az)/(1+Bz)$ for $\beta \geq \beta_0$.
		\item [$(iv)$] $\mho_\beta(z, p(z)) \prec 1+\sin{z}$ for $\beta \geq(e-1)\sum_{n=0}^{\infty}\frac{(-1)^n}{(2n+1)!(2n+1)}\approx1.62563$.
		%\item [$(vi)$] $\mho_\beta(z, p(z)) \prec z+\sqrt{1+z^2}$ for $\beta\geq$.
		\item [$(v)$] $\mho_\beta(z, p(z)) \prec e^z$ for $\beta\geq((e+1)/e)\sum_{n=1}^{\infty}\frac{1}{n!n}\approx1.80273$.
	\end{itemize}
	Then $p \prec \wp$ and the bounds are sharp.
\end{theorem}

Let $f\in\mathcal{A}$ and set $p(z)=zf'(z)/f(z)$ and $p(z)=f(z)/z$. Then using Theorem \ref{sb3}, we obtain the following results:

\begin{corollary}
	Let $f\in\mathcal{A}$ and set
	\begin{equation*}
	\mho_\beta\biggl(z,\frac{zf'(z)}{f(z)}\biggl)=1+\beta\biggl(\frac{zf'(z)}{f(z)}\biggl)^{-1}\biggl(1-\frac{zf'(z)}{f(z)}+\frac{zf''(z)}{f(z)}\biggl).    	
	\end{equation*}
	If $ \mho_\beta\left(z,\frac{zf'(z)}{f(z)}\right)$ satisfies any of the conditions (i)-(v) of the Theorem \ref{sb3}, then $ f\in \mathcal{S}^*_{\wp} $.
\end{corollary}
\begin{corollary}
	Let $f\in\mathcal{A}$ and set
	\begin{equation*}
	\mho_\beta\biggl(z,\frac{f(z)}{z}\biggl)=1+\beta\left(\frac{z}{f(z)}\left(\frac{zf'(z)}{f(z)}-1\right)\right).    	
	\end{equation*}
	If $ \mho_\beta\left(z,\frac{f(z)}{z}\right)$ satisfies any of the conditions (i)-(v) of the Theorem \ref{sb3}, then $ f\in \mathcal{S}^*_{\wp} $.
\end{corollary}

	
	
	
	
\section*{Conflict of interest}
	The authors declare that they have no conflict of interest.

\begin{thebibliography}{99}
	\setlength{\itemsep}{5pt}

	\bibitem{abo-2018} Akta\c{s} \.{I}, Baricz \'{A} and Orhan H. Bounds for radii of starlikeness and convexity of some special functions. Turkish J Math, 2018, {\bf 42}(1): 211--226

\bibitem{con2010}  Ali R M, Ravichandran V and Jain N K. Convolution of certain analytic functions. J Anal, 2010, {\bf 18}: 1--8

\bibitem{BP-mitiig-2016} Bansal D and  Prajapat J K. Certain geometric properties of the Mittag-Leffler functions. Complex Variables and Elliptic Equations, 2016, {\bf61}(3): 338--350

\bibitem{Banga-2020} Banga S and Kumar S S. Applications of differential subordinations to certain classes of starlike functions. J Korean Math Soc, 2020, {\bf57}(2): 331--357

\bibitem{bdoy-2016}Baricz \'{A}, Dimitrov D K, Orhan H, and Ya\u{g}mur N. Radii of starlikeness of some special functions. Proc Amer Math Soc, 2016, {\bf 144}(8): 3355--3367

\bibitem{btk-2-18}Baricz \'{A}, Toklu E and Kadio\u{g}lu E. Radii of starlikeness and convexity of Wright functions. Math Commun, 2018, {\bf 23}(1): 97--117

\bibitem{lommel-hadmrd} Biernacki M and Krzy\.{z} J. On the monotonity of certain functionals in the theory of analytic functions. Ann Univ Mariae Curie-Sk\l odowska Sect A,1957, {\bf 9}(1955): 135--147

\bibitem{bulut-engel-2019} Bulut S and Engel O. The radius of starlikeness, convexity and uniform convexity of the Legendre polynomials of odd degree. Results Math, 2019, {\bf 74}(1), Art 48, 9 pp.

\bibitem{cho2019} Cho N E, Kumar S, Kumar v and Ravichandran V. Convolution and radius problems of analytic functions associated with the tilted Carath\'{e}odory functions. Math Commun, 2019, {\bf 24}(2): 165--179

\bibitem{sinefun}  Cho N E, Kumar V,  Kumar S S and Ravichandran V. Radius problems for starlike functions associated with the sine function. Bull Iranian Math Soc, 2019, {\bf 45}(1): 213--232

\bibitem{ganga1997} Gangadharan A, Ravichandran V and Shanmugam T N. Radii of convexity and strong starlikeness for some classes of analytic functions. J Math Anal Appl, 1997, {\bf 211}(1): 301--313

\bibitem{goel2020}Goel P and Kumar S S. Certain Class of Starlike Functions Associated with Modified Sigmoid Function. Bull Malays Math Sci Soc, 2020, {\bf 43}(1): 957--991

\bibitem{golu}  Goluzin G M. On a variational method in the theory of analytic functions. Amer Math Soc Transl, 1961, {\bf 18}(2): 1--14

\bibitem{kargar-ebadian} Kargar R, Ebadian A and {Sok\'{o}\l} J. Radius problems for some subclasses of analytic functions. Complex Anal Oper Theory, 2017, {\bf11}(7): 1639--1649

\bibitem{virendraBell} Kumar V, Cho N E, Ravichandran V and Srivastava H M. Sharp coefficient bounds for starlike functions associated with the Bell numbers. Math Slovaca, 2019, {\bf69}(5): 1053--1064

\bibitem{Kumar-cardioid}  Kumar S S and Kamaljeet G. A Cardioid Domain and Starlike Functions. Analysis and Mathematical Physics, 2020 (Communicated)

\bibitem{kumarravi2016} Kumar S and Ravichandran V. A subclass of starlike functions associated with a rational function. Southeast Asian Bull Math, 2016, {\bf 40}(2): 199--212	

\bibitem{kurokiOwa2011} Kuroki K, Owa S. Notes on new class for certain analytic functions. RIMS Kokyuroku, 2011, {\bf1772}: 21--25

\bibitem{minda94}  Ma W C and Minda D. A unified treatment of some special classes of univalent functions. {\it Proceedings of the Conference on Complex Analysis, Tianjin}, Conf Proc Lecture Notes Anal, I Int Press, Cambridge, MA, 1992:  157--169

\bibitem{mc}  MacGregor T H. Majorization by univalent functions. Duke Math J, 1967, {\bf 34}: 95--102

\bibitem{T.H.Mac1973}  MacGregor T H. Hull subordination and extremal problems for starlike and spirallike mappings. Trans Amer Math Soc, 1973, {\bf 183}: 499--510

\bibitem{mendi2exp}  Mendiratta R, Nagpal S and Ravichandran V. On a subclass of strongly starlike functions associated with exponential function. Bull Malays Math Sci Soc, 2015, {\bf 38}(1): 365--386

\bibitem{sub-pg132}  Miller S S and Mocanu P T. {\it Differential subordinations}, Monographs and Textbooks in Pure and Applied Mathematics. New York: Marcel Dekker, Inc, 2000

\bibitem{raina2015} Raina R K and {Sok\'{o}\l} J. Some properties related to a certain class of starlike functions. C R Math Acad Sci Paris, 2015, {\bf 353}(11): 973--978

\bibitem{rush-sheil-1973} Ruscheweyh S and Sheil-Small T. Hadamard products of Schlicht functions and the P\'{o}lya-Schoenberg conjecture. Comment Math Helv, 1973, {\bf 48}: 119--135

\bibitem{naveen14} Sharma K,  Jain N K and Ravichandran V. Starlike functions associated with a cardioid. Afr Mat, 2016, {\bf 27}(5-6): 923--939

\bibitem{sokol1996}  {Sok\'{o}\l} J and Stankiewicz J. Radius of convexity of some subclasses of strongly starlike functions. Zeszyty Nauk Politech Rzeszowskiej Mat, 1996, (19): 101--105

\bibitem{tang}  Tang H and Deng G. Majorization problems for some subclasses of starlike functions. J Math Res Appl, 2019, {\bf 39}(2): 153--159. 

%\bibitem{uralegaddi}  Uralegaddi B A,  Ganigi M D and  Sarangi S M. Univalent functions with positive coefficients. Tamkang J Math, 1994, {\bf 25}(3): 225--230

\bibitem{tilt}  Wang L M. The tilted Carath\'{e}odory class and its applications. J Korean Math Soc, 2012, {\bf49}:  671--686

\end{thebibliography}
\end{document}	