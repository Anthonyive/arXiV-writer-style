\documentclass [11pt]{amsart}
\usepackage{amsmath}
\usepackage{amssymb}
\usepackage{comment}
\usepackage{tikz-cd}
\renewcommand{\baselinestretch}{1.2}
\newtheorem{Def}{Definition}[section]
\newtheorem{Thm}{Theorem}[section]
\newtheorem{rem}{Remark}[section]
\newtheorem{Cor}{Corollary}[section]
\newtheorem{Lem}{Lemma}[section]
\newcommand{\R}{\mathbb R}
\newcommand{\C}{\mathbb C}
\newcommand{\T}{\mathbb T}
\newcommand{\N}{\mathbb N}
\newcommand{\Q}{\mathbb Q}
\newcommand{\Z}{\mathbb Z}
\newcommand{\la}{\lambda}
\newcommand{\ra} {\rightarrow}
\newcommand{\txt} {\textmd}
\newcommand{\ds} {\displaystyle}
\newcommand{\be} {\begin{equation}}
\newcommand{\ee} {\end{equation}}
\newcommand{\bes} {\begin{equation*}}
\newcommand{\ees} {\end{equation*}}
\newcommand{\bea} {\begin{eqnarray}}
\newcommand{\eea} {\end{eqnarray}}
\newcommand{\beas} {\begin{eqnarray*}}
\newcommand{\eeas} {\end{eqnarray*}}
\usepackage{array}
\usepackage{tabularx,multicol}
\usepackage{enumerate}
\usepackage{mathrsfs}
\date{}
\usepackage[margin=1in]{geometry}
\setlength\extrarowheight{2pt}
\numberwithin{equation}{section}




%%% ----------------------------------------------------------------------
\begin{document}

\title[\tiny{Completeness of exponentials and Beurling's Theorem on $\R^n$ and $\T^n$}]{Completeness of exponentials and Beurling's Theorem regarding Fourier transform on $\R^n$ and $\T^n$}

\author{\tiny{ Santanu Debnath and Suparna Sen}}

\address{Department of Pure Mathematics, University of Calcutta, India.}

\email{ santanudebnath1804@gmail.com , suparna29@gmail.com }


%%% ----------------------------------------------------------------------

\begin{abstract}
A classical result of A. Beurling gives a relation between the decay of a complex Borel measure on $\R$ and the vanishing set of its Fourier transform. We prove several variable analogues of this result on the Euclidean space $\R^n$ and the $n$-dimensional torus $\T^n.$ We also prove some results on the well known weighted approximation problem of exponentials on $\R^n$ and $\T^n$ by establishing an equivalence with Beurling's theorem. 
 \end{abstract}

\subjclass[2010]{Primary 22E30; Secondary 30B60, 46E15}

\keywords{Fourier Transform, Beurling's Theorem, Completeness of Exponentials}

%%% ----------------------------------------------------------------------
\maketitle
%%% ----------------------------------------------------------------------
\section{Introduction}

Several classical results (referred as uncertainty principles in general) in harmonic analysis  deal with the phenomenon that a function on the real line and its Fourier transform cannot simultaneously be `small', for example if one decays rapidly at infinity then the other cannot vanish on a `large set' unless both vanish identically. As a manifestation of this fact, we note the following: if a function $f$ on $\R$ satisfies the estimate 
$$ |f(x)| \leq e^{-|x|}, \quad \txt{ for all } x \in \R,$$
and its Fourier transform $\widehat{f}$ vanishes on a set of positive Lebesgue measure, then $f$ is zero almost everywhere. This is due to the fact that here $\widehat{f}$ turns out to be holomorphic on a domain containing the real line because of the rapid decay on $f.$ The same conclusion holds if we interchange the conditions on $f \in L^1(\R)$ and $\widehat{f}.$ So it is natural to study the interrelation of the optimal nature of the set on which one vanishes and the allowable decay of the other. 

Plenty of classical literature is available on this \cite{B, I, K, L1, PW} which studied results of this kind on the real line $\R.$ Among these results, perhaps the most general one is due to Beurling \cite{B, K}: 
 \begin{Thm}[Beurling]\label{ber-th}
 	Let $\mu$ be a complex Borel measure on $\R$ such that 
 	\begin{equation*}\label{eq;1}
 	\int_0^{\infty}\dfrac{1}{1+x^2}\log\left(\dfrac{1}{\int_x^{\infty}|d\mu(t)|}\right)dx=\infty,
 	\end{equation*}
where $|d\mu(t)| = d|\mu|(t),$ $|\mu|$ being the total variation of the measure $\mu$. If the Fourier transform $\widehat{\mu}$ of $\mu$ vanishes on a set $\Lambda$ of positive Lebesgue measure in $\R,$ then $\mu $ is identically zero 
 \end{Thm}
\noindent As a corollary of Beurling's Theorem on $\R$ (Theorem \ref{ber-th}), we also have the following result (see \cite{B, K}) on the circle group $\T$:
\begin{Thm}[Beurling]\label{ber-th-c}
Let $f\in L^2(\T)$ satisfy
\begin{equation*}\label{eq;7c}
\sum_{k=0}^\infty\dfrac{1}{1+k^2} \log\left(\dfrac{1}{\sum_{m=k}^\infty |\widehat{f}(m)|^2}\right)=\infty,
\end{equation*}
where $\widehat{f}(k)$ denote the Fourier coefficients of $f.$ If $f$ vanishes on a set of positive Lebesgue measure in $\T,$ then $f$ is zero almost everywhere.
\end{Thm}

Recently the classical results due to Ingham, Levinson and Paley-Wiener have been extended to the Euclidean space $\R^n$ and the torus $\T^n$ in \cite{BRS}. Along similar lines, we have proved several variable analogues of Beurling's Theorem on $\R^n$ (Theorem \ref{several beurling}) and $\T^n$ (Theorem \ref{sev-fn-c}) in this paper. We note here that Beurling's result can be considered as a straightforward generalization of Levinson's result, which considers functions vanishing on open sets instead of positive Lebesgue measure sets in $\R.$ We recall that the approach to the extension of Levinson's theorem on $\R^n$ (see \cite{BRS}) and some other noncommutative setting relies heavily on an alternative proof suggested in \cite{K} where it was shown that the theorem of Levinson can also be obtained as a consequence of completeness of linear span of exponentials in certain normed linear space of continuous functions. This leads us to the famous weighted approximation problem regarding completeness of exponentials.

Several versions of this problem and its multiple reformulations have been studied throughout the last century by many prominent mathematicians including Akhiezer, Bernstein, de Branges, Krein, Koosis, Levinson, Mergelyan and many others. This is still a very active area of research with more recent significant contributions by Bakan, de Jeu, Poltoratski among others. Besides the inherent beauty of the original problem, such an extensive interest is largely because of its numerous links with other areas of classical analysis such as spectral problems for differential operators, gap and density problems, type problem etc. There is still a lot to be explored in these areas and the connections between them.

In the study of weighted approximation problem on $\R^n,$ we start with a function $\psi:[0,\infty) \rightarrow [0,\infty)$ such that $\psi(x)\to \infty$ as $x\to \infty.$ We then define a weighted space of continuous functions $$C_{\psi}(\R^n)=\left\lbrace f : \R^n \to \C ~ : ~ f \txt{ is continuous and } \lim_{|x|\to \infty}\dfrac{f(x)}{e^{\psi(|x|)}}=0\right\rbrace,$$
equipped with the weighted uniform norm 
$$\|f\|_{\psi}=\sup_{x \in \R^n}\dfrac{|f(x)|}{e^{\psi(|x|)}},\quad \txt{ for } f\in C_{\psi}(\R^n).$$ For $\Lambda \subset \R^n,$ we consider the linear span of the exponential functions given by 
$$\Phi_{\Lambda}(\R^n)= span \lbrace e_{\la}:\la \in \Lambda \rbrace,$$ where $e_{\la}(x)=e^{-i\la \cdot x}$ for $x \in \R^n.$ A version of the weighted approximation problem deals with the conditions on $\Lambda\subset \R^n$ and $\psi,$ for which the space of exponentials $\Phi_{\Lambda}(\R^n)$ is dense in $C_{\psi}(\R^n).$ Another version of this problem is to study approximation using $L^p$ norms with respect to a finite positive measure $\mu.$ In this version, one studies conditions on $\Lambda\subset \R^n$ and $\mu$ that ensure completeness, that is, density of $\Phi_{\Lambda}(\R^n)$ in $L^p(\R^n,\mu).$ For some results of this type, see \cite{DJ, K, P}. Similar questions may arise in the context of the torus $\T^n.$  We have proved some results of this genre on $\R^n$ (Theorem \ref{several density} and Theorem \ref{Lp dense}) and $\T^n$ (Theorem \ref{sev-den-c}) in this paper.

In \cite{BRS, K}, we have seen that the theorem of Levinson was obtained as a consequence of the density of $\Phi_\Lambda$ (for an open set $\Lambda$) in $(C_\psi, \|\cdot\|_\psi)$ (for $\psi$ increasing and satisfying certain integrability condition). So while attempting to prove a several variable version of Beurling's Theorem, it is natural to seek an analogous exponential density result for a set $\Lambda$ of positive Lebesgue measure in $\R.$ Quite surprisingly, we have been able to prove such an exponential density result (Theorem \ref{density}) as a consequence of Beurling's result on $\R$ (Theorem \ref{ber-th}). This also establishes an equivalence between the above two problems of different genre. In fact, both of these results are true for more general $\Lambda,$ that is, a set $\Lambda$ whose closure $\overline{\Lambda}$ is of positive Lebesgue measure in $\R.$ Similar ideas have been used to prove analogous results in $\R^n$ and $\T^n.$ 

This paper is organised as follows: The next section is devoted to the results on $\R^n.$ First, Beurling's theorem on $\R$ is restated (Theorem \ref{psi-ber}) in a way similar to Levinson's Theorem along with a short proof. Next, an exponential density result (Theorem \ref{density}) is proved on $\R$ using the explicit definition of the dual of $(C_\psi, \|\cdot\|_\psi)$ calculated in Lemma \ref{dual lem}. Following standard techniques, the exponential density result is extended to $\R^n$ (Theorem \ref{several density}). Next, following a well-known result by A. Bakan, another version of an exponential density result (Theorem \ref{Lp dense}) is proved on $\R^n.$ Thereafter, Beurling's theorem on $\R^n$ (Theorem \ref{several beurling}) is proved using Lemma \ref{dual lem} and Theorem \ref{several density}. As a consequence of Theorem \ref{Lp dense}, another version of Beurling's result (Theorem \ref{beurlinglp}) is also obtained. In the last section, the analogous results on $\T^n$ are proved. After restating Beurling's theorem on $\T$ (Theorem \ref{psi-ber-c}), the dual of the space of sequences $(c_\psi(\Z^n), \|\cdot\|_\psi)$ is calculated in Lemma \ref{dual-lem-c}. Using this, an exponential density result (Theorem \ref{density-c}) is proved on $\T.$ Thereafter, the exponential density result is extended to $\T^n$ (Theorem \ref{sev-den-c}). Next, a result (Lemma \ref{positive rectangle type}) on positive rectangle type sets (to be defined subsequently) is proved. Finally, Beurling's Theorem on $\T^n$ (Theorem \ref{sev-fn-c}) is proved using Lemma \ref{dual-lem-c}, Lemma \ref{positive rectangle type} and Theorem \ref{sev-den-c}. 

We will adopt the following notation and convention throughout this paper: 
The Lebesgue measure of a set $E\subset \R$ is denoted by $m(E)$ but for convenience $dx$ will be used instead of $dm(x).$ For a Banach space $X$ endowed with a norm $\|\cdot\|,$ the dual of $X$ is defined to be the set of all bounded linear functionals on $X$ denoted by $(X,\|\cdot\|)^*.$ The space of all smooth functions on $X$ is denoted by $C^\infty(X)$ and $C_c^\infty(X)$ denotes the space of all smooth, compactly supported functions on $X.$ The open ball of radius $l>0$ centered at $a$ in $\R^n$ is denoted by $B(a,l).$ For $x, y \in \R^n,$ $|x|$ is used to denote the Euclidean norm of the vector $x$ and $x \cdot y$ to denote the Euclidean inner product of the vectors $x$ and $y.$

\section{On the Euclidean space $\R^n$}
This section is broadly divided into two subsections. In the first subsection our aim is to prove some results on density of exponentials in certain Banach spaces of functions on $\R^n$ (Theorem \ref{several density} and Theorem \ref{Lp dense}). We will use Beurling's result on $\R$ (Theorem \ref{ber-th}) for proving these results. In the second subsection, we will use the exponential density results from the first subsection to prove several variable analogues (Theorem \ref{several beurling} and Theorem \ref{beurlinglp}) of Beurling's one-variable result (Theorem \ref{ber-th}). In this way the results on density of exponentials are equivalent to Beurling's theorem.

First we will restate Beurling's result in a way similar to Levinson's Theorem (see \cite{BRS}). Here the Fourier transform $\widehat{\mu}$ of a complex Borel measure $\mu$ on $\R$ is defined by  
$$\widehat{\mu}(\la)=\int_{\R} e^{-i\la t}d\mu(t), \text{ for } \la \in \R.$$
\begin{Thm}\label{psi-ber}
Let $\psi:[0,\infty)\rightarrow [0,\infty)$ be an increasing function  and $\mu$ be a complex Borel measure on $\R$ satisfying the following conditions:
\begin{equation}\label{mu decay}
\int_{0}^\infty e^{\psi(|x|)}|d\mu (x)|<\infty,
\end{equation}
\begin{equation}\label{psi cond}
\int_0^{\infty}\dfrac{\psi(x)}{1+x^2}dx=\infty.
\end{equation}
If $\widehat{\mu}$ vanishes on a set $\Lambda \subset \R$ such that $\overline{\Lambda}$ is of positive Lebesgue measure in $\R,$ then $\mu $ is identically zero.
\end{Thm}
\begin{proof} Since  $\psi$ is increasing on $[0,\infty)$ we have for all $x\geq 0,$
	$$e^{\psi(x)}\int_x^\infty|d\mu(t)| \leq \int_{0}^\infty e^{\psi(|t|)}|d\mu (t)|.$$
From (\ref{mu decay}), it follows that for all $x \geq 0,$ there exists a constant $C>0$ such that 
 $$\int_x^{\infty} |d\mu(t)| \leq Ce^{-\psi (x)} .$$
Since $\log$ is an increasing function, we get that
$$\int_0^{\infty}\dfrac{1}{1+x^2}\log\left(\dfrac{1}{\int_x^{\infty}|d\mu(t)|}\right)dx \geq C\int_0^{\infty}\dfrac{\psi(x)}{1+x^2}dx dx.$$
So by equation (\ref{psi cond}), we conclude that $$\int_0^{\infty}\dfrac{1}{1+x^2}\log\left(\dfrac{1}{\int_x^{\infty}|d\mu(t)|}\right)dx=\infty.$$
Since $\widehat{\mu}$ is a continuous function, we get that $\widehat{\mu}$ vanishes on $\overline{\Lambda},$ a set of positive Lebesgue measure in $\R.$ Hence the result follows from Theorem \ref{ber-th}.
\end{proof}

\begin{rem}
It is interesting to note that the result is true even for those $\mu,$ for which $\widehat{\mu}$ vanishes on a set $\Lambda$ of zero Lebesgue measure in $\R,$ if its closure $\overline{\Lambda}$ is of positive Lebesgue measure.
\end{rem}

\subsection{Density of Exponentials}

We consider the space of exponentials $\Phi_{\Lambda}(\R^n)$ and the weighted space of continuous functions $C_\psi(\R^n)$ defined in the introduction. Since we assume that $\psi(x)\to \infty $ as $x\to \infty,$ it is easy to see that for any $\Lambda \subset \R^n,$ $\Phi_{\Lambda}(\R^n)$ is a subspace of $C_{\psi}(\R^n).$ For a finite positive measure $\mu$ on $\R^n,$ $\Phi_{\Lambda}(\R^n)$ is also a subspace of $L^p(\R^n, \mu),$ the space of all $L^p$ functions on $\R^n$ with respect to the measure $\mu.$ It is a well known problem in harmonic analysis to study for which $\Lambda\subset \R^n$ the space of exponentials $\Phi_{\Lambda}(\R^n)$ is dense in certain Banach spaces. We will prove two such results (Theorem \ref{several density} and Theorem \ref{Lp dense}) in this subsection, for the Banach spaces $C_\psi(\R^n)$ and $L^p(\R^n, \mu).$

We will calculate $(C_{\psi}(\R^n),\|\cdot \|_{\psi})^*$ explicitly for continuous $\psi$ in order to prove our result. Let $C_0(\R^n)$ denote the Banach space of all continuous functions on $\R^n$ vanishing at infinity with respect to the supremum norm $\|\cdot\|_{\infty}$ and $\mathcal{M}(\R^n)$ denote the collection of all complex Borel measures on $ \R^n,$ which is the dual of $(C_0( \R^n),\|\cdot\|_{\infty}) .$ Using this fact, we prove the following lemma:
\begin{Lem}\label{dual lem}
If $\psi$ is a non-negative continuous function on $[0,\infty),$ then $(C_{\psi}( \R^n),\|\cdot\|_{\psi})$ is a Banach space isometrically isomorphic to $(C_0( \R^n),\|\cdot\|_{\infty})$ and its dual is given by
 $$(C_{\psi}( \R^n),\|\cdot\|_{\psi})^*=\left\lbrace\beta \in \mathcal{M}( \R^n) : 
 \int_{ \R^n}e^{\psi(|x|)}|d\beta(x)|<\infty \right\rbrace .$$
\end{Lem}
\begin{proof}
For $f \in C_0( \R^n),$ using the continuity of $\psi,$ we define a function $f_\psi \in C_{\psi}( \R^n)$ by 
\beas f_\psi(x)=f(x)e^{\psi(|x|)}, \quad \txt{ for } x \in  \R^n.\eeas
Thus we get a bijective linear map $A_\psi: C_0( \R^n) \to C_\psi( \R^n)$ such that $A_\psi(f) = f_\psi.$ It is easy to see that the map $A_\psi$ is also an isometry from the Banach space $(C_0( \R^n),\|\cdot\|_{\infty})$ onto $(C_{\psi}( \R^n),\|\cdot\|_{\psi}).$ From this, we get a natural bijection between the respective dual spaces, given by 
\beas 
(C_{\psi}( \R^n),\|\cdot\|_{\psi})^* &\to & (C_0( \R^n),\|\cdot\|_{\infty})^* \\
T & \mapsto & T \circ A_\psi.
\eeas
We know that, $(C_0( \R^n),\|\cdot\|_\infty)^*=\mathcal{M}( \R^n)$ in the sense that, corresponding to a given bounded linear functional $T_0$ on $C_0( \R^n)$ there is a unique $\alpha_0 \in \mathcal{M}( \R^n)$ such that 
\bes T_0(f)=\int_{ \R^n}f(t)~ d\alpha_0(t), \quad \txt{ for all } f\in C_0( \R^n).\ees
It follows that for $T\in (C_{\psi}( \R^n),\|\cdot\|_{\psi})^*,$ we get $\alpha \in \mathcal{M}( \R^n)$ such that 
\bes 
T(g) = (T \circ A_\psi)(g_{-\psi}) = \int_{ \R^n} \frac{g(t)}{e^{\psi(|t|)}}~ d\alpha(t)  =\int_{\R^n} g(t) ~ d\beta(t), 	\quad \quad \txt{ for } g\in C_\psi( \R^n),
\ees
where $g_{-\psi} = A_{\psi}^{-1}(g) \in C_0( \R^n)$ and $\ds{ d\beta(t) = e^{-\psi(|t|)} d \alpha(t)}$ for $t \in \R^n.$ Since $ e^{-\psi(|t|)} \leq 1,$ for all $t \in \R^n,$ it is easy to see that $\beta$ is also a complex Borel measure on $ \R^n$ satisfying $\ds{\int_{ \R^n}e^{\psi(|x|)}|d\beta(x)|<\infty.}$ Moreover, for any $\beta \in \mathcal{M}( \R^n)$ satisfying the given condition, it defines a bounded linear functional on the space $(C_\psi( \R^n), \|\cdot\|_{\psi})$ in a similar manner as above. Hence the lemma is proved.
\end{proof}

First we will use Beurling's result (Theorem \ref{psi-ber}) to prove the following weighted approximation result on $\R:$ 
\begin{Thm}\label{density}
Let $\psi:[0,\infty) \rightarrow [0,\infty)$ be a continuous, increasing function satisfying 
\be \label{intinf} \int_0^{\infty}\dfrac{\psi(x)}{1+x^2}dx=\infty. \ee
For any $\Lambda \subset \R$ such that $\overline{\Lambda}$ is of positive Lebesgue measure in $\R,$ $\Phi_{\Lambda}(\R)$ is dense in $(C_{\psi}(\R),\|\cdot \|_{\psi}).$
\end{Thm}
\begin{proof}
Since $\Phi_{\Lambda}(\R)$ is a subspace of $C_{\psi}(\R),$ in order to prove the result, it is enough to show that for any bounded linear functional $T$ on $(C_{\psi}(\R),\|\cdot \|_{\psi}),$ if $T$ vanishes on the space $\Phi_{\Lambda}(\R),$ then $T$ is identically zero. Let us consider such a $T \in (C_{\psi}(\R),\|\cdot \|_{\psi})^*.$ From Lemma \ref{dual lem}, we get a complex  Borel measure $\beta$ on $\R$ such that
$$T(f)=\int_\R f(t) ~ d\beta(t), \quad \txt{ for } f \in C_{\psi}(\R),$$
where $\beta$ satisfies $$\int_{ \R}e^{\psi(|x|)}|d\beta(x)|<\infty.$$
Since $T$ vanishes on the space $\Phi_{\Lambda}(\R),$ we get that  
$$ \int_\R e^{-i\lambda\cdot t}d\beta(t)=0, \quad \quad ~\forall ~ \lambda \in \Lambda.$$
This implies that the function $\hat{\beta}$ vanishes on the set $\Lambda \subset \R.$ So $\beta$ satisfies all the conditions of Theorem \ref{psi-ber}, from which it follows that $\beta$ is identically zero. Hence we conclude that $ T \equiv 0.$
\end{proof}

We wish to prove an analogue of the above result on $\R^n,$ in a way similar to Lemma $2.4$ in \cite{BRS} which involves open sets in $\R^n.$ We note that an open set $U\subseteq \R^n$ always contains a set of the form $U_1\times U_2 \times \cdots \times U_n$ where each $U_j \subseteq \R$ is open in $\R$ for $1\leq j \leq n.$ This is an important tool in the technique of the proof of Lemma $2.4$ in \cite{BRS}. However, this property does not hold for sets of positive Lebesgue measure in $\R^n$. For example in $\mathbb{R}^2$ consider the set 
$$E=\lbrace (x,y)\in [0,1]\times [0,1] : x-y \notin \mathbb{Q} \rbrace .$$ Since $E$ is a set of positive Lebesgue measure in $\R^2,$ by inner regularity, there exists a compact set $K\subset E,$ also of positive Lebesgue measure. 
It will follow that $K$ cannot contain $\Lambda_1 \times \Lambda_2$ such that $\Lambda_1, \Lambda_2 \subset \mathbb{R}$ with $m(\overline{\Lambda_1}), m(\overline{\Lambda_2})>0.$ If not, then $\overline{\Lambda_1} \times \overline{\Lambda_2} =\overline{\Lambda_1 \times \Lambda_2} \subset K.$ But as $m(\overline{\Lambda_1}), m(\overline{\Lambda_2})>0,$ it is known that $\overline{\Lambda_1} - \overline{\Lambda_2}$ must contain an interval, which contradicts the definition of $E$. It is due to this reason that we are led to incorporate the above property in the following definition.

We call $\Lambda \subset \R^n$ to be `a set of positive rectangle type' if $\Lambda$ contains a set of the form $\Lambda_1 \times \cdots \times \Lambda_n,$ where $\Lambda_j \subset \R$ for each $1 \leq j \leq n$ such that $\overline{\Lambda_j}$ has positive Lebesgue measure in $\R,$ that is, $m(\overline{\Lambda_j})>0.$ In the following result, we will prove that if $\Lambda \subset \R^n$ is of positive rectangle type, then the density of $\Phi_{\Lambda}(\R^n)$ in $(C_{\psi}(\R^n),\|\cdot \|_{\psi})$ is characterized by the integral condition on $\psi:$  
\begin{comment}
\begin{Thm}\label{several density}
Let $\psi:[0,\infty) \rightarrow [0,\infty)$ be a continuous, increasing function satisfying 
$$\int_0^{\infty}\dfrac{\psi(x)}{1+x^2}dx=\infty.$$
For any set $\Lambda \subset \R^n$ of positive rectangle type, the subspace $\Phi_{\Lambda}(\R^n)$ is dense in $(C_{\psi}(\R^n),\|\cdot \|_{\psi}).$

Conversely if for any set $\Lambda \subset \R^n$ of positive rectangle type, the subspace $\Phi_{\Lambda}(\R^n)$ is dense in $(C_{\psi}(\R^n),\|\cdot \|_{\psi})$ where $\psi:[0,\infty) \rightarrow [0,\infty)$ is a continuous, increasing function, then   $$\int_0^{\infty}\dfrac{\psi(x)}{1+x^2}dx=\infty.$$ 
\end{Thm}
\end{comment}


\begin{Thm}\label{several density}
Let $\psi$ be a non-negative, continuous, increasing function on $[0,\infty).$ $\Phi_{\Lambda}(\R^n)$ is dense in $(C_{\psi}(\R^n),\|\cdot \|_{\psi})$ for any positive rectangle type set $\Lambda \subset \R^n$ if and only if 
\be \label{psiint} \int_0^{\infty}\dfrac{\psi(x)}{1+x^2}dx=\infty. \ee
\end{Thm}

\begin{proof} 
First, we will assume (\ref{psiint}) and prove the density of $\Phi_{\Lambda}(\R^n)$ in $(C_{\psi}(\R^n),\|\cdot \|_{\psi}).$
Let us define, 
$$\psi_0 (x)= \dfrac{\psi (x)}{n}, \quad \quad \txt{for } x\in [0,\infty).$$ 
Since $\Lambda$ is of positive rectangle type, there exists a set $\Gamma$ of the form $\Gamma = \Lambda_1 \times \cdots \times \Lambda_n \subseteq \Lambda,$ where $\Lambda_j \subset \R$ for each $1 \leq j \leq n$ such that $m(\overline{\Lambda_j})>0.$ We now consider the following spaces of functions
\beas 
\mathcal{P}\Phi_{\Gamma}(\R^n) &=& \text{span} \left\{ f:\R^n \to \C :  f(x_1, \cdots, x_n)=f_1(x_1)\cdots f_n(x_n), f_j\in \Phi_{\Lambda_j}(\R), 1\leq j \leq n  \right\},\\
\mathcal{P}C_{\psi_0}(\R^n) &=& \text{span} \left\{f:\R^n \to \C :  f(x_1, \cdots, x_n)=f_1(x_1)\cdots f_n(x_n), f_j\in C_{\psi_0}(\R), 1\leq j \leq n  \right\}. 
\eeas
It is easy to see that $\mathcal{P}\Phi_{\Gamma}(\R^n) \subseteq \Phi_{\Lambda}(\R^n)$ from the relation 
$$e_{\lambda_1}(x_1)\cdots e_{\lambda_n}(x_n)=e_{\lambda}(x), \txt{ for }\la = (\la_1, \cdots, \la_n) \in \Lambda_1 \times \cdots \times \Lambda_n \txt{ and }x = (x_1, \cdots, x_n) \in \R^n.$$ It was shown in \cite{BRS} that $ \mathcal{P}C_{\psi_0}(\R^n)$ is dense in $(C_\psi(\R^n),\|\cdot \|_\psi).$ 
Since $\Phi_{\Lambda}(\R^n) \subseteq (C_{\psi}(\R^n),\|\cdot \|_{\psi}),$ in order to prove the result it is enough to prove that $ \mathcal{P}\Phi_{\Gamma}(\R^n)$ is dense in $ (\mathcal{P}C_{\psi_0}(\R^n),\|\cdot \|_\psi).$ We consider functions in $\mathcal{P}C_{\psi_0}(\R^n)$ of the form $$f(x)= f_1(x_1)\cdots f_n(x_n),$$ for $f_j\in C_{\psi_0}(\R),$ for all $1\leq j \leq n.$ Since $m(\overline{\Lambda_j})>0$ for each $j,$ for any $0<\epsilon<1$ applying Theorem \ref{density} we get $g_j\in \Phi_{\Lambda_j}(\R)$ such that for all $1\leq j\leq n$
\begin{equation*}
\sup_{s\in \R}\frac{|f_j(s)- g_j(s)|}{e^{\psi_0(|s|)}}<\epsilon .
\end{equation*}
By triangle inequality we have,
$$ \sup_{s\in \R} \dfrac{|g_j(s)|}{e^{\psi_0(|s|)}}\leq 1+\|f_j\|_{\psi_0}, \quad \txt{ for } 1\leq j \leq n . $$ 
Now we define $g\in \mathcal{P}\Phi_{\Gamma}(\R^n)$ by $$g(x)= g_1(x_1)\cdots g_n(x_n),\quad \txt{ for }  x=(x_1, \cdots , x_n)\in \R^n.$$ By defining $$g_0(y)=e^{\psi_0(|y|)}=f_{n+1}(y),~~ y\in \R,$$ we have for all $x=(x_1,\cdots , x_n)\in \R^n,$
\begin{eqnarray*}
\frac{|f(x)- g(x)|}{e^{\psi(|x|)}} &\leq& \frac{|f_1(x_1)\cdots f_n(x_n)- g_1(x_1)\cdots g_n(x_n)|}{e^{\psi_0(|x_1|)}\cdots e^{\psi_0(|x_n|)}}\\
&\leq & \sum_{k=1}^n\dfrac{|f_k(x_k)-g_k(x_k)|}{e^{\psi_0(|x_k|)}}\left(   \prod_{j=k+1}^{n+1}\dfrac{|f_j(x_j)|}{e^{\psi_0(|x_j|)}}\prod_{j=0}^{k-1}\dfrac{|g_j(x_j)|}{e^{\psi_0(|x_j|)}} \right) \\ 
&\leq & \epsilon ~ n \prod_{j=1}^n (1+\|f_j\|_{\psi_0}) \\ 
&\leq & C\epsilon.
\end{eqnarray*}
This proves the first part of the result.

For the converse part, let us assume that $\Phi_{\Lambda}(\R^n)$ is dense in $(C_{\psi}(\R^n),\|\cdot \|_{\psi})$ for any positive rectangle type set $\Lambda \subset \R^n.$ If possible, let $$\int_0^{\infty}\dfrac{\psi(x)}{1+x^2}dx<\infty.$$ By Theorem 2.6(b) of \cite{BRS}, we get a non-zero $f\in C_c^\infty(\R^n)$ satisfying $$|\widehat{f}(\xi)|\leq Ce^{-\psi(|\xi|)}\quad  \text{ for all } \xi\in \R^n.$$
Again, applying Lemma 2.5 of \cite{BRS}, there exists a non-zero continuous function $F$ on $\R^n$ which vanishes on a non-empty open subset of $\R^n$ and satisfies
\be\label{F cond}
\int_{\R^n}|\widehat{F}(\xi)|e^{\psi(|\xi|)}d\xi<\infty.\ee
We define a non-zero complex Borel measure $\mu$ on $\R^n$ by 
\bes \label{mu} d\mu(x)=\widehat{F}(x)dx. \ees 
By Lemma \ref{dual lem}, it follows from (\ref{F cond}) that $\mu\in (C_\psi(\R^n),\|\cdot\|_\psi)^*.$ Moreover, since both $\widehat{\mu}$ and $F$ are continuous functions on $\R^n,$ $\widehat{\mu}$ also vanishes on an open set, say $U \subset \R^n.$
This implies that as a bounded linear functional on $(C_\psi(\R^n),\|\cdot\|_\psi),$ $\mu$ vanishes on $\Phi_U(\R^n).$ Since $U$ is a set of positive rectangle type, using the density of  $\Phi_U(\R^n)$ in $(C_\psi(\R^n),\|\cdot\|_\psi),$ we can conclude that $\mu$ is identically zero, which gives a contradiction. Hence the result is proved.
\end{proof}
 

Now we will prove the density of $\Phi_\Lambda(\R^n)$ in another class of Banach space, $L^p(\R^n,\mu)$ for $1\leq p < \infty,$ the space of $L^p$ functions with respect to $\mu,$  where $\mu$ is a finite positive measure satisfying certain conditions. This is actually a consequence of the above result (Theorem \ref{several density}) along with a well-known result by A. Bakan (see Theorem 14 of \cite{P1}). For the sake of completeness we will provide a direct proof. 
\begin{Thm}\label{Lp dense}
Let $\psi:[0,\infty)\rightarrow [0,\infty)$ be a continuous increasing function such that 
\be\label{p cond} \int_0^\infty\dfrac{\psi(x)}{1+x^2}dx=\infty \ee 
and $\mu$ be a positive measure satisfying 
\be \label{measure estimate} \int_{\R^n} e^{\psi(|x|)}d\mu(x)<\infty. \ee
For any set of positive rectangle type $\Lambda \subset \R^n,$ the space $\Phi_\Lambda(\R^n)$ is dense in $L^p(\R^n,\mu),~ 1\leq p < \infty.$ 
\end{Thm}
\begin{proof}
Since $C_c(\R^n)$ is dense in $L^p(\R^n,\mu),$ for $1\leq p < \infty$ it is enough to prove that $\Phi_\Lambda(\R^n)$ is dense in $C_c(\R^n)$ with respect to the corresponding $\|\cdot\|_p$ norm of $L^p(\R^n,\mu),$ for $1\leq p < \infty.$ We define $$\psi_1(x)=\dfrac{\psi(x)}{p}, \quad \quad  \txt{ for all } x\in [0,\infty).$$ From (\ref{p cond}), it easily follows that $\psi_1$ is a continuous increasing function satisfying 
	$$\int_0^\infty\dfrac{\psi_1(x)}{1+x^2}dx=\infty.$$ 
We consider $f\in C_c(\R^n) \subseteq C_{\psi_1}(\R^n).$ Given any $\epsilon >0,$ from Theorem \ref{several density} we get $g\in \Phi_\Lambda(\R^n)$ such that $\|f-g\|_{\psi_1}<\epsilon.$ From (\ref{measure estimate}) we get $C>0$ such that 
$$\|f-g\|_p^p ~=~ \int_{\R^n}\dfrac{|f(x)-g(x)|^p}{e^{p\psi_1(|x|)}}e^{\psi(|x|)}d\mu(x) ~\leq~ \|f-g\|_{\psi_1}^p \int_{\R^n} e^{\psi(|x|)}d\mu(x) ~<~ C \epsilon^p.$$
Hence the result is proved.
\end{proof}
\begin{rem}
It follows from the proof of Theorem 14 of \cite{P1} that if $\Phi_{\Lambda}(\R^n)$ is dense in $L^p(\R^n,\mu),$ for a finite positive measure $\mu$ on $\R^n,$ then there exists $\psi$ satisfying (\ref{measure estimate}) such that $\Phi_{\Lambda}(\R^n)$ is dense in $(C_\psi(\R^n),\|\cdot\|_\psi).$
\end{rem}

\subsection{Beurling's Theorem}
We will now prove some several variable analogues of Beurling's result on $\R$ (Theorem \ref{psi-ber}). Our proof will follow the ideas used in \cite{BRS} to prove the analogue of Levinson's Theorem. Accordingly, a main ingredient of the proof is the density of a linear span of exponentials $\Phi_{\Lambda}(\R^n)$ in the Banach spaces $(C_{\psi}(\R^n), \|\cdot\|)$ and $L^p(\R^n, \mu),$ which we have already proved in Theorem \ref{several density} and Theorem \ref{Lp dense}. We have the following analogue of Theorem \ref{psi-ber} on $\R^n$:

\begin{Thm}\label{several beurling}
	Let $\psi:[0,\infty) \rightarrow [0,\infty)$ be a continuous, increasing function such that $\psi(x)\to \infty $ as $x\to \infty$ and
	\begin{equation*}
	I = \int_0^\infty \dfrac{\psi(x)}{1+x^2}dx.
	\end{equation*}
	\begin{itemize}
	\item[(a)] Let $\mu $ be a complex Borel measure on $\R^n$ satisfying 
	\begin{equation}\label{eq;4}
	\int_{\R^n}e^{\psi(|x|)}|d\mu(x)|<\infty.
	\end{equation}
	If $\widehat{\mu}$ vanishes on a set $\Lambda \subset \R^n$ of positive rectangle type and $I=\infty,$ then $\mu $ is identically zero.
	\item[(b)] If $I$ is finite, then there exists a non-trivial complex Borel measure $\mu$ on $\R^n$ satisfying (\ref{eq;4}) such that $\widehat{\mu}$ vanishes on a set of positive rectangle type.	
	\end{itemize}
		\end{Thm}

\begin{proof}
First we shall prove (a). Since the complex Borel measure $\mu$ satisfies (\ref{eq;4}), from Lemma \ref{dual lem} we have that $\mu \in (C_\psi(\R^n),\|\cdot\|_\psi)^*,$ that is,  there is a bounded linear functional $T_{\mu}$ on $C_{\psi}(\R^n)$ defined by
	$$T_\mu(g)=\int_{\R^n}g(x)d\mu(x), \quad \txt{for } g\in C_{\psi}(\R^n).$$  
	Since $\widehat{\mu}$ vanishes on $\Lambda \subset \R^n,$ we have 
	$$T_\mu(e_\la) = \widehat{\mu}(\la) = 0, \quad \txt{ for } \la \in \Lambda.$$ Moreover, as $\Phi_{\Lambda}(\R^n)$ is spanned by such $e_\la$'s, it follows that $$T_\mu(\phi) = 0, \quad \txt{ for } \phi \in \Phi_{\Lambda}(\R^n).$$
Since $\Lambda \subset \R^n$ is a set of positive rectangle type, applying Theorem \ref{several density} we get that $T_\mu$ is identically zero on $C_{\psi}(\R^n).$ Hence we can conclude that $\mu$ is identically zero. This proves part (a).

Now we shall prove (b). If $I$ is finite, from Theorem \ref{several density} we obtain a set $\Lambda \subset \R^n$ of positive rectangle type such that the subspace $\Phi_\Lambda(\R^n)$ is not dense in $(C_\psi(\R^n),\|\cdot\|_\psi).$ So there exists a non-zero complex Borel measure $\mu \in (C_\psi(\R^n),\|\cdot\|_\psi)^*$ satisfying equation (\ref{eq;4}) (follows from Lemma \ref{dual lem}) such that $\mu(\Phi_{\Lambda})\equiv 0.$ This implies that $\widehat{\mu}$ vanishes on $\Lambda.$ Hence the theorem is proved.
\end{proof}

It clearly follows that the above result is true for $f\in L^p(\R^n),$ for $1 \leq p \leq 2$ instead of the complex measure $\mu.$ However, using the duality of $L^p$ spaces and Theorem \ref{Lp dense}, in a similar manner as above we get the following analogue of Theorem \ref{psi-ber} on $\R^n$ for $f \in L^p(\R^n,\mu),$ $1 < p \leq 2,$ where $\mu$ is a positive measure satisfying certain condition. We note that since we do not have Theorem \ref{Lp dense} for $p = \infty$ and we have used duality argument in the proof, we do not get the result for $L^1(\R^n,\mu).$ 
\begin{Thm} \label{beurlinglp}
Let $\psi:[0,\infty)\rightarrow [0,\infty)$ be a continuous increasing function such that 
$$\int_0^\infty\dfrac{\psi(x)}{1+x^2}dx=\infty$$ 
and $\mu$ be a positive measure satisfying 
$$ \int_{\R^n} e^{\psi(|x|)}d\mu(x)<\infty.$$
If $f\in L^p(\R^n,\mu),$ for $1<p\leq 2,$ is such that $\widehat{f}$ vanishes on $\Lambda\subset \R^n,$ where $\Lambda$ is a set of positive rectangle type, then $f$ is identically zero. 
\end{Thm}



\section{On the $n$-dimensional torus $\T^n$}

In this section we will prove analogues of Theorem \ref{several density} and Theorem \ref{several beurling} on the $n$-dimensional torus
$$\T^n=\left\{\left(e^{ix_1}, \cdots , e^{ix_n}\right):~ x_1, \cdots ,x_n\in [-\pi,\pi)\right\}.$$ Clearly we can identify $\T^n$ with $[-\pi,\pi)^n$ which is of finite Lebesgue measure in $\R^n.$ As in the case of $\R^n,$ functions vanishing on a set of positive Lebesgue measure inside $\T^n$ certainly makes sense. Moreover, $\T^n$ being a set of finite Lebesgue measure, it is enough to consider functions in $L^1(\T^n).$ For $f\in L^1(\T^n),$ we define its Fourier coefficients by the formula,
$$ \widehat{f}(k)= \frac{1}{(2\pi)^n}\int_{\T^n}f(x)e^{-ix\cdot k}dx, \text{ for } k\in \Z^n.$$ In order to prove these results, we need to proceed as in the previous section. First, in a similar way as $\R^n,$ we will restate Theorem \ref{ber-th-c}:

\begin{Thm}\label{psi-ber-c}
Let $f\in L^2(\T)$ and $\psi$ be an increasing non-negative function on $\N \cup \{0\}$ satisfying the following conditions:
\beas
\sum_{k=0}^\infty|\widehat{f}(k) ~ e^{\psi(k)} |^2 &<& \infty,  \\
\sum_{k=0}^\infty\dfrac{\psi(k)}{1+k^2} &=& \infty.
\eeas
If $f$ vanishes on a set of positive Lebesgue measure in $\T,$ then $f$ is zero almost everywhere.
\end{Thm}

\subsection{Density of Exponentials}

For any function $\psi:[0,\infty) \rightarrow [0,\infty)$ such that $\psi(x)\to \infty $ as $x\to \infty,$ we consider the space of sequences over $\Z^n$ defined by $$c_{\psi}(\Z^n)=\left\lbrace (a_k)_{k\in \Z^n} : \lim_{|k|\to \infty}\dfrac{a_k}{e^{\psi(|k|)}}=0\right\rbrace .$$ It is easy to see that $(c_{\psi}(\Z^n),\|\cdot \|_{\psi})$ is a normed linear space where 
$$\|A\|_{\psi}=\sup_{k\in\Z^n}\dfrac{|a_k|}{e^{\psi(|k|)}}, \quad \txt{ for } A=(a_k)_{k\in \Z^n}\in c_{\psi}(\Z^n).$$ 
We define the Banach space $(c_0(\Z^n),\|\cdot\|_{\infty})$ of sequences vanishing at infinity by 
$$ c_0(\Z^n)=\ds{\left\lbrace (a_k)_{k\in\Z^n} : \lim_{|k|\to \infty}a_k=0 \right\rbrace }.$$
As in the case of $\R^n,$ we will need to calculate the dual of $(c_{\psi}(\Z^n),\|\cdot \|_{\psi})$ explicitly. 

\begin{Lem}\label{dual-lem-c}
	$(c_{\psi}(\Z^n),\|\cdot\|_{\psi})$ is a Banach space isometrically isomorphic to $(c_0(\Z^n),\|\cdot\|_{\infty})$ and its dual is given by
	$$(c_{\psi}(\Z^n),\|\cdot\|_{\psi})^*=\left\lbrace (d_k)_{k\in \Z^n} \in l^1( \Z^n) :
	\sum_{k\in\Z^n}e^{\psi(|k|)}|d_k|<\infty \right\rbrace .$$
\end{Lem}
\begin{proof}
First, we define a bijective linear isometry $T_\psi: (c_0(\Z^n),\|\cdot\|_{\infty}) \to (c_{\psi}(\Z^n),\|\cdot\|_{\psi})$ such that 
$$T_\psi(A) = A^\psi = (a^\psi_k)_{k\in\Z^n}, \quad \txt{ for } A = (a_k)_{k\in\Z^n} \in c_0(\Z^n),$$ 
where $a^\psi_k=e^{\psi(|k|)}a_k$ for all $k\in \Z^n.$ So we get a natural bijection between the respective dual spaces, given by 
\beas 
(c_{\psi}(\Z^n),\|\cdot\|_{\psi})^* &\to & (c_0(\Z^n),\|\cdot\|_{\infty})^* \\
L & \mapsto & L \circ T_\psi.
\eeas
It is well known that the dual of $(c_0(\Z^n),\|\cdot\|_{\infty}) $ is $(l^1(\Z^n),\|\cdot\|_{1})$ where
$$ l^1(\Z^n)=\left\{ (a_k)_{k\in\Z^n} : \sum_{k\in\Z^n}|a_k|<\infty \right\},$$
that is, given any bounded linear functional $L_0$ on $c_0(\Z^n),$ we get $C=(c_k)_{k\in\Z^n}\in l^1( \Z^n)$ such that 
\bes L_0(A)=\sum_{k\in\Z^n}a_kc_k, \quad \txt{ for } A=(a_k)_{k\in\Z^n}\in c_0(\Z^n).\ees
Thus for $L\in (c_{\psi}(\Z^n),\|\cdot\|_{\psi})^*,$ we get $C=(c_k)_{k\in\Z^n}\in l^1(\Z^n)$ such that 
\bes 
L(B)=(L\circ T_\psi)(B^{-\psi})=\sum_{k\in \Z^n}b^{-\psi}_kc_k =\sum_{k\in\Z^n}b_k e^{-\psi(|k|)} c_k = \sum_{k\in\Z^n}b_k d_k, \quad \txt{ for } B=(b_k)_{k\in\Z^n}\in c_\psi(\Z^n),
\ees
where $$B^{-\psi}=(b^{-\psi}_k)_{k\in\Z^n} = (e^{-\psi(|k|)}b_k)_{k\in\Z^n} \in c_0(\Z^n),$$
$$(d_k)_{k\in\Z^n} = (e^{-\psi(|k|)}c_k)_{k\in\Z^n}.$$ So $D =(d_k)_{k\in\Z^n} \in l^1( \Z^n)$ satisfies $\ds{\sum_{k\in\Z^n}e^{\psi(|k|)}|d_k|<\infty.}$ Again, for $D=(d_k)_{k\in\Z^n} \in l^1(\Z^n)$ satisfying the given condition, it defines a bounded linear functional on the space $(c_\psi( \Z^n), \|\cdot\|_{\psi})$ in a similar manner as above. Hence the lemma is proved.	
\end{proof}

Given $\Lambda \subset \T^n,$ we consider the linear span of exponentials given by 
$$\Phi_{\Lambda}(\Z^n)= span \lbrace E_{\la}: \la \in \Lambda \rbrace, \quad \txt{ where } E_\lambda=\left(e^{i\la \cdot k}\right)_{k\in\Z^n}.$$ 
We will now prove the density of the subspace $\Phi_{\Lambda}(\Z)$ in $c_{\psi}(\Z)$ for certain $\Lambda \subset \T.$
\begin{Thm}\label{density-c}
	Let $\psi:[0,\infty) \rightarrow [0,\infty)$ be an increasing function satisfying 
	\bes \sum_{k=0}^\infty\dfrac{\psi(k)}{1+k^2} = \infty. \ees
	For any $\Lambda \subset \T$ such that $\Lambda$ is of positive Lebesgue measure, $\Phi_{\Lambda}(\Z)$ is dense in $(c_{\psi}(\Z),\|\cdot \|_{\psi}).$
\end{Thm}
 
\begin{proof}
	We consider a bounded linear functional $L$ on $(c_{\psi}(\Z),\|\cdot \|_{\psi})$ which vanishes on the space $\Phi_{\Lambda}(\Z).$ From Lemma \ref{dual-lem-c}, we get a sequence $D=(d_k)_{k\in\Z} \in l^1(\Z)$ satisfying 
\be \label{dk} \ds{\sum_{k\in\Z}e^{\psi(|k|)}|d_k|<\infty} \ee such that
	\be \label{dualrel}  L(B)=\sum_{k\in\Z} b_k~d_k, \quad \txt{ for } B=(b_k)_{k\in\Z} \in c_{\psi}(\Z). \ee
Since $l^1(\Z)\subset l^2(\Z),$ we have $D \in l^2(\Z).$ So we get $g \in L^2(\T)$ such that $\widehat{g}(k) = d_k$ for all $k \in \Z.$ Since $L$ vanishes on $\Phi_{\Lambda}(\Z),$ using (\ref{dualrel}) we have $$L(E_\la) = \sum_{k\in\Z} e^{i\lambda \cdot k}d_k=0, \quad \txt{ for all } \lambda\in \Lambda .$$ It follows that $g(\la) = 0$ for almost every $\la \in \Lambda,$ a set of positive Lebesgue measure inside $\T.$ Moreover, from (\ref{dk}) we get that  
    $$ \sum_{k=0}^\infty e^{2\psi(k)}|\widehat{g}(k)|^2<\infty .$$
So we can apply Theorem \ref{psi-ber-c} to $g,$ to conclude that $g\equiv 0.$ It follows that $d_k=0,\text{ for all } k \in \Z.$ Hence $L$ is identically zero and the result follows.
\end{proof}

\vspace{0.1in}

Now, we will look at the $n$-dimensional case. By slight abuse of notation, we call $\Lambda \subset \T^n$ to be `a set of positive rectangle type' if there exists $\Lambda_1,\cdots ,\Lambda_n\subset \T$ such that $\Lambda_1\times \cdots \times \Lambda_n \subseteq \Lambda$ and $m(\Lambda_j)>0$ for each $j.$ We will prove that the density of  $\Phi_\Lambda(\Z^n)$ in $(c_\psi(\Z^n),\|\cdot\|_\psi)$ is also characterized by the integrability condition on $\psi:$

\vspace{0.1in}
\begin{Thm}\label{sev-den-c}
Let $\psi$ be a non-negative increasing function on $[0,\infty).$ $\Phi_\Lambda(\Z^n)$ is a dense subspace in $(c_\psi(\Z^n),\|\cdot\|_\psi)$ for any set $\Lambda \subset \T^n$ of positive rectangle type if and only if 
	\be \label{psiinf} \sum_{k=0}^\infty\dfrac{\psi(k)}{1+k^2} = \infty. \ee
\end{Thm}
\begin{proof}
First we will prove the density of $\Phi_\Lambda(\Z^n)$ in $(c_\psi(\Z^n),\|\cdot\|_\psi)$ assuming (\ref{psiinf}). We will reduce the problem to the case $n=1$ and then apply Theorem \ref{density-c} to get the result. Let us define, for $x\in [0,\infty),$ $$\psi_0 (x)= \dfrac{\psi (x)}{n}.$$ Since $\Lambda$ is of positive rectangle type, there exists a set $\Gamma$ of the form $\Gamma = \Lambda_1 \times \cdots \times \Lambda_n \subseteq \Lambda,$ where $\Lambda_j \subset \T$ for each $1 \leq j \leq n$ such that $m(\Lambda_j)>0.$

We consider the space $c_{00}(\Z^n)$ of eventually zero sequences over $\Z^n$ given by  
\bes c_{00}(\Z^n)=\left\{ (a_k)_{k\in \Z^n}: \exists m \geq 0 \txt{ such that } a_k=0 \txt{ for } |k|>m \right\}. \ees 
Now, we will define the following spaces of functions
\beas 
\mathcal{P}\Phi_{\Gamma}(\Z^n) &=& \text{span}\left\{ (e_k)_{k  =(k_1, \cdots, k_n) \in \Z^n} :  e_k=e^1_{k_1}\cdots e^n_{k_n}, (e^j_{m})_{m\in \Z}\in \Phi_{\Lambda_j}(\Z), 1\leq j \leq n  \right\} \subseteq  \Phi_{\Lambda}(\Z^n), \\
\mathcal{P}c_{\psi_0}(\Z^n) &=& \text{span} \left\{ (b_k)_{k  =(k_1, \cdots, k_n) \in \Z^n} :  b_k=b^1_{k_1}\cdots b^n_{k_n}, (b^j_{m})_{m\in \Z}\in c_{\psi_0}(\Z), 1\leq j \leq n  \right\} \subseteq  c_\psi(\Z^n), \\
\mathcal{P}c_{00}(\Z^n) &=& \text{span} \left\{ (a_k)_{k =(k_1, \cdots, k_n) \in \Z^n} :  a_k=a^1_{k_1}\cdots a^n_{k_n}, (a^j_m)_{m\in \Z}\in c_{00}(\Z), 1\leq j \leq n  \right\} \subseteq  c_{00}(\Z^n).
\eeas
We consider the following diagram for our convenience:
\begin{center}
	\begin{tikzcd}
	\mathcal{P}\Phi_{\Gamma}(\Z^n)\arrow[r, "\tau_1"] \arrow[d, "j_1"]
	& \Phi_{\Lambda}(\Z^n) \arrow[d, "i_1"] \\
	\mathcal{P}c_{\psi_0}(\Z^n) \arrow[r, "\tau"]
	& c_\psi(\Z^n)\\
	\mathcal{P}c_{00}(\Z^n)\arrow[u, "j_2"]\arrow[r, "\tau_2"]
	& c_{00}(\Z^n)\arrow[u, "i_2"]
	\end{tikzcd} 
\end{center}

Since $\Gamma \subset \Lambda,$ the inclusion $\tau_1: \mathcal{P}\Phi_{\Gamma}(\Z^n) \ra \Phi_{\Lambda}(\Z^n)$ is easy to see. It follows from the definition of $c_{00}(\Z^n)$ that the inclusion $\tau_2: \mathcal{P}c_{00}(\Z^n) \ra c_{00}(\Z^n)$ is actually an identity. Now, we will prove that $\tau : \mathcal{P}c_{\psi_0}(\Z^n) \ra c_{\psi}(\Z^n)$ is an inclusion. Since $\psi_0$ is increasing, for any $(b_k)_{k =(k_1, \cdots, k_n) \in \Z^n}\in \mathcal{P}c_{\psi_0}(\Z^n),$ we have
$$\dfrac{|b_k|}{e^{\psi(|k|)}}=\dfrac{|b^1_{k_1}|\cdots |b^n_{k_n}|}{e^{n\psi_0(|k|)}}\leq \dfrac{|b^1_{k_1}|}{e^{\psi_0(|k_1|)}}\cdots \dfrac{|b^n_{k_n}|}{e^{\psi_0(|k_n|)}}.$$
If $|k|\to \infty$ for $k \in \Z^n,$ there is $1\leq p \leq n$ such that $|k_p|\to \infty.$ Since $(b^p_{m})_{m\in \Z}\in c_{\psi_0}(\Z),$ we get that 
$$\ds{\lim_{|k_p|\to \infty}\dfrac{|b^p_{k_j}|}{e^{\psi_0(|k_p|)}}=0}.$$ Also, $\ds{\left(\dfrac{|b^j_{m}|}{e^{\psi_0(|m|)}}\right)_{m\in \Z}}$ are bounded for all $1\leq j \leq n.$ So we conclude that $(b_k)_{k \in \Z^n}\in c_{\psi}(\Z^n).$

Modifying the standard argument used to prove that $c_{00}$ is dense in $(c_0, \|\cdot\|_\infty),$ it is easy to see that $c_{00}(\Z^n)$ is dense in $(c_{\psi}(\Z^n),\|\cdot \|_{\psi}).$ Moreover, since $c_{00}(\Z^n)=\mathcal{P}c_{00}(\Z^n)\subset \mathcal{P}c_{\psi_0}(\Z^n) \subset c_\psi(\Z^n),$ it follows that $\mathcal{P}c_{\psi_0}(\Z^n)$ is dense in $(c_\psi(\Z^n),\|\cdot\|_\psi).$ 

In order to prove the result that $\Phi_\Lambda(\Z^n)$ is dense in $(c_\psi(\Z^n),\|\cdot\|_\psi),$ it remains to show that $\mathcal{P}\Phi_{\Gamma}(\Z^n)$ is dense in $(\mathcal{P}c_{\psi_0}(\Z^n),\|\cdot\|_\psi).$ For this, it is enough to consider $B = (b_k)_{k \in \Z^n} \in \mathcal{P}c_{\psi_0}(\Z^n)$ of the form $$b_k=b^1_{k_1}\cdots b^n_{k_n},\quad \txt{ for }  k=(k_1, \cdots , k_n)\in \Z^n,$$
where $ B^j=(b^j_m)_{m\in\Z}\in c_{\psi_0}(\Z)$ for $1\leq j \leq n.$ Since $m(\Lambda_j)>0$ for each $j$ and (\ref{psiinf}) is true for $\psi_0$ as well, applying Theorem \ref{density-c} we get that $\Phi_{\Lambda_j}(\Z)$ is dense in $(c_{\psi_0}(\Z), \|\cdot\|_{\psi_0}).$ So for any $0<\epsilon<1,$ we get $E^j=(e^j_m)_{m\in\Z}\in \Phi_{\Lambda_j}(\Z)$  such that
\begin{equation*}
\|E^j - B^j\|_{\psi_0} = \sup_{m \in \Z} \frac{|b^j_{m}- e^j_{m}|}{e^{\psi_0(|m|)}}<\epsilon, \quad \txt{ for each } 1\leq j\leq n. 
\end{equation*}
By triangle inequality we have,
$$ \|E^j\|_{\psi_0} < 1+\|B^j\|_{\psi_0}, \quad \txt{ for each } 1\leq j \leq n . $$ 
Now we define $E=(e_k)_{k \in \Z^n} \in \mathcal{P}\Phi_{\Gamma}(\Z^n)$ by
\beas e_k &=& e^1_{k_1}\cdots e^n_{k_n},\quad \txt{ for }  k=(k_1, \cdots , k_n)\in \Z^n, \\  
\txt{ and } \quad \quad \quad \quad  e^0_{m} &=& e^{\psi_0(|m|)}=b^{n+1}_{m},\quad \txt{ for } m \in \Z.\eeas Since $\psi = n \psi_0$ is increasing, for $k=(k_1,\cdots , k_n)\in \Z^n$ we get that
\begin{eqnarray*}
\frac{|b_k- e_k|}{e^{\psi(|k|)}} &\leq& \frac{|b^1_{k_1}\cdots b^n_{k_n}- e^1_{k_1}\cdots e^n_{k_n}|}{e^{\psi_0(|k_1|)}\cdots e^{\psi_0(|k_n|)}}\\
&\leq & \sum_{j=1}^n\dfrac{|b^j_{k_j}-e^j_{k_j}|}{e^{\psi_0(|k_j|)}}\left( \prod_{p=0}^{j-1}\dfrac{|e^p_{k_p}|}{e^{\psi_0(|k_p|)}} \prod_{p=j+1}^{n+1}\dfrac{|b^p_{k_p}|}{e^{\psi_0(|k_p|)}} \right) \\ 
&< & \epsilon ~ n \prod_{j=1}^n (1+\|B^j\|_{\psi_0}) \\ 
&\leq & C\epsilon.
\end{eqnarray*}
So we get that $\|B-E\|_\psi < C \epsilon.$ This proves the first part.

For the converse part, let us assume that $\psi:[0,\infty) \rightarrow [0,\infty)$ is an increasing function satisfying 
\bes \sum_{k=0}^\infty\dfrac{\psi(k)}{1+k^2} = \infty. \ees 
Since $\psi$ is increasing, it is easy to see that \bes \int_0^{\infty}\dfrac{\psi(x)}{1+x^2}dx<\infty. \ees 
Now, we can choose $\epsilon>0$ and $a \in \T^n$ in such a way that $B(0,\epsilon)$ and $B(a,2\epsilon)$ are disjoint subsets inside $\T^n.$
By Theorem 2.6 (b) of \cite{BRS}, we get a non-zero radial function $g\in C_c^\infty(\R^n)$ supported inside $B(0,\epsilon)$  satisfying 
$$|\widehat{g}(\xi)|\leq Ce^{-\psi(|\xi|)}, \quad \text{ for all } \xi \in \R^n.$$ 
 We define  
$$h(x)= (2 \pi)^n \sum_{m\in \Z^n}g(x+ 2 \pi m), \quad \txt{for } x\in \T^n.$$ 
We note that the choice of $\epsilon$ implies that $h=g|_{\T^n}$ and $h\in C^\infty(\T^n).$ Moreover, by Poisson summation formula, we have
$$|\widehat{h}(m)|=|\widehat{g}(m)|\leq Ce^{-\psi(|m|)}, \quad \txt{ for all } m\in \Z^n.$$
Now, let us define $f=h*h \in C^\infty(\T^n)$ which is non-zero (since $h$ is non-zero) and vanishes on $B(a,\epsilon)$ because $h$ vanishes on $B(a,2\epsilon).$ So we get that 
$$ |\widehat{f}(m)|\leq Ce^{-\psi(|m|)}|\widehat{h}(m)|, \quad \txt{ for all } m\in \Z^n.$$
Since $h\in C^\infty(\T^n),$ we obtain $\widehat{h} \in l^1(\Z^n)$ which implies that 
$$\ds{\sum_{m\in \Z^n}|\widehat{f}(m)|e^{\psi(|m|)}<\infty.}$$
Therefore by Lemma \ref{dual-lem-c}, we get that $\widehat{f}\in (C_\psi(\Z^n),\|\cdot\|_\psi)^*.$ 
Moreover, since $f \in C^\infty(\T^n)$ vanishes on the set $\Lambda=B (a,\epsilon),$ we have $\widehat{f}$ vanishes on $\Phi_\Lambda(\Z^n).$ Since $f$ is non-zero, we can conclude that the space $\Phi_\Lambda(\Z^n)$ is not dense in $(C_\psi(\Z^n),\|\cdot\|_\psi)$ for some positive rectangle type
set $\Lambda.$
\end{proof}

\subsection{Beurling's Theorem}

We will prove Beurling's result on the torus $\T^n.$ For this, we will need the following lemma regarding a set of positive rectangle type.
\begin{Lem}\label{positive rectangle type}
	If $\Lambda\subset \T^n$ is a set of positive rectangle type and $E$ is a closed set of zero Lebesgue measure, then $\Lambda \setminus E$ is also a set of positive rectangle type. 
\end{Lem} 
\begin{proof}
	 It is known that every open set in $\R^n$ can be written as a countable union of almost disjoint closed cubes. Since $E$ is closed in $\T^n,$ there exist almost disjoint closed rectangles $\{A_k\}_{k\in \N}$ inside $\T^n$ such that 
	$$ E^c=\bigcup_{k\in \N} A_k.$$
	Since $\Lambda$ is a set of positive rectangle type, we have $\Lambda_j\subset \T$ such that $\Gamma = \Lambda_1\times \cdots \times \Lambda_n \subseteq \Lambda$ and $m(\Lambda_j)>0,$ for all $1\leq j \leq n.$ So we get that $$\Gamma \setminus E= \bigcup_{k\in \N}\Gamma \cap A_k.$$
	As $E$ is a set of zero Lebesgue measure, $\Gamma \setminus E$ is a set of positive Lebesgue measure and hence there exists $m\in \N$ such that $\Gamma \cap A_m$ is a set of positive Lebesgue measure. Now, we define  
	$$\Gamma_j= (\text{Projection of  } A_m \text{ on } j^{\txt{th}} \text{ coordinate axis}) \cap \Lambda_j, \quad \txt{ for } 1\leq j \leq n. $$
	Since $A_m$ is a rectangle in $\T^n,$ it is easy to see that 
	$$\Gamma_1\times \cdots \times \Gamma_n =\Gamma \cap A_m.$$
	It follows that $m(\Gamma_j)>0$ for all $1\leq j \leq n.$ Hence we can conclude that $\Lambda\setminus E$ is a set of positive rectangle type.
\end{proof}

\begin{rem}
The above result may not be true if the set $E$ is not assumed to be closed. For example, if we consider the set 
$$F=\lbrace (x,y)\in [0,1]\times [0,1] : x-y \in \mathbb{Q} \rbrace$$ 
of zero Lebesgue measure inside the positive rectangle type set $\Lambda = [0,1]\times [0,1],$ then $\Lambda \setminus F$ is not a set of positive rectangle type.
\end{rem}

Now, we will use Theorem \ref{sev-den-c} to prove Beurling's theorem on $\T^n.$ We note here that the original result on the unit circle $\T$ (Theorem \ref{psi-ber-c}) was for functions in $L^2(\T).$ However, we have been able to improve the following result for functions in $L^1(\T^n).$

\begin{Thm}\label{sev-fn-c}
Let $\psi:[0,\infty)\to [0,\infty)$ be an increasing function such that $\psi(x) \ra \infty$ as $          x \ra \infty$ and 
\begin{equation*}
S = \sum_{k=0}^\infty\dfrac{\psi(k)}{1+k^2}.
\end{equation*}
\begin{itemize}
\item[(a)] Suppose $f\in L^1(\T^n)$ is such that 
\begin{equation}\label{eq;10c}
\sum_{k\in\Z^n}|\widehat{f}(k)|e^{\psi(|k|)}<\infty .
\end{equation}
If $f$ vanishes on a set $\Lambda \subset \T^n$ of positive rectangle type and $I = \infty,$ then $f$ is zero almost everywhere.
\item[(b)] If $I$ is finite, then there is a non-zero $f\in C^\infty(\T^n)$ vanishing on a set of positive rectangle type $\Lambda \subset \T^n$ satisfying (\ref{eq;10c}).
\end{itemize}
\end{Thm}

\begin{proof}
First we shall prove (a). Since $f\in L^1(\T^n)$ and $(\widehat{f}(k))_{k\in\Z^n} \in l^1(\Z^n)$ (follows from (\ref{eq;10c})), the Fourier series of $f$ converges absolutely and uniformly to a continuous function  $g$ on $\T,$ that is, 
\be \label{fourierseries} g(\lambda)=\sum_{k\in \Z^n} \widehat{f}(k)e^{ik\cdot \lambda}, \quad \txt{ for all } \lambda\in \T^n. \ee
Moreover, $f=g$ almost everywhere. So we get that 
$$N=\{x\in \T^n:f(x)\neq g(x)\}$$ 
is a set of zero Lebesgue measure. 

We claim that $g$ also vanishes on a set of positive rectangle type. To prove this, we consider the open set  
$$O=\{x\in \T^n:g(x)\neq 0\}.$$ 
Since $\Lambda$ is a set of positive rectangle type in $\T^n,$ using regularity of the Lebesgue measure,  without loss of generality we can assume that $\Lambda$ is closed in $\T^n.$ Now we define 
$$E=\Lambda\cap O,$$
which is a set of zero Lebesgue measure as $E\subseteq N.$ Since $\Lambda$ is closed and $O$ is open, it follows that 
$$\overline{E}\setminus E \subseteq \Lambda \cap (\overline{O}\setminus O)$$ 
is a set of zero Lebesgue measure. So we get that $\overline{E}$ is also a set of zero Lebesgue measure. Thus by Lemma \ref{positive rectangle type}, $F = \Lambda\setminus \overline{E}$ is a set of positive rectangle type in $\T^n.$ Moreover, $g$ vanishes on $F$ because 
$$F \subseteq \Lambda\setminus E= \Lambda\setminus O \subseteq O^c.$$ This proves our claim. 

Finally, we will prove that $\widehat{f}(k)=0 \txt{ for all } k\in \Z^n.$ Since $(\widehat{f}(k))_{k\in\Z^n}$ satisfies (\ref{eq;10c}), from Lemma \ref{dual-lem-c} we get that $(\widehat{f}(k))_{k\in \Z^n}\in (c_{\psi}(\Z^n),\|\cdot \|_{\psi})^*,$ that is, there is a bounded linear functional $T_f$ on $c_{\psi}(\Z^n)$ given by
\be \label{functional} T_f(A) = \sum_{k \in \Z^n} \widehat{f}(k) ~ a_k, \quad \txt{ for } A = (a_k)_{k \in \Z^n} \in c_{\psi}(\Z^n). \ee
Now, as $g$ vanishes on $F,$ from (\ref{fourierseries}) and (\ref{functional}) we get that $T_f$ vanishes on the set $\Phi_{F}(\Z^n).$ Since $F$ is a set of positive rectangle type and $S=\infty$, we get from Theorem \ref{sev-den-c} that $\Phi_{F}(\Z^n)$ is dense in $(c_{\psi}(\Z^n),\|\cdot \|_{\psi}).$ So it follows that $T_f$ is identically zero. Hence we conclude that $\widehat{f}(k)=0 \txt{ for all } k\in \Z^n.$ This proves (a). 

For part (b), if $S<\infty,$ the function $f$ constructed in the proof of converse part of Theorem \ref{sev-den-c} serves our purpose. This completes the proof.
\end{proof}

\textbf{Acknowledgement:} The authors are thankful to Professor Swagato Kumar Ray for several fruitful discussions and suggestions. The first author is supported by CSIR Senior Research Fellowship (Enrollment Id : 09/028(1002)/2017-EMR-I).


\begin{thebibliography}{99}

	
	\bibitem{B} Beurling, A. \textit{The collected works of Arne Beurling. Volume 1. Complex analysis. Edited by L. Carleson, P. Malliavin, J. Neuberger and J. Wermer. Contemporary Mathematicians.}, Birkh\"auser Boston, Inc., Boston, MA,  1989. MR1057613
		 
%	\bibitem{BR} Bhowmik, M.; Ray, S. \textit{ A theorem of Levinson for Riemannian symmetric spaces of noncompact type}	 
		 
	\bibitem{BRS} Bhowmik, M.; Ray, S.; Sen, S. \textit{Around Theorems of Ingham-type Regarding Decay of Fourier Transform on $\R^n,$ $\T^n$ and Two Step Nilpotent Lie Groups}, Bulletin des Sciences Mathématiques, 155 (2019), 33-73. MR3944135 
		
%	\bibitem{BS1}Bhowmik, M.; Sen, S. \textit{ An Uncertainty Principle of Paley and Wiener on Euclidean Motion Group} Journal of Fourier Analysis and Applications volume 23, pages1445–1464(2017)
	
%	\bibitem{BS2}Bhowmik, M.; Sen, S. \textit{ Uncertainty Principles of Ingham and Paley-Wiener on Semisimple Lie Groups} Israel Journal of Mathematics 225(1) · June 2016
		
	
%	\bibitem{CG} Corwin,L. J.; Greenleaf, F. P. \textit{Representations of nilpotent Lie groups and their applications} Cambridge University Press, Cambridge, 1990. MR1070979 (92b:22007)
	
	\bibitem{DJ} De Jeu, M.; \textit{Subspaces with equal closure}, Constructive Approximation 20 (2004), no. 1, 93-157. MR2025416
	
	
%	\bibitem{HJ} Havin, V.; J\"oricke, B. \textit{The Uncertainty Principle in Harmonic Analysis} Ergebnisse der Mathematik und ihrer Grenzgebiete, 3, Folge, 28, Berlin, Springer-Verlag, 1994. MR1303780 (96c:42001)
	
		
	
	\bibitem{I} Ingham, A. E. \textit{A Note on Fourier Transforms}, J. London Math. Soc. 9 (1934) no. 1 , 29-32. MR1574706	
		
	\bibitem{K} Koosis, P. \textit{The logarithmic integral I}, Cambridge Studies in Advanced Mathematics, 12, Cambridge University Press, Cambridge, 1998. MR1670244 (99j:30001)
	
	\bibitem{L1} Levinson, N. \textit{Gap and Density Theorems}, American Mathematical Society Colloquium Publications, v. 26. American Mathematical Society, New York, 1940. MR0003208 (2,180d)
	
%	\bibitem{L2} Levinson, N. \textit{On a Class of Non-Vanishing Functions} Proc. London Math. Soc. 41 (1936) no. 1, 393-407. MR1576177
	
		
%	\bibitem{N} Nachbin, L. \textit{Weighted approximation for algebras and modules of continuous functions: Real and self-adjoint complex cases} Ann. of Math. (2) 81 1965 289-302. 	MR0176353
	
	\bibitem{PW} Paley, R. E. A. C.; Wiener, N. \textit{Fourier transforms in the complex domain (Reprint of the 1934 original)}, American Mathematical Society Colloquium Publications, 19. American Mathematical Society, Providence, RI, 1987. MR1451142 (98a:01023)
	
%	\bibitem{PW1} Paley, R. E. A. C.; Wiener, N. \textit{Notes on the theory and application of Fourier transforms. I, II.} Trans. Amer. Math. Soc. 35 (1933), no. 2, 348–355. MR1501688
	
	\bibitem{P1} Poltoratski, A. \textit{A problem on completeness of exponentials}, Ann. of Math. (2) 178 (2013) no. 3, 983–1016. MR3092474
	
	\bibitem{P} Poltoratski, A. \textit{Toeplitz Approach to Problems of the Uncertainty Principle}, American Mathematical Society Providence, Rhode Island, 2015. MR3309830
	
	%	\bibitem{PS1} Parui, S.; Sarkar, R. P. \textit{Beurling's theorem and $L^p$-$L^q$ Morgan's theorem for step two nilpotent Lie groups.} Publ. Res. Inst. Math. Sci. 44 (2008), no. 4, 1027–1056. MR2477903 (2009j:22014) 
	
	
	%\bibitem{R} Ray, S. K. \textit{Uncertainty principles on two step nilpotent Lie groups} Proc. Indian Acad.Sci. 111 (2001), no. 3, 293-318. MR1851093 (2002g:22016).
	
%	\bibitem{Ru} Rudin, W. \textit{Real and complex analysis} Third edition. McGraw-Hill Book Co., New York, 1987. MR0924157 (88k:00002)
	
	


		
	
		
	
	
	
	
	
	
	
	
	
\end{thebibliography}














\end{document}

