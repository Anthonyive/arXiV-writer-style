\documentclass[12pt]{amsart}
\usepackage{amssymb,amscd}
%\usepackage{comment}


%\usepackage[notref,notcite]{showkeys}

\usepackage{tikz-cd}
%\usepackage{tikz}
%\usepackage[all,cmtip]{xy}
%\usetikzlibrary{calc,matrix}

%\let\mc\mathcal
%\usepackage{eucal}
%\let\euc\mathcal
%\let\mathcal\mc

%\usepackage{verbatim}

%\usepackage{amsmath,amssymb,graphicx,mathrsfs}   % my new
\usepackage[colorlinks=true,allcolors = blue]{hyperref} % my new
%\usepackage{mathtools}
%\DeclarePairedDelimiter{\ceil}{\lceil}{\rceil}


\textwidth 6.5truein
\textheight 8.67truein
\oddsidemargin 0truein
\evensidemargin 0truein
\topmargin 0truein

\let\frak\mathfrak
\let\Bbb\mathbb

\def\>{\relax\ifmmode\mskip.666667\thinmuskip\relax\else\kern.111111em\fi}
\def\<{\relax\ifmmode\mskip-.333333\thinmuskip\relax\else\kern-.0555556em\fi}
\def\vsk#1>{\vskip#1\baselineskip}
\def\vv#1>{\vadjust{\vsk#1>}\ignorespaces}
\def\vvn#1>{\vadjust{\nobreak\vsk#1>\nobreak}\ignorespaces}
\def\vvgood{\vadjust{\penalty-500}} \let\alb\allowbreak
\def\fratop{\genfrac{}{}{0pt}1}
\def\satop#1#2{\fratop{\scriptstyle#1}{\scriptstyle#2}}
\def\tfrac{\textstyle\frac}

\def\sskip{\par\vsk.2>}
\let\Medskip\medskip
\def\medskip{\par\Medskip}
\let\Bigskip\bigskip
\def\bigskip{\par\Bigskip}

\let\Maketitle\maketitle
\def\maketitle{\Maketitle\thispagestyle{empty}\let\maketitle\empty}

\newtheorem{thm}{Theorem}[section]
\newtheorem{cor}[thm]{Corollary}
\newtheorem{lem}[thm]{Lemma}
\newtheorem{prop}[thm]{Proposition}
\newtheorem{conj}[thm]{Conjecture}
\newtheorem{defn}[thm]{Definition}

\numberwithin{equation}{section}

\theoremstyle{definition}
\newtheorem*{rem}{Remark}
\newtheorem*{example}{Example}

\let\mc\mathcal
\let\nc\newcommand

\def\flati{\def\=##1{\rlap##1\hphantom b)}}

\let\al\alpha
\let\bt\beta
\let\dl\delta
\let\Dl\Delta
\let\eps\varepsilon
\let\gm\gamma
\let\Gm\Gamma
\let\la\lambda
\let\La\Lambda
\let\pho\phi
\let\phi\varphi
\let\si\sigma
\let\Si\Sigma
\let\Sig\varSigma
\let\Tht\Theta
\let\tht\theta
\let\thi\vartheta
\let\Ups\Upsilon
\let\om\omega
\let\Om\Omega

\let\der\partial
\let\Hat\widehat
\let\minus\setminus
\let\ox\otimes
\let\wt\widetilde

\let\ge\geqslant
\let\geq\geqslant
\let\le\leqslant
\let\leq\leqslant

\let\on\operatorname
\let\bi\bibitem
\let\bs\boldsymbol

\def\C{{\mathbb C}}
\def\Z{{\mathbb Z}}

\def\Ac{{\mc A}}
\def\B{{\mc B}}
\def\D{{\mc D}}
\def\E{{\mc E}}
\def\F{{\mc F}}
\def\L{{\mc L}}
\def\M{{\mc M}}
\def\N{{\mc N}}
\def\O{{\mc O}}
\def\Pc{{\mc P}}
\def\Q{{\mc Q}}
\def\Sc{{\mc S}}
\def\V{{\mc V}}
\def\K{{\mc K}}
\def\O{{\mc O}}
\def\T{{\mc T}}
\def\l{{\mathfrak l}}
\def\m{{\mathfrak m}}

\def\+#1{^{\{#1\}}}
\def\lsym#1{#1\alb\dots\relax#1\alb}
\def\lc{\lsym,}

\def\End{\on{End}}
\def\Gr{\on{Gr}}
\def\Hom{\on{Hom}}
\def\id{\on{id}}
\def\Res{\on{Res}}
\def\res{\on{res}}
\def\qdet{\on{qdet}}
\def\rdet{\on{rdet}}
\def\sign{\on{sign}}
\def\const{\on{const}}
\def\ch{\on{ch}}
\def\tbigoplus{\mathop{\textstyle{\bigoplus}}\limits}
\def\tbigcap{\mathrel{\textstyle{\bigcap}}}
\def\Wr{\on{Wr}}
\def\Sym{\on{Sym}}
\def\Sing{\on{Sing}}

\def\codim{\on{codim}}%MF
\def\im{\on{im}}%MF

\def\ii{i,\<\>i}
\def\ij{i,\<\>j}
\def\ik{i,\<\>k}
\def\il{i,\<\>l}
\def\ji{j,\<\>i}
\def\jk{j,\<\>k}
\def\kj{k,\<\>j}
\def\kl{k,\<\>l}
\def\II{I\<\<,\<\>I}
\def\IJ{I\<\<,J}
\def\ioi{i+1,\<\>i}
\def\ppo{p,\<\>p+1}
\def\pop{p+1,\<\>p}
\def\pci{p,\<\>i}
\def\pcj{p,\<\>j}
\def\poi{p+1,\<\>i}
\def\poj{p+1,\<\>j}

\def\gln{\mathfrak{gl}_N}
\def\sln{\mathfrak{sl}_N}

\def\fb{\mathfrak{b}}
\def\fg{\mathfrak{g}}
\def\fh{\mathfrak{h}}
\def\fn{\mathfrak{n}}

\def\glnt{\gln[t]}
\def\Ugln{U(\gln)}
\def\Uglnt{U(\glnt)}
\def\Yn{Y\<(\gln)}

\def\slnt{\sln[t]}
\def\Usln{U(\sln)}
\def\Uslnt{U(\slnt)}

\def\beq{\begin{equation}}
\def\eeq{\end{equation}}
\def\be{\begin{equation*}}
\def\ee{\end{equation*}}

\nc{\bea}{\begin{eqnarray*}}
\nc{\eea}{\end{eqnarray*}}
\nc{\bean}{\begin{eqnarray}}
\nc{\eean}{\end{eqnarray}}
%\nc{\Ref}[1]{{\rm(\ref{#1})}}    %\eqref{} instead of \Ref

\def\g{{\mathfrak g}}
\def\h{{\mathfrak h}}
\def\n{{\mathfrak n}}
\def\fsl{\mathfrak{sl}}


\def\CD{{\mathcal{D}}}

\let\ga\gamma
\let\Ga\Gamma

\nc{\Il}{{\mc I_{\bs\la}}}
\nc{\bla}{{\bs\la}}
\nc{\Fla}{\F_{\bs\la}}
\nc{\tfl}{{T^*\Fla}}
\nc{\GL}{{GL_n(\C)}}
\nc{\GLC}{{GL_n(\C)\times\C^*}}

\def\Di{{\tfrac 1D}}
\def\DV{\Di\V^-}
\def\DL{\DV_\bla}

\def\zzz{z_1\lc z_n}
\def\Czh{\C[\zzz,h]}
\def\Vz{V\ox\Czh}
\def\ty{\Tilde Y\<(\gln)}
\def\tb{\Tilde\B}
\def\IMA{{I^{\<\>\max}}}
\def\IMI{{I^{\<\>\min}}}
\def\CZH{\C[\zzz]^{\>S_n}\!\ox\C[h]}
\def\Sla{S_{\la_1}\!\lsym\times S_{\la_N}}
\def\SlN{S_{\la_N}\!\lsym\times S_{\la_1}}
\def\Czhl{\C[\zzz]^{\>\Sla}\!\ox\C[h]}
\def\Czghl{\C[\zb;\Gmm;h]^{\>S_n\times\Sla}}

\def\sh{\hat s}
\def\st{\tilde s}
\def\ib{\bs i}
\def\jb{\bs j}
\def\kb{\bs k}
\def\iib{\ib,\<\>\ib}
\def\ijb{\ib,\<\>\jb}
\def\ikb{\ib,\<\>\kb}
\def\jib{\jb,\<\>\ib}
\def\jkb{\jb,\<\>\kb}
\def\kjb{\kb,\<\>\jb}
\def\zb{\bs z}
\def\zzi{z_1\lc z_i,z_{i+1}\lc z_n}
\def\zzii{z_1\lc z_{i+1},z_i\lc z_n}
\def\zzzn{z_n\lc z_1}
\def\ip{\<\>i\>\prime}
\def\ipi{\>\prime\<\>i}
\def\jp{\prime j}
\def\iset{\{\<\>i\<\>\}}
\def\jset{\{\<\>j\<\>\}}
\def\IMIp{{I^{\min,i\prime}}}
\let\Gmm\Gamma
\def\fc{\check f}
\def\sih{\hat\si}
\def\top{\on{top}}
\def\ka{{\kappa}}
\def\slt{{\frak{sl}_2}}
\def\OT{{\otimes_{a=1}^nM_{m_a}}}
\def\mm{{\bs m}}
\def\zz{{\bs z}}

\def\lra{\longrightarrow}
\nc{\glt}{{\frak{gl}_2}}
\def\KZ/{{\slshape KZ\/}}
\def\qKZ/{{\slshape qKZ\/}}
\def\XXX/{{\slshape XXX\/}}
\nc{\arr}{\rightarrow}
\nc{\larr}{\longrightarrow}
\nc{\A}{{\mc A}}
\nc{\Ax}{{\mc A(\xi)}}
\nc{\cdet}{{\on{cdet}\,}}
\nc{\Fun}{{\on{Fun}_{\slt}\!\!V[0]\,}}
\nc{\Oz}{{\O(\zeta,m,\ell)}}
\nc{\Bm}{{\B(\mu, m,\ell)}}
\nc{\sk}{{\sqrt{-1}}}



\begin{document}

\hrule width0pt
\vsk->
\title[Dynamical Bethe algebra and quasi-polynomials]
{Dynamical $\slt$ Bethe algebra and functions
\\
 on pairs of quasi-polynomials}

\author[A.\,Slinkin, D.\,Thompson, A.\,Varchenko]
{A.\,Slinkin$^{\diamond}$, D.\,Thompson$^*$, A.\,Varchenko$^{\star}$}

\maketitle


\begin{center}
{\it $^{\diamond, *, \star}\<$Department of Mathematics, University
of North Carolina at Chapel Hill\\ Chapel Hill, NC 27599-3250, USA\/}


\vsk.5>
{\it $^{\diamond}\<$
National Research University Higher School of Economics\\ 20 Myasnitskaya Street, 101000 Moscow, Russia\/}

\vsk.5>
{\it $^{\star}\<$Faculty of Mathematics and Mechanics, Lomonosov Moscow State
University\\ Leninskiye Gory 1, 119991 Moscow GSP-1, Russia\/}

\end{center}

\bigskip
\hfill
On the Occasion of the 70th Birthday of Igor Krichiver

\bigskip

\vsk>
{\leftskip3pc \rightskip\leftskip \parindent0pt \Small
{\it Key words\/}: Commuting differential operators, eigenfunctions, Weyl group invariance,
\\
\phantom{aaaaaaaaqq}
Bethe ansatz, Wronskian equation,
quasi-polynomials


\vsk.6>
{\it 2010 Mathematics Subject Classification\/}: 17B80, 81R12, 14M15
\par}



{\let\thefootnote\relax
\footnotetext{\vsk-.8>\noindent
$^\diamond\<$
{\it E\>-mail}: slinalex@live.unc.edu
\\
$^*$
{\it E\>-mail}:
dthomp@email.unc.edu
\\
$^\star\<$
{\it E\>-mail}:
anv@email.unc.edu\,, supported in part by NSF grants
DMS-1665239, DMS-1954266}}

\begin{abstract} 

We consider  the space $\text{Fun}_{\frak{sl}_2}\!V[0]$ of functions
on the Cartan subalgebra of $ \frak{sl}_2$ with values
in the zero weight subspace $V[0]$ of a tensor product of irreducible finite-dimensional
$\frak{sl}_2$-modules. We consider the algebra $\mathcal B$
of commuting differential  operators on $\text{Fun}_{\frak{sl}_2}\,V[0]$, constructed
by  V.\,Rubtsov, A.\,Silantyev, D.\,Talalaev in 2009. We describe the relations between
the action of $\mathcal B$ on $\text{Fun}_{\frak{sl}_2}\,V[0]$ and spaces of pairs 
of quasi-polynomials.






\end{abstract}

{\small\tableofcontents\par}

\setcounter{footnote}{0}
\renewcommand{\thefootnote}{\arabic{footnote}}





\section{Introduction}








A  quantum integrable model is a vector space  $V$
and an algebra $\B$ of commuting linear operators on $V$, 
 called  the Bethe algebra of Hamiltonians. 
The problem is to find eigenvectors and eigenvalues.
If the vector space is a space of functions, then the Hamiltonians are  differential or difference operators.

We say that a quantum integrable model can be geometrized,
 if there is a topological space (a scheme) $X$ with an algebra
$\O_X$ of functions on $X$,  an isomorphism of vector spaces $\psi :\O_X\to V$,  an isomorphism
of algebras $\tau:\O_X\to\B$ such that
\bea
\psi(fg) = \tau(f)\,\psi(g),\qquad \forall f,g\in\O_X.
\eea
These objects $\O_X, \psi,\tau$ identify the  $\B$-module $V$ with the regular representation of the algebra
  $\O_X$ of functions.
 


If a quantum integrable model $(V,\B)$ is geometrized, then the eigenvectors 
of $\B$ in $V$ are identified with delta-functions of points of $X$ and the eigenvalues of an eigenvector
in $ V$ correspond to evaluations of functions on $X$ at the corresponding  point of $X$.

\smallskip

Our motivation to geometrize the Bethe algebras  came from the examples
considered in \cite{MTV3, MTV5}, where the algebra of Hamiltonians acting on a subspace 
of a tensor product of $\gln$-modules was identified with the
algebra of functions on the intersection of suitable Schubert cycles in a 
Grassmannian. That identification gave an unexpected relation between the representation
theory and Schubert calculus. 

\smallskip

The examples in \cite{MTV3, MTV5} are related to models with  a finite-dimensional vector space $V$.
How to proceed in the infinite-dimensional case of commuting differential operators
 is not clear yet. In this paper we discuss an example. 
 In our infinite-dimensional space $V$
 we distinguish a family of finite-dimensional subspaces $E[\mu]$, $\mu\in\C$, each of which is invariant
 with respect to the algebra $\B$ of commuting differential operators. We geometrize each of the pairs
 $(E[\mu], \B\big\vert_{E[\mu]})$, thus constructing a family of topological spaces $X[\mu]$, $\mu\in\C$. 
 We observe that natural interrelations between the subspaces $E[\mu]$ correspond to natural interrelations
 between the topological spaces $X[\mu]$. For example,  the Weyl involution
$V\to V$, available in our case, identifies $E[\mu]$ and 
 $E[-\mu]$. We show that this identification corresponds to a natural
 isomorphism $X[\mu]\to X[-\mu]$.
 
 
 


\smallskip


Representation theory provides a source of commuting differential or difference operators. In this
paper we discuss the  construction due to 
  V.\,Rubtsov, A.\,Silantyev, D.\,Talalaev, \cite{RST}. That quantum integrable model
   is called the {\it quantum dynamical Gaudin model}. We study the $\slt$ trigonometric version
   of the quantum dynamical Gaudin model, while in \cite{RST}  the $\gln$ elliptic version
was considered. 
\smallskip

Consider the Lie algebra $\slt$ and its Cartan subalgebra $\h\subset \slt$,
$\dim \h = 1$.
For $s=1,\dots,n$, let    $V_{m_s}$ be the irreducible  $\slt$-module of dimension $m_s+1$.
Let  $V=\otimes_{s=1}^nV_{m_s}$,
\bea
V[0] = \{ v\in\ V\ | \ hv=0, \  \ \forall h\in\h\}\,,
\eea
the zero weight subspace.  The space $V[0]$ is nontrivial if $M=\sum_{s=1}^nm_s$ is even.
Let 
$\Fun$  be the
  space of $V[0]$-valued functions on $\h$.
Fix a subset $z=\{z_1,\dots,z_n\}\subset \C^\times$. 
Having these data,  Rubtsov, Silantyev, and Talalaev  construct a family of commuting differential operators acting on $\Fun$. 


First,  one  constructs a $2\times 2$-matrix
$\left[
\begin{matrix}
x \der_x & 0  
\\
0 & x \der_x 
\end{matrix} 
\right]+ L(x)\,
=(\delta_{ij}\der_x +  L_{ij}(x))$, 
where  $x$ is a parameter,  
$\der_x =\frac{\der}{\der x}$,
and $L_{ij}(x)$  are differential operators on $\Fun$  
depending on $x$. 
Let
$\mc C = \on{cdet}\big[\delta_{ij}\der_x +  L_{ij}(x)\big],$
where {\it cdet} is the column determinant of the matrix with non-commuting entries,
$\on{cdet}
\left[\begin{array}{cc}
a & b\\
c & d
\end{array}\right] = ad-cb.$
The operator $\mc C$ can be rewritten in the form
\bea
\der_x^2 + C_1(x)\der_x+C_2(x) ,
\eea
where $C_1(x), C_2(x)$  are differential operators on $\Fun$, whose coefficients are rational functions of $x$.
For any $a,b\in \C -\{z_1,\dots,z_n\}$ and $i,j=1,2$ the operators $C_i(a)$, $C_j(b)$ commute. 
The space $\Fun$ with the algebra
$\B$ generated by these commuting differential operators
 is called the {\it quantum dynamical Gaudin model}.

\smallskip
We show that the  algebra $\B$ is generated by the trigonometric KZB operators $H_0$,
$H_1(z)$, \dots, $H_n(z)$, see them in \cite{FW, JV}. 
The KZB operator $H_0$ is also known as the
trigonometric  Hamiltonian operator of the quantum two-particle
Calogero-Moser system with spin space $V$. The operator $H_0$
is the second order differential operator independent of $z$. 
\smallskip

For any $\mu\notin\Z$ we define the subspace $E[\mu]\subset \Fun$ as
the space of meromorphic
eigenfunctions of $H_0$ with eigenvalue $\pi \sqrt{-1}\, \frac{\mu^2}2$ and prescribed poles. The subspaces
$E[\mu]$ were introduced in \cite{FV2} and studied in \cite{JV}. We have $\dim E[\mu]=\dim V[0]$.
The Bethe algebra $\B$ preserves each of $E[\mu]$.

\smallskip
The $\slt$ Weyl involution acts on $\Fun$. The Bethe algebra $\B$ is Weyl group invariant. 
The Weyl involution induces an isomorphism  $E[\mu]\to E[-\mu]$, which is called in \cite{FV2}
the scattering matrix.  The scattering matrix 
$E[\mu]\to E[-\mu]$ is an isomorphism of $\B$-modules.

\smallskip
The basis of the geometrization procedure lies in the following observation. Let $\psi\in E[\mu]$ be an eigenvector of $\B$,
\bea
C_i(x) \psi = E_i(x, \psi) \psi, \qquad i=1,2,
\eea
where $E_i(x,\psi)$ are scalar eigenvalue functions of $x$. We assign to $\psi$ the scalar differential operator
\bea
\mc E_\psi = \der_x^2 + E_1(x,\psi)\der_x+E_2(x,\psi).
\eea
 We show that the kernel of $\psi$ is spanned by 
two quasi-polynomials
$x^{-\mu/2} f(x), x^{\mu/2} g(x)$, where $f(x),g(x)$ are monic polynomials of  degree $M/2$, with the property
that the  Wronskian 
of the two quasi-polynomials  is
\bean
\label{f1}
\Wr(x^{-\mu/2} f(x), x^{\mu/2} g(x))\, = \,\frac {\mu}x\,\prod_{s=1}^n (x-z_s)^{m_s}.
\eean
This fact suggests that the space $X[\mu]$ geometrizing $(E[\mu], \B\big\vert_E[\mu])$
is the space of pairs  $(x^{-\mu/2} f(x)$, $x^{\mu/2} g(x))$ of quasi-polynomials
 with Wronskian given by \eqref{f1}.  
 

 
 \smallskip
 In this paper we show that this is indeed so. 
We also show that 
 the scattering matrix isomorphism
$E[\mu]\to E[-\mu]$ corresponds to the natural isomorphism $X[\mu]\to X[-\mu]$ defined by the transposition of 
the quasi-polynomials. 

\smallskip
The main message of this paper is the deep relation between the quantum dynamical Gaudin model $(\B, \Fun)$ and
the spaces of pairs of quasi-polynomials.

\smallskip
It would be interesting to develop the elliptic version of this correspondence. In the elliptic version the pairs of quasi-polynomials
are replaced with pairs of theta-polynomials, see \cite{ThV}, but the elliptic KZB operator
$H_0$  does depend on $z$ and does not have apparent analogs of the 
subspaces $E[\mu]$.

\medskip
The paper is organized as follows. In Section \ref{sec 2} we define the
$\slt$ quantum dynamical Gaudin model. 
In Section \ref{sec 3} we discuss properties of the spaces $E[\mu]$. 
In Section \ref{sec 4} we introduce the quantum trigonometric Gaudin model 
$(V[\nu],\B(z,\mu,V[\nu])$ on a weight subspace $V[\nu]\subset V$ and show that
the quantum dynamical Gaudin model $(E[\mu], \B\big\vert_{E[\nu]})$
is isomorphic to the quantum trigonometric  Gaudin model 
$(V[0],\B(z,\mu,V[0])$ on the zero weight subspace. In Section \ref{sec 5} we describe
the Bethe ansatz for the quantum trigonometric  Gaudin model.
In Sections \ref{sec 6} and \ref{sec 7} we describe the kernel of the operator $\mc E_\psi$.
In Sections \ref{sec 8} - \ref{sec 11} we develop the geometrization procedure. The constructions
of Sections \ref{sec 9} - \ref{sec 11} are parallel to the geometrization constructions in
\cite{MTV3, MTV2}.

\medskip

The authors thank V.\,Tarasov for useful discussions.


\section{Quantum dynamical Gaudin model}
\label{sec 2}

\subsection{$\glt$ $RST$-operator}
\label{ssec RST} 
\label{ssec not}

 Consider the complex Lie algebra $\glt$ with standard basis  $e_{11}$, $e_{12}$, $e_{21}$, $e_{22}$.
 Denote by $\h$  the Cartan subalgebra of $\glt$ with basis  $e_{11},e_{22}$ and elements 
 $\la_1 e_{11}+\la_2 e_{22}$.
 Denote 
 \bea
 \la :=\la_1-\la_2.
 \eea
Let $z=\{z_1,\dots,z_n\}\subset \C^\times$ be a set of nonzero pairwise distinct numbers.

\smallskip

Let $V^1,\dots,V^n$ be $\glt$-modules and $V=\otimes_{k=1}^nV^k$.  
Let  $V = \oplus_{\nu\in\h^*} V[\nu]$
be  the weight decomposition, where $ V[\nu] = \{v\in V\ |\ e_{jj}v = \nu (e_{jj})v \ \on{for}\ j=1,2 \}$.
In particular,
\bea
V[0] = \{v\in V\ |\ e_{11}v =  e_{22}v = 0\}.
\eea

For $g\in \glt$, denote $g^{(s)} = 1 \otimes \cdots \otimes g \otimes \cdots \otimes 1\in \End(V)$, with  $g$ in the $s$-th factor.
An element $e_{jk}$ acts on $V$ by $e_{jk}^{(1)}+ \dots + e_{jk}^{(n)}$.
\smallskip

Let $u$ be a variable. Denote
\bea
x= e^{-2\pi \sqrt{-1}u}.
\eea
Let $\der_u = \frac{\der }{\der u}$, \, $\der_x = \frac{\der }{\der x}$,\, $\der_{\la_j} = \frac{\der }{\der \la_j}$ and so on.
\smallskip


Introduce a $2\times 2$-matrix $\L$,
 \bean
\label{matrixL0}
\phantom{aaa}
\\
\notag
  \begin{bmatrix} \L_{11}  &\L_{12}
  \\
   \L_{21}  &\L_{22}
\end{bmatrix} 
 =
 \begin{bmatrix} 
 \pi 
 \sqrt{-1} 
\sum_{s=1}^n \frac{z_s+x}{z_s-x} \,e_{11}^{(s)} 
+ \pi \cot (\pi \la) \, e_{22} \; & \; 
\pi \sqrt{-1} \sum_{s=1}^n \frac{z_s+x}{z_s-x} \,e_{21}^{(s)} - \pi \cot (\pi \la) e_{21} 
 \\
\pi \sqrt{-1} \sum_{s=1}^n \frac{z_s+x}{z_s-x} \,e_{12}^{(s)} + \pi \cot (\pi \la) e_{12} \; 
& \; \pi \sqrt{-1} \sum_{s=1}^n \frac{z_s+x}{z_s-x} \,e_{22}^{(s)} - \pi \cot (\pi \la) e_{11} 
\\
\end{bmatrix},
\eean

\vsk.2>
\noindent
The entries of $\L$ are $\End(V)$-valued trigonometric functions of $u$ and $\la$.

\smallskip
The {\it universal dynamical differential operator} (or the {\it $RST$-operator})
is defined by the formula 
\bean
\label{trigRST}
\mc C = \cdet (\delta_{jk}\partial_u - \delta_{jk} \, \partial_{\la_{j}} + \L_{jk}),
\eean
where for a $2\times 2$-matrix $A = (a_{jk})$ with noncommuting entries the column determinant
 is defined by the formula 
\bea
\cdet A = a_{11}a_{22} - a_{21}a_{12}\,.
\eea
Write the $RST$-operator in the form
\bean
\label{D}
\mc C = \partial_u^2 \,+ \,C_1(x)\, \partial_u \,+ \,C_2(x),
\eean
where $C_1(x)$ and $C_2(x)$ are functions in $x$ with 
values in the space of linear differential operators in variables $\la_1,\la_2$ with coefficients in $\End(V)$.



\begin{thm}
[\cite{RST}]
\label{thm RST}

Fix  $z=\{z_1,\dots,z_n\}\subset \C^\times$.  
Then for any $a \in \C-\{z_1,\dots,z_n\}$ the operators $C_1(a), C_2(a)$, 
restricted to $V[0]$-valued functions of $\la_1,\la_2$,
 define 
linear differential operators in $\la_1, \la_2$ with coefficients in $\End(V[0])$. Moreover, for any 
$a,b \in \C -\{z_1,\dots,z_n\}$, 
the differential operators $C_j(a)$, $C_k(b)$, $j,k=1,2$, acting on the space of $V[0]$-valued functions of $\la_1,\la_2$
commute: 
\bean
\label{cC}
[C_j(a), C_k(b)] = 0, \qquad j,k=1,2.
\eean
\end{thm} 

\smallskip
The  elliptic version of the $RST$-operator for $\gln$ was introduced by V.\,Rubtsov, A.\,Silantyev, D.\,Talalaev in \cite{RST}.
The elliptic version of the $\gln$ $RST$-operator, in particular for the case of $N=2$, 
was discussed in \cite{ThV}. The $RST$-operator,
defined in \eqref{D}, is the trigonometric 
degeneration of the elliptic $\glt$  $RST$-operator.







\subsection{Dynamical Bethe algebra of $\Fun$} 

In this paper, we are interested in  the $\slt$ version of the $RST$-operator.

\smallskip
The Lie algebra $\slt$ is a Lie subalgebra of $\glt$. We have  $\glt=\slt\oplus \C(e_{11}+e_{22})$,
where $e_{11}+e_{22}$ is a central element. 
Let $V^1,\dots,V^n$ be $\slt$-modules, thought of as $\glt$-modules,
 where the central element $e_{11}+e_{22}$ acts by zero.
Let $V=\ox_{k=1}^nV^k$ be the tensor product of the $\slt$-modules. 

\smallskip

 In this paper {\it we consider only such tensor products.}


\smallskip
We consider the Cartan subalgebra of $\slt$ consisting of elements $\la_1e_{11}+\la_2e_{22}$
with $\la_1+\la_2=0$.
 We identify the algebra of functions 
on the Cartan subalgebra of $\slt$ with the algebra of functions in
the variable 
\bea
\la=\la_1-\la_2\,,
\eea
 since the elements
$\la_1e_{11}+\la_2e_{22}$
with $\la_1+\la_2=0$ are uniquely determined by the difference of coordinates.


Denote by $\Fun$ the space of $V[0]$-valued meromorphic functions on the Cartan subalgebra of $\slt$.
In other words, $\on{Fun}_{\slt}\!\!V[0]$ is the space of $V[0]$-valued meromorphic functions in
one variable $\la$.



\smallskip
Each coefficient $C_1(x), C_2(x)$ of the $RST$-operator, defines a differential operator
acting on  $\Fun$. From now on {\it we consider  the coefficients $C_1(x)$, $C_2(x)$ as 
a family of  commuting differential operators  on  $\Fun$}, depending on the parameter $x$.

\smallskip
The commutative algebra of differential operators on  $\Fun$  generated by the identity operator
 and the operators $\{ C_j(a)\ |\ j=1,2, \ a\in \C -\{z_1,\dots,z_n \}\}$
is called the {\it dynamical Bethe algebra} of $\Fun$. The dynamical Bethe algebra depends on the choice of
the numbers $\{z_1,\dots,z_n\}$.


\subsection{Tensor product of $\slt$-modules}
\label{sec TP}

Given $m\in \Z_{\geq 0}$, denote by $V_{m}$ the 
 irreducible $\slt$-module with highest weight $m$. 
 It has a basis $v^m_0,\dots,v^m_m$ such that
 \bean
 \label{V_m}
 \phantom{aaaaa}
 (e_{11}-e_{22})v^m_k=(m-2k)v^m_k,
 \quad
 e_{21}v^m_k=(k+1)v^m_{k+1},
\quad
 e_{12}v^m_k=(m-k+1)v^m_{k-1}.
 \eean
 
 \smallskip
From now on our tensor product $V$ is of the form
\bean
\label{ten V}
V\,=\, \ox_{s=1}^n V_{m_s}\,, \qquad m_s\in\Z_{> 0}\,.
\eean
We have the weight decomposition $V =\oplus_{\nu\in\Z}V[\nu]$ consisting of  weight subspaces
\bean
\label{wdec}
V[\nu] \,=\, \{v\in V\ |\ (e_{11}-e_{22}) v = \nu v\,\}\,.
\eean
If $V[\nu]$ is nonzero, then
\bean
\label{mu m}
\nu = \sum_{s=1}^n m_s - 2k,
\eean
for some nonnegative integer $k$. 
 The dimension of $V[0]$ is positive if the sum $\sum_{s=1}^n m_s$ is even. 
 
 
 \subsection{Operator $\mc A(\mu)$} 
 \label{sec WG}
 
 
The $\slt$ Weyl group $W$ consists of two elements: identity and involution $\si$.
The projective action of $W$ on  $V_m$  is given by the formula
\bea
\si: v^m_k \mapsto  (-1)^kv^m_{m-k}
\eea
for any $k$. We have $\si^2=(-1)^m$. The Weyl group $W$  acts on the tensor product $V$ diagonally.


\smallskip

Following \cite{TV}, introduce
\bea
p(\mu) \,=\, \sum_{k=0}^\infty \,e_{21}^k e_{12}^k\,\frac1{k!}\, 
\prod_{j=0}^{k-1}\,\frac 1{\mu +e_{22}-e_{11}-j}\,,
\qquad \mu\in \C.
\eea 
The series $p(\mu)$ acts on $V_m$, since only a finite number of terms acts nontrivially.
The formula for the action of $p(\mu)$ on a basis vector $v^m_k$
becomes more symmetric if $\mu$ is replaced by $\mu+\frac {\nu}2-1$, where
$\nu=m-2k$ is the weight of $v^m_k$,
\bean
\label{ppro}
p\Big(\mu+\frac {\nu}2-1\Big) v^m_k\,=\,\prod_{j=0}^{k-1}\, \frac{\mu +\frac m2-j}{\mu -\frac m2 + j} \,v^m_k\,,
\eean
see \cite[Section 2.5]{TV}. 

\smallskip
The series $p(\mu)$ acts  on the tensor product $V$ in the standard way. 
Introduce the operator 
\bean
\label{mc A}
\mc  A(\mu) \,:\,V\,\to\,  V,
\quad
v \ \mapsto\ \si p(\mu) \,v\,.
\eean
The  operator $\mc A(\mu)$ is a meromorphic function of $\mu$. For any $\nu$, we have
$\mc A(\mu) V[\nu] \subset V[-\nu]$, and
$\lim_{\mu\to \infty} \mc A(\mu) = \si$\,.  The operator $\mc A(\mu)$ may be considered as a deformation of the Weyl group
operator $\si$.



\begin{lem}
\label{lem v iso}

For  $V= \ox_{s=1}^n V_{m_s}$ as in \eqref{ten V}, denote $M=\sum_{s=1}^n m_s$. 
Assume that $\mu \notin \frac M2+\Z$.
Then for any $\nu$ the  operator
\bean
\label{A mu}
\mc A\Big(\mu +\frac {\nu}2-1\Big)\Big\vert_{V[\nu]}\,:\,V[\nu]\,\to\,V[-\nu]
\eean
is an isomorphism of vector spaces. The composition of the operator
$\mc A\Big(\mu +\frac {\nu}2-1\Big)\Big\vert_{V[\nu]}$ and the operator
\bean
\label{A-mu}
\mc A\Big(-\mu -\frac {\nu}2+1\Big)\Big\vert_{V[-\nu]}\,:\,V[-\nu]\,\to\,V[\nu]
\eean
is the scalar operator on $V[\nu]$ of multiplication by $(-1)^M \frac{\mu -\nu/2}{\mu +\nu/2}$\,.



\end{lem}


\begin{proof} 
The  $\slt$ irreducible decomposition $V=\oplus _m V_m$ of the tensor product $V$
has the highest weights  $m$  of  the form  $m=M-2k$ for $k\in \Z_{\geq 0}$, only.
Now \eqref{A mu} is an isomorphism by formula \eqref{ppro}.
The statement on the composition is \cite[Theorem 10]{TV}.
\end{proof}

\begin{rem}
The operator $\mc A(\mu)$ is the (only) nontrivial element of the $\slt$ dynamical Weyl group of $V$, see definitions in
\cite{EV}.


\end{rem}

\subsection{KZB operators}
Introduce the following elements of $\glt\otimes\glt$\,,
\bea
&&
 \Om_{12} = e_{12}\otimes e_{21},   \phantom{aaaaaaaa}  \qquad \Om_{21} = e_{21}\otimes e_{12}, 
 \\
&&
\Om_0=e_{11}\otimes e_{11} + e_{22}\otimes e_{22},   \qquad
\Om = \Om_0 + \Om_{12} + \Om_{21}. 
\eea
The {\emph{KZB operators}} $H_0, H_1(z), \dots,H_n(z)$ 
are the following differential operators in variables $\la_1,\la_2$ acting on the space $\Fun$\,,
\bean
\label{tKZB} 
&&
\\
\notag
H_0 
&=& 
\frac{1}{4\pi \sqrt{-1}} (\partial_{\la_1}^2 + \partial_{\la_2}^2) + \frac{\pi \sqrt{-1}}{4} 
\sum_{s,t=1}^n \left[ \frac{1}{2} \Om^{(s,t)}_0 + \frac{1}{\sin^2(\pi\la)}
 \left(\Om_{12}^{(s,t)} + \Om_{21}^{(s,t)} \right) \right],
\\
\notag
H_s(z)
&=&
 - (e_{11}^{(s)} \partial_{\la_1} + e_{22}^{(s)} \partial_{\la_2}) + 
\sum_{t:\, t \ne s} \left[ \pi \sqrt{-1} \frac{z_t+ z_s}{z_t-z_s} \Om^{(s,t)} - \pi \cot (\pi \la) 
\left( \Om_{12}^{(s,t)} - \Om_{21}^{(s,t)} \right) \right],
\eean
cf. formulas in Section 3.4 of \cite{JV}.  The elliptic KZB operators were introduced in \cite{FW}.
In \eqref{tKZB} we consider the trigonometric degeneration of the elliptic KZB operators.

\smallskip
By \cite{FW} the operators $H_0, H_1(z), \dots, H_n(z)$ commute and 
$\sum_{s=1}^n H_s(z) = 0$.


\begin{rem}
The differential operator $H_0$ is the {\it Hamiltonian operator of the trigonometric quantum two-particle
system with spin space} $V$.
\end{rem}





\subsection{Coefficients $C_1(x)$, $C_2(x)$}


\begin{lem} 
We have
\bea
C_1(x) = \L^0_{11}(x)+\L^0_{22}(x) - \partial_{\la_1} - \partial_{\la_2}.
\eea
Hence the coefficient $C_1(x)$ acts by zero  on $\Fun$. 
\qed
\end{lem}

\begin{cor} The $RST$-operator \eqref{D} has the form
\bean
\label{mc C}
\mc C=\der_u^2+C_2(x)
\eean
as an operator on $\Fun$.
\end{cor}

\begin{thm} [\cite{ThV}]  
\label{S_2(X)} 

We have 
\bean
\label{C2}
&&
\\
&&
\notag
C_2(x) = -2\pi \sqrt{-1} H_0 - \sum_{s=1}^n 
\left[ 2\pi \sqrt{-1}  \frac{H_s(z)}{1-x/z_s} +  
4\pi^2 \Big(-\frac{c_2^{(s)}}{1-x/z_s} + \frac{c_2^{(s)}}{(1-x/z_s)^2}\Big) \right],
\eean
where $c_2 = e_{11}e_{22} - e_{12}e_{21} + e_{11}$ is a central element of $\glt$.
\end{thm}



\begin{proof}
This is the trigonometric degeneration of the elliptic version of this theorem, see 
  \cite[Theorem 4.9]{ThV}.
\end{proof}

\begin{cor} The dynamical Bethe algebra of $\Fun$ is generated by the identity operator and
the KZB operators $H_0, H_1(z), \dots, H_n(z)$.

\end{cor}

The commutativity of the KZB operators and formulas \eqref{mc C}, 
\eqref{C2} imply the commutativity $[C_2(a),C_2(b)]=0$  independently of  Theorem  
\ref{thm RST}.





\subsection{Weyl group invariance}
\label{sec Wgi}


The Weyl group acts on  $V[0]$
as explained in Section \ref{sec WG}.
Hence the Weyl group acts on $\Fun$  by the formula

\bean
\label{siF}
\si : \psi(\la) \mapsto \si (\psi (-\la)), \qquad
\psi \in\Fun\,.
\eean

\smallskip
\noindent
This action extends to a Weyl group action on
 $\End(\Fun)$, where for  
$T \in \End(\Fun)$ the operator $\si(T)$ is defined as the product 
$\si T\si^{-1}$ of the three elements of $\End(\Fun)$.


\begin{lem} [\cite{ThV}]
\label{weyl inv}
For any  $a\in \C-\{z_1,\dots,z_n\}$ the operator $C_2(a) \in \End(\Fun)$ is Weyl group invariant.
\end{lem}



\begin{proof}
By \cite{FW} the KZB operators  $H_0,H_1(z),\dots,H_n(z)$ are Weyl group invariant.
The lemma  follows from formula \eqref{C2}.
\end{proof}








\section{Eigenfunctions of $H_0$}
\label{sec 3}



\subsection{Trigonometric Gaudin operators}

The trigonometric $r$-matrix is defined by 
\bean
\label{r-matrix}
r(x) = \frac{\Om_+ x + \Om_-}{x-1},
\eean
where
$\Om_+ = \frac12 \Om_0 + \Om_{12}$,\ $\Om_- = \frac12 \Om_0 + \Om_{21}$.





For $\mu \in \C$ the trigonometric Gaudin operators acting on $V$ are defined as 
\bean
\label{trig Gaudin} 
\K_s( z,\mu) = \frac{\mu}{2}\,(e_{11}-e_{22})^{(s)} + \sum_{t:\, t \ne s} r^{(s,t)} (z_s/ z_t), \qquad s=1,\dots,n.
\eean
 Each operator $\K_s(z,\mu)$ preserves each of  the weight subspaces $V[\nu]$
and
 \bea
 [\K_s(z,\mu), \K_t( z,\mu)] = 0
 \eea
  for all $s,t$, see \cite{Ch, EFK}.


\subsection{Dynamical Bethe algebra of  $E(\mu)$} 
\label{sec BE}

Let 
\bea
\La=e^{-2\pi \sqrt{-1} \la}\,, \qquad\on{where}\   \la=\la_1-\la_2\,.
\eea
Let $\A$ be the algebra of functions in $\la$, which can be represented 
as meromorphic functions of $\La$ with poles only at the set $\{\La=1\}$.

\smallskip

For $\mu\in\C$ introduce  the $\A$-module $\A[\mu]$ 
of functions of the form
$e^{\pi \sqrt{-1} \mu\la}f,$ where $f\in\A$.  This module
is preserved by derivatives with respect to $\la_1,\la_2$. Therefore the KZB operator $H_0$ preserves the space
$\A[\mu]\otimes V[0]$.  Any $\psi \in \A[\mu]\otimes V[0]$
has the form
\bea
\psi(\la) = e^{\pi \sqrt{-1} \mu\la}\sum_{k=0}^\infty \La^k \psi^k, \qquad \psi^k\in V[0].
\eea

\begin{thm} [\cite{FV2}]
\label{prop}
Let $\mu \not\in\Z_{>0}$.
Then for any nonzero $v \in V[0]$, there exists a unique $\psi \in \A[\mu] \ox V[0]$ such that 
\bea
H_0\, \psi\, = \,\epsilon \,\psi\,,
\eea
for some $\epsilon\in\C$ and  $\psi^0=v$. Moreover,
$\epsilon = \,\pi \sqrt{-1}\, \frac{\mu^2}2$\,.




\end{thm}

Cf. \cite{JV}.  This function $\psi$ is denoted by $\psi_v$.
\smallskip

We denote by $E[\mu]$ the vector space of functions $\psi\in \A[\mu] \ox V[0]$ such that $H_0\,\psi =\,\pi \sqrt{-1}
\, \frac{\mu^2}2\,\psi$. For more  information on this space see \cite[Section 9]{JV}.







\begin{cor}
 For $\mu \notin \Z_{>0}$, the map 
\bean
\label{is VE}
V[0]\to E[\mu], \qquad v \mapsto \psi_v,
\eean
 is an isomorphism.
\end{cor}

\begin{thm} [\cite{JV}] 
\label{H_s eigen}
Let $\mu \notin \Z_{>0}$. Then for $s=1,\dots,n$, the KZB operators $H_s(z)$ preserve the space $E[\mu]$. 
Moreover, for any $v \in V[0]$ we have
\bea
H_s(z) \psi_v = \psi_w, 
\eea
where $w=-2\pi \sqrt{-1} \, \K_s (z,\mu)\, v$.
\end{thm}

\begin{thm} 
\label{B20 thm}

Let  $\mu \notin \Z_{>0}$, $V= \ox_{s=1}^n V_{m_s}$\,,  and $v\in V[0]$.  Then
\bea
C_2(x)\,\psi_v\,=\,\psi_w\,,
\eea
where
\bean
\label{B20}
\phantom{aaa}
w\,=\, (2\pi \sqrt{-1})^2 \bigg[-\frac{\mu^2}{4} + \sum_{s=1}^n \Big[\frac{m_s(m_s+2)/4+\K_s(z,\mu)}{1-x/ z_s}  
- \; \frac{m_s(m_s+2)/4}{(1-x/z_s)^2} \Big] \bigg] v\,.
\eean
\end{thm}


\begin{proof}
One computes the action of $C_2(X)$ on $\psi_v$ using Theorem \ref{S_2(X)}. The computation is based on 
Theorem \ref{H_s eigen} and the fact that $c_2$ acts on $V_{m_s}$ as multiplication by 
$-\frac{m_s(m_s+2)}{4}$.
\end{proof}


By Theorem \ref{H_s eigen} the subspace $E[\mu]\subset \Fun$ is invariant with respect to the 
action of the dynamical Bethe algebra. The restriction
of the dynamical Bethe algebra  to $E[\mu]$ is called the
{\it dynamical Bethe algebra} of $E[\mu]$ and  denoted by
$\B(z; E[\mu])$.  

\vsk.2>
Notice that $E[\mu]$ 
is a finite-dimensional vector space of dimension $\dim V[0]$. The space $E[\mu]$
does not depend on $ z$,
since the KZB operator $H_0$ does not depend on $ z$. 
The algebra $\B(z; E[\mu])$ is generated by the identity operator and
the KZB operators $H_1(z), \dots, H_s(z)$ and does depend on $ z$.

\subsection{Two-particle scattering matrix}

\begin{thm}
[{\cite[Lemma 6.2]{FV2}}]

\label{thm x=A}

For  $\mu\notin\Z$,  the action \eqref{siF} of the Weyl group involution $\si$ on $\Fun$ identifies
the spaces $E[\mu]$ and $E[-\mu]$. More precisely, for any $v\in V[0]$ we have
\bean
\label{vAw}
\si (\psi_v^\mu(-\la)) \,=\, \psi_{\mc A(\mu-1)v}^{-\mu}(\la),
\eean
where $\psi_v^\mu(\la)$ is the element of $E[\mu]$ with initial term $v$ and 
$\psi_{\mc A(\mu-1)v}^{-\mu}(\la)$ is the element of $E[-\mu]$ with 
initial term $\mc A(\mu-1)v$. Here $\mc A(\mu-1):V[0]\to V[0]$ is the vector isomorphism, defined in \eqref{mc A}.

\end{thm} 

\begin{proof} 
Formula \eqref{vAw} is proved in the example next to Lemma 6.2 in \cite{FV2}.
\end{proof}










\section{Quantum trigonometric Gaudin model}
\label{sec 4}


\subsection{Universal differential operator}
Let $V= \ox_{s=1}^n V_{m_s}$. 
Introduce a $2\times 2$-matrix
\bean
\label{matM}
\mc M = \begin{bmatrix} \mc M_{11} \; & \; \mc M_{12} \\
                              \mc M_{21} \; & \; \mc M_{22} \\
        \end{bmatrix} = -2 \pi \sqrt{-1} \sum_{s=1}^n r^{(0,s)}(x/z_s),
\eean
where $r(x)$ is the trigonometric $r$-matrix defined in \eqref{r-matrix}.
More explicitly,
\bea
\M = \begin{bmatrix} 
2\pi \sqrt{-1} \sum_{s=1}^n \frac{1}{1-x/ z_s} e_{11}^{(s)} - \pi \sqrt{-1} e_{11} \; & 
\; 2 \pi \sqrt{-1} \sum_{s=1}^n \frac{1}{1-x/ z_s} e_{21}^{(s)} - 2 \pi \sqrt{-1} e_{21}  \\
2 \pi \sqrt{-1} \sum_{s=1}^n \frac{1}{1-x/ z_s} e_{12}^{(s)} \; & 
\; 2 \pi \sqrt{-1} \sum_{s=1}^n \frac{1}{1-x/ z_s} e_{22}^{(s)} - \pi \sqrt{-1} e_{22} \\
\end{bmatrix}.
\eea
 The {\it universal (trigonometric) differential operator} for $V$ with parameter $\mu\in\C$
is defined by the formula 
\bean
\label{Dcdet}
\D = \cdet \begin{bmatrix} 
\der_u  -\pi \sqrt{-1} \mu + \M_{11} \; & \; \M_{12} 
\\
      \M_{21} \; & \; \der_u + \pi \sqrt{-1} \mu + \M_{22} 
\\
                    \end{bmatrix}.
\eean
Write the operator $\D$ in the form
\bea
\D = \der_u^2 +D_1(x)\der_u + D_2(x),
\eea
where $D_1(x)$, $D_2(x)$ are $\End(V)$-valued functions of $x$. It is clear that
$\D$ commutes with the action on $V$ of the Cartan subalgebra of $\slt$. In particular,  it means that
$D_1(x)$, $D_2(x)$ preserve the weight decomposition of $V$.


\subsection{Coefficients and Gaudin operators}
\label{sec ctG}


\begin{thm} 
\label{Dthm}
We have  $D_1(x)=0$ and $(2\pi \sqrt{-1})^{-2} D_2(x)$ equals
\bean
\label{DD2}
\phantom{aaaqa}
-\frac{\mu^2+\mu (e_{11} - e_{22}) -  e_{11} e_{22}}4
 + \sum_{s=1}^n \Big[ \frac{m_s(m_s+2)/4 + \K_s ( Z,\mu)}{1-x/z_s} - \frac{m_s(m_s+2)/4}{(1-x/z_s)^2} \Big] .
\eean


\end{thm}






\begin{proof}
The proof is by straightforward calculation. We have
\bea
\D
&=&\Big(\der_u - \pi \sqrt{-1} \mu + 2\pi \sqrt{-1} \sum_{s=1}^n \frac{1}{1-x/ z_s} e_{11}^{(s)} - \pi \sqrt{-1} e_{11}\Big)
 \\
&
\times  
&
\Big(\der_u + \pi \sqrt{-1} \mu + 2 \pi \sqrt{-1} \sum_{s=1}^n \frac{1}{1-x/ z_s} e_{22}^{(s)} - \pi \sqrt{-1} e_{22}\Big) 
\\
&-&
\Big( 2 \pi \sqrt{-1} \sum_{s=1}^n \frac{1}{1-x/ z_s} e_{12}^{(s)}\Big)
 \Big( 2 \pi \sqrt{-1} \sum_{s=1}^n \frac{1}{1-x/ z_s} e_{21}^{(s)} - 2 \pi \sqrt{-1} e_{21} \Big) .
\eea
Then
\bea
D_1(x) =  2 \pi \sqrt{-1} \sum_{s=1}^n \frac{e_{11}^{(s)}+e_{22}^{(s)}}{1-x/ z_s} - \pi \sqrt{-1} (e_{11} + e_{22}) = 0.
\eea
Since  $x=e^{-2 \pi \sqrt{-1} u}$ and $\der_u = -2 \pi \sqrt{-1} \,x \der_x$, the coefficient 
of $(2\pi \sqrt{-1} )^{-2}D_2(x)$ equals
\bean
\label{D/}
&&
 -\frac{\mu^2}{4} - \sum_{s=1}^n \frac{e_{22}^{(s)}}{(1-x/z_s)^2} + \sum_{s=1}^n \frac{e_{22}^{(s)}}{1-x/z_s}
  + \frac{\mu}{2} \sum_{s=1}^n \frac{e_{11}^{(s)}-e_{22}^{(s)}}{1-x/z_s}
   - \frac{\mu}{4}(e_{11}-e_{22}) 
   \\
\notag
&&
+ \sum_{s=1}^n \frac{e_{11}^{(s)} e_{22}^{(s)}}{(1-x/z_s)^2}
 + \sum_{s=1}^n \Big( \sum_{t: \, t\ne s} \frac{e_{11}^{(s)} e_{22}^{(t)} + e_{11}^{(t)} e_{22}^{(s)}}{1- z_s/ z_t} \Big) 
 \frac{1}{1-x/z_s} - \sum_{s=1}^n \frac{e_{11}^{(s)} e_{22}^{(s)}}{1-x/z_s} 
 \\
\notag
 &&
- \sum_{s=1}^n \Big( \sum_{t: \, t\ne s} e_{22}^{(t)} \Big) \frac{e_{11}^{(s)}}{1-x/z_s}
 + \frac14 e_{11} e_{22} - \sum_{s=1}^n \frac{e_{12}^{(s)} e_{21}^{(s)}}{(1-x/z_s)^2} \phantom{aaaaaaaaaa} 
 \\
\notag
 &&
- \sum_{s=1}^n \Big( \sum_{t: \, t\ne s} \frac{e_{12}^{(s)}e_{21}^{(t)} + e_{12}^{(t)} e_{21}^{(s)}}{1- z_s/ z_t} \Big) 
\frac{1}{1-x/z_s} + 
\sum_{s=1}^n \frac{e_{12}^{(s)} e_{21}^{(s)}}{1-x/z_s} + \sum_{s=1}^n \Big( \sum_{t: \, t\ne s} e_{21}^{(t)} \Big) \frac{e_{12}^{(s)}}{1-x/z_s} .
\eean 
The constant term  in \eqref{D/} equals $-\frac{\mu^2+\mu (e_{11} - e_{22}) -  e_{11} e_{22}}4$.
For $s=1,\dots,n$, the coefficient of $\frac{1}{1-x/z_s}$ in \eqref{D/} equals
\bea
&&
-c_2^{(s)}\, +\, \frac{\mu}{2}\, (e_{11}-e_{22})^{(s)} \,+ 
\,\sum_{t: \, t \ne s} \frac{e_{12}^{(s)} e_{21}^{(t)}  z_s 
+ e_{12}^{(t)} e_{21}^{(s)}  z_t}{ z_s -  z_t} + e_{22}^{(s)} \big( e_{22} - e_{22}^{(s)} \big)
\\
&&
\phantom{aaa}
 = \ -c_2^{(s)} \,+ \,\K_s( z,\mu) \,= \,m_s(m_s+2)/4\, +\, K_s(z, \mu).
\eea
The coefficient of $\frac{1}{(1-x/z_s)^2}$ in \eqref{D/} equals 
\bea
-e_{22}^{(s)} + e_{11}^{(s)}e_{22}^{(s)} - e_{12}^{(s)}e_{21}^{(s)} = (e_{11} e_{22} - e_{12}e_{21} + e_{11})^{(s)} = c_2^{(s)}.
\eea
The theorem is proved.
\end{proof}

\begin{lem}
\label{D2 comm}
For any $a,b\in \C-\{z_1,\dots,z_n\}$ the operators
 $D_2(a), D_2(b)\in \End(V)$ commute.
They also commute with  the $\glt$ Cartan subalgebra.


\end{lem}

\begin{proof} 
It is clear that the trigonometric Gaudin operators $\K_s(z,\mu)$ commute with the 
$\glt$ Cartan subalgebra.  Now the lemma follows from the commutativity of
trigonometric Gaudin operators.
\end{proof}


\begin{cor}
\label{cor D2 mu}


Choose a weight subspace $V[\nu]$ of $V$. Then 
$(2\pi \sqrt{-1})^{-2} D_2(x)$ restricted to $V[\nu]$ equals 
\bean
\label{D2}
 -\, \frac{(\mu+\nu/2)^2}{4} + \sum_{s=1}^n \Big[ \frac{m_s(m_s+2)/4 + \K_s ( z,\mu)}{1-x/z_s} - \frac{m_s(m_s+2)/4}{(1-x/z_s)^2} \Big] .
\eean
\qed
\end{cor}

The commutative algebra of  operators on  $V[\nu]$  generated by the identity operator
 and the operators $\{ D_2(a)\ |\ a\in \C -\{z_1,\dots,z_n \}\}$
is called the {\it  Bethe algebra} of $V[\nu]$ with parameter $\mu$ and denoted by
$\B(z;\mu; V[\nu])$.  The  Bethe algebra  $\B(z;\mu; V[\nu])$ 
is generated by the identity operator and the trigonometric Gaudin operators 
$\K_1 ( z,\mu), \dots,\K_n ( z,\mu)$.

\smallskip

The pair  $(V[\nu],  \B(z;\mu; V[\nu]))$ is called the {\it trigonometric Gaudin model on $V[\nu]$}.

\begin{cor}
\label{iso B} 

If $\mu\notin \Z_{>0}$,  the isomorphism $V[0]\to E[\mu]$ in \eqref{is VE} induces an isomorphism
$\B(z;\mu;V[0])\to B(z;E[\mu])$ between the  Bethe algebra of $V[0]$ with parameter $\mu$ and the
dynamical Bethe algebra of the space $E[\mu]$.



\end{cor}

\begin{proof}
The corollary is proved by comparing formulas \eqref{B20} and \eqref{D2}.
\end{proof}



\subsection{Gaudin operators and Weyl group}



\begin{lem}
[{\cite[Lemma 18]{TV}},  cf. {\cite[Lemma 5.5]{MV2}}]
\label{lem AK}
For any weight subspace $V[\nu]$, any $v\in V[\nu]$, $s=1,\dots,n$, we have
\bean
\label{AK}
\mc A\Big(\mu +\frac {\nu}2-1\Big) \K_s(z,\mu) v\,=\,
 \K_s(z, - \mu)\mc A\Big(\mu +\frac {\nu}2-1\Big) v.
 \eean
 
 
\end{lem}


\begin{thm}
\label{thm isom mu}

For  $V= \ox_{s=1}^n V_{m_s}$ as in \eqref{ten V}, denote $M=\sum_{s=1}^n m_s$. 
Assume that $\mu \notin \frac M2+\Z$. 
Then for any $\nu$ the isomorphism of vector spaces
\bean
\label{A mu n}
\mc A\Big(\mu +\frac {\nu}2-1\Big)\Big\vert_{V[\nu]}\,:\,V[\nu]\,\to\,V[-\nu]
\eean
induces  an isomorphism of  Bethe algebras
\bean
\label{B iso}
\phantom{aaa}
\B(z;\mu; V[\nu]) \to  \B(z;-\mu; V[-\nu]), \quad
T\mapsto\mc A\Big(\mu +\frac {\nu}2-1\Big) T
\mc A\Big(\mu +\frac {\nu}2-1\Big)^{-1}.
\eean


\end{thm}

\begin{proof}
The theorem is a corollary of Lemmas    \ref{lem v iso} and \ref{lem AK}.
\end{proof}


\subsection{Commutative diagram}

Assume that $\mu\notin\Z$ and $M$ is even. Then $V[0]$ is a nonzero weight subspace.

\smallskip

Consider the $\B(z;\mu;V[0])$-module $V[0]$ and $\B(z;-\mu;V[0])$-module $V[0]$.
 Consider the $\B(z; E[\mu])$-module $E[\mu]$ and $\B(z; E[-\mu])$-module $E[-\mu]$.
Consider the diagram relating these modules
\bean
\label{comD}
\begin{tikzcd}
(\B(z;\mu;V[0]),\, V[0])   \arrow[r, ] \arrow[d, ] & (\B(z;-\mu;V[0]), \,V[0]) \arrow[d, ] 
\\
(\B(z; E[\mu]),\, E[\mu]) \arrow[r,   ]  & (\B(z; E[-\mu]),\, E[-\mu])
\end{tikzcd}\ \  .
\eean

\noindent
Here the map
$(\B(z;\mu;V[0]),\, V[0]) \to (\B(z;-\mu;V[0]), \,V[0])$ is the module isomorphism of Theorem \ref{thm isom mu}.
The map 
$(\B(z; E[\mu]),\, E[\mu]) \to (\B(z; E[-\mu]),\, E[-\mu])$ is the module isomorphism induced by the action
of the Weyl involution $\si$
and the fact that the $RST$-operator is Weyl group invariant, see Lemma \ref{weyl inv}.
The maps 
$(\B(z;\mu;V[0]),\, V[0])\to (\B(z; E[\mu]),\, E[\mu])$ and  
$(\B(z;-\mu;V[0]),\, V[0])\to (\B(z; E[-\mu]),\, E[-\mu])$
are the module isomorphisms of Corollary \ref{iso B}.


\begin{thm}
\label{thm tra}
Diagram \eqref{comD} is commutative.

\end{thm}

\begin{proof}
The theorem follows from Theorems \ref{H_s eigen}, \ref{thm x=A}, \ref{thm isom mu}.
\end{proof}








\section{Bethe ansatz}
\label{sec 5}






\subsection{Bethe ansatz equations for  triple $(z;\mu;V[\nu])$}
\label{sec BAE}

Let
$V= \ox_{s=1}^n V_{m_s}$, as in \eqref{ten V},
and $M=\sum_{s=1}^n m_s$.  Let $V[\nu]$ be a nonzero 
weight subspace of $V$.
Then $\nu = M-2m$ for some nonnegative integer $m$.
\smallskip

Let $z=\{z_1,\dots,z_n\}\subset \C^\times$ be a set of nonzero pairwise distinct numbers,
as in Section \ref{ssec not}. Let $\mu\in\C$.
\smallskip

Introduce the  {\emph{master function}} of the variables $t=(t_1,\dots,t_m), \mu, z$,
\bea
\Phi(t,z,\mu) \, 
&=&
  \, \Big(1-\mu + \frac{\nu}{2} \Big) \sum_{i=1}^m \,\ln t_i \,+ 
  \,\sum_{s=1}^n \,\frac{m_s}{4}\, (2\mu + m_s - \nu) \,\ln z_s     
 \\
& + &
2 \sum_{1 \leqslant i < j \leqslant m} \ln(t_i - t_j) 
- \sum_{i=1}^m  \sum_{s=1}^n m_s \ln(t_i - z_s) + 
\sum_{1 \le s < r \leqslant n} \frac{m_s m_r}{2} \ln(z_s - z_r).
\eea
The {\emph{Bethe ansatz equations}} are the critical point equations for
the master function $\Phi(t,z,\mu)$ with respect to the variables $t_1,\dots,t_m$,
\bean
\label{tr.BAE}
\frac{1-\mu+\nu/2}{t_i} \,+ \,\sum_{j \ne i} \frac{2}{t_i-t_j}\, -\, \sum_{s=1}^n \frac{m_s}{t_i-z_s}\, = \,0, 
\qquad i=1,\dots,m.
\eean
The master function $\Phi(t,z,\mu)$ is the trigonometric degeneration 
 of the elliptic master function considered in Section 5 of \cite{ThV}, see also \cite{FV1, MaV}.
 
\smallskip
The symmetric group $S_m$ acts on the critical set. If $(t_1^0,\dots, t^0_m; z;\mu)$
 is a solution of the Bethe ansatz equations, then for any $\rho\in S_m$, 
 the point $(t_{\rho(1)}^0,\dots, t^0_{\rho(m)};z;\mu)$ is also a solution.

\subsection{Bethe vectors}
Define
\bea
\mc{C} &=& \{\ell = (\ell_1,\dots,\ell_n)\in\Z_{\geq 0}\ |\  \ell_s\leq m_s, \, \ell_1+\dots+\ell_n=m\},
\\
\om_{\ell} (t,z) 
&=&
 \Sym \; \Big[
\prod_{s=1}^n  \prod_{i=\ell_1+\dots+\ell_{s-1}+1}^{\ell_1+\dots+\ell_s} \frac{1}{t_i-z_s}\Big],
\eea
where $\Sym f(t_1,\dots,t_m) = \sum_{\rho \in S_m} f(t_{\rho(1)}, \dots,t_{\rho(m)})$.
Introduce the {\it weight function} 
\bean
\label{wght_f}
\om(t,z) \,=\, \sum_{\ell \in \mc{C}} \,\om_{\ell} (t,z)\, 
v^{m_1}_{\ell_1}\otimes \dots\otimes v^{m_n}_{\ell_n}\,,
\eean
see Section \ref{sec TP}.  This weight function see in \cite{MV2}, also in \cite{JV, MaV, SV}.
\smallskip

Notice that the weight function is a symmetric function of the variables $t_1,\dots,t_m$.

\smallskip
 If $(t^0\!\,;z;\mu)$ is a solution of the Bethe ansatz equations \eqref{tr.BAE}, 
 then the vector $\om(t^0\!, z)$ is called the {\it Bethe vector}.  


\begin{thm}
[\cite{MTV6,V}]
\label{thm Bnon}


Let $(t^0\!; z; \mu)$ be a solution of the Bethe ansatz equations \eqref{tr.BAE}.
Then the Bethe vector $\om(t^0\!, z)$ is nonzero.

\end{thm}



\begin{thm} [\cite{FV1, JV}, cf. \cite{RV}] 
\label{eigenv}



Let $(t^0 ; z; \mu)$ be a solution of the Bethe ansatz equations \eqref{tr.BAE}.
 Then the Bethe vector $\om(t^0\!, z)$ is an eigenvector of the trigonometric Gaudin operators,
\bea
\K_s(z,\mu) \, \om(t^0\!,z) \,=\, z_s\,\frac{\der \Phi}{\der z_s}(t^0\!, z,\mu) \, \om(t^0\!, z)\,,
\qquad s=1,\dots,n.
\eea
\end{thm}


Denote 
\bean
\label{k_s}
\phantom{aaa}
k_s(t^0\!,z,\mu)
&=&
 z_s\,\frac{\der \Phi}{\der z_s}(t^0\!,z,\mu)
 \\
 \notag
 &=&
  \frac{m_s}{2} \Big[ (\mu - \nu/2 + m_s/2)\, +\, 
 \sum_{p: \,p \ne s} m_p \,\frac{z_s}{z_s-z_p}
\, + \,
2 \sum_{i=1}^m \frac{z_s}{t_i^0-z_s} \Big]. 
 \eean

\subsection{Bethe vectors and coefficient $D_2(x)$}

\begin{lem}
\label{hatB2 lem}

If $(t^0\!\,; z; \mu)$ is a solution of the Bethe ansatz equations 
\eqref{tr.BAE}, then the Bethe vector $\om(t^0\!,z)$ is an eigenvector of
all operators of the Bethe algebra $\B(z;\mu; V[\nu])$. In particular, the 
operator $D_2(x)$ acts on $\om(t^0\!,z)$ by multiplication by the scalar
\bean
\label{sc D2}
\phantom{aaaa}
(2\pi \sqrt{-1})^2\Big[- \frac{(\mu+\nu/2)^2}{4} + \sum_{s=1}^n \Big[ \frac{m_s(m_s+2)/4 + k_s (t^0\!, z,\mu)}{1-x/z_s}
 - \frac{m_s(m_s+2)/4}{(1-x/z_s)^2} \Big]\Big] .
\eean

\end{lem}



\begin{proof}
The lemma follows from Theorem \ref{eigenv} and Corollary \ref{cor D2 mu}.
\end{proof}

For  a  solution  $(t^0\!; z; \mu)$  of the Bethe ansatz equations 
\eqref{tr.BAE}, we introduce the {\it fundamental differential operator}
\bean
\label{hatD}
\mc E_{\,t^0\!,z,\mu} = \der_u^2 + E_2(x, t^0\!,z,\mu),
\eean
where the function $E_2(x, t^0\!,z,\mu)$ is given by formula \eqref{sc D2}.


\subsection{Basis of Bethe vectors}

The Bethe ansatz method is the method to construct eigenvectors of commuting operators,
see Lemma  \ref{hatB2 lem} as an example. The standard problem 
is to determine if  the Bethe ansatz method gives a basis of eigenvectors  of the vector
space, on which the commuting operators act.
In the case of Lemma  \ref{hatB2 lem} the answer is positive.




\begin{lem} 
\label{basis}

Let $\mu\notin \frac\nu 2+\Z_{>0}$.  Then for generic $z=\{z_1,\dots,z_n\}\subset \C^\times$,
the set of solutions $(t^0;z;\mu)$ 
 of   system \eqref{tr.BAE}
of the Bethe ansatz equations
 is such that
 the corresponding Bethe vectors $\om(t^0\!,z,\mu)$ form a basis of the space $V[\nu]$.

\end{lem}

\begin{proof}
Here the word generic means 
that the subset of all acceptable  sets $\{z_1,\dots,z_n\}$ forms a Zariski open subset in the space
of all sets $\{z_1,\dots,z_n\}$. The proof of the lemma is standard. It is a modification of \cite[Theorem 8]{ScV},
cf. \cite[Section 4.4]{MV1}, \cite[Section 5.4]{MV2}, \cite[Section 10.6]{MTV1}.
\end{proof}



\section{Function $w(x)$ in the kernel of  $\mc E_{\,t^0;z;\mu}$}
\label{sec 6}





Let $(t^0;z;\mu)$  be a solution of system \eqref{tr.BAE} of  Bethe ansatz equations, where
$t^0=(t^0_1,\dots,t^0_m)$. Define 
\bean
\label{Y&u}
y(x) \,=\, \prod_{i=1}^m (x- t^0_i), \qquad  
w(x) \,=\,y(x)\, x^{\frac{\nu/2-\mu}{2}} \,\prod_{s=1}^n (x-z_s)^{-m_s/2}\,.
\eean

\begin{thm}
\label{Dthm}


We have
\bean
\label{Dform}
\mc E_{\,t^0;z;\mu}\, =\, \big( \der_u + (\ln w)' \big) \big( \der_u - (\ln w)' \big).
\eean
where  $' = \der/\der u$. In other words,  
\bean
\label{hatB2form}
E_2(x, t^0\!,z,\mu)  = - (\ln w)'' - ((\ln w)')^2.
\eean
\end{thm}

\begin{rem}
For $\nu=0$ this statement is the trigonometric degeneration of its elliptic version \cite[Theorem 5.3]{ThV}.
\end{rem}

\begin{proof} Recall that $\partial_u\, =\, -2\pi \sqrt{-1} \,x \partial_x$. We have 
\bea
(\ln w)' 
&=&
 -2\pi \sqrt{-1} \,\Big[-  \frac{\nu/2+\mu }{2} + \sum_{i=1}^m \frac{t^0_i}{x-t^0_i}
  - \frac12 \sum_{s=1}^n \frac{z_s m_s}{x- z_s} \Big], 
\\
(\ln w)'' 
&=&
 (2 \pi \sqrt{-1} )^2 \Big[ - \sum_{i=1}^m \frac{t^0_i}{x-t^0_i} 
 - \sum_{i=1}^m \frac{(t^0_i)^2}{(x-t^0_i)^2} 
 + \frac12 \sum_{s=1}^n \frac{z_s m_s}{x-z_s} 
 + \frac12 \sum_{s=1}^n \frac{z_s^2 m_s}{(x-z_s)^2} \Big].
\eea
Hence, $(2\pi \sqrt{-1})^{-2}(- (\ln w)'' - ((\ln w)')^2)$  equals
\bea
&&  \sum_{i=1}^m \frac{t^0_i}{x-t^0_i} 
 + \sum_{i=1}^m \frac{(t^0_i)^2}{(x-t^0_i)^2} 
 - \frac12 \sum_{s=1}^n \frac{z_s m_s}{x-z_s} 
 - \frac12 \sum_{s=1}^n \frac{z_s^2 m_s}{(x-z_s)^2} 
 \\
 && 
 \phantom{aa}
- \frac14 (\mu+\nu/2)^2 - \sum_{i=1}^m \frac{(t_i^0)^2}{(x-t^0_i)^2} 
- 2 \sum_{i=1}^m  \sum_{j:\,j \ne i} \frac{t^0_it^0_j}{t^0_i - t^0_j}  \frac{1}{x-t^0_i} 
\\
&&
 \phantom{aa}
- \frac14 \sum_{s=1}^n \frac{z_s^2 m_s^2}{(x-z_s)^2}
- \frac 12\sum_{s=1}^n   \sum_{p:\,p \ne s} \frac{z_s z_pm_s m_p}{z_s-z_p} 
 \frac{1}{x-z_s} + (\mu+\nu/2) \sum_{i=1}^m \frac{t^0_i}{x-t^0_i} 
\\
&&
 \phantom{aa}
-\frac12\,(\mu+\nu/2)\sum_{s=1}^n \frac{z_s m_s}{x-z_s}
+ \sum_{i=1}^m \sum_{s=1}^n \frac{t^0_i z_s m_s}{t^0_i-z_s}  \frac{1}{x-t^0_i} -
\sum_{i=1}^m  \sum_{s=1}^n \frac{t^0_i z_s m_s}{t^0_i-z_s}  \frac{1}{x-z_s} .
\eea
In the expression above for each $i=1,\dots,m$ the coefficient of $\frac{1}{(x-t^0_i)^2}$ 
equals zero. The coefficient of $\frac{1}{x-t^0_i}$ equals
\bea
&&
(\mu+\nu/2+1)t^0_i - 
\sum_{j:\,j \ne i} \frac{2t^0_it^0_j}{t^0_i-t^0_j} 
+ \sum_{s=1}^n \frac{t^0_i z_sm_s}{t^0_i-z_s} 
\\
&&
 \phantom{aa}
= t^0_i 
\Big[ \mu+\nu/2+1  +  2 \sum_{j:\,j \ne i} \frac{t^0_j-t^0_i+t^0_i}{t^0_j-t^0_i}
 - \sum_{s=1}^n \frac{(z_s - t^0_i+t^0_i)m_s}{z_s-t^0_i} \Big] 
 \\
&&
 \phantom{aa}
= t^0_i \Big[ \mu+\nu/2+1  + 2(m-1) + \sum_{j:\,j \ne i} \frac{2t^0_i}
{t^0_j-t^0_i} - \sum_{s=1}^n m_s - \sum_{s=1}^n \frac{t^0_i m_s}{z_s-t^0_i} \Big]
\\
&&
 \phantom{aa}
= -(t^0_i)^2 \Big[ \frac{1-\mu+\nu/2}{t^0_i} + 
\sum_{j:\,j \ne i} \frac{2}{t^0_i-t^0_j} - 
\sum_{s=1}^n \frac{m_s}{t^0_i-z_s} \Big] = 0, 
\eea
where the last equality follows from
the Bethe ansatz equations \eqref{tr.BAE}. For each $s=1,\dots,n$
 the coefficient of $\frac{1}{(1-x/z_s)^2}$ equals
  $-m_s(m_s+2)/4$. The coefficient of $\frac{1}{1-x/z_s}$ equals
\bea
&&
\frac12 m_s + \frac12 m_s \sum_{p:\,p \ne s} \frac{z_p m_p}{z_s-z_p} 
+ \frac12 (\mu+\nu/2) m_s + \sum_{i=1}^m \frac{t^0_i m_s}{t^0_i-z_s} 
\\
&&
 \phantom{aa}
= \frac{m_s}{2} \Big[ 1+ \mu+ \nu/2 - \sum_{p:\,p \ne s} 
\frac{(z_p-z_s+z_s)m_p}{z_p-z_s} + 2 \sum_{i=1}^m \frac{t^0_i-z_s+z_s}{t^0_i-z_s} \Big] 
\\
&&
 \phantom{aa}
= \frac{m_s}{2} \Big[ 1+ \mu+ \nu/2 - \sum_{p \ne s} m_p - \sum_{p:\,p \ne s} 
\frac{z_s m_p}{z_p-z_s} +
 2\sum_{i=1}^m1 + 2 \sum_{i=1}^m \frac{z_s}{t^0_i-z_s} \Big] 
\\
&&
 \phantom{aa}
= \frac{m_s}{2} \Big[ (1+ \mu - \nu/2 + m_s) + \sum_{p:\,p \ne s}
 \frac{z_s m_p}{z_s-z_p} + 2 \sum_{i=1}^m \frac{z_s}{t^0_i-z_s} \Big] 
\\
&&
 \phantom{aa}
= m_s(m_s+2)/4 + \frac{m_s}{2} \Big[ (\mu - \nu/2 + m_s/2) + \sum_{p:\,p \ne s}
 m_p \frac{z_s}{z_s-z_p} + 2 \sum_{i=1}^m \frac{z_s}{t^0_i-z_s} \Big] 
 \\
&&
 \phantom{aa}
= m_s(m_s+2)/4 + k_s(t^0,z,\mu), 
\eea
where $k_s(t^0,z,\mu)$ are defined in \eqref{k_s}. Hence, $ E_2 = - (\ln w)'' - ((\ln w)')^2$.
\end{proof}

\begin{cor}
The function $w(x)$  lies in the kernel of $\mc E_{\,t^0;z;\mu}$.
\end{cor}



\section{Function $\tilde w(x)$ in the kernel of $\mc E_{\,t^0;z;\mu}$}
\label{sec 7}


\subsection{Wronskian}


 \label{ssec Wr}
The {\emph{Wronskian}} of two  functions $f(a)$ and $g(a)$ is 
\bean
\label{Wr}
\Wr_a (f,g) = f \frac{d g}{d a} - \frac{d f}{d a} g.
\eean
We have 
\bean
\label{h^2}
\Wr_a(hf, hg) = h^2 \Wr(f,g)
\eean
for any function $h(a)$.

\subsection{Wronskian  and Bethe ansatz equations}

\begin{lem} 
\label{lem WBA}

The following two statements hold:

\begin{enumerate}

\item[(i)]

Let $\mu\notin \frac\nu2+\Z_{\geq 0}$. Let   $(t^0;z;\mu)$ be a solution
 of the Bethe ansatz equations \eqref{tr.BAE} and
$y(x)=  \prod_{i=1}^m(x-t^0_i)$.  Then there exists a unique 
monic polynomial $\tilde y(x)$ of degree $M-m$, such that 
\bean
\label{Wr.eqn22}
\Wr_x (y(x),x^{\mu-\nu/2} \tilde y(x)) \,=\,\const\, x^{\mu-\nu/2-1} \prod_{s=1}^n (x-z_s)^{m_s},
\eean
where $\const$ is a nonzero constant.


\item[(ii)]

Let  $\mu\ne \frac\nu2$. Assume that $y(x)=  \prod_{i=1}^m(x-t^0_i)$ is a polynomial
with distinct roots
 such that   $y(z_s)\ne 0$,  $s=1,\dots,n$. Assume that
 there exists a polynomial $\tilde y(x)$ such that
 equation \eqref{Wr.eqn22} holds. Then $(t^0_1,\dots,t^0_m;z;\mu)$ is a solution of the Bethe ansatz equations \eqref{tr.BAE}.
\end{enumerate}

\end{lem}

\begin{proof}
This lemma is a reformulation of Theorem 3.2 and Corollary 3.3 in \cite{MV2}.
\end{proof}

\subsection{Function $\tilde w(x)$} \label{ssec utild}


Recall that we have  a solution $(t^0; z; \mu)$ of the Bethe ansatz equations, the differential
operator $\mc E_{\,t^0\!,z,\mu}$  and the function
$w(x) \,=\,y(x)\, x^{\frac{\nu/2-\mu}{2}} \,\prod_{s=1}^n (x-z_s)^{-m_s/2}$, where
$y(x) =\prod_{i=1}^m (x- t^0_i)$.


\begin{thm}
\label{quasipthm}


Let $\mu\notin \frac\nu2+\Z_{\geq 0}$. Then there exists a unique 
monic polynomial $\tilde y(x)$ of degree $M-m$, such that the function
\bean
\label{utild}
\tilde w(x) \,= \,\tilde y(x)\, x^{\frac{\mu-\nu/2}{2} }\, \prod_{s=1}^n (x-z_s)^{-m_s/2}
\eean 
lies in the kernel of $\mc E_{\,t^0; z;\mu}$. The functions $w(x), \tilde w(x)$ span the kernel of
 $\mc E_{\,t^0 ; z; \mu}$.

\end{thm}







\begin{proof}


The differential operator $\mc E_{\,t^0; z; \mu}$ introduced in \eqref{hatD} has no first order term. 
Hence the kernel of $\mc E_{\,t^0; z; \mu}$ consists of  the functions $\tilde w(x)$ satisfying  the equation
\bean
\label{WSo}
\Wr_u (w(x), \tilde w(x)) = \const\,.
\eean
 By Lemma \ref{lem WBA},  there exists a unique monic polynomial $\tilde y(x)$ of degree $M-m$, such that 
equation \eqref{Wr.eqn22} holds.
Dividing both sides of \eqref{Wr.eqn22} by $x^{\mu-\nu/2} \prod_{s=1}^n (x-z_s)^{m_s}$ we obtain
\bean
\label{Wr.eqn2}
\phantom{aaa}
\Wr_x\! \Big( y(x)\, x^{\frac{\nu/2-\mu}{2}} \prod_{s=1}^n (x-z_s)^{-m_s/2},\,
\tilde y(x) \,x^{\frac{\mu-\nu/2}{2} } \prod_{s=1}^n (x-z_s)^{-m_s/2} \Big) = \,\const\,x^{-1}.
\eean
Recall that $x=e^{-2\pi \sqrt{-1} u}$, hence $\der_u  = -2\pi \sqrt{-1}\, x \der_x$. This implies equation \eqref{WSo}.
The theorem is proved.
\end{proof}

\subsection{Bethe ansatz equations for triples $(z;\mu; V[\nu])$ and $(z; -\mu; V[-\nu])$}

\begin{lem}
\label{lem yty}
Let $\mu\notin \frac\nu2+\Z_{\geq 0}$.
Let $(t^0; z; \mu)$ be a solution of the Bethe ansatz equations \eqref{tr.BAE} assigned to
the triple $(z; \mu; V[\nu])$ in Section \ref{sec BAE}.  Let 
\bean
\label{roty}
\tilde y(x) \,=\, \prod_{i=1}^{M-m}(x-\tilde t^{\,0}_i)
\eean
be the polynomials assigned to $(t^0; z; \mu)$ in Theorem \ref{quasipthm}. If
 $\tilde y(x)$ has distinct roots and $\tilde y(z_s)\ne 0$ for $s=1,\dots,n$,
 then
$(\tilde t^{\,0}_1,\dots,\tilde t^{\,0}_{M-m}; z; -\mu)$ is a solution of the Bethe ansatz equations 
\eqref{tr.BAE} assigned to the triple $(z; -\mu;  V[-\nu])$.





\end{lem}


\begin{proof}
Equation \eqref{Wr.eqn22} can be rewritten as 
\bean
\label{Wr2}
\Wr_x (x^{-\mu+\nu/2}y(x), \tilde y(x)) \,=\,\const\, x^{-\mu+\nu/2-1} \prod_{s=1}^n (x-z_s)^{m_s}.
\eean
Now the lemma follows from the equalities $-\nu = -M+2m=M-2(M-m)$
and Lemma \ref{lem WBA}.
\end{proof}


\begin{thm}
[{\cite[Theorem 5.7]{MV2}}]
\label{thm WB}

 Under assumptions of Lemma \ref{lem yty} consider 
the Bethe vectors  $\om(t^0\,,z, \mu) \in V[\nu]$ and
 $\om(\tilde t^{\,0}\!,z,-\mu) \in V[\nu]$.  Then
 \bean
 \label{BVA}
\mc A\Big(\mu +\frac {\nu}2-1\Big)
\om(t^0\,,z,\mu) =\, \const\,
\om(\tilde t^{\,0}\!,z,-\mu),
\eean
where $\const$ is a nonzero constant.


\end{thm}

\begin{cor}
\label{cor =eig}
 Under assumptions of Lemma \ref{lem yty}, for $s=1,\dots,n$,
 the eigenvalue of $\K_s(z,\mu)$ on $\om(t^0\!,z,\mu)$ equals 
 the eigenvalue of $\K_s(z,-\mu)$ on $\om(\tilde t^{\,0}\!,z,\mu)$.
\end{cor}

\begin{proof}
The corollary follows from Lemma \ref{lem AK} and Theorem \ref{thm isom mu}.
\end{proof}



%\begin{thm}
%[{\cite[Section 5.4]{MV2}}]\label{thm ba}


%Let $\{z_1,\dots,z_n\}\subset \C^\times$ be distinct numbers. Let $V= \ox_{s=1}^n V_{m_s}$ be as in \eqref{ten V} and $V[\nu]$ a nonzero weight subspace. Then there exists a Zarisky open subspace $U\subset \C$, such that for any $\mu\in U$ the following statements hold:
%\begin{enumerate}

%\item[(i)] 
%The  Bethe vectors $\om(t^0\!,z,\mu)$ assigned to solutions 
%of the Bethe ansatz equations \eqref{tr.BAE}
 %form a basis
%of $V[\nu]$.  


%\item[(ii)] 

%Let $(t^0; z; \mu)$ be any of these solutions. Let 
%$\tilde y(x)  = \prod_{i=1}^{M-m}(x-\tilde t^{\,0}_i)$
%be the polynomial assigned to $(t^0; z; \mu)$ in Theorem \ref{quasipthm}. 
%Then  $\tilde y(x)$ has distinct roots and $\tilde y(z_s)\ne 0$ for $s=1,\dots,n$;
%hence $(\tilde t^{\,0}_1, \dots, \tilde t^{\,0}_{M-m}; z; -\mu)$ is a solution of the Bethe ansatz equations 
%\eqref{tr.BAE} assigned to the triple $(z; -\mu; V[-\nu])$.

%\item[(iii)] 
%Consider the Bethe vectors $\om(\tilde t^{\,0}\!,z,-\mu)$, assigned to solutions 
% $(\tilde t^{\,0}_1, \dots, \tilde t^{\,0}_{M-m}; z;-\mu)$
%constructed in part (ii). Then these vectors  form a basis of $V[-\nu]$.

%\end{enumerate}
%\end{thm}





\section{Conjugates of $\D$ and  $\mc E_{\,t^0; z; \mu}$}
\label{sec 8}



\subsection{Conjugate of $\D$}


Recall the universal differential operator $\D=\der_u^2+D_2(x)$  introduced in \eqref{Dcdet}, where
the coefficient $D_2(x)$ is determined by formula \eqref{DD2}. 
We introduce the operator 
\bean
\label{Dconj}
\D^{c} = \frac{1}{(2\pi \sqrt{-1}\,x)^2}\,\prod_{s=1}^n (x-z_s)^{m_s/2} \cdot \D \cdot \prod_{s=1}^n (x-z_s)^{-m_s/2},
\eean
where the superscript $^c$ stays for the word ``conjugated''.

\begin{thm}
\label{Dconjthm}
We have 
\bean
\label{Dconjform}
\phantom{aaaaaa}
\D^{c} 
&=&
 \der_x^2 + \Big[ \frac{1}{x} - \sum_{s=1}^n \frac{m_s}{x- z_s} \Big] \der_x     
-  \frac{1}{x} \sum_{s=1}^n \frac{m_s/2}{x- z_s} + \sum_{s=1}^n \frac{m_s(m_s+2)/4}{(x- z_s)^2}
\\ 
 &+&
    \sum_{s \ne p} \frac{m_s m_p/4}{(x- z_s)(x- z_p)}  
-\frac{\mu^2+\mu (e_{11} - e_{22}) -  e_{11} e_{22}}{4x^2}    
 \notag
 \\  
&-&
\frac{1}{x^2} \,\sum_{s=1}^n \left[  z_s \frac{m_s(m_s+2)/4 + 
\K_s(z,\mu)}{x- z_s} +  z_s^2 \frac{m_s(m_s+2)/4}{(x- z_s)^2} 
\right].
 \nonumber
\eean

\end{thm}



\begin{proof} 

 Recall that $x = e^{-2 \pi \sqrt{-1} u}$, 
 $\partial_u = -2 \pi \sqrt{-1} \,x \partial_x, \; \partial_u^2 = (2\pi \sqrt{-1})^2 ( x \partial_x + x^2 \partial_x^2 )$. 
 Denote \\
 $f = \prod_{s=1}^n (x-z_s)^{-m_s/2}$.  We have
$f' =   -\sum_{s=1}^n \frac{m_s/2}{x-z_s}  \,f\,$,
\bea
f'' =  \Big( \sum_{s=1}^n \frac{m_s^2/4}{(x-z_s)^2} +
 \sum_{s \ne p} \frac{m_s m_p / 4}{(x-z_s)(x-z_p)} + \sum_{s=1}^n \frac{m_s/2}{(x-z_s)^2} \Big) f,
\eea
where $' = \der/\der x$. Therefore,
\bea
\D^{c}
& =& \frac{1}{x^2} f^{-1} \Big[x^2 \der_x^2 + x\der_x + \frac{1}{(2\pi \sqrt{-1})^2} D_2(x) \Big] f 
\\
&=&
 \frac{1}{x^2} f^{-1} \Big[ x^2 \big( f \der_x^2 + 2 f' \der_x + f'' \big) 
 + x \big(f \der_x + f' \big) + \frac{1}{(2\pi\sqrt{-1} )^2} D_2(x) f \Big]
\\
&=&
 \der_x^2 + \Big[ 2 f^{-1} f' + \frac{1}{x} \Big] \der_x + \Big[ f^{-1} f'' + \frac{1}{x} f^{-1} f' + 
 \frac{1}{x^2} \frac{1}{(2\pi \sqrt{-1})^2} D_2(x) \Big],
\eea
which gives the right-hand side of formula \eqref{Dconjform}.
\end{proof}

\subsection{Conjugate of $\mc E_{\,t^0;z;\mu}$\,} Similarly to the conjugation of $\D$ we conjugate 
$\mc E_{\,t^0\!,z,\mu}$ and consider the differential operator
\bean
\label{conj Dt}
\mc E_{\,t^0; z; \mu}^c = \frac{1}{(2\pi \sqrt{-1}\,x)^2}\,\prod_{s=1}^n (x-z_s)^{m_s/2} \cdot 
\mc E_{\,t^0; z; \mu} \cdot \prod_{s=1}^n (x-z_s)^{-m_s/2}.
\eean


\begin{lem}
\label{kerDconj}

The kernel of $\mc E_{\,t^0; z; \mu}^c$ is spanned by quasi-polynomials 
\bean
\label{qp PQ}
 x^{\frac{\nu/2-\mu}{2}} y(x),
 \qquad
  x^{\frac{\mu-\nu/2}{2} }\tilde y(x),
  \eean
where $y(x)$ is the monic polynomial of degree $m$, defined in \eqref{Y&u}, and $\tilde y(x)$ is the monic polynomial of degree
$M-m$, defined in Theorem \ref{quasipthm}.
\qed  
\end{lem} 

\begin{lem}
\label{lem e=e}
Under assumptions of Lemma \ref{lem yty}, let
$(t^0; z; \mu)$ be a solution of the Bethe ansatz equations \eqref{tr.BAE} assigned to
the triple $(z; \mu; V[\nu])$. Assume that 
the numbers $\tilde t^{\,0}_1,\dots,\tilde t^{\,0}_{M-m}$ defined in Lemma \ref{lem yty}
are such that 
$(\tilde t^{\,0}_1,\dots,\tilde t^{\,0}_{M-m}; z; -\mu)$ is a solution of the Bethe ansatz equations 
\eqref{tr.BAE} assigned to the triple $(z; -\mu;  V[-\nu])$. Then
\bean
\label{e=e}
\mc E_{\,\tilde t^{\,0}; z; -\mu}^c
=
\mc E_{\,t^0; z; \mu}^c\,.
\eean
\qed
\end{lem}









\section{Space of $V$-valued functions of $z_1,\dots,z_n$}
\label{sec 9}

\subsection{Space $V_1^{\ox n}[\nu]$}
\label{sec V1}

Recall the two-dimensional irreducible $\slt$-module $V_1$ 
with basis 
$v^1_0\,,\,v^1_1$\,,  see \eqref{V_m}.
In the rest of the paper we assume that $V$ is the tensor power of $V_1$, 
\bean
\label{V1n}
V\,=\, V_1^{\ox n}\,, \qquad\on{where} \ \ n>1.
\eean
The space $V$ has a basis of vectors
\bea
v_I=v_{i_1}^1\ox\dots\ox v_{i_n}^1\,,
\eea
labeled by partitions $I=(I_1,I_2)$ of $\{1,\dots, n\}$, where
 $\,i_j =0$ if $j\in I_1$, and $\,i_j =1$ if $j\in I_2$.
 We have the weight decomposition $V=\oplus_{m=0}^n V[n-2m]$, where
$V[n-2m]$
is of dimension $\binom{n}{m}$ and
has the basis  $\{v_I\ |\ I=(I_1,I_2), \ |I_1|=m, \ |I_2|=n-m\}$.

\smallskip
We use notations \
$\nu=n-2m$, \ $\ell = n-m$,\ and hence\ $ m+\ell=n$.


\subsection{Space $\V^{S}$} 
Let $z=(z_1,\dots,z_n)$ be variables. The symmetric group 
$S_n$  acts on the algebra $\C[z_1,\dots,z_n]$ by permuting the variables. Let $\si_s(z)$,
$s=1,\dots,n$, be the $s$-th elementary symmetric polynomial in $z_1,\dots, z_n$.
 The algebra of
symmetric polynomials $\C[z_1,\dots, z_n]^S$ is a free polynomial algebra with generators 
$\si_1(z),\dots,\si_n(z)$.

\smallskip


Let $\V$ be the space of polynomials in $z_1,\dots,z_n$ with coefficients
in $V_1^{\ox n}$,
\bea
\V = V_1^{\ox n}\ox \C[z_1,\dots,z_n].
\eea
The symmetric group $S_n$ acts on $\V$ by permuting the factors of $V_1^{\ox n}$
and the variables $z_1,\dots,z_n$ simultaneously,
\bea
\rho(v_1\ox\dots\ox v_n\ox p(z_1,\dots,z_n))=
v_{(\rho^{-1})(1)}\ox\dots\ox v_{(\rho^{-1})(n)}\ox p(z_{\rho(1)}, \dots, z_{\rho(n)}),\quad \rho\in S_n.
\eea
We denote by $\V^S$ the subspace of $S_n$-invariants in $\V$.

\begin{lem}
[\cite{MTV3}]
 The space $\V^S$ is a free $\C[z_1,\dots, z_n]^S$-module of rank $2^n$.
\end{lem}


Consider the grading on $\C[z_1,\dots,z_n]$
 such that $\deg z_s = 1$ for all $s = 1,\dots, n$. 
 We define a
grading on $\V$ by setting $\deg(v \ox p) = \deg p$ for any $v \in V_1^{\ox n}$ and 
$p\in\C[z_1,\dots,z_n]$. The
grading on $\V$ induces a  grading on $\End(\V)$.

\smallskip
The Lie algebras $\slt\subset\glt$ naturally act on $\V^S$. We have the weight decomposition
\bea
\V^S=\oplus_{m=0}^n \V^S[n-2m],
\qquad
\V^S[n-2m] = (V[n-2m]\ox\C[z_1,\dots,z_n])^{S}\,.
\eea


\smallskip
Let $M$ be a $\Z_{>0}$-graded space with finite-dimensional homogeneous components. Let
$M_j\subset M$ be the homogeneous component of degree $j$. The formal power series in a
variable $\al$,
$\ch_M(\al) =\sum_{j=0}^\infty (\dim M_j)\, \al^j,$\,
is called the graded character of $M$.

\begin{lem}
[\cite{MTV2}]
\label{lem frV}
The space $\V^S[n-2m]$  is a free $\C[z_1,\dots, z_n]^S$-module of rank $\binom{n}{m}$
and
\bean
\label{ch V}
\ch_{\V^S[n-2m]}(\al) \,=\, \prod_{i=1}^m \frac 1{1-\al^i} \cdot \prod_{i=1}^{n-m} \frac 1{1-\al^i}\,.
\eean
\end{lem}


\subsection{Bethe algebra of $\V^S[\nu]$}

Recall the differential operator $\D^c$ introduced in \eqref{Dconjform}
for $V=\oplus_{s=1}^nV_{m_s}$ and depending on  parameter $\mu\in \C$.
 For $V=V_1^{\ox n}$ the operator $\D^c$ takes the form
\bean
\label{Dz}
\mc F= \der_x^2 + F_1(x)\der_x + F_2(x),
\eean
where
\bean
\label{B}
\phantom{aaa}
F_1(x) 
&=&
  \frac{1}{x} - \sum_{s=1}^n \frac{1}{x- z_s}\, ,
\\
\notag
F_2(x) 
&=&
-  \frac{1}{x} \, \sum_{s=1}^n \frac{1/2}{x- z_s} + \sum_{s=1}^n \frac{3/4}{(x- z_s)^2}
+ \sum_{s \ne p} \frac{1/4}{(x- z_s)(x- z_p)}
\\ 
 &-&
    \frac{\mu^2+\mu (e_{11} - e_{22}) -  e_{11} e_{22}}{4x^2}    
-
\frac{1}{x^2} \,\sum_{s=1}^n \left[  z_s \frac{3/4 + 
\K_s(z,\mu)}{x- z_s} +  z_s^2 \frac{3/4}{(x- z_s)^2} 
\right].
\notag
\eean
In formula \eqref{Dconjform} we had $\{z_1,\dots,z_n\}$ being
 a subset of $ \C^\times$. From now on we consider
$z_1,\dots,z_n$ as independent variables.

\smallskip
The operator $\mc F$ in formula \eqref{Dz} with variables $z_1,\dots,z_n$ 
is called the {\it universal differential operator} for $\V^S$ with parameter $\mu\in\C$.



\begin{lem}
[{cf. {\cite[Section 2.7]{MTV3}}}]
\label{lem LG}
The Laurent expansions of $F_1(x)$ and $F_2(x)$ at infinity have the form
\bean
\label{Fij}
F_1(x) = \sum_{j=1}^\infty F_{1j} x^{-j}\,, 
\qquad  
F_2 (x) = \sum_{j=2}^\infty F_{2j} x^{-j}\,,
\eean
where 
\bea
F_{11} 
&=&
1 - n, 
\qquad 
F_{1j} = - \sum_{s=1}^n z_s^{j-1} \quad \on{for} \; j\geq 2 .  
%\\
%F_{22}
% &=&
% \frac14\,\Big[n^2-\mu^2-\mu (e_{11} - e_{22}) + e_{11} e_{22}\Big]\,.
\eea
For any $j\geq 2$, the element $F_{2j}$ is a homogeneous polynomial
in $z_1,\dots,z_n$ of degree $j-2$ with coefficients in
$\End(V)$. The element $F_{2j}$ preserves the weight decomposition of
$\V$. Each of the elements  $F_{1j}$, $j\geq 1$, $F_{2j}$, $j\geq 2$, defines an endomorphism 
of the $\C[z_1,\dots,z_n]^{S}$-module $\V^S$.
\end{lem}

\begin{proof}
The proof follows from  straightforward calculations.
\end{proof}


\begin{lem}
The elements $F_{1j}$, $j\geq 1$, $F_{2j}$, $j\geq 2$, considered as endomorphisms of
 the $\C[z_1,\dots,z_n]^{S}$-module  $\V^S$, commute.
\end{lem}

\begin{proof}
The commutativity follows from the commutativity of the trigonometric Gaudin operators
in formula \eqref{Dconjform}.
\end{proof}

For a  weight subspace $\V^S[\nu]$, $\nu = n-2m$, $\ell = n-m$,
consider the commutative subalgebra
$\B(\mu; m;\ell)$ 
of the algebra of endomorphisms of the  $\C[z_1,\dots,z_n]^{S}$-module  $\V^S[\nu]$,
generated by the elements $F_{1j}$, $j\geq 1$, $F_{2j}$, $j\geq 2$.
The subalgebra $\B(\mu; m;\ell)$ 
 is called the {\it Bethe algebra} of $\V^S[\nu]$ with parameter $\mu\in \C$.

\begin{lem}
\label{lem z in B}
The Bethe algebra $\B(\mu; m;\ell)$ contains the subalgebra of
operators of  multiplication by
elements of $\C[z_1,\dots,z_n]^S$.
\end{lem}

\begin{proof}
The subalgebra of  operators of multiplication by elements of 
$\C[z_1,\dots,z_n]^S$ is generated by
the elements $F_{1j}$, $j\geq 1$, see Lemma \ref{lem LG}.
\end{proof}

Lemma \ref{lem z in B} makes the Bethe algebra $\B(\mu; m;\ell)$
a $\C[z_1,\dots,z_n]^S$-module.



\subsection{Weyl group invariance}

For a weight subspace $V[\nu]=V_1^{\ox n}[\nu]$ recall   the operator 
$\mc A\big(\mu+ \nu/2-1\big) : V[\nu] \to V[-\nu]$,  defined in \eqref{mc A}.
It is an isomorphism of vector spaces,  if $\mu \notin \frac n2 + \Z$.
That operator induces an isomorphism of  $\C[z_1,\dots,z_n]^S$-modules,

\bean
\label{mc Amu}
\mc A\big(\mu+ \nu/2-1\big) : \V^S[\nu] \to \V^S[-\nu].
\eean
\vsk.2>


\begin{lem}
\label{lem BB iso}
Let $\mu \notin \frac n2 + \Z$. Let $F_{ij}(\mu, m,\ell)$ be the generators
of $\B(\mu;m;\ell)$, defined in \eqref{Fij},  and
$F_{ij}(-\mu,\ell, m)$  the generators
of $\B(-\mu;\ell;m)$.  Then
\bean
\label{Fij iso}
F_{ij}(-\mu,\ell, m) \,=\,
\mc A\Big(\mu +\frac {\nu}2-1\Big) F_{ij}(\mu, m,\ell)\,
\mc A\Big(\mu +\frac {\nu}2-1\Big)^{-1}
\eean
for all $i,j$. 
The map
 
\bean
\label{mu-mu}
\B(\mu;m;\ell) \to \B(\mu;m;\ell), \quad F_{ij}(\mu; m; \ell)
\mapsto
F_{ij}(-\mu,\ell, m),
\eean

\smallskip
\noindent
is an isomorphism of algebras and of $\C[z_1,\dots,z_n]^S$-modules.
The maps  in \eqref{mc Amu} and \eqref{mu-mu} define an isomorphism
between the $\B(\mu;m;\ell)$-module $\V^S[\nu]$
and the $\B(-\mu;\ell;m)$-module $\V^S[-\nu]$.


\end{lem}

\begin{proof}
The lemma follows from Lemma \ref{lem AK}.
\end{proof}

\subsection{Generic fibers of $\V^S[\nu]$}

Given $a=(a_1,\dots,a_n)\in \C^n$, denote by $I_a\subset \C[z_1,\dots, z_n]$ the ideal
generated by the polynomials  $\si_s(z)-a_s$, $s=1,\dots,n$. Define
\bean
\label{ide}
 I_a\V^S[\nu] \,:=\, \V^S \cap (V[\nu]\ox I_a).
\eean
Assume that $a$ is such that the polynomial $x^n + \sum_{s=1}^n(-1)^s a_sx^{n-s}$ has distinct
nonzero roots  $b_1,\dots,b_n$.

\begin{lem}
[{\cite[Lemma 2.13]{MTV3}}]
\label{lem fib}

The quotient $\V^S[\nu]/ I_a\V^S[\nu]$ is a finite-dimensional complex vector space canonically isomorphic
to $V[\nu]$. Under this isomorphism
the  Bethe algebra $\B(\mu; m;\ell)$ induces a commutative algebra of  operators on 
$V[\nu]$. That commutative algebra of operators is canonically isomorphic to the Bethe algebra
$\B(b_1,\dots,b_n; \mu; V[\nu])$ introduced in Section \ref{sec ctG}.

\end{lem}





\section{Functions on pairs of quasi-polynomials}
\label{sec 10}


\subsection{Space of pairs of quasi-polynomials}

Let $m,\ell,n$ be positive integers, $m+\ell=n$. Denote
$\nu=n-2m$, cf. Section \ref{sec V1}.
Let 
\bea
\zeta\,\in \,\C - \frac 12\,\Z\,.
\eea
 Let $\Om(\zeta,m,\ell)$ be the affine $n$-dimensional space
with coordinates $p_i$, $i=1,\dots,m$,  $q_j$, $j=1,\dots,\ell$.
Introduce the generating functions
\bean
\label{pq}
p(x) &=& x^{-\zeta}\,(x^m + p_1x^{m-1} + \dots + p_m),
\\
\notag
q(x) &=& x^{\zeta}\,(x^\ell + q_1x^{\ell-1} + \dots + q_\ell).
\eean
We identify points $U$ of $ \Om(\zeta, m,\ell)$ with  two-dimensional complex vector spaces 
generated by quasi-polynomials
\bean
\label{pqU}
p(x,U) &=& x^{-\zeta}\,(x^m + p_1(U)x^{m-1} + \dots + p_m(U)),
\\
\notag
q(x,U) &=& x^{\zeta}\,(x^\ell + q_1(U)x^{\ell-1} + \dots + q_\ell(U)).
\eean
Denote by $\O(\zeta,m,\ell)$
the algebra of regular functions on $\Om(\zeta,m,\ell)$,
\bea
\O(\zeta,m,\ell) = \C[p_1,\dots,p_m, q_1,\dots,q_\ell].
\eea
 Define the  grading on 
$\O(\zeta,m,\ell)$ by $\deg p_i=\deg q_i=i$ for
all $i$.



\begin{lem}
\label{lem grO}
The graded character of the algebra $\O(\zeta, m,\ell)$ equals
\bean
\ch_{\O(\zeta, m,\ell)} (\al) = \prod_{i=1}^m \frac{1}{1-\al^i} \cdot \prod_{j=1}^\ell \frac{1}{1-\al^j}.
\eean
\qed
\end{lem}


\subsection{Wronski map}

Let $p(x), q(x)$  be the generating functions in \eqref{pq}.  We have
\bean
\label{Wpq}
\Wr_x(p,q) = \frac {2\zeta + \ell-m}x\,
\Big(x^n + \sum_{s=1}^n\,(-1)^s\,\Si_s \,x^{n-s}\Big),
\eean
where $\Si_1,\dots,\Si_n$ are elements of $\O(\zeta, m,\ell)$. 
 Notice that 
$2\zeta + \ell-m = 2\zeta +\nu\,\notin\Z$ according to our assumptions. 
The elements  $\Si_1,\dots,\Si_n$ are homogeneous with
$\deg \Si_s = s$.


\smallskip
Define the {\it Wronski map}
\bea
\Wr\, :\, \Om(\zeta, m,\ell) \to \C^n, \quad
U \mapsto 
(\Si_1(U), \dots, \Si_n(U)).
\eea

\begin{lem}
\label{lem pdeg}
For  $\zeta\in \C-\frac 12\Z$, \,the Wronski map is a map of positive degree. 

\end{lem}

\begin{proof} 
The proof is a slight modification of the proof of \cite[Proposition 3.1]{MTV4}. 
\end{proof}

Let $\O^S \subset \Oz$ be the subalgebra generated by $\Si_1,\dots,\Si_n$.
Let $\si_1,\dots,\si_n$ be coordinates on  $\C^n$, which is the image of the Wronski
map. Introduce the grading on $\C[\si_1,\dots,\si_n]$ by $\deg \si_s=s$ for all $s$.
The Wronski map induces the isomorphism
$\C[\si_1,\dots,\si_n]\to \O^S$, $\si_s \mapsto \Si_s$, of graded algebras, see Lemma
\ref{lem pdeg}.
This isomorphism makes $\Oz$ a $\C[\si_1,\dots,\si_n]$-module.


\subsection{Another realization of $\O(\zeta, m,\ell)$}


Define the differential operator $\mc G$ by
\bean
\label{DO1}
\mc G = \frac{1}{\Wr_x(p,q)} \, 
\on{rdet}\begin{bmatrix} p & p' & p'' 
\\
 q & q' & q'' 
 \\ 
 1 & \der_x & \der^2_x \end{bmatrix},
\eean
where $\on{rdet}$ is the row determinant.
 We have
\bean
\label{DO2}
\mc G = \der_x^2 + G_1(x) \der_x + G_2(x),
\eean

\medskip
\noindent
cf. \cite{MTV3}.
It is a differential operator in variable 
$x$ and \  $ G_1(x)$, $G_2(x)$  are rational functions in $x$  with coefficients in $\O(\zeta, m,\ell)$.

\smallskip
Notice  that
\bean
\label{G1}
G_1 \,=\,-\,\frac{(\Wr_x(p,q))'}{\Wr_x(p,q)}\,.
\eean 
\vsk.2>

\begin{lem}
[{cf. {\cite[Section 2.7]{MTV3}}}]
\label{lem LGG}
The Laurent expansions of $G_1(x)$ and $G_2(x)$ at infinity have the form
\bean
\label{LaG} 
G_i (x) = \sum_{j=i}^\infty G_{ij} x^{-j}, \qquad  i=1,2, 
\eean
where for
any $i,j$, the element $G_{ij}$ is a homogeneous element of $\O(\zeta, m,\ell)$
 of degree $j-i$.
 % and 
% \bea
%G_{11} 
%= 1 - n, 
%\qquad 
%G_{22} = (m-\zeta)(\ell+\zeta).
%\eea
\end{lem}

\begin{proof}
The proof is by  straightforward calculation.
\end{proof}


%Define the {\it indicial polynomial} of the differential operator $\mc G$
%at infinity  by 
%\\
%$\chi_{\mc G}(\al) = \al(\al-1) + G_{11}\al+ G_{22}$\,.


%\begin{lem}

%We have
%\bean
%\label{char mc G}
%\chi_{\mc G}(\al) = \big(\al - (m-\zeta)\big)
%\big(\al-(\ell +\zeta)\big).
%\eean
%\qed
%\end{lem}




\begin{lem} 
[{\cite[Lemma 3.4]{MTV3}}, {\cite[Lemma 4.3]{MTV2}}]

\label{O_thm}
Let  $\zeta\in \C-\frac 12\Z$.
Then  the elements $G_{ij}$,   $i=1,2$,  $j\geq i$,
 generate the algebra $\O(\zeta, m,\ell)$. 
 \qed
 \end{lem}
 
 \subsection{Fibers of  Wronski map}
 
 

Given $a=(a_1,\dots,a_n)\in \C^n$, denote by $J_a\subset \O(\zeta,m,\ell)$ the ideal
generated by the elements  $\Si_s-a_s$, $s=1,\dots,n$. Define
\bean
\label{ide}
 \O_a(\zeta,m,\ell) \,:=\, \O(\zeta,m,\ell)\big/ J_a .
 \eean

\noindent
 The algebra $\O_a(\zeta,m,\ell)$ is the algebra of functions on the fiber $\Wr^{-1}(a)$ of the Wronski map.

\smallskip
Let 
\bean
\label{ab}
x^n+\sum_{s=1}^n\,(-1)^{n-s}\,a_s\,x^{n-s} =   \prod_{s=1}^n(x-b_s)
\eean
 for some $b_s\in\C$.  Let  $U=\langle p(x,U), q(x,U)\rangle$ be a point of
$\Om(\zeta,m,\ell)$ and
\bea
p(x,U)=x^{-\zeta}\prod_{i=1}^m(x-t^0_i),
\qquad
q(x,U)=x^{\zeta}\prod_{i=1}^\ell(x-\tilde t^{\,0}_i),
\eea
for some $t_i^0, \tilde t^{\,0}_i\in\C$.

\begin{lem}
\label{lem gen f}
Let $\zeta\in \C-\frac 12\Z$. Then there exists
 a Zariski open subset $X\subset \C^n$ such that for any  $a\in X$ 
 all the numbers $b_1,\dots,b_n$ are nonzero and distinct.
Moreover, for any point $U\in\Wr^{-1}(a)$ all the numbers
$b_1,\dots,b_n$, $t^0_1,\dots,t^0_m$, $\tilde t^{\,0}_1,\dots,\tilde t^{\,0}_\ell$ are distinct.
\qed


\end{lem}

\begin{lem}
\label{cor gen f}
If $a\in X$ and $U\in \Wr^{-1}(a)$, then
$(t^0_1,\dots,t^0_m; b_1,\dots,b_n; 2\zeta + \nu/2)$ is a solution
of the Bethe ansatz equations \eqref{tr.BAE} assigned to the triple
$(b_1,\dots,b_n; 2\zeta + \nu/2; V[\nu])$, and
$(\tilde t^{\,0}_1,\dots,\tilde t^{\,0}_\ell; b_1,\dots,b_n; - 2\zeta - \nu/2)$ is a solution
of the Bethe ansatz equations \eqref{tr.BAE} assigned to the triple
$(b_1,\dots,b_n; -2\zeta - \nu/2; V[-\nu])$.


 

\end{lem}

\begin{proof}
We have 
\bea
\Wr_x(p(x,U),q(x,U))= \frac {2\zeta + \ell-m}x\,
\Big(x^n + \sum_{s=1}^n\,(-1)^s\,a_s \,x^{n-s}\Big).
\eea
Now the lemma follows from Lemmas \ref{lem gen f}, \ref{lem WBA}.
\end{proof}


For $U\in \Om(\zeta, m,\ell)$ denote by $\mc G_U$ the monic differential operator
with kernel $U$, 
\bean
\label{GU}
\mc G_U = \der_x^2 + F_{1;U}(x)\der_x + F_{2,U}(x).
\eean
The operator $\mc G_U$ is 
obtained from the operator $\mc G$ by evaluating the generating functions $p,q$ 
at the point $U$.

\begin{lem}
\label{lem e=g}

Let $a\in X$ and $U\in \Wr^{-1}(a)$. Let $(t^0; b; 2\zeta + \nu/2)$ be the solution of the Bethe ansatz equations
described in Lemma \ref{cor gen f}. Let $\mc E_{t^0;\, b;\, 2\zeta + \nu/2}^c$ 
be the differential operator defined in \eqref{conj Dt}.
Then
\bea
\mc E_{t^0;\, b; \,2\zeta + \nu/2}^c = \mc G_U\,.
\eea
 \end{lem}
 
 \begin{proof}
 The lemma follows from Lemma \ref{kerDconj}.
  \end{proof}
 
 
 

\section{Isomorphisms}
\label{sec 11}

In Section \ref{sec 9} we introduced the $\B(\mu, m,\ell)$-module $\V^S[\nu]$, where
 $\mu\in \C$, \ $\nu = n-2m$, $m+\ell=n$.
In Section \ref{sec 10} we discussed the properties of
the algebra $\Oz$ under the assumption that 
$\zeta\in \C-\frac 12\Z$. 

\smallskip
 We  consider $\Oz$ as the $\Oz$-module with action defined by
 multiplication. 

\smallskip
In this section we construct an isomorphism between the $\B(\mu, m,\ell)$-module $\V^S[\nu]$
and the $\Oz$-module $\Oz$ under the assumption that
\bean
\label{ass}
\zeta = \frac\mu 2 - \frac \nu 4\quad \on{and} \quad \zeta\in \C-\frac 12\Z\,,
\eean
where the last inclusion can be reformulated as 
\bean
\label{assm}
\mu\notin\ \frac n2 + \Z\,,
\eean
cf. the
assumptions on $\mu$ and $\zeta$ in Theorems \ref{thm isom mu}, \ref{quasipthm}, 
 Lemmas \ref{lem WBA}, \ref{lem yty} and Section \ref{sec 10}.

\smallskip
The construction of the isomorphism is similar to the constructions in \cite{MTV3, MTV2}.


\subsection{Isomorphism of algebras}

Consider the map

\bea
\tau  :  \Oz \to \B(\mu, m,\ell), \quad G_{ij}\mapsto F_{ij}.
\eea
${}$

\begin{thm}
[cf. {\cite[Theorem 5.3]{MTV3}}, {\cite[Theorem 6.3]{MTV2}}]
\label{thm isoa}

Under the assumptions \eqref{ass}
the map $\tau$ is a well-defined 
isomorphism of graded algebras.

\end{thm}

\begin{proof} 
Let a polynomial $R(G_{ij})$ in generators $G_{ij}$  be equal to zero in
$\Oz$. Let us prove that the
corresponding polynomial $R(F_{ij})$ is equal to zero in $\Bm$.
Indeed, $R(F_{ij})$ is a polynomial
in $z_1,\dots,z_n$ with values in $\End(V[\nu])$. By Lemmas \ref{lem gen f} - \ref{lem e=g},
 \ref{basis}, for generic $b_1,\dots,b_n$ the
 value of the polynomial $R(F_{ij})$ at
$z_1 = b_1,\dots, z_n=b_n$ equals zero. Hence, the polynomial $R(F_{ij})$ equals zero identically and
the map $\tau$ is a well-defined defined homomorphism of algebras.

The elements $G_{ij}$, $F_{ij}$ are of the same degree. Hence  $\tau$ is a graded homomorphism.

Let a polynomial $R(G_{ij})$ in generators  $G_{ij}$ be a nonzero element of
$\Oz$. Then the value of
$R(G_{ij})$ at a generic point $U \in \Om(\zeta, m,\ell)$ is not equal to zero by Lemma \ref{lem e=g}.
Then the polynomial $R(F_{ij})$ is not identically equal to zero. Therefore, the map $\tau$ is injective.
Since the elements $F_{ij}$ generate the algebra $\Bm$, the map $\tau$ is surjective.
\end{proof}

The algebra $\C[z_1,\dots,z_n]^S$
 is embedded into the algebra $\Bm$ as the subalgebra of operators
of multiplication by symmetric polynomials.
 The
algebra $\C[z_1,\dots,z_n]^S$ is embedded into the algebra $\Oz$, the elementary symmetric polynomials
$\si_1(z),\dots,\si_n(z)$ being mapped to the elements
$\Si_1,\dots,\Si_n$. These
embeddings give the algebras $\Bm$ and $\Oz$ the structure of $\C[z_1,\dots,z_n]^S$-modules.

\begin{lem}
[{\cite[Lemma 6.4]{MTV3}}]
\label{lem iso m}
Under assumptions \eqref{ass} the map $\tau$ is an isomorphism of $\C[z_1,\dots,z_n]^S$-modules.

\end{lem}

\begin{proof}
The lemma follows from formulas \eqref{Wr.eqn2}, \eqref{G1}.
\end{proof}

\subsection{Isomorphism of modules}
\label{sec imo}

The subspace of $\V^S[\nu]$ of all elements of degree $0$ is of dimension one and is generated by the vector
\bea
v_+ = \sum_{I=(I_1,I_2),\, |I_1|=m, |I_2|=\ell} v_I\,.
\eea
The subspace of $\Oz$ of all elements of degree $0$ is of dimension one and is generated by the element $1$.
Define the $\C[z_1,\dots,z_n]^S$-linear map

\bean
\label{ups}
\phi : \Oz \to \V^S[\nu], \quad G\mapsto \tau(G)\,v_+\,.
\eean
\medskip




\begin{thm}
[{\cite[Theorem 6.7]{MTV3}}]

\label{thm ups}

Under assumptions \eqref{ass}, the map $\phi$ is a graded isomorphism of graded $\C[z_1,\dots,z_n]^S$-modules. 
The maps $\tau$ and $\phi$ intertwine the action of multiplication
operators on $\Oz$  and the action of the Bethe algebra $\Bm$  on $\V^S[\nu]$, that is, for any 
$f,g \in\Oz$, we have

\bean
\label{inter}
\phi(fg) = \tau(f)\,\phi(g).
\eean
${}$

\noindent
In other words, the maps $\tau$ and $\phi$ define an isomorphism between the $\Oz$-module $\Oz$ and the
$\Bm$-module $\V^S[\nu]$.


\end{thm}

\begin{proof}
First we show that the map $\phi$ is injective. Indeed,
the algebra $\Oz$ is a free polynomial algebra containing the subalgebra $\C[z_1,\dots,z_n]^S$.
The quotient algebra $\Oz/\C[z_1,\dots,z_n]^S$ is finite-dimensional by Lemma \ref{lem pdeg}.
The  kernel of $\phi$ is a proper ideal $\mc I$ in $\Oz$. Then $\tau(\mc I)$ is an ideal in $\Bm$.
Any proper ideal in $\Bm$ has zero intersection with $\C[z_1,\dots,z_n]^S$. Hence
$\mc I$ has zero intersection with $\C[z_1,\dots,z_n]^S$ and therefore is 
the zero ideal. The injectivity is proved.

The map $\phi$ is graded. 
The graded characters of $\V^S[\nu]$ and $\Oz$ are equal by Lemmas \ref{lem frV}    and  \ref{lem grO}.
Hence $\phi$ is an isomorphism.
\end{proof}



\begin{cor}
\label{lem isofi}
Assume that $a=(a_1,\dots,a_n)\in \C^n$ is such that the polynomial 
$x^n + \sum_{s=1}^n(-1)^s a_sx^{n-s}$ has distinct roots
$b_1,\dots,b_n$. Then under  assumptions \eqref{ass}, the isomorphisms 
$\tau$, $\phi$ induce the isomorphism of  the $\B(b_1,\dots,b_n;\mu; V[\nu])$-module
$V[\nu]$ and the $\O_a(\zeta;m,\ell)$-module $\O_a(\zeta;m,\ell)$, where
$\O_a(\zeta;m,\ell)$ is the algebra of functions on the fiber $\Wr^{-1}(a)$
of the Wronski map, see  \eqref{ide}.
\end{cor}


\begin{proof}
The corollary follows from Lemma \ref{lem fib} and Theorems \ref{thm isoa}, \ref{thm ups}.
\end{proof}




\begin{cor}
\label{cor deg}
The degree of the Wronski map $\Wr$ equals $\dim V[\nu] = \binom{n}{m}$.
\qed
\end{cor}

\subsection{Dynamical Bethe algebra and quasi-polynomials}

The space $V = V_1^{\ox n}$ has a nontrivial zero weight subspace if $n$ is even.
Let $n=2m$. For the zero weight subspace $V[0]$,
 we have $\nu=0, \,m=\ell$, and 
assumptions \eqref{ass} take the form
\bean
\label{asss}
\zeta = \frac\mu 2 \quad \on{and} \quad 
\mu\notin\  \Z\,.
\eean
\smallskip


Let $a=(a_1,\dots,a_n)\in \C^n$ be such that the polynomial 
$x^n + \sum_{s=1}^n(-1)^s a_sx^{n-s}$ has distinct nonzero roots
$b_1,\dots,b_n$. 
Consider the functional space $E[\mu]$ as the module
over the  dynamical Bethe algebra $\B(b_1,\dots,b_n;E[\mu])$, see
Section \ref{sec BE}.  Consider the
$\O_a(\zeta;m,m)$-module $\O_a(\zeta;m,m)$, where
$\O_a(\zeta;m,m)$ is the algebra of functions on the fiber $\Wr^{-1}(a)$
of the Wronski map.



\begin{cor}
\label{lem isofi}

Under  assumptions \eqref{asss}, the isomorphisms 
$\tau$, $\phi$ and the isomorphism $V[0]\to E[\mu]$  in Corollary \ref{iso B}
induce the isomorphism of the $\B(b_1,\dots,b_n;E[\mu])$-module $E[\mu]$ and
  the $\O_a(\zeta;m,m)$-module $\O_a(\zeta;m,m)$.
  \qed
  \end{cor}

\subsection{Weyl involution and transposition of quasi-polynomials}



Consider the 
\\
$\Bm$-module $\V^S[\nu]$ and  $\B(-\mu, \ell,m)$-module
$\V^S[-\nu]$. Consider the $\Oz$-module $\Oz$ and 
$\mc O(-\zeta, \ell, m)$-module $\mc O(-\zeta, \ell, m)$.

Under assumptions \eqref{ass}, consider the diagram,

\bean
\label{comd}
\begin{tikzcd}
 (\Bm,\,  \V^S[\nu])  \arrow[r, ] \arrow[d, ] & (\B(-\mu, \ell,m),\,\V^S[-\nu])  \arrow[d, ] 
\\
 (\Oz,\,  \Oz) \arrow[r,   ]  & (\mc O(-\zeta, \ell, m), \,\mc O(-\zeta, \ell, m)
)
\end{tikzcd}\ \   .
\eean
Here $\V^S[\nu] \to    \Oz$ and 
$\V^S[-\nu] \to \mc O(-\zeta, \ell, m)$ are  the module isomorphisms of Theorem \ref{thm ups}.
The map $ \V^S[\nu] \to \V^S[-\nu]$ is the module isomorphism of Lemma \ref{lem BB iso}.
The map    $\Oz \to  \mc O(-\zeta, \ell, m)$ is the module isomorphism defined by the transposition
of the quasi-polynomials $p,q$.

\begin{thm}
\label{thm tra}
The diagram \eqref{comd} is commutative.

\end{thm}

\begin{proof}
The theorem follows from Lemma \ref{e=e}.
\end{proof}

The commutativity of diagram \eqref{comd} implies the commutativity of the 
diagram of fibers over a generic point $a\in\C^n$,
\bean
\label{cofd}
\begin{tikzcd}
 (\B(b_1,\dots,b_n;\mu, V[\nu]), \,  V[\nu])  \arrow[r, ] \arrow[d, ] &
 (\B(b_1,\dots,b_n;-\mu, V[-\nu]), \,  V[-\nu])  \arrow[d, ] 
\\
   (\O_a(\zeta, m,\ell),\,\O_a(\zeta, m,\ell)) \arrow[r,   ]  & (\mc O_a(-\zeta, \ell, m),\,\O_a(-\zeta,\ell,m))
\end{tikzcd}  \ \ ,
\eean
see notations in Section \ref{sec imo}. 

Combining commutative diagrams \eqref{cofd} and \eqref{comD} we obtain the commutative
diagram
\bean
\label{cofd}
\begin{tikzcd}
 (\B(z; E[\mu]),\, E[\mu]) \arrow[r,   ]\arrow[d,]  & (\B(z; E[-\mu]),\, E[-\mu])\arrow[d, ]
\\
  (\O_a(\zeta, m,m),\,\O_a(\zeta, m,m)) \arrow[r,   ]  & (\mc O_a(-\zeta, m, m),\,\O_a(\zeta, m,m))
\end{tikzcd}  \ \ ,
\eean
which holds if $n=2m$ is even and $\mu\notin\Z$. The diagram 
identifies the Weyl involution $E[\mu]\to E[-\mu]$ in the functional spaces
of eigenfunctions of the KZB operator $H_0$ 
with the isomorphism $\O_a(\zeta, m,m)\to \O_a(-\zeta, m,m)$
induced by the transposition  of quasi-polynomials.






\bigskip
\begin{thebibliography}{[COGP]}
\normalsize
\frenchspacing
\raggedbottom

\bibitem[Ch]{Ch} I.\,Cherednik,
{\it Generalized braid groups and local $r$-matrix systems}, (in Russian)
\href{http://mi.mathnet.ru/eng/dan7069}{Dokl. Akad. Nauk SSSR}, {\bf 307} (1989), no.\;1, 49–53 . 
% translation in {Soviet Mathematics Doklady} {\bf 40}, no.1 (1990), 43--8


\bibitem[EFK]{EFK} P.\,Etingof, I.\,Frenkel, and A.\,Kirillov Jr, {\it
Lectures on Representation Theory and Knizhnik-Zamolodchikov Equations},
Providence: \href{http://dx.doi.org/10.1090/surv/058}{American Mathematical Society} {\bf 58} (1998)


\bi[EV]{EV} P.\,Etingof, A.\,Varchenko,  {\it Dynamical Weyl groups and  applications}, \href{https://doi.org/10.1006/aima.2001.2034}{Adv. Math.} {\bf 167} (2002), no.\;1, 74--127 


\bi[FV1]{FV1} G.\,Felder and A.\,Varchenko, {\it Integral representation of solutions of the elliptic
Knizhnik-Zamolodchikov-Bernard equations}, \href{https://doi.org/10.1155/S1073792895000171}{Int. Math. Res. Notices} 1995, no.\;5, 221–233


\bi[FV2]{FV2} G.\,Felder, A.\,Varchenko,  
 {\it Three formulas for eigenfunctions of integrable Schrodinger operators},
 \href{https://doi.org/10.1023/A:1000138423050}{\em Compositio Math.} {\bf 107} (1997), no.\;2, 143--175 


\bi [FW]{FW} G.\,Felder and C.\,Wieszerkowski, {\it Conformal blocks on elliptic curves and the Knizhnik-Zamolodchikov-Bernard equations}, \href{https://doi.org/10.1007/BF02099366}{CMP} {\bf 176} (1996), 133-161


\bi[JV]{JV} E.\,Jensen, A.\,Varchenko, {\it
Norms of eigenfunctions of trigonometric KZB operators},
 \href{https://doi.org/10.1093/imrn/rns017}{IMRN}, Volume 2013 (2013), no.\;6, 1230--1267 


\bi[MaV]{MaV}
Y.\,Markov, A.\,Varchenko,  {\it
Hypergeometric Solutions of Trigonometric KZ Equations satisfy
Dynamical Difference  Equations}, \href{https://doi.org/10.1006/aima.2001.2027}{Adv. Math.} 166 (2002), no.\;1, 100--147


\bi[MTV1]{MTV1}
E. Mukhin, V. Tarasov, A. Varchenko, {\it
Generating operator of XXX or Gaudin transfer matrices has quasi-exponential kernel}, \href{https://doi.org/10.3842/SIGMA.2007.060}{SIGMA} {\bf 3} (2007), 060, 31 pages 


\bi[MTV2]{MTV2}
E.\,Mukhin, V.\,Tarasov, A.\,Varchenko,
{\it Spaces of quasi-exponentials and representations of \;$\gln$\>}, \href{https://doi.org/10.1088/1751-8113/41/19/194017}{J.~Phys.~A} {\bf 41} (2008) 194017, 1--28. % no.\;19


\bi[MTV3]{MTV3}
E.\,Mukhin, V.\,Tarasov, A.\,Varchenko, {\it
Schubert calculus and representations of general linear group\/}, \href{https://doi.org/10.1090/S0894-0347-09-00640-7}{J. Amer. Math. Soc.} {\bf 22} (2009), 909--940


\bi[MTV4]{MTV4}
E.\,Mukhin, V.\,Tarasov, A.\,Varchenko,
{\it On reality property of Wronski maps},
\href{https://doi.org/10.1142/S1793744209000092}{Confluentes Mathematici} {\bf 1} (2009), no.\;2, 225--247


\bi[MTV5]{MTV5}
E.\,Mukhin, V.\,Tarasov, A.\,Varchenko,
{\it
The B. and M. Shapiro conjecture in real algebraic geometry and the Bethe ansatz},
\href{https://doi.org/10.4007/annals.2009.170.863}{Ann. of Math.} {\bf 170}  (2009), no.\;2, 863--881


\bi[MTV6]{MTV6}
E.\,Mukhin, V.\,Tarasov, A.\,Varchenko,
 {\it
Bethe algebra of the $\frak{gl}_{N+1}$ Gaudin model and algebra of functions on the critical set of the master function}, \href{https://doi.org/10.1142/9789814324373_0016}{New trends in quantum integrable systems}, World Sci. Publ., Hackensack, NJ, 2011, 307--324


\bi[MV1]{MV1}
E.\,Mukhin and A.\,Varchenko,
 {\it
Norm of a Bethe Vector and the Hessian of the Master Function},
 \href{https://doi.org/10.1112/S0010437X05001569}{Compos. Math.} {\bf 141} (2005), no.\;4, 1012--1028


\bi[MV2]{MV2}
E.\,Mukhin and A.\,Varchenko,
{\it Quasi-polynomials and the Bethe Ansatz}, \href{https://doi.org/10.2140/gtm.2008.13.385}{Geometry $\&$ Topology Monographs} {\bf 13} (2008), 385-420


\bi[RST]{RST}  V.\,Rubtsov, A.\,Silantyev, D.\,Talalaev,
{\it  Manin Matrices, Quantum Elliptic Commutative Families and Characteristic Polynomial of Elliptic Gaudin Model},
\href{https://doi.org/10.3842/SIGMA.2009.110}{SIGMA} {\bf 5} (2009), 110, 22 pages


\bi[RV]{RV}
N.\,Reshetikhin, A.\,Varchenko, {\it
Quasiclassical asymptotics of solutions to the KZ equations},
 Geometry, Topology and Physics for R. Bott, Intern. Press, 1995, 293--322, 


\bi[SV]{SV} V.\,Schechtman and A.\,Varchenko, {\it Arrangements of hyperplanes
and Lie algebra homology\/}, \href{https://doi.org/10.1007/BF01243909}{Invent. Math.} {\bf 106} (1991), 139--194


\bi[ScV]{ScV} I.\,Scherbak, A.\,Varchenko, {\it
Critical points of functions, $\slt$ representations, and
Fuchsian differential equations with  only univalued solutions}, \href{https://doi.org/10.17323/1609-4514-2003-3-2-621-645}{Moscow Math. J.} {\bf 3} (2003), no.\;2, 621--645
 

\bi[TV]{TV}
V.\,Tarasov and A.\,Varchenko,  {\it Difference Equations Compatible with Trigonometric KZ Differential Equations}, \href{https://doi.org/10.1155/S1073792800000441}{IMRN} 2000, no.\;15, 801--829


\bi[ThV]{ThV} D.\,Thompson, A.\,Varchenko, {\it Dynamical Elliptic Bethe Algebra, KZB Eigenfunctions, and Theta-Polynomials},
\href{https://www.prior-sci-pub.com/lims_2019art6.html} {CJAM} {\bf 1} (2020), no.\;1, 78--125 
%{\tt  	arXiv:1810.09001}, 2018

\bi[V]{V} A.\,Varchenko, {\it Quantum Integrable Model of an Arrangement of Hyperplanes}, \href{https://doi.org/10.3842/SIGMA.2011.032}{SIGMA} 7 (2011), 032, 55 pages

\end{thebibliography}






\end{document}



