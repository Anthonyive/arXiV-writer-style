\documentclass[12pt]{amsart}
\usepackage{amssymb,latexsym,amsmath,amscd,amsthm,graphicx, color}
\usepackage[all]{xy}
\usepackage{pgf,tikz}
\usepackage{mathrsfs}


\usetikzlibrary{arrows}
\definecolor{uuuuuu}{rgb}{0.26666666666666666,0.26666666666666666,0.26666666666666666}
\definecolor{xdxdff}{rgb}{0.49019607843137253,0.49019607843137253,1.}
\definecolor{ffqqqq}{rgb}{1.,0.,0.}
\definecolor{ffqqqq}{rgb}{1.,0.,0.}
\definecolor{ffxfqq}{rgb}{1.,0.4980392156862745,0.}
\definecolor{xdxdff}{rgb}{0.49019607843137253,0.49019607843137253,1.}
\definecolor{ffqqqq}{rgb}{1.,0.,0.}
\definecolor{ududff}{rgb}{0.30196078431372547,0.30196078431372547,1.}
\raggedbottom

\pagestyle{empty}
\newcommand{\degre}{\ensuremath{^\circ}}
\definecolor{uuuuuu}{rgb}{0.26666666666666666,0.26666666666666666,0.26666666666666666}
\definecolor{qqwuqq}{rgb}{0.,0.39215686274509803,0.}
\definecolor{zzttqq}{rgb}{0.6,0.2,0.}
\definecolor{xdxdff}{rgb}{0.49019607843137253,0.49019607843137253,1.}
\definecolor{qqqqff}{rgb}{0.,0.,1.}
\definecolor{cqcqcq}{rgb}{0.7529411764705882,0.7529411764705882,0.7529411764705882}
\definecolor{sqsqsq}{rgb}{0.12549019607843137,0.12549019607843137,0.12549019607843137}


\setlength{\oddsidemargin}{0 in} \setlength{\evensidemargin}{0 in}
\setlength{\textwidth}{6.75 in} \setlength{\topmargin}{-.6 in}
\setlength{\headheight}{.00 in} \setlength{\headsep}{.3 in }
\setlength{\textheight}{10 in} \setlength{\footskip}{0 in}


\theoremstyle{plain}
\newtheorem{Q}{Question}
\newtheorem{theorem}[subsection]{Theorem}
\newtheorem*{con}{Conclusion}
\newtheorem{corollary}[subsection]{Corollary}
\newtheorem{lemma}[subsection]{Lemma}
\newtheorem{defi}[subsection]{Definition}
\newtheorem{prop}[subsection]{Proposition}

\theoremstyle{definition}
\newtheorem{defi1}[subsection]{Definition}
\newtheorem{prop1}[subsection]{Proposition}
\newtheorem{cor}[subsection]{Corollary}
\newtheorem{exam}[subsection]{Example}
\newtheorem{fact}[subsection]{Fact}
\newtheorem{remark}[subsection]{Remark}
\newtheorem{rem}[subsection]{Remark}
\newtheorem{claim}[subsection]{Claim}
\newtheorem{imp}[section]{}
\newtheorem{notation}[subsection]{Notation}
\newtheorem{note}[subsection]{Note}
\newcommand{\tarc}{\mbox{\large$\frown$}}
\newcommand{\arc}[1]{\stackrel{\tarc}{#1}}


%Lattice operations
\newcommand{\jj}{\vee}% join
\newcommand{\mm}{\wedge}% meet
\newcommand{\JJ}{\bigvee}% big join
\newcommand{\MM}{\bigwedge}% big meet
\newcommand{\JJm}[2]{\JJ(\,#1\mid#2\,)}% big join with a middle
\newcommand{\MMm}[2]{\MM(\,#1\mid#2\,)}% big meet with a middle


%Set operation
\newcommand{\uu}{\cup}% union
\newcommand{\ii}{\cap}% intersection
\newcommand{\UU}{\bigcup}% big union
\newcommand{\II}{\bigcap}% big intersection
\newcommand{\UUm}[2]{\UU\{\,#1\mid#2\,\}}% big union with a middle
\newcommand{\IIm}[2]{\II\{\,#1\mid#2\,\}}% big intersection with a middle

%Sets
\newcommand{\ci}{\subseteq}% contained in with equality
\newcommand{\nc}{\nsubseteq}% not \ci
\newcommand{\sci}{\subset}% strictly contained in
\newcommand{\nci}{\nc}% not \ci
\newcommand{\ce}{\supseteq}% containing with equality
\newcommand{\sce}{\supset}
\newcommand{\nce}{\nsupseteq}% not \ce
\newcommand{\nin}{\notin}% not \in
\newcommand{\es}{\emptyset}% the empty set
\newcommand{\set}[1]{\{#1\}}% set
\newcommand{\setm}[2]{\{\,#1\mid#2\,\}}% set with a middle

%Partial ordering
\newcommand{\nle}{\nleq}% not \leq

%Greek letters
\newcommand{\ga}{\alpha}
\newcommand{\gb}{\beta}
\newcommand{\gc}{\chi}
\newcommand{\gd}{\delta}
\renewcommand{\gg}{\gamma}% old use >>
\newcommand{\gh}{\eta}
\newcommand{\gi}{\iota}
\newcommand{\gk}{\kappa}
\newcommand{\gl}{\lambda}
\newcommand{\gm}{\mu}
\newcommand{\gn}{\nu}
\newcommand{\go}{\omega}
\newcommand{\gp}{\pi}
\newcommand{\gq}{\theta}
\newcommand{\gs}{\sigma}
\newcommand{\gt}{\tau}
\newcommand{\gx}{\xi}
\newcommand{\gy}{\psi}
\newcommand{\gz}{\zeta}
\newcommand{\vp}{\varphi}

\newcommand{\gG}{\Gamma}
\newcommand{\gD}{\Delta}
\newcommand{\gF}{\Phi}
\newcommand{\gL}{\Lambda}
\newcommand{\gO}{\Omega}
\newcommand{\gP}{\Pi}
\newcommand{\gQ}{\Theta}
\newcommand{\gS}{\Sigma}
\newcommand{\gX}{\Xi}
\newcommand{\gY}{\Psi}

%Font command
\newcommand{\tbf}{\textbf}% text bold
\newcommand{\tit}{\textit}% text italic

\newcommand{\mbf}{\mathbf}% math bold
\newcommand{\B}{\boldsymbol}% Bold math symbol, use as \B{a}
\newcommand{\C}[1]{\mathcal{#1}}% Euler Script - only caps, use as \C{A}
\newcommand{\D}[1]{\mathbb{#1}}% Doubled - blackboard bold - only caps, uas as \D{A}
\newcommand{\F}[1]{\mathfrak{#1}}% Fraktur, use as \F{a}

%Miscellaneous
\newcommand{\te}{\text}% same as \mathrm command.
\newcommand{\tei}{\textit}
\newcommand{\im}{\implies}
\newcommand{\ti}{\times}
\newcommand{\exi}{\ \exists \ }
\newcommand{\ther}{\therefore}
\newcommand{\be}{\because}
\newcommand{\ep}{\epsilon}
\newcommand{\la}{\langle}
\newcommand{\ra}{\rangle}
\newcommand{\ol}{\overline}
\newcommand{\ul}{\underline}
\newcommand{\q}{\quad}% spacing
\newcommand{\qq}{\qquad}% morepspacing
\newcommand{\fa}{\ \forall \, }
\newcommand{\nd}{\noindent}
\newcommand{\bs}{\backslash}
\newcommand{\pa}{\partial}
\newcommand{\para}{\parallel}
\newcommand{\sm}{\setminus}
\newcommand{\tl}{\tilde}
\begin{document}

\title{Quantization coefficients for uniform distributions on the boundaries of regular polygons}

%\title{Quantization coefficients for uniform distributions on the boundaries of regular polygons}


\address{School of Mathematical and Statistical Sciences\\
University of Texas Rio Grande Valley\\
1201 West University Drive\\
Edinburg, TX 78539-2999, USA.}


\email{\{$^1$joel.hansen01, $^2$itzamar.marquez01, $^3$mrinal.roychowdhury, $^4$eduardo.torres05\}\linebreak@utrgv.edu}


\author{$^1$Joel Hansen}
\author{$^2$Itzamar Marquez}
\author{$^3$Mrinal K. Roychowdhury}
\author{$^4$Eduardo Torres}




\subjclass[2010]{60Exx, 94A34.}
\keywords{Uniform distribution, optimal sets, quantization error, quantization coefficient, regular polygon}


\date{}
\maketitle

\pagestyle{myheadings}\markboth{J. Hansen, I. Marquez, M.K. Roychowdhury, and E. Torres}
{Quantization coefficients for uniform distributions on the boundaries of regular polygons}

\begin{abstract}
In this paper, we give a general formula to determine the quantization coefficients for uniform distributions defined on the boundaries of different regular $m$-sided polygons inscribed in a circle. The result shows that the quantization coefficient for the uniform distribution on the boundary of a regular $m$-sided polygon inscribed in a circle is an increasing function of $m$, and approaches to the quantization coefficient for the uniform distribution on the circle as $m$ tends to infinity.
\end{abstract}








\section{Introduction}

Let $\D R^d$ denote the $d$-dimensional Euclidean space, $\|\cdot\|$ denote the Euclidean norm on $\D R^d$ for any $d\geq 1$, and $n\in \D N$. For a finite set $\ga\sci \D R^d$, the \tit{cost} or \tit{distortion error} for $P$ with respect to the set $\ga$, denoted by $V(P; \ga)$, is defined by
\[V(P; \ga):= \int \min_{a\in\alpha} \|x-a\|^2 dP(x).\]
Then, the $n$th quantization error for $P$, denoted by $V_n:=V_n(P)$, is defined by
\begin{equation*} \label{eq0} V_n:=V_n(P)=\inf \Big\{V(P; \ga) : \ga\sci \D R^d, 1\leq \text{ card}(\ga) \leq n \Big\}.\end{equation*}
A set $\ga$ for which the infimum is achieved and contains no more than $n$ points is called an \textit{optimal set of $n$-means}. It is well-known that for a continuous probability measure an optimal set of $n$-means always contains exactly $n$ elements. If $P$ is the probability distribution, then an optimal set of $n$-means is denoted by $\ga_n:=\ga_n(P)$. Optimal sets of $n$-means for different probability distributions were determined by several authors, for example, see \cite{CR1, CR2, DR1, DR2, GL2, L, R1, R2, R3, R4, R5, R6, RR1, RS}. It has broad applications in engineering and technology (see \cite{GG, GN, Z}). For any $s\in (0, +\infty)$, the number \[\lim_{n\to \infty} n^{\frac 2 s} V_n(P),\]
if it exists, is called the $s$-dimensional \tit{quantization coefficient} for $P$. Bucklew and Wise (see \cite{BW}) showed that for a Borel probability measure $P$ with non-vanishing absolutely continuous part the quantization coefficient exists as a finite positive number.
For some more details interested readers can also see \cite{GL1, P}.
Let $E(X)$ represent the expected value of a random variable $X$ associated with a probability distribution $P$. Let $\ga$ be an optimal set of $n$-means for $P$, and $a\in \ga$. Then, it is well-known that
$a=E(X : X \in M(a|\ga)),$
where $M(a|\ga)$ is the Voronoi region of $a\in \ga , $ i.e.,  $M(a|\ga)$ is the set of all elements $x$ in $\D R^d$ which are closest to $a$ among all the elements in $\ga$ (see \cite{GG, GL1}).


From the work of Rosenblatt and Roychowdhury (see \cite{RR2}), it is known that the quantization coefficient for the uniform distribution on a unit circle is $\frac {\pi^2}3$; on the other hand, from the work of Pena et al. (see \cite{PRRSS}), it is known that the quantization coefficient for the uniform distribution on the boundary of a regular hexagon inscribed in a unit circle is $3$. Notice that a regular $m$-sided polygon inscribed in a circle tends to the circle as $m$ tends to infinity. Pena et al. conjectured that the quantization coefficient for the uniform distribution on the boundary of a regular $m$-sided polygon inscribed in a circle is an increasing function of $m$ (see \cite{PRRSS}), and approaches to the quantization coefficient for the uniform distribution on the circle as $m$ tends to infinity. In this paper, we prove that the conjecture is true.

The arrangement of the paper is as follows: First, we prove a theorem Theorem~\ref{theo00}, which gives a technique how to calculate the optimal sets of $n$-means and the $n$th quantization errors for all positive integers $n$ for a uniform distribution defined on any line segment. Next, let $P$ be the uniform distribution defined on the boundary of a regular $m$-sided polygon inscribed in a unit circle. In Proposition~\ref{prop22}, for $k\geq 2$, we determine the optimal set of $mk$-means and the $mk$th quantization error for the probability distribution $P$. Then, with the help of the proposition, in Theorem~\ref{theo01}, we have shown that the quantization coefficient for $P$ exists, and equals $\frac{1}{3} m^2 \sin ^2\left(\frac{\pi }{m}\right)$, i.e.,
\[\lim_{n\to \infty} n^2 V_n(P)=\frac{1}{3} m^2 \sin ^2\left(\frac{\pi }{m}\right).\]
Notice that $\frac{1}{3} m^2 \sin ^2\left(\frac{\pi }{m}\right)$ is an increasing function of $m$, and $\lim\limits_{m\to \infty} \frac{1}{3} m^2 \sin ^2\left(\frac{\pi }{m}\right)=\frac{\pi^2}{3},$
which is the quantization coefficient for a uniform distribution on the unit circle (see \cite{RR2}). Thus, the result in this paper, shows that the conjecture given by Pena et al. in \cite{PRRSS} is true.


\section{Main result}

In this section, first we give some basic definitions.

%In Euclidean geometry, a regular polygon is a polygon that is equiangular, i.e., all angles are equal in measure; and equilateral, i.e., all sides have the same length. A regular $m$-sided polygon has rotational symmetry of order $m$. All vertices of a regular polygon lie on a common circle, i.e., the circumscribed circle; i.e., they are concyclic points. That is, a regular polygon is a cyclic polygon. Let $P$ be a uniform distribution on the boundary of a regular $m$-sided polygon for a give $m\geq 3$. Let $\ga_n$ be an optimal set of $n$-means for $P$. Since $P$ is uniform, and regular $m$ sided polygon has symmetry of order $m$, it is not difficult to show that $\ga_m$ will contain $m$ elements, each from the interior of the $m$ angles of the regular $m$-sided polygon; and for any positive integer $k\geq 2$, $\ga_{mk}$ will contain $m$ points, each from the interior of the $m$ angles, and $(k-1)$ points from each side of the regular $m$-sided polygon. Moreover, the following is true: Let $A$ be one of the vertices of the regular $m$-sided polygon, and for $k\geq 1$, let $a$ be an element of an optimal set of $mk$-means that lies in the interior of $\angle A$. Further, let $AA_1$ and $AA_2$ be the two adjacent sides of the vertex $A$. Then, the boundary of the Voronoi region of $a$ will cut $AA_1$ and $AA_2$ at two points $D_1$ and $D_2$ such that $|AD_1|=|AD_2|=r$ for some real $r$ such that $0<r\leq \frac {\ell} 2$, where $\ell$ is the length of the sides of the polygon.

Let $i$ and $j$ be the unit vectors in the positive directions of the $x_1$- and $x_2$-axes, respectively. By the position vector $a$ of a point $A$, it is meant that $\overrightarrow{OA}=a$. We will identify the position vector of a point $(a_1, a_2)$ by $(a_1, a_2):=a_1 i +a_2 j$, and apologize for any abuse in notation. For any two position vectors $  a:=( a_1, a_2)$ and $  b:=( b_1, b_2)$, we write $\rho(a, b):=\|(a_1, b_1)-(a_2, b_2)\|^2=(a_1-a_2)^2 +(b_1-b_2)^2$, which gives the squared Euclidean distance between the two points $(a_1, a_2)$ and $(b_1, b_2)$.  Let $P$ and $Q$ belong to an optimal set of $n$-means for some positive integer $n$, and let $D$ be a point on the boundary of the Voronoi regions of the points $P$ and $Q$. Since the boundary of the Voronoi regions of any two points is the perpendicular bisector of the line segment joining the points, we have
$|\overrightarrow{DP}|=|\overrightarrow{DQ}|, \te{ i.e., } (\overrightarrow{DP})^2=(\overrightarrow{DQ})^2$ implying
$(  p-  d)^2=(  q-  d)^2$, i.e., $\rho(  d,   p)-\rho(  d,   q)=0$, where $p, q, d$ are, respectively, the position vectors of the points $P, Q, D$. We call such an equation a \tit{canonical equation}.

Let us now give the following theorem.



\begin{theorem} \label{theo00}
Let $AB$ be a line segment joining the two points $A$ and $B$ given by the position vectors $a:=(a_1, b_1)$ and $b:=(a_2, b_2)$, respectively. Let $\mu$ be a uniform distribution on $AB$. Let $M(t)$ be the parametric representation of $AB$ for $0\leq t\leq 1$ such that $M(0)=a$, and $M(1)=b$. Let $D_1$ and $D_2$ be two points on the segment $AB$ at distances $r_1$ and $r_2$ from $A$ and $B$, respectively (see Figure~\ref{Fig1}). Then, the optimal set of $n$-means for $\mu$ on the segment $D_1D_2$, is given by
\[\ga_n(\mu,  \, D_1D_2):=\Big\{M(\frac {r_1}{\ell}+\frac {2j-1}{2n} (1-\frac{r_2}{\ell}-\frac{r_1}{\ell})) : 1\leq j\leq n \Big\},\]
with the  $n$th quantization error for $\mu$ on the segment $D_1D_2$,
\[V_n(\mu, \, D_1D_2):=n \int_{\frac {r_1}{\ell}}^{\frac {r_1}{\ell}+\frac {1}{n} (1-\frac{r_2}{\ell}-\frac{r_1}{\ell})}\rho\Big(M(t), M(\frac {r_1}{\ell}+\frac {1}{2n} (1-\frac{r_2}{\ell}-\frac{r_1}{\ell}))\Big) d\mu,\]
where $\ell$ is the length of the line segment $AB$.
\end{theorem}

\begin{figure}
\begin{tikzpicture}[line cap=round,line join=round,>=triangle 45,x=1.0cm,y=1.0cm]
\clip(-4.434877892537599,-0.5065236901434486) rectangle (11.67941711302465,5.21943965200402);
\draw [line width=0.5pt,color=ffqqqq] (-4.,1.)-- (9.,5.);
\draw (-4.331619416758946,1.0466997550116428) node[anchor=north west] {$A(M(0)=a)$};
\draw (8.775247841033814,5.110995388362796) node[anchor=north west] {$B(M(1)=b)$};
\draw (6.616698485521949,4.383174795378465) node[anchor=north west] {$D_2(d_2=M(1-\frac{r_2}{\ell}))$};
\draw (-2.425873138919643,1.624555378619519) node[anchor=north west] {$D_1(d_1=M(\frac{r_1}{\ell}))$};
\draw [line width=0.5 pt,dotted] (0.,0.)-- (9.,5.);
\draw [line width=0.5 pt,dotted] (0.,0.)-- (-4.,1.);
\draw (-0.07285987750580951,0.02995135022337276) node[anchor=north west] {$O$};
\draw (1.7244421366922663,3.055424263641221) node[anchor=north west] {$|AB|=\ell$};
\draw (-4.331619416758946,2.124555378619519) node[anchor=north west] {$|AD_1|=r_1$};
\draw (6.616698485521949,5.383174795378465) node[anchor=north west] {$|BD_2|=r_2$};
\draw [->,line width=0.5pt,dotted] (0.,0.) -- (6.,0.);
\draw [->,line width=0.5pt,dotted] (0.,0.) -- (-3.,0.);
\draw [->,line width=0.5pt,dotted] (0.,0.) -- (0.,4.);
\draw [->,line width=0.5 pt,dotted] (0.,0.) -- (0.02,-.5377855623607878);
\begin{scriptsize}
\draw [fill=xdxdff] (-4.,1.) circle (1.5pt);
\draw [fill=ududff] (9.,5.) circle (1.5pt);
\draw [fill=xdxdff] (-2.0015135135135136,1.614918918918919) circle (1.5pt);
\draw [fill=xdxdff] (7.016972972972972,4.389837837837838) circle (1.5pt);
\draw [fill=xdxdff] (0.,0.) circle (1.5pt);
\end{scriptsize}
\end{tikzpicture}
\caption{ }
\label{Fig1}
\end{figure}

\begin{proof}
Since $\ell$ is the length of the line segment $AB$, the probability density function (pdf) $f$ of the uniform distribution $\mu$ on $AB$ is given by $f(x_1, x_2)=\frac 1{\ell}$ for all $(x_1, x_2)\in AB$, and zero, otherwise.
Let $s$ represent the distance of any point on $AB$ from the point $A$. Then, we have $d\mu=d\mu(s)=\mu(ds)=f(x_1, x_2) ds=\frac 1 {\ell} \,ds$. Notice that $ds=\sqrt{(\frac {dx_1}{dt})^2 +(\frac {dx_2}{dt})^2}\,dt=\ell \,dt$ yielding $d\mu=dt$. Given, the parametric representation of the line segment $AB$ is $M(t)$ for $0\leq t\leq 1$ with $M(0)=a$ and $M(1)=b$. Hence, the parameters for the points $D_1$ and $D_2$, which are at distances $r_1$ and $r_2$ from $A$ and $B$ are, respectively, given by $t=\frac {r_1}{\ell}$ and $t=1-\frac {r_2}{\ell}$, i.e., if $d_1$ and $d_2$ are the position vectors of the points $D_1$ and $D_2$ (see Figure~\ref{Fig1}), then we have
\[d_1=M(\frac {r_1}{\ell}), \te{ and } d_2=M(1-\frac {r_2}{\ell}).\]
In fact, we can identify the line segment $D_1D_2$ by its parameters in the closed interval $[\frac {r_1}{\ell}, 1-\frac {r_2}{\ell}].$ By \cite{RR2}, we know that the optimal set of $n$-means with respect to an uniform distribution in the closed interval  $[\frac {r_1}{\ell}, 1-\frac {r_2}{\ell}]$ is given by the set
\[\Big \{\frac {r_1}{\ell}+\frac {2j-1}{n}(1-\frac {r_2}{\ell}-\frac {r_1}{\ell}) : 1\leq j\leq n\Big\}.\]
Hence, the optimal set of $n$-means for $\mu$ on the segment $D_1D_2$, is given by
\[\ga_n(\mu,  \, D_1D_2):=\Big\{M(\frac {r_1}{\ell}+\frac {2j-1}{2n} (1-\frac{r_2}{\ell}-\frac{r_1}{\ell})) : 1\leq j\leq n \Big\}.\]
If $V_n(\mu, \, D_1D_2)$ is the corresponding quantization error, we have
\begin{align*} V_n(\mu, \, D_1D_2)=n\Big(\te{Quantization error due to the point } M(\frac {r_1}{\ell}+\frac {1}{2n} (1-\frac{r_2}{\ell}-\frac{r_1}{\ell}))\Big).
\end{align*}
Again, notice that any point on the line segment $D_1D_2$ is given by $M(t)$ for $\frac{r_1}{\ell}\leq t\leq 1-\frac{r_2}{\ell}$, and the parameters for the points at which the boundary of the Voronoi region of $M(\frac {r_1}{\ell}+\frac {1}{2n} (1-\frac{r_2}{\ell}-\frac{r_1}{\ell}))$ cuts the segment $D_1D_2$ are given by $t=\frac{r_1}{\ell}$, and $t=\frac {r_1}{\ell}+\frac {1}{n} (1-\frac{r_2}{\ell}-\frac{r_1}{\ell})$. Hence, we have
\[V_n(\mu, \, D_1D_2)=n \int_{\frac {r_1}{\ell}}^{\frac {r_1}{\ell}+\frac {1}{n} (1-\frac{r_2}{\ell}-\frac{r_1}{\ell})}\rho\Big(M(t), M(\frac {r_1}{\ell}+\frac {1}{2n} (1-\frac{r_2}{\ell}-\frac{r_1}{\ell}))\Big) d\mu.\]
Thus, the proof of the theorem is complete.
\end{proof}
Let the equation of the unit circle be $x_1^2+x_2^2=1$. Let $A_1A_2A_3\cdots A_m$ be a regular $m$-sided polygon for some $m\geq 3$ inscribed in the circle. Without any loss of generality due to rotational symmetry, we can always assume that the vertex $A_1$ lies on the $x_1$-axis, i.e., the vertex $A_1$ is the point where the circle intersects the positive direction of the $x_1$-axis. Again, notice that each side of the regular $m$-sided polygon subtends a central angle of radian $\frac {2\pi}m$. Thus, the position vectors $\tilde a_j$ of the vertices $A_j$ are given by $\tilde a_j=(\cos \frac{2 \pi }{m} (j-1), \sin\frac{2 \pi }{m} (j-1))$ for $1\leq j\leq m$. Let $\ell$ be the length of each side of the polygon, then we have
\begin{equation} \label{eq100} \ell=\|\tilde a_m-\tilde a_{m-1}\|=\|\tilde a_{m-1}-\tilde a_{m-2}\|=\cdots=\|\tilde a_2-\tilde a_1\|=2 \sin \frac{\pi }{m}.
\end{equation}
Let $L$ be the boundary of the polygon. Then, we can write
\[L=\UU_{j=1}^m L_j,\]
where $L_j$ is the side $A_jA_{j+1}$, and $A_{m+1}$ is identified as the vertex $A_1$. Then, for $1\leq j\leq m$, we can write
\[L_j:=A_jA_{j+1}=\set{M_j :  0\leq t\leq 1}, \te{ where } M_j=\tl a_{j+1}t+(1-t)\tl a_j.\]
Notice that $M_j$ is a function of $t$, and any point on the side $A_jA_{j+1}$ can be represented by $M_j:=M_j(t)$ for $0\leq t\leq 1$. Thus, we see that $M_j(0)=\tilde a_j$, and $M_j(1)=\tilde a_{j+1}$ for $1\leq j\leq m$.
Let $P$ be the uniform distribution defined on the boundary $L$ of the polygon. Then, the probability density function (pdf) $f$ of the uniform distribution $P$ is given by $f(x_1, x_2)=\frac 1{m\ell}$ for all $(x_1, x_2)\in L$, and zero, otherwise.
Let $s$ represent the distance of any point on $L$ from the vertex $A_1$ tracing along the boundary $L$ in the counterclockwise direction. Then, we have $dP=dP(s)=P(ds)=f(x_1, x_2) ds=\frac 1 {m\ell} ds$. For $1\leq j\leq m$, on each $L_j$,  we have $ds=\sqrt{(\frac {dx_1}{dt})^2 +(\frac {dx_2}{dt})^2}|dt|=\ell |dt|$ yielding $dP(s)=\frac \ell{m\ell} |dt|=\frac 1m |dt|$.



\begin{remark} \label{rem1}
Since $P$ is uniform, and a regular $m$-sided polygon has symmetry of order $m$, it is not difficult to show that an optimal set $\ga_m$ will contain $m$ points, each from the interior of the $m$ angles of the regular $m$-sided polygon; and for any positive integer $k\geq 2$, $\ga_{mk}$ will contain $m$ points, each from the interior of the $m$ angles, and $(k-1)$ points from each side of the regular $m$-sided polygon. Moreover, the following is true: Let $A$ be one of the vertices of the regular $m$-sided polygon, and for $k\geq 2$, let $a$ be an element of an optimal set of $mk$-means that lies in the interior of $\angle A$. Further, let $AA_1$ and $AA_2$ be the two adjacent sides of the vertex $A$. Then, the boundary of the Voronoi region of $a$ will cut $AA_1$ and $AA_2$ at two points $D_1$ and $D_2$ such that $|AD_1|=|AD_2|=r$ for some real $r$ such that $0<r\leq \frac {\ell} 2$, where $\ell$ is the length of the sides of the polygon.
\end{remark}
\begin{prop} \label{prop22}
Let $\ga_n$ be an optimal set of $n$-means such that $n=mk$, where $k\in \D N$, and $k\geq 2$. Let $a_j$ be the points that $\ga_n$ contains from the interior of the angles $A_j$ of the regular $m$-sided polygon, $1\leq j\leq m$. Then,
\[\ga_n=\set{a_j : 1\leq j\leq m}\uu \UU_{j=1}^m \ga_{j, k-1},\]
where\begin{align*}
a_1&=(1-\frac{1}{2} r \sin (\frac{\pi }{m}),0),\\
a_j&=(\frac{1}{4} \cos \frac{2 \pi  (j-1)}{m}(r (\cos (\frac{2 \pi }{m})-1) \csc (\frac{\pi }{m})+4), \sin \frac{2 \pi  (j-1)}{m} (\frac{1}{4} r (\cos (\frac{2 \pi }{m})-1) \csc (\frac{\pi }{m})+1))
\end{align*}
for $2\leq j\leq m$, and
$\ga_{j, k-1}:=\set{M_j(\frac r \ell+\frac {2i-1}{2(k-1)}(1-\frac {2r}{\ell})) : 1\leq i\leq k-1}$ for $1\leq j\leq m$, and
\[r=\frac{4 \sin (\frac{\pi }{m})}{2 (k-1) \sqrt{3 \cos ^2(\frac{\pi }{m})+1}+4}.\] Moreover, the quantization error for $n$-means is given by
\begin{align*}
V_n=\frac{2 \sin ^2(\frac{\pi }{m}) (3 \cos (\frac{2 \pi }{m})+5)}{3 \Big(k \sqrt{6 \cos (\frac{2 \pi }{m})+10}-\sqrt{6 \cos (\frac{2 \pi }{m})+10}+4\Big)^2}.
\end{align*}
\end{prop}










\begin{proof}
Let $\ga_n$ be an optimal set of $n$-means, where $n=mk$ for some positive integer $k\geq 2$. Since $a_j$ are the points that $\ga_n$ contains from the interior of the angles $A_j$, by Remark~\ref{rem1}, due to uniform distribution and symmetry, we can say that there exists a real number $r$, where $0<r\leq \frac {\ell} 2$, such that the boundary of the Voronoi region of each $a_j$ will cut the the two adjacent sides at distances $r$ from the vertex $A_j$. Notice that the two adjacent sides of the vertex $A_1$ are $A_mA_1$ and $A_1A_2$ in the polygon. Again, by the hypothesis $a_1$ is the point that $\ga_n$ contains from $\angle A_1$. If the boundary of the Voronoi region of $a_1$ cuts $A_mA_1$ and $A_1A_2$ at $D_1$ and $D_2$, respectively, we have
\[a_1=E (X : X\in D_1A_1\uu A_1D_2)=\frac {\int_{D_1A_1}(x_1, x_2) dP+\int_{A_1D_2}(x_1, x_2) dP}{\int_{D_1A_1}1 dP+\int_{A_1D_2}1 dP},\]
which implies
\[a_1=\frac{\int_{1-\frac{r}{L}}^1 M_n(t) \, dt+\int_0^{\frac{r}{L}} M_1(t) \, dt}{\int_0^{\frac{r}{L}} 1 \, dt+\int_{1-\frac{r}{L}}^1 1 \, dt}=(1-\frac{1}{2} r \sin \left(\frac{\pi }{m}\right),0).\]
Similarly, for $2\leq j\leq m$, we obtain
\[a_j=\frac{\int_{1-\frac{r}{L}}^1 M_{j-1}(t) \, dt+\int_0^{\frac{r}{L}} M_j(t) \, dt}{\int_0^{\frac{r}{L}} 1 \, dt+\int_{1-\frac{r}{L}}^1 1 \, dt}\]
yielding
\[a_j=(\frac{1}{4} \cos \frac{2 \pi  (j-1)}{m}(r (\cos (\frac{2 \pi }{m})-1) \csc (\frac{\pi }{m})+4), \sin \frac{2 \pi  (j-1)}{m} (\frac{1}{4} r (\cos (\frac{2 \pi }{m})-1) \csc (\frac{\pi }{m})+1)).\]
Again, by Remark~\ref{rem1}, $\ga_n$ contains $(k-1)$ points from each side of the regular $m$-sided polygon. For $1\leq j\leq m$, let $\ga_{j, k-1}$ be the optimal set of $(k-1)$-means that $\ga_n$ contains from the side $A_jA_{j+1}$. Recall that the parametric representation of the side $A_jA_{j+1}$ is $M_j(t)$, and the $(k-1)$ means from each side occur due to an uniform distribution on the segment bounded by the two points represented by the parameters $t=\frac{r}{\ell}$ and $t=1-\frac{r}{\ell}$. Hence, by Theorem~\ref{theo00}, we have
\[\ga_{j, k-1}:=\Big\{M_j(\frac r \ell+\frac {2i-1}{2(k-1)}(1-\frac {2r}{\ell})) : 1\leq i\leq k-1\Big\}.\]
To calculate the quantization error, we proceed as follows:
By symmetry, the quantization error contributed by all the points $a_j$ for $1\leq j\leq m$ is given by
\begin{align*} m \int_{D_1A_1\uu A_1D_2}\rho((x_1, x_2), a_1) dP&=2 m\int_{A_1D_2}\rho((x_1, x_2), a_1) dP=2 \int_{0}^{\frac {r}{\ell}}\rho(M_1(t), a_1)) dt,
\end{align*}
implying
\begin{equation} \label{eq1}  m \int_{D_1A_1\uu A_1D_2}\rho((x_1, x_2), a_1) dP=\frac{1}{24} r^3 (3 \cos (\frac{2 \pi }{m})+5) \csc (\frac{\pi }{m}).
\end{equation}
Again, by Theorem~\ref{theo00}, the quantization error contributed by all the sets $\ga_{j, k-1}$ for $1\leq j\leq m$ is given by
\[m V_n(\mu, \, \ga_{j, k-1}):=(k-1)\int_{\frac {r}{\ell}}^{\frac {r}{\ell}+\frac {1}{k-1} (1-\frac{2r}{\ell})}\rho\Big(M(t), M(\frac {r}{\ell}+\frac {1}{2(k-1)} (1-\frac{2r}{\ell}))\Big) dt.\]
implying
\begin{equation} \label{eq2} m V_n(\mu, \, \ga_{j, k-1})=\frac 1{3(k-1)^2}\csc (\frac{\pi }{m}) (\sin (\frac{\pi }{m})-r)^3.
\end{equation}
Hence, by \eqref{eq1} and \eqref{eq2}, the quantization error for $n$-means is given by
\begin{align*}
V_n= \frac{1}{24} \csc (\frac{\pi }{m})\Big(r^3 (3 \cos(\frac{2 \pi }{m})+5)+\frac 8{(k-1)^2}(\sin(\frac{\pi }{m})-r)^3\Big).
\end{align*}
Notice that for a given $k$, the quantization error $V_n$ is a function of $r$. Solving $\frac{\pa V_n}{\pa r}=0$, we have $r=\frac{4 \sin (\frac{\pi }{m})}{2 (k-1) \sqrt{3 \cos ^2(\frac{\pi }{m})+1}+4}$. Putting $r=\frac{4 \sin (\frac{\pi }{m})}{2 (k-1) \sqrt{3 \cos ^2(\frac{\pi }{m})+1}+4}$, we have
\[V_n=\frac{2 \sin ^2(\frac{\pi }{m}) (3 \cos (\frac{2 \pi }{m})+5)}{3 \Big(k \sqrt{6 \cos (\frac{2 \pi }{m})+10}-\sqrt{6 \cos (\frac{2 \pi }{m})+10}+4\Big)^2}.\]
Thus, the proof the proposition is complete.
\end{proof}

Let us now prove the following theorem.

\begin{theorem}\label{theo01}
Let $P$ be the uniform distribution on the boundary of a regular $m$-sided polygon inscribed in a unit circle. Then, the quantization coefficient for $P$ exists as a finite positive number which equals $\frac{1}{3} m^2 \sin ^2(\frac{\pi }{m})$, i.e.,
$\lim\limits_{n\to \infty} n^2 V_n=\frac{1}{3} m^2 \sin ^2(\frac{\pi }{m}).$
\end{theorem}

\begin{proof} Let $n\in \D N$ be such that $n\geq 2m$. Then, there exists a unique positive integer $\ell(n)\geq 2$ such that $m\ell(n)\leq n<m(\ell(n)+1)$. Then,
\begin{equation} \label{eq46} (m\ell(n))^2V_{m(\ell(n)+1)}<n^2 V_n<(m(\ell(n)+1))^2V_{m\ell(n)}.
\end{equation}
We have
\begin{align*}
&\lim_{n \to \infty} (m\ell(n))^2V_{m(\ell(n)+1)}\\
&=\underset{\ell(n)\to \infty }{\text{lim}}(m \ell(n))^2 \frac{2 \sin ^2(\frac{\pi }{m}) (3 \cos (\frac{2 \pi }{m})+5)}{3 \Big((\ell(n)+1) \sqrt{6 \cos (\frac{2 \pi }{m})+10}-\sqrt{6 \cos (\frac{2 \pi }{m})+10}+4\Big)^2}=\frac{1}{3} m^2 \sin ^2(\frac{\pi }{m}),\end{align*}
and
\begin{align*}
&\lim_{n \to \infty} (m(\ell(n)+1))^2V_{m\ell(n)}\\
&=\underset{\ell(n)\to \infty }{\text{lim}}(m(\ell(n)+1))^2\frac{2 \sin ^2(\frac{\pi }{m}) (3 \cos (\frac{2 \pi }{m})+5)}{3 \Big(\ell(n) \sqrt{6 \cos (\frac{2 \pi }{m})+10}-\sqrt{6 \cos (\frac{2 \pi }{m})+10}+4\Big)^2}=\frac{1}{3} m^2 \sin ^2(\frac{\pi }{m}),
\end{align*}
and hence, by \eqref{eq46} we have
$\mathop{\lim}\limits_{n\to\infty} n^2 V_n=\frac{1}{3} m^2 \sin ^2(\frac{\pi }{m})$, i.e., the quantization coefficient exists as a finite positive number which equals $\frac{1}{3} m^2 \sin ^2(\frac{\pi }{m})$.
Thus, the proof of the theorem is complete.
\end{proof}

\begin{remark} Since  $\lim\limits_{m\to \infty}2 \sin \frac{\pi }{m}=0$, by \eqref{eq100}, we can conclude that when $m$ tends to $\infty$, then the length of each side of the regular $m$-sided polygon becomes zero, i.e., the regular $m$-sided polygon coincides with the circle. Moreover, for $m\geq 3$, we have
 \[\frac{d}{dm}(\frac{1}{3} m^2 \sin ^2(\frac{\pi }{m}))=\frac{2}{3} \sin (\frac{\pi }{m}) (m \sin (\frac{\pi }{m})-\pi  \cos (\frac{\pi }{m}))>0\]
yielding the fact that the quantization coefficient $\frac{1}{3} m^2 \sin ^2(\frac{\pi }{m})$ for the uniform distribution on the boundary of the regular $m$-sided polygon is an increasing function of $m$. Again,
 \[\lim_{m\to\infty}\frac{1}{3} m^2 \sin ^2(\frac{\pi }{m})=\frac{\pi ^2}{3},\]
 i.e., when $m$ tends to infinity, then the quantization coefficient of the regular $m$-sided polygon equals $\frac{\pi ^2}{3}$.
 Recall that $\frac{\pi ^2}{3}$ is the quantization coefficient for the uniform distribution on the unit circle. Thus, the result in this paper, proves the conjecture given by Pena et al. in the paper \cite{PRRSS}.
 \end{remark}

 \begin{remark}
 If $m=6$, we see that $\lim\limits_{n\to \infty} n^2V_n=3$, which is the quantization coefficient for the uniform distribution on the boundary of a hexagon inscribed in a unit circle. Thus, the result in this paper, also generalizes a result given by Pena et al. in the paper \cite{PRRSS}.
 \end{remark}

 We now end the paper with an open problem given in the following remark.

 \begin{remark}
 Quantization coefficients were also investigated for several singular continuous probability measures, but in each of the known cases, it is seen that the quantization coefficient for the singular continuous probability measure does not exist, see \cite{CR1, CR2, CR3, GL2, R3, R6}. It is still not known whether there is any singular continuous probability measure for which the quantization coefficient exists as a finite positive number.
 \end{remark}

\begin{thebibliography}{9999}


%\bibitem[AW]{AW} E.F. Abaya and G.L. Wise, \emph{Some remarks on the existence of optimal quantizers}, Statistics \& Probability Letters, Volume 2, Issue 6, December 1984, Pages 349-351.

    \bibitem[BW]{BW} J.A. Bucklew and G.L. Wise, \emph{Multidimensional asymptotic quantization theory with $r$th power distortion measures}, IEEE Transactions on Information Theory, 1982, Vol. 28, Issue 2, 239-247.


\bibitem[CR1]{CR1} D. \c C\"omez and M.K. Roychowdhury, \emph{Quantization for uniform distributions on stretched Sierpinski triangles}, Monatshefte f\"ur Mathematik, Volume 190, Issue 1,  79-100 (2019).


\bibitem[CR2]{CR2} D. \c C\"omez and M.K. Roychowdhury, \emph{Quantization for uniform distributions of Cantor dusts on $\mathbb{R}^2$}, Topology Proceedings, Volume 56 (2020), Pages 195-218.
    
    \bibitem[CR3]{CR3} D. \c C\"omez and M.K. Roychowdhury, \emph{Quantization for infinite affine transformations}, arXiv:1604.04261 [math.DS].


%\bibitem[DFG]{DFG} Q. Du, V. Faber and M. Gunzburger, \emph{Centroidal Voronoi Tessellations: Applications and Algorithms}, SIAM Review, Vol. 41, No. 4 (1999), pp. 637-676.

\bibitem[DR1]{DR1} C.P. Dettmann and M.K. Roychowdhury, \emph{Quantization for uniform distributions on equilateral triangles}, Real Analysis Exchange, Vol. 42(1), 2017, pp. 149-166.

\bibitem[DR2]{DR2} C.P. Dettmann and M.K. Roychowdhury, \emph{An algorithm to compute CVTs for finitely generated Cantor distributions}, to appear, Southeast Asian Bulletin of Mathematics.



\bibitem[GG]{GG} A. Gersho and R.M. Gray, \emph{Vector quantization and signal compression}, Kluwer Academy publishers: Boston, 1992.


%\bibitem[GKL]{GKL}  R.M. Gray, J.C. Kieffer and Y. Linde, \emph{Locally optimal block quantizer design}, Information and Control, 45 (1980), pp. 178-198.


\bibitem[GL1]{GL1} S. Graf and H. Luschgy, \emph{Foundations of quantization for probability distributions}, Lecture Notes in Mathematics 1730, Springer, Berlin, 2000.

%\bibitem[GL2]{GL2} A. Gy\"orgy and T. Linder, \emph{On the structure of optimal entropy-constrained scalar quantizers},  IEEE transactions on information theory, vol. 48, no. 2 (2002), pp. 416-427.

\bibitem[GL2]{GL2} S. Graf and H. Luschgy, \emph{The Quantization of the Cantor Distribution}, Math. Nachr., 183 (1997), 113-133.


    \bibitem[GN]{GN}  R. Gray and D. Neuhoff, \emph{Quantization}, IEEE Trans. Inform. Theory,  44 (1998), pp. 2325-2383.

%
%\bibitem[L]{L} L.J. Lindsay, \emph{Quantization dimension for probability distributions}, PhD dissertation, 2001, University of North Texas, Texas, USA.

\bibitem[L]{L} L. Roychowdhury, \emph{Optimal quantization for nonuniform Cantor distributions}, Journal of Interdisciplinary Mathematics, Vol 22 (2019), pp. 1325-1348.
%\bibitem[MKT]{MKT} S. Matsuura, H. Kurata and T. Tarpey, \emph{Optimal estimators of principal points for minimizing expected mean squared distance}, Journal of Statistical Planning and Inference, 167 (2015), 102-122.
\bibitem[P] {P} K. P\"otzelberger, \emph{The quantization dimension of distributions}, Math. Proc. Camb. Phil. Soc., 131, 507-519 (2001).


\bibitem[PRRSS]{PRRSS} G. Pena, H. Rodrigo, M.K. Roychowdhury, J. Sifuentes, and E. Suazo, \emph{Quantization for uniform distributions on hexagonal, semicircular, and elliptical curves}, arXiv:1902.03887 [math.PR], to appear, Journal of Optimization Theory and Applications.

\bibitem[R1]{R1} M.K. Roychowdhury, \emph{Quantization and centroidal Voronoi tessellations for probability measures on dyadic Cantor sets}, Journal of Fractal Geometry, 4 (2017), 127-146.


\bibitem[R2]{R2} M.K. Roychowdhury, \emph{Optimal quantizers for some absolutely continuous probability measures}, Real Analysis Exchange, Vol. 43(1), 2017, pp. 105-136.
\bibitem[R3]{R3} M.K. Roychowdhury, \emph{Optimal quantization for the Cantor distribution generated by infinite similitudes},  Israel Journal of Mathematics 231 (2019), 437-466.



\bibitem[R4]{R4} M.K. Roychowdhury, \emph{Least upper bound of the exact formula for optimal quantization of some uniform Cantor distributions}, Discrete and Continuous Dynamical Systems- Series A, Volume 38, Number 9, September 2018, pp. 4555-4570.

\bibitem[R5]{R5} M.K. Roychowdhury, \emph{Center of mass and the optimal quantizers for some continuous and discrete uniform distributions}, Journal of Interdisciplinary Mathematics, Vol. 22 (2019), No. 4,  pp. 451-471.

\bibitem[R6]{R6} M.K. Roychowdhury, \emph{The quantization of the standard triadic Cantor distribution}, to appear, Houston Journal of Mathematics.

\bibitem[RR1]{RR1} J. Rosenblatt and M.K. Roychowdhury, \emph{Optimal quantization for piecewise uniform distributions}, Uniform Distribution Theory 13 (2018), no. 2, 23-55.

\bibitem[RR2]{RR2} J. Rosenblatt and M.K. Roychowdhury, \emph{Uniform distributions on curves and quantization}, arXiv:1809.08364 [math.PR].


\bibitem[RS]{RS} M.K. Roychowdhury, and Wasiela Salinas, \emph{Quantization for a mixture of uniform distributions associated with probability vectors}, Uniform Distribution Theory 15 (2020), no. 1, 105-142.







\bibitem[Z]{Z} R. Zam, \emph{Lattice Coding for Signals and Networks: A Structured Coding Approach to Quantization, Modulation, and Multiuser Information Theory}, Cambridge University Press, 2014.



\end{thebibliography}

\end{document}

