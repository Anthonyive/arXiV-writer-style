      \documentclass[a4paper,11pt,reqno]{amsart}
      \usepackage{etex}
      \usepackage[OT2, T1]{fontenc}
      \usepackage{url}
      \usepackage{amsmath}
      \usepackage{lmodern}
      \usepackage{array}
      \usepackage{graphicx}
      \usepackage{amsfonts}
      \usepackage{amssymb}
      \usepackage{xcolor}
      \definecolor{imperialred}{RGB}{237, 41, 57}
      \definecolor{royalblue}{RGB}{64, 106, 212}
      \definecolor{link}{RGB}{11,0,128}
      \usepackage[colorlinks=true, citecolor=royalblue,linkcolor=imperialred,urlcolor=link]{hyperref}
      \usepackage{amsthm}
      \usepackage{esint}
      
      
      \newcommand{\temp}[1]{{\color{Gray}   [#1]}}
      \newcommand{\ning}[1]{{\color{royalblue} \sf  [#1]}}
      \newcommand{\revise}[1]{{\color{imperialred}   [#1]}}
      \newcommand{\ready}[1]{{\color{Brown}   [#1]}}
      \usepackage{exscale, relsize}
      \usepackage{stackengine}
      \usepackage{scalerel}
      \usepackage{setspace}
      \usepackage{paralist}
      \usepackage{colonequals}		% for nice :=
      \usepackage{mathabx}		% for \widec
      \usepackage[left]{lineno}
      \usepackage{amscd}
      \usepackage[all,cmtip]{xy}	% for commutative diagrams
      \usepackage{sseq}			% for spectral sequences
      \usepackage{verbatim}
      \usepackage{parskip}
      \usepackage{microtype}
      \usepackage[in]{fullpage}
      \usepackage{graphicx}
      \usepackage{mathrsfs} 			% for \mathscr (script letters)
      \newcommand\hmmax{0} % default 3
      \newcommand\bmmax{0} % default 4
      \usepackage{bm} 				% for bold Greek letters
      \usepackage{cleveref}
      \usepackage{enumitem}
      \usepackage{moreenum}			% for enumeration via Greek letters
      \usepackage{stmaryrd}
      \usepackage{blindtext}
      \usepackage{tikz-cd}
      \usepackage{tikz}
      \usetikzlibrary{bending,matrix,arrows,decorations.pathmorphing,backgrounds,positioning,fit,petri,arrows.meta}
      \usepackage{textcomp}
      \usepackage{indentfirst}
      \usetikzlibrary{arrows}
      \usepackage{enumitem}
      \tikzset{commutative diagrams/.cd,arrow style=tikz,diagrams={>=latex'}}
      \usepackage{filecontents}
      \usepackage[alphabetic,lite]{amsrefs} 	% for bibliography
      \usepackage{stmaryrd}	% this is for \llbracket and \rrbracket
      \usepackage[foot]{amsaddr}
      \usepackage{subfiles}
      \usepackage{ragged2e}
      
      % This is for Sha
      % \DeclareSymbolFont{cyrletters}{OT2}{wncyr}{m}{n}
      % \DeclareMathSymbol{\Sha}{\mathalpha}{cyrletters}58}
      
      
      % -----------------------Colors
         \newcommand{\coR}{\textcolor{red}}
         \newcommand{\coG}{\textcolor{green}}
         \newcommand{\coB}{\textcolor{blue}}
      
      % -----------------------Greek letters
         \newcommand{\gA}{\alpha}
         \newcommand{\gB}{\beta}
         \newcommand{\gG}{\gamma}
         \newcommand{\gD}{\delta}
         \newcommand{\gE}{\epsilon}
         \newcommand{\gZ}{\zeta}
         \newcommand{\gI}{\iota}
         \newcommand{\gK}{\kappa}
         \newcommand{\gL}{\lambda}
         \newcommand{\gM}{\mu}
         \newcommand{\gN}{\nu}
         \newcommand{\gX}{\xi}
         \newcommand{\gC}{\chi}
         \newcommand{\gO}{\omega}
         \newcommand{\gP}{\pi}
         \newcommand{\gF}{\phi}
         \newcommand{\gT}{\tau}
         \newcommand{\GG}{\Gamma}
         \newcommand{\GD}{\Delta}
         \newcommand{\GGL}{\Lambda}
         \newcommand{\GO}{\Omega}
         \newcommand{\GP}{\Pi}
         \newcommand{\GF}{\Phi}
      
      % -----------------------Blackboard letters
         \newcommand{\bA}{\mathbb{A}}
         \newcommand{\bC}{\mathbb{C}}
         \newcommand{\bD}{\mathbb{D}}
         \newcommand{\bE}{\mathbb{E}}
         \newcommand{\bF}{\mathbb{F}}
         \newcommand{\bG}{\mathbb{G}}
         \newcommand{\bH}{\mathbb{H}}
         \newcommand{\bL}{\mathbb{L}}
         \newcommand{\bN}{\mathbb{N}}
         \newcommand{\bP}{\mathbb{P}}
         \newcommand{\bQ}{\mathbb{Q}}
         \newcommand{\bR}{\mathbb{R}}
         \newcommand{\bS}{\mathbb{S}}
         \newcommand{\bT}{\mathbb{T}}
         \newcommand{\bU}{\mathbb{U}}
         \newcommand{\bV}{\mathbb{V}}
         \newcommand{\bZ}{\mathbb{Z}}
      
      %%-----------------------Bold letters
         \newcommand{\bbA}{\mathbf{A}}
         \newcommand{\bbB}{\mathbf{B}}
         \newcommand{\bbC}{\mathbf{C}}
         \newcommand{\bbD}{\mathbf{D}}
         \newcommand{\bbE}{\mathbf{E}}
         \newcommand{\bbF}{\mathbf{F}}
         \newcommand{\bbG}{\mathbf{G}}
         \newcommand{\bbH}{\mathbf{H}}
         \newcommand{\bbK}{\mathbf{K}}
         \newcommand{\bbL}{\mathbf{L}}
         \newcommand{\bbN}{\mathbf{N}}
         \newcommand{\bbP}{\mathbf{P}}
         \newcommand{\bbQ}{\mathbf{Q}}
         \newcommand{\bbR}{\mathbf{R}}
         \newcommand{\bbS}{\mathbf{S}}
         \newcommand{\bbT}{\mathbf{T}}
         \newcommand{\bbU}{\mathbf{U}}
         \newcommand{\bbX}{\mathbf{X}}
         \newcommand{\bbY}{\mathbf{Y}}
         \newcommand{\bbZ}{\mathbf{Z}}
      
      %-----------------------Calligraphic letters
         \newcommand{\cA}{\mathcal{A}}
         \newcommand{\cB}{\mathcal{B}}
         \newcommand{\cC}{\mathcal{C}}
         \newcommand{\cD}{\mathcal{D}}
         \newcommand{\cE}{\mathcal{E}}
         \newcommand{\cF}{\mathcal{F}}
         \newcommand{\cG}{\mathcal{G}}
         \newcommand{\cH}{\mathcal{H}}
         \newcommand{\cI}{\mathcal{I}}
         \newcommand{\cJ}{\mathcal{J}}
         \newcommand{\cK}{\mathcal{K}}
         \newcommand{\cL}{\mathcal{L}}
         \newcommand{\cM}{\mathcal{M}}
         \newcommand{\cN}{\mathcal{N}}
         \newcommand{\cO}{\mathcal{O}}
         \newcommand{\cP}{\mathcal{P}}
         \newcommand{\cQ}{\mathcal{Q}}
         \newcommand{\cR}{\mathcal{R}}
         \newcommand{\cS}{\mathcal{S}}
         \newcommand{\cT}{\mathcal{T}}
         \newcommand{\cU}{\mathcal{U}}
         \newcommand{\cV}{\mathcal{V}}
         \newcommand{\cW}{\mathcal{W}}
         \newcommand{\cX}{\mathcal{X}}
         \newcommand{\cY}{\mathcal{Y}}
         \newcommand{\cZ}{\mathcal{Z}}
      
      %%-----------------------Fraktur letters
         \newcommand{\fa}{\mathfrak{a}}
         \newcommand{\fb}{\mathfrak{b}}
         \newcommand{\fc}{\mathfrak{c}}
         \newcommand{\fd}{\mathfrak{d}}
         \newcommand{\ff}{\mathfrak{f}}
         \newcommand{\fg}{\mathfrak{g}}
         \newcommand{\fh}{\mathfrak{h}}
         \newcommand{\fm}{\mathfrak{m}}
         \newcommand{\fo}{\mathfrak{o}}
         \newcommand{\fp}{\mathfrak{p}}
         \newcommand{\fq}{\mathfrak{q}}
         \newcommand{\fr}{\mathfrak{r}}
         \newcommand{\ft}{\mathfrak{t}}
         \newcommand{\fA}{\mathfrak{A}}
         \newcommand{\fB}{\mathfrak{B}}
         \newcommand{\fC}{\mathfrak{C}}
         \newcommand{\fD}{\mathfrak{D}}
         \newcommand{\fE}{\mathfrak{E}}
         \newcommand{\fF}{\mathfrak{F}}
         \newcommand{\fG}{\mathfrak{G}}
         \newcommand{\fH}{\mathfrak{H}}
         \newcommand{\fI}{\mathfrak{I}}
         \newcommand{\fK}{\mathfrak{K}}
         \newcommand{\fL}{\mathfrak{L}}
         \newcommand{\fM}{\mathfrak{M}}
         \newcommand{\fN}{\mathfrak{N}}
         \newcommand{\fO}{\mathfrak{O}}
         \newcommand{\fP}{\mathfrak{P}}
         \newcommand{\fQ}{\mathfrak{Q}}
         \newcommand{\fR}{\mathfrak{R}}
         \newcommand{\fS}{\mathfrak{S}}
         \newcommand{\fT}{\mathfrak{T}}
         \newcommand{\fU}{\mathfrak{U}}
         \newcommand{\fW}{\mathfrak{W}}
         \newcommand{\fX}{\mathfrak{X}}
      
      %%-----------------------Script letters
         \newcommand{\sA}{\mathscr{A}}
         \newcommand{\sB}{\mathscr{B}}
         \newcommand{\sC}{\mathscr{C}}
         \newcommand{\sD}{\mathscr{D}}
         \newcommand{\sE}{\mathscr{E}}
         \newcommand{\sF}{\mathscr{F}}
         \newcommand{\sG}{\mathscr{G}}
         \newcommand{\sH}{\mathscr{H}}
         \newcommand{\sI}{\mathscr{I}}
         \newcommand{\sK}{\mathscr{K}}
         \newcommand{\sL}{\mathscr{L}}
         \newcommand{\sM}{\mathscr{M}}
         \newcommand{\sN}{\mathscr{N}}
         \newcommand{\sO}{\mathscr{O}}
         \newcommand{\sP}{\mathscr{P}}
         \newcommand{\sQ}{\mathscr{Q}}
         \newcommand{\sR}{\mathscr{R}}
         \newcommand{\sS}{\mathscr{S}}
         \newcommand{\sT}{\mathscr{T}}
         \newcommand{\sU}{\mathscr{U}}
         \newcommand{\sV}{\mathscr{V}}
         \newcommand{\sW}{\mathscr{W}}
         \newcommand{\sX}{\mathscr{X}}
         \newcommand{\sY}{\mathscr{Y}}
         \newcommand{\sZ}{\mathscr{Z}}
      
      %%-----------------------Math shortcuts
         \newcommand{\ra}{\rightarrow}
         \newcommand{\la}{\leftarrow}
         \newcommand{\Ra}{\Rightarrow}
         \newcommand{\La}{\Leftarrow}
         \newcommand{\Lra}{\Leftrightarrow}
         \newcommand{\xra}{\xrightarrow}
         \newcommand{\xla}{\xleftarrow}
         \newcommand{\lra}{\longrightarrow}
         \newcommand{\lla}{\longleftarrow}
         \newcommand{\hra}{\hookrightarrow}
         \newcommand{\hla}{\hookleftarrow}
         \newcommand{\I}{^{\infty}}
         \newcommand{\wt}{\widetilde}
         \newcommand{\wh}{\widehat}
         \newcommand{\eps}{\epsilon}
         \newcommand{\del}{\partial}
         \newcommand{\pr}{^{\prime}}
         \newcommand{\prpr}{^{\prime\prime}}
         \newcommand{\op}{^{\mathrm{op}}}		 
         \newcommand{\sep}{^{\mathrm{sep}}}
      
         \newcommand{\ce}{\colonequals}
         \newcommand{\ov}{\overline}
         \newcommand{\un}{\underline}
         \newcommand{\sm}{\mathrm{sm}}
         \newcommand{\sing}{\mathrm{sing}}
         \renewcommand{\b}{\textbf}
         \newcommand{\surjects}{\twoheadrightarrow}
         \newcommand{\injects}{\hookrightarrow}
         \newcommand{\isom}{\simeq}			                                               % Isomorphic*/
         \newcommand{\Qlbar}{\ov{\bQ}_\ell}		                                           % Algebraic closure of l-adic numbers*/
         \newcommand{\notdiv}{\nmid}
         \newcommand{\Intersection}{\bigcap}                                                 % intersection of a collection*/
         \newcommand{\intersect}{\cap} 		                                               % binary intersection*/
         \newcommand{\Union}{\bigcup} 		                                               % union of a collection*/
         \newcommand{\union}{\cup} 			                                               % binary union*/
         \newcommand{\tensor}{\otimes} 		                                               % binary tensor product*/
         \newcommand{\Tensor}{\bigotimes} 	                                               % tensor product of a collection*/
         \newcommand{\directsum}{\oplus} 		                                               % binary direct sum*/
         \newcommand{\Directsum}{\bigoplus}                                                 	% direct sum of a collection*/
         \newcommand{\isoto}{\overset{\sim}{\longrightarrow}}
         \newcommand{\isomfrom}{\overset{\sim}{\longleftarrow}}
         \newcommand{\heart}{^{\heartsuit}}		                                           % heart of a t-structure*/
         \newcommand{\nr}{{\mathrm{nr}}}		                                               % unramified cohomology or max unramified extension*/
         \newcommand{\nd}{{\mathrm{nd}}}		                                               % the maximal nondivisible quotient of an abelian group*/
         \newcommand{\gp}{{\mathrm{gp}}}		                                               % group completion of a monoid*/
         \newcommand{\normal}{{\lhd}}			                                               % normal subgroup*/
         \newcommand{\cont}{{\mathrm{cont}}}                                                 % continuous (mainly used in subscripts)*/
         \newcommand{\cris}{{\mathrm{cris}}}                                                 % crystalline (mainly used in subscripts)*/
         \newcommand{\dR}{{\mathrm{dR}}}		                                               % de Rham (mainly used in subscripts)*/
         \newcommand{\st}{{\mathrm{st}}}		                                               % semistable (mainly used in subscripts)*/
         \newcommand{\pst}{{\mathrm{pst}}}		                                           % potentially semistable (mainly used in subscripts)*/
         \newcommand{\can}{{\mathrm{can}}}		                                           % canonical (mainly used in subscripts)*/
         \newcommand{\bb}{\boldsymbol}		                                               % for bold symbol in math mode (esp. greek letters)*/
         \newcommand{\cycl}{{\mathrm{cycl}}}		                                           % cyclotomic (mainly used in subscripts)*/
         \newcommand{\fppf}{\mathrm{fppf}}		                                           % for fppf cohomology (mainly used in subscripts)*/
         \newcommand{\fpqf}{\mathrm{fpqf}}		                                           % flat quasi-finite site (mainly used in subscripts)*/
         \newcommand{\et}{\mathrm{\acute{e}t}}	                                           % for etale cohomology (mainly used in subscripts)*/
         \newcommand{\Zar}{\mathrm{Zar}}		                                               % for Zariski cohomology (mainly used in subscripts)*/
         \newcommand{\red}{{\mathrm{red}}}		                                           % cyclotomic (mainly used in subscripts)*/
         \newcommand{\alg}{\mathrm{alg}}		                                               % algebraic closure (mainly used in superscripts)*/
         \newcommand{\llb}{\llbracket}		                                               % [[*/
         \newcommand{\rrb}{\rrbracket}		                                               % ]]*/
         \newcommand{\lb}{[}		                                                           % [ for the bpp hack*/
         \newcommand{\rb}{]}		                                                           % ] for the bpp hack*/
         \newcommand{\tors}{\mathrm{tors}}		                                           % torsion subgroup (mainly used in subscripts)*/
         \newcommand{\sh}{\mathrm{sh}}		                                              % strict henselization (mainly used in superscripts)*/
         \newcommand{\psh}{\mathrm{psh}}		                                          % presheaf*/
         \newcommand{\shv}{\mathrm{shv}}		                                          % sheaf*/
         \newcommand{\adjoint}{\dashv}		                                              % LHS is left adjoint to RHS*/
         \DeclareMathOperator{\gr}{gr}			                                              % associated graded*/
         \newcommand{\Cong}{\mathrm{cong}}		                                              % for congruence numbers*/
         \newcommand{\FEt}{\textbf{F\'Et}}
         \newcommand{\longeq}{{=\joinrel=\joinrel=}}
      
         \newcommand{\hva}{\!\stackon[-8pt]{\;$V^{\raisebox{1pt}{$\mathsmaller{\mathsmaller{a}}$}}$}{\vstretch{1.5}{\hstretch{1.7}{\widehat{\phantom{}}}}}}
         \newcommand{\hRa}{\!\stackon[-8pt]{\;$R^{\raisebox{1pt}{$\mathsmaller{\mathsmaller{a}}$}}$}{\vstretch{1.5}{\hstretch{1.7}{\widehat{\phantom{}}}}}}
         \newcommand{\hka}{\!\stackon[-8pt]{\;$K^{\raisebox{1pt}{$\mathsmaller{\mathsmaller{a}}$}}$}{\vstretch{1.5}{\hstretch{1.7}{\widehat{\phantom{}}}}}}
         \newcommand{\hvai}{\!\!\stackon[-8pt]{\;$V^{\raisebox{1pt}{$\mathsmaller{\mathsmaller{a}}$}}$}{\vstretch{1.5}{\hstretch{1.7}{\widehat{\phantom{}}}}} {[\f{1}{a}]}}
         \newcommand{\via}{V{[\f{1}{a}]}}
         \newcommand{\nvia}{V\mathsmaller{[\f{1}{a}]}}
         \newcommand{\nhvai}{\wh{V}^a[\f{1}{a}]}
         \newcommand{\nhka}{\wh{K}^a}
         \newcommand{\nhva}{\wh{V}^a}
         \newcommand{\etp}{\pi_1^{\mathrm{\acute{e}t}}}
         \renewcommand{\th}{^{\mathrm{th}}}
         \renewcommand{\implies}{\Rightarrow}
         \renewcommand{\div}{{\mathrm{div}}}	                                               % maximal divisible subgroup (mainly used in subscripts)*/
         \newcommand{\Div}{{\mathrm{Div}}}	                                               % Cartier divisor associated to a generically exact complex*/
         \newcommand{\PD}{{\mathrm{PD}}}	                                                   % Divided powers*/
         \newcommand{\mt}{\un{\mathrm{Tor}}}
         \newcommand{\normlizer}{\un{\mathrm{Norm}}}
         \newcommand{\mtg}{\un{\mathrm{Tor}}(G)}
         \newcommand{\leftexp}[2]{{\vphantom{#2}}^{#1}{#2}}
         \newcommand{\leftsub}[2]{{\vphantom{#2}}_{#1}{#2}}
         \providecommand{\abs}[1]{\left\lvert#1\right\rvert}
         \providecommand{\norm}[1]{\lVert#1\rVert}
         \providecommand{\In}[1]{\left\langle#1\right\rangle}
         \providecommand{\p}[1]{\left(#1\right)}
         \providecommand{\SP}[1]{\cite[\href{https://stacks.math.columbia.edu/tag/#1}{#1}]{SP}}
         \providecommand{\hkh}[1]{\left\{#1\right\}}
         \providecommand{\up}[1]{{\upshape(}#1{\upshape)}}
         \providecommand{\uref}[1]{{\upshape\ref{#1}}}
         \providecommand{\uS}{{\upshape\S}}
         \providecommand{\ucolon}{{\upshape:}}
         \providecommand{\uscolon}{{\upshape;} }
         \providecommand{\sqp}[1]{\left[#1\right]}
         \providecommand{\f}[2]{\frac{#1}{#2}}
         
         \DeclareMathOperator{\dist}{dist}		                       % Distance*/
         \DeclareMathOperator{\Ker}{Ker}			                       % Kernel*/
         \DeclareMathOperator{\coker}{coker}		                       % cokernel*/
         \DeclareMathOperator{\Coker}{Coker}		                       % Cokernel*/
         \DeclareMathOperator{\im}{Im}			                       % Imaginary part*/
         \DeclareMathOperator{\Spec}{Spec}		                       % Spectrum of a ring*/
         \DeclareMathOperator{\Spf}{Spf}		                           % Formal spectrum of an adic ring*/
         \DeclareMathOperator{\Spa}{Spa}		                           % Adic spectrum of an f-adic ring*/
         \DeclareMathOperator{\Proj}{Proj}		                       % Proj of a graded ring*/
         \DeclareMathOperator{\rad}{rad}			                       % Radical of an ideal ?*/
         \DeclareMathOperator{\Hom}{Hom}			                       % Set of arrows between two object*/
         \DeclareMathOperator{\Gr}{Gr}			                       % Associated graded ?*/
         \DeclareMathOperator{\Ann}{Ann}			                       % Annihilator of a module*/
         \DeclareMathOperator{\Fitt}{Fitt}		                       	% Fitting ideal*/
         \DeclareMathOperator{\Ass}{Ass}			                       % Associated primes*/
         \DeclareMathOperator{\Span}{span}		                       % Span*/
         \DeclareMathOperator{\Char}{char}		                       % Characteristic of a field*/
         \DeclareMathOperator{\Frac}{Frac}		                       % Field of fractions*/
         \DeclareMathOperator{\depth}{depth}		                       % Depth of a module*/
         \DeclareMathOperator{\trdeg}{tr.deg}		                       % Transcendence degree*/
         \DeclareMathOperator{\supp}{supp}		                       % support of a function*/
         \DeclareMathOperator{\Supp}{Supp}		                       % Support of a function*/
         \DeclareMathOperator{\Ht}{ht}			                       % Height of an ideal*/
         \DeclareMathOperator{\PGL}{PGL}			                       % Projective general linear group*/
         \DeclareMathOperator{\SL}{SL}			                       % Special linear group*/
         \DeclareMathOperator{\re}{Re}			                       % Real part*/
         \DeclareMathOperator{\cyl}{cyl}			                       % Mapping cylinder*/
         \DeclareMathOperator{\Nat}{Nat}			                       % Natural transformations*/
         \DeclareMathOperator{\Map}{Map}			                       % Mapping space*/
         \DeclareMathOperator{\id}{id}			                       % identity*/
         \DeclareMathOperator{\Id}{Id}			                           % Identity*/
         \DeclareMathOperator{\Ob}{Ob}			                           % Objects of a category*/
         \DeclareMathOperator{\const}{const}	                           	% Constant*/
         \DeclareMathOperator{\rel}{rel}		                           	% Homotopy rel something*/
         \DeclareMathOperator{\Cone}{Cone}		                           % Unreduced cone*/
         \DeclareMathOperator{\ev}{ev}			                           % Evaluation at a point map*/
         \DeclareMathOperator{\Susp}{Susp}		                           % Unreduced suspension*/
         \DeclareMathOperator{\Var}{Var}		                           	% Variance or the category of varieties*/
         \DeclareMathOperator{\Ext}{Ext}		                           	% Derived functors of Hom*/
         \DeclareMathOperator{\Extrig}{Extrig}	                           		% Rigidified extensions*/
         \DeclareMathOperator{\Tor}{Tor}		                           	% Derived functors of tensor product*/
         \DeclareMathOperator{\pt}{pt}			                           % One point space*/
         \DeclareMathOperator{\Tube}{Tube}		                                                  % Turbular neighborhood*/
         \DeclareMathOperator{\Sq}{Sq}			                                                  % Steenrod squares*/
         \DeclareMathOperator{\Sets}{\textbf{Set}}
         \DeclareMathOperator{\Sch}{\textbf{Sch}}		                                                  % Category of sets*/
         \DeclareMathOperator{\colim}{colim}		                                                  % colimit*/
         \DeclareMathOperator{\Ab}{Ab}		                                                  % Category of abelian groups*/
         \DeclareMathOperator{\Sh}{Sh}		                                                  % Category of sheaves*/
         \DeclareMathOperator{\Mod}{Mod}		                                                  % Category of modules*/
         \DeclareMathOperator{\Top}{Top}		                                                  % Category of topological spaces*/
         \DeclareMathOperator{\loc}{loc}		                                                  %*/
         \DeclareMathOperator{\obs}{obs}		                                                  % Obstruction*/
         \DeclareMathOperator{\Br}{Br}		                                                  % Brauer group*/
         \DeclareMathOperator{\Pro}{Pro}		                                                  % Prolongation to a pro category*/
         \DeclareMathOperator{\Ex}{Ex}		                                                  % Right adjoint to subdivision functor*/
         \DeclareMathOperator{\Gal}{Gal}	                                                  % Galois group*/
         \DeclareMathOperator{\Perv}{Perv}	                                                  % Perverse sheaves*/
         \DeclareMathOperator{\tr}{tr}		                                                  % Truncation of a simplicial object or a trace*/
         \DeclareMathOperator{\Tr}{Tr}		                                                  % Trace*/
         \DeclareMathOperator{\Trace}{Trace}		                                                  % More trace*/
         \DeclareMathOperator{\Norm}{Norm}		                                                  % Norm*/
         \DeclareMathOperator{\Ran}{Ran}		                                                  % Right Kan extension or Ran space*/
         \DeclareMathOperator{\Lan}{Lan}		                                                  % Left Kan extension*/
         \DeclareMathOperator{\cosk}{cosk}		                                                  % coskeleton of a truncated simplicial object in a complete category*/
         \DeclareMathOperator{\sk}{sk}			                                                  % skeleton of a truncated simplicial object in a cocomplete category*/
         \DeclareMathOperator{\Et}{\acute{E}t}	                                                  % For etale site or etale topos or etale homotopy type*/
         \DeclareMathOperator{\HR}{HR}	                                                  % The category of hypercoverings*/
         \DeclareMathOperator{\Ho}{Ho}	                                                  % Homotopy category of a model category*/
         \DeclareMathOperator{\inv}{inv}	                                                  % The invariant map from local class field theory*/
         \DeclareMathOperator{\desc}{desc}	                                                  % The descent obstruction*/
         \DeclareMathOperator{\fin}{fin}		                                                  % The descent obstruction when one restricts to torsors under finite linear alg. groups*/
         \DeclareMathOperator{\finab}{fin-ab}	                                                  % The descent obstruction when one restricts to torsors under finite abelian groups*/
         \DeclareMathOperator{\con}{con}	                                                  % The descent obstruction when one restricts to torsors under connected lin. alg. groups*/
         \DeclareMathOperator{\Sym}{Sym}			                                                  % symmetric power*/
         \DeclareMathOperator{\ord}{ord}	                                                  % order*/
         \DeclareMathOperator{\ver}{ver}	                                                  % verlagerung = transfer*/
         \DeclareMathOperator{\Ver}{Ver}	                                                  % Verlagerung = transfer*/
         \DeclareMathOperator{\ab}{ab}		                                                  % abelianization*/
         \DeclareMathOperator{\Ind}{Ind}		                                                  % Induced representation*/
         \DeclareMathOperator{\Res}{Res}		                                                  % Restriction of the representation*/
         \DeclareMathOperator{\GL}{GL}		                                                  % The general linear group*/
         \DeclareMathOperator{\End}{End}		                                                  % The algebra of endomorphisms*/
         \DeclareMathOperator{\Aut}{Aut}		                                                  % The group of automorphisms*/
         \DeclareMathOperator{\Iso}{Iso}		                                                  % The isomorphism functor*/
         \DeclareMathOperator{\rk}{rk}		                                                  % rank*/
         \DeclareMathOperator{\cork}{cork}		                                                  % corank*/
         \DeclareMathOperator{\Sel}{Sel}		                                                  % Selmer group*/
         \DeclareMathOperator{\Art}{Art}		                                                  % local Artin homomorphism*/
         \DeclareMathOperator{\rec}{rec}		                                                  % reciprocity homomorphism*/
         \DeclareMathOperator{\Lie}{Lie}		                                                  % Lie algebra*/
         \DeclareMathOperator{\length}{length}		                                                  % length of a module*/
         \DeclareMathOperator{\Crys}{Crys}		                                                  % The functor ( -- \tensor B_\crys )^{G_K}*/
         \DeclareMathOperator{\DR}{DR}		                                                  % The functor ( -- \tensor B_\dR )^{G_K}*/
         \DeclareMathOperator{\Irr}{Irr}		                                                  % The set of isomorphism classes of irreducible representations*/
         \DeclareMathOperator{\Mat}{Mat}		                                                  % Matrices*/
         \DeclareMathOperator{\KS}{\mathbf{KS}}		                                                  % Kolyvagin systems*/
         \DeclareMathOperator{\Pic}{Pic}		                                                  % Picard group*/
         \DeclareMathOperator{\Cl}{Cl}		                                                  % Class group*/
         \DeclareMathOperator{\codim}{codim}		                                                  % codimension*/
         \DeclareMathOperator{\Sw}{Sw}		                                                  % Swan conductor*/
         \DeclareMathOperator{\Frob}{Frob}		                                                  % Frobenius*/
         \DeclareMathOperator{\lcm}{lcm}		                                                  % least common multiple*/
         \DeclareMathOperator{\NS}{NS}		                                             % Neron--Severi group*/
         \DeclareMathOperator{\Isom}{Isom}	                                             	% Isomorphism scheme*/
         \DeclareMathOperator{\SEC}{SEC}		                                             % Stack of transverse multisections*/
         \DeclareMathOperator{\ET}{ET}		                                             % Stack of transverse multisections up to permutation*/
         \DeclareMathOperator{\Jac}{Jac}		                                             % Jacobian*/
         \DeclareMathOperator{\Fil}{Fil}		                                             	% filtration*/
         \DeclareMathOperator{\Cont}{Cont}	                                             		% space of continuous valuations*/
         \newcommand{\ba}{\begin{aligned}}
         \newcommand{\ea}{\end{aligned}}
         \newcommand{\be}{\begin{equation}}
         \newcommand{\ee}{\end{equation}}
         \newcommand{\pf}{\begin{proof}}
         \newcommand{\bpf}{\begin{proof}}
         \newcommand{\epf}{\end{proof}}
         \newcommand{\bthm}{\begin{thm}}
         \newcommand{\ethm}{\end{thm}}
         \newcommand{\bthmt}{\begin{thm-tweak}}
         \newcommand{\ethmt}{\end{thm-tweak}}
         \newcommand{\bprop}{\begin{prop}}
         \newcommand{\eprop}{\end{prop}}
         \newcommand{\bcor}{\begin{cor}}
         \newcommand{\ecor}{\end{cor}}
         \newcommand{\bcort}{\begin{cor-tweak}}
         \newcommand{\ecort}{\end{cor-tweak}}
         \newcommand{\brem}{\begin{rem}}
         \newcommand{\erem}{\end{rem}}
         \newcommand{\bremt}{\begin{rem-tweak}}
         \newcommand{\eremt}{\end{rem-tweak}}
         \newcommand{\brems}{\begin{rems} \hfill \begin{enumerate}[label=\b{\thenumberingbase.},ref=\thenumberingbase]}
         \newcommand{\remi}{\addtocounter{numberingbase}{1} \item}
         \newcommand{\erems}{\end{enumerate} \end{rems}}
         \newcommand{\begs}{\begin{egs} \hfill \begin{enumerate}[label=\b{\thenumberingbase.},ref=\thenumberingbase]}
         \newcommand{\egi}{\addtocounter{numberingbase}{1} \item}
         \newcommand{\eegs}{\end{enumerate} \end{egs}}
         \newcommand{\eremstweak}{\end{enumerate} \end{rems-tweak}}
         \newcommand{\eremst}{\end{enumerate} \end{rems-tweak}}
         \newcommand{\blem}{\begin{lemma}}
         \newcommand{\elem}{\end{lemma}}
         \newcommand{\blemt}{\begin{lemma-tweak}}
         \newcommand{\elemt}{\end{lemma-tweak}}
         \newcommand{\bconj}{\begin{conj}}
         \newcommand{\econj}{\end{conj}}
         \newcommand{\bprob}{\begin{Problem}}
         \newcommand{\eprob}{\end{Problem}}
         \newcommand{\bpropt}{\begin{prop-tweak}}
         \newcommand{\epropt}{\end{prop-tweak}}
         \newcommand{\bq}{\begin{Q}}
         \newcommand{\eq}{\end{Q}}
         \newcommand{\benum}{\begin{enumerate}[label={{\upshape(\alph*)}}]}
         \newcommand{\benuma}{\begin{enumerate}[label={{\upshape(\arabic*)}}]}
         \newcommand{\benumr}{\begin{enumerate}[label={{\upshape(\roman*)}}]}
         \newcommand{\eenum}{\end{enumerate}}
         \newcommand{\bc}{\begin{comment}}
         \newcommand{\ec}{\end{comment}}
         \newcommand{\bd}{\begin{defn}}
         \newcommand{\ed}{\end{defn}}
         \newcommand{\bdt}{\begin{defn-tweak}}
         \newcommand{\edt}{\end{defn-tweak}}
         \newcommand{\beg}{\begin{eg}}
         \newcommand{\eeg}{\end{eg}}
         \newcommand{\begt}{\begin{eg-tweak}}
         \newcommand{\eegt}{\end{eg-tweak}}
         \newcommand{\bcl}{\begin{claim}}
         \newcommand{\ecl}{\end{claim}}
         \newcommand{\lab}{\label}
         \newcommand{\ssk}{\smallskip}
         \newcommand{\msk}{\medskip}
         \newcommand{\bsk}{\bigskip}
         \newcommand{\x}{\text}
         \newcommand{\rv}{\revise}
         \newcommand{\rd}{\ready}
         \newcommand{\q}{\quad}
         \newcommand{\qq}{\quad\quad}
         \newcommand{\qqq}{\quad\quad\quad}
         \newcommand{\qqqq}{\quad\quad\quad\quad}
         \newcommand{\qqqqq}{\quad\quad\quad\quad\quad}
         \newcommand{\qqqqqq}{\quad\quad\quad\quad\quad\quad}
         \newcommand{\qqqqqqq}{\quad\quad\quad\quad\quad\quad\quad}
         \newcommand{\qqqqqqqq}{\quad\quad\quad\quad\quad\quad\quad\quad}
         \newcommand{\qqqqqqqqq}{\quad\quad\quad\quad\quad\quad\quad\quad\quad}
         \newcommand{\tst}{\textstyle}
         \newcommand{\Ell}{\cE\ell\ell}
         \newcommand{\EEl}{\overline{\cE\ell\ell}}
         \newcommand{\bal}{\mathrm{bal}}
         \newcommand{\naive}{\mathrm{naive}}
         \newcommand{\perf}{\mathrm{perf}}			                                                                               % perfection*/
         \newcommand{\Tate}{\underline{\mathrm{Tate}}}
         \newcommand{\KM}{\mathrm{KM}}
         \newcommand{\Ner}{\mathrm{Ner}}
         \newcommand{\val}{\mathrm{val}}
         \newcommand{\Inf}{\mathrm{inf}}
         \newcommand{\new}{\mathrm{new}}			                                                                                   % new quotient of a modular Jacobian*/
         \newcommand{\old}{\mathrm{old}}			                                                                                   % old quotient of a modular Jacobian*/
         \newcommand{\ad}{\mathrm{ad}}			                                                                                   % associated adic space*/
         \newcommand{\proet}{\mathrm{pro\acute{e}t}}			                                                                   % associated adic space*/
         \newcommand{\cusps}{\mathrm{cusps}}
         \newcommand{\rr}{\raggedright}
         \newcommand{\bconjt}{\begin{conj-tweak}}
         \newcommand{\econjt}{\end{conj-tweak}}
         \newcommand{\forg}{\mathrm{forg}}
         \newcommand{\quot}{\mathrm{quot}}
         \newcommand{\normalizer}{\underline{\mathrm{Norm}}}
         \newcommand{\centr}{\underline{\mathrm{Centr}}}
         \newcommand{\urad}{\mathrm{rad}^u}
         \newcommand{\Par}{\un{\mathrm{Par}}}
         \renewcommand{\Isom}{\un{\mathrm{Isom}}}
         \newcommand{\reg}{^{\mathrm{reg}}} 
      
        % -----------------------Theorems and numbering
        % \newcommand*{\QED}{\hfill\ensuremath{\qed}}
        % \newcommand*{\QEDD}{\hfill\ensuremath{\qed\qed}
      
      
      \usepackage{aliascnt}
      \newaliascnt{numberingbase}{subsection}
      
      \theoremstyle{plain}
      \newtheorem{thm}[numberingbase]{Theorem}
      \Crefname{thm}{Theorem}{Theorems}
      \newtheorem{rethm}{Theorem}
      \Crefname{rethm}{Theorem}{Theorem}
      \newtheorem{prop}[numberingbase]{Proposition}
      \Crefname{prop}{Proposition}{Propositions}
      \newtheorem{Q}[numberingbase]{Question}
      \Crefname{Q}{Question}{Questions}
      \newtheorem{Problem}[subsection]{Problem}
      \Crefname{Problem}{Problem}{Problems}
      \newtheorem{conj}[numberingbase]{Conjecture}
      \Crefname{conj}{Conjecture}{Conjectures}
      \newtheorem{cor}[numberingbase]{Corollary}
      \Crefname{cor}{Corollary}{Corollaries}
      \newtheorem{lemma}[numberingbase]{Lemma}
      
      \newtheorem{subegs}[equation]{Examples}
      \newtheorem{subprop}[equation]{Proposition}
      \Crefname{subprop}{Proposition}{Propositions}
      \newtheorem{subcor}[equation]{Corollary}
      \Crefname{subcor}{Corollary}{Corollaries}
      \newtheorem{sublem}[equation]{Lemma}
      \Crefname{sublem}{Lemma}{Lemmas}
      
      \theoremstyle{remark}
      \newtheorem{claim}[equation]{Claim}
      \Crefname{claim}{Claim}{Claims}
      \newtheorem{subrem}[equation]{Remark}
      \Crefname{subrem}{Remark}{Remarks}
      
      \theoremstyle{definition}
      \newtheorem{defn}[numberingbase]{Definition}
      \Crefname{defn}{Definition}{Definitions}
      \newtheorem{conv}[numberingbase]{Convention}
      \Crefname{conv}{Convention}{Conventions}
      \newtheorem{eg}[numberingbase]{Example}
      \Crefname{eg}{Example}{Examples}
      \newtheorem{rem}[numberingbase]{Remark}
      \Crefname{rem}{Remark}{Remarks}
      \newtheorem*{rems}{Remarks}
      \newtheorem*{egs}{Examples}
      
      
      \theoremstyle{plain}
      \newtheorem{thm-tweak}[subsection]{Theorem}
      \Crefname{thm-tweak}{Theorem}{Theorems}
      \newtheorem{lemma-tweak}[subsection]{Lemma}
      \Crefname{lemma-tweak}{Lemma}{Lemmas}
      \newtheorem{cor-tweak}[subsection]{Corollary}
      \Crefname{cor-tweak}{Corollary}{Corollaries}
      \newtheorem{prop-tweak}[subsection]{Proposition}
      \Crefname{prop-tweak}{Proposition}{Propositions}
      \newtheorem{conj-tweak}[subsection]{Conjecture}
      \Crefname{conj-tweak}{Conjecture}{Conjectures}
      
      \theoremstyle{definition}
      \newtheorem{defn-tweak}[subsection]{Definition}
      \Crefname{defn-tweak}{Definition}{Definitions}
      \newtheorem{eg-tweak}[subsection]{Example}
      \Crefname{eg-tweak}{Example}{Examples}
      \newtheorem*{rems-tweak}{Remarks}
      \newtheorem{rem-tweak}[subsection]{Remark}
      \Crefname{rem-tweak}{Remark}{Remarks}
      \makeatletter
      \renewcommand*\env@matrix[1][\arraystretch]{%
        \edef\arraystretch{#1}%
        \hskip -\arraycolsep
        \let\@ifnextchar\new@ifnextchar
        \array{*\c@MaxMatrixCols c}}
      \makeatother
      \newtheoremstyle{subsection-tweak}
         {11pt}
         {3pt}%
         {}
         {}%
         {\bfseries}
         {}%
         {.5em}
         {\thmnumber{\@{#1}{}\@{#2}.}%
          \thmnote{~{\bfseries#3.}}}
      
      \theoremstyle{subsection-tweak}
      \newtheorem{pp}[numberingbase]{}
      \newcommand{\bpp}{\begin{pp}}
      \newcommand{\epp}{\end{pp}}
      
      \theoremstyle{subsection-tweak}
      \newtheorem{pp-tweak}[subsection]{}
      
      
      \numberwithin{equation}{subsection}
%       \numberwithin{equation}{section}
%       \makeatletter
% \@addtoreset{equation}{section}
% \makeatother
      %-----------------------Enumeration
      \renewcommand{\labelenumi}{\b{\arabic{enumi}.}}
      \renewcommand{\labelenumii}{(\alph{enumii})}
      
      %-----------------------Page layout
      %\setlength{\parindent}{0pt} % 0 pt  = indentation
      %\pagenumbering{gobble}          % Uncomment to suppress page numbers
      
      
      %--------------------------Table of contents customization
      
      \makeatletter
      \def\@tocline#1#2#3#4#5#6#7{
      %    \par \addpenalty\@secpenalty\addvspace{#2}%
          \begingroup %\hyphenpenalty\@M
          \@ifempty{#4}{%
      %      \@tempdima\csname r@tocindent\number#1\endcsname\relax
          }{%
      %      \@tempdima#4\relax
          }%
      
          \parindent\z@ \leftskip#3\relax \advance\leftskip\@tempdima\relax
      %    \rightskip\@pnumwidth plus4em \parfillskip-\@pnumwidth
          #5\hskip-\@tempdima
            \ifcase #1
             \or\or \hskip 2em \or \hskip 1em \else \hskip 3em \fi%
            #6\nobreak\relax
          \dotfill\hbox to\@pnumwidth{\@tocpagenum{#7}}\par
          \nobreak
          \endgroup
        }
       \def\l@section{\@tocline{1}{0pt}{1pc}{}{}}
      
      \renewcommand{\tocsection}[3]{%
        \indentlabel{\@ifnotempty{#2}{\makebox[1.3em][l]{%
          \ignorespaces#1 \bfseries{#2}.\hfill}}}\bfseries{#3}
          \vspace{-0.5pt}}
      
      \renewcommand{\tocsubsection}[3]{%
        \indentlabel{\@ifnotempty{#2}{\hspace*{-0.5em}\makebox[2.1em][l]{%
          \ignorespaces#1#2.\hfill}}}#3
          \vspace{1.5pt}}
      
      %\renewcommand{\tocsectionvskip}{10pt}
      %\set{10pt}
      %\settocsectionvskip{10pt}
      
      \makeatother
      
      
      %----------------------------------------------Showkeys
      
      %\usepackage[notcite]{showkeys}   % for drafts to show all the labels in pdf
      
      
      \DeclareMathVersion{normal2}
      
      %\usepackage[titletoc]{appendix}
      %----------------------------------------------Showkeys
      
      %\usepackage[notcite]{showkeys}   % for drafts to show all the labels in pdf
      
      
      
      %-------------------------------------------------------------END OF ALL THAT CRAP!
      
      
      
      % Checklist to do before making a public update to the manuscript:
      %   	(a) Spell check;  done 11/12/17
      %     (b) Eliminate numbering of unreferenced equations;   done 11/12/17
      %     (c) Check if cited preprints have appeared and for discrepancies in numbering;
      %     (d) Thank the referee in the acknowledgements (if applicable).
      %     (e) Go through the manuscript and check for unnecessary italicization of references, reference numbers, parentheses, semicolons, etc. in theorem statements.
      % }
      
\setcounter{tocdepth}{2}
\setcounter{secnumdepth}{2}
      
      
      \begin{document}
      
      
      \title{THE GROTHENDIECK--SERRE CONJECTURE OVER VALUATION RINGS}
      
      \author{NING GUO}
      \address{D\'{e}partement de Math\'{e}matiques, Universit\'{e} Paris-Sud, Orsay Cedex, France }
      \email{ning.guo@math.u-psud.fr}
      \date{\today}
      
      
      \begin{abstract}
      In this article, we establish the Grothendieck--Serre conjecture over valuation rings: for a reductive group scheme $G$ over a valuation ring $V$ with fraction field $K$, a $G$-torsor over $V$ is trivial if it is trivial over $K$.
      This result is predicted by the original Grothendieck--Serre conjecture and the resolution of singularities.
      The novelty of our proof lies in overcoming subtleties brought by general nondiscrete valuation rings.
      By using flasque resolutions and inducting with local cohomology, we
      prove a non-Noetherian counterpart of Colliot-Th\'el\`ene--Sansuc's case of tori.
      Then, taking advantage of techniques in algebraization, we obtain the passage to the Henselian rank one case.
      Finally, we induct on Levi subgroups and use the integrality of rational points of anisotropic groups to reduce to the semisimple anisotropic case, in which we appeal to properties of parahoric subgroups in Bruhat--Tits theory to conclude.
      \end{abstract}
      
      \small\maketitle
      
      \hypersetup{
          linktoc=page,     %set to all if you want both sections and subsections linked
      }
      \renewcommand*\contentsname{}
      %\q\\
      \tableofcontents
      
      
      \section{The Grothendieck--Serre conjecture and Zariski's local uniformization}
      Originally conceived by  A. Grothendieck \cite{Gro58}*{pp.~26-27, Rem.~3} and J.-P. Serre \cite{Ser58}*{p.~31, Rem.} in 1958, the prototype of the Grothendieck--Serre conjecture predicted that for an algebraic group $G$ over an algebraically closed field $k$, a $G$-torsor over a nonsingular $k$-variety is Zariski-locally trivial if it is generically trivial.
      With its subsequent generalization to regular base schemes by A. Grothendieck \cite{Gro68}*{Rem.~1.11.a} and the localization by spreading out, the conjecture became the following.
      \bconj[Grothendieck--Serre]\label{GSconj}
      For a reductive group scheme $G$ over a regular local ring $R$ with fraction field $K$, the following map between nonabelian \'etale cohomology pointed sets has trivial kernel\ucolon
      \[
      H^1_{\et}(R,G)\ra H^1_{\et}(K,G);
      \]
      in other words, a $G$-torsor over $R$ is trivial if its restriction over $K$ is trivial.
      \econj
      Diverse variants and cases of \Cref{GSconj} were derived in the last several decades.
      For the history of the topic, we refer to \cites{Guo19, FP15}.
      The goal of this article is to settle the analogue of \Cref{GSconj} when $R$ is instead assumed to be a valuation ring.
      This variant is expected because of the following consequence of the resolution of singularities conjecture, a weak form of Zariski's local uniformization.
      \bconj[Zariski]\label{local-unif}
      Every valuation ring is a filtered direct limit of regular local rings.
      \econj
      By assuming \Cref{local-unif}, a limit argument \cite{Gir71}*{VII, 2.1.6} reduces the Grothendieck--Serre conjecture over valuation rings to the original \Cref{GSconj}.
      In particular, the combination of \Cref{GSconj} and \ref{local-unif} predicts our following main result.
      \bthm\label{GSVal}
      For a reductive group scheme $G$ over a valuation ring $V$ with fraction field $K$, the map
      \[
      \text{\tag{$\diamondsuit$}  $H^1_{\et}(V,G)\ra H^1_{\et}(K,G) \label{GSV}$\qq  is injective.}
      \]
      \ethm
      The special case of \Cref{GSVal} when $G$ is an orthogonal group for a nondegenerate quadratic form and $V$ is a valuation ring in which $2$ is invertible was proved in \cite{CTS87}*{6.4} and \cite{CLRR80}*{Thm.~4.5}. 

      


      The non-Noetherianness of general valuation rings introduces considerable subtleties, even when $G$ is a torus.
      Namely, in this case we can no longer adopt the method of \cite{CTS87}*{4.1} and need to devise alternative arguments.
      For instance, a crucial ingredient of \emph{loc.~cit.} is the exact sequence of \'etale sheaves
       \be\label{fes}
       0\ra \bG_{m, S}\ra i_{\ast}(\bG_{m,\xi})\ra \oplus_{x\in S^{(1)}}i_{x \ast}(\un{\b{Z}}_{x})\ra 0,
       \ee
      where $S$ is a semilocal regular scheme with the union of generic points $i: \xi\ra S$ and $x$ ranges over the points of codimension $1$.
      Being used in the proof of \cite{CTS87}*{2.2}, however, the right-exactness of (\ref{fes}) holds on the premise that all Weil divisors are principal, which fails for valuation rings, see \cite{EGAIV4}*{21.6.9}.
      To circumvent this, after using a flasque resolution of tori, we take advantage of local cohomology techniques to induct on the rank of the valuation ring.
      This reduces us to the following:
       \[
        \text{\tag{$\ast$} for a flasque torus $F$ over a valuation ring $(V,\fm_V)$, we have\qq  $H^2_{\fm_V}(V,F)=0$.}
       \]
      For a flasque torus with character group $\GGL$, by definition (\S\ref{fr}), the Galois action on $\GGL$ has special properties, so certain Galois cohomology of $\GGL$ vanishes, which leads to the vanishing of local cohomology ($\ast$) and therefore the case of tori:
      \bprop[\Cref{GS-tori}]
          For a torus $T$ over a valuation ring $V$ with fraction field $K$,
      \[\textstyle \text{the map \q $H^1_{\et}(V,T)\hra H^1_{\et}(K,T)$ \q is injective.}
      \]
      \eprop
      This case of tori, in turn, yields the simplest case of the product formula stated in (\ref{pdx}) below (or, see \Cref{pd-tori}), which is essential for further reduction of \Cref{GSVal}.
      
      
      A practical advantage of Henselian rank-one valuation rings is that several techniques of Bruhat--Tits theory, especially in \cite{BrT2}*{\S4-5}, become available.
      The goal of \S\ref{wk-ap} and \S\ref{passage-h} is to reduce \Cref{GSVal} to this case: after a limit argument that leads to the case of finite rank, we induct on the rank $n$ of a valuation ring $V$ by patching torsors.
      The induction hypothesis implies that our $G$-torsor over $V$ is a gluing of trivial torsors.
      For this gluing, we choose an $a\in V$ such that the $a$-adic completion $\hva$ is a Henselian valuation ring of rank one with $\hka\ce \Frac(\hva)$; so that, $\via$ is a valuation ring of rank $n-1$.
      % Subsequently, trivializations of a $G$-torsor over $\via$ and $\hva$ correspond to elements in $G(\via)$ and $G(\hva)$ resepectively.
      Similar to the Beauville--Laszlo's gluing of bundles, our patching is reformulated as the product formula
      \begin{equation}\label{pdx}
          \tst G(\hka)=\mathrm{Im}\bigl(G(\via)\ra G(\hka)\bigr)\cdot G(\hva).
      \end{equation}
      The strategy for proving this formula is a ``d\'evissage'' that establishes approximation properties of certain subgroups of $G_{\nhva}$.
      In this procedure, techniques of algebraization \cite{BC20}*{\S2} plays an important role, especially for a Harder-type approximation (see \S\ref{wk-ap}) and for the following integrality of rational points.
      \bprop[\Cref{aniso-int-rat}]\label{a-i-r}
      For a reductive anisotropic group scheme $G$ over a Henselian valuation ring $V$ with fraction field $K$, we have $ G(V)=G(K)$.
      \eprop
      Based on its special case when $K=\hka$ is complete due to Maculan \cite{Mac17}*{Thm.~1.1}, our approach to \Cref{a-i-r} is a reduction to completion that rests on techniques of algebraization to approximate schemes characterizing the anisotropicity of $G_{\nhva}$. 
      Indeed, \Cref{a-i-r} is an anisotropic version of the product formula (\ref{pdx}).
      \Cref{a-i-r} is helpful, not only for the reduction to the Henselian rank-one case, but also for the induction on Levi subgroups when reducing to the semisimple anisotropic case in \S\ref{passage-ss-ani}.
      After these reductions, we transfer \Cref{GSVal} into the injectivity of a map of Galois cohomologies.
      We conclude by taking advantage of properties of parahoric subgroups in Bruhat--Tits theory, see \Cref{final-proof}.
      
      In addition to techniques of algebraization, another crucial element of our reduction to the Henselian rank-one case is a lifting property of maximal tori of reductive group schemes.
      \blem[\Cref{extend-tor}]\label{lift-tor}
      Let $G$ be a reductive group scheme over a local ring $(R,\fm_{R})$ with a maximal $R/\fm_{R}$-torus $T$. If the cardinality of $R/\fm_{R}$ is at least $\dim(G)$, then $G$ has a maximal $R$-torus $\sT$ such that 
      \[
      \tst \sT_{R/\fm_R}=T.
      \]
      \elem
      This strengthens a result of Grothendieck \cite{SGA3II}*{XIV, 3.20} that a maximal torus of a reductive group scheme exists Zariski-locally on the base.
      By a correspondence of maximal tori and regular sections, the novelty is to lift regular sections instead of merely proving their existence Zariski-locally.
      Depending on inspection of the reasoning for \emph{loc.~cit.}, the key point is \cite{Bar67}, which guarantees that Lie algebras over fields with large cardinalities contain regular sections.
      For lifting regular sections, we need the functorial property of Killing polynomials.
      Indeed, Killing polynomials over rings were defined ambiguously in
      in the original literature, see \cite{SGA3II}*{XIV, 2.2}.
      Therefore, to establish \Cref{lift-tor}, we first add the supplementary details \S\ref{C} for Killing polynomials over rings.
      Subsequently, for a Lie algebra with locally constant nilpotent rank, we use the functoriality of Killing polynomials to deduce the openness of the regular locus. This openness permits us to lift regular sections, which amounts to lifting maximal tori.   
      \bpp[Notation and conventions]\label{convention}
      We always assume that each fibre of a reductive group scheme is connected. For a valuation ring $V$, we denote by $\fm_V$ the maximal ideal of $V$. 
      When $V$ has finite rank $n$, for the prime $\fp\subset V$ of height $n-1$ and $a\in \fm_V\backslash \fp$,  we denote by $\hva$ the $a$-adic completion of $V$. 
      For a module $M$ finitely generated over a topological ring $A$, we endow $M$ with the \emph{canonical topology} as the quotient of the product topology via $\pi\colon A^{\oplus n}\surjects M$. 
      By \cite{GR18}*{8.3.34}, this topology on $M$ is independent of the choice of $\pi$.
      In particular, we endow each finitely generated $\via$-module with the ``$a$-adic'' topology. For a reductive group scheme $G$ over a scheme $S$, by \cite{SGA3II}*{XIV, 6.1}, the functor
      \[
      \text{$\mtg\colon \Sch_{/S}\op\ra \Sets$,\qq $S\pr\mapsto \{\text{maximal tori of $G_{S\pr}$}\}$.}
      \]
      is representable by a smooth affine $S$-scheme.
      \epp
      \subsection*{Acknowledgements} I especially thank my advisor K\k{e}stutis \v{C}esnavi\v{c}ius for his kindness, helpful advice, and extensive comments for revising. I thank Ofer Gabber and Philippe Gille for useful conversations about properties of anisotropic groups. I thank Fei Xu, Yang Cao, and Xiaozong Wang for useful information. I also thank Jiandi Zou and Yisheng Tian for discussions. This article is supported by the EDMH doctoral program and the excellent condition for research at the Universit\'e Paris-Saclay.
 
      \section{The case of tori}
      The goal of this section is to prove the Grothendieck--Serre conjecture over valuation rings for tori (\Cref{GS-tori}), which is a non-Noetherian counterpart of Colliot-Th\'el\`      ene--Sansuc's result \cite{CTS87}*{4.1}.
      In \cite{CTS87}, the authors defined flasque resolutions of tori over arbitrary base schemes, which yielded several cohomological properties of tori over a regular scheme.
      In particular, they proved that for a torus $T$ over a semilocal regular ring $R$ with total ring of fractions $K$, the following map is injective:
      \begin{equation}\label{gs-old-tori}
        H^1_{\et}(R,T)\hra H^1_{\et}(K,T),
      \end{equation}
      which is a stronger version of the Grothendieck--Serre conjecture for tori, see \cite{CTS87}*{4.1}.
      Nevertheless, if we substitute $R$ in (\ref{gs-old-tori}) with a valuation ring $V$, then the method in \emph{loc.~cit.} does not work any more because of the non-Noetherianness of       $V$.
      Seeking an alternative argument in this case, we induct on the rank of $V$ and use local cohomology.
      This case of tori obtained in \Cref{GS-tori} is crucial for subsequent steps of the proof of \Cref{GSVal}, such as for patching torsors (see \Cref{decomp-gp} and \ref{rank-one-kernel-trivial}).
      
      \bpp[Flasque resolution of tori]\label{fr}
      The concepts of quasitrivial and flasque tori are rooted in two special Galois modules that then serve as character groups: permutation and flasque modules.
      A \emph{permutation} module $M$ over a finite group $G$ is a finite type $\b{Z}$-free module with a $\b{Z}$-basis on which $G$ acts via permutations; in this case, $M$ is of the form $\oplus_{i}\b{Z}[G/H_i]$ for subgroups $H_{i}\subset G$.
      If, in addition, $H^1(G, \mathrm{Hom}_{\b Z}(M,Q))=0$ for any permutation module $Q$, then $M$ is \emph{flasque}.
      For instance, if $Q$ is a finite $\b{Z}$-lattice with trivial $G$-action, then $Q$ is permutation and $H^{1}(G,\Hom(M,Q))=0$ for any flasque module $M$.
      Recall that the \emph{Cartier dual} of a torus $T$ over a scheme $S$ is a sheaf $\sD(T)\ce \cH om_{\text{$S$-gr.}}(T,\bG_{m,S})$.
      If for every connected component $Z$ of $S$ and every connected Galois cover $Z\pr\ra Z$ with Galois group $G$ splitting $T$, the $G$-module $(\sD(T))(Z\pr)$ is flasque (resp.,  permutation), then $T$ is \emph{flasque} (resp., \emph{quasitrivial}).
      Indeed, when $S$ is connected, every quasitrivial torus is a finite product of Weil restrictions $\mathrm{Res}_{S\pr_i/S}(\bG_{m})$ for finite \'etale  connected covers $S\pr_i\ra S$.
      As proved in \cite{CTS87}*{1.3.3}, for a torus $T$ over a scheme $S$ with finitely many connected components, there is an exact sequence of $S$-tori, a \emph{flasque resolution} of $T$:
      \begin{equation}\label{flasque-resolution}
        \text{$1\ra F\ra P\ra T\ra 1$,\qqq where $F$ is flasque and $P$ is quasitrivial.}
      \end{equation}
      \epp
      \blem\label{isotrivial}
          Let $T$ be a torus over a valuation ring $V$. 
          Then, $T$ is isotrivial\footnote{
      Recall from \cite{SGA3II}*{IX, 1.1} that a torus $\sT$ over a scheme $S$ is \emph{isotrivial}, if there is a finite \'etale surjective morphism of schemes $S\pr\ra S$ such that $\sT_{S\pr}$ splits.}. 
      To be precise, there is a Galois cover of affine schemes $\Spec R\ra \Spec V$ such that $T_{R}$ is split.
      \elem
      \bpf
          By a limit argument \cite{SGA3II}*{XV, 3.6}, there is a torus $T_{0}$ over a Noetherian domain $V_{0}\subset V$ that descends $T$ over $\Spec V_0$.
      Taking normalization of $V_{0}$ if necessary, we may assume that $V_{0}$ is normal.
      By the isotriviality of tori over locally Noetherian geometrically unibranch bases \cite{SGA3II}*{X, 5.16}, after base change to $V$, we get a minimal Galois cover $V \ra R$ splitting $T$.
      \epf
      \blem\label{2-vanish}
      For a flasque torus $F$ over a valuation ring $V$, the following local cohomology vanishes:
      \[
      H^2_{\fm_V}(V,F)=0.
      \]
      \elem
      \bpf
      We denote $X=\Spec(V)$ and $Z=\Spec(V/\fm_V)$.
      By excision \cite{Mil80}*{III, 1.28}, we may replace $X$ by its Henselization $X^{\mathrm{h}}$.
      For a variable $X$-\'etale scheme $X\pr$ with preimage $Z\pr\ce X\pr\times_{X} Z$, let $\cH^q_{Z}(-,F)$ be the \'etale sheafification of the presheaf  $X\pr\mapsto H^q_{Z\pr}(X\pr, F)$.
      By the local-to-global spectral sequence 
      \[
      \text{\qqqqqqqq\qqqqqqq$H^p_{\et}(X,\cH^q_Z(X,F)) \Ra H^{p+q}_Z(X,F)$,\qqqqq\qq (\cite{SGA4II}*{V, 6.4})}
      \]
      the desired $H^2_{Z}(X,F)=0$ follows from  $H^0_{\et}(X,\cH^2_{Z}(X, F))=H^1_{\et}(X,\cH^1_Z(X, F))=H^2_{\et}(X, \cH^0_Z(X, F))=0$.
      So first, we use \cite{SGA4II}*{VII, 5.9} to calculate the $\cH^i_Z(X,F)$ \'etale locally. In particular, it suffices to compute $H^q_{\ov{\fm}_V}(V^{\sh},F)$ for $0\leq q\leq 2$, where $\ov{\fm}_{V}$ is a geometric point above $\fm_{V}$ and $V^{\sh}$ is the strict Henselization of $V$ at $\ov{\fm}_{V}$.
      By \cite{SGA4II}*{V, 6.5}, we have the following long exact sequence
      \begin{equation}\label{str-loc-ext}
       \cdots\ra H^i_{\et}(V^{\sh}, F) \ra H^i_{\et}((V^{\sh})_{\mathfrak{p}},F)\ra H^{i+1}_{\ov{\fm}_V}(V^{\sh},F)\ra H^{i+1}_{\et}(V^{\sh},F)\ra \cdots.
      \end{equation}
      First, we compute $H^{q}_{\ov{\fm}_V}(V^{\sh},F)$ when $q=0$ and $2$. 
      The injectivity of $H^0_{\et}(V^{\sh},F)\ra H^0_{\et}((V^{\sh})_{\fp},F)$, combined with the vanishing of $H^1_{\et}((V^{\sh})_{\fp},F)$ and $H^i_{\et}(V^{\sh},F)$ for $i=1,2$ (see \SP{03QO}) imply that 
      \[
      \cH^0_{Z}(X,F)=\cH^2_{Z}(X,F)=0.
      \]
      It remains to calculate $\cH^{1}_{\ov{\fm}_V}(V^{\sh},F)$. 
      From (\ref{str-loc-ext}), we obtain the following short exact sequence 
      \[
      0\ra H^0_{\et}(V^{\sh},F)\ra H^0_{\et}((V^{\sh})_{\fp}, F)\ra H^1_{\ov{\fm}_V}(V^{\sh}, F)\ra H^1_{\et}(V^{\sh}, F)=0,
      \]
      where the last vanishing is Hilbert's theorem 90. 
      For the Cartier dual $\sD(F)$ of $F$, we denote by $\Lambda\ce \sD(F)(V^{\sh})$ the character group of $F$ as a $\b{Z}$-lattice and by $\Lambda^{\vee}\ce \Hom_{\b{Z}}(\Lambda,\b{Z}) $ its dual.
      The Cartier duality $F\cong\sD(\sD(F))\cong\cH om_{\text{$V$-gr.}}(\sD(F),\bG_m)$ provides
      \[
      \begin{matrix}[1.5]
        H^0_{\et}(V^{\sh},F)\cong F\p{V^{\sh}}\cong \cH om_{\text{$V$-gr.}}(\sD(F),\bG_m)(V^{\sh})=\Hom_{\b{Z}}(\Lambda, ({V^{\sh}})^{\times})\cong\Lambda^{\vee}\otimes_{\b{Z}}({V^{\sh}})^{\times}, \\
        \text{\!\!\!\!\!\!\!\!\!\!and similarly,\qqqqq $H^0_{\et}((V^{\sh})_{\fp},F)\cong \Lambda^{\vee}\otimes_{\b{Z}}(V^{\sh})_{\fp}^{\times}$.}
      \end{matrix}
      \]
      Subsequently, for the value group $\GG_{V^{\sh}/\fp}$ of $V^{\sh}/\fp$\footnote{Here, we use the fact that for every valuation ring $V$ with prime $\fp$, there is an isomorphism of nonunital rings $\fp V\isoto \fp V_{\fp}$.
      To see that, we write every element in $\fp V_{\fp}$ as the form $a/b$, where $a\in \fp V$ and $b\in V\backslash \fp$. 
      Since $V$ is the valuation ring of $K$, if $b/a\in V$ then $b\in \fp V$, which leads to a contradiction.
      Therefore, we have $a/b\in V$, so $a/b\in V\cap \fp V_{\fp}=\fp V$. In particular, we have $\fp V^{\sh}\isoto \fp (V^{\sh})_{\fp}$ so the fraction field of $V^{\sh}/\fp$ is $(V^{\sh})_{\fp}/\fp=V^{\sh}/\fp (V^{\sh})_{\fp}$.}, we have the following isomorphism
      \[
      H^1_{\ov{\fm}_V}(V^{\sh},F)=(\GGL^{\vee}\otimes_{\b{Z}}(V^{\sh})_{\fp}^{\times})/(\GGL^{\vee}\otimes_{\b{Z}}(V^{\sh})^{\times})\cong \GGL^{\vee}\otimes_{\b{Z}}\GG_{V^{\sh}/\fp}.
      \]
      The Henselianity of $X$ permits us to view $\cH^1_Z(X,F)$ as a sheaf over the site of profinite $\etp(V)\ce\etp(X,\ov{\fm}_V)$-sets.
      In particular, $\Spec(V/\fm_V)$ is the one-point set with trivial $\etp(V)$-action.
      By \cite{Sch13}*{3.7(iii)} derived from the Cartan--Leray spectral sequence, we obtain the first isomorphism below:
      % \begin{equation}\label{h1}
      % \begin{split}
      % \!\!\!\!\!\!\!\!\!\!\!\!\!\!\!\!\!\!\!\!\!\!\!\!\!\!\!\!\!\!\!\!\!\! H^1_{\et}(X,\cH^1_Z(X,F)) & \isom \q  H^1_{\et}(\etp(V), \mathrm{Hom}_{\b{Z}}(\Lambda,\Gamma_{V/\fp})) \\
      %  & = \varinjlim_{\text{$Y/X$ Galois}}H^1_{\et}(\Gal(Y/X), \Hom_{\b{Z}}(\Lambda, \Gamma_{V/\fp})^{\etp(Y)}), \\
      % \end{split}
      % \end{equation}
      \begin{equation}\label{h1}
        H^1_{\et}(X,\cH^1_Z(X,F)) \cong H^1(\etp(V),H^1_{\ov{\fm}_V}(V^{\sh},F)) \cong  H^1(\etp(V), \mathrm{Hom}_{\b{Z}}(\GGL,\GG_{V^{\sh}/\fp})).
      \end{equation}
      To see the action of $\etp(V)$ on $\mathrm{Hom}_{\b{Z}}(\GGL, \GG_{V^{\sh}/\fp})$, by \Cref{isotrivial}, we first note that the $\etp(V)$-action on $\GGL$ factors through its quotient $\Gal(Y/X)$, where $Y$ is the minimal Galois cover of $X$ splitting $F$. 
      Besides, 
      \[
         \GG_{V^{\sh}/\fp}\stackrel{\text{\SP{05WS}}}{=\joinrel=\joinrel=\joinrel=} \GG_{\p{V/\fp}^{\sh}} \overset{\text{\SP{0ASK}}}{=\joinrel=\joinrel=\joinrel=} \GG_{V/\fp},
      \]
      so $\etp(V)$ acts trivially on $\GG_{V^{\sh}/\fp}\cong\Frac(V/\fp)^{\times}/(V/\fp)^{\times}$.
      Thus, the $\etp(V)$-action on $\mathrm{Hom}_{\b{Z}}(\GGL, \GG_{V/\fp})$ factors through $\Gal(Y/X)$.
      Since $\etp(V)$ is a direct limit of the $\Gal(X_{\alpha}/X)$, where $X_{\alpha}$ are Galois covers of $X$, a limit argument \cite{Ser02}*{\S2.2, Cor.~1} reduces (\ref{h1}) to
      \begin{equation}\label{h2}
       \tst  H^1_{\et}(X,\cH^1_Z(X,F))\isom \varinjlim_{\alpha}  H^1\Bigl(\Gal(X_{\alpha}/X), \mathrm{Hom}_{\b{Z}}(\GGL,\GG_{V/\fp})^{\etp(X_{\alpha})}\Bigr).
      \end{equation}
      We express $\Gamma_{V/\fp}$ as a direct limit of finitely generated $\b{Z}$-submodules $\Gamma_i$. 
      Since $\GGL$ is $\b{Z}$-finitely presented,
      \begin{equation}\label{colimit}
       \tst \varinjlim_{i\in I}\Hom_{\b{Z}}(\Lambda, \Gamma_{i})\isoto \Hom_{\b{Z}}(\Lambda,\Gamma_{V/\fp}).
      \end{equation}
      Combining the isomorphism (\ref{colimit}) with a limit argument \cite{Ser02}*{\S2.2, Prop.~8}, we reduce (\ref{h2}) to
      \begin{equation*}
           \tst \varinjlim_{\alpha}H^1\Bigl(\Gal(X_{\alpha}/X), \varinjlim_{i\in I}\Hom_{\b{Z}}(\GGL,\GG_i)^{\etp(X_{\alpha})}\Bigr)  =  \varinjlim_{\alpha}\varinjlim_{i\in I}H^1\Bigl(\Gal(X_{\alpha}/X), \Hom_{\b{Z}}(\GGL,\GG_i)^{\etp(X_{\alpha})}\Bigr).
      \end{equation*}
      It suffices to calculate for a large $\alpha_{0}$ such that $X_{\alpha_0}$ splits $F$.
      In this situation, $\etp(X_{\alpha_0})$ acts trivially on $\mathrm{Hom}_{\b{Z}}(\GGL,\GG_{i})$. 
      Since $F$ is a flasque torus, its character group $\GGL$ is a flasque $\Gal(X_{\alpha_0}/X)$-module.
      As aforementioned, $\Gal(X_{\alpha_0}/X)$ acts trivially on $\GG_{V/\fp}$, so the $\GG_{i}$ are finite $\b{Z}$-lattices with trivial $\Gal(X_{\alpha_0}/X)$-action.
      The example in \S\ref{fr} implies that $H^1(\Gal(X_{\alpha_0}/X),\Hom_{\b{Z}}(\Lambda,\Gamma_{V/\fp}))=0$, which verifies that
      \[
       H^1_{\et}(X, \cH^1_Z(X,F))=0. \qedhere
      \]
      \epf
      \bprop\label{GS-tori}
      For a torus $T$ over a valuation ring $V$ with fraction field $K$, the following map
      \[
      \textstyle \text{$H^1_{\et}(V,T)\hra H^1_{\et}(K,T)$ \qq is injective.}
      \]
      \eprop
      \bpf
      Because $V$ is a direct limit of valuation rings of finite rank (see, for instance, \cite{BM20}*{2.22}), a limit argument for abelian sheaves \cite{SGA4II}*{VII, 5.7} reduces us to the case when $V$       has finite rank $n$.
      Since $T$ is commutative, proving that $H^1_{\et}(V,T)\ra H^1_{\et}(K,T)$ has trivial kernel suffices.
      A flasque resolution $1\ra F \ra P \ra T \ra 1$ of $T$ as (\ref{flasque-resolution}) leads to the following commutative diagram with exact rows
      \[
      \begin{tikzcd}
        H^1_{\et}(V,P)\ar{r} \ar{d} & H^1_{\et}(V,T) \ar{r} \ar{d} & H^2_{\et}(V,F)  \ar{d} \\
         H^1_{\et}(K,P)\ar{r}  & H^1_{\et}(K,T) \ar{r} & H^2_{\et}(K,F).
      \end{tikzcd}
      \]
      Since $P$ is quasitrivial, we have $P\isom \prod_{S\pr_i}\mathrm{Res}_{S\pr_i/\Spec V}\bG_{m}$ for  finite \'etale connected $V$-schemes $S_{i}\pr$, so \cite{SGA3IIInew}*{XIX, 8.4} gives an isomorphism $H^{1}_{\et}(V,P)\cong \prod_{S_{i}\pr}H^{1}_{\et}(S_{i}\pr, \bG_m)$.
      By Hilbert's theorem 90, $H^1_{\et}(V,P)=H^1_{\et}(K,P)=0$.
      Therefore, it suffices to verify that $H_{\et}^2(V,F)\ra H_{\et}^2(K,F)$ has trivial kernel.

      For this, we induct on the rank $n$ of $V$.
      The case of $V=K$ is trivial, so when $n\geq 1$, for the prime $\mathfrak{p}$ of $V$ of height $n-1$, we assume that the assertion holds for $V_{\mathfrak{p}}$ (which has rank $n-1$).
      Denote $X=\Spec(V)$ and $Z=\Spec(V/\fm_V)$. 
      By \cite{SGA4II}*{V, 6.5}, we have the following long exact sequence:
      \begin{equation}\label{loc-seq}
        \text{$\cdots \ra H^2_{Z}(X, F)\ra H^2_{\et}(X,F)\ra H^2_{\et}(X-Z,F)\ra H^3_{Z}(X,F)\ra \cdots$.}
      \end{equation}
      The induction hypothesis reduces us to showing that $H^2_{Z}(X,F)=0$, which is proved in \Cref{2-vanish}.      
      \epf
      With \Cref{GS-tori}, we use \Cref{2-vanish} to the sequence (\ref{loc-seq}) to obtain the following corollary.  
      \bcor
      Let $F$ be a flasque torus over a valuation ring $V$ with fraction field $K$, then the map
      \[
      \text{\qqq $H^1_{\et}(V,F)\isoto H^1_{\et}(K,F)$ \qq is an isomorphism.}
      \]
      \ecor
      
      
      
      \section{Algebraizations and a Harder-type approximation}\label{wk-ap}
      In \Cref{open-normal},  we acquire a higher-height analog of Harder's weak approximation \cite{Har68}*{Satz.~2.1} to reduce \Cref{GSVal} to the case of Henselian rank one valuation rings. 
      However, if we consider higher-height valuations, then Harder's argument no longer applies, see \Cref{rem-harder}. 
      To nevertheless generalize Harder's result, we approximate by integral points and take advantage of techniques of algebraization from \cite{BC20}*{\S2}.
      These procedures invoke a lifting \Cref{extend-tor} for maximal tori of reductive group schemes over local rings, which is essentially a lifting of Cartan subalgebras.
      \bpp[Regular sections, Cartan subalgebras and subgroups of type (C)]\label{C}
      For a ring $R$, a Lie $R$-algebra $\fh$ that is locally free as a module and a variable $R$-algebra $R\pr$ such that $\fh_{R\pr}\ce \fh\otimes_{R}R\pr$ has constant rank, every $a\in \fh_{R\pr}$ has an adjoint action on $x\in \fh_{R\pr}$, denoted by $\mathrm{ad}(a)(x)\ce [a,x]$ with characteristic polynomial $P_{\fh_{R\pr},a}(t)=t^{n}+c_{1}(a)t^{n-1}+\cdots+c_{n}(a)\in R\pr[t]$, where $n$ is the rank of $\fh_{R\pr}$. 
      For the sheaf of symmetric algebras $\sA\ce \un{\mathrm{Sym}}_{\cO_R}(\fh^{\vee})$ of the dual $R$-module $\fh^{\vee}$, the  associated vector bundle of $\fh$ is $\b{W}(\fh)\ce \un{\Spec}(\sA)$.
      By the universal property \cite{EGAII}*{1.7.4}, we identify 
      $\b{W}(\fh)(R\pr)$ with $\fh_{R\pr}$. 
      For each $i$, consider the assignment:
      \[
        c_i: \q \b{W}(\fh)(R\pr)\ra \b{A}^1_{R}(R\pr),\qq a\mapsto (R[\tau]\ra R\pr, \tau\mapsto c_i(a)).
      \]
      Each $c_{i}$ is a natural transformation and determines a morphism $R[\tau]\ra \GG(\sA)$ by sending $\tau$ to an element $c_{i}\in \GG(\sA)$.
      We define the \emph{Killing polynomial} of $\fh$ as $P_{\fh}(t)\ce t^{n}+c_{1}t^{n-1}+\cdots+c_{n}\in \GG(\sA)[t]$.
      By the functorialities of the $c_{i}$, the formation of Killing polynomials commutes with base change.
      When $R$ is a field $k$, the largest $r$ such that $P_{\fh}(t)$ is divisible by $t^r$ is the \emph{nilpotent rank} of $\fh$.
      The nilpotent rank of the Lie algebra of a reductive group scheme is \'etale-locally constant (see \cite{SGA3II}*{XV, 7.3} and \cite{SGA3IIInew}*{XXII, 5.1.2, 5.1.3}).
      Every $a\in \fh$ satisfying $c_{n-r}(a)\neq 0$ is called a \emph{regular element}. 
      Let $G$ be a reductive group scheme over a scheme $S$. 
      For the Lie algebra $\fg$ of $G$, if a subalgebra $\fd\subset \fg$ is Zariski-locally a direct summand such that its geometric fibre $\fd_{\ov{s}}$ at each $s\in S$ is nilpotent and equals to its own normalizer, then $\sigma$ is a \emph{Cartan subalgebra} of $\fg$ (\cite{SGA3II}*{XIV, 2.4}). 
      We say an $S$-subgroup $D\subset G$ is \emph{of type (C)}, if $D$ is $S$-smooth with connected fibres, and $\mathrm{Lie}(D)\subset \fg$ is a Cartan subalgebra.
      A section $\sigma$ of $\fg$ is a \emph{regular section}, if $\sigma$ is in a Cartan subalgebra such that $\sigma(s)\in \fg_s$ is a regular element for all $s\in S$.
      A section of $\fg$ with regular fibres is \emph{quasi-regular}, hence regular sections are quasi-regular.
      \bpp[Schemes of maximal tori]\label{max-tor}
      For a reductive group scheme $G$ defined over a scheme $S$, the functor
      \[
      \text{$\mtg\colon \Sch_{/S}\op\ra \Sets$,\qq $S\pr\mapsto \{\text{maximal tori of $G_{S\pr}$}\}$.}
      \]
      is representable by a smooth affine $S$-scheme (\cite{SGA3II}*{XIV, 6.1}).
      For an $S$-scheme $S\pr$ and a maximal torus $T\in \mtg(S\pr)$ of $G_{S\pr}$, by \cite{SGA3IIInew}*{XXII, 5.8.3}, the morphism
         \begin{equation}\label{conj-tor}
            \tst \text{$G_{S\pr}\ra \un{\mathrm{Tor}}(G_{S\pr}),\qq g\mapsto gTg^{-1}$}
         \end{equation}
         induces an isomorphism $G_{S\pr}/\un{\mathrm{Norm}}_{G_{S\pr}}(T)\cong \un{\mathrm{Tor}}(G_{S\pr})$.
         Here, $\un{\mathrm{Norm}}_{G_{S\pr}}(T)$ is an $S\pr$-smooth scheme (see \cite{SGA3II}*{XI, 2.4bis}).
         Now, we establish the lifting property of $\mtg$.
            \epp      
      \epp
      \blem\label{extend-tor}
      Let $G$ be a reductive group scheme over a local ring $R$ with residue field $\kappa$ and let $Z$ be the center of $G$.
      If the cardinality of $\kappa$ is at least $\dim(G/Z)$, then the following map is surjective:
      \[
         \mtg(R)\surjects \mtg(\kappa).
      \]
      \elem
      \bpf
      By \cite{SGA3II}*{XII, 4.7c)}, an isomorphism of schemes $\mtg\isom \un{\mathrm{Tor}}(G/Z)$ reduces us to the semisimple adjoint case, where the maximal tori of $G$ are exactly the subgroups of type (C) (\cite{SGA3II}*{XIV, 3.18}).
      These subgroups are bijectively assigned by $D\mapsto \mathrm{Lie}(D)$ to the Cartan subalgebras of $\fg\ce \mathrm{Lie}(G)$, see \cite{SGA3II}*{XIV, 3.9}.
      It suffices to lift a Cartan subalgebra $\fc_0\subset \fg_{\kappa}$ to that of $\fg$.
      Since the cardinality of $\kappa$ is at least $\dim(G/Z)=\dim(G)$, by \cite{Bar67}*{Thm.~1}, $\fc_{0}$ is of the form $\mathrm{Nil}(a_s)\ce \bigcap_{n}\ker(\mathrm{ad}(a_s^n))$ for some $a_{s}\in \fc_{0}$. 
      Hence \cite{SGA3II}*{XIII, 5.7} implies that $a_{s}\in \fc_{0}$ is a regular element of $\fg_{s}$.
      We take a section $a$ of $\fg$ passing through $a_{0}$ and claim that $V\ce \{\text{$s\in \Spec R$ such that $a_{s}\in \fg_{s}$ is regular}\}$ is an open subset of $\Spec R$.
      We may assume that $R$ is reduced.
      Since the nilpotent rank of $\fg$ is locally constant,  the Killing polynomial of $\fg$ at every $s\in \Spec R$ is uniformly $P_{\fg_s}(t)=t^r(t^{n-r}+(c_1)_st^{n-r-1}+\cdots+(c_{n-r})_{s})$ such that $(c_{n-r})_s$ is nonzero.
      Thus, the regular locus in $\fg$ is the principle open subset $\{c_{n-r}\neq 0\}\subset \b{W}(\fg)$ so $V$ is nonempty and open, which implies that $V=\Spec R$.
      In particular, the regular element $a_0\in \fc_0$ is lifted to a quasi-regular section $a\in \fg$, which by \cite{SGA3IIInew}*{XIV, 3.7}, is regular.
      By definition of regular sections, there is a Cartan subalgebra of $\fg$ containing $a$ and is the desired lifting of $\fc_{0}$.
      \epf
      Next, we combine this lifting property with techniques of algebraization to deduce a density \Cref{alg}. 
      Roughly speaking, this density permits us to ``replace'' maximal tori of $G_{\nhka}$ by those of $G_{\via}$.
      Subsequently, we prove an approximation of $\hka$-points of a maximal torus of $G_{\nhka}$ by $\via$-points (\Cref{open-in-torus}).
      \blem\label{alg}
      For a valuation ring $V$ of finite rank $n$ with prime $\mathfrak{p}$ of height $n-1$,  we choose an $a\in \fm_V\backslash \fp$ and form the $a$-adic completion $\hva$ with $\hka\ce \Frac(\hva)$.
      Let $G$ be a reductive group scheme over $V$ and $\mtg$ the scheme of maximal tori of $G$.
      We endow $\mtg(\hka)$ with the induced $a$-adic topology. Then, 
      \[
      \tst \text{the image of \q $\mtg(\via)\ra \mtg(\hka)$\q is dense.}
      \]
      \bpf
      Let $(a_N)$ be a Cauchy sequence in $\via$ with respect to the $a$-adic topology. 
      For an integer $m$, we denote the truncated sequence of $(a_N)$ by $(a_N)_{N\geq m}$. 
      With termwise addition and multiplication, all truncated Cauchy sequences form a ring $\mathrm{Cauchy}^{\geq m}(\via)$. 
      Passing to the limit, it fits into the surjection 
      \be\label{cauchy}
      \tst \varinjlim_{m\geq 0}\bigl(\mathrm{Cauchy}^{\geq m}(\via)\bigr)\surjects \hka,\qq (a_N)_N\mapsto \lim_{N\ra \infty}a_N.
      \ee
      Since $\hka$ is a field, the kernel $I$ of this map is a maximal ideal of the source.
      For an element $x\in \varinjlim_{m\geq 0}\bigl(\mathrm{Cauchy}^{\geq m}(\via)\bigr)\backslash I$, we can write it as a sequence $(x_N)_{N\geq N_0}$ such that all $x_{N}$ are nonzero. So $(x_{N})_{N\geq N_0}$ has an inverse by taking inverses termwisely.  
      Therefore, $\varinjlim_{m\geq 0}\bigl(\mathrm{Cauchy}^{\geq m}(\via)\bigr)$ is a local ring, whose unique maximal ideal $I$ satisfies $\varinjlim_{m\geq 0}\bigl(\mathrm{Cauchy}^{\geq m}(\via)\bigr)/I=\hka$.
      Since $\mtg$ is finitely presented and affine over $\via$, by the lifting property \Cref{extend-tor}, we obtain the following surjection
      \[
      \tst \qq\varinjlim_{m\geq 0}\Bigl(\mtg\bigl(\mathrm{Cauchy}^{\geq m}(\via)\bigr)\Bigr)\isom \mtg\Bigl(\varinjlim_{m\geq 0}\bigl(\mathrm{Cauchy}^{\geq m}(\via)\bigr)\Bigr)\surjects \mtg(\hka). \qedhere
      \]
      \epf
      \elem
      Next, we establish several open properties of certain maps to construct an open normal subgroup contained in the closure of the image of $G(\via)\ra G(\hka)$.
      \blem\label{open-in-torus}
      Let $V$ be a valuation ring of finite rank $n$ with prime $\fp$ of height $n-1$. 
      For an $a\in \fm_{V}\backslash \fp$, let $\hva$ be the $a$-adic completion of $V$ and $\hka\ce \mathrm{Frac}(\hva)$.
      For a reductive group scheme $G$ over $V$, we endow $G(\hka)$ with $a$-adic topology.
      Let $\ov{G(\via)}$ be the closure of the image of $G(\via)\ra G(\hka)$.
      For a fixed maximal torus $T$ of $G_{\wh{K}^a}$ with minimal splitting field $L_0$, consider the norm map
      \[
      N_{L_0/\wh{K}^a}\colon T(L_0)\ra T(\hka).
      \]
      Then, the image $U$ of this norm map is an open subgroup of $T(\hka)$ and is contained in $\ov{G(\via)}$.
      \elem
      \bpf Since $\mtg$ is of finite type over $V$, we endow $\mtg(\hka)$ with the induced $a$-adic topology.
      First, we prove that $U$ is open.
      Note that $\hka$ is a Henselian valued field, hence is \'etale-open in the sense of \cite{Ces15d}*{2.8~(2)}. 
      the kernel $R^1T$ of the norm map $\mathrm{Res}_{L_0/\nhka}(T_{L_0})\ra T$ is a torus\footnote{One can check that $R^1T\pr_{\nhka}$ is a torus: when $T\pr$ splits after a base change (of rank $k$), the associated $\b{Z}$-module of the corresponding base change of $R^1T\pr_{\nhka}$ has trivial Galois action and is the following
          \[
          \tst \text{$\mathrm{Coker}\p{\b{Z}^k\ra \b{Z}[\Gal(L_0/\hka)]^k, (n_i)\mapsto (n_i\cdot \id)}\isom \b{Z}[\Gal(L_0/\hka)-\{\id\}]^k$,\qq which is torsion-free}.
          \]}
      hence by \cite{SGA3II}*{IX, 2.1~e)} is $\hka$-smooth. % see .
      By the criterion \cite{Ces15d}*{4.3}, the norm map $N_{L_0/\nhka}\colon T(L_0)\ra T(\hka)$ is open so the image $U\ce N_{L_0/\nhka}(T(L_0))\subset T(\hka)$ is open.
      The proof for $U\subset \ov{G(\via)}$ proceeds as follows.
      % The proof proceeds as follows.
      \benumr
      \item 
         For the morphism (\ref{conj-tor}), a criterion for openness \cite{Ces15d}*{4.3} applies and we have the open map:
         \[
         \tst \phi\colon G(\hka)\ra \mtg(\hka),\qq g\mapsto gTg^{-1}.
         \]
         Consequently, $\phi$ sends every open neighborhood $W$ of $\id\in G(\hka)$ to an open neighborhood of $T$.
         By \Cref{alg}, the density of $\mtg(\via)$ in $\mtg(\hka)$  implies that
         \[
         \tst \phi(W)\cap \mathrm{Im}(\mtg(\via)\ra \mtg(\hka))\neq \emptyset.
         \]
         Hence, there are $T\pr\in \mtg(\via)$ and $g\in W$ such that $gTg^{-1}=T\pr_{\nhka}\in \phi(W)$.
      \item For the maximal torus $T\pr\in \mtg(\via)$ found in (i), by \Cref{isotrivial}, there is a minimal Galois cover $\via \ra R$ splitting $T\pr$.
      The base change $R\otimes_{\via}\hka\isom \prod_{i=1}^{r}L_i$ is a product of $a$-adically complete\footnote{Since $R$ is a finite flat $\via$-module, it is free and we have $\hRa=R\otimes_{\via}\hka$ so the $L_i$ are $a$-adically complete.} fields $L_i$.
          Let $\rho\colon \etp(\via)\ra \mathrm{GL}_n(\b{Z})$ be the $\etp(\via)$-action on the lattice $\b{Z}^n\isom \Hom_{\text{$R$-gr.}}\p{T\pr_{R},\bG_{m}}$ (\cite{SGA3II}*{X, 1.2}) with image $Q\ce \rho(\etp(\via))$.
          The minimality of $R$ amounts to that $\ker \rho=\etp(R)$.
          The functoriality of \'etale fundamental groups \cite{SGA1new}*{V, 6.1~\emph{ff.}} yields $\tau\colon \etp(\hka)\ra \etp(\via)$.
          Therefore, the kernel of $\rho\circ\tau\colon \etp(\hka)\ra Q$ is the \'etale fundamental group of a connected component of $\Spec(R\otimes_{\nvia}\hka)$, that is, $\etp(L_i)$ for all $i$. Consequently, $L_i$ are minimal splitting fields of $T\pr_{\nhka}$, so we may assume that $L_1\isom L_0$.
      \item For the maximal torus $T\pr\in \mtg(\via)$ found in (i), by the argument for $T$ in the very beginning, the image $U\pr\ce N_{L_0/\nhka}(T\pr(L_0))\subset T\pr(\hka)$ is open.
      Since the following map is surjective
          \[
          \tst \text{$\varinjlim_{m\geq 0}\mathrm{Cauchy}^{\geq m}(R)\surjects \hRa\cong \prod_{i=1}^{r}L_i$,}
          \]
          the image of $R^{\times}\ra \prod_{i=1}^{r}L_i^{\times}$ is dense.
          Note that $T\pr_{R}$ is split, so the image of $T\pr(R)\ra \prod_{i=1}^{r}T\pr(L_i)\overset{pr_1}{\ra} T\pr(L_0) $ is dense.
          For a $u\pr\in U\pr$, there is $v\pr\in T\pr(L_0)$ such that $N_{L_0/\nhka}(v\pr)=u\pr$. 
          Hence, there is a variable $r\pr\in T\pr(R)$ whose image approximates $v\pr$. 
          Taking the norm image $N_{R/\via}(r\pr)\in T\pr(\via)$ of $r\pr$, we see that $N_{R/\via}(r\pr)$ approximates $u\pr$.
          Consequently, we have 
          \[
          \tst U\pr\subset \ov{T\pr(\via)}.
          \]
          \item Let $(W_j)_{j\in J}$ be a filtered basis of open neighborhoods of $\id\in G(\hka)$.
          Just as the construction in (i), for each $j\in J$, there are $g_j\in W_j$ and $T\pr_j\in \mtg(\via)$ such that $T^{g_j}\ce g_jTg_j^{-1}=(T\pr_j)_{\nhka}\in \phi(W_j)$.
          By transport of structure, the norm map $N_{L_0/\nhka}$ sends $T^{g_j}(L_0)$ to $g_j(N_{L_0/\nhka}(T(L_0)))g_j^{-1}$.
          Therefore, every $U_j\ce N_{L_0/\nhka}(T^{g_j}(L_0))$ satisfies $U_j=g_jUg_j^{-1}$.
          For every open neighborhood $B_u\subset U$ of an element $u\in U$, the subsets $(W_j^{-1} u W_j)_{j\in J}$ form a basis of open neighborhood of $u$, hence $\{g_j^{-1}ug_j\}_{j\in J}\cap B_u\neq \emptyset$, which combined with $g_jB_ug_j^{-1}\subset U_j\subset \ov{G(\via)}$ implies that $u\in \ov{G(\via)}$.
          Thus, $U$ is contained in $\ov{G(\via)}$.     \qedhere
          \eenum
      \epf
      Now, we establish the main result \Cref{open-normal} of this section by constructing an open subgroup in the closure of $G(\via)$.
      Moreover, by lumping together the approximations in the case of maximal tori (\Cref{open-in-torus}), we prove that the constructed open subgroup is normal.
      We will see, this normality is crucial for the dynamic argument for root groups when proving the product formula \Cref{decomp-gp}.
      \bprop\label{open-normal}
      Let $V$ be a valuation ring of finite rank $n$ with prime $\fp$ of height $n-1$. 
      For an $a\in \fm_{V}\backslash \fp$, let $\hva$ be the $a$-adic completion of $V$ and $\hka\ce \mathrm{Frac}(\hva)$.
      For a reductive group scheme $G$ over $V$, we endow $G(\hka)$ with $a$-adic topology. Let $\ov{G(\via)}$ be the closure of the image of $G(\via)\ra G(\hka)$.
      Then
      \[
      \tst \text{$\ov{G(\via)}$ contains an open normal subgroup $N$ of $G(\hka)$.}
      \]
      \eprop
      \bpf
      \hfil
      \benumr
      \item We fix a maximal torus $T\subset G_{\nhka}$. We denote by $\fg$ the Lie algebra of $G_{\nhka}$ and by $\fh$ the Lie algebra of $T$. 
      For each $g\in G_{\nhka}$, by \cite{SGA3II}*{XIII, 2.6~b)}, the subspace $\fg^{\mathrm{ad}(g)}\subset \fg$ fixed by $\mathrm{ad}(g)$ has dimension at least $\dim T$. 
      We define the regular locus $G\reg\subset G_{\nhka}$ as the subscheme containing all $g\in G_{\nhka}$ that satisfy $\dim(\fg^{\mathrm{ad}(g)})=\dim T$.
      For a $t\in T$, by the following equation 
      \[
      \dim(\fg^{\mathrm{ad}(t)})=\dim(\fh^{\mathrm{ad}(t)})+\dim((\fg/\fh)^{\mathrm{ad}(t)}),
      \]
      $t$ is regular in $G_{\nhka}$ if and only if $t$ is regular in $T$, in which case we have $(\fg/\fh)^{\mathrm{ad}(t)}=0$. 
      \item Recall the open subgroup $U\subset T(\hka)$ constructed in \Cref{open-in-torus}.  We claim that $U\cap T^{\mathrm{reg}}(\hka)\neq \emptyset$. 
      For the minimal splitting field $L_0$ of $T$, we consider the norm map $\mathrm{Nm}\colon \Res_{L_0/\nhka}(T_{L_0})\ra T$.
      We note that $T_{L_0}\isom \bG_{m,L_0}^n$ is isomorphic to a dense open of $\bA^n_{L_0}$.
      Therefore, $\Res_{L_0/\nhka}(T_{L_0})$  is also a dense open subset of $\bA^{mn}_{\nhka}$ for $m\ce[L_0:\hka]$.
      The field $\hka$ is infinite, so $\p{\Res_{L_0/\nhka}(T_{L_0})}(\hka)$ is Zariski dense and $\p{\Res_{L_0/\nhka}(T_{L_0})}(\hka)\cap \mathrm{Nm}^{-1}(T^{\mathrm{reg}})(\hka)\neq \emptyset$.
      Applying $\mathrm{Nm}$ to this nonempty intersection proves the claim.
      \item For a fixed $t_{0}\in U\cap T\reg(\hka)$, by (i), we have $(\fg/\fh)^{\mathrm{ad}(t_0)}=0$. 
      So \cite{SGA3II}*{XIII, 2.2} implies that
      \[
      f\colon G_{\nhka}\times T\ra G_{\nhka}, \qq (g,t)\mapsto gtg^{-1}
      \]
      is smooth at $(\id, t_0)$.
      Thus, there is an open neighborhood $W$ of $(\id,t_0)$ such that $W(\hka)\subset G(\hka)\times U$ and $f|_{W}$ is smooth.
      Translate $W$ by $G(\hka)$-action, there is an open neighborhood $U_{0}\subset U$ of $t_{0}$ such that $U_{0}\subset W(\hka)$ and $f|_{G(\nhka)\times U_{0}}$ is smooth.
      The criterion \cite{Ces15d}*{2.9~(a)} implies that $E\ce f(G(\hka)\times U_{0})$ is an open subset of $G(\hka)$.
      Let $N$ be the subgroup of $G(\hka)$ generated by $E$.
      Hence, $N$ contains the open subset $f(G(\hka)\times U_0)$ so is an open subgroup. 
      \item Since $E$ is stable under $G(\hka)$-conjugation, $N$ is a normal subgroup of $G(\hka)$.
      For each $g\in G(\hka)$, we denote $T^{g}\ce gTg^{-1}$.
      The image $U^{g}\ce N_{L_0/\nhka}(T^g(L_0))$ satisfies $U^{g}=gUg^{-1}$. 
      Using \Cref{open-in-torus} to $T^{g}$,  we have $U^{g}\subset \ov{G(\via)}$.
      Since $E$ is the union of $gUg^{-1}$ for all $g\in G(\hka)$, the closure of $G(\via)$ contains $E$, hence contains $N$. \qedhere
      \eenum
      \epf 
     \bcor\label{clopen}
     With the notations in \Cref{open-normal}, then $\ov{G(\via)}$ is an open subgroup of $G(\hka)$ and
     \[
     \tst \ov{G(\via)}\cdot G(\hva)=\mathrm{Im}\bigl(G(\via)\ra G(\hka)\bigr)\cdot G(\hva).
     \]
     \ecor
     \bpf
       The image of $G(\via)\ra G(\hka)$ is a subgroup of $G(\hka)$, hence so is its closure $\ov{G(\via)}$.
       Since $\ov{G(\via)}$ contains an open subset $N$, it is an open (and therefore closed) subgroup of $G(\hka)$.
       Because $G$ is affine, by \cite{Con12}*{2.2}, the subgroup $G(\hva)\subset G(\hka)$ is open, hence the equality follows. 
     \epf
     \brem\label{rem-harder}
     Here, we compare Harder's original argument with the proof of \Cref{open-normal}.
     For a reductive group scheme $H$ over a valued field $F$ equipped with finitely many nontrivial nonequivalent valuations $v\in \cV$ of rank one, we consider the completions $F_{v}$ of $F$ at $v\in \cV$. 
     The product $\prod_{v\in \cV}H(F_v)$ is endowed with the product topology of $v$-adic topologies.
     For the diagonal map $i\colon H(F)\ra \prod_{v\in \cV}H(F_v)$, Harder  proved that the closure of $i(H(F))$ contains an open normal subgroup of $\prod_{v\in \cV}H(F_v)$ (see \cite{Har68}). 
     However, Harder's argument is only feasible for the case of rank one.
     Let $v$ be a higher-rank valuation of $F$
      % and $v\pr$ the rank-one valuation of $F$ generalizing $v$.
     % By \cite{EP05}*{2.3.4}, two completions coincide $F_{v}=F_{v\pr}$. 
     The point is, $F_{v}$ is not Henselian in general, see \cite{EP05}*{2.4.6}.
     Without the Henselianity, the criterion \cite{Ces15d}*{4.3}, which is important for Harder's rank-one case \cite{Guo19}*{proof of 2.1}, does not apply.
     \erem
      \section{Passage to the Henselian rank one case: patching by a product formula}\label{passage-h}
      The aim of this section is to reduce \Cref{GSVal} to the case when $V$ is a Henselian valuation ring of rank one.
      The key of our reduction is the following product formula  for patching trivial torsors
      \[
          \tst  \text{\qqqqqq\qqqqqq$G(\hka)=\mathrm{Im}\bigl(G(\via)\ra G(\hka)\bigr)\cdot G(\hva).$ \qqqqqq (\Cref{decomp-gp})}
      \]
      To show this, we establish several properties (\Cref{aniso-int-rat}) of an anisotropic group over a Henselian valuation ring.
      Based on Maculan's compactness \cite{Mac17}*{Thm.~1.1} of anisotropic groups over complete nonarchimedean valued fields, we use the algebraization technique in \cite{BC20}*{\S2} to prove the same for the Henselian case, generalizing its discrete valued case \cite{Guo19}*{3.6}.
      Subsequently, by the Harder-type approximation in \Cref{open-normal}, we prove the product formula, which gives the reduction \Cref{rank-one-kernel-trivial}.

      First, we recall a criterion for anisotropicity \cite{SGA3IIInew}*{XXVI, 6.14}, which is practically useful.
      \blem\label{ani-cri}
        A reductive group scheme $G$ over a semilocal connected scheme $S$ is anisotropic if and only if $G$ has no proper parabolic subgroup and $\rad(G)$ contains no copy of $\bG_{m}$.
      \elem
      Precisely, to determine whether $G$ is anisotropic, we consider the functor
      \[
      \Par(H)\colon \Sch_{/S}\op\ra \Sets,\qq S\pr\mapsto \{\x{parabolic subgroups of $H_{S\pr}$}\},
      \]
      which is representable by a smooth projective $S$-scheme (see \cite{SGA3IIInew}*{XXVI, 3.5}).
      To characterize the splitting behaviour of tori, recall the anti-equivalence of categories in \cite{SGA3II}*{X, 1.2}
      \[\qq \hkh{\begin{matrix}
                     \text{isotrivial tori over $S$}
                   \end{matrix} } \isoto
            \hkh{\begin{matrix}
                   \text{finite $\mathbb{Z}$-lattices with} \\
                   \text{continuous $\etp(S)$-action}
                 \end{matrix} } \qq T\mapsto M(T)\ce \mathrm{Hom}_{\x{$\ov{s}$-gr.}}(T_{\ov{s}}, \bG_{m,\ov{s}}),
      \]
      where $\ov{s}$ is a fixed geometric point of $S$ that is used as the base point when forming the \'etale fundamental group $\etp(S)$. 
      Therefore, a split subtorus of $T$ corresponds to a quotient lattice of $M(T)$ with trivial $\etp(S)$-action.
      By \Cref{isotrivial}, when $S$ is a valuation ring, the left-hand side is the category of all $S$-tori. 
      
      Then, we use \Cref{ani-cri} to establish several properties about anisotropic groups over a valuation ring.
      \bprop\label{aniso-int-rat}
      Let $G$ be a reductive group scheme over a valuation ring $V$ with fraction field $K$.
      \benum
      \item\label{ani-frac} $G$ is anisotropic if and only if so is $G_K$.
      \item\label{hen-ani}
      Suppose that $V$ is Henselian with an element $a\in \fm_{V}$.
      We form the $a$-adic completion $\hva$ of $V$. 
      Then, $G$ is anisotropic if and only if so is $G_{\nhva}$.
      \item\label{ani-eq} If $V$ is Henselian and $G$ is anisotropic, then $G(V)=G(K)$.
      \eenum
      \eprop
      \bpf\hfill     
      \benum
      \item The ``if'' part follows, since the base change of a nontrivial split torus of $G$ is a nontrivial split torus $G_{K}$.
      For the ``only if'' part, we begin by assuming that $G$ is anisotropic, which, by \Cref{ani-cri}, means that $G$ has no proper parabolic subgroup and $\rad(G)$ contains no copy of $\bG_{m}$.
      The valuative criterion for the projectivity of $\Par(G)$ implies that $\Par(G)(K)=\Par(G)(V)=\{\ast\}$, where the one-point set denotes the trivial parabolic $G$ and $G_{K}$.
      If $\rad(G_K)=\rad(G)_K$ contains a nontrivial split torus, then $M(\rad(G_K))$ has a quotient lattice permitting trivial $\etp(K)$-action.
      Since the valuation ring $V$ is normal, hence, by \SP{0BQM}, the map $\etp(K)\ra \etp(V)$ is surjective.
      Because $M(\rad(G_K))=M(\rad(G))$, the right-hand term has a nontrivial quotient lattice with trivial $\etp(V)$-action, giving a nontrivial split subtorus of $\rad(G)$.
      Hence, $G_{K}$ is also anisotropic.
      \item
      The deduction from the anisotropicity of $G_{\nhva}$ to that of $G$ is trivial.
      Assume that $G_{\nhva}$ is not anisotropic, it remains to show that either $\Par(G)(V)\neq \{\ast\}$, or $\rad(G)$ contains a copy of $\bG_{m}$.
      If $\Par(G)(\hva)\neq \{\ast\}$, then the Henselianity of $\hva$ implies that $\Par(G)(\hva)\neq \{\ast\}$.
      Note that $V/\fm_V\ra \hva/\fm_{\nhva}$ is an isomorphism (see \SP{05GG}), so $\Par(G)(V/\fm_V)=\Par(G)(\hva/\fm_{\nhva})\neq \{\ast\}$.
      Since $V$ is $a$-adically Henselian, by \SP{09XJ},  $(V,\fm_{V})$ is a Henselian pair.
      Subsequently, by the case $t=1$ of \cite{BC20}*{2.2.16}, the map $\Par(G)(V)\ra \Par(G)(V/\fm_V)$ is surjective so $\Par(G)(V)\neq \{\ast\}$.
      Now, assume that $\rad(G_{\nhva})$ contains a copy of $\bG_{m}$.
      We consider the functor
      \[
      \un{\Hom}(\bG_m, \rad(G))\colon \Sch\op_{/V}\ra \Sets, \qq R\mapsto \Hom_{\text{$R$-gr.}}(\mathbb{G}_{m,R}, \rad(G)_R),
      \]
      which is representable by an \'etale locally constant group scheme (see \cite{SGA3II}*{X, 5.6}).
      Since $\un{\Hom}(\bG_m,\rad(G))(\hva)\neq \emptyset$, the sets $\un{\Hom}(\bG_m, \rad(G))(V/\fm_V)=\un{\Hom}(\bG_m,\rad(G))(\hva/\fm_{\nhva})$ are nonempty.
      Since $V$ is Henselian and $\un{\Hom}(\bG_m,\rad(G))$ is $V$-smooth, the map 
      \[ 
         \un{\Hom}(\bG_m,\rad(G))(V)\ra \un{\Hom}(\bG_m,\rad(G))(V/\fm_V)\neq \emptyset
       \]
      is surjective. 
      Consequently, $\rad(G)$ contains a copy of $\bG_{m}$, so by \Cref{ani-cri}, $G$ is not anisotropic.
      \item
      By \cite{BM20}*{2.22}, $V$ is a filtered direct union of valuation subrings $V_{i}$ of finite rank, such that each $V_{i}\ra V$ is a local ring map.
      Further, the Henselization $V_{i}^{\mathrm{h}}$ of each $V_{i}$ is also a valuation subring of $V$ and by \SP{0ASK} is of finite rank.
      Since the filtered inductive limit of Henselian local rings along local ring maps is a Henselian local ring (\SP{04GI}), $V$ is a filtered direct union of Henselian valuation subrings $V_{i}^{\mathrm{h}}$ of finite rank.
      Similarly, $K$ is a filtered direct union of $K_i^{\mathrm{h}}\ce\Frac(V_i^{\mathrm{h}})$.
      Since $G$ is finitely presented over $V$, there are affine group schemes $G_{i}$ smooth of finitely presented (\cite{Nag66}*{Thm.~3'}) over $V_{i}^{\mathrm{h}}$ such that each base change of $G_{i}$ over $V$ is isomorphic to $G$.
      Further, by \cite{Con14}*{3.1.11}, these $G_{i}$ are reductive group schemes.
      It is clear that all $G_{i}$ are anisotropic.
      By a limit argument \SP{01ZC}, we have $G(V)=\varinjlim_{i}G(V_i^{\mathrm{h}})$ and $G(K)=\varinjlim_{i}G(K_i^{\mathrm{h}})$.
      Subsequently, we reduce to the case when $V$ is Henselian of finite rank $n$.

      When $n=0$, we have $V=K$ is a field and the assertion is trivial.
      Now, we prove the case when $V$ is of rank one.
      For $a\in \fm_{V}\backslash\{0\}$, we form the $a$-adic completion $\hva$ of $V$ with $\hka\ce \Frac(\hva)$.
      By \ref{hen-ani}, $G_{\nhva}$ is anisotropic.
      For the nonarchimedean complete valued field $\hka$, by \cite{Mac17}*{Thm.~1.1}, $G(\hva)$ is a maximal bounded subgroup of $G(\hka)$.
      On the other hand, a result of Bruhat--Tits--Rousseau \cite{Rou77}*{Thm.~5.2.3} shows that $G(\hka)$ is bounded.
      Consequently, we have $G(\hva)=G(\hka)$.
      The rank-one assumption ensures that $V\hookrightarrow \hva$, so the map $G(V)\hra G(\hva)$ is injective.
      The fibre product $V=K\times_{\nhka}\hva$ (see \SP{0BNR}) and the affineness of $G$ yield
      \[
      \tst G(V)\isoto G(K)\times_{G(\nhka)}G(\hva)\cong G(K). 
      \]
      When $V$ is of rank $n>1$, we assume the assertion holds for the case of rank $\leq n-1$ and prove by induction.
      For the prime $\fp\subset V$ of height $n-1$, the localization $V_{\fp}$ and the quotient $V/\fp$ are Henselian valuation rings\footnote{By \SP{05WQ}, the rank-one valuation ring $V/\fp$ is Henselian.  
      For $V_{\fp}$, we use Gabber's criterion \SP{09XI}: that is, we need to check that every monic polynomial of the form
      \[
      \text{\qq $f(T)=T^N(T-1)+a_NT^N+\cdots+a_1T+a_0$,\qq where $a_{i}\in \fp V_{\fp}$ for $i=0,\cdots, N$ and $N\geq 1$}
      \]       
      has a root in $1+\fp V_{\fp}$. 
      Here, by the footnote $2$, we identify $\fp V_{\fp}$ with $\fp$.
      By \SP{0DYD}, the Henselianity of $V$ implies that $(V,\fp V)$ is a Henselian pair, hence $(V_{\fp}, \fp V_{\fp})$ satisfies Gabber's criterion.}. 
      Since $V$ is Henselian, a section of $\un{\Hom}(\bG_m, G)$ over $V/\fm_{V}$ lifts to a section over $V$.
      Hence, $G_{V/\fm_V}$ is anisotropic and so is $G_{V/\fp}$.
      By \ref{ani-frac}, $G$ is anisotropic implies that $G_{K}$ and $G_{V_{\fp}}$ are anisotropic. 
      By the settled rank-one case and the induction hypothesis, we have 
      \begin{equation}\label{ani-one}
       \text{$G(V/\fp)=G(V_{\fp}/\fp V_{\fp})$\qq and \qq $G(V_{\fp})=G(K)$.}
      \end{equation}
      The affineness of $G$ and the fibre product $V\isoto V_{\fp}\times_{V_{\fp}/\fp V_{\fp}}V/\fp$ imply the fibre product
      \begin{equation}\label{ani-fib-pro}
            G(V)\isoto G(V_{\fp})\times_{G(V_{\fp}/\fp V_{\fp})} G(V/\fp V).
      \end{equation}
      Therefore,  the combination of (\ref{ani-fib-pro}) and  (\ref{ani-one}) gives us the desired equation $G(V)=G(K)$.    \qedhere
      \eenum
      \epf
      The following lemma provides the tori version of the product formula.
      \blem\label{pd-tori}
        For a valuation ring $V$ of rank $n>0$ with prime $\fp$ of height $n-1$, we take an $a\in \fm_{V}\backslash \fp$ and form the $a$-adic completion $\hva$ with $\hka\ce \Frac(\hva)$.
        For a $V$-torus $T$, we have the product formula
        \[
         \tst  T(\hka)=\mathrm{Im}\bigl(T(\via)\ra T(\hka)\bigr)\cdot T(\hva).
        \]
      \elem
      \bpf
          The left-hand side contains the right-hand side, so it remains to show that every element of $T(\hka)$ is a product of elements of $\mathrm{Im}\bigl(T(\via)\ra T(\hka)\bigr)$ and $T(\hva)$. 
          Consider the commutative diagram
     \[
      \tst \tikz {
      \node  (A) at (0.5,0.6) {$0$};
      \node  (B) at (2,0.6) {$T(V)$};
      \node  (C) at (4,0.6) {$T(\via)$};
      \node  (D) at (6.5,0.6) {$H^1_{V/(a)}(V,T)$};
      \node  (E) at (9.3,0.6) {$H^1(V,T)$};
      \node  (F) at (12,0.6) {$H^1(\via,T)$};
      \node  (a) at (0.5,-0.5)   {$0$};
      \node  (b) at (2,-0.5) {$T(V^{\mathrm{h}})$};
      \node  (c) at (4,-0.5) {$T(V^{\mathrm{h}}[\f{1}{a}])$};
      \node  (d) at (6.5,-0.5) {$H^1_{\small{V^{\mathrm{h}}/(a)}}(V^{\mathrm{h}},T)$};
      \node  (e) at (9.3,-0.5) {$H^1(V^{\mathrm{h}}, T)$};
      \node  (f) at (12,-0.5) {$H^1(V^{\mathrm{h}}[\f{1}{a}],T)$};
      \node  (aa) at (0.5,-1.7) {$0$};
      \node  (bb) at (2,-1.7) {$T(\hva)$};
      \node  (cc) at (4,-1.7) {$T(\hka)$};
      \node  (dd) at (6.5,-1.7) {$H^1_{\small{\nhva/(a)}}(\hva, T)$};
      \node  (ee) at (9.3,-1.7) {$H^1(\hva,T)$};
      \node  (ff) at (12,-1.7) {$H^1(\hka, T)$,};
      \draw [draw = black, thin,
      arrows={
      - Stealth }]
      (A) edge  (B);
      \draw [draw = black, thin,
      arrows={
      - Stealth }]
      (B) edge  (C);
      \draw [draw = black, thin,
      arrows={
      - Stealth }]
      (C) edge (D);
      \draw [draw = black, thin,
      arrows={
      - Stealth }]
      (D) edge (E);
      \draw [draw = black, thin,
      arrows={
      - Stealth }]
      (E) edge (F);
      \draw [draw = black, thin,
      arrows={
      - Stealth }]
      (B) edge (b);
      \draw [draw = black, thin,
      arrows={
      - Stealth }]
      (C) edge (c);
      \draw [draw = black, thin,
      arrows={
      - Stealth }]
      (D) edge (d);
      \draw [draw = black, thin,
      arrows={
      - Stealth }]
      (E) edge (e);
      \draw [draw = black, thin,
      arrows={
      - Stealth }]
      (F) edge (f); 
      \draw [draw = black, thin,
      arrows={
      - Stealth }]
      (a) edge (b);
      \draw [draw = black, thin,
      arrows={
      - Stealth }]
      (b) edge (c);
      \draw [draw = black, thin,
      arrows={
      - Stealth }]
      (c) edge (d);
      \draw [draw = black, thin,
      arrows={
      - Stealth }]
      (d) edge (e);
      \draw [draw = black, thin,
      arrows={
      - Stealth }]
      (e) edge (f);
      \draw [draw = black, thin,
      arrows={
      - Stealth }]
      (aa) edge (bb);
      \draw [draw = black, thin,
      arrows={
      - Stealth }]
      (bb) edge (cc);
      \draw [draw = black, thin,
      arrows={
      - Stealth }]
      (cc) edge (dd);
      \draw [draw = black, thin,
      arrows={
      - Stealth }]
      (dd) edge (ee);
      \draw [draw = black, thin,
      arrows={
      - Stealth }]
      (ee) edge (ff);
      \draw [draw = black, thin,
      arrows={
      - Stealth }]
      (b) edge (bb);
      \draw [draw = black, thin,
      arrows={
      - Stealth }]
      (c) edge (cc);
      \draw [draw = black, thin,
      arrows={
      - Stealth }]
      (d) edge (dd);
      \draw [draw = black, thin,
      arrows={
      - Stealth }]
      (e) edge (ee);
      \draw [draw = black, thin,
      arrows={
      - Stealth }]
      (f) edge (ff);
      }
      \]
       where $V^{\mathrm{h}}$ is the Henselization of $V$ and the rows are exact sequences of local cohomology \cite{SGA4II}*{V, 6.5.3}. 
       By \SP{0F0L}, $V^{\mathrm{h}}$ is also the $a$-Henselization of $V$, hence the $a$-adic completion of $V^{\mathrm{h}}$ is $\hva$ (see \cite{FK18}*{0, 7.3.5}).  
       By the tori case \Cref{GS-tori}, the horizontal morphisms in the two rightmost squares are injective. 
       The excision \cite{Mil80}*{III, 1.28} combined with a limit argument yield an isomorphism $H^{1}_{V/(a)}(V,G)\cong H^{1}_{V^{\mathrm{h}}/(a)}(V^{\mathrm{h}})$. 
       Therefore, a diagram chase gives the following product
       \begin{equation}\label{pd1}
         \tst  T(V^{\mathrm{h}}[\f{1}{a}])=\mathrm{Im}\p{T(\via)\ra T(V^{\mathrm{h}}[\f{1}{a}])}\cdot T(V^{\mathrm{h}})
       \end{equation}
       By \cite{BC20}*{2.2.16}, the image of $T(V^{\mathrm{h}}[\f{1}{a}])\ra T(\hka)$ is dense. 
       The openness of $T(\hva)$ in $T(\hka)$ implies
       \begin{equation}\label{pd2}
         \tst \mathrm{Im}\bigl(T(V^{\mathrm{h}}[\f{1}{a}])\ra T(\hka)\bigr)\cdot T(\hva)=\ov{\mathrm{Im}\bigl(T(V^{\mathrm{h}}[\f{1}{a}])\ra T(\hka)}\bigr)\cdot T(\hva)=T(\hka).
       \end{equation}
       Combining (\ref{pd1}) and (\ref{pd2}), we obtain the product formula for the case of tori.
      \epf
      \bprop\label{decomp-gp}
      For a valuation ring $V$ of rank $n>0$ with prime $\fp$ of height $n-1$, we take $a\in \fm_V\backslash\fp$ to form the $a$-adic completion $\hva$ of $V$ with $\hka\ce \Frac(\hva)$.
      For a reductive $V$-group scheme $G$, we denote by $G(\hva)$ its image in $G(\hka)$ and by $\mathrm{Im}(G(\via))$ the image of $G(\via)$ in $G(\hka)$. 
      Then, 
      \[
      G(\hka)=\mathrm{Im}\p{G(V\mathsmaller{[\f{1}{a}]})}\cdot G(\hva).
      \]
      \eprop
      \bpf
      The right-hand side is contained in the left-hand side, so we only need to show that every element of $G(\hka)$ is a product of elements of $\mathrm{Im}\p{G(\via)}$ and $G(\hva)$. 

      The case when $G_{\nhva}$ is anisotropic follows from \Cref{aniso-int-rat}. 
      When $G_{\nhva}$ contains no proper parabolic subgroup and $\rad(G_{\nhva})$ contains a nontrivial split torus of $G_{\nhva}$, we consider the commutative diagram
      \begin{equation}
      \begin{tikzcd}
      0  \arrow{r}
      & \rad(G)(\hva) \arrow{d} \arrow{r} & G(\hva) \arrow{d} \arrow{r} & (G/\rad(G))(\hva) \ar[equal]{d}        \arrow{r} & H^1(\hva, \rad(G)) \arrow{d} \\
      0 \arrow{r} & \rad(G)(\hka) \arrow{r} & G(\hka) \arrow{r} & (G/\rad(G))(\hka) \arrow{r} & H^1(\hka,\rad(G)),
      \end{tikzcd}
      \end{equation}
      where the rows are exact, and the equality follows from \Cref{ani-cri} and \Cref{aniso-int-rat}~\ref{ani-eq}. 
      Since $\rad(G_{\nhva})$ is a torus, by \Cref{GS-tori}, the last vertical arrow is injective. 
      Thus, a diagram chase gives $G(\hka)=\rad(G)(\hka)\cdot G(\hva)$ so the product formula for $\rad(G)$ (\Cref{pd-tori}) leads to the assertion.

      By \Cref{ani-cri}, the remaining case is when $G_{\nhva}$ contains a proper parabolic subgroup. 
       For a minimal parabolic subgroup $P$ of $G_{\nhva}$, we denote its unipotent radical by $U\ce \urad(P)$. 
       As exhibited in \cite{SGA3IIInew}*{XXVI, 6.11},  the centralizer of a maximal split torus $T\subset G_{\nhva}$ is a Levi subgroup $L$ of $P$.
       By \emph{op.~cit.}~2.4~\emph{ff.}, there is a maximal torus $\wt{T}$ of $G$ containing $T$.
       We denote the base change by $\wh{T}\ce\wt{T}_{\nhka}$, which is a maximal torus of $G_{\nhka}$. 
       As mentioned in (i) of the proof of \Cref{open-in-torus}, the following map 
       \[
       \phi\colon G(\hka)\ra \mtg(\hka),\qq g\mapsto g\wh{T}g^{-1}
       \]
          is open and sends an open neighborhood $W$ of $\mathrm{id}\in G(\hka)$ to an open neighborhood $\phi(W)$ of $\wh{T}\in \mtg(\hka)$.
        By \Cref{open-normal}, there is an open normal subgroup $N\subset G(\hka)$ contained in $\ov{\mathrm{Im}\p{G(\via)}}$.
        Since $G$ is affine, by \cite{Con12}*{2.2}, $G(\hva)\subset G(\hka)$ is closed and open.
        So, $N\cap G(\hva)$ is an open neighborhood of identity.
        Similarly, the affineness of $\mtg$ implies that $\mtg(\hva)\subset \mtg(\hka)$ is closed and open.
        Now, we choose $W$ such that $W\subset N\cap G(\hva)$. 
        Therefore, every $g\in W$ is contained in $\ov{\mathrm{Im}\p{G(\via)}}\cap G(\hva)$ and $\phi(W)\cap \mtg(\hva)$ is an open subset of $\mtg(\hka)$ containing $\wh{T}$.
        The fibre product $V\isoto \via\times_{\nhka}\hva$ (\SP{0BNR}) and the affineness of $\mtg$ give the isomorphism
        \begin{equation}\label{fp-tor}
          \mtg(V)\isoto \mtg(V\mathsmaller{[\f{1}{a}]})\times_{\mtg(\nhka)}\mtg(\hva).
        \end{equation}
        The density \Cref{alg} of $\mtg(\via)\ra \mtg(\hka)$ implies that the intersection of $\phi(W)\cap \mtg(\hva)$ and $\mathrm{Im}\p{\mtg(\via)}$ is nonempty, yielding a maximal torus $T_0\in \mtg(V)$ and a $g\in W$ such that $(T_0)_{\nhka}=g\wh{T}g^{-1}\in \phi(W)$.
          By the product formula for tori (\Cref{pd-tori}), we have $T_0(\hka)=\mathrm{Im}\p{T_{0}(\via)}\cdot T_0(\hva)$.
          Taking $\hka$-points of $\wh{T}=g^{-1}(T_0)_{\nhka}g$, we deduce that
          \begin{equation}\label{z-inside}
          \tst \wh{T}(\hka)=g^{-1}T_0(\hka)g=g^{-1}\mathrm{Im}\p{T_0(\via)}T_0(\hva)g\subset g^{-1}\mathrm{Im}\p{G(\via)}G(\hva)g.
          \end{equation}
        Since $g\in W$ is contained in $\ov{\mathrm{Im}\p{G(\via)}}\cap G(\hva)$, (\ref{z-inside}) implies that $\wh{T}(\hka)\subset \ov{\mathrm{Im}\p{G(\via)}}G(\hva)$.
        Note that \Cref{clopen} gives us $\ov{\mathrm{Im}(G(\via))}G(\hva)=\mathrm{Im}(G(\via))G(\hva)$. 
        Consequently, we have
        \begin{equation}\label{tor-decomp}
         \tst T(\hka)\subset \wt{T}(\hka)=\wh{T}(\hka)\subset \mathrm{Im}\p{G(\via)}G(\hva).
        \end{equation}
      In this paragraph, we prove that $U(\hka)\subset \ov{\mathrm{Im}\p{G(\via)}}$.
      The maximal split torus $T$ acts on $G_{\nhva}$
\[
   T\times G_{\nhva}\ra G_{\nhva},\qq (t, g)\mapsto tgt^{-1},
\]
inducing a weight decomposition $\mathrm{Lie}(G_{\nhva})=\bigoplus_{\alpha\in X^{\ast}(T)}\Lie(G_{\nhva})^{\alpha}$, where $X^{\ast}(T)$ is the character lattice of $T$. 
The subset $\Phi \subset X^{\ast}(T)-\{0\}$ such that $\mathrm{Lie}(G_{\nhva})^{\alpha}\neq 0$ is the relative root system of $(G_{\nhva}, T)$. 
By \cite{SGA3IIInew}*{XXVI, 6.1;  7.4}, $\mathrm{Lie}(L)$ is the zero-weight space of $\Lie(G_{\nhva})$ and the set $\Phi_{+}$ of positive roots of the relative root datum $((G_{\nhva}, T), X^{\ast}(T), \Phi)$ fits into the decomposition  
\[
   \tst \text{$\mathrm{Lie}(P)=\mathrm{Lie}(L)\oplus \p{\bigoplus_{\alpha\in \Phi_+}\mathrm{Lie}(G_{\nhva})^{\alpha}}$\qq  with\qq  $\Lie(U)=\bigoplus_{\alpha\in \Phi_+}\Lie(G_{\nhva})^{\alpha}$.}      
\]
Let $\wt{K}/\hka$ be a Galois field extension splitting $G_{\nhva}$.
By \emph{op.~cit.}~2.4~\emph{ff.}, there is a split maximal torus $T\pr\subset L_{\wt{K}}\subset P_{\wt{K}}$ of $G_{\wt{K}}$ containing $T_{\wt{K}}$. 
The centralizer of $T\pr$ in $G_{\wt{K}}$ is itself, which is also a Levi subgroup of a Borel $\wt{K}$-subgroup $B\subset P_{\wt{K}}$.
The adjoint action of $T\pr$ on $G_{\wt{K}}$ induces a decomposition $\Lie(G_{\wt{K}})=\bigoplus_{\alpha\in X^{\ast}(T\pr)}\Lie(G_{\wt{K}})^{\alpha}$, which is a coarsening of the base change of $\Lie(G_{\nhva})=\bigoplus_{\alpha\in X^{\ast}(T)}\Lie(G_{\nhva})^{\alpha}$ over $\wt{K}$. 
For the root system $\Phi\pr$ with the positive set $\Phi\pr_{+}$ for the Borel $B$, \emph{op.~cit.}~7.12 gives us a surjection $\eta \colon X^{\ast}(T\pr) \surjects X^{\ast}(T)$ such that $\Phi_{+}\subset u(\Phi\pr_+)\subset \Phi_{+}\cup \{0\}$.
By \emph{op.~cit.} 1.12, we have a decomposition
\[
\tst  U_{\wt{K}}=\prod_{\alpha\in \Phi\prpr}U_{\wt{K},\alpha},\qq \Lie(U_{\wt{K}})=\bigoplus_{\alpha\in \Phi\prpr}\Lie(G_{\wt{K}})^{\alpha},
   \] 
where $\Phi\prpr\subset \Phi\pr_{+}$ and we have isomorphisms $f_{\alpha}\colon U_{\wt{K},\alpha}\isoto \bG_{a,\wt{K}}$.
Since $\Lie(L)\subset \Lie(G_{\nhva})$ is the zero-weight space for the $T$-action, the restriction to $T$ of weights in $\Lie(U_{\wt{K}})$ must be nonzero, that is $\eta(\Phi\prpr)\subset \Phi_{+}$. 
For a cocharacter $\xi\colon \bG_m\ra T$, the dual map $\eta^{\ast}\colon X_{\ast}(T)\injects X_{\ast}(T\pr)$ of $u$ sends $\xi$ to a cocharacter $\eta^{\ast}(\xi)\in X_{\ast}(T\pr)$ of $T_{\wt{K}}$.
The adjoint action of $\bG_{m}$ on $U$ induced by $\xi$ is denoted by
\[
  \mathrm{ad}\colon  \bG_m(\hka)\times U(\hka)\ra U(\hka),\qq (t,u)\mapsto \xi(t)u\xi(t)^{-1}.
\]
For the open normal subgroup $N\subset G(\hka)$ constructed in \Cref{open-normal}, the intersection $N\cap U(\hka)$ is open, nonempty and stable under $\bG_{m}(\hka)$-action.
We consider the following commutative diagram
\[
\tikz {
\node  (C) at (-5.2,0) {$\bG_m(\hka)\times (N\cap U(\hka))$};
\node  (A) at (0,0) {$T(\hka)\times (N\cap U(\hka))$};
\node  (B) at (4.5,0) {$N\cap U(\hka)$};
\node  (c) at (-5.2,-1.5) {$\bG_m(\hka)\times U(\hka)$};
\node  (a) at (0,-1.5) {$T(\hka)\times U(\hka)$};
\node  (b) at (4.5,-1.5)   {$U(\hka)$};
\node  (d) at (-5.2,-3) {$\bG_m(\wt{K})\times U(\wt{K})$};
\node  (e) at (0,-3) {$T(\wt{K})\times U(\wt{K})$};
\node  (f) at (4.5,-3)   {$U(\wt{K})$.};
\node  (l) at (-2.6, 0.2) {$\xi\times \id$};
\node  (ll) at (2.55,0.2) {$\mathrm{ad}$};
\node  (m) at (-2.6, -1.3) {$\xi\times \id$};
\node  (mm) at (2.55, -1.3) {$\mathrm{ad}$};
\node  (n) at (-2.6, -2.8) {$\xi\times \id$};
\node  (nn) at (2.55, -2.8) {$\mathrm{ad}$};
\draw [draw = black, thin,
arrows={
- Stealth }]
(e) edge  (f);
\draw [draw = black, thin,
arrows={
- Stealth }]
(d) edge  (e);
\draw [draw = black, thin,
arrows={
- Stealth }]
(b) edge  (f);
\draw [draw = black, thin,
arrows={
- Stealth }]
(a) edge  (e);
\draw [draw = black, thin,
arrows={
- Stealth }]
(c) edge  (d);
\draw [draw = black, thin,
arrows={
- Stealth }]
(A) edge  (B);
\draw [draw = black, thin,
arrows={
- Stealth }]
(a) edge (b);
\draw [draw = black, thin,
arrows={
- Stealth }]
(B) edge (b);
\draw [draw = black, thin,
arrows={
- Stealth }]
(A) edge (a);
\draw [draw = black, thin,
arrows={
- Stealth }]
(C) edge  (c);
\draw [draw = black, thin,
arrows={
- Stealth }]
(C) edge  (A);
\draw [draw = black, thin,
arrows={
- Stealth }]
(c) edge  (a);
}
\]
Let $\varpi$ be a pseudo-uniformizer\footnote{We say that $\varpi\in \hka$ is a pseudo-uniformizer, if $\varpi\in \hva$ and $0<|\varpi|<1$, where $|\cdot |$ is the absolute value on $\hka$.
      Since $\hka$ is of rank one, the notion of pseudo-uniformizers coincides with that of topologically nilpotent units, which are elements of $(\hka)^{\times}$ such that their powers converge to $0$.} of $(\hka)^{\times}=\bG_m(\hka)$.
For an integer $m$, the action of $\varpi^{m}$ on $u\in U(\hka)$ is denoted by $(\varpi^m)\cdot u$.
Let $\wt{u}$ be the image of $u$ in $U(\wt{K})$.
Since $\wt{u}=\prod_{\alpha\in \Phi\prpr}f_{\alpha}({g}_{\alpha})$ with ${g}_{\alpha}\in \wt{K}$,  the image  of $(\varpi^m)\cdot u$ in $U(\wt{K})$ is $\p{\eta^{\ast}(\xi)(\varpi^m)} \wt{u} \p{\eta^{\ast}(\xi)(\varpi^m)}^{-1}$, expressed as the following
\[
\tst  \prod_{\alpha\in \Phi\prpr}\p{\eta^{\ast}(\xi)(\varpi^m)}f_{\alpha}(g_{\alpha})\p{\eta^{\ast}(\xi)(\varpi^m)}^{-1}=\prod_{\alpha\in \Phi\prpr}f_{\alpha}\bigl((\varpi^m)^{\langle \eta^{\ast}(\xi),\alpha\rangle}g_{\alpha}\bigr)=\prod_{\alpha\in \Phi\prpr}f_{\alpha}\bigl((\varpi^m)^{\langle \xi,\eta(\alpha)\rangle}g_{\alpha}\bigr). 
\]
Because $\eta(\Phi\prpr)\subset \Phi_{+}$, we can choose a cocharacter $\xi$ such that $\langle \xi, \eta(\alpha)\rangle$ are strictly positive. 
Then, when $m$ grows, the element $(\varpi^m)\cdot y\in U(\wt{K})$ gets closer to identity, and so the same holds in $U(\hka)$.
Thus, since $N\cap U(\hka)$ is an open neighbourhood of identity, every orbit of the $T$-action on $U(\hka)$ intersects with $N\cap U(\hka)$ nontrivially.
So, we have $U(\hka)=\bigcup_{t\in T(\nhka)}t(N\cap U(\hka))t^{-1}=N\cap U(\hka)$,
which implies that $U(\hka)\subset N$.
By combining with \Cref{open-normal}, we conclude that 
\begin{equation}\label{uni-in}
\tst U(\hka)\subset \ov{\mathrm{Im}\p{G(\via)}}.
\end{equation}
   

Now we show that $P(\hka)\subset \mathrm{Im}\p{G(\via)}G(\hva)$. By \Cref{aniso-int-rat}, the quotient $H\ce L/T$ satisfies $H(\hka)=H(\hva)$.
Since $T$ is split, Hilbert's theorem 90 gives the vanishing in the commutative diagram
      \begin{equation}\label{aniso-quot}
      \begin{tikzcd}
      0  \arrow{r}
      & T(\hva) \arrow{d} \arrow{r} & L(\hva) \arrow{d} \arrow{r} & H(\hva) \ar[equal]{d}        \arrow{r} & H^1(\hva, T)=0 \arrow{d} \\
      0 \arrow{r} & T(\hka) \arrow{r} & L(\hka) \arrow{r} & H(\hka) \arrow{r} & H^1(\hka,T)=0
      \end{tikzcd}
      \end{equation}
      with exact rows.
      A diagram chase yields $L(\hka)=T(\hka)L(\hva)$.
      Combining this with (\ref{tor-decomp}) and (\ref{uni-in}), we conclude that
      \begin{equation}\label{par-decomp}
       \tst P(\hka)\subset \mathrm{Im}\p{G(\via)}G(\hva).
      \end{equation}
      By \cite{SGA3IIInew}*{XXVI, 4.3.2, 5.2}, there is a parabolic $Q$ such that $P\cap Q=L$ fitting into the surjection
      \begin{equation}\label{rad-surj}
        \urad(P)(\hka)\cdot \urad(Q)(\hka)\surjects G(\hka)/P(\hka).
      \end{equation}
      Applying (\ref{uni-in}) to (\ref{rad-surj}) gives $G(\hka)\subset \ov{\mathrm{Im}(G(\via))}P(\hka)$, which combined with (\ref{par-decomp}) yields $G(\hka)\subset \ov{\mathrm{Im}(G(\via))}G(\hva)$.
      With the equality $\ov{\mathrm{Im}(G(\via))}G(\hva)=\mathrm{Im}(G(\via))G(\hva)$ verified in \Cref{clopen}, we obtain the desired product formula $G(\hka)=\mathrm{Im}(G(\via))G(\hva)$.\qedhere

      % Finally, when $G_{\nhva}$ contains no proper parabolic subgroup, there are two cases\footnote{Recall that a reductive group scheme over a semilocal connected scheme is anisotropic, if and only if it contains no proper parabolic subgroups and its radical subgroup contains no copy of $\bG_{m}$, see \cite{SGA3IIInew}*{XXVI, 6.14}.}: (1) $G_{\nhva}$ is anisotropic; (2) $\rad(G_{\nhva})$ contains a nontrivial maximal split subtorus $T_{r}$ of $G_{\nhva}$.
      % The case (1) follows directly from \Cref{aniso-int-rat}.
      % For (2), since $\rad(G_{\nhva})$ lies in the center of $G_{\nhva}$ (\cite{SGA3IIInew}*{XXII, 4.3.6}), we consider $G_{\nhva}/T_{r}$ and argue similarly as $H$ in (\ref{aniso-quot}), where $L$ is replaced by $G_{\nhva}$ and $T$ by $T_r$:
      % by diagram chase, we have $G(\hka)=T_{r}(\hka)G(\hva)\subset T_{r}(\via)T_{r}(\hva)G(\hva)\subset G(\via)G(\hva)$.
      \epf
      \bprop\label{rank-one-kernel-trivial}
      For \Cref{GSVal}, proving that \ref{GSV} has trivial kernel for rank one Henselian $V$ suffices.
      \eprop
      \bpf
      A twisting technique \cite{Gir71}*{III, 2.6.1(1)} reduces us to showing that the map \ref{GSV} has trivial kernel.
      The valuation ring $V$ is a filtered direct union of valuation subrings $V_i$ of finite rank (see, for instance, \cite{BM20}*{2.22}).
      Since direct limits commute with localizations, the fraction field $K=\Frac(V)$ is also a filtered direct union of $K_i\ce \Frac(V_i)$.
      A limit argument \cite{Gir71}*{VII, 2.1.6} gives compatible isomorphisms $\tst H^1_{\et}(V,G)\cong \varinjlim_{i\in I}H^1_{\et}(V_i,G)$ and $\tst H^1_{\et}(K,G)\cong \varinjlim_{i\in I}H^1_{\et}(K_i,G)$.
      Thus, it suffices to prove that \ref{GSV} has trivial kernel for $V$ of finite rank, say $n\geq 0$.
      When $n=0$, the valuation ring $V=K$ is a field, so this case is trivial.
      Our induction hypothesis is to assume that \Cref{GSVal} holds for two kinds of valuation rings $V\pr$: (1) for $V\pr$ Henselian of rank one; (2) for $V\pr$ of rank $n-1$.
      Indeed, (1) is only used for the case $n=1$.
      
      Let $\cX$ be a $G$-torsor lying in the kernel of $H^1_{\et}(V,G)\ra H^1_{\et}(K,G)$. For the prime $\mathfrak{p}\subset V$ of height $n-1$, we choose an element $a\in \fm_V\backslash\mathfrak{p}$ and consider the $a$-adic completion $\hva$ of $V$ with fraction field $\hka$.
      The induction hypothesis gives  the triviality of $\cX|_{\via}$ hence a section $s_1\in \cX(\via)$.
      Consequently, $\cX$ is trivial over $\hka$ and by induction hypothesis again, trivial over $\hva$ with $s_2\in \cX(\hva)$.
      By the product formula $G(\hka)=G(\via)G(\hva)$ in \Cref{decomp-gp}, there are $g_1\in G(\via)$ and $g_2\in G(\hva)$ such that $g_1 s_1$ and $g_2s_2$ have the same image in $\cX(\hka)$.
      Since $\cX$ is affine over $V$, by \SP{0BNR}, we have the fibre product $\cX(V)\isoto \cX(\via)\times_{\cX(\nhka)}\cX(\hva)$, which is nonempty, so the triviality of $\cX$ follows.
      \epf
      
      \section{Passage to the semisimple anisotropic case}\label{passage-ss-ani}
      After the passage to the Henselian rank one case (see \Cref{rank-one-kernel-trivial}), in this section, we further reduce \Cref{GSVal} to the case when $G$ is semisimple anisotropic, see \Cref{no-para}.
      For this, by induction on Levi subgroups, we reduce to the case when $G$ contains no proper parabolic subgroups.
      Subsequently, we consider the semisimple quotient of $G$, which is semisimple anisotropic.
      By using the integrality of rational points of anisotropic groups and a diagram chase, we obtain the desired reduction.
      \bprop\label{no-para}
      To prove \Cref{GSVal}, it suffices to show that \ref{GSV} has trivial kernel in the case when $V$ is a Henselian valuation ring of rank one and $G$ is semisimple anisotropic.
      \eprop
      \bpf
      First, we reduce to the case when $G$ contains no proper parabolic subgroups. 
      If $G$ contains a proper minimal parabolic $P$ with Levi $L$ and unipotent radical $\rad^u(P)$, consider the commutative diagram
      \[
      \tikz {
      \node  (C) at (-3.2,0) {$H_{\et}^1(V,L)$};
      \node  (A) at (-0.6,0) {$H_{\et}^1(V,P)$};
      \node  (B) at (2,0) {$H_{\et}^1(V,G)$};
      \node  (c) at (-3.2,-1.5) {$H_{\et}^1(K,L)$};
      \node  (a) at (-0.6,-1.5) {$H_{\et}^1(K,P)$};
      \node  (b) at (2.01,-1.5)   {$H_{\et}^1(K,G).$};
      \node  (L) at (-2.9,-0.73) {$l_L$};
      \node  (P) at (-0.3,-0.73) {$l_P$};
      \node  (G) at (2.3,-0.73) {$l_G$};
      \draw [draw = black, thin,
      arrows={
      - Stealth }]
      (A) edge  (B);
      \draw [draw = black, thin,
      arrows={
      - Stealth }]
      (a) edge (b);
      \draw [draw = black, thin,
      arrows={
      - Stealth }]
      (B) edge (b);
      \draw [draw = black, thin,
      arrows={
      - Stealth }]
      (A) edge (a);
      \draw [draw = black, thin,
      arrows={
      - Stealth }]
      (C) edge  (c);
      \draw [draw = black, thin,
      arrows={
      - Stealth }]
      (C) edge  (A);
      \draw [draw = black, thin,
      arrows={
      - Stealth }]
      (c) edge  (a);
      }
      \]
      By \cite{SGA3IIInew}*{XXVI, 2.3}, the left horizontal arrows are bijective.
      If a $G$-torsor $\cX$ lies in $\ker(l_G)$, then it satisfies $\cX(K)\neq \emptyset$.
      By \cite{SGA3IIInew}*{XXVI, 3.3; 3.20}, the fpqc quotient $\cX/P$ is representable by a scheme which is projective over $V$.
      The valuative criterion of properness gives $(\cX/P)(V)=(\cX/P)(K)\neq \emptyset$, so we can form a fibre product $\cY\ce \cX\times_{\cX/P}\Spec V$ from a $V$-point of $\cX/P      $.
      % Indeed, by \cite{SGA1new}*{VIII, 7.9}, the $P$-torsor $\cY$ is representable by a $V$-scheme.
      Since $\cY(K)\neq \emptyset$, the class $[\cY]\in \ker(l_P)$.
      On the other hand, the image of $[\cY]$ in $H_{\et}^1(V, G)$ coincides with $[\cX]$.
      Consequently, the triviality of $\ker(l_L)$ amounts to the triviality of $\ker(l_G)$.
      By \cite{SGA3IIInew}*{XXVI, 1.20} and \Cref{rank-one-kernel-trivial}, we reduce to proving \Cref{GSVal} for $V$ Henselian of rank one and $G$ without proper parabolic subgroup, more precisely, to showing that $\ker(H^1(V,G)\ra H^1(K,G))=\{\ast\}$ for such $V$ and $G$.
      
      For the radical $\rad(G)$ of $G$, the quotient $G/\rad(G)$ is $V$-anisotropic and by \Cref{aniso-int-rat} satisfies $(G/\rad(G))(V)=(G/\rad(G))(K)$, fitting into the following commutative diagram
      \[
      \tikz {
      \node  (A) at (-0.6,0) {$(G/\rad(G))(V)$};
      \node  (B) at (3,0) {$H_{\et}^1(V,\rad(G))$};
      \node  (C) at (6,0) {$H_{\et}^1(V,G)$};
      \node  (D) at (9,0) {$H_{\et}^1(V,G/\rad(G))$};
      \node  (a) at (-0.6,-1.5) {$(G/\rad(G))(K)$};
      \node  (b) at (3,-1.5)   {$H_{\et}^1(K,\rad(G))$};
      \node  (c) at (6,-1.5) {$H_{\et}^1(K,G)$};
      \node  (d) at (9,-1.5) {$H_{\et}^1(K,G/\rad(G))$};
      \node  (l) at (3.7,-0.75)  {$\scriptstyle{l(\rad(G))}$};
      \node  (ll) at (6.4,-0.75) {$\scriptstyle{l(G)}$};
      \node  (lll) at (9.9,-0.75) {$\scriptstyle{l(G/\rad(G))}$};
      \draw [draw = black, thin,
      arrows={
      - Stealth }]
      (A) edge  (B);
      \draw [draw = black, thin,
      arrows={
      - Stealth }]
      (B) edge  (C);
      \draw [draw = black, thin,
      arrows={
      - Stealth }]
      (C) edge (D);
      \draw [draw = black, thin,
      arrows={
      - Stealth }]
      (a) edge (b);
      \draw [draw = black, thin,
      arrows={
      - Stealth }]
      (b) edge (c);
      \draw [draw = black, thin,
      arrows={
      - Stealth }]
      (c) edge (d);
      \draw [draw = black, thin,
      arrows={
      - Stealth }]
      (B) edge (b);
      \draw [draw = black, thin,
      arrows={
      - Stealth }]
      (C) edge (c);
      \draw (-0.64cm,-3.2em) -- (-0.64cm,-0.8em);
      \draw (-0.56cm,-3.2em) -- (-0.56cm,-0.8em);
      \draw [draw = black, thin,
      arrows={
      - Stealth }]
      (D) edge (d);
      }
      \]
      If $\ker(l(G/\rad(G)))$ is trivial, then by the tori case \Cref{GS-tori} and a diagram chase, we conclude.
      \epf
      
      \section{Proof of the main theorem}\label{pf}
      In this section, we finish the proof of our main result \Cref{GSVal}.
      By the reduction of \Cref{no-para}, it suffices to deal with semisimple anisotropic group schemes over Henselian valuation rings of rank one.
      In this situation, we argue by using techniques in Bruhat--Tits theory and Galois cohomology to conclude.
      \bthm\label{final-proof}
      For a semisimple anisotropic group scheme $G$ over a Henselian valuation ring $V$ of rank one with fraction field $K$, the following map has trivial kernel\ucolon
      \[
      H_{\et}^1(V,G)\ra H_{\et}^1(K,G).
      \]
      \ethm
      \bpf
      Let $\wt{V}$ be a strict Henselization of $V$ at $\fm_V$ with fraction field $\wt{K}$ as a subfield of a separable closure $K\sep$.
      We consider three Galois groups $\GG\ce \Gal(\wt{V}/V)$, $\GG_{\wt{K}}\ce \Gal(K\sep/\wt{K})$ and $\GG_K\ce \Gal(K\sep/K)$.
      Since $V$ is Henselian, an application \cite{Sch13}*{3.7(iii)} of the Cartan--Leray spectral sequence yields an isomorphism $H^1_{\et}(V,G)\isom H^1(\GG,G(\wt{V}))$.
      By \cite{SGA4II}*{VIII, 2.1}, we have $H^1_{\et}(K,G)\isom H^1(\GG_K, G(K\sep))$.
      With these bijections, it suffices to show that each of the following maps has trivial kernel
      \[
      H^1(\GG,G(\wt{V}))\stackrel{\alpha}{\ra} H^1(\GG, G(\wt{K}))\stackrel{\beta}{\ra} H^1(\GG_K,G(K\sep)).
      \]
      For $\beta\colon H^1(\GG,G(\wt{K}))\ra H^1(\GG_K,G(K\sep))$, we invoke the inflation-restriction exact sequence \cite{Ser02}*{5.8~a)}:
      \[
      0\ra H^1(G_1/G_2, A^{G_2})\ra H^1(G_1,A)\ra H^1(G_2,A)^{G_1/G_2},
      \]
      for which $G_2$ is a closed normal subgroup of a group $G_1$ and $A$ is a $G_1$-group. It suffices to take
      \[
      \text{$G_1\ce \GG_K$, \qq $G_2\ce \GG_{\wt{K}}$,\qq and $A\ce G(K\sep)$.}
      \]
      For $\alpha\colon H^1(\GG, G(\wt{V}))\ra H^1(\GG, G(\wt{K}))$, let $z\in H^1(\GG,G(\wt{V}))$ be a cocycle in $\ker \alpha$, which signifies that
      \begin{equation}\label{cocycle}
        \text{there is an $h\in G(\wt{K})$ such that for every $s\in \GG$,\qq $z(s)=h^{-1} s(h)\in G(\wt{V})$.}
      \end{equation}
      Now we come to Bruhat--Tits theory and consider $G(\wt{V})$ and $hG(\wt{V})h^{-1}$ as two subgroups of $G(\wt{K})$.
      Let $\wt{\sI}(G)$ be the building of $G_{\wt{K}}$.
      Since $G_{\wt{K}}$ is semisimple, the extended building $\wt{\sI}(G)^{ext}\ce \wt{\sI}(G)\times (\Hom_{\text{$\wt{K}$-gr.}}(G,\bG_{m,\wt{K}})^{\vee}\otimes_{\b{Z}}\b{R})$ has       trivial vectorial part and equals to $\wt{\sI}(G)$.
      The elements of $G(\wt{K})$ act on the building $\wt{\sI}(G)$ and for each facet $F\subset \wt{\sI}(G)$, we consider its stabilizer $P^{\dagger}_F$ and its connected pointwise stabilizer $P^0_F$.
      In fact, there are group schemes $\mathfrak{G}^{\dagger}_{F}$ and $\mathfrak{G}^0_F$ over $\wt{V}$ such that $\mathfrak{G}^{\dagger}_F(\wt{V})=P^{\dagger}_F$ and $\mathfrak{G}^0_F(\wt{V})=P^0_F$, see \cite{BrT2}*{4.6.28}.
      Note that the residue field of $\wt{V}$ is separably closed and the closed fibre of $G_{\wt{V}}$ is reductive, so, by \cite{BrT2}*{4.6.22, 4.6.31}, there is a special point $x$ in the building $\wt{\sI}(G)$ such that the Chevalley group $G_{\wt{V}}$ is the stabilizer $\mathfrak{G}^{\dagger}_x=\mathfrak{G}_x^0$ of $x$ with connected fibres.
      By definition \cite{BrT2}*{5.2.6}, $G(\wt{V})$ is a parahoric subgroup of $G(\wt{K})$.
      Therefore, its conjugate $hG(\wt{V})h^{-1}$ is also a parahoric subgroup $P^0_{h^{-1}\cdot x}$.
      Since $G(\wt{V})$ is $\GG$-invariant, every $s\in \GG$ acts on $hG(\wt{V})h^{-1}$ as follows
      \[
      \tst \text{$s(hG(\wt{V})h^{-1})=s(h)G(\wt{V})s(h^{-1})\stackrel{(\ref{cocycle})}{\longeq}hG(\wt{V})h^{-1}$.}
      \]
      The $\GG$-invariance of $G(\wt{V})$ and $hG(\wt{V})h^{-1}$ amounts to that $x$ and $h\cdot x$ are two fixed points of $\GG$ in $\wt{\sI}(G)$.
      But by \cite{BrT2}*{5.2.7}, the anisotropicity of $G_{K}$ gives the uniqueness of fixed points in $\wt{\sI}(G)$.
      Thus, we have $G(\wt{V})=hG(\wt{V})h^{-1}$, which means that for every $g\in G(\wt{V})$ its conjugate $hgh^{-1}$ fixes $x$.
      This is equivalent to that $g$ fixes $h^{-1}\cdot x$ and to the inclusion of stabilizers $P_x^{\dagger}\subset P^{\dagger}_{h^{-1}\cdot x}$.
      On the other hand, every $\tau\in P^{\dagger}_{h^{-1}\cdot x}$ satisfies $h\tau h^{-1}\cdot x=x$, so $h\tau h^{-1}\in P^{\dagger}_x= G(\wt{V})$.
      Since $h$ normalizes $G(\wt{V})$, this inclusion implies that $\tau\in G(\wt{V})$ and $P^{\dagger}_{h^{-1}\cdot x}\subset G(\wt{V})$.
      Combined with $P^{\dagger}_x\subset P^{\dagger}_{h^{-1}\cdot x}$, this gives $P^{\dagger}_x=P^{\dagger}_{h^{-1}\cdot x}=G(\wt{V})$.
      Therefore, the stabilizer $P^{\dagger}_{h^{-1}\cdot x}$ is also a parahoric subgroup and equals to $P^0_{h^{-1}\cdot x}$.
      By \cite{BrT2}*{4.6.29}, the equality $P^0_x=P^0_{h^{-1}\cdot x}$ implies that $h^{-1}\cdot x=x$, so $h\in P^0_x=G(\wt{V})$, which gives the triviality of $z$.\qedhere
      \epf

      
      %\brem
      %
      %\erem
      
      
      



















\begin{bibdiv}
\begin{biblist}
% \bibselect{big}

\bibselect{gsval4}

\end{biblist}
\end{bibdiv}





%\bibliographystyle{habbrv.bst}
%\bibliography{gsval}




\end{document} 