\chapter{Down-up Chains for Multifurcating Trees: introducing the $(\alpha, \gamma)$-chain}\label{ch:downupmultifurcating}

To introduce an appropriate down-move in the multifurcating case we start by introducing a more detailed version of the $(\alpha, \gamma)$-growth model.
The main inspiration for what we will call the \textit{planar $(\alpha, \gamma)$-growth model} is the properties of the $(\alpha, \gamma)$-growth model given in Theorem~\ref{spinaldecomp}.
Here we basically characterize the tree structure in terms of an $\ocrp$, a $\crp$, and the weights associated with branch points and edges.
The main aim in the following will be to use a version of an $\ocrp$ instead of a $\crp$, whilst retaining the structure determined by first mentioned $\ocrp$ and the weights.\\

\noindent
The outline of this chapter is as follows:
%
\begin{enumerate}
  \item Introduce a planar $(\alpha, \gamma)$-growth model.
  \item Motivated by the definition of the $\alpha$-chain, generalize this label swapping approach to the multifurcating setting using this additional detail of the tree.
\end{enumerate}

\section{A planar $\left( \alpha, \gamma \right)$-growth model}
%
We begin by defining planar trees which will provide the state space for the process we are studying later on.
%
\fxnote*{New definition of planar trees, building on Chapter 2.}{%
\begin{defi}[Planar Tree]\label{planartree}
  Let $A$ be a finite set and fix $\tree \in \T_A$.
  Let $\treen = {\left( n_B \right)}_{B \in \tree}$ be a collection of non-negative integers, satisfying that for any collection of siblings $C_1, \ldots, C_k \in \tree$ it holds that $\left(n_{C_1}, \ldots, n_{C_k}\right)$ is a permutation of $\left( 1, \ldots, k \right)$. 
  We call $\hat{\tree} = (\tree, \treen)$ a \textit{planar tree}, and denote the space of planar trees with $n$ leaves by $\widehat{\T_n}$.
\end{defi}
%
To limit notation in the remainder of this exposition, we will introduce some vocabulary for planar trees, so let $\hat{\tree} = \left( \tree, \treen \right) \in \widehat{\T_A}$ be given.
If $C, D \in \tree$ such that $\overleftarrow{C} = \overleftarrow{D}$ and $n_{C} < n_{D}$ we will say that $C$ is \textit{to the left of} $D$.
An important note is, that we will often use this term to mean \textit{to the immediate left of}, which is the case if $n_{C} = n_{D} - 1$.
In the exact same manner, we can talk about edges being \textit{to the right} or even \textit{immediate right} of other edges.
A natural consequence is that we can define the \textit{leftmost} and \textit{rightmost} edge in collection of siblings $C_1, \ldots, C_k$.
The leftmost edge will be the unique $C \in \left\{ C_1, \ldots, C_k \right\}$ satisfying that $n_C = 1$, while the rightmost is of course the one satisfying that $n_C = k$. \\
}

\noindent
It is clear that there is a surjective projection map from the space of planar trees to the space of trees, $\rho \colon \widehat{\T_n} \to \T_n$, defined by
%
\begin{align}\label{projectionmap}
  \rho \left( \hat{\tree} \right) = \rho \left( \tree, \treen \right) = \tree
\end{align}
%
Note that this map is clearly \textit{not} injective for $n > 3$, which is illustrated in Figure~\ref{fig:planartree}.
We will continue to use the terms \textit{ancestral line} and \textit{spine} in some planar tree, $\hat{\tree}$.
By this we will simply mean the corresponding version $\rho{\hat{\tree}}$, defined earlier, which still makes fine sense for planar trees. \\
%
\begin{figure}[t]
    \centering
    \begin{tikzpicture}[%
    node distance = 0.8cm,
        table/.style={circle, draw, minimum width=0.7cm},
        bp/.style={circle, draw, fill=oxfordblue, inner sep=0pt, minimum width=4pt},
        comment/.style={circle, draw, color=red},
        label/.style={text=oxfordblue},
        line/.style={-, line width = 2pt, color = oxfordblue},
        leaf/.style={}<++>
        scale=1
    ]

%    \node[anchor = south] (0) at (0,0) {\verb|Root|};
%    \node[bp] (bp) at (0,-1) {};
%    \node[anchor = north east] (1) at (-0.5,-2.5) {1};
%    \node (2) at (0.5,-2) {2.5};

    \node (0) {\verb|Root|};
    
    \node at (-1,0) {$\hat{\tree}_1$};

    \node[bp, below=of 0] (bp2) {};

    \node[below left=of bp2] (1) {1};
    \node[below=of bp2] (2) {2};
    \node[below right=of bp2] (3) {3};

    \draw[line] (0) to (bp2) to (1);
    \draw[line] (bp2) to (2);
    \draw[line] (bp2) to (3);

    \begin{scope}[shift = {(3.5,0)}]
        \node (0) {\verb|Root|};
        
        \node at (-1,0) {$\hat{\tree}_2$};

        \node[bp, below=of 0] (bp2) {};

        \node[below left=of bp2] (1) {2};
        \node[below=of bp2] (2) {1};
        \node[below right=of bp2] (3) {3};

        \draw[line] (0) to (bp2) to (1);
        \draw[line] (bp2) to (2);
        \draw[line] (bp2) to (3);
    \end{scope}

    \begin{scope}[shift = {(7,0)}]
        \node (0) {\verb|Root|};

        \node at (-1,0) {$\hat{\tree}_3$};

        \node[bp, below=of 0] (bp2) {};

        \node[below left=of bp2] (1) {1};
        \node[below=of bp2] (2) {3};
        \node[below right=of bp2] (3) {2};

        \draw[line] (0) to (bp2) to (1);
        \draw[line] (bp2) to (2);
        \draw[line] (bp2) to (3);
    \end{scope}
    
    \begin{scope}[shift = {(10.5,0)}]
        \node (0) {\verb|Root|};

        \node at (-1,0) {$\hat{\tree}_4$};

        \node[bp, below=of 0] (bp2) {};

        \node[below left=of bp2] (1) {3};
        \node[below=of bp2] (2) {2};
        \node[below right=of bp2] (3) {1};

        \draw[line] (0) to (bp2) to (1);
        \draw[line] (bp2) to (2);
        \draw[line] (bp2) to (3);
    \end{scope}
%    \begin{scope}[shift = {(0,-4)}]
%        \node (0) at (0,0) {};
%        \node [bp] (1bp_start) at (0,0.5) {};
%        \node [bp] (1bp_end) at (1.5,0.5) {};
%        \node [anchor = south east] (tree1) at (0.5,1.5) {$\tilde{\text{\i}}$};
%        \node (bp_start) at (1.5,2) {};
%        \node (bp_mid2) at (3,2) {};
%        \node (i) at (3,3) {$i$};
%        \node [bp] (bp_end) at (4,2) {};
%        \node [anchor = south] (1) at (4.5,3) {};
%        \node [anchor = south east] (2) at (3.5,3) {};
%
%        \draw[line, dotted] (0) to (1bp_start);
%        \draw[line] (1bp_start) to (1bp_end);
%        \draw[line, red] (1bp_end) to node[auto, swap] {\footnotesize $\gamma$} (bp_start.center) to node[auto, swap] {\footnotesize $\alpha-\gamma$} (bp_mid2.center) to node[auto] {\footnotesize $1-\alpha$} (i);
%        \draw[line] (bp_mid2.center) to (bp_end);
%        \draw[line] (1bp_end) to (tree1);
%        \draw[line] (bp_end.center) [bend left] to (1.center) [bend left] to (bp_end.center) [bend left] to (2.center) [bend left] to (bp_end.center);
%
%        \node[label, anchor = south] (T_itilde) at (0,2) {$T_{\tilde{\text{i}}}$};
%    \end{scope}
%
%    \begin{scope}[shift = {(6,-4)}]
%        \node (0) at (0,0) {};
%        \node [bp] (1bp_start) at (0,0.5) {};
%        \node [bp] (1bp_end) at (1,0.5) {};
%        \node (tree1) at (0.5,1.5) {};
%        \node (bp_start) at (1,2) {};
%        \node (bp_mid1) at (2,2) {};
%        \node (itilde) at (2,3) {$\tilde{\text{\i}}$};
%        \node (bp_mid2) at (3,2) {};
%        \node (i) at (3,3) {$i$};
%        \node [bp] (bp_end) at (4,2) {};
%        \node [anchor = south] (1) at (4.5,3) {};
%        \node [anchor = south east] (2) at (3.5,3) {};
%
%        \draw[line, dotted] (0) to (1bp_start);
%        \draw[line] (1bp_start) to (1bp_end) to (bp_start.center) to (bp_mid1.center) to (itilde);
%        \draw[line, red] (bp_mid1.center) to node[auto, swap] {\footnotesize $\alpha$} (bp_mid2.center) to node[auto] {\footnotesize $1-\alpha$} (i);
%        \draw[line] (bp_mid2.center) to (bp_end);
%        \draw[line] (1bp_end) [bend left] to (tree1) [bend left] to (1bp_end);
%        \draw[line] (bp_end.center) [bend left] to (1.center) [bend left] to (bp_end.center) [bend left] to (2.center) [bend left] to (bp_end.center);
%
%        \node[label, anchor = south] (T_itilde) at (0,2) {$T_{\tilde{\text{i}}}$};
%    \end{scope}
\end{tikzpicture}

    \captionsetup{singlelinecheck=off}
    \caption[fig:planartree]{Four different elements of $\widehat{\T_3}$, $\hat{\tree}_1, \hat{\tree}_2, \hat{\tree}_3, \hat{\tree}_4$, where the non-planar version is the same tree, i.e.\ $\rho\left( \hat{\tree}_1 \right) = \rho\left( \hat{\tree}_2 \right) = \rho\left( \hat{\tree}_3 \right) = \rho\left( \hat{\tree}_4 \right) = \tree$.
        Specifically, we have for all $i \in [4]$ that $\hat{\tree}_i = \left( \tree, \treen^i \right)$ where $\treen^i = \left( n_1^i, n_2^i, n_3^i \right)$ are
    \begin{align*}
        \treen^1 = \left( 1, 2, 3 \right),
        \quad
        \treen^2 = \left( 2, 1, 3 \right),
        \quad
        \treen^3 = \left( 1, 3, 2 \right),
        \quad \text{and} \quad
        \treen^4 = \left( 3, 2, 1 \right).
    \end{align*}
}\label{fig:planartree}
\end{figure}
%
Before introducing a planar version of the $(\alpha, \gamma)$-growth model, let us transfer the insertion and deletion operations to this richer setting.
In the following, recall that $\pi$ and $\varphi$ denotes the deletion and insertion map, respectively, in the setting of Chapter~\ref{ch:prelims}.
%
\fxwarning*{I've changed these definitions quite a lot.}{%
\begin{defi}[Deletion of leaf in $\widehat{\T}$]\label{deletionplanar}
    Let $A$ be a finite set and fix $\hat{\tree} \in \widehat{\T_A}$.
    The operation of deleting a leaf, $\{ \tildei \} \subseteq A$, from $\hat{\tree}$ is a map
    %
    \begin{align}
        \hat{\pi} \left( \hat{\tree}, \tildei \right)
        = \hat{\pi} \left( \tree, \treen, \tildei \right)
        \overset{\hat{\pi}}{\mapsto}
        \left( \pi \left( \tree, \tildei \right), \treen^\pi \right)
        \in \widehat{\T_{A \setminus \{\tildei\}}}
        \label{deletionmapplanar}
    \end{align}
    %
    where $\treen^\pi = {\left( n_B^\pi \right)}_{B \in \pi\left( \tree, \tildei \right)}$ is defined by setting
    %
    \begin{align}
        n_B^\pi
        =
        \begin{cases}
            n_B - 1 & \text{if}\ \overleftarrow{\{\tildei\}} = \overleftarrow{B}\ \text{and}\ n_{\{\tildei\}} > n_B \\
            n_B & \text{otherwise} 
        \end{cases}
    \end{align}
    %
    for each $B \in \pi\left( \tree, \tildei \right)$.
\end{defi}
%
\begin{defi}[Insertion of leaf in $\widehat{\T}$]\label{insertionplanar}
    Let $A$ be a finite set and fix $\hat{\tree[s]} = \left( \tree[s], \treen \right) \in \widehat{\T_A}$.
    The operation of inserting a leaf, $\{ j \} \subseteq \N \setminus A$, to the vertex or edge, $C \in \tree[s]$, is a map
    %
    \begin{align}
        \left( \hat{\tree[s]}, C, j, l \right)
        =
        \left( \tree[s], \treen, C, j, l \right)
        \overset{\hat{\varphi}}{\mapsto}
        \left( \varphi \left( \tree[s], C, j \right), \treen^\varphi \right) \in \widehat{\T_{A \cup \{j\}}}
        \label{insertionsmapplanar}
    \end{align}
    %
    where $\treen^\varphi$ is obtained in the following way:
    %
    \begin{itemize}
        \item If $C$ is a vertex with $k$ children, $l \in [k+1]$ is allowed, and we set $\treen^\varphi = {\left( n_B^\varphi \right)}_{B \in \varphi \left( \tree[s], C, j \right)}$ with
            %
            \begin{align}
                n_B^\varphi
                =
                \begin{cases}
                    l & \text{if}\ B = \{j\} \\
                    n_B + 1 & \text{if}\ C = \overleftarrow{B}\ \text{and}\ n_B \geq l \\
                    n_{B \setminus \{j\}} & \text{otherwise}
                \end{cases}
                \label{insertionnumbersplanar1}
            \end{align}
            %
            for each $B \in \varphi \left( \tree[s], C, j \right)$.
            %
        \item If $C$ is an edge, $l \in \{1, 2\}$ is allowed, and we set $\treen^\varphi = {\left( n_B^\varphi \right)}_{B \in \varphi \left( \tree[s], C, j \right)}$ with
            %
            \begin{align}
                n_B^\varphi
                =
                \begin{cases}
                    l & \text{if}\ B = \{j\} \\
                    3 - l & \text{if}\ B = C \\
                    n_{B \setminus \{j\}} & \text{otherwise}
                \end{cases}
                \label{insertionnumbersplanar2}
            \end{align}
            %
            for each $B \in \varphi \left( \tree[s], C, j \right)$.
            %
    \end{itemize}
    %
\end{defi}
%
The above definitions are notionally very heavy but this is mainly due to the attempt on making it a formal definition.
However, one should keep in mind that the actual insertion or deletion procedure is quite simpel.
If $B_l$ and $B_r$ was to the immediate left and immediate right, respectively, of a leaf $\{ \tildei \}$ before deletion, we are simply saying that $B_l$ is to the immediate left of $B_r$ after deletion.
Similarly, when inserting a new leaf $\{j\}$ into a vertex we can insert it to the left or right, respectively, or inbetween (meaning the obvious thing) any two children of that vertex.
And finally, when inserting $\{j\}$ to an edge, we allow it to be the left- or rightmost, respectively, child of the newly created vertex.
}
%
We are now ready to define a down-up chain in this more detailed setting:
%
\begin{defi}[Planar $(\alpha, \gamma)$-growth model]
    Let $\hat{T}_1$ be the deterministic elements of $\widehat{\T_1}$.
    For any $\nin$ construct $\hat{T}_{n+1}$ conditional on $\hat{T}_n = \hat{\tree}$, by for each $C \in \tree$ setting 
    %
    \begin{align}
        w_{C, l}
        =
        \begin{cases}
            1 - \alpha & \text{if}\ C = C_e\ \text{is a leaf edge and}\ l = 2 \\
            \gamma & \text{if}\ C = C_e\ \text{is an internal edge and}\ l = 2 \\
            \alpha - \gamma & \text{if}\ C = C_v\ \text{with $k$ children and}\ l = 1 \\
            \alpha & \text{if}\ C = C_v\ \text{with $k$ children and}\ 1 < l < k \\
            0 & \text{otherwise}
        \end{cases}
        \label{weightsplanar}
    \end{align}
    %
    and then defining the distribution of $\hat{T}_{n+1}$ by
    %
    \begin{align}
        \P \left( \hat{T}_{n+1} = \hat{\varphi} \left( \hat{\tree}, C, j, l \right) \mid \hat{T}_n = \hat{\tree} \right)
        = \frac{w_{C,l}}{n - \alpha}
        \label{InsertionPlanarAlphaGamma}
    \end{align}
    %
    for each $C \in \rho \left( \tree \right)$.
    The sequence ${\left( \hat{T}_n \right)}_\nin$ is referred to as the \textit{planar $(\alpha, \gamma)$-growth model}.
\end{defi}
%
While we have not shown any results about the planar $(\alpha, \gamma)$-growth model, ${\left( \hat{T}_n \right)}_\nin$, we showed numerous results on ${\left( \rho \left( \hat{T}_n \right) \right)}_\nin$ in Chapter~\ref{ch:prelims}, since using the projection map simply corresponds to forgetting the internal structure amongst siblings, and the weights defined for the planar $(\alpha, \gamma)$-growth rule is on aggregate exactly the same as for the $(\alpha, \gamma)$-gamma growth
rule of Chapter~\ref{ch:prelims}.
This will play a significant role later on. \\

One of the advantages of working with planar trees as opposed to regular trees, is that we are able to make a total ordering of all subtrees along the spine from any leaf.
%
\begin{defi}[Total ordering from leaf]\label{totalordering}
  %
  Fix $\tree \in \T_n$ and some leaf $\{i\} \in \tree$ and form the spine $\{i\}, B_1, \ldots, B_k$.
  We will say that any subtree of $B_j$ is to \textit{left} of any subtree in $B_i$ for $j > i \geq 1$, which combined with the ordering inherited from the definition of a planar tree, gives a \textit{total ordering} of any subtree encountered on the spine from ${i}$.    
  %
\end{defi}
%
\section{A down-up chain for planar $(\alpha, \gamma)$-trees}
%
The entire motivation for introducing a planar version of the $(\alpha, \gamma)$-growth model, was fundamentally to arrive at the total ordering described in Definition~\ref{totalordering}, so that we can generalize the $\alpha$-chain from Definition~\ref{alphachain} and thereby obtain a result similar to Theorem~\ref{StatDistBin}.
The definition of a general $(f, G)$-down-up chain from Definition~\ref{downupchain_general} carries over without any difficulty to the planar setting, by using the corresponding growth rules and deletions maps, and will therefore not be repeated here.
We will however update the notion of label swapping to avoid any ambiguity.
%
\begin{defi}[Label swapping in $\widehat{\T_A}$]\label{labelswappingplanar}
  Let $A$ be a finite set and fix $\hat{\tree} \in \widehat{\T_A}$.
  The operation of \textit{swapping labels} is a map $\hat{\tau} \colon \widehat{\T_A} \times A \times A \to \widehat{\T_A}$ defined by
  %
  \begin{align*}
  \hat{\tau} \left( \hat{\tree}, i, j \right)
  = \left( \tau(\tree), \treen^\tau \right)
  \end{align*}
  %
  where $\treen^\tau = {\left( n_B^\tau \right)}_{B \in \tau(\tree, i, j)}$ with
  %
  \begin{align}
    n_B^\tau
    =
    \begin{cases}
      n_{\{j\}} & \text{if}\ B = \{i\} \\
      n_{\{i\}} & \text{if}\ B = \{j\} \\
      n_B & \text{otherwise}
    \end{cases}
    \label{labelswappingnumbers}
  \end{align}
  %
  for each $B \in \tau(\tree, i, j)$.
  %
\end{defi}
%
We are now ready to introduce a down-up chain for the planar $(\alpha, \gamma)$-trees.
%
% It turns out that under fairly weak assumptions a down-up chain will be Markovian, and so it is natural to ask what properties this Markov Chain holds.
%
\begin{defi}[$(\alpha, \gamma)$-chain]
  The \textit{$(\alpha, \gamma)$-chain} is an $(f, (\alpha, \gamma))$-down-up chain where $f$ is defined by setting $f(\hat{\tree[s]}, i) = \left( \tau\left( \hat{\tree[s]}, i, \tildei \right), \tildei \right)$, with
  %
  \begin{align}
    &a = \min
    \begin{Bmatrix}
      \text{leaves in 1st spinal bush} \\ \text{on spine from}\ i
    \end{Bmatrix}, \\
    &b = \min
    \begin{Bmatrix}
      \text{leaves in 1st subtree to the} \\ \text{left of $i$ not containing $a$}
    \end{Bmatrix},
    \label{searchplanar}
  \end{align}
  %
  and $\tildei = \max \left\{ i, a, b \right\}$.
\end{defi}
%
Using the proof technique from Theorem~\ref{StatDistBin} (Theorem 3 of~\cite{RefWorks:doc:5b4cbc93e4b07f5746e47014}) we recently managed to prove the following result:
%
\begin{thm}\label{StationarityTheorem}
  Fix $n \geq 3$, let $\hat{T}_n$ denote the $n$'th step of the planar $(\alpha, \gamma)$-growth model and let $\hat{T} = {\left( \hat{T}_n(m) \right)}_{m \in \N_0}$ be the $(\alpha, \gamma)$-chain.
  It holds that $\hat{T}$ has a stationary distribution, and that $\hat{T}_n(m) \dcon \hat{T}_n$ as $m \to \infty$.
\end{thm}
%
In our approach to prove the statement above, we will need the following seemingly technical lemma.
The statement of the lemma as well as the proof technique is an adaption of Lemma $4$ in~\cite{RefWorks:doc:5b4cbc93e4b07f5746e47014}.
%
\begin{lemma}\label{StatImportantLemma}
  Fix $\nin$ and let ${\left( \hat{T}_m \right)}_{1 \leq m \leq  n}$ be the first $n$ steps of a planar $(\alpha, \gamma)$-growth model.
    Fix $1 \leq i \leq \tildei \leq n$ and define
    %
    \begin{align}
        E_{i, \tildei} = \left( \tildei = \max\left\{ i, a, b \right\}\ \text{in}\ \hat{T}_n \right)
        \label{LemmaEii}
    \end{align}
    %
    It holds that $E_{i, \tildei} \independent \hat{T}_{\tildei - 1}$ and that $\hat{\pi} \left( \hat{\tau} \left( \hat{T}_n, i, \tildei \right), \tildei \right) \mid E_{i, \tildei} \deq \hat{T}_{n-1}$.
\end{lemma}
%
\begin{proof}
    We will prove the first assertion by splitting up in two cases.
    The overall proof technique will be to translate $E_{i, \tildei}$ to facts regarding the growth process --- and use this characterization to achieve the result.
    %
    \begin{itemize}
      \item[Case 1: $i < \tildei$.] In this case $E_{i, \tildei}$ is equivalent to either
            %
            \begin{itemize}
              \item inserting $\tildei$ to the immediate left of $i$ in the first branch point on the ancenstral path from $i$ in $\hat{T}_{\tildei - 1}$, and
              \item \textit{not} inserting $\tildei + 1, \ldots, n$ on leaf $i$ or between $i$ and the subtree containing $\tildei$ in the first branch point on the ancestral path from $i$ in $\hat{T}_{\tildei}, \ldots, \hat{T}_{n-1}$, respectively.
            \end{itemize}
            %
            or
            %
            \begin{itemize}
              \item inserting $\tildei$ on leaf $i$ in $\hat{T}_{\tildei - 1}$, and
              \item \textit{not} inserting $\tildei + 1, \ldots, n$ on the first two edges nor in the first branch point on the ancenstral path from $i$ in $\hat{T}_{\tildei}, \ldots, \hat{T}_{n-1}$, respectively.
            \end{itemize}
            %
            or if $i$ is located leftmost in the branch point
            %
            \begin{itemize}
              \item inserting $\tildei$ on second edge on the ancestral path from $i$ in $\hat{T}_{\tildei - 1}$, and
              \item \textit{not} inserting $\tildei + 1, \ldots, n$ on the first two edges nor to the left of $i$ in the first branch point on the ancenstral path from $i$ in $\hat{T}_{\tildei}, \ldots, \hat{T}_{n-1}$, respectively.
            \end{itemize}
            %
            \begin{figure}[t]
                \centering
                \begin{tikzpicture}[%
        table/.style={circle, draw, minimum width=0.7cm},
        bp/.style={circle, draw, fill=black, inner sep=0pt, minimum width=4pt},
        comment/.style={circle, draw, color=red},
        label/.style={text=oxfordblue},
        line/.style={-, line width = 2pt, color = oxfordblue},
        scale=1
    ]

    \begin{scope}[xscale = -1]
      \node (0) at (0,0) {};
      \node (1bp_start) at (0,0.5) {};
      \node [bp] (1bp_end) at (1,0.5) {};
      \node (tree1) at (0.5,1.5) {};
      \node (bp_start) at (1,2) {};
      \node (bp_mid2) at (2.5,2) {};
      \node[anchor=south] (2) at (2.5,3) {};
      \node [bp] (bp_end) at (4,2) {};
      \node [anchor = south] (1) at (4.5,3) {};
      \node [anchor = south] (i) at (3.5,3) {$i$};

      \draw[line, dotted] (0) to (1bp_start.center);
      \draw[line] (1bp_start.center) to (1bp_end);

      \draw[line] (bp_start.center) to node[auto, outer sep = 1pt] {$\alpha-\gamma$} (bp_mid2.center);
      \draw[line, green] (bp_mid2.center) to node[auto, outer sep = 2pt] {$\alpha$} (bp_end);

      \draw[line, bend left] (bp_mid2.center) to (2.center) to (bp_mid2.center);
      \draw[line] (1bp_end) [bend left] to (tree1) [bend left] to (1bp_end);
      \draw[line, bend left] (bp_end.center) to (1.center) to (bp_end.center);
      \draw[line] (bp_end.center) to node[auto, swap, anchor = west] {$1 - \alpha$} (i);

      \draw[line] (1bp_end) to node[auto] {$\gamma$} (bp_start.center);

      \node[label, anchor = south] (T_itilde) at (0,2) {$\That_{\tildei-1}$};

      \begin{scope}[shift = {(-6,0)}]
        \node (0) at (0,0) {};
        \node [bp] (1bp_start) at (0,0.5) {};
        \node [bp] (1bp_end) at (1,0.5) {};
        \node (tree1) at (0.5,1.5) {};
        \node (bp_start) at (1,2) {};
        \node (bp_mid1) at (2,2) {};
        \node (2) at (2,3) {};
        \node (bp_mid2) at (3,2) {};
        \node (itilde) at (3,3) {$\tildei$};
        \node [bp] (bp_end) at (4,2) {};
        \node [anchor = south] (1) at (4.5,3) {};
        \node [anchor = south] (i) at (3.5,3) {$i$};

        \draw[line, dotted] (0) to (1bp_start);
        \draw[line] (1bp_start) to (1bp_end) to (bp_start.center) to (bp_mid1.center);
        \draw[line] (itilde) to (bp_mid2.center) to (bp_mid1.center);
        \draw[line, bend left] (bp_mid1.center) to (2.center) to (bp_mid1.center);
        \draw[line, bend left] (1bp_end) to (tree1) to (1bp_end);
        \draw[line] (bp_end.center) [bend left] to (1.center) [bend left] to (bp_end.center);
        \draw[line, red] (i) to node[auto, swap, anchor = west, inner sep = 1pt] {\tiny $1 - \alpha$} (bp_end.center) to node[auto] {\tiny $\alpha$} (bp_mid2.center);

        \node[label, anchor = south] (T_itilde) at (0,2) {$\That_{\tildei}$};
      \end{scope}

      \begin{scope}[shift = {(0,-4)}]
        \node (0) at (0,0) {};
        \node (1bp_start) at (0,0.5) {};
        \node [bp] (1bp_end) at (1,0.5) {};
        \node (tree1) at (0.5,1.5) {};
        \node (bp_start) at (1,2) {};
        \node (bp_mid2) at (2.5,2) {};
        \node[anchor=south] (i) at (2.5,3) {$i$};
        \node [bp] (bp_end) at (4,2) {};
        \node [anchor = south] (1) at (4.5,3) {};
        \node [anchor = south east] (2) at (3.5,3) {};

        \draw[line, dotted] (0) to (1bp_start.center);
        \draw[line] (1bp_start.center) to (1bp_end);

        \draw[line] (1bp_end) to node[auto] {$\gamma$} (bp_start.center);

        \draw[line, green] (bp_mid2.center) to node[auto] {$\alpha$} (bp_end);

        \draw[line] (bp_mid2.center) to node[auto, swap] {$1-\alpha$} (i);
        \draw[line] (1bp_end) [bend left] to (tree1) [bend left] to (1bp_end);
        \draw[line] (bp_end.center) [bend left] to (1.center) [bend left] to (bp_end.center) [bend left] to (2.center) [bend left] to (bp_end.center);

        \draw[line] (bp_start.center) to node[auto] {$\alpha - \gamma$} (bp_mid2.center);

        \node[label, anchor = south] (T_itilde) at (0,2) {$\That_{\tildei-1}$};
      \end{scope}

      \begin{scope}[shift = {(-6,-4)}]
        \node (0) at (0,0) {};
        \node [bp] (1bp_start) at (0,0.5) {};
        \node [bp] (1bp_end) at (1,0.5) {};
        \node (tree1) at (0.5,1.5) {};
        \node (bp_start) at (1,2) {};
        \node (bp_mid1) at (2,2) {};
        \node (i) at (2,3) {$i$};
        \node (bp_mid2) at (3,2) {};
        \node (itilde) at (3,3) {$\tildei$};
        \node [bp] (bp_end) at (4,2) {};
        \node [anchor = south] (1) at (4.5,3) {};
        \node [anchor = south east] (2) at (3.5,3) {};

        \draw[line, dotted] (0) to (1bp_start);
        \draw[line] (1bp_start) to (1bp_end) to (bp_start.center) to (bp_mid1.center);
        \draw[line] (bp_end) to (bp_mid2.center) to (itilde);
        \draw[line, red] (bp_mid2.center) to node[auto] {\footnotesize $\alpha$} (bp_mid1.center) to node[auto, swap] {\footnotesize $1-\alpha$} (i);
        \draw[line] (1bp_end) [bend left] to (tree1) [bend left] to (1bp_end);
        \draw[line] (bp_end.center) [bend left] to (1.center) [bend left] to (bp_end.center) [bend left] to (2.center) [bend left] to (bp_end.center);

        \node[label, anchor = south] (T_itilde) at (0,2) {$\That_{\tildei}$};
      \end{scope}

      \begin{scope}[shift = {(0,-8)}]
        \node (0) at (0,0) {};
        \node (1bp_start) at (0,0.5) {};
        \node [bp] (1bp_end) at (1,0.5) {};
        \node (tree1) at (0.5,1.5) {};
        \node (bp_start) at (1,2) {};
        \node (bp_mid2) at (2.5,2) {};
        \node[anchor=south] (i) at (2.5,3) {$i$};
        \node [bp] (bp_end) at (4,2) {};
        \node [anchor = south] (1) at (4.5,3) {};
        \node [anchor = south east] (2) at (3.5,3) {};

        \draw[line, dotted] (0) to (1bp_start.center);
        \draw[line] (1bp_start.center) to (1bp_end);

        \draw[line] (1bp_end) to node[auto] {$\gamma$} (bp_start.center);

        \draw[line] (bp_mid2.center) to node[auto] {$\alpha$} (bp_end);

        \draw[line, green] (bp_mid2.center) to node[auto, swap] {$1-\alpha$} (i);
        \draw[line] (1bp_end) [bend left] to (tree1) [bend left] to (1bp_end);
        \draw[line] (bp_end.center) [bend left] to (1.center) [bend left] to (bp_end.center) [bend left] to (2.center) [bend left] to (bp_end.center);

        \draw[line] (bp_start.center) to node[auto] {$\alpha - \gamma$} (bp_mid2.center);

        \node[label, anchor = south] (T_itilde) at (0,2) {$\That_{\tildei-1}$};
      \end{scope}

      \begin{scope}[shift = {(-6,-7.5)}]
        \node (0) at (-1,-1) {};
        \node (0bp_start) at (-1, -0.5) {};
        \node [bp] (0bp_end) at (0, -0.5) {};
        \node (tree1) at (-0.5,0.5) {};
        \node (1bp_start) at (0,0.5) {};
        \node (1bp_mid) at (1.4,0.5) {};
        \node [bp] (1bp_end) at (3.5,0.5) {};
        \node [anchor = south east] (tree2) at (3,1.5) {};
        \node [anchor = south] (tree3) at (4,1.5) {};

        \node (bp_start) at (1.4,2) {};
        \node [bp] (bp_end) at (2.7,2) {};
        \node [anchor = south] (1) at (3.2,3) {$i$};
        \node [anchor = south east] (2) at (2.3,3) {$\tildei$};

        \draw[line, dotted] (0) to (0bp_start.center);
        \draw[line] (0bp_start.center) to (0bp_end) to (1bp_start.center) to (1bp_mid.center) to (1bp_end);
        \draw[line] (0bp_end) [bend left] to (tree1) [bend left] to (0bp_end);
        \draw[line] (1bp_end.center) [bend left] to (tree3.center) [bend left] to (1bp_end.center);
        \draw[line] (1bp_end.center) [bend left] to (tree2.center) [bend left] to (1bp_end.center);
        \draw[line, red] (1bp_mid.center) to node[auto] {\footnotesize $\gamma$} (bp_start.center) to node[auto] {\footnotesize $\alpha - \gamma$} (bp_end) to node[auto] {\footnotesize $1-\alpha$} (1);
        \draw[line] (bp_end) to (2);

        \node[label, anchor = south] (T_itilde) at (0,2) {$\That_{\tildei}$};
      \end{scope} 
    \end{scope}
\end{tikzpicture}

                \caption{Test}
                \label{fig:ForbiddenEdges_1}
            \end{figure}
            %
            Since we have now characterized the event $E_{i,\tildei}$ in terms of insertions to specific edges summing to a total weight of $1$ in all cases --- and the fact that all of these insertion are independent of one another --- this is enough to argue that $\hat{T}_{\tildei - 1} \independent E_{i, \tildei}$ when $i < \tildei$.

          \item[Case 2: $i = \tildei$.] Here $E_{i, \tildei}$ is equivalent to
            %
            \begin{itemize}
              \item \textit{not} inserting $\tildei + 1, \ldots, n$ on the first two edges nor to the left of $i = \tildei$ in the first branch point on the ancestral path from $i$ in $\hat{T}_{\tildei}, \ldots \hat{T}_{n-1}$, respectively, if $i = \tildei$ is placed leftmost in the branch point.
            \end{itemize}
            %
            or
            \begin{itemize}
              \item \textit{not} inserting $\tildei + 1, \ldots, n$ on the leaf or to the immediate left of $i = \tildei$ in the first branch point on the ancestral path from $i$ in $\hat{T}_{\tildei}, \ldots \hat{T}_{n-1}$, respectively, if $i = \tildei$ is places anywhere else.
            \end{itemize}
            %
            Following this characterization we have shown that $\hat{T}_{\tildei - 1} \independent E_{i, \tildei}$ for all $1 \leq i \leq \tildei \leq n$, proving the first assertion of the lemma.
            
            For the second assertion we start by noting that
            %
            \begin{align}\label{ImpLemmaStartPoint}
              \P \left( \hat{T}_{\tildei - 1} \in \cdot \mid E_{i,\tildei} \right) = \P \left( \hat{T}_{\tildei - 1} \in \cdot \right)
            \end{align}
            %
            due to the independence proven above.
            Now that we can construct $\tau \left( \hat{T}_n, i, \tildei \right)$ from $\hat{T}_{\tildei - 1}$ under the conditional law of $E_{i, \tildei}$:
            The leaf $\tildei$ is never attached and so the `forbidden edges', characterized above, does not exist.
            But every leaf that was previously labelled $\tildei, \ldots, n$ will still be attached according to the $(\alpha, \gamma)$-growth rule.
            \fxfatal{Should this argument be more elaborate? Maybe provide figures for the various cases?}
            This shows that
            %
            \begin{align}\label{ImpLemmaEnd}
              \P \left( \hat{\pi} \left( \hat{\tau} \left( \hat{T}_n, i, \tildei \right) \right) \in \cdot \mid E_{i, \tildei} \right)
              = \P \left( \hat{T}_{n-1} \in \cdot \right)
            \end{align}
            %
            finalizing the proof.
    \end{itemize}
\end{proof}
%
With the above lemma in hand Theorem~\ref{StationarityTheorem} is within reach, by conditioning on $E_{i, \tildei}$ for fixed values of $i$ and $\tildei$:
%
\begin{proof}[Proof of Theorem~\ref{StationarityTheorem}]
    Fix $n \in \N$.
    That the $(\alpha, \gamma)$-chain on $\T_n^*$ is stationary is easily seen, since the state space is finite and there is a unique, recurrent communicating class for
    %
    \begin{itemize}
        \item $\alpha = 1$ and all $\gamma \in (0,1)$ (only insertions in branch points or internal edges allowed),
        \item $\gamma = 0$ and all $\alpha \in (0,1)$ (only insertions in branch points or leaves allowed),
        \item $\gamma = \alpha$ and all $\alpha \in (0,1)$ (binary trees),
    \end{itemize}
    %
    and that the Markov Chain is irreducible otherwise.
    Thus we are done if we can show that $T_1^n \deq T_0^n = T_n$, which we will do by utilizing Lemma~\ref{StatImportantLemma}.
    Let $U$ be a uniform random variable on $[n]$, independent of everything else, and note that for all $t \in \T_{n-1}$ it holds that
    %
    \begin{align*}
        \P \left( \tau \left( T_n, U \right) = t \right)
        &= \sum_{1 \leq i \leq \tildei \leq n} \P \left( \tau \left( T_n, i, \tildei \right) = t \mid E_{i, \tildei} \cap (U = i) \right) \P \left( E_{i, \tildei} \cap (U = i) \right) \\
        &= \sum_{1 \leq i \leq \tildei \leq n} \P \left( T_{n-1} = t \right) \P \left( E_{i, \tildei} \cap (U = i) \right) \\
        &= \P \left( T_{n-1} = t \right)
    \end{align*}
    %
    where we have used the independence of $U$ as well as Lemma~\ref{StatImportantLemma}.
    This shows that $\tau \left( T_n, U \right) \deq T_{n-1}$, and since $T_1^n$ is constructed from the former using the exact same operation as $T_n$ is constructed from the latter, this finishes our proof.
\end{proof}
