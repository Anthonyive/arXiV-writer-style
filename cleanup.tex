
\documentclass{amsart}
\makeatletter
\renewcommand{\theequation}{\thesection.\arabic{equation}}
\@addtoreset{equation}{section}
\makeatother
\usepackage{setspace}
\usepackage{xcolor}
\singlespacing
\usepackage{amsmath}
\usepackage{amssymb}
\usepackage{latexsym}
\usepackage{amsthm}
\usepackage{hyperref}
\usepackage{mathscinet}
\newtheorem{Thm}{Theorem}
\newtheorem{Que}{Question}
\newtheorem{Lem}{Lemma}
\newtheorem{Prop}{Proposition}
\newtheorem{Conj}{Conjecture}
\newtheorem{Cor}{Corollary}
\newtheorem{Cla}{Claim}
\newtheorem{Fac}{Fact}
\theoremstyle{definition}
\newtheorem{Axi}{Axiom}
\newtheorem{Ass}{Assumption}
\newtheorem{Def}{Definition}
\newtheorem{Rem}{Remark}
\newtheorem{Exa}{Example}
\begin{document}
\title{Characterizations of the Cauchy distribution associated with integral transforms}
\author{Kazuki Okamura}
\address{School of General Education, Shinshu University}
\email{kazukio@shinshu-u.ac.jp}
\subjclass{60E10, 62E10}
\keywords{characterization of the Cauchy distribution, M\"obius transforms, Mellin transforms}
\date{\today}
\maketitle
\begin{abstract}
We give two new simple characterizations of the Cauchy distribution by using the
M\"obius and Mellin transforms.
They
also yield characterizations of the circular Cauchy distribution and the mixture Cauchy model.
\end{abstract}
\section{Introduction}
The Cauchy distribution is a statistical model with a heavy-tailed symmetric distribution.
We cannot define its expected value and its variance, and it has no moment generating function, due to its heavy tails.
It also appears in physics, and is called the Lorentz distribution alternatively.
Characterizations of probability distributions are interesting in itself and useful when we choose a suitable statistical model.
Various characterizations of the Cauchy distribution have been considered by many authors \cite{Arnold1979,Arnold1990, Bell1985, Chin2020, Dunau1987, Hamedani1993, Hassenforder1988, Knight1976b, Knight1976a, Letac1977, Menon1962, Menon1966, Norton1983, Obretenov1961, Ramachandran1970, Williams1969, Yanushkevichius2007,Yanushkevichius2014}.
This paper proposes yet another type of characterizations of the Cauchy distribution.
Our characterizations concern
integral transforms, specifically, the M\"obius and Mellin transforms.
The M\"obius and Mellin transforms of the Cauchy distribution have somewhat simpler forms than the characteristic function of it, that is, the Fourier transform of it.
Our proofs utilize some basic facts of complex analysis and functional analysis.
Furthermore, our approach immediately yields characterizations of the circular Cauchy distribution and the mixture Cauchy model.
This paper adopts McCullagh's parametrization of the Cauchy distribution \cite{McCullagh1996}.
Let   be the imaginary unit.
For a complex number  , let   and   be the real and imaginary parts of   respectively.
We denote the distribution with density function
\ by  , where we let  .
This paper is organized as follows.
In Section 2, we give a characterization of the Cauchy distribution by the M\"obius transforms with an application to a characterization of the circular Cauchy distribution.
In Section 3, we give a characterization of the Cauchy distribution by the Mellin transforms with an application to a characterization of the mixture Cauchy model.
\section{Characterization by M\"obius transforms}
Hereafter the symbol   denotes the notation of the expectation of random variables and we denote the complex conjugate of a complex number   by  .
\begin{Thm}\label{Mobius}
Let   be a real-valued random variable such that there exists
  such that
\}{E\left}\]
for every   in a subset   of   having a limit point in  .
Then,
  follows  .
\end{Thm}
Let   be the set of continuous functions vanishing at infinity.
The following is standard.
\begin{Lem}\label{Cc}
Let  .
Then,
\ \end{Lem}
\begin{proof}
We have that
\ Since  ,   is uniformly continuous on  , that is,
it holds that
for every
 , there exists   such that for every  ,
\ By this and the fact that  ,
we have that
\} \frac{b}{\pi(t^2 + b^2)} dt, \]
where   denotes the supremum norm of  .
Since   is increasing as a function of  ,
by applying the Lebesgue dominated convergence theorem,
we have that
\} \frac{b}{\pi(t^2 + b^2)} dt = 0.
\]
\end{proof}
For  ,
we let   be the function defined by
\ which is a M\"obius transform and could be regarded as a certain generalization of the Cayley transform.
\begin{proof}
We have that for every  ,
\begin{equation}\label{Mobius-mean}
E\left = \phi_{\gamma}(\alpha).
\end{equation}
By the residue theorem,
we see that \eqref{Mobius-mean} holds for   following  .
Let   be the Borel probability measure on   induced by  .
Then,
\begin{equation}\label{eq-compare}
\int_{\mathbb{R}} \phi_{\gamma}(x) \mu(dx) = \int_{\mathbb{R}} \phi_{\gamma}(x)
\nu(dx),
\end{equation}
where   and
we let
\ Let  F_{\mu}(a+bi) := \frac{1}{\pi} \int_{\mathbb{R}} \frac{1}{x - (a+bi)} \mu(dx), \ a+bi \in \mathbb{H}. 
This is holomorphic on  .
Let   and   be the real and imaginary parts of   respectively.
By replacing   with  , we define  ,   and   in the same manner.
By comparing the real and imaginary parts of \eqref{eq-compare},
we have that
for every  ,
\ and
\ Therefore,
  on  .
By applying
the identity theorem for holomorphic functions \cite{Rudin1987},
  on  .
By this and Fubini's theorem,
\ \ By Lemma \ref{Cc}, we have that
\ and
\ Thus we have that
\ Since   and   are both
regular, by the Riesz-Markov-Kakutani theorem \cite{Rudin1987},
we have that  , which means that   follows the Cauchy distribution with parameter  .
\end{proof}
\begin{Rem}\label{upper}
Let   be the closure of  , that is,  .
Let  .
Since
\E\left - E\left^2}{\left|E\left\right|^2}i,\]
where we let  ,
it holds that for every  ,  , and furthermore, it holds that
  if and only if the distribution of   is not a point mass.
It holds that
  is the maximal likelihood estimator of Cauchy samples  ,
if and only if
 
where the expectation
is taken with respect to  .
\end{Rem}
The circular Cauchy distribution, also known as the wrapped Cauchy distribution, appears in the area of directional statistics.
It is a distribution on the unit circle and is connected with the Cauchy distribution via M\"obius transforms.
Such connection is considered by
\cite{McCullagh1996}.
Let  .
The circular-Cauchy distribution   with parameter   is the continuous distribution on $
We remark that   is a bijection between   and  , and furthermore its inverse is given by
\ We can extend the domain of   to  .
  defines a bijection between   and  .
If a random variable   follows the circular-Cauchy distribution  ,
then,   follows the Cauchy distribution with parameter  .
Therefore, by computations with Theorem \ref{Mobius},
we have that
\begin{Cor}
Let   be a $}{E\left} \]
for every   in a subset   of   having a limit point in  .
We also assume that  .
Then,   follows the circular-Cauchy distribution  .
\end{Cor}
Let   be the closure of  , that is,  .
Let
 .
Assume that  .
Then, for every  ,
  where we let  .
Then, by Remark \ref{upper}, it holds that for every  ,  , and furthermore,
  if and only if the distribution of   is not a point mass.
\section{Characterization by Mellin transforms}
We define the logarithm for complex numbers as follows.
For   where   and  ,
we let
\ This is holomorphic on  .
Then,
\begin{equation}\label{def-log}
\log x = \log |x| + i \pi \mathbf{1}_{(-\infty, 0)}(x), \ \ x \in \mathbb{R} \setminus \{0\},
\end{equation}
where   denotes the indicator function of  .
For every  , we let
\ For every  , we let  .
This definition is also adopted for  .
We remark that   is{\it not} a real number if   and  .
For example,  .
In this paper, we call   the Mellin transform of the random variable  .
We deal with the powers of{\it negative} numbers by allowing the powers to be{\it complex-valued}.
In this point, our definition of
the powers of
random variables
is different from the one given in Zolotarev \cite{Zolotarev1986}.
\begin{Thm}\label{power}
Let   be a real-valued random variable such that  
for some  .
If it holds that   for
a subset   having a limit point in   and some  ,
then,   follows
 .
\end{Thm}
Our proof of this assertion depends on Galambos and Simonelli \cite{galambos2004}.
However their definition of
the Mellin transform of random variables is somewhat different from ours, so we need some arguments.
\begin{proof}
\begin{Lem}\label{F-hol}
Let   and   be the closure of  .
Let  .
Then,   is well-defined and continuous on   and holomorphic on  .
\end{Lem}
\begin{proof}
Since   and  ,
we see that   is well-defined and continuous on  .
Let   and  .
We remark that  .
Then we have that
\ = E\left.
\]
If  , then,
\ \ If   and  , then,
\ By the assumption,
\ < +\infty. \]
If   and  , then,
\ By the assumption and the fact that   for every  ,
\ < +\infty. \]
By the Lebesgue dominated convergence theorem,
\ \to 0, \ h \to 0. \]
\end{proof}
\begin{Lem}\label{F-rep}
\ \end{Lem}
\begin{proof}
Let  .
This is holomorphic on  .
By the assumption of Theorem \ref{power},
it holds that
 .
By Lemma \ref{F-hol},   is holomorphic on  .
Hence, by the identity theorem for holomorphic functions,
 .
Since   and   are both continuous on  ,
we have the assertion.
\end{proof}
\begin{Lem}\label{g-exp-pre}
Let
\begin{equation}\label{g-def}
g(a) := E\left + i E\left.
\end{equation}
Then,
  is well-defined and continuous on   and holomorphic on  .
Furthermore,
\begin{equation}\label{g-strip}
g(a) = r^a \left(\cos(a \theta) - \frac{\sin(a\theta)}{\sin(a\pi)} \cos(a\pi) + i \frac{\sin(a\theta)}{\sin(a\pi)} \right), \ a \in J.
\end{equation}
\end{Lem}
We remark that \eqref{g-def} is equivalent to the definition of
the Mellin transform of   in \cite{galambos2004}.
\begin{proof}
Since  ,
  is well-defined and continuous on  .
If  , then,
\begin{equation}\label{f-re-im}
f(a) = E\left + E\left \cos(a \pi) +
i E\left \sin(a \pi).
\end{equation}
If  , then, by Lemma \ref{F-rep},
\begin{equation}\label{x-nega}
E\left = r^a \frac{\sin(a\theta)}{\sin(a\pi)},
\end{equation}
where we let  .
Hence, as a function of  ,
  is holomorphic on  .
By using
Lemma \ref{F-rep}, we have that
\ + E\left \cos(a \pi) = r^a \cos(a\theta), \ 0 < a < \delta. \]
Therefore we have that as a function of  ,
  is holomorphic on   and
\begin{equation}\label{x-posi}
E\left = r^a \left(\cos(a\theta) - \frac{\sin(a\theta)}{\sin(a\pi)}\cos(a\pi) \right).
\end{equation}
By \eqref{f-re-im}, \eqref{x-nega} and \eqref{x-posi},
we have \eqref{g-strip}.
\end{proof}
Now we return to the proof of Theorem \ref{power}.
Since   for   and  ,
we could continuously extend the function   in
\eqref{g-strip} to the left boundary of  , which is the imaginary axis  .
If   follows the Cauchy distribution  , then, by the residue theorem,
\ = \frac{\textup{Im}(\gamma)}{\pi} \int_{\mathbb{R}}
\frac{x^a}{|x - \gamma|^2} dx = \gamma^a, \ a \in J. \]
Hence if we define   for   following
the Cauchy distribution   in the same manner as in \eqref{g-def},
then we have the same expression for   as in \eqref{g-strip},
and in particular, they are identical with each other on
the imaginary axis  .
Now Theorem \ref{power} follows from \cite{galambos2004}.
\end{proof}
We also have the following claim which is similar to Theorem \ref{power}.
\begin{Thm}\label{power-2}
Let   be a real-valued random variable such that   and
 
for some  .
If   for
a subset   having a limit point in   and some  ,
then,   follows  .
\end{Thm}
The proof of Theorem \ref{power-2} goes in the same manner as in the proof of Theorem \ref{power}.
\begin{Cor}
Let   be a real-valued random variable such that   and
 
for some  . Then, \\ (i)
If   for an infinite increasing sequence  ,
then,   follows
 . \\ (ii) If   and   for every  ,
then,   follows
 .
\end{Cor}
We remark that if   and  , then,  , and furthermore,   is non-atomic.
\begin{proof}
Assertion (i) follows from Theorem \ref{power-2}.
(ii) We remark that for  ,
\ \le \exp(|p| \pi) E \le E\left < +\infty. \]
By the Lebesgue convergence theorem and the assumption, we have that for  ,
\ = E\left = \sum_{n=0}^{\infty} \frac{p^n E\left}{n!}
= \exp(p E). \]
By \eqref{def-log}, we have that
\) = \exp(E) \exp(i \pi P(X < 0)).
\]
By the assumption, we have that  .
Hence, it holds that
 .
Now apply Theorem \ref{power}.
\end{proof}
It might be
interesting to consider sufficient conditions for   for every  .
It is not sufficient that  .
For example, if we consider the expectation with respect to
 \mu = \frac{1}{3} \left( \delta_{\{-1\}} + \delta_{\{\exp(\pi/\sqrt{3})\}} + \delta_{\{\exp(-\pi/\sqrt{3})\}}
\right), 
then, we have that  .
We can also give a characterization for the mixture Cauchy model.
If the probability density function is given by
\ for some   and  ,
then, we call the model the mixture Cauchy model $C\left(t; \mu_1 + \sigma_1 i;
\mu_2 + \sigma_2 i\right)$.
See Lehmann \cite{Lehmann1999} for mixture models of location-scale families.
We can show the following in the same manner as in the proof of Theorem \ref{power}.
\begin{Cor}
Let   be a real-valued random variable such that  
for some  .
If   for
a subset   having a limit point in   and some   and  ,
then,   follows the mixture Cauchy model  .
\end{Cor}
\noindent{\it Acknowledgements} \ The author appreciates the referee for careful reading of the manuscript and giving helpful comments.
The author was supported by JSPS KAKENHI
19K14549.
\bibliographystyle{amsplain}
\bibliography{GFT-Cauchy}
\end{document}
